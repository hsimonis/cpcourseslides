
\usepackage{../template/beamerthemeinsight/insight}
\usepackage[medium]{ubuntu}  % May need to be installed separately from https://github.com/tzwenn/ubuntu-latex-fonts
\usepackage{url}
\usepackage[normalem]{ulem}
\usepackage{listings}
\usepackage{graphics}
\usepackage{tikz}
\usetikzlibrary{shapes,calc,through,backgrounds,arrows,automata,positioning}

\usepackage{multicol}
\usepackage{subfigure}
\usepackage{booktabs}
\usepackage{amsmath}
\usepackage{hyperref}


\definecolor{pantone174-6}{RGB}{127,138,154}
\definecolor{insight-gray}{RGB}{127,138,154}
\definecolor{pantone117-8}{RGB}{0,107,148}
\definecolor{insight-blue}{RGB}{0,107,148}
\definecolor{pantone127-4}{RGB}{114,202,195}
\definecolor{insight-aqua}{RGB}{114,202,195}
\definecolor{pantone157-8}{RGB}{146,200,62}
\definecolor{insight-lime}{RGB}{146,200,62}
\definecolor{pantone126-8}{RGB}{0,105,106}
\definecolor{insight-forestgreen}{RGB}{0,105,106}
\definecolor{pantone151-8}{RGB}{76,183,72}
\definecolor{insight-green}{RGB}{76,183,72}
\definecolor{pantone119-1}{RGB}{176,215,227}
\definecolor{insight-babyblue}{RGB}{176,215,227}

\definecolor{pantone10-7}{RGB}{255,198,52}
\definecolor{insight-yellow}{RGB}{255,198,52}
\definecolor{pantone24-8}{RGB}{246,135,31}
\definecolor{insight-orange}{RGB}{246,135,31}
\definecolor{pantone46-8}{RGB}{193,49,34}
\definecolor{insight-burntorange}{RGB}{193,49,34}
\definecolor{pantone74-7}{RGB}{186,52,111}
\definecolor{insight-pink}{RGB}{186,52,111}
\definecolor{pantone92-7}{RGB}{105,44,122}
\definecolor{insight-plum}{RGB}{105,44,122}
\definecolor{pantone72-16}{RGB}{145,0,69}
\definecolor{insight-maroon}{RGB}{145,0,69}
\definecolor{pantone120-2}{RGB}{128,184,200}
\definecolor{insight-dodgerblue}{RGB}{128,184,200}
\definecolor{pantone149-4}{RGB}{115,175,117}
\definecolor{insight-palegreen}{RGB}{115,175,117}
\definecolor{pantone124-16}{RGB}{0,152,153}
\definecolor{insight-turquoise}{RGB}{0,152,153}
\definecolor{pantone116-14}{RGB}{0,149,191}
\definecolor{insight-royalblue}{RGB}{0,149,191}
\definecolor{pantone102-7}{RGB}{61,91,169}
\definecolor{insight-purple}{RGB}{61,91,169}
\definecolor{pantone108-16}{RGB}{0,52,98}
\definecolor{insight-darknavy}{RGB}{0,52,98}

\usepackage{xspace}
\usepackage{mhchem}
\newcommand{\sol}{\textsf{Solutions}\xspace}
\newcommand{\pat}{\textsf{Generalisations/Patterns}\xspace}
\newcommand{\obs}{\textsf{Observations}\xspace}

\newcommand{\rcp}{\textsf{read-for-CP}\xspace}
\newcommand{\wcp}{\textsf{write-from-CP}\xspace}
\newcommand{\rdm}{\textsf{read-for-DM}\xspace}
\newcommand{\rw}{\textsf{Apply-to-World}\xspace}
\newcommand{\wdm}{\textsf{write-from-DM}\xspace}

\newcommand{\evalw}{\ensuremath{eval\_world}\xspace}


% Define block styles
\tikzstyle{world} = [cloud, draw,cloud puffs=10,cloud puff arc=120, aspect=2, inner ysep=1em]
\tikzstyle{block} = [rectangle, draw, fill=blue!20, text width=4em, text centered, rounded corners, minimum height=3em]
\tikzstyle{data} = [draw, ellipse,fill=red!20, minimum height=4em, minimum width=6em]
\tikzstyle{line} = [draw, -latex]
\tikzstyle{n} = [font=\itshape]
 
\usepackage{amssymb,marvosym}
\usepackage{colortbl}
\usepackage{array}
\usepackage{textpos}
\usepackage{textcomp,pifont}
\newcommand{\REM}[2]{\only<presentation>{\only<#1>{\textcolor{pantone116-14}{\ding{233} \textit{#2}}}}}
\newcommand{\G}{\cellcolor[gray]{0.7}}
\newcommand{\R}{\cellcolor[gray]{0.5}\ding{54}}
\newcommand{\X}{\cellcolor[gray]{0.5}-}
\newcommand{\D}{\cellcolor[gray]{0.5}|}
\newcommand{\A}{\cellcolor{pantone72-16}} % dark red
\newcommand{\Y}{\cellcolor{pantone10-7}} % yellow
\newcommand{\F}{\cellcolor{pantone24-8}\ding{89}} % red
\newcommand{\links}[1]{\href{../#1/VIDEO/web/web.html}{\beamerbutton{Video}}~\href{../#1/VIDEO/iphone/iphone.m4v}{\beamerbutton{iPhone}}~\href{../#1/slides.pdf}{\beamerbutton{Slides}}~\href{../#1/handout.pdf}{\beamerbutton{Handout}}~\href{http://4c114:8080/dokuwiki/doku.php?id=#1}{\beamerbutton{Wiki}}}
\newcommand{\pred}[1]{\texttt{#1}}
\mode<all>{
\tikzset{router/.style={shape=circle,draw=pantone116-14,fill=pantone120-2,thick,inner sep=0pt,minimum size=6mm}}
\tikzset{ring/.style={shape=circle,draw=pantone116-14,fill=pantone120-2,thick,inner sep=0pt,minimum size=6mm}}
\tikzset{box/.style={shape=rectangle,draw=pantone116-14,fill=pantone120-2,thick,inner sep=0pt,minimum size=6mm}}
}


\usepackage[english]{babel}
% or whatever

\usepackage[latin1]{inputenc}
% or whatever

\usepackage{times}
\usepackage[T1]{fontenc}
% Or whatever. Note that the encoding and the font should match. If T1
% does not look nice, try deleting the line with the fontenc.

\mode<article>{
\usepackage{makeidx}
\makeindex
\usepackage{fullpage}
\usepackage{hyperref}
}
\hypersetup{pdfauthor={Helmut Simonis},pdfsubject={Introduction to Constraint Programming},pdfkeywords={Introduction
    to Constraint Programming;ECLiPSe;ELearning;Saros;Constraint
    Programming;Prolog;Logic Programming;Insight;Insight Centre for
    Data Analytics;University College Cork;UCC;MiniZinc;CRT-AI;Choco-Solver;ACP Winterschool 2024}}

\mode<presentation>{
\author{Helmut Simonis}
%\email{helmut.simonis@insight-centre.org}
}
\mode<article>{
\author{Helmut Simonis\\%
\\%
email: \url{helmut.simonis@insight-centre.org}\\%
homepage: \url{http://http://insight-centre.org/}\\%
\\%
Insight SFI Centre for Data Analytics\\%
School of Computer Science and Information Technology\\%
University College Cork\\
Ireland}
}
\institute[Insight] % (optional, but mostly needed)
{Insight SFI Centre for Data Analytics\\School of Computer Science and Information Technology\\University College Cork\\Ireland}

\date[March 25th, 2024] % (optional)
{ACP Winterschool 2024}



% Delete this, if you do not want the table of contents to pop up at
% the beginning of each subsection:
\AtBeginSection[]
{
  \begin{frame}<beamer>
    \frametitle{Outline}
    \tableofcontents[sectionstyle=show/shaded,subsectionstyle=show/show/hide]
  \end{frame}
}


% If you wish to uncover everything in a step-wise fashion, uncomment
% the following command: 

%\beamerdefaultoverlayspecification{<+->}

