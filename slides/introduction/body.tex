\begin{frame}
\frametitle{Objectives}
\begin{itemize}
\item Overview of Core Constraint Programming
\item Three Main Concepts
\begin{itemize}
\item Constraint Propagation
\item Global Constraints
\item Customizing Search
\end{itemize}
%\item Get Some Experience with MiniZinc
\item Topics will be treated in more detail in later parts of the school
\item Based on Examples, not Formal Description
\end{itemize}
\end{frame}

\begin{frame}
\frametitle{Outline}
\begin{itemize}
\item Why Constraint Programming?
\item Constraint Propagation
\item Global Constraints
\item Customizing Search
%\item Choices in Modelling
%\item Decomposition
%\item More Global Constraints
%\item Methodology
%\item What is missing?
\end{itemize}
\end{frame}

% \begin{frame}
  % \frametitle{Using MiniZinc IDE}
  % \begin{itemize}
  % \item Developed in the Australian NICTA project
  % \item Maintained by Monash University
  % \item Modelling tool with multiple back-end solvers
  % \item Available from \url{https://www.minizinc.org/}
  % \end{itemize}
% \end{frame}


% \begin{frame}
% \frametitle{Examples in ECLiPSe}
% \begin{itemize}
% \item Open sourced constraint programming language
% \item Development goes back to 1985
% \item ECRC, ICL, IC-Parc, PTL, Cisco
% \item \url{https://eclipseclp.org/}
% \item Specialities
% \begin{itemize}
% \item Develop new solvers for specific domains
% \item Integration with MIP
% \end{itemize}
% \item Not included in bundled MiniZinc IDE
% \item Specialized visualization tools used here
  % \begin{itemize}
  % \item CP-Viz, Simonis et al. 2010
    % \end{itemize}
% \end{itemize}
% \end{frame}

\begin{frame}
\frametitle{Tutorial Based on ECLiPSe ELearning Course}
\begin{itemize}
\item Self-study course in constraint programming
\item Supported by Cisco Systems and Silicon Valley Community Foundation
\item Multi-media format, video lectures, slides, handout etc
\item {\small \url{https://eclipseclp.org/ELearning/index.html}}
\end{itemize}
\end{frame}

\begin{frame}
\frametitle{Also Part of CRT-AI Constraint Week}
\begin{itemize}
\item Annual one week course on CP and Optimization in Ireland
\item Part of national training program for PhD students in AI
\item \url{https://www.crt-ai.ie/}
\end{itemize}
\end{frame}



\begin{frame}
\frametitle{Constraint Programming - in a nutshell}
\begin{itemize}
\item Declarative description of problems with
\begin{itemize}
\item {\em Variables} which range over (finite) sets of values
\item {\em Constraints} over subsets of variables which restrict possible value combinations
\item A {\em solution} is a value assignment which satisfies all constraints
\end{itemize}

\item Constraint propagation/reasoning
\begin{itemize}
\item Removing inconsistent values for variables
\item Detect failure if constraint can not be satisfied
\item Interaction of constraints via shared variables
\item Incomplete
\end{itemize}

\item Search
\begin{itemize}
\item User controlled assignment of values to variables
\item Each step triggers constraint propagation 
\end{itemize}
\item Different domains require/allow different methods
\end{itemize}
\end{frame}

\begin{frame}
  \frametitle{Constraint Programming is Different}
  \begin{itemize}
  \item Declarative Programming
    \begin{itemize}
    \item Concentrate on what you want
      \item Not how to get there
      \item Program != Algorithm
      \item Program = Model
    \end{itemize}
    \item Applied to Combinatorial Problems
      \begin{itemize}
        \item No complete polynomial algorithms known (exist?)
        \item CP less ad-hoc than heuristics
        \item Models can evolve
  \end{itemize}
  \end{itemize}
  \end{frame}
    
\begin{frame}
  \frametitle{A Subtractive Process}
  \begin{textblock}{4}(8,-3)
    \includegraphics[width=4cm]{../introduction/images/stages}
  \end{textblock}
  \vfill
  \begin{quote}
    ``Oh, bosh, as Mr. Ruskin says. Sculpture, per se, is the simplest thing in the world. All you have to do is to take a big chunk of marble and a hammer and chisel, make up your mind what you are about to create and chip off all the marble you don't want.''-Paris Gaulois.
  \end{quote}
  
  {\tiny Source: \url{https://quoteinvestigator.com/2014/06/22/chip-away/}}
\end{frame}

\begin{frame}
\frametitle{Basic Process}
\begin{center}
\begin{tikzpicture}
\node[shape=rectangle,fill=pantone127-4] (problem) at (2,5) {Problem};
\node (human) at (2,4) {Human};
\node[shape=rectangle,fill=pantone127-4] (model) at (2,3) {Model};
\node (solver) at (2,2) {Constraint Solver/Search};
\node[shape=rectangle,fill=pantone157-8] (solution) at (2,1) {Solution};
\draw[-] (problem) -- (human);
\draw[->] (human) -- (model);
\draw[-] (model) -- (solver);
\draw[->] (solver) -- (solution);
\end{tikzpicture}
\end{center}
\end{frame}

\begin{frame}
\frametitle{More Realistic}
\begin{center}
\begin{tikzpicture}[xscale=4.0]
\node[shape=rectangle,fill=pantone127-4] (problem) at (2,5) {Problem};
\node (human) at (2,4) {Human};
\node[shape=rectangle,fill=pantone127-4] (model) at (2,3) {Model};
\node (solver) at (2,2) {Constraint Solver/Search};
\node[shape=rectangle,fill=pantone24-8] (hangs) at (1,1) {Hangs};
\node[shape=rectangle,fill=pantone157-8] (solution) at (2,1) {Solution};
\node[shape=rectangle,fill=pantone24-8] (wrong) at (3,1) {Wrong Solution};
\draw[-] (problem) -- (human);
\draw[->] (human) -- (model);
\draw[-] (model) -- (solver);
\draw[->] (solver) -- (hangs);
\draw[->] (solver) -- (solution);
\draw[->] (solver) -- (wrong);
\draw[->] (hangs) -- (1,4) -- (human);
\draw[->] (wrong) -- (3,4) -- (human);
\end{tikzpicture}
\end{center}
\end{frame}

\begin{frame}
\frametitle{Dual Role of Model}
\begin{itemize}
\item Allows Human to Express Problem
\begin{itemize}
\item Close to Problem Domain
\item Constraints as Abstractions
\end{itemize}
\item Allows Solver to Execute
\begin{itemize}
\item Variables as Communication Mechanism
\item Constraints as Algorithms
\end{itemize}
\end{itemize}
\end{frame}

\begin{frame}
\frametitle{Modelling Frameworks}
\begin{itemize}
\item \href{https://www.minizinc.org/}{MiniZinc} (NICTA, Monash University, Australia)
\item \href{https://github.com/eomahony/Numberjack}{NumberJack} (Insight, Ireland) 
\item \href{https://www-users.york.ac.uk/peter.nightingale/savilerow//}{EssencePrime/SavilleRow} (UK) 
\item \href{https://cpmpy.readthedocs.io/en/latest/}{CPMpy} (KU Leuven)
\item Allow use of multiple back-end solvers
\item Compile model into variants for each solver
\item A priori solver independent model(CP, MIP, SAT)
\end{itemize}
\end{frame} 

\begin{frame}
\frametitle{Framework Process}
\begin{center}
\begin{tikzpicture}[xscale=2]
\node[shape=rectangle,fill=pantone127-4] (problem) at (2.5,5) {Problem};
\node (human) at (2.5,4) {Human};
\node[shape=rectangle,fill=pantone127-4] (model) at (2.5,3) {Model};
\node (compile) at (2.5,2) {Compile/Reformulate};
\node (solver1) at (1,1) {CP};
\node (solver2) at (2,1) {MIP};
\node (solver3) at (3,1) {SAT};
\node (solver4) at (4,1) {Other};
\node[shape=rectangle,fill=pantone157-8] (solution1) at (1,0) {Solution};
\node[shape=rectangle,fill=pantone157-8] (solution2) at (2,0) {Solution};
\node[shape=rectangle,fill=pantone157-8] (solution3) at (3,0) {Solution};
\node[shape=rectangle,fill=pantone157-8] (solution4) at (4,0) {Solution};
\draw[-] (problem) -- (human);
\draw[->] (human) -- (model);
\draw[-] (model) -- (compile);
\draw[-] (compile) -- (solver1);
\draw[-] (compile) -- (solver2);
\draw[-] (compile) -- (solver3);
\draw[-] (compile) -- (solver4);
\draw[->] (solver1) -- (solution1);
\draw[->] (solver2) -- (solution2);
\draw[->] (solver3) -- (solution3);
\draw[->] (solver4) -- (solution4);
\end{tikzpicture}
\end{center}
\end{frame}

% \begin{frame}
  % \frametitle{Do It Now!}
  % \begin{itemize}
  % \item Download and install Minizinc
  % \item \url{https://www.minizinc.org/}
  % \end{itemize}
% \end{frame}


\begin{frame}
\frametitle{Why use Puzzles as Examples?}
\begin{itemize}
\item Easy to understand the problem
\item Solvable by hand without specialized knowledge
\item Possible to compare automated to manual solving process
\end{itemize}

\begin{quote}
The puzzle, though inanimate, is presented as a solvable problem without lasting negative consequences, a very low-risk low-reward situation. By being a puzzle, the object is attempting to convince the user that it must be completed.
\end{quote}
{\tiny Source: \href{https://www.everydayrhetoric.com/2020/03/the-rhetoric-of-puzzles/}{Every Day Rhetoric}}
\end{frame}


\mode<all>{
\part{Basic Constraint Propagation}
}
\begin{frame}
\partpage
\end{frame}
\mode<all>{
\begin{frame}
\frametitle{Objectives}
\begin{itemize}
\item Overview of Core Constraint Programming
\item Three Main Concepts
\begin{itemize}
\item Constraint Propagation
\item Global Constraints
\item Customizing Search
\end{itemize}
%\item Get Some Experience with MiniZinc
\item Topics will be treated in more detail in later parts of the school
\item Based on Examples, not Formal Description
\end{itemize}
\end{frame}

\begin{frame}
\frametitle{Outline}
\begin{itemize}
\item Why Constraint Programming?
\item Constraint Propagation
\item Global Constraints
\item Customizing Search
%\item Choices in Modelling
%\item Decomposition
%\item More Global Constraints
%\item Methodology
%\item What is missing?
\end{itemize}
\end{frame}

% \begin{frame}
  % \frametitle{Using MiniZinc IDE}
  % \begin{itemize}
  % \item Developed in the Australian NICTA project
  % \item Maintained by Monash University
  % \item Modelling tool with multiple back-end solvers
  % \item Available from \url{https://www.minizinc.org/}
  % \end{itemize}
% \end{frame}


% \begin{frame}
% \frametitle{Examples in ECLiPSe}
% \begin{itemize}
% \item Open sourced constraint programming language
% \item Development goes back to 1985
% \item ECRC, ICL, IC-Parc, PTL, Cisco
% \item \url{https://eclipseclp.org/}
% \item Specialities
% \begin{itemize}
% \item Develop new solvers for specific domains
% \item Integration with MIP
% \end{itemize}
% \item Not included in bundled MiniZinc IDE
% \item Specialized visualization tools used here
  % \begin{itemize}
  % \item CP-Viz, Simonis et al. 2010
    % \end{itemize}
% \end{itemize}
% \end{frame}

\begin{frame}
\frametitle{Tutorial Based on ECLiPSe ELearning Course}
\begin{itemize}
\item Self-study course in constraint programming
\item Supported by Cisco Systems and Silicon Valley Community Foundation
\item Multi-media format, video lectures, slides, handout etc
\item {\small \url{https://eclipseclp.org/ELearning/index.html}}
\end{itemize}
\end{frame}

\begin{frame}
\frametitle{Also Part of CRT-AI Constraint Week}
\begin{itemize}
\item Annual one week course on CP and Optimization in Ireland
\item Part of national training program for PhD students in AI
\item \url{https://www.crt-ai.ie/}
\end{itemize}
\end{frame}



\begin{frame}
\frametitle{Constraint Programming - in a nutshell}
\begin{itemize}
\item Declarative description of problems with
\begin{itemize}
\item {\em Variables} which range over (finite) sets of values
\item {\em Constraints} over subsets of variables which restrict possible value combinations
\item A {\em solution} is a value assignment which satisfies all constraints
\end{itemize}

\item Constraint propagation/reasoning
\begin{itemize}
\item Removing inconsistent values for variables
\item Detect failure if constraint can not be satisfied
\item Interaction of constraints via shared variables
\item Incomplete
\end{itemize}

\item Search
\begin{itemize}
\item User controlled assignment of values to variables
\item Each step triggers constraint propagation 
\end{itemize}
\item Different domains require/allow different methods
\end{itemize}
\end{frame}

\begin{frame}
  \frametitle{Constraint Programming is Different}
  \begin{itemize}
  \item Declarative Programming
    \begin{itemize}
    \item Concentrate on what you want
      \item Not how to get there
      \item Program != Algorithm
      \item Program = Model
    \end{itemize}
    \item Applied to Combinatorial Problems
      \begin{itemize}
        \item No complete polynomial algorithms known (exist?)
        \item CP less ad-hoc than heuristics
        \item Models can evolve
  \end{itemize}
  \end{itemize}
  \end{frame}
    
\begin{frame}
  \frametitle{A Subtractive Process}
  \begin{textblock}{4}(8,-3)
    \includegraphics[width=4cm]{../introduction/images/stages}
  \end{textblock}
  \vfill
  \begin{quote}
    ``Oh, bosh, as Mr. Ruskin says. Sculpture, per se, is the simplest thing in the world. All you have to do is to take a big chunk of marble and a hammer and chisel, make up your mind what you are about to create and chip off all the marble you don't want.''-Paris Gaulois.
  \end{quote}
  
  {\tiny Source: \url{https://quoteinvestigator.com/2014/06/22/chip-away/}}
\end{frame}

\begin{frame}
\frametitle{Basic Process}
\begin{center}
\begin{tikzpicture}
\node[shape=rectangle,fill=pantone127-4] (problem) at (2,5) {Problem};
\node (human) at (2,4) {Human};
\node[shape=rectangle,fill=pantone127-4] (model) at (2,3) {Model};
\node (solver) at (2,2) {Constraint Solver/Search};
\node[shape=rectangle,fill=pantone157-8] (solution) at (2,1) {Solution};
\draw[-] (problem) -- (human);
\draw[->] (human) -- (model);
\draw[-] (model) -- (solver);
\draw[->] (solver) -- (solution);
\end{tikzpicture}
\end{center}
\end{frame}

\begin{frame}
\frametitle{More Realistic}
\begin{center}
\begin{tikzpicture}[xscale=4.0]
\node[shape=rectangle,fill=pantone127-4] (problem) at (2,5) {Problem};
\node (human) at (2,4) {Human};
\node[shape=rectangle,fill=pantone127-4] (model) at (2,3) {Model};
\node (solver) at (2,2) {Constraint Solver/Search};
\node[shape=rectangle,fill=pantone24-8] (hangs) at (1,1) {Hangs};
\node[shape=rectangle,fill=pantone157-8] (solution) at (2,1) {Solution};
\node[shape=rectangle,fill=pantone24-8] (wrong) at (3,1) {Wrong Solution};
\draw[-] (problem) -- (human);
\draw[->] (human) -- (model);
\draw[-] (model) -- (solver);
\draw[->] (solver) -- (hangs);
\draw[->] (solver) -- (solution);
\draw[->] (solver) -- (wrong);
\draw[->] (hangs) -- (1,4) -- (human);
\draw[->] (wrong) -- (3,4) -- (human);
\end{tikzpicture}
\end{center}
\end{frame}

\begin{frame}
\frametitle{Dual Role of Model}
\begin{itemize}
\item Allows Human to Express Problem
\begin{itemize}
\item Close to Problem Domain
\item Constraints as Abstractions
\end{itemize}
\item Allows Solver to Execute
\begin{itemize}
\item Variables as Communication Mechanism
\item Constraints as Algorithms
\end{itemize}
\end{itemize}
\end{frame}

\begin{frame}
\frametitle{Modelling Frameworks}
\begin{itemize}
\item \href{https://www.minizinc.org/}{MiniZinc} (NICTA, Monash University, Australia)
\item \href{https://github.com/eomahony/Numberjack}{NumberJack} (Insight, Ireland) 
\item \href{https://www-users.york.ac.uk/peter.nightingale/savilerow//}{EssencePrime/SavilleRow} (UK) 
\item \href{https://cpmpy.readthedocs.io/en/latest/}{CPMpy} (KU Leuven)
\item Allow use of multiple back-end solvers
\item Compile model into variants for each solver
\item A priori solver independent model(CP, MIP, SAT)
\end{itemize}
\end{frame} 

\begin{frame}
\frametitle{Framework Process}
\begin{center}
\begin{tikzpicture}[xscale=2]
\node[shape=rectangle,fill=pantone127-4] (problem) at (2.5,5) {Problem};
\node (human) at (2.5,4) {Human};
\node[shape=rectangle,fill=pantone127-4] (model) at (2.5,3) {Model};
\node (compile) at (2.5,2) {Compile/Reformulate};
\node (solver1) at (1,1) {CP};
\node (solver2) at (2,1) {MIP};
\node (solver3) at (3,1) {SAT};
\node (solver4) at (4,1) {Other};
\node[shape=rectangle,fill=pantone157-8] (solution1) at (1,0) {Solution};
\node[shape=rectangle,fill=pantone157-8] (solution2) at (2,0) {Solution};
\node[shape=rectangle,fill=pantone157-8] (solution3) at (3,0) {Solution};
\node[shape=rectangle,fill=pantone157-8] (solution4) at (4,0) {Solution};
\draw[-] (problem) -- (human);
\draw[->] (human) -- (model);
\draw[-] (model) -- (compile);
\draw[-] (compile) -- (solver1);
\draw[-] (compile) -- (solver2);
\draw[-] (compile) -- (solver3);
\draw[-] (compile) -- (solver4);
\draw[->] (solver1) -- (solution1);
\draw[->] (solver2) -- (solution2);
\draw[->] (solver3) -- (solution3);
\draw[->] (solver4) -- (solution4);
\end{tikzpicture}
\end{center}
\end{frame}

% \begin{frame}
  % \frametitle{Do It Now!}
  % \begin{itemize}
  % \item Download and install Minizinc
  % \item \url{https://www.minizinc.org/}
  % \end{itemize}
% \end{frame}


\begin{frame}
\frametitle{Why use Puzzles as Examples?}
\begin{itemize}
\item Easy to understand the problem
\item Solvable by hand without specialized knowledge
\item Possible to compare automated to manual solving process
\end{itemize}

\begin{quote}
The puzzle, though inanimate, is presented as a solvable problem without lasting negative consequences, a very low-risk low-reward situation. By being a puzzle, the object is attempting to convince the user that it must be completed.
\end{quote}
{\tiny Source: \href{https://www.everydayrhetoric.com/2020/03/the-rhetoric-of-puzzles/}{Every Day Rhetoric}}
\end{frame}


\mode<all>{
\part{Basic Constraint Propagation}
}
\begin{frame}
\partpage
\end{frame}
\mode<all>{
\begin{frame}
\frametitle{Objectives}
\begin{itemize}
\item Overview of Core Constraint Programming
\item Three Main Concepts
\begin{itemize}
\item Constraint Propagation
\item Global Constraints
\item Customizing Search
\end{itemize}
%\item Get Some Experience with MiniZinc
\item Topics will be treated in more detail in later parts of the school
\item Based on Examples, not Formal Description
\end{itemize}
\end{frame}

\begin{frame}
\frametitle{Outline}
\begin{itemize}
\item Why Constraint Programming?
\item Constraint Propagation
\item Global Constraints
\item Customizing Search
%\item Choices in Modelling
%\item Decomposition
%\item More Global Constraints
%\item Methodology
%\item What is missing?
\end{itemize}
\end{frame}

% \begin{frame}
  % \frametitle{Using MiniZinc IDE}
  % \begin{itemize}
  % \item Developed in the Australian NICTA project
  % \item Maintained by Monash University
  % \item Modelling tool with multiple back-end solvers
  % \item Available from \url{https://www.minizinc.org/}
  % \end{itemize}
% \end{frame}


% \begin{frame}
% \frametitle{Examples in ECLiPSe}
% \begin{itemize}
% \item Open sourced constraint programming language
% \item Development goes back to 1985
% \item ECRC, ICL, IC-Parc, PTL, Cisco
% \item \url{https://eclipseclp.org/}
% \item Specialities
% \begin{itemize}
% \item Develop new solvers for specific domains
% \item Integration with MIP
% \end{itemize}
% \item Not included in bundled MiniZinc IDE
% \item Specialized visualization tools used here
  % \begin{itemize}
  % \item CP-Viz, Simonis et al. 2010
    % \end{itemize}
% \end{itemize}
% \end{frame}

\begin{frame}
\frametitle{Tutorial Based on ECLiPSe ELearning Course}
\begin{itemize}
\item Self-study course in constraint programming
\item Supported by Cisco Systems and Silicon Valley Community Foundation
\item Multi-media format, video lectures, slides, handout etc
\item {\small \url{https://eclipseclp.org/ELearning/index.html}}
\end{itemize}
\end{frame}

\begin{frame}
\frametitle{Also Part of CRT-AI Constraint Week}
\begin{itemize}
\item Annual one week course on CP and Optimization in Ireland
\item Part of national training program for PhD students in AI
\item \url{https://www.crt-ai.ie/}
\end{itemize}
\end{frame}



\begin{frame}
\frametitle{Constraint Programming - in a nutshell}
\begin{itemize}
\item Declarative description of problems with
\begin{itemize}
\item {\em Variables} which range over (finite) sets of values
\item {\em Constraints} over subsets of variables which restrict possible value combinations
\item A {\em solution} is a value assignment which satisfies all constraints
\end{itemize}

\item Constraint propagation/reasoning
\begin{itemize}
\item Removing inconsistent values for variables
\item Detect failure if constraint can not be satisfied
\item Interaction of constraints via shared variables
\item Incomplete
\end{itemize}

\item Search
\begin{itemize}
\item User controlled assignment of values to variables
\item Each step triggers constraint propagation 
\end{itemize}
\item Different domains require/allow different methods
\end{itemize}
\end{frame}

\begin{frame}
  \frametitle{Constraint Programming is Different}
  \begin{itemize}
  \item Declarative Programming
    \begin{itemize}
    \item Concentrate on what you want
      \item Not how to get there
      \item Program != Algorithm
      \item Program = Model
    \end{itemize}
    \item Applied to Combinatorial Problems
      \begin{itemize}
        \item No complete polynomial algorithms known (exist?)
        \item CP less ad-hoc than heuristics
        \item Models can evolve
  \end{itemize}
  \end{itemize}
  \end{frame}
    
\begin{frame}
  \frametitle{A Subtractive Process}
  \begin{textblock}{4}(8,-3)
    \includegraphics[width=4cm]{../introduction/images/stages}
  \end{textblock}
  \vfill
  \begin{quote}
    ``Oh, bosh, as Mr. Ruskin says. Sculpture, per se, is the simplest thing in the world. All you have to do is to take a big chunk of marble and a hammer and chisel, make up your mind what you are about to create and chip off all the marble you don't want.''-Paris Gaulois.
  \end{quote}
  
  {\tiny Source: \url{https://quoteinvestigator.com/2014/06/22/chip-away/}}
\end{frame}

\begin{frame}
\frametitle{Basic Process}
\begin{center}
\begin{tikzpicture}
\node[shape=rectangle,fill=pantone127-4] (problem) at (2,5) {Problem};
\node (human) at (2,4) {Human};
\node[shape=rectangle,fill=pantone127-4] (model) at (2,3) {Model};
\node (solver) at (2,2) {Constraint Solver/Search};
\node[shape=rectangle,fill=pantone157-8] (solution) at (2,1) {Solution};
\draw[-] (problem) -- (human);
\draw[->] (human) -- (model);
\draw[-] (model) -- (solver);
\draw[->] (solver) -- (solution);
\end{tikzpicture}
\end{center}
\end{frame}

\begin{frame}
\frametitle{More Realistic}
\begin{center}
\begin{tikzpicture}[xscale=4.0]
\node[shape=rectangle,fill=pantone127-4] (problem) at (2,5) {Problem};
\node (human) at (2,4) {Human};
\node[shape=rectangle,fill=pantone127-4] (model) at (2,3) {Model};
\node (solver) at (2,2) {Constraint Solver/Search};
\node[shape=rectangle,fill=pantone24-8] (hangs) at (1,1) {Hangs};
\node[shape=rectangle,fill=pantone157-8] (solution) at (2,1) {Solution};
\node[shape=rectangle,fill=pantone24-8] (wrong) at (3,1) {Wrong Solution};
\draw[-] (problem) -- (human);
\draw[->] (human) -- (model);
\draw[-] (model) -- (solver);
\draw[->] (solver) -- (hangs);
\draw[->] (solver) -- (solution);
\draw[->] (solver) -- (wrong);
\draw[->] (hangs) -- (1,4) -- (human);
\draw[->] (wrong) -- (3,4) -- (human);
\end{tikzpicture}
\end{center}
\end{frame}

\begin{frame}
\frametitle{Dual Role of Model}
\begin{itemize}
\item Allows Human to Express Problem
\begin{itemize}
\item Close to Problem Domain
\item Constraints as Abstractions
\end{itemize}
\item Allows Solver to Execute
\begin{itemize}
\item Variables as Communication Mechanism
\item Constraints as Algorithms
\end{itemize}
\end{itemize}
\end{frame}

\begin{frame}
\frametitle{Modelling Frameworks}
\begin{itemize}
\item \href{https://www.minizinc.org/}{MiniZinc} (NICTA, Monash University, Australia)
\item \href{https://github.com/eomahony/Numberjack}{NumberJack} (Insight, Ireland) 
\item \href{https://www-users.york.ac.uk/peter.nightingale/savilerow//}{EssencePrime/SavilleRow} (UK) 
\item \href{https://cpmpy.readthedocs.io/en/latest/}{CPMpy} (KU Leuven)
\item Allow use of multiple back-end solvers
\item Compile model into variants for each solver
\item A priori solver independent model(CP, MIP, SAT)
\end{itemize}
\end{frame} 

\begin{frame}
\frametitle{Framework Process}
\begin{center}
\begin{tikzpicture}[xscale=2]
\node[shape=rectangle,fill=pantone127-4] (problem) at (2.5,5) {Problem};
\node (human) at (2.5,4) {Human};
\node[shape=rectangle,fill=pantone127-4] (model) at (2.5,3) {Model};
\node (compile) at (2.5,2) {Compile/Reformulate};
\node (solver1) at (1,1) {CP};
\node (solver2) at (2,1) {MIP};
\node (solver3) at (3,1) {SAT};
\node (solver4) at (4,1) {Other};
\node[shape=rectangle,fill=pantone157-8] (solution1) at (1,0) {Solution};
\node[shape=rectangle,fill=pantone157-8] (solution2) at (2,0) {Solution};
\node[shape=rectangle,fill=pantone157-8] (solution3) at (3,0) {Solution};
\node[shape=rectangle,fill=pantone157-8] (solution4) at (4,0) {Solution};
\draw[-] (problem) -- (human);
\draw[->] (human) -- (model);
\draw[-] (model) -- (compile);
\draw[-] (compile) -- (solver1);
\draw[-] (compile) -- (solver2);
\draw[-] (compile) -- (solver3);
\draw[-] (compile) -- (solver4);
\draw[->] (solver1) -- (solution1);
\draw[->] (solver2) -- (solution2);
\draw[->] (solver3) -- (solution3);
\draw[->] (solver4) -- (solution4);
\end{tikzpicture}
\end{center}
\end{frame}

% \begin{frame}
  % \frametitle{Do It Now!}
  % \begin{itemize}
  % \item Download and install Minizinc
  % \item \url{https://www.minizinc.org/}
  % \end{itemize}
% \end{frame}


\begin{frame}
\frametitle{Why use Puzzles as Examples?}
\begin{itemize}
\item Easy to understand the problem
\item Solvable by hand without specialized knowledge
\item Possible to compare automated to manual solving process
\end{itemize}

\begin{quote}
The puzzle, though inanimate, is presented as a solvable problem without lasting negative consequences, a very low-risk low-reward situation. By being a puzzle, the object is attempting to convince the user that it must be completed.
\end{quote}
{\tiny Source: \href{https://www.everydayrhetoric.com/2020/03/the-rhetoric-of-puzzles/}{Every Day Rhetoric}}
\end{frame}


\mode<all>{
\part{Basic Constraint Propagation}
}
\begin{frame}
\partpage
\end{frame}
\mode<all>{
\begin{frame}
\frametitle{Objectives}
\begin{itemize}
\item Overview of Core Constraint Programming
\item Three Main Concepts
\begin{itemize}
\item Constraint Propagation
\item Global Constraints
\item Customizing Search
\end{itemize}
%\item Get Some Experience with MiniZinc
\item Topics will be treated in more detail in later parts of the school
\item Based on Examples, not Formal Description
\end{itemize}
\end{frame}

\begin{frame}
\frametitle{Outline}
\begin{itemize}
\item Why Constraint Programming?
\item Constraint Propagation
\item Global Constraints
\item Customizing Search
%\item Choices in Modelling
%\item Decomposition
%\item More Global Constraints
%\item Methodology
%\item What is missing?
\end{itemize}
\end{frame}

% \begin{frame}
  % \frametitle{Using MiniZinc IDE}
  % \begin{itemize}
  % \item Developed in the Australian NICTA project
  % \item Maintained by Monash University
  % \item Modelling tool with multiple back-end solvers
  % \item Available from \url{https://www.minizinc.org/}
  % \end{itemize}
% \end{frame}


% \begin{frame}
% \frametitle{Examples in ECLiPSe}
% \begin{itemize}
% \item Open sourced constraint programming language
% \item Development goes back to 1985
% \item ECRC, ICL, IC-Parc, PTL, Cisco
% \item \url{https://eclipseclp.org/}
% \item Specialities
% \begin{itemize}
% \item Develop new solvers for specific domains
% \item Integration with MIP
% \end{itemize}
% \item Not included in bundled MiniZinc IDE
% \item Specialized visualization tools used here
  % \begin{itemize}
  % \item CP-Viz, Simonis et al. 2010
    % \end{itemize}
% \end{itemize}
% \end{frame}

\begin{frame}
\frametitle{Tutorial Based on ECLiPSe ELearning Course}
\begin{itemize}
\item Self-study course in constraint programming
\item Supported by Cisco Systems and Silicon Valley Community Foundation
\item Multi-media format, video lectures, slides, handout etc
\item {\small \url{https://eclipseclp.org/ELearning/index.html}}
\end{itemize}
\end{frame}

\begin{frame}
\frametitle{Also Part of CRT-AI Constraint Week}
\begin{itemize}
\item Annual one week course on CP and Optimization in Ireland
\item Part of national training program for PhD students in AI
\item \url{https://www.crt-ai.ie/}
\end{itemize}
\end{frame}



\begin{frame}
\frametitle{Constraint Programming - in a nutshell}
\begin{itemize}
\item Declarative description of problems with
\begin{itemize}
\item {\em Variables} which range over (finite) sets of values
\item {\em Constraints} over subsets of variables which restrict possible value combinations
\item A {\em solution} is a value assignment which satisfies all constraints
\end{itemize}

\item Constraint propagation/reasoning
\begin{itemize}
\item Removing inconsistent values for variables
\item Detect failure if constraint can not be satisfied
\item Interaction of constraints via shared variables
\item Incomplete
\end{itemize}

\item Search
\begin{itemize}
\item User controlled assignment of values to variables
\item Each step triggers constraint propagation 
\end{itemize}
\item Different domains require/allow different methods
\end{itemize}
\end{frame}

\begin{frame}
  \frametitle{Constraint Programming is Different}
  \begin{itemize}
  \item Declarative Programming
    \begin{itemize}
    \item Concentrate on what you want
      \item Not how to get there
      \item Program != Algorithm
      \item Program = Model
    \end{itemize}
    \item Applied to Combinatorial Problems
      \begin{itemize}
        \item No complete polynomial algorithms known (exist?)
        \item CP less ad-hoc than heuristics
        \item Models can evolve
  \end{itemize}
  \end{itemize}
  \end{frame}
    
\begin{frame}
  \frametitle{A Subtractive Process}
  \begin{textblock}{4}(8,-3)
    \includegraphics[width=4cm]{../introduction/images/stages}
  \end{textblock}
  \vfill
  \begin{quote}
    ``Oh, bosh, as Mr. Ruskin says. Sculpture, per se, is the simplest thing in the world. All you have to do is to take a big chunk of marble and a hammer and chisel, make up your mind what you are about to create and chip off all the marble you don't want.''-Paris Gaulois.
  \end{quote}
  
  {\tiny Source: \url{https://quoteinvestigator.com/2014/06/22/chip-away/}}
\end{frame}

\begin{frame}
\frametitle{Basic Process}
\begin{center}
\begin{tikzpicture}
\node[shape=rectangle,fill=pantone127-4] (problem) at (2,5) {Problem};
\node (human) at (2,4) {Human};
\node[shape=rectangle,fill=pantone127-4] (model) at (2,3) {Model};
\node (solver) at (2,2) {Constraint Solver/Search};
\node[shape=rectangle,fill=pantone157-8] (solution) at (2,1) {Solution};
\draw[-] (problem) -- (human);
\draw[->] (human) -- (model);
\draw[-] (model) -- (solver);
\draw[->] (solver) -- (solution);
\end{tikzpicture}
\end{center}
\end{frame}

\begin{frame}
\frametitle{More Realistic}
\begin{center}
\begin{tikzpicture}[xscale=4.0]
\node[shape=rectangle,fill=pantone127-4] (problem) at (2,5) {Problem};
\node (human) at (2,4) {Human};
\node[shape=rectangle,fill=pantone127-4] (model) at (2,3) {Model};
\node (solver) at (2,2) {Constraint Solver/Search};
\node[shape=rectangle,fill=pantone24-8] (hangs) at (1,1) {Hangs};
\node[shape=rectangle,fill=pantone157-8] (solution) at (2,1) {Solution};
\node[shape=rectangle,fill=pantone24-8] (wrong) at (3,1) {Wrong Solution};
\draw[-] (problem) -- (human);
\draw[->] (human) -- (model);
\draw[-] (model) -- (solver);
\draw[->] (solver) -- (hangs);
\draw[->] (solver) -- (solution);
\draw[->] (solver) -- (wrong);
\draw[->] (hangs) -- (1,4) -- (human);
\draw[->] (wrong) -- (3,4) -- (human);
\end{tikzpicture}
\end{center}
\end{frame}

\begin{frame}
\frametitle{Dual Role of Model}
\begin{itemize}
\item Allows Human to Express Problem
\begin{itemize}
\item Close to Problem Domain
\item Constraints as Abstractions
\end{itemize}
\item Allows Solver to Execute
\begin{itemize}
\item Variables as Communication Mechanism
\item Constraints as Algorithms
\end{itemize}
\end{itemize}
\end{frame}

\begin{frame}
\frametitle{Modelling Frameworks}
\begin{itemize}
\item \href{https://www.minizinc.org/}{MiniZinc} (NICTA, Monash University, Australia)
\item \href{https://github.com/eomahony/Numberjack}{NumberJack} (Insight, Ireland) 
\item \href{https://www-users.york.ac.uk/peter.nightingale/savilerow//}{EssencePrime/SavilleRow} (UK) 
\item \href{https://cpmpy.readthedocs.io/en/latest/}{CPMpy} (KU Leuven)
\item Allow use of multiple back-end solvers
\item Compile model into variants for each solver
\item A priori solver independent model(CP, MIP, SAT)
\end{itemize}
\end{frame} 

\begin{frame}
\frametitle{Framework Process}
\begin{center}
\begin{tikzpicture}[xscale=2]
\node[shape=rectangle,fill=pantone127-4] (problem) at (2.5,5) {Problem};
\node (human) at (2.5,4) {Human};
\node[shape=rectangle,fill=pantone127-4] (model) at (2.5,3) {Model};
\node (compile) at (2.5,2) {Compile/Reformulate};
\node (solver1) at (1,1) {CP};
\node (solver2) at (2,1) {MIP};
\node (solver3) at (3,1) {SAT};
\node (solver4) at (4,1) {Other};
\node[shape=rectangle,fill=pantone157-8] (solution1) at (1,0) {Solution};
\node[shape=rectangle,fill=pantone157-8] (solution2) at (2,0) {Solution};
\node[shape=rectangle,fill=pantone157-8] (solution3) at (3,0) {Solution};
\node[shape=rectangle,fill=pantone157-8] (solution4) at (4,0) {Solution};
\draw[-] (problem) -- (human);
\draw[->] (human) -- (model);
\draw[-] (model) -- (compile);
\draw[-] (compile) -- (solver1);
\draw[-] (compile) -- (solver2);
\draw[-] (compile) -- (solver3);
\draw[-] (compile) -- (solver4);
\draw[->] (solver1) -- (solution1);
\draw[->] (solver2) -- (solution2);
\draw[->] (solver3) -- (solution3);
\draw[->] (solver4) -- (solution4);
\end{tikzpicture}
\end{center}
\end{frame}

% \begin{frame}
  % \frametitle{Do It Now!}
  % \begin{itemize}
  % \item Download and install Minizinc
  % \item \url{https://www.minizinc.org/}
  % \end{itemize}
% \end{frame}


\begin{frame}
\frametitle{Why use Puzzles as Examples?}
\begin{itemize}
\item Easy to understand the problem
\item Solvable by hand without specialized knowledge
\item Possible to compare automated to manual solving process
\end{itemize}

\begin{quote}
The puzzle, though inanimate, is presented as a solvable problem without lasting negative consequences, a very low-risk low-reward situation. By being a puzzle, the object is attempting to convince the user that it must be completed.
\end{quote}
{\tiny Source: \href{https://www.everydayrhetoric.com/2020/03/the-rhetoric-of-puzzles/}{Every Day Rhetoric}}
\end{frame}


\mode<all>{
\part{Basic Constraint Propagation}
}
\begin{frame}
\partpage
\end{frame}
\mode<all>{
\input{../sendmore/body}
}

\mode<all>{
\part{Global Constraints}
}
\begin{frame}
\partpage
\end{frame}
\mode<all>{
\input{../sudoku/body}
}

\clearpage
\mode<all>{
\part{Customizing Search}
}
\begin{frame}
\partpage
\end{frame}
\mode<all>{
\input{../nqueen/body}
}


%% \clearpage
%% \mode<all>{
%%   \part{Problem Decomposition}
%% }
%% \begin{frame}
%%   \partpage
%% \end{frame}
%% \mode<all>{
%%   \input{../party/body}
%%   }

%% \clearpage
%% \mode<all>{
%%   \part{More Global Constraints}
%% }
%% \begin{frame}
%%   \partpage
%% \end{frame}
%% \mode<all>{
%%   \input{../car/body}
%%   }

%% \clearpage
%% \mode<all>{
%%   \part{Methodology}
%% }
%% \begin{frame}
%%   \partpage
%% \end{frame}
%% \mode<all>{
%%   \input{../methodology/body}
%%   }

\clearpage
\mode<all>{
\part{What is missing?}
}
\begin{frame}
\partpage
\end{frame}

\begin{frame}
\frametitle{Many Specialized Topics}
\begin{itemize}
\item How to design efficient core engine
\item Hybrids with LP/MIP tools
\item Hybrids with SAT
\item Symmetry breaking
\item Use of MDD/BDD to encode sets of solutions
\item High level modelling tools
\item Debugging/visualization
\end{itemize}
\end{frame}

\begin{frame}
\frametitle{Reformulation}
\begin{itemize}
\item Just because the user has modelled it this way, it doesn't mean we have to solve it that way
\begin{itemize}
\item Replace some constraint(s) by other, equivalent constraints
\item Because we don't have that constraint in our system
\item For performance
\end{itemize}
\end{itemize}
\end{frame}

\begin{frame}
\frametitle{Learning}
\begin{itemize}
\item While solving the problem we can learn how to strengthen the model/search 
\begin{itemize}
\item Understand which constraints/method contribute to propagation and change schedule
\item Learn no-good constraints by explaining failure
\item Adapt search strategy based on search experience
\end{itemize}
\end{itemize}
\end{frame}

%% \begin{frame}
%% \frametitle{Refined Process}
%% \begin{center}
%% \begin{tikzpicture}
%% \node[shape=rectangle,fill=pantone127-4] (problem) at (3,7) {Problem};
%% \node (human) at (3,6) {Human};
%% \node[shape=rectangle,fill=pantone127-4] (model) at (3,5) {User Model};
%% \node (reformulation) at (3,4) {Reformulation};
%% \node[shape=rectangle,fill=pantone127-4] (implementation) at (3,3) {Implementation Model};
%% \node (solver) at (3,2) {Constraint Solver/Search};
%% \node[shape=rectangle,fill=pantone157-8] (solution) at (3,1) {Solution};
%% \draw[-] (problem) -- (human);
%% \draw[->] (human) -- (model);
%% \draw[-] (model) -- (reformulation);
%% \draw[->] (reformulation) -- (implementation);
%% \draw[-] (implementation) -- (solver);
%% \draw[->] (solver) -- (0,2) -- (0,3) -- (implementation);
%% \draw[->] (solver) -- (solution);
%% \end{tikzpicture}
%% \end{center}
%% \end{frame}

%\begin{frame}
%\frametitle{What is CP actually used for?}
%\url{http://hsimonis.wordpress.com}
%\includegraphics[width=10cm]{../introduction/images/blog}
%\end{frame}

%% \begin{frame}
%% \frametitle{Exercises}
%% \begin{itemize}
%% \item Install Constraint Solver of your Choice
%% \item MiniZinc (MiningZinc) will be used in later sessions
%% \item Run Sendmoremoney, Sudoku, Queens examples
%% \item Can you choose which propagator to use for alldifferent?
%% \item Can you use different search strategies?
%% \end{itemize}
%% \end{frame}

\begin{frame}
\frametitle{More Learning Resources}
\begin{itemize}
\item Survey of Methods, Resources, and Formats for Teaching Constraint Programming
\begin{itemize}
\item by Tejas Santanam, Helmut Simonis
\item \url{https://doi.org/10.48550/arXiv.2403.12717}
\item Based on survey of community for WTCP 2023
\item \url{https://hsimonis.github.io/WTCP2023/}
\end{itemize}
\end{itemize}
\end{frame}


}

\mode<all>{
\part{Global Constraints}
}
\begin{frame}
\partpage
\end{frame}
\mode<all>{
\begin{frame}
\frametitle{Objectives}
\begin{itemize}
\item Overview of Core Constraint Programming
\item Three Main Concepts
\begin{itemize}
\item Constraint Propagation
\item Global Constraints
\item Customizing Search
\end{itemize}
%\item Get Some Experience with MiniZinc
\item Topics will be treated in more detail in later parts of the school
\item Based on Examples, not Formal Description
\end{itemize}
\end{frame}

\begin{frame}
\frametitle{Outline}
\begin{itemize}
\item Why Constraint Programming?
\item Constraint Propagation
\item Global Constraints
\item Customizing Search
%\item Choices in Modelling
%\item Decomposition
%\item More Global Constraints
%\item Methodology
%\item What is missing?
\end{itemize}
\end{frame}

% \begin{frame}
  % \frametitle{Using MiniZinc IDE}
  % \begin{itemize}
  % \item Developed in the Australian NICTA project
  % \item Maintained by Monash University
  % \item Modelling tool with multiple back-end solvers
  % \item Available from \url{https://www.minizinc.org/}
  % \end{itemize}
% \end{frame}


% \begin{frame}
% \frametitle{Examples in ECLiPSe}
% \begin{itemize}
% \item Open sourced constraint programming language
% \item Development goes back to 1985
% \item ECRC, ICL, IC-Parc, PTL, Cisco
% \item \url{https://eclipseclp.org/}
% \item Specialities
% \begin{itemize}
% \item Develop new solvers for specific domains
% \item Integration with MIP
% \end{itemize}
% \item Not included in bundled MiniZinc IDE
% \item Specialized visualization tools used here
  % \begin{itemize}
  % \item CP-Viz, Simonis et al. 2010
    % \end{itemize}
% \end{itemize}
% \end{frame}

\begin{frame}
\frametitle{Tutorial Based on ECLiPSe ELearning Course}
\begin{itemize}
\item Self-study course in constraint programming
\item Supported by Cisco Systems and Silicon Valley Community Foundation
\item Multi-media format, video lectures, slides, handout etc
\item {\small \url{https://eclipseclp.org/ELearning/index.html}}
\end{itemize}
\end{frame}

\begin{frame}
\frametitle{Also Part of CRT-AI Constraint Week}
\begin{itemize}
\item Annual one week course on CP and Optimization in Ireland
\item Part of national training program for PhD students in AI
\item \url{https://www.crt-ai.ie/}
\end{itemize}
\end{frame}



\begin{frame}
\frametitle{Constraint Programming - in a nutshell}
\begin{itemize}
\item Declarative description of problems with
\begin{itemize}
\item {\em Variables} which range over (finite) sets of values
\item {\em Constraints} over subsets of variables which restrict possible value combinations
\item A {\em solution} is a value assignment which satisfies all constraints
\end{itemize}

\item Constraint propagation/reasoning
\begin{itemize}
\item Removing inconsistent values for variables
\item Detect failure if constraint can not be satisfied
\item Interaction of constraints via shared variables
\item Incomplete
\end{itemize}

\item Search
\begin{itemize}
\item User controlled assignment of values to variables
\item Each step triggers constraint propagation 
\end{itemize}
\item Different domains require/allow different methods
\end{itemize}
\end{frame}

\begin{frame}
  \frametitle{Constraint Programming is Different}
  \begin{itemize}
  \item Declarative Programming
    \begin{itemize}
    \item Concentrate on what you want
      \item Not how to get there
      \item Program != Algorithm
      \item Program = Model
    \end{itemize}
    \item Applied to Combinatorial Problems
      \begin{itemize}
        \item No complete polynomial algorithms known (exist?)
        \item CP less ad-hoc than heuristics
        \item Models can evolve
  \end{itemize}
  \end{itemize}
  \end{frame}
    
\begin{frame}
  \frametitle{A Subtractive Process}
  \begin{textblock}{4}(8,-3)
    \includegraphics[width=4cm]{../introduction/images/stages}
  \end{textblock}
  \vfill
  \begin{quote}
    ``Oh, bosh, as Mr. Ruskin says. Sculpture, per se, is the simplest thing in the world. All you have to do is to take a big chunk of marble and a hammer and chisel, make up your mind what you are about to create and chip off all the marble you don't want.''-Paris Gaulois.
  \end{quote}
  
  {\tiny Source: \url{https://quoteinvestigator.com/2014/06/22/chip-away/}}
\end{frame}

\begin{frame}
\frametitle{Basic Process}
\begin{center}
\begin{tikzpicture}
\node[shape=rectangle,fill=pantone127-4] (problem) at (2,5) {Problem};
\node (human) at (2,4) {Human};
\node[shape=rectangle,fill=pantone127-4] (model) at (2,3) {Model};
\node (solver) at (2,2) {Constraint Solver/Search};
\node[shape=rectangle,fill=pantone157-8] (solution) at (2,1) {Solution};
\draw[-] (problem) -- (human);
\draw[->] (human) -- (model);
\draw[-] (model) -- (solver);
\draw[->] (solver) -- (solution);
\end{tikzpicture}
\end{center}
\end{frame}

\begin{frame}
\frametitle{More Realistic}
\begin{center}
\begin{tikzpicture}[xscale=4.0]
\node[shape=rectangle,fill=pantone127-4] (problem) at (2,5) {Problem};
\node (human) at (2,4) {Human};
\node[shape=rectangle,fill=pantone127-4] (model) at (2,3) {Model};
\node (solver) at (2,2) {Constraint Solver/Search};
\node[shape=rectangle,fill=pantone24-8] (hangs) at (1,1) {Hangs};
\node[shape=rectangle,fill=pantone157-8] (solution) at (2,1) {Solution};
\node[shape=rectangle,fill=pantone24-8] (wrong) at (3,1) {Wrong Solution};
\draw[-] (problem) -- (human);
\draw[->] (human) -- (model);
\draw[-] (model) -- (solver);
\draw[->] (solver) -- (hangs);
\draw[->] (solver) -- (solution);
\draw[->] (solver) -- (wrong);
\draw[->] (hangs) -- (1,4) -- (human);
\draw[->] (wrong) -- (3,4) -- (human);
\end{tikzpicture}
\end{center}
\end{frame}

\begin{frame}
\frametitle{Dual Role of Model}
\begin{itemize}
\item Allows Human to Express Problem
\begin{itemize}
\item Close to Problem Domain
\item Constraints as Abstractions
\end{itemize}
\item Allows Solver to Execute
\begin{itemize}
\item Variables as Communication Mechanism
\item Constraints as Algorithms
\end{itemize}
\end{itemize}
\end{frame}

\begin{frame}
\frametitle{Modelling Frameworks}
\begin{itemize}
\item \href{https://www.minizinc.org/}{MiniZinc} (NICTA, Monash University, Australia)
\item \href{https://github.com/eomahony/Numberjack}{NumberJack} (Insight, Ireland) 
\item \href{https://www-users.york.ac.uk/peter.nightingale/savilerow//}{EssencePrime/SavilleRow} (UK) 
\item \href{https://cpmpy.readthedocs.io/en/latest/}{CPMpy} (KU Leuven)
\item Allow use of multiple back-end solvers
\item Compile model into variants for each solver
\item A priori solver independent model(CP, MIP, SAT)
\end{itemize}
\end{frame} 

\begin{frame}
\frametitle{Framework Process}
\begin{center}
\begin{tikzpicture}[xscale=2]
\node[shape=rectangle,fill=pantone127-4] (problem) at (2.5,5) {Problem};
\node (human) at (2.5,4) {Human};
\node[shape=rectangle,fill=pantone127-4] (model) at (2.5,3) {Model};
\node (compile) at (2.5,2) {Compile/Reformulate};
\node (solver1) at (1,1) {CP};
\node (solver2) at (2,1) {MIP};
\node (solver3) at (3,1) {SAT};
\node (solver4) at (4,1) {Other};
\node[shape=rectangle,fill=pantone157-8] (solution1) at (1,0) {Solution};
\node[shape=rectangle,fill=pantone157-8] (solution2) at (2,0) {Solution};
\node[shape=rectangle,fill=pantone157-8] (solution3) at (3,0) {Solution};
\node[shape=rectangle,fill=pantone157-8] (solution4) at (4,0) {Solution};
\draw[-] (problem) -- (human);
\draw[->] (human) -- (model);
\draw[-] (model) -- (compile);
\draw[-] (compile) -- (solver1);
\draw[-] (compile) -- (solver2);
\draw[-] (compile) -- (solver3);
\draw[-] (compile) -- (solver4);
\draw[->] (solver1) -- (solution1);
\draw[->] (solver2) -- (solution2);
\draw[->] (solver3) -- (solution3);
\draw[->] (solver4) -- (solution4);
\end{tikzpicture}
\end{center}
\end{frame}

% \begin{frame}
  % \frametitle{Do It Now!}
  % \begin{itemize}
  % \item Download and install Minizinc
  % \item \url{https://www.minizinc.org/}
  % \end{itemize}
% \end{frame}


\begin{frame}
\frametitle{Why use Puzzles as Examples?}
\begin{itemize}
\item Easy to understand the problem
\item Solvable by hand without specialized knowledge
\item Possible to compare automated to manual solving process
\end{itemize}

\begin{quote}
The puzzle, though inanimate, is presented as a solvable problem without lasting negative consequences, a very low-risk low-reward situation. By being a puzzle, the object is attempting to convince the user that it must be completed.
\end{quote}
{\tiny Source: \href{https://www.everydayrhetoric.com/2020/03/the-rhetoric-of-puzzles/}{Every Day Rhetoric}}
\end{frame}


\mode<all>{
\part{Basic Constraint Propagation}
}
\begin{frame}
\partpage
\end{frame}
\mode<all>{
\input{../sendmore/body}
}

\mode<all>{
\part{Global Constraints}
}
\begin{frame}
\partpage
\end{frame}
\mode<all>{
\input{../sudoku/body}
}

\clearpage
\mode<all>{
\part{Customizing Search}
}
\begin{frame}
\partpage
\end{frame}
\mode<all>{
\input{../nqueen/body}
}


%% \clearpage
%% \mode<all>{
%%   \part{Problem Decomposition}
%% }
%% \begin{frame}
%%   \partpage
%% \end{frame}
%% \mode<all>{
%%   \input{../party/body}
%%   }

%% \clearpage
%% \mode<all>{
%%   \part{More Global Constraints}
%% }
%% \begin{frame}
%%   \partpage
%% \end{frame}
%% \mode<all>{
%%   \input{../car/body}
%%   }

%% \clearpage
%% \mode<all>{
%%   \part{Methodology}
%% }
%% \begin{frame}
%%   \partpage
%% \end{frame}
%% \mode<all>{
%%   \input{../methodology/body}
%%   }

\clearpage
\mode<all>{
\part{What is missing?}
}
\begin{frame}
\partpage
\end{frame}

\begin{frame}
\frametitle{Many Specialized Topics}
\begin{itemize}
\item How to design efficient core engine
\item Hybrids with LP/MIP tools
\item Hybrids with SAT
\item Symmetry breaking
\item Use of MDD/BDD to encode sets of solutions
\item High level modelling tools
\item Debugging/visualization
\end{itemize}
\end{frame}

\begin{frame}
\frametitle{Reformulation}
\begin{itemize}
\item Just because the user has modelled it this way, it doesn't mean we have to solve it that way
\begin{itemize}
\item Replace some constraint(s) by other, equivalent constraints
\item Because we don't have that constraint in our system
\item For performance
\end{itemize}
\end{itemize}
\end{frame}

\begin{frame}
\frametitle{Learning}
\begin{itemize}
\item While solving the problem we can learn how to strengthen the model/search 
\begin{itemize}
\item Understand which constraints/method contribute to propagation and change schedule
\item Learn no-good constraints by explaining failure
\item Adapt search strategy based on search experience
\end{itemize}
\end{itemize}
\end{frame}

%% \begin{frame}
%% \frametitle{Refined Process}
%% \begin{center}
%% \begin{tikzpicture}
%% \node[shape=rectangle,fill=pantone127-4] (problem) at (3,7) {Problem};
%% \node (human) at (3,6) {Human};
%% \node[shape=rectangle,fill=pantone127-4] (model) at (3,5) {User Model};
%% \node (reformulation) at (3,4) {Reformulation};
%% \node[shape=rectangle,fill=pantone127-4] (implementation) at (3,3) {Implementation Model};
%% \node (solver) at (3,2) {Constraint Solver/Search};
%% \node[shape=rectangle,fill=pantone157-8] (solution) at (3,1) {Solution};
%% \draw[-] (problem) -- (human);
%% \draw[->] (human) -- (model);
%% \draw[-] (model) -- (reformulation);
%% \draw[->] (reformulation) -- (implementation);
%% \draw[-] (implementation) -- (solver);
%% \draw[->] (solver) -- (0,2) -- (0,3) -- (implementation);
%% \draw[->] (solver) -- (solution);
%% \end{tikzpicture}
%% \end{center}
%% \end{frame}

%\begin{frame}
%\frametitle{What is CP actually used for?}
%\url{http://hsimonis.wordpress.com}
%\includegraphics[width=10cm]{../introduction/images/blog}
%\end{frame}

%% \begin{frame}
%% \frametitle{Exercises}
%% \begin{itemize}
%% \item Install Constraint Solver of your Choice
%% \item MiniZinc (MiningZinc) will be used in later sessions
%% \item Run Sendmoremoney, Sudoku, Queens examples
%% \item Can you choose which propagator to use for alldifferent?
%% \item Can you use different search strategies?
%% \end{itemize}
%% \end{frame}

\begin{frame}
\frametitle{More Learning Resources}
\begin{itemize}
\item Survey of Methods, Resources, and Formats for Teaching Constraint Programming
\begin{itemize}
\item by Tejas Santanam, Helmut Simonis
\item \url{https://doi.org/10.48550/arXiv.2403.12717}
\item Based on survey of community for WTCP 2023
\item \url{https://hsimonis.github.io/WTCP2023/}
\end{itemize}
\end{itemize}
\end{frame}


}

\clearpage
\mode<all>{
\part{Customizing Search}
}
\begin{frame}
\partpage
\end{frame}
\mode<all>{
\begin{frame}
\frametitle{Objectives}
\begin{itemize}
\item Overview of Core Constraint Programming
\item Three Main Concepts
\begin{itemize}
\item Constraint Propagation
\item Global Constraints
\item Customizing Search
\end{itemize}
%\item Get Some Experience with MiniZinc
\item Topics will be treated in more detail in later parts of the school
\item Based on Examples, not Formal Description
\end{itemize}
\end{frame}

\begin{frame}
\frametitle{Outline}
\begin{itemize}
\item Why Constraint Programming?
\item Constraint Propagation
\item Global Constraints
\item Customizing Search
%\item Choices in Modelling
%\item Decomposition
%\item More Global Constraints
%\item Methodology
%\item What is missing?
\end{itemize}
\end{frame}

% \begin{frame}
  % \frametitle{Using MiniZinc IDE}
  % \begin{itemize}
  % \item Developed in the Australian NICTA project
  % \item Maintained by Monash University
  % \item Modelling tool with multiple back-end solvers
  % \item Available from \url{https://www.minizinc.org/}
  % \end{itemize}
% \end{frame}


% \begin{frame}
% \frametitle{Examples in ECLiPSe}
% \begin{itemize}
% \item Open sourced constraint programming language
% \item Development goes back to 1985
% \item ECRC, ICL, IC-Parc, PTL, Cisco
% \item \url{https://eclipseclp.org/}
% \item Specialities
% \begin{itemize}
% \item Develop new solvers for specific domains
% \item Integration with MIP
% \end{itemize}
% \item Not included in bundled MiniZinc IDE
% \item Specialized visualization tools used here
  % \begin{itemize}
  % \item CP-Viz, Simonis et al. 2010
    % \end{itemize}
% \end{itemize}
% \end{frame}

\begin{frame}
\frametitle{Tutorial Based on ECLiPSe ELearning Course}
\begin{itemize}
\item Self-study course in constraint programming
\item Supported by Cisco Systems and Silicon Valley Community Foundation
\item Multi-media format, video lectures, slides, handout etc
\item {\small \url{https://eclipseclp.org/ELearning/index.html}}
\end{itemize}
\end{frame}

\begin{frame}
\frametitle{Also Part of CRT-AI Constraint Week}
\begin{itemize}
\item Annual one week course on CP and Optimization in Ireland
\item Part of national training program for PhD students in AI
\item \url{https://www.crt-ai.ie/}
\end{itemize}
\end{frame}



\begin{frame}
\frametitle{Constraint Programming - in a nutshell}
\begin{itemize}
\item Declarative description of problems with
\begin{itemize}
\item {\em Variables} which range over (finite) sets of values
\item {\em Constraints} over subsets of variables which restrict possible value combinations
\item A {\em solution} is a value assignment which satisfies all constraints
\end{itemize}

\item Constraint propagation/reasoning
\begin{itemize}
\item Removing inconsistent values for variables
\item Detect failure if constraint can not be satisfied
\item Interaction of constraints via shared variables
\item Incomplete
\end{itemize}

\item Search
\begin{itemize}
\item User controlled assignment of values to variables
\item Each step triggers constraint propagation 
\end{itemize}
\item Different domains require/allow different methods
\end{itemize}
\end{frame}

\begin{frame}
  \frametitle{Constraint Programming is Different}
  \begin{itemize}
  \item Declarative Programming
    \begin{itemize}
    \item Concentrate on what you want
      \item Not how to get there
      \item Program != Algorithm
      \item Program = Model
    \end{itemize}
    \item Applied to Combinatorial Problems
      \begin{itemize}
        \item No complete polynomial algorithms known (exist?)
        \item CP less ad-hoc than heuristics
        \item Models can evolve
  \end{itemize}
  \end{itemize}
  \end{frame}
    
\begin{frame}
  \frametitle{A Subtractive Process}
  \begin{textblock}{4}(8,-3)
    \includegraphics[width=4cm]{../introduction/images/stages}
  \end{textblock}
  \vfill
  \begin{quote}
    ``Oh, bosh, as Mr. Ruskin says. Sculpture, per se, is the simplest thing in the world. All you have to do is to take a big chunk of marble and a hammer and chisel, make up your mind what you are about to create and chip off all the marble you don't want.''-Paris Gaulois.
  \end{quote}
  
  {\tiny Source: \url{https://quoteinvestigator.com/2014/06/22/chip-away/}}
\end{frame}

\begin{frame}
\frametitle{Basic Process}
\begin{center}
\begin{tikzpicture}
\node[shape=rectangle,fill=pantone127-4] (problem) at (2,5) {Problem};
\node (human) at (2,4) {Human};
\node[shape=rectangle,fill=pantone127-4] (model) at (2,3) {Model};
\node (solver) at (2,2) {Constraint Solver/Search};
\node[shape=rectangle,fill=pantone157-8] (solution) at (2,1) {Solution};
\draw[-] (problem) -- (human);
\draw[->] (human) -- (model);
\draw[-] (model) -- (solver);
\draw[->] (solver) -- (solution);
\end{tikzpicture}
\end{center}
\end{frame}

\begin{frame}
\frametitle{More Realistic}
\begin{center}
\begin{tikzpicture}[xscale=4.0]
\node[shape=rectangle,fill=pantone127-4] (problem) at (2,5) {Problem};
\node (human) at (2,4) {Human};
\node[shape=rectangle,fill=pantone127-4] (model) at (2,3) {Model};
\node (solver) at (2,2) {Constraint Solver/Search};
\node[shape=rectangle,fill=pantone24-8] (hangs) at (1,1) {Hangs};
\node[shape=rectangle,fill=pantone157-8] (solution) at (2,1) {Solution};
\node[shape=rectangle,fill=pantone24-8] (wrong) at (3,1) {Wrong Solution};
\draw[-] (problem) -- (human);
\draw[->] (human) -- (model);
\draw[-] (model) -- (solver);
\draw[->] (solver) -- (hangs);
\draw[->] (solver) -- (solution);
\draw[->] (solver) -- (wrong);
\draw[->] (hangs) -- (1,4) -- (human);
\draw[->] (wrong) -- (3,4) -- (human);
\end{tikzpicture}
\end{center}
\end{frame}

\begin{frame}
\frametitle{Dual Role of Model}
\begin{itemize}
\item Allows Human to Express Problem
\begin{itemize}
\item Close to Problem Domain
\item Constraints as Abstractions
\end{itemize}
\item Allows Solver to Execute
\begin{itemize}
\item Variables as Communication Mechanism
\item Constraints as Algorithms
\end{itemize}
\end{itemize}
\end{frame}

\begin{frame}
\frametitle{Modelling Frameworks}
\begin{itemize}
\item \href{https://www.minizinc.org/}{MiniZinc} (NICTA, Monash University, Australia)
\item \href{https://github.com/eomahony/Numberjack}{NumberJack} (Insight, Ireland) 
\item \href{https://www-users.york.ac.uk/peter.nightingale/savilerow//}{EssencePrime/SavilleRow} (UK) 
\item \href{https://cpmpy.readthedocs.io/en/latest/}{CPMpy} (KU Leuven)
\item Allow use of multiple back-end solvers
\item Compile model into variants for each solver
\item A priori solver independent model(CP, MIP, SAT)
\end{itemize}
\end{frame} 

\begin{frame}
\frametitle{Framework Process}
\begin{center}
\begin{tikzpicture}[xscale=2]
\node[shape=rectangle,fill=pantone127-4] (problem) at (2.5,5) {Problem};
\node (human) at (2.5,4) {Human};
\node[shape=rectangle,fill=pantone127-4] (model) at (2.5,3) {Model};
\node (compile) at (2.5,2) {Compile/Reformulate};
\node (solver1) at (1,1) {CP};
\node (solver2) at (2,1) {MIP};
\node (solver3) at (3,1) {SAT};
\node (solver4) at (4,1) {Other};
\node[shape=rectangle,fill=pantone157-8] (solution1) at (1,0) {Solution};
\node[shape=rectangle,fill=pantone157-8] (solution2) at (2,0) {Solution};
\node[shape=rectangle,fill=pantone157-8] (solution3) at (3,0) {Solution};
\node[shape=rectangle,fill=pantone157-8] (solution4) at (4,0) {Solution};
\draw[-] (problem) -- (human);
\draw[->] (human) -- (model);
\draw[-] (model) -- (compile);
\draw[-] (compile) -- (solver1);
\draw[-] (compile) -- (solver2);
\draw[-] (compile) -- (solver3);
\draw[-] (compile) -- (solver4);
\draw[->] (solver1) -- (solution1);
\draw[->] (solver2) -- (solution2);
\draw[->] (solver3) -- (solution3);
\draw[->] (solver4) -- (solution4);
\end{tikzpicture}
\end{center}
\end{frame}

% \begin{frame}
  % \frametitle{Do It Now!}
  % \begin{itemize}
  % \item Download and install Minizinc
  % \item \url{https://www.minizinc.org/}
  % \end{itemize}
% \end{frame}


\begin{frame}
\frametitle{Why use Puzzles as Examples?}
\begin{itemize}
\item Easy to understand the problem
\item Solvable by hand without specialized knowledge
\item Possible to compare automated to manual solving process
\end{itemize}

\begin{quote}
The puzzle, though inanimate, is presented as a solvable problem without lasting negative consequences, a very low-risk low-reward situation. By being a puzzle, the object is attempting to convince the user that it must be completed.
\end{quote}
{\tiny Source: \href{https://www.everydayrhetoric.com/2020/03/the-rhetoric-of-puzzles/}{Every Day Rhetoric}}
\end{frame}


\mode<all>{
\part{Basic Constraint Propagation}
}
\begin{frame}
\partpage
\end{frame}
\mode<all>{
\input{../sendmore/body}
}

\mode<all>{
\part{Global Constraints}
}
\begin{frame}
\partpage
\end{frame}
\mode<all>{
\input{../sudoku/body}
}

\clearpage
\mode<all>{
\part{Customizing Search}
}
\begin{frame}
\partpage
\end{frame}
\mode<all>{
\input{../nqueen/body}
}


%% \clearpage
%% \mode<all>{
%%   \part{Problem Decomposition}
%% }
%% \begin{frame}
%%   \partpage
%% \end{frame}
%% \mode<all>{
%%   \input{../party/body}
%%   }

%% \clearpage
%% \mode<all>{
%%   \part{More Global Constraints}
%% }
%% \begin{frame}
%%   \partpage
%% \end{frame}
%% \mode<all>{
%%   \input{../car/body}
%%   }

%% \clearpage
%% \mode<all>{
%%   \part{Methodology}
%% }
%% \begin{frame}
%%   \partpage
%% \end{frame}
%% \mode<all>{
%%   \input{../methodology/body}
%%   }

\clearpage
\mode<all>{
\part{What is missing?}
}
\begin{frame}
\partpage
\end{frame}

\begin{frame}
\frametitle{Many Specialized Topics}
\begin{itemize}
\item How to design efficient core engine
\item Hybrids with LP/MIP tools
\item Hybrids with SAT
\item Symmetry breaking
\item Use of MDD/BDD to encode sets of solutions
\item High level modelling tools
\item Debugging/visualization
\end{itemize}
\end{frame}

\begin{frame}
\frametitle{Reformulation}
\begin{itemize}
\item Just because the user has modelled it this way, it doesn't mean we have to solve it that way
\begin{itemize}
\item Replace some constraint(s) by other, equivalent constraints
\item Because we don't have that constraint in our system
\item For performance
\end{itemize}
\end{itemize}
\end{frame}

\begin{frame}
\frametitle{Learning}
\begin{itemize}
\item While solving the problem we can learn how to strengthen the model/search 
\begin{itemize}
\item Understand which constraints/method contribute to propagation and change schedule
\item Learn no-good constraints by explaining failure
\item Adapt search strategy based on search experience
\end{itemize}
\end{itemize}
\end{frame}

%% \begin{frame}
%% \frametitle{Refined Process}
%% \begin{center}
%% \begin{tikzpicture}
%% \node[shape=rectangle,fill=pantone127-4] (problem) at (3,7) {Problem};
%% \node (human) at (3,6) {Human};
%% \node[shape=rectangle,fill=pantone127-4] (model) at (3,5) {User Model};
%% \node (reformulation) at (3,4) {Reformulation};
%% \node[shape=rectangle,fill=pantone127-4] (implementation) at (3,3) {Implementation Model};
%% \node (solver) at (3,2) {Constraint Solver/Search};
%% \node[shape=rectangle,fill=pantone157-8] (solution) at (3,1) {Solution};
%% \draw[-] (problem) -- (human);
%% \draw[->] (human) -- (model);
%% \draw[-] (model) -- (reformulation);
%% \draw[->] (reformulation) -- (implementation);
%% \draw[-] (implementation) -- (solver);
%% \draw[->] (solver) -- (0,2) -- (0,3) -- (implementation);
%% \draw[->] (solver) -- (solution);
%% \end{tikzpicture}
%% \end{center}
%% \end{frame}

%\begin{frame}
%\frametitle{What is CP actually used for?}
%\url{http://hsimonis.wordpress.com}
%\includegraphics[width=10cm]{../introduction/images/blog}
%\end{frame}

%% \begin{frame}
%% \frametitle{Exercises}
%% \begin{itemize}
%% \item Install Constraint Solver of your Choice
%% \item MiniZinc (MiningZinc) will be used in later sessions
%% \item Run Sendmoremoney, Sudoku, Queens examples
%% \item Can you choose which propagator to use for alldifferent?
%% \item Can you use different search strategies?
%% \end{itemize}
%% \end{frame}

\begin{frame}
\frametitle{More Learning Resources}
\begin{itemize}
\item Survey of Methods, Resources, and Formats for Teaching Constraint Programming
\begin{itemize}
\item by Tejas Santanam, Helmut Simonis
\item \url{https://doi.org/10.48550/arXiv.2403.12717}
\item Based on survey of community for WTCP 2023
\item \url{https://hsimonis.github.io/WTCP2023/}
\end{itemize}
\end{itemize}
\end{frame}


}


%% \clearpage
%% \mode<all>{
%%   \part{Problem Decomposition}
%% }
%% \begin{frame}
%%   \partpage
%% \end{frame}
%% \mode<all>{
%%   \begin{frame}
\frametitle{Objectives}
\begin{itemize}
\item Overview of Core Constraint Programming
\item Three Main Concepts
\begin{itemize}
\item Constraint Propagation
\item Global Constraints
\item Customizing Search
\end{itemize}
%\item Get Some Experience with MiniZinc
\item Topics will be treated in more detail in later parts of the school
\item Based on Examples, not Formal Description
\end{itemize}
\end{frame}

\begin{frame}
\frametitle{Outline}
\begin{itemize}
\item Why Constraint Programming?
\item Constraint Propagation
\item Global Constraints
\item Customizing Search
%\item Choices in Modelling
%\item Decomposition
%\item More Global Constraints
%\item Methodology
%\item What is missing?
\end{itemize}
\end{frame}

% \begin{frame}
  % \frametitle{Using MiniZinc IDE}
  % \begin{itemize}
  % \item Developed in the Australian NICTA project
  % \item Maintained by Monash University
  % \item Modelling tool with multiple back-end solvers
  % \item Available from \url{https://www.minizinc.org/}
  % \end{itemize}
% \end{frame}


% \begin{frame}
% \frametitle{Examples in ECLiPSe}
% \begin{itemize}
% \item Open sourced constraint programming language
% \item Development goes back to 1985
% \item ECRC, ICL, IC-Parc, PTL, Cisco
% \item \url{https://eclipseclp.org/}
% \item Specialities
% \begin{itemize}
% \item Develop new solvers for specific domains
% \item Integration with MIP
% \end{itemize}
% \item Not included in bundled MiniZinc IDE
% \item Specialized visualization tools used here
  % \begin{itemize}
  % \item CP-Viz, Simonis et al. 2010
    % \end{itemize}
% \end{itemize}
% \end{frame}

\begin{frame}
\frametitle{Tutorial Based on ECLiPSe ELearning Course}
\begin{itemize}
\item Self-study course in constraint programming
\item Supported by Cisco Systems and Silicon Valley Community Foundation
\item Multi-media format, video lectures, slides, handout etc
\item {\small \url{https://eclipseclp.org/ELearning/index.html}}
\end{itemize}
\end{frame}

\begin{frame}
\frametitle{Also Part of CRT-AI Constraint Week}
\begin{itemize}
\item Annual one week course on CP and Optimization in Ireland
\item Part of national training program for PhD students in AI
\item \url{https://www.crt-ai.ie/}
\end{itemize}
\end{frame}



\begin{frame}
\frametitle{Constraint Programming - in a nutshell}
\begin{itemize}
\item Declarative description of problems with
\begin{itemize}
\item {\em Variables} which range over (finite) sets of values
\item {\em Constraints} over subsets of variables which restrict possible value combinations
\item A {\em solution} is a value assignment which satisfies all constraints
\end{itemize}

\item Constraint propagation/reasoning
\begin{itemize}
\item Removing inconsistent values for variables
\item Detect failure if constraint can not be satisfied
\item Interaction of constraints via shared variables
\item Incomplete
\end{itemize}

\item Search
\begin{itemize}
\item User controlled assignment of values to variables
\item Each step triggers constraint propagation 
\end{itemize}
\item Different domains require/allow different methods
\end{itemize}
\end{frame}

\begin{frame}
  \frametitle{Constraint Programming is Different}
  \begin{itemize}
  \item Declarative Programming
    \begin{itemize}
    \item Concentrate on what you want
      \item Not how to get there
      \item Program != Algorithm
      \item Program = Model
    \end{itemize}
    \item Applied to Combinatorial Problems
      \begin{itemize}
        \item No complete polynomial algorithms known (exist?)
        \item CP less ad-hoc than heuristics
        \item Models can evolve
  \end{itemize}
  \end{itemize}
  \end{frame}
    
\begin{frame}
  \frametitle{A Subtractive Process}
  \begin{textblock}{4}(8,-3)
    \includegraphics[width=4cm]{../introduction/images/stages}
  \end{textblock}
  \vfill
  \begin{quote}
    ``Oh, bosh, as Mr. Ruskin says. Sculpture, per se, is the simplest thing in the world. All you have to do is to take a big chunk of marble and a hammer and chisel, make up your mind what you are about to create and chip off all the marble you don't want.''-Paris Gaulois.
  \end{quote}
  
  {\tiny Source: \url{https://quoteinvestigator.com/2014/06/22/chip-away/}}
\end{frame}

\begin{frame}
\frametitle{Basic Process}
\begin{center}
\begin{tikzpicture}
\node[shape=rectangle,fill=pantone127-4] (problem) at (2,5) {Problem};
\node (human) at (2,4) {Human};
\node[shape=rectangle,fill=pantone127-4] (model) at (2,3) {Model};
\node (solver) at (2,2) {Constraint Solver/Search};
\node[shape=rectangle,fill=pantone157-8] (solution) at (2,1) {Solution};
\draw[-] (problem) -- (human);
\draw[->] (human) -- (model);
\draw[-] (model) -- (solver);
\draw[->] (solver) -- (solution);
\end{tikzpicture}
\end{center}
\end{frame}

\begin{frame}
\frametitle{More Realistic}
\begin{center}
\begin{tikzpicture}[xscale=4.0]
\node[shape=rectangle,fill=pantone127-4] (problem) at (2,5) {Problem};
\node (human) at (2,4) {Human};
\node[shape=rectangle,fill=pantone127-4] (model) at (2,3) {Model};
\node (solver) at (2,2) {Constraint Solver/Search};
\node[shape=rectangle,fill=pantone24-8] (hangs) at (1,1) {Hangs};
\node[shape=rectangle,fill=pantone157-8] (solution) at (2,1) {Solution};
\node[shape=rectangle,fill=pantone24-8] (wrong) at (3,1) {Wrong Solution};
\draw[-] (problem) -- (human);
\draw[->] (human) -- (model);
\draw[-] (model) -- (solver);
\draw[->] (solver) -- (hangs);
\draw[->] (solver) -- (solution);
\draw[->] (solver) -- (wrong);
\draw[->] (hangs) -- (1,4) -- (human);
\draw[->] (wrong) -- (3,4) -- (human);
\end{tikzpicture}
\end{center}
\end{frame}

\begin{frame}
\frametitle{Dual Role of Model}
\begin{itemize}
\item Allows Human to Express Problem
\begin{itemize}
\item Close to Problem Domain
\item Constraints as Abstractions
\end{itemize}
\item Allows Solver to Execute
\begin{itemize}
\item Variables as Communication Mechanism
\item Constraints as Algorithms
\end{itemize}
\end{itemize}
\end{frame}

\begin{frame}
\frametitle{Modelling Frameworks}
\begin{itemize}
\item \href{https://www.minizinc.org/}{MiniZinc} (NICTA, Monash University, Australia)
\item \href{https://github.com/eomahony/Numberjack}{NumberJack} (Insight, Ireland) 
\item \href{https://www-users.york.ac.uk/peter.nightingale/savilerow//}{EssencePrime/SavilleRow} (UK) 
\item \href{https://cpmpy.readthedocs.io/en/latest/}{CPMpy} (KU Leuven)
\item Allow use of multiple back-end solvers
\item Compile model into variants for each solver
\item A priori solver independent model(CP, MIP, SAT)
\end{itemize}
\end{frame} 

\begin{frame}
\frametitle{Framework Process}
\begin{center}
\begin{tikzpicture}[xscale=2]
\node[shape=rectangle,fill=pantone127-4] (problem) at (2.5,5) {Problem};
\node (human) at (2.5,4) {Human};
\node[shape=rectangle,fill=pantone127-4] (model) at (2.5,3) {Model};
\node (compile) at (2.5,2) {Compile/Reformulate};
\node (solver1) at (1,1) {CP};
\node (solver2) at (2,1) {MIP};
\node (solver3) at (3,1) {SAT};
\node (solver4) at (4,1) {Other};
\node[shape=rectangle,fill=pantone157-8] (solution1) at (1,0) {Solution};
\node[shape=rectangle,fill=pantone157-8] (solution2) at (2,0) {Solution};
\node[shape=rectangle,fill=pantone157-8] (solution3) at (3,0) {Solution};
\node[shape=rectangle,fill=pantone157-8] (solution4) at (4,0) {Solution};
\draw[-] (problem) -- (human);
\draw[->] (human) -- (model);
\draw[-] (model) -- (compile);
\draw[-] (compile) -- (solver1);
\draw[-] (compile) -- (solver2);
\draw[-] (compile) -- (solver3);
\draw[-] (compile) -- (solver4);
\draw[->] (solver1) -- (solution1);
\draw[->] (solver2) -- (solution2);
\draw[->] (solver3) -- (solution3);
\draw[->] (solver4) -- (solution4);
\end{tikzpicture}
\end{center}
\end{frame}

% \begin{frame}
  % \frametitle{Do It Now!}
  % \begin{itemize}
  % \item Download and install Minizinc
  % \item \url{https://www.minizinc.org/}
  % \end{itemize}
% \end{frame}


\begin{frame}
\frametitle{Why use Puzzles as Examples?}
\begin{itemize}
\item Easy to understand the problem
\item Solvable by hand without specialized knowledge
\item Possible to compare automated to manual solving process
\end{itemize}

\begin{quote}
The puzzle, though inanimate, is presented as a solvable problem without lasting negative consequences, a very low-risk low-reward situation. By being a puzzle, the object is attempting to convince the user that it must be completed.
\end{quote}
{\tiny Source: \href{https://www.everydayrhetoric.com/2020/03/the-rhetoric-of-puzzles/}{Every Day Rhetoric}}
\end{frame}


\mode<all>{
\part{Basic Constraint Propagation}
}
\begin{frame}
\partpage
\end{frame}
\mode<all>{
\input{../sendmore/body}
}

\mode<all>{
\part{Global Constraints}
}
\begin{frame}
\partpage
\end{frame}
\mode<all>{
\input{../sudoku/body}
}

\clearpage
\mode<all>{
\part{Customizing Search}
}
\begin{frame}
\partpage
\end{frame}
\mode<all>{
\input{../nqueen/body}
}


%% \clearpage
%% \mode<all>{
%%   \part{Problem Decomposition}
%% }
%% \begin{frame}
%%   \partpage
%% \end{frame}
%% \mode<all>{
%%   \input{../party/body}
%%   }

%% \clearpage
%% \mode<all>{
%%   \part{More Global Constraints}
%% }
%% \begin{frame}
%%   \partpage
%% \end{frame}
%% \mode<all>{
%%   \input{../car/body}
%%   }

%% \clearpage
%% \mode<all>{
%%   \part{Methodology}
%% }
%% \begin{frame}
%%   \partpage
%% \end{frame}
%% \mode<all>{
%%   \input{../methodology/body}
%%   }

\clearpage
\mode<all>{
\part{What is missing?}
}
\begin{frame}
\partpage
\end{frame}

\begin{frame}
\frametitle{Many Specialized Topics}
\begin{itemize}
\item How to design efficient core engine
\item Hybrids with LP/MIP tools
\item Hybrids with SAT
\item Symmetry breaking
\item Use of MDD/BDD to encode sets of solutions
\item High level modelling tools
\item Debugging/visualization
\end{itemize}
\end{frame}

\begin{frame}
\frametitle{Reformulation}
\begin{itemize}
\item Just because the user has modelled it this way, it doesn't mean we have to solve it that way
\begin{itemize}
\item Replace some constraint(s) by other, equivalent constraints
\item Because we don't have that constraint in our system
\item For performance
\end{itemize}
\end{itemize}
\end{frame}

\begin{frame}
\frametitle{Learning}
\begin{itemize}
\item While solving the problem we can learn how to strengthen the model/search 
\begin{itemize}
\item Understand which constraints/method contribute to propagation and change schedule
\item Learn no-good constraints by explaining failure
\item Adapt search strategy based on search experience
\end{itemize}
\end{itemize}
\end{frame}

%% \begin{frame}
%% \frametitle{Refined Process}
%% \begin{center}
%% \begin{tikzpicture}
%% \node[shape=rectangle,fill=pantone127-4] (problem) at (3,7) {Problem};
%% \node (human) at (3,6) {Human};
%% \node[shape=rectangle,fill=pantone127-4] (model) at (3,5) {User Model};
%% \node (reformulation) at (3,4) {Reformulation};
%% \node[shape=rectangle,fill=pantone127-4] (implementation) at (3,3) {Implementation Model};
%% \node (solver) at (3,2) {Constraint Solver/Search};
%% \node[shape=rectangle,fill=pantone157-8] (solution) at (3,1) {Solution};
%% \draw[-] (problem) -- (human);
%% \draw[->] (human) -- (model);
%% \draw[-] (model) -- (reformulation);
%% \draw[->] (reformulation) -- (implementation);
%% \draw[-] (implementation) -- (solver);
%% \draw[->] (solver) -- (0,2) -- (0,3) -- (implementation);
%% \draw[->] (solver) -- (solution);
%% \end{tikzpicture}
%% \end{center}
%% \end{frame}

%\begin{frame}
%\frametitle{What is CP actually used for?}
%\url{http://hsimonis.wordpress.com}
%\includegraphics[width=10cm]{../introduction/images/blog}
%\end{frame}

%% \begin{frame}
%% \frametitle{Exercises}
%% \begin{itemize}
%% \item Install Constraint Solver of your Choice
%% \item MiniZinc (MiningZinc) will be used in later sessions
%% \item Run Sendmoremoney, Sudoku, Queens examples
%% \item Can you choose which propagator to use for alldifferent?
%% \item Can you use different search strategies?
%% \end{itemize}
%% \end{frame}

\begin{frame}
\frametitle{More Learning Resources}
\begin{itemize}
\item Survey of Methods, Resources, and Formats for Teaching Constraint Programming
\begin{itemize}
\item by Tejas Santanam, Helmut Simonis
\item \url{https://doi.org/10.48550/arXiv.2403.12717}
\item Based on survey of community for WTCP 2023
\item \url{https://hsimonis.github.io/WTCP2023/}
\end{itemize}
\end{itemize}
\end{frame}


%%   }

%% \clearpage
%% \mode<all>{
%%   \part{More Global Constraints}
%% }
%% \begin{frame}
%%   \partpage
%% \end{frame}
%% \mode<all>{
%%   \begin{frame}
\frametitle{Objectives}
\begin{itemize}
\item Overview of Core Constraint Programming
\item Three Main Concepts
\begin{itemize}
\item Constraint Propagation
\item Global Constraints
\item Customizing Search
\end{itemize}
%\item Get Some Experience with MiniZinc
\item Topics will be treated in more detail in later parts of the school
\item Based on Examples, not Formal Description
\end{itemize}
\end{frame}

\begin{frame}
\frametitle{Outline}
\begin{itemize}
\item Why Constraint Programming?
\item Constraint Propagation
\item Global Constraints
\item Customizing Search
%\item Choices in Modelling
%\item Decomposition
%\item More Global Constraints
%\item Methodology
%\item What is missing?
\end{itemize}
\end{frame}

% \begin{frame}
  % \frametitle{Using MiniZinc IDE}
  % \begin{itemize}
  % \item Developed in the Australian NICTA project
  % \item Maintained by Monash University
  % \item Modelling tool with multiple back-end solvers
  % \item Available from \url{https://www.minizinc.org/}
  % \end{itemize}
% \end{frame}


% \begin{frame}
% \frametitle{Examples in ECLiPSe}
% \begin{itemize}
% \item Open sourced constraint programming language
% \item Development goes back to 1985
% \item ECRC, ICL, IC-Parc, PTL, Cisco
% \item \url{https://eclipseclp.org/}
% \item Specialities
% \begin{itemize}
% \item Develop new solvers for specific domains
% \item Integration with MIP
% \end{itemize}
% \item Not included in bundled MiniZinc IDE
% \item Specialized visualization tools used here
  % \begin{itemize}
  % \item CP-Viz, Simonis et al. 2010
    % \end{itemize}
% \end{itemize}
% \end{frame}

\begin{frame}
\frametitle{Tutorial Based on ECLiPSe ELearning Course}
\begin{itemize}
\item Self-study course in constraint programming
\item Supported by Cisco Systems and Silicon Valley Community Foundation
\item Multi-media format, video lectures, slides, handout etc
\item {\small \url{https://eclipseclp.org/ELearning/index.html}}
\end{itemize}
\end{frame}

\begin{frame}
\frametitle{Also Part of CRT-AI Constraint Week}
\begin{itemize}
\item Annual one week course on CP and Optimization in Ireland
\item Part of national training program for PhD students in AI
\item \url{https://www.crt-ai.ie/}
\end{itemize}
\end{frame}



\begin{frame}
\frametitle{Constraint Programming - in a nutshell}
\begin{itemize}
\item Declarative description of problems with
\begin{itemize}
\item {\em Variables} which range over (finite) sets of values
\item {\em Constraints} over subsets of variables which restrict possible value combinations
\item A {\em solution} is a value assignment which satisfies all constraints
\end{itemize}

\item Constraint propagation/reasoning
\begin{itemize}
\item Removing inconsistent values for variables
\item Detect failure if constraint can not be satisfied
\item Interaction of constraints via shared variables
\item Incomplete
\end{itemize}

\item Search
\begin{itemize}
\item User controlled assignment of values to variables
\item Each step triggers constraint propagation 
\end{itemize}
\item Different domains require/allow different methods
\end{itemize}
\end{frame}

\begin{frame}
  \frametitle{Constraint Programming is Different}
  \begin{itemize}
  \item Declarative Programming
    \begin{itemize}
    \item Concentrate on what you want
      \item Not how to get there
      \item Program != Algorithm
      \item Program = Model
    \end{itemize}
    \item Applied to Combinatorial Problems
      \begin{itemize}
        \item No complete polynomial algorithms known (exist?)
        \item CP less ad-hoc than heuristics
        \item Models can evolve
  \end{itemize}
  \end{itemize}
  \end{frame}
    
\begin{frame}
  \frametitle{A Subtractive Process}
  \begin{textblock}{4}(8,-3)
    \includegraphics[width=4cm]{../introduction/images/stages}
  \end{textblock}
  \vfill
  \begin{quote}
    ``Oh, bosh, as Mr. Ruskin says. Sculpture, per se, is the simplest thing in the world. All you have to do is to take a big chunk of marble and a hammer and chisel, make up your mind what you are about to create and chip off all the marble you don't want.''-Paris Gaulois.
  \end{quote}
  
  {\tiny Source: \url{https://quoteinvestigator.com/2014/06/22/chip-away/}}
\end{frame}

\begin{frame}
\frametitle{Basic Process}
\begin{center}
\begin{tikzpicture}
\node[shape=rectangle,fill=pantone127-4] (problem) at (2,5) {Problem};
\node (human) at (2,4) {Human};
\node[shape=rectangle,fill=pantone127-4] (model) at (2,3) {Model};
\node (solver) at (2,2) {Constraint Solver/Search};
\node[shape=rectangle,fill=pantone157-8] (solution) at (2,1) {Solution};
\draw[-] (problem) -- (human);
\draw[->] (human) -- (model);
\draw[-] (model) -- (solver);
\draw[->] (solver) -- (solution);
\end{tikzpicture}
\end{center}
\end{frame}

\begin{frame}
\frametitle{More Realistic}
\begin{center}
\begin{tikzpicture}[xscale=4.0]
\node[shape=rectangle,fill=pantone127-4] (problem) at (2,5) {Problem};
\node (human) at (2,4) {Human};
\node[shape=rectangle,fill=pantone127-4] (model) at (2,3) {Model};
\node (solver) at (2,2) {Constraint Solver/Search};
\node[shape=rectangle,fill=pantone24-8] (hangs) at (1,1) {Hangs};
\node[shape=rectangle,fill=pantone157-8] (solution) at (2,1) {Solution};
\node[shape=rectangle,fill=pantone24-8] (wrong) at (3,1) {Wrong Solution};
\draw[-] (problem) -- (human);
\draw[->] (human) -- (model);
\draw[-] (model) -- (solver);
\draw[->] (solver) -- (hangs);
\draw[->] (solver) -- (solution);
\draw[->] (solver) -- (wrong);
\draw[->] (hangs) -- (1,4) -- (human);
\draw[->] (wrong) -- (3,4) -- (human);
\end{tikzpicture}
\end{center}
\end{frame}

\begin{frame}
\frametitle{Dual Role of Model}
\begin{itemize}
\item Allows Human to Express Problem
\begin{itemize}
\item Close to Problem Domain
\item Constraints as Abstractions
\end{itemize}
\item Allows Solver to Execute
\begin{itemize}
\item Variables as Communication Mechanism
\item Constraints as Algorithms
\end{itemize}
\end{itemize}
\end{frame}

\begin{frame}
\frametitle{Modelling Frameworks}
\begin{itemize}
\item \href{https://www.minizinc.org/}{MiniZinc} (NICTA, Monash University, Australia)
\item \href{https://github.com/eomahony/Numberjack}{NumberJack} (Insight, Ireland) 
\item \href{https://www-users.york.ac.uk/peter.nightingale/savilerow//}{EssencePrime/SavilleRow} (UK) 
\item \href{https://cpmpy.readthedocs.io/en/latest/}{CPMpy} (KU Leuven)
\item Allow use of multiple back-end solvers
\item Compile model into variants for each solver
\item A priori solver independent model(CP, MIP, SAT)
\end{itemize}
\end{frame} 

\begin{frame}
\frametitle{Framework Process}
\begin{center}
\begin{tikzpicture}[xscale=2]
\node[shape=rectangle,fill=pantone127-4] (problem) at (2.5,5) {Problem};
\node (human) at (2.5,4) {Human};
\node[shape=rectangle,fill=pantone127-4] (model) at (2.5,3) {Model};
\node (compile) at (2.5,2) {Compile/Reformulate};
\node (solver1) at (1,1) {CP};
\node (solver2) at (2,1) {MIP};
\node (solver3) at (3,1) {SAT};
\node (solver4) at (4,1) {Other};
\node[shape=rectangle,fill=pantone157-8] (solution1) at (1,0) {Solution};
\node[shape=rectangle,fill=pantone157-8] (solution2) at (2,0) {Solution};
\node[shape=rectangle,fill=pantone157-8] (solution3) at (3,0) {Solution};
\node[shape=rectangle,fill=pantone157-8] (solution4) at (4,0) {Solution};
\draw[-] (problem) -- (human);
\draw[->] (human) -- (model);
\draw[-] (model) -- (compile);
\draw[-] (compile) -- (solver1);
\draw[-] (compile) -- (solver2);
\draw[-] (compile) -- (solver3);
\draw[-] (compile) -- (solver4);
\draw[->] (solver1) -- (solution1);
\draw[->] (solver2) -- (solution2);
\draw[->] (solver3) -- (solution3);
\draw[->] (solver4) -- (solution4);
\end{tikzpicture}
\end{center}
\end{frame}

% \begin{frame}
  % \frametitle{Do It Now!}
  % \begin{itemize}
  % \item Download and install Minizinc
  % \item \url{https://www.minizinc.org/}
  % \end{itemize}
% \end{frame}


\begin{frame}
\frametitle{Why use Puzzles as Examples?}
\begin{itemize}
\item Easy to understand the problem
\item Solvable by hand without specialized knowledge
\item Possible to compare automated to manual solving process
\end{itemize}

\begin{quote}
The puzzle, though inanimate, is presented as a solvable problem without lasting negative consequences, a very low-risk low-reward situation. By being a puzzle, the object is attempting to convince the user that it must be completed.
\end{quote}
{\tiny Source: \href{https://www.everydayrhetoric.com/2020/03/the-rhetoric-of-puzzles/}{Every Day Rhetoric}}
\end{frame}


\mode<all>{
\part{Basic Constraint Propagation}
}
\begin{frame}
\partpage
\end{frame}
\mode<all>{
\input{../sendmore/body}
}

\mode<all>{
\part{Global Constraints}
}
\begin{frame}
\partpage
\end{frame}
\mode<all>{
\input{../sudoku/body}
}

\clearpage
\mode<all>{
\part{Customizing Search}
}
\begin{frame}
\partpage
\end{frame}
\mode<all>{
\input{../nqueen/body}
}


%% \clearpage
%% \mode<all>{
%%   \part{Problem Decomposition}
%% }
%% \begin{frame}
%%   \partpage
%% \end{frame}
%% \mode<all>{
%%   \input{../party/body}
%%   }

%% \clearpage
%% \mode<all>{
%%   \part{More Global Constraints}
%% }
%% \begin{frame}
%%   \partpage
%% \end{frame}
%% \mode<all>{
%%   \input{../car/body}
%%   }

%% \clearpage
%% \mode<all>{
%%   \part{Methodology}
%% }
%% \begin{frame}
%%   \partpage
%% \end{frame}
%% \mode<all>{
%%   \input{../methodology/body}
%%   }

\clearpage
\mode<all>{
\part{What is missing?}
}
\begin{frame}
\partpage
\end{frame}

\begin{frame}
\frametitle{Many Specialized Topics}
\begin{itemize}
\item How to design efficient core engine
\item Hybrids with LP/MIP tools
\item Hybrids with SAT
\item Symmetry breaking
\item Use of MDD/BDD to encode sets of solutions
\item High level modelling tools
\item Debugging/visualization
\end{itemize}
\end{frame}

\begin{frame}
\frametitle{Reformulation}
\begin{itemize}
\item Just because the user has modelled it this way, it doesn't mean we have to solve it that way
\begin{itemize}
\item Replace some constraint(s) by other, equivalent constraints
\item Because we don't have that constraint in our system
\item For performance
\end{itemize}
\end{itemize}
\end{frame}

\begin{frame}
\frametitle{Learning}
\begin{itemize}
\item While solving the problem we can learn how to strengthen the model/search 
\begin{itemize}
\item Understand which constraints/method contribute to propagation and change schedule
\item Learn no-good constraints by explaining failure
\item Adapt search strategy based on search experience
\end{itemize}
\end{itemize}
\end{frame}

%% \begin{frame}
%% \frametitle{Refined Process}
%% \begin{center}
%% \begin{tikzpicture}
%% \node[shape=rectangle,fill=pantone127-4] (problem) at (3,7) {Problem};
%% \node (human) at (3,6) {Human};
%% \node[shape=rectangle,fill=pantone127-4] (model) at (3,5) {User Model};
%% \node (reformulation) at (3,4) {Reformulation};
%% \node[shape=rectangle,fill=pantone127-4] (implementation) at (3,3) {Implementation Model};
%% \node (solver) at (3,2) {Constraint Solver/Search};
%% \node[shape=rectangle,fill=pantone157-8] (solution) at (3,1) {Solution};
%% \draw[-] (problem) -- (human);
%% \draw[->] (human) -- (model);
%% \draw[-] (model) -- (reformulation);
%% \draw[->] (reformulation) -- (implementation);
%% \draw[-] (implementation) -- (solver);
%% \draw[->] (solver) -- (0,2) -- (0,3) -- (implementation);
%% \draw[->] (solver) -- (solution);
%% \end{tikzpicture}
%% \end{center}
%% \end{frame}

%\begin{frame}
%\frametitle{What is CP actually used for?}
%\url{http://hsimonis.wordpress.com}
%\includegraphics[width=10cm]{../introduction/images/blog}
%\end{frame}

%% \begin{frame}
%% \frametitle{Exercises}
%% \begin{itemize}
%% \item Install Constraint Solver of your Choice
%% \item MiniZinc (MiningZinc) will be used in later sessions
%% \item Run Sendmoremoney, Sudoku, Queens examples
%% \item Can you choose which propagator to use for alldifferent?
%% \item Can you use different search strategies?
%% \end{itemize}
%% \end{frame}

\begin{frame}
\frametitle{More Learning Resources}
\begin{itemize}
\item Survey of Methods, Resources, and Formats for Teaching Constraint Programming
\begin{itemize}
\item by Tejas Santanam, Helmut Simonis
\item \url{https://doi.org/10.48550/arXiv.2403.12717}
\item Based on survey of community for WTCP 2023
\item \url{https://hsimonis.github.io/WTCP2023/}
\end{itemize}
\end{itemize}
\end{frame}


%%   }

%% \clearpage
%% \mode<all>{
%%   \part{Methodology}
%% }
%% \begin{frame}
%%   \partpage
%% \end{frame}
%% \mode<all>{
%%   \begin{frame}
\frametitle{Objectives}
\begin{itemize}
\item Overview of Core Constraint Programming
\item Three Main Concepts
\begin{itemize}
\item Constraint Propagation
\item Global Constraints
\item Customizing Search
\end{itemize}
%\item Get Some Experience with MiniZinc
\item Topics will be treated in more detail in later parts of the school
\item Based on Examples, not Formal Description
\end{itemize}
\end{frame}

\begin{frame}
\frametitle{Outline}
\begin{itemize}
\item Why Constraint Programming?
\item Constraint Propagation
\item Global Constraints
\item Customizing Search
%\item Choices in Modelling
%\item Decomposition
%\item More Global Constraints
%\item Methodology
%\item What is missing?
\end{itemize}
\end{frame}

% \begin{frame}
  % \frametitle{Using MiniZinc IDE}
  % \begin{itemize}
  % \item Developed in the Australian NICTA project
  % \item Maintained by Monash University
  % \item Modelling tool with multiple back-end solvers
  % \item Available from \url{https://www.minizinc.org/}
  % \end{itemize}
% \end{frame}


% \begin{frame}
% \frametitle{Examples in ECLiPSe}
% \begin{itemize}
% \item Open sourced constraint programming language
% \item Development goes back to 1985
% \item ECRC, ICL, IC-Parc, PTL, Cisco
% \item \url{https://eclipseclp.org/}
% \item Specialities
% \begin{itemize}
% \item Develop new solvers for specific domains
% \item Integration with MIP
% \end{itemize}
% \item Not included in bundled MiniZinc IDE
% \item Specialized visualization tools used here
  % \begin{itemize}
  % \item CP-Viz, Simonis et al. 2010
    % \end{itemize}
% \end{itemize}
% \end{frame}

\begin{frame}
\frametitle{Tutorial Based on ECLiPSe ELearning Course}
\begin{itemize}
\item Self-study course in constraint programming
\item Supported by Cisco Systems and Silicon Valley Community Foundation
\item Multi-media format, video lectures, slides, handout etc
\item {\small \url{https://eclipseclp.org/ELearning/index.html}}
\end{itemize}
\end{frame}

\begin{frame}
\frametitle{Also Part of CRT-AI Constraint Week}
\begin{itemize}
\item Annual one week course on CP and Optimization in Ireland
\item Part of national training program for PhD students in AI
\item \url{https://www.crt-ai.ie/}
\end{itemize}
\end{frame}



\begin{frame}
\frametitle{Constraint Programming - in a nutshell}
\begin{itemize}
\item Declarative description of problems with
\begin{itemize}
\item {\em Variables} which range over (finite) sets of values
\item {\em Constraints} over subsets of variables which restrict possible value combinations
\item A {\em solution} is a value assignment which satisfies all constraints
\end{itemize}

\item Constraint propagation/reasoning
\begin{itemize}
\item Removing inconsistent values for variables
\item Detect failure if constraint can not be satisfied
\item Interaction of constraints via shared variables
\item Incomplete
\end{itemize}

\item Search
\begin{itemize}
\item User controlled assignment of values to variables
\item Each step triggers constraint propagation 
\end{itemize}
\item Different domains require/allow different methods
\end{itemize}
\end{frame}

\begin{frame}
  \frametitle{Constraint Programming is Different}
  \begin{itemize}
  \item Declarative Programming
    \begin{itemize}
    \item Concentrate on what you want
      \item Not how to get there
      \item Program != Algorithm
      \item Program = Model
    \end{itemize}
    \item Applied to Combinatorial Problems
      \begin{itemize}
        \item No complete polynomial algorithms known (exist?)
        \item CP less ad-hoc than heuristics
        \item Models can evolve
  \end{itemize}
  \end{itemize}
  \end{frame}
    
\begin{frame}
  \frametitle{A Subtractive Process}
  \begin{textblock}{4}(8,-3)
    \includegraphics[width=4cm]{../introduction/images/stages}
  \end{textblock}
  \vfill
  \begin{quote}
    ``Oh, bosh, as Mr. Ruskin says. Sculpture, per se, is the simplest thing in the world. All you have to do is to take a big chunk of marble and a hammer and chisel, make up your mind what you are about to create and chip off all the marble you don't want.''-Paris Gaulois.
  \end{quote}
  
  {\tiny Source: \url{https://quoteinvestigator.com/2014/06/22/chip-away/}}
\end{frame}

\begin{frame}
\frametitle{Basic Process}
\begin{center}
\begin{tikzpicture}
\node[shape=rectangle,fill=pantone127-4] (problem) at (2,5) {Problem};
\node (human) at (2,4) {Human};
\node[shape=rectangle,fill=pantone127-4] (model) at (2,3) {Model};
\node (solver) at (2,2) {Constraint Solver/Search};
\node[shape=rectangle,fill=pantone157-8] (solution) at (2,1) {Solution};
\draw[-] (problem) -- (human);
\draw[->] (human) -- (model);
\draw[-] (model) -- (solver);
\draw[->] (solver) -- (solution);
\end{tikzpicture}
\end{center}
\end{frame}

\begin{frame}
\frametitle{More Realistic}
\begin{center}
\begin{tikzpicture}[xscale=4.0]
\node[shape=rectangle,fill=pantone127-4] (problem) at (2,5) {Problem};
\node (human) at (2,4) {Human};
\node[shape=rectangle,fill=pantone127-4] (model) at (2,3) {Model};
\node (solver) at (2,2) {Constraint Solver/Search};
\node[shape=rectangle,fill=pantone24-8] (hangs) at (1,1) {Hangs};
\node[shape=rectangle,fill=pantone157-8] (solution) at (2,1) {Solution};
\node[shape=rectangle,fill=pantone24-8] (wrong) at (3,1) {Wrong Solution};
\draw[-] (problem) -- (human);
\draw[->] (human) -- (model);
\draw[-] (model) -- (solver);
\draw[->] (solver) -- (hangs);
\draw[->] (solver) -- (solution);
\draw[->] (solver) -- (wrong);
\draw[->] (hangs) -- (1,4) -- (human);
\draw[->] (wrong) -- (3,4) -- (human);
\end{tikzpicture}
\end{center}
\end{frame}

\begin{frame}
\frametitle{Dual Role of Model}
\begin{itemize}
\item Allows Human to Express Problem
\begin{itemize}
\item Close to Problem Domain
\item Constraints as Abstractions
\end{itemize}
\item Allows Solver to Execute
\begin{itemize}
\item Variables as Communication Mechanism
\item Constraints as Algorithms
\end{itemize}
\end{itemize}
\end{frame}

\begin{frame}
\frametitle{Modelling Frameworks}
\begin{itemize}
\item \href{https://www.minizinc.org/}{MiniZinc} (NICTA, Monash University, Australia)
\item \href{https://github.com/eomahony/Numberjack}{NumberJack} (Insight, Ireland) 
\item \href{https://www-users.york.ac.uk/peter.nightingale/savilerow//}{EssencePrime/SavilleRow} (UK) 
\item \href{https://cpmpy.readthedocs.io/en/latest/}{CPMpy} (KU Leuven)
\item Allow use of multiple back-end solvers
\item Compile model into variants for each solver
\item A priori solver independent model(CP, MIP, SAT)
\end{itemize}
\end{frame} 

\begin{frame}
\frametitle{Framework Process}
\begin{center}
\begin{tikzpicture}[xscale=2]
\node[shape=rectangle,fill=pantone127-4] (problem) at (2.5,5) {Problem};
\node (human) at (2.5,4) {Human};
\node[shape=rectangle,fill=pantone127-4] (model) at (2.5,3) {Model};
\node (compile) at (2.5,2) {Compile/Reformulate};
\node (solver1) at (1,1) {CP};
\node (solver2) at (2,1) {MIP};
\node (solver3) at (3,1) {SAT};
\node (solver4) at (4,1) {Other};
\node[shape=rectangle,fill=pantone157-8] (solution1) at (1,0) {Solution};
\node[shape=rectangle,fill=pantone157-8] (solution2) at (2,0) {Solution};
\node[shape=rectangle,fill=pantone157-8] (solution3) at (3,0) {Solution};
\node[shape=rectangle,fill=pantone157-8] (solution4) at (4,0) {Solution};
\draw[-] (problem) -- (human);
\draw[->] (human) -- (model);
\draw[-] (model) -- (compile);
\draw[-] (compile) -- (solver1);
\draw[-] (compile) -- (solver2);
\draw[-] (compile) -- (solver3);
\draw[-] (compile) -- (solver4);
\draw[->] (solver1) -- (solution1);
\draw[->] (solver2) -- (solution2);
\draw[->] (solver3) -- (solution3);
\draw[->] (solver4) -- (solution4);
\end{tikzpicture}
\end{center}
\end{frame}

% \begin{frame}
  % \frametitle{Do It Now!}
  % \begin{itemize}
  % \item Download and install Minizinc
  % \item \url{https://www.minizinc.org/}
  % \end{itemize}
% \end{frame}


\begin{frame}
\frametitle{Why use Puzzles as Examples?}
\begin{itemize}
\item Easy to understand the problem
\item Solvable by hand without specialized knowledge
\item Possible to compare automated to manual solving process
\end{itemize}

\begin{quote}
The puzzle, though inanimate, is presented as a solvable problem without lasting negative consequences, a very low-risk low-reward situation. By being a puzzle, the object is attempting to convince the user that it must be completed.
\end{quote}
{\tiny Source: \href{https://www.everydayrhetoric.com/2020/03/the-rhetoric-of-puzzles/}{Every Day Rhetoric}}
\end{frame}


\mode<all>{
\part{Basic Constraint Propagation}
}
\begin{frame}
\partpage
\end{frame}
\mode<all>{
\input{../sendmore/body}
}

\mode<all>{
\part{Global Constraints}
}
\begin{frame}
\partpage
\end{frame}
\mode<all>{
\input{../sudoku/body}
}

\clearpage
\mode<all>{
\part{Customizing Search}
}
\begin{frame}
\partpage
\end{frame}
\mode<all>{
\input{../nqueen/body}
}


%% \clearpage
%% \mode<all>{
%%   \part{Problem Decomposition}
%% }
%% \begin{frame}
%%   \partpage
%% \end{frame}
%% \mode<all>{
%%   \input{../party/body}
%%   }

%% \clearpage
%% \mode<all>{
%%   \part{More Global Constraints}
%% }
%% \begin{frame}
%%   \partpage
%% \end{frame}
%% \mode<all>{
%%   \input{../car/body}
%%   }

%% \clearpage
%% \mode<all>{
%%   \part{Methodology}
%% }
%% \begin{frame}
%%   \partpage
%% \end{frame}
%% \mode<all>{
%%   \input{../methodology/body}
%%   }

\clearpage
\mode<all>{
\part{What is missing?}
}
\begin{frame}
\partpage
\end{frame}

\begin{frame}
\frametitle{Many Specialized Topics}
\begin{itemize}
\item How to design efficient core engine
\item Hybrids with LP/MIP tools
\item Hybrids with SAT
\item Symmetry breaking
\item Use of MDD/BDD to encode sets of solutions
\item High level modelling tools
\item Debugging/visualization
\end{itemize}
\end{frame}

\begin{frame}
\frametitle{Reformulation}
\begin{itemize}
\item Just because the user has modelled it this way, it doesn't mean we have to solve it that way
\begin{itemize}
\item Replace some constraint(s) by other, equivalent constraints
\item Because we don't have that constraint in our system
\item For performance
\end{itemize}
\end{itemize}
\end{frame}

\begin{frame}
\frametitle{Learning}
\begin{itemize}
\item While solving the problem we can learn how to strengthen the model/search 
\begin{itemize}
\item Understand which constraints/method contribute to propagation and change schedule
\item Learn no-good constraints by explaining failure
\item Adapt search strategy based on search experience
\end{itemize}
\end{itemize}
\end{frame}

%% \begin{frame}
%% \frametitle{Refined Process}
%% \begin{center}
%% \begin{tikzpicture}
%% \node[shape=rectangle,fill=pantone127-4] (problem) at (3,7) {Problem};
%% \node (human) at (3,6) {Human};
%% \node[shape=rectangle,fill=pantone127-4] (model) at (3,5) {User Model};
%% \node (reformulation) at (3,4) {Reformulation};
%% \node[shape=rectangle,fill=pantone127-4] (implementation) at (3,3) {Implementation Model};
%% \node (solver) at (3,2) {Constraint Solver/Search};
%% \node[shape=rectangle,fill=pantone157-8] (solution) at (3,1) {Solution};
%% \draw[-] (problem) -- (human);
%% \draw[->] (human) -- (model);
%% \draw[-] (model) -- (reformulation);
%% \draw[->] (reformulation) -- (implementation);
%% \draw[-] (implementation) -- (solver);
%% \draw[->] (solver) -- (0,2) -- (0,3) -- (implementation);
%% \draw[->] (solver) -- (solution);
%% \end{tikzpicture}
%% \end{center}
%% \end{frame}

%\begin{frame}
%\frametitle{What is CP actually used for?}
%\url{http://hsimonis.wordpress.com}
%\includegraphics[width=10cm]{../introduction/images/blog}
%\end{frame}

%% \begin{frame}
%% \frametitle{Exercises}
%% \begin{itemize}
%% \item Install Constraint Solver of your Choice
%% \item MiniZinc (MiningZinc) will be used in later sessions
%% \item Run Sendmoremoney, Sudoku, Queens examples
%% \item Can you choose which propagator to use for alldifferent?
%% \item Can you use different search strategies?
%% \end{itemize}
%% \end{frame}

\begin{frame}
\frametitle{More Learning Resources}
\begin{itemize}
\item Survey of Methods, Resources, and Formats for Teaching Constraint Programming
\begin{itemize}
\item by Tejas Santanam, Helmut Simonis
\item \url{https://doi.org/10.48550/arXiv.2403.12717}
\item Based on survey of community for WTCP 2023
\item \url{https://hsimonis.github.io/WTCP2023/}
\end{itemize}
\end{itemize}
\end{frame}


%%   }

\clearpage
\mode<all>{
\part{What is missing?}
}
\begin{frame}
\partpage
\end{frame}

\begin{frame}
\frametitle{Many Specialized Topics}
\begin{itemize}
\item How to design efficient core engine
\item Hybrids with LP/MIP tools
\item Hybrids with SAT
\item Symmetry breaking
\item Use of MDD/BDD to encode sets of solutions
\item High level modelling tools
\item Debugging/visualization
\end{itemize}
\end{frame}

\begin{frame}
\frametitle{Reformulation}
\begin{itemize}
\item Just because the user has modelled it this way, it doesn't mean we have to solve it that way
\begin{itemize}
\item Replace some constraint(s) by other, equivalent constraints
\item Because we don't have that constraint in our system
\item For performance
\end{itemize}
\end{itemize}
\end{frame}

\begin{frame}
\frametitle{Learning}
\begin{itemize}
\item While solving the problem we can learn how to strengthen the model/search 
\begin{itemize}
\item Understand which constraints/method contribute to propagation and change schedule
\item Learn no-good constraints by explaining failure
\item Adapt search strategy based on search experience
\end{itemize}
\end{itemize}
\end{frame}

%% \begin{frame}
%% \frametitle{Refined Process}
%% \begin{center}
%% \begin{tikzpicture}
%% \node[shape=rectangle,fill=pantone127-4] (problem) at (3,7) {Problem};
%% \node (human) at (3,6) {Human};
%% \node[shape=rectangle,fill=pantone127-4] (model) at (3,5) {User Model};
%% \node (reformulation) at (3,4) {Reformulation};
%% \node[shape=rectangle,fill=pantone127-4] (implementation) at (3,3) {Implementation Model};
%% \node (solver) at (3,2) {Constraint Solver/Search};
%% \node[shape=rectangle,fill=pantone157-8] (solution) at (3,1) {Solution};
%% \draw[-] (problem) -- (human);
%% \draw[->] (human) -- (model);
%% \draw[-] (model) -- (reformulation);
%% \draw[->] (reformulation) -- (implementation);
%% \draw[-] (implementation) -- (solver);
%% \draw[->] (solver) -- (0,2) -- (0,3) -- (implementation);
%% \draw[->] (solver) -- (solution);
%% \end{tikzpicture}
%% \end{center}
%% \end{frame}

%\begin{frame}
%\frametitle{What is CP actually used for?}
%\url{http://hsimonis.wordpress.com}
%\includegraphics[width=10cm]{../introduction/images/blog}
%\end{frame}

%% \begin{frame}
%% \frametitle{Exercises}
%% \begin{itemize}
%% \item Install Constraint Solver of your Choice
%% \item MiniZinc (MiningZinc) will be used in later sessions
%% \item Run Sendmoremoney, Sudoku, Queens examples
%% \item Can you choose which propagator to use for alldifferent?
%% \item Can you use different search strategies?
%% \end{itemize}
%% \end{frame}

\begin{frame}
\frametitle{More Learning Resources}
\begin{itemize}
\item Survey of Methods, Resources, and Formats for Teaching Constraint Programming
\begin{itemize}
\item by Tejas Santanam, Helmut Simonis
\item \url{https://doi.org/10.48550/arXiv.2403.12717}
\item Based on survey of community for WTCP 2023
\item \url{https://hsimonis.github.io/WTCP2023/}
\end{itemize}
\end{itemize}
\end{frame}


}

\mode<all>{
\part{Global Constraints}
}
\begin{frame}
\partpage
\end{frame}
\mode<all>{
\begin{frame}
\frametitle{Objectives}
\begin{itemize}
\item Overview of Core Constraint Programming
\item Three Main Concepts
\begin{itemize}
\item Constraint Propagation
\item Global Constraints
\item Customizing Search
\end{itemize}
%\item Get Some Experience with MiniZinc
\item Topics will be treated in more detail in later parts of the school
\item Based on Examples, not Formal Description
\end{itemize}
\end{frame}

\begin{frame}
\frametitle{Outline}
\begin{itemize}
\item Why Constraint Programming?
\item Constraint Propagation
\item Global Constraints
\item Customizing Search
%\item Choices in Modelling
%\item Decomposition
%\item More Global Constraints
%\item Methodology
%\item What is missing?
\end{itemize}
\end{frame}

% \begin{frame}
  % \frametitle{Using MiniZinc IDE}
  % \begin{itemize}
  % \item Developed in the Australian NICTA project
  % \item Maintained by Monash University
  % \item Modelling tool with multiple back-end solvers
  % \item Available from \url{https://www.minizinc.org/}
  % \end{itemize}
% \end{frame}


% \begin{frame}
% \frametitle{Examples in ECLiPSe}
% \begin{itemize}
% \item Open sourced constraint programming language
% \item Development goes back to 1985
% \item ECRC, ICL, IC-Parc, PTL, Cisco
% \item \url{https://eclipseclp.org/}
% \item Specialities
% \begin{itemize}
% \item Develop new solvers for specific domains
% \item Integration with MIP
% \end{itemize}
% \item Not included in bundled MiniZinc IDE
% \item Specialized visualization tools used here
  % \begin{itemize}
  % \item CP-Viz, Simonis et al. 2010
    % \end{itemize}
% \end{itemize}
% \end{frame}

\begin{frame}
\frametitle{Tutorial Based on ECLiPSe ELearning Course}
\begin{itemize}
\item Self-study course in constraint programming
\item Supported by Cisco Systems and Silicon Valley Community Foundation
\item Multi-media format, video lectures, slides, handout etc
\item {\small \url{https://eclipseclp.org/ELearning/index.html}}
\end{itemize}
\end{frame}

\begin{frame}
\frametitle{Also Part of CRT-AI Constraint Week}
\begin{itemize}
\item Annual one week course on CP and Optimization in Ireland
\item Part of national training program for PhD students in AI
\item \url{https://www.crt-ai.ie/}
\end{itemize}
\end{frame}



\begin{frame}
\frametitle{Constraint Programming - in a nutshell}
\begin{itemize}
\item Declarative description of problems with
\begin{itemize}
\item {\em Variables} which range over (finite) sets of values
\item {\em Constraints} over subsets of variables which restrict possible value combinations
\item A {\em solution} is a value assignment which satisfies all constraints
\end{itemize}

\item Constraint propagation/reasoning
\begin{itemize}
\item Removing inconsistent values for variables
\item Detect failure if constraint can not be satisfied
\item Interaction of constraints via shared variables
\item Incomplete
\end{itemize}

\item Search
\begin{itemize}
\item User controlled assignment of values to variables
\item Each step triggers constraint propagation 
\end{itemize}
\item Different domains require/allow different methods
\end{itemize}
\end{frame}

\begin{frame}
  \frametitle{Constraint Programming is Different}
  \begin{itemize}
  \item Declarative Programming
    \begin{itemize}
    \item Concentrate on what you want
      \item Not how to get there
      \item Program != Algorithm
      \item Program = Model
    \end{itemize}
    \item Applied to Combinatorial Problems
      \begin{itemize}
        \item No complete polynomial algorithms known (exist?)
        \item CP less ad-hoc than heuristics
        \item Models can evolve
  \end{itemize}
  \end{itemize}
  \end{frame}
    
\begin{frame}
  \frametitle{A Subtractive Process}
  \begin{textblock}{4}(8,-3)
    \includegraphics[width=4cm]{../introduction/images/stages}
  \end{textblock}
  \vfill
  \begin{quote}
    ``Oh, bosh, as Mr. Ruskin says. Sculpture, per se, is the simplest thing in the world. All you have to do is to take a big chunk of marble and a hammer and chisel, make up your mind what you are about to create and chip off all the marble you don't want.''-Paris Gaulois.
  \end{quote}
  
  {\tiny Source: \url{https://quoteinvestigator.com/2014/06/22/chip-away/}}
\end{frame}

\begin{frame}
\frametitle{Basic Process}
\begin{center}
\begin{tikzpicture}
\node[shape=rectangle,fill=pantone127-4] (problem) at (2,5) {Problem};
\node (human) at (2,4) {Human};
\node[shape=rectangle,fill=pantone127-4] (model) at (2,3) {Model};
\node (solver) at (2,2) {Constraint Solver/Search};
\node[shape=rectangle,fill=pantone157-8] (solution) at (2,1) {Solution};
\draw[-] (problem) -- (human);
\draw[->] (human) -- (model);
\draw[-] (model) -- (solver);
\draw[->] (solver) -- (solution);
\end{tikzpicture}
\end{center}
\end{frame}

\begin{frame}
\frametitle{More Realistic}
\begin{center}
\begin{tikzpicture}[xscale=4.0]
\node[shape=rectangle,fill=pantone127-4] (problem) at (2,5) {Problem};
\node (human) at (2,4) {Human};
\node[shape=rectangle,fill=pantone127-4] (model) at (2,3) {Model};
\node (solver) at (2,2) {Constraint Solver/Search};
\node[shape=rectangle,fill=pantone24-8] (hangs) at (1,1) {Hangs};
\node[shape=rectangle,fill=pantone157-8] (solution) at (2,1) {Solution};
\node[shape=rectangle,fill=pantone24-8] (wrong) at (3,1) {Wrong Solution};
\draw[-] (problem) -- (human);
\draw[->] (human) -- (model);
\draw[-] (model) -- (solver);
\draw[->] (solver) -- (hangs);
\draw[->] (solver) -- (solution);
\draw[->] (solver) -- (wrong);
\draw[->] (hangs) -- (1,4) -- (human);
\draw[->] (wrong) -- (3,4) -- (human);
\end{tikzpicture}
\end{center}
\end{frame}

\begin{frame}
\frametitle{Dual Role of Model}
\begin{itemize}
\item Allows Human to Express Problem
\begin{itemize}
\item Close to Problem Domain
\item Constraints as Abstractions
\end{itemize}
\item Allows Solver to Execute
\begin{itemize}
\item Variables as Communication Mechanism
\item Constraints as Algorithms
\end{itemize}
\end{itemize}
\end{frame}

\begin{frame}
\frametitle{Modelling Frameworks}
\begin{itemize}
\item \href{https://www.minizinc.org/}{MiniZinc} (NICTA, Monash University, Australia)
\item \href{https://github.com/eomahony/Numberjack}{NumberJack} (Insight, Ireland) 
\item \href{https://www-users.york.ac.uk/peter.nightingale/savilerow//}{EssencePrime/SavilleRow} (UK) 
\item \href{https://cpmpy.readthedocs.io/en/latest/}{CPMpy} (KU Leuven)
\item Allow use of multiple back-end solvers
\item Compile model into variants for each solver
\item A priori solver independent model(CP, MIP, SAT)
\end{itemize}
\end{frame} 

\begin{frame}
\frametitle{Framework Process}
\begin{center}
\begin{tikzpicture}[xscale=2]
\node[shape=rectangle,fill=pantone127-4] (problem) at (2.5,5) {Problem};
\node (human) at (2.5,4) {Human};
\node[shape=rectangle,fill=pantone127-4] (model) at (2.5,3) {Model};
\node (compile) at (2.5,2) {Compile/Reformulate};
\node (solver1) at (1,1) {CP};
\node (solver2) at (2,1) {MIP};
\node (solver3) at (3,1) {SAT};
\node (solver4) at (4,1) {Other};
\node[shape=rectangle,fill=pantone157-8] (solution1) at (1,0) {Solution};
\node[shape=rectangle,fill=pantone157-8] (solution2) at (2,0) {Solution};
\node[shape=rectangle,fill=pantone157-8] (solution3) at (3,0) {Solution};
\node[shape=rectangle,fill=pantone157-8] (solution4) at (4,0) {Solution};
\draw[-] (problem) -- (human);
\draw[->] (human) -- (model);
\draw[-] (model) -- (compile);
\draw[-] (compile) -- (solver1);
\draw[-] (compile) -- (solver2);
\draw[-] (compile) -- (solver3);
\draw[-] (compile) -- (solver4);
\draw[->] (solver1) -- (solution1);
\draw[->] (solver2) -- (solution2);
\draw[->] (solver3) -- (solution3);
\draw[->] (solver4) -- (solution4);
\end{tikzpicture}
\end{center}
\end{frame}

% \begin{frame}
  % \frametitle{Do It Now!}
  % \begin{itemize}
  % \item Download and install Minizinc
  % \item \url{https://www.minizinc.org/}
  % \end{itemize}
% \end{frame}


\begin{frame}
\frametitle{Why use Puzzles as Examples?}
\begin{itemize}
\item Easy to understand the problem
\item Solvable by hand without specialized knowledge
\item Possible to compare automated to manual solving process
\end{itemize}

\begin{quote}
The puzzle, though inanimate, is presented as a solvable problem without lasting negative consequences, a very low-risk low-reward situation. By being a puzzle, the object is attempting to convince the user that it must be completed.
\end{quote}
{\tiny Source: \href{https://www.everydayrhetoric.com/2020/03/the-rhetoric-of-puzzles/}{Every Day Rhetoric}}
\end{frame}


\mode<all>{
\part{Basic Constraint Propagation}
}
\begin{frame}
\partpage
\end{frame}
\mode<all>{
\begin{frame}
\frametitle{Objectives}
\begin{itemize}
\item Overview of Core Constraint Programming
\item Three Main Concepts
\begin{itemize}
\item Constraint Propagation
\item Global Constraints
\item Customizing Search
\end{itemize}
%\item Get Some Experience with MiniZinc
\item Topics will be treated in more detail in later parts of the school
\item Based on Examples, not Formal Description
\end{itemize}
\end{frame}

\begin{frame}
\frametitle{Outline}
\begin{itemize}
\item Why Constraint Programming?
\item Constraint Propagation
\item Global Constraints
\item Customizing Search
%\item Choices in Modelling
%\item Decomposition
%\item More Global Constraints
%\item Methodology
%\item What is missing?
\end{itemize}
\end{frame}

% \begin{frame}
  % \frametitle{Using MiniZinc IDE}
  % \begin{itemize}
  % \item Developed in the Australian NICTA project
  % \item Maintained by Monash University
  % \item Modelling tool with multiple back-end solvers
  % \item Available from \url{https://www.minizinc.org/}
  % \end{itemize}
% \end{frame}


% \begin{frame}
% \frametitle{Examples in ECLiPSe}
% \begin{itemize}
% \item Open sourced constraint programming language
% \item Development goes back to 1985
% \item ECRC, ICL, IC-Parc, PTL, Cisco
% \item \url{https://eclipseclp.org/}
% \item Specialities
% \begin{itemize}
% \item Develop new solvers for specific domains
% \item Integration with MIP
% \end{itemize}
% \item Not included in bundled MiniZinc IDE
% \item Specialized visualization tools used here
  % \begin{itemize}
  % \item CP-Viz, Simonis et al. 2010
    % \end{itemize}
% \end{itemize}
% \end{frame}

\begin{frame}
\frametitle{Tutorial Based on ECLiPSe ELearning Course}
\begin{itemize}
\item Self-study course in constraint programming
\item Supported by Cisco Systems and Silicon Valley Community Foundation
\item Multi-media format, video lectures, slides, handout etc
\item {\small \url{https://eclipseclp.org/ELearning/index.html}}
\end{itemize}
\end{frame}

\begin{frame}
\frametitle{Also Part of CRT-AI Constraint Week}
\begin{itemize}
\item Annual one week course on CP and Optimization in Ireland
\item Part of national training program for PhD students in AI
\item \url{https://www.crt-ai.ie/}
\end{itemize}
\end{frame}



\begin{frame}
\frametitle{Constraint Programming - in a nutshell}
\begin{itemize}
\item Declarative description of problems with
\begin{itemize}
\item {\em Variables} which range over (finite) sets of values
\item {\em Constraints} over subsets of variables which restrict possible value combinations
\item A {\em solution} is a value assignment which satisfies all constraints
\end{itemize}

\item Constraint propagation/reasoning
\begin{itemize}
\item Removing inconsistent values for variables
\item Detect failure if constraint can not be satisfied
\item Interaction of constraints via shared variables
\item Incomplete
\end{itemize}

\item Search
\begin{itemize}
\item User controlled assignment of values to variables
\item Each step triggers constraint propagation 
\end{itemize}
\item Different domains require/allow different methods
\end{itemize}
\end{frame}

\begin{frame}
  \frametitle{Constraint Programming is Different}
  \begin{itemize}
  \item Declarative Programming
    \begin{itemize}
    \item Concentrate on what you want
      \item Not how to get there
      \item Program != Algorithm
      \item Program = Model
    \end{itemize}
    \item Applied to Combinatorial Problems
      \begin{itemize}
        \item No complete polynomial algorithms known (exist?)
        \item CP less ad-hoc than heuristics
        \item Models can evolve
  \end{itemize}
  \end{itemize}
  \end{frame}
    
\begin{frame}
  \frametitle{A Subtractive Process}
  \begin{textblock}{4}(8,-3)
    \includegraphics[width=4cm]{../introduction/images/stages}
  \end{textblock}
  \vfill
  \begin{quote}
    ``Oh, bosh, as Mr. Ruskin says. Sculpture, per se, is the simplest thing in the world. All you have to do is to take a big chunk of marble and a hammer and chisel, make up your mind what you are about to create and chip off all the marble you don't want.''-Paris Gaulois.
  \end{quote}
  
  {\tiny Source: \url{https://quoteinvestigator.com/2014/06/22/chip-away/}}
\end{frame}

\begin{frame}
\frametitle{Basic Process}
\begin{center}
\begin{tikzpicture}
\node[shape=rectangle,fill=pantone127-4] (problem) at (2,5) {Problem};
\node (human) at (2,4) {Human};
\node[shape=rectangle,fill=pantone127-4] (model) at (2,3) {Model};
\node (solver) at (2,2) {Constraint Solver/Search};
\node[shape=rectangle,fill=pantone157-8] (solution) at (2,1) {Solution};
\draw[-] (problem) -- (human);
\draw[->] (human) -- (model);
\draw[-] (model) -- (solver);
\draw[->] (solver) -- (solution);
\end{tikzpicture}
\end{center}
\end{frame}

\begin{frame}
\frametitle{More Realistic}
\begin{center}
\begin{tikzpicture}[xscale=4.0]
\node[shape=rectangle,fill=pantone127-4] (problem) at (2,5) {Problem};
\node (human) at (2,4) {Human};
\node[shape=rectangle,fill=pantone127-4] (model) at (2,3) {Model};
\node (solver) at (2,2) {Constraint Solver/Search};
\node[shape=rectangle,fill=pantone24-8] (hangs) at (1,1) {Hangs};
\node[shape=rectangle,fill=pantone157-8] (solution) at (2,1) {Solution};
\node[shape=rectangle,fill=pantone24-8] (wrong) at (3,1) {Wrong Solution};
\draw[-] (problem) -- (human);
\draw[->] (human) -- (model);
\draw[-] (model) -- (solver);
\draw[->] (solver) -- (hangs);
\draw[->] (solver) -- (solution);
\draw[->] (solver) -- (wrong);
\draw[->] (hangs) -- (1,4) -- (human);
\draw[->] (wrong) -- (3,4) -- (human);
\end{tikzpicture}
\end{center}
\end{frame}

\begin{frame}
\frametitle{Dual Role of Model}
\begin{itemize}
\item Allows Human to Express Problem
\begin{itemize}
\item Close to Problem Domain
\item Constraints as Abstractions
\end{itemize}
\item Allows Solver to Execute
\begin{itemize}
\item Variables as Communication Mechanism
\item Constraints as Algorithms
\end{itemize}
\end{itemize}
\end{frame}

\begin{frame}
\frametitle{Modelling Frameworks}
\begin{itemize}
\item \href{https://www.minizinc.org/}{MiniZinc} (NICTA, Monash University, Australia)
\item \href{https://github.com/eomahony/Numberjack}{NumberJack} (Insight, Ireland) 
\item \href{https://www-users.york.ac.uk/peter.nightingale/savilerow//}{EssencePrime/SavilleRow} (UK) 
\item \href{https://cpmpy.readthedocs.io/en/latest/}{CPMpy} (KU Leuven)
\item Allow use of multiple back-end solvers
\item Compile model into variants for each solver
\item A priori solver independent model(CP, MIP, SAT)
\end{itemize}
\end{frame} 

\begin{frame}
\frametitle{Framework Process}
\begin{center}
\begin{tikzpicture}[xscale=2]
\node[shape=rectangle,fill=pantone127-4] (problem) at (2.5,5) {Problem};
\node (human) at (2.5,4) {Human};
\node[shape=rectangle,fill=pantone127-4] (model) at (2.5,3) {Model};
\node (compile) at (2.5,2) {Compile/Reformulate};
\node (solver1) at (1,1) {CP};
\node (solver2) at (2,1) {MIP};
\node (solver3) at (3,1) {SAT};
\node (solver4) at (4,1) {Other};
\node[shape=rectangle,fill=pantone157-8] (solution1) at (1,0) {Solution};
\node[shape=rectangle,fill=pantone157-8] (solution2) at (2,0) {Solution};
\node[shape=rectangle,fill=pantone157-8] (solution3) at (3,0) {Solution};
\node[shape=rectangle,fill=pantone157-8] (solution4) at (4,0) {Solution};
\draw[-] (problem) -- (human);
\draw[->] (human) -- (model);
\draw[-] (model) -- (compile);
\draw[-] (compile) -- (solver1);
\draw[-] (compile) -- (solver2);
\draw[-] (compile) -- (solver3);
\draw[-] (compile) -- (solver4);
\draw[->] (solver1) -- (solution1);
\draw[->] (solver2) -- (solution2);
\draw[->] (solver3) -- (solution3);
\draw[->] (solver4) -- (solution4);
\end{tikzpicture}
\end{center}
\end{frame}

% \begin{frame}
  % \frametitle{Do It Now!}
  % \begin{itemize}
  % \item Download and install Minizinc
  % \item \url{https://www.minizinc.org/}
  % \end{itemize}
% \end{frame}


\begin{frame}
\frametitle{Why use Puzzles as Examples?}
\begin{itemize}
\item Easy to understand the problem
\item Solvable by hand without specialized knowledge
\item Possible to compare automated to manual solving process
\end{itemize}

\begin{quote}
The puzzle, though inanimate, is presented as a solvable problem without lasting negative consequences, a very low-risk low-reward situation. By being a puzzle, the object is attempting to convince the user that it must be completed.
\end{quote}
{\tiny Source: \href{https://www.everydayrhetoric.com/2020/03/the-rhetoric-of-puzzles/}{Every Day Rhetoric}}
\end{frame}


\mode<all>{
\part{Basic Constraint Propagation}
}
\begin{frame}
\partpage
\end{frame}
\mode<all>{
\input{../sendmore/body}
}

\mode<all>{
\part{Global Constraints}
}
\begin{frame}
\partpage
\end{frame}
\mode<all>{
\input{../sudoku/body}
}

\clearpage
\mode<all>{
\part{Customizing Search}
}
\begin{frame}
\partpage
\end{frame}
\mode<all>{
\input{../nqueen/body}
}


%% \clearpage
%% \mode<all>{
%%   \part{Problem Decomposition}
%% }
%% \begin{frame}
%%   \partpage
%% \end{frame}
%% \mode<all>{
%%   \input{../party/body}
%%   }

%% \clearpage
%% \mode<all>{
%%   \part{More Global Constraints}
%% }
%% \begin{frame}
%%   \partpage
%% \end{frame}
%% \mode<all>{
%%   \input{../car/body}
%%   }

%% \clearpage
%% \mode<all>{
%%   \part{Methodology}
%% }
%% \begin{frame}
%%   \partpage
%% \end{frame}
%% \mode<all>{
%%   \input{../methodology/body}
%%   }

\clearpage
\mode<all>{
\part{What is missing?}
}
\begin{frame}
\partpage
\end{frame}

\begin{frame}
\frametitle{Many Specialized Topics}
\begin{itemize}
\item How to design efficient core engine
\item Hybrids with LP/MIP tools
\item Hybrids with SAT
\item Symmetry breaking
\item Use of MDD/BDD to encode sets of solutions
\item High level modelling tools
\item Debugging/visualization
\end{itemize}
\end{frame}

\begin{frame}
\frametitle{Reformulation}
\begin{itemize}
\item Just because the user has modelled it this way, it doesn't mean we have to solve it that way
\begin{itemize}
\item Replace some constraint(s) by other, equivalent constraints
\item Because we don't have that constraint in our system
\item For performance
\end{itemize}
\end{itemize}
\end{frame}

\begin{frame}
\frametitle{Learning}
\begin{itemize}
\item While solving the problem we can learn how to strengthen the model/search 
\begin{itemize}
\item Understand which constraints/method contribute to propagation and change schedule
\item Learn no-good constraints by explaining failure
\item Adapt search strategy based on search experience
\end{itemize}
\end{itemize}
\end{frame}

%% \begin{frame}
%% \frametitle{Refined Process}
%% \begin{center}
%% \begin{tikzpicture}
%% \node[shape=rectangle,fill=pantone127-4] (problem) at (3,7) {Problem};
%% \node (human) at (3,6) {Human};
%% \node[shape=rectangle,fill=pantone127-4] (model) at (3,5) {User Model};
%% \node (reformulation) at (3,4) {Reformulation};
%% \node[shape=rectangle,fill=pantone127-4] (implementation) at (3,3) {Implementation Model};
%% \node (solver) at (3,2) {Constraint Solver/Search};
%% \node[shape=rectangle,fill=pantone157-8] (solution) at (3,1) {Solution};
%% \draw[-] (problem) -- (human);
%% \draw[->] (human) -- (model);
%% \draw[-] (model) -- (reformulation);
%% \draw[->] (reformulation) -- (implementation);
%% \draw[-] (implementation) -- (solver);
%% \draw[->] (solver) -- (0,2) -- (0,3) -- (implementation);
%% \draw[->] (solver) -- (solution);
%% \end{tikzpicture}
%% \end{center}
%% \end{frame}

%\begin{frame}
%\frametitle{What is CP actually used for?}
%\url{http://hsimonis.wordpress.com}
%\includegraphics[width=10cm]{../introduction/images/blog}
%\end{frame}

%% \begin{frame}
%% \frametitle{Exercises}
%% \begin{itemize}
%% \item Install Constraint Solver of your Choice
%% \item MiniZinc (MiningZinc) will be used in later sessions
%% \item Run Sendmoremoney, Sudoku, Queens examples
%% \item Can you choose which propagator to use for alldifferent?
%% \item Can you use different search strategies?
%% \end{itemize}
%% \end{frame}

\begin{frame}
\frametitle{More Learning Resources}
\begin{itemize}
\item Survey of Methods, Resources, and Formats for Teaching Constraint Programming
\begin{itemize}
\item by Tejas Santanam, Helmut Simonis
\item \url{https://doi.org/10.48550/arXiv.2403.12717}
\item Based on survey of community for WTCP 2023
\item \url{https://hsimonis.github.io/WTCP2023/}
\end{itemize}
\end{itemize}
\end{frame}


}

\mode<all>{
\part{Global Constraints}
}
\begin{frame}
\partpage
\end{frame}
\mode<all>{
\begin{frame}
\frametitle{Objectives}
\begin{itemize}
\item Overview of Core Constraint Programming
\item Three Main Concepts
\begin{itemize}
\item Constraint Propagation
\item Global Constraints
\item Customizing Search
\end{itemize}
%\item Get Some Experience with MiniZinc
\item Topics will be treated in more detail in later parts of the school
\item Based on Examples, not Formal Description
\end{itemize}
\end{frame}

\begin{frame}
\frametitle{Outline}
\begin{itemize}
\item Why Constraint Programming?
\item Constraint Propagation
\item Global Constraints
\item Customizing Search
%\item Choices in Modelling
%\item Decomposition
%\item More Global Constraints
%\item Methodology
%\item What is missing?
\end{itemize}
\end{frame}

% \begin{frame}
  % \frametitle{Using MiniZinc IDE}
  % \begin{itemize}
  % \item Developed in the Australian NICTA project
  % \item Maintained by Monash University
  % \item Modelling tool with multiple back-end solvers
  % \item Available from \url{https://www.minizinc.org/}
  % \end{itemize}
% \end{frame}


% \begin{frame}
% \frametitle{Examples in ECLiPSe}
% \begin{itemize}
% \item Open sourced constraint programming language
% \item Development goes back to 1985
% \item ECRC, ICL, IC-Parc, PTL, Cisco
% \item \url{https://eclipseclp.org/}
% \item Specialities
% \begin{itemize}
% \item Develop new solvers for specific domains
% \item Integration with MIP
% \end{itemize}
% \item Not included in bundled MiniZinc IDE
% \item Specialized visualization tools used here
  % \begin{itemize}
  % \item CP-Viz, Simonis et al. 2010
    % \end{itemize}
% \end{itemize}
% \end{frame}

\begin{frame}
\frametitle{Tutorial Based on ECLiPSe ELearning Course}
\begin{itemize}
\item Self-study course in constraint programming
\item Supported by Cisco Systems and Silicon Valley Community Foundation
\item Multi-media format, video lectures, slides, handout etc
\item {\small \url{https://eclipseclp.org/ELearning/index.html}}
\end{itemize}
\end{frame}

\begin{frame}
\frametitle{Also Part of CRT-AI Constraint Week}
\begin{itemize}
\item Annual one week course on CP and Optimization in Ireland
\item Part of national training program for PhD students in AI
\item \url{https://www.crt-ai.ie/}
\end{itemize}
\end{frame}



\begin{frame}
\frametitle{Constraint Programming - in a nutshell}
\begin{itemize}
\item Declarative description of problems with
\begin{itemize}
\item {\em Variables} which range over (finite) sets of values
\item {\em Constraints} over subsets of variables which restrict possible value combinations
\item A {\em solution} is a value assignment which satisfies all constraints
\end{itemize}

\item Constraint propagation/reasoning
\begin{itemize}
\item Removing inconsistent values for variables
\item Detect failure if constraint can not be satisfied
\item Interaction of constraints via shared variables
\item Incomplete
\end{itemize}

\item Search
\begin{itemize}
\item User controlled assignment of values to variables
\item Each step triggers constraint propagation 
\end{itemize}
\item Different domains require/allow different methods
\end{itemize}
\end{frame}

\begin{frame}
  \frametitle{Constraint Programming is Different}
  \begin{itemize}
  \item Declarative Programming
    \begin{itemize}
    \item Concentrate on what you want
      \item Not how to get there
      \item Program != Algorithm
      \item Program = Model
    \end{itemize}
    \item Applied to Combinatorial Problems
      \begin{itemize}
        \item No complete polynomial algorithms known (exist?)
        \item CP less ad-hoc than heuristics
        \item Models can evolve
  \end{itemize}
  \end{itemize}
  \end{frame}
    
\begin{frame}
  \frametitle{A Subtractive Process}
  \begin{textblock}{4}(8,-3)
    \includegraphics[width=4cm]{../introduction/images/stages}
  \end{textblock}
  \vfill
  \begin{quote}
    ``Oh, bosh, as Mr. Ruskin says. Sculpture, per se, is the simplest thing in the world. All you have to do is to take a big chunk of marble and a hammer and chisel, make up your mind what you are about to create and chip off all the marble you don't want.''-Paris Gaulois.
  \end{quote}
  
  {\tiny Source: \url{https://quoteinvestigator.com/2014/06/22/chip-away/}}
\end{frame}

\begin{frame}
\frametitle{Basic Process}
\begin{center}
\begin{tikzpicture}
\node[shape=rectangle,fill=pantone127-4] (problem) at (2,5) {Problem};
\node (human) at (2,4) {Human};
\node[shape=rectangle,fill=pantone127-4] (model) at (2,3) {Model};
\node (solver) at (2,2) {Constraint Solver/Search};
\node[shape=rectangle,fill=pantone157-8] (solution) at (2,1) {Solution};
\draw[-] (problem) -- (human);
\draw[->] (human) -- (model);
\draw[-] (model) -- (solver);
\draw[->] (solver) -- (solution);
\end{tikzpicture}
\end{center}
\end{frame}

\begin{frame}
\frametitle{More Realistic}
\begin{center}
\begin{tikzpicture}[xscale=4.0]
\node[shape=rectangle,fill=pantone127-4] (problem) at (2,5) {Problem};
\node (human) at (2,4) {Human};
\node[shape=rectangle,fill=pantone127-4] (model) at (2,3) {Model};
\node (solver) at (2,2) {Constraint Solver/Search};
\node[shape=rectangle,fill=pantone24-8] (hangs) at (1,1) {Hangs};
\node[shape=rectangle,fill=pantone157-8] (solution) at (2,1) {Solution};
\node[shape=rectangle,fill=pantone24-8] (wrong) at (3,1) {Wrong Solution};
\draw[-] (problem) -- (human);
\draw[->] (human) -- (model);
\draw[-] (model) -- (solver);
\draw[->] (solver) -- (hangs);
\draw[->] (solver) -- (solution);
\draw[->] (solver) -- (wrong);
\draw[->] (hangs) -- (1,4) -- (human);
\draw[->] (wrong) -- (3,4) -- (human);
\end{tikzpicture}
\end{center}
\end{frame}

\begin{frame}
\frametitle{Dual Role of Model}
\begin{itemize}
\item Allows Human to Express Problem
\begin{itemize}
\item Close to Problem Domain
\item Constraints as Abstractions
\end{itemize}
\item Allows Solver to Execute
\begin{itemize}
\item Variables as Communication Mechanism
\item Constraints as Algorithms
\end{itemize}
\end{itemize}
\end{frame}

\begin{frame}
\frametitle{Modelling Frameworks}
\begin{itemize}
\item \href{https://www.minizinc.org/}{MiniZinc} (NICTA, Monash University, Australia)
\item \href{https://github.com/eomahony/Numberjack}{NumberJack} (Insight, Ireland) 
\item \href{https://www-users.york.ac.uk/peter.nightingale/savilerow//}{EssencePrime/SavilleRow} (UK) 
\item \href{https://cpmpy.readthedocs.io/en/latest/}{CPMpy} (KU Leuven)
\item Allow use of multiple back-end solvers
\item Compile model into variants for each solver
\item A priori solver independent model(CP, MIP, SAT)
\end{itemize}
\end{frame} 

\begin{frame}
\frametitle{Framework Process}
\begin{center}
\begin{tikzpicture}[xscale=2]
\node[shape=rectangle,fill=pantone127-4] (problem) at (2.5,5) {Problem};
\node (human) at (2.5,4) {Human};
\node[shape=rectangle,fill=pantone127-4] (model) at (2.5,3) {Model};
\node (compile) at (2.5,2) {Compile/Reformulate};
\node (solver1) at (1,1) {CP};
\node (solver2) at (2,1) {MIP};
\node (solver3) at (3,1) {SAT};
\node (solver4) at (4,1) {Other};
\node[shape=rectangle,fill=pantone157-8] (solution1) at (1,0) {Solution};
\node[shape=rectangle,fill=pantone157-8] (solution2) at (2,0) {Solution};
\node[shape=rectangle,fill=pantone157-8] (solution3) at (3,0) {Solution};
\node[shape=rectangle,fill=pantone157-8] (solution4) at (4,0) {Solution};
\draw[-] (problem) -- (human);
\draw[->] (human) -- (model);
\draw[-] (model) -- (compile);
\draw[-] (compile) -- (solver1);
\draw[-] (compile) -- (solver2);
\draw[-] (compile) -- (solver3);
\draw[-] (compile) -- (solver4);
\draw[->] (solver1) -- (solution1);
\draw[->] (solver2) -- (solution2);
\draw[->] (solver3) -- (solution3);
\draw[->] (solver4) -- (solution4);
\end{tikzpicture}
\end{center}
\end{frame}

% \begin{frame}
  % \frametitle{Do It Now!}
  % \begin{itemize}
  % \item Download and install Minizinc
  % \item \url{https://www.minizinc.org/}
  % \end{itemize}
% \end{frame}


\begin{frame}
\frametitle{Why use Puzzles as Examples?}
\begin{itemize}
\item Easy to understand the problem
\item Solvable by hand without specialized knowledge
\item Possible to compare automated to manual solving process
\end{itemize}

\begin{quote}
The puzzle, though inanimate, is presented as a solvable problem without lasting negative consequences, a very low-risk low-reward situation. By being a puzzle, the object is attempting to convince the user that it must be completed.
\end{quote}
{\tiny Source: \href{https://www.everydayrhetoric.com/2020/03/the-rhetoric-of-puzzles/}{Every Day Rhetoric}}
\end{frame}


\mode<all>{
\part{Basic Constraint Propagation}
}
\begin{frame}
\partpage
\end{frame}
\mode<all>{
\input{../sendmore/body}
}

\mode<all>{
\part{Global Constraints}
}
\begin{frame}
\partpage
\end{frame}
\mode<all>{
\input{../sudoku/body}
}

\clearpage
\mode<all>{
\part{Customizing Search}
}
\begin{frame}
\partpage
\end{frame}
\mode<all>{
\input{../nqueen/body}
}


%% \clearpage
%% \mode<all>{
%%   \part{Problem Decomposition}
%% }
%% \begin{frame}
%%   \partpage
%% \end{frame}
%% \mode<all>{
%%   \input{../party/body}
%%   }

%% \clearpage
%% \mode<all>{
%%   \part{More Global Constraints}
%% }
%% \begin{frame}
%%   \partpage
%% \end{frame}
%% \mode<all>{
%%   \input{../car/body}
%%   }

%% \clearpage
%% \mode<all>{
%%   \part{Methodology}
%% }
%% \begin{frame}
%%   \partpage
%% \end{frame}
%% \mode<all>{
%%   \input{../methodology/body}
%%   }

\clearpage
\mode<all>{
\part{What is missing?}
}
\begin{frame}
\partpage
\end{frame}

\begin{frame}
\frametitle{Many Specialized Topics}
\begin{itemize}
\item How to design efficient core engine
\item Hybrids with LP/MIP tools
\item Hybrids with SAT
\item Symmetry breaking
\item Use of MDD/BDD to encode sets of solutions
\item High level modelling tools
\item Debugging/visualization
\end{itemize}
\end{frame}

\begin{frame}
\frametitle{Reformulation}
\begin{itemize}
\item Just because the user has modelled it this way, it doesn't mean we have to solve it that way
\begin{itemize}
\item Replace some constraint(s) by other, equivalent constraints
\item Because we don't have that constraint in our system
\item For performance
\end{itemize}
\end{itemize}
\end{frame}

\begin{frame}
\frametitle{Learning}
\begin{itemize}
\item While solving the problem we can learn how to strengthen the model/search 
\begin{itemize}
\item Understand which constraints/method contribute to propagation and change schedule
\item Learn no-good constraints by explaining failure
\item Adapt search strategy based on search experience
\end{itemize}
\end{itemize}
\end{frame}

%% \begin{frame}
%% \frametitle{Refined Process}
%% \begin{center}
%% \begin{tikzpicture}
%% \node[shape=rectangle,fill=pantone127-4] (problem) at (3,7) {Problem};
%% \node (human) at (3,6) {Human};
%% \node[shape=rectangle,fill=pantone127-4] (model) at (3,5) {User Model};
%% \node (reformulation) at (3,4) {Reformulation};
%% \node[shape=rectangle,fill=pantone127-4] (implementation) at (3,3) {Implementation Model};
%% \node (solver) at (3,2) {Constraint Solver/Search};
%% \node[shape=rectangle,fill=pantone157-8] (solution) at (3,1) {Solution};
%% \draw[-] (problem) -- (human);
%% \draw[->] (human) -- (model);
%% \draw[-] (model) -- (reformulation);
%% \draw[->] (reformulation) -- (implementation);
%% \draw[-] (implementation) -- (solver);
%% \draw[->] (solver) -- (0,2) -- (0,3) -- (implementation);
%% \draw[->] (solver) -- (solution);
%% \end{tikzpicture}
%% \end{center}
%% \end{frame}

%\begin{frame}
%\frametitle{What is CP actually used for?}
%\url{http://hsimonis.wordpress.com}
%\includegraphics[width=10cm]{../introduction/images/blog}
%\end{frame}

%% \begin{frame}
%% \frametitle{Exercises}
%% \begin{itemize}
%% \item Install Constraint Solver of your Choice
%% \item MiniZinc (MiningZinc) will be used in later sessions
%% \item Run Sendmoremoney, Sudoku, Queens examples
%% \item Can you choose which propagator to use for alldifferent?
%% \item Can you use different search strategies?
%% \end{itemize}
%% \end{frame}

\begin{frame}
\frametitle{More Learning Resources}
\begin{itemize}
\item Survey of Methods, Resources, and Formats for Teaching Constraint Programming
\begin{itemize}
\item by Tejas Santanam, Helmut Simonis
\item \url{https://doi.org/10.48550/arXiv.2403.12717}
\item Based on survey of community for WTCP 2023
\item \url{https://hsimonis.github.io/WTCP2023/}
\end{itemize}
\end{itemize}
\end{frame}


}

\clearpage
\mode<all>{
\part{Customizing Search}
}
\begin{frame}
\partpage
\end{frame}
\mode<all>{
\begin{frame}
\frametitle{Objectives}
\begin{itemize}
\item Overview of Core Constraint Programming
\item Three Main Concepts
\begin{itemize}
\item Constraint Propagation
\item Global Constraints
\item Customizing Search
\end{itemize}
%\item Get Some Experience with MiniZinc
\item Topics will be treated in more detail in later parts of the school
\item Based on Examples, not Formal Description
\end{itemize}
\end{frame}

\begin{frame}
\frametitle{Outline}
\begin{itemize}
\item Why Constraint Programming?
\item Constraint Propagation
\item Global Constraints
\item Customizing Search
%\item Choices in Modelling
%\item Decomposition
%\item More Global Constraints
%\item Methodology
%\item What is missing?
\end{itemize}
\end{frame}

% \begin{frame}
  % \frametitle{Using MiniZinc IDE}
  % \begin{itemize}
  % \item Developed in the Australian NICTA project
  % \item Maintained by Monash University
  % \item Modelling tool with multiple back-end solvers
  % \item Available from \url{https://www.minizinc.org/}
  % \end{itemize}
% \end{frame}


% \begin{frame}
% \frametitle{Examples in ECLiPSe}
% \begin{itemize}
% \item Open sourced constraint programming language
% \item Development goes back to 1985
% \item ECRC, ICL, IC-Parc, PTL, Cisco
% \item \url{https://eclipseclp.org/}
% \item Specialities
% \begin{itemize}
% \item Develop new solvers for specific domains
% \item Integration with MIP
% \end{itemize}
% \item Not included in bundled MiniZinc IDE
% \item Specialized visualization tools used here
  % \begin{itemize}
  % \item CP-Viz, Simonis et al. 2010
    % \end{itemize}
% \end{itemize}
% \end{frame}

\begin{frame}
\frametitle{Tutorial Based on ECLiPSe ELearning Course}
\begin{itemize}
\item Self-study course in constraint programming
\item Supported by Cisco Systems and Silicon Valley Community Foundation
\item Multi-media format, video lectures, slides, handout etc
\item {\small \url{https://eclipseclp.org/ELearning/index.html}}
\end{itemize}
\end{frame}

\begin{frame}
\frametitle{Also Part of CRT-AI Constraint Week}
\begin{itemize}
\item Annual one week course on CP and Optimization in Ireland
\item Part of national training program for PhD students in AI
\item \url{https://www.crt-ai.ie/}
\end{itemize}
\end{frame}



\begin{frame}
\frametitle{Constraint Programming - in a nutshell}
\begin{itemize}
\item Declarative description of problems with
\begin{itemize}
\item {\em Variables} which range over (finite) sets of values
\item {\em Constraints} over subsets of variables which restrict possible value combinations
\item A {\em solution} is a value assignment which satisfies all constraints
\end{itemize}

\item Constraint propagation/reasoning
\begin{itemize}
\item Removing inconsistent values for variables
\item Detect failure if constraint can not be satisfied
\item Interaction of constraints via shared variables
\item Incomplete
\end{itemize}

\item Search
\begin{itemize}
\item User controlled assignment of values to variables
\item Each step triggers constraint propagation 
\end{itemize}
\item Different domains require/allow different methods
\end{itemize}
\end{frame}

\begin{frame}
  \frametitle{Constraint Programming is Different}
  \begin{itemize}
  \item Declarative Programming
    \begin{itemize}
    \item Concentrate on what you want
      \item Not how to get there
      \item Program != Algorithm
      \item Program = Model
    \end{itemize}
    \item Applied to Combinatorial Problems
      \begin{itemize}
        \item No complete polynomial algorithms known (exist?)
        \item CP less ad-hoc than heuristics
        \item Models can evolve
  \end{itemize}
  \end{itemize}
  \end{frame}
    
\begin{frame}
  \frametitle{A Subtractive Process}
  \begin{textblock}{4}(8,-3)
    \includegraphics[width=4cm]{../introduction/images/stages}
  \end{textblock}
  \vfill
  \begin{quote}
    ``Oh, bosh, as Mr. Ruskin says. Sculpture, per se, is the simplest thing in the world. All you have to do is to take a big chunk of marble and a hammer and chisel, make up your mind what you are about to create and chip off all the marble you don't want.''-Paris Gaulois.
  \end{quote}
  
  {\tiny Source: \url{https://quoteinvestigator.com/2014/06/22/chip-away/}}
\end{frame}

\begin{frame}
\frametitle{Basic Process}
\begin{center}
\begin{tikzpicture}
\node[shape=rectangle,fill=pantone127-4] (problem) at (2,5) {Problem};
\node (human) at (2,4) {Human};
\node[shape=rectangle,fill=pantone127-4] (model) at (2,3) {Model};
\node (solver) at (2,2) {Constraint Solver/Search};
\node[shape=rectangle,fill=pantone157-8] (solution) at (2,1) {Solution};
\draw[-] (problem) -- (human);
\draw[->] (human) -- (model);
\draw[-] (model) -- (solver);
\draw[->] (solver) -- (solution);
\end{tikzpicture}
\end{center}
\end{frame}

\begin{frame}
\frametitle{More Realistic}
\begin{center}
\begin{tikzpicture}[xscale=4.0]
\node[shape=rectangle,fill=pantone127-4] (problem) at (2,5) {Problem};
\node (human) at (2,4) {Human};
\node[shape=rectangle,fill=pantone127-4] (model) at (2,3) {Model};
\node (solver) at (2,2) {Constraint Solver/Search};
\node[shape=rectangle,fill=pantone24-8] (hangs) at (1,1) {Hangs};
\node[shape=rectangle,fill=pantone157-8] (solution) at (2,1) {Solution};
\node[shape=rectangle,fill=pantone24-8] (wrong) at (3,1) {Wrong Solution};
\draw[-] (problem) -- (human);
\draw[->] (human) -- (model);
\draw[-] (model) -- (solver);
\draw[->] (solver) -- (hangs);
\draw[->] (solver) -- (solution);
\draw[->] (solver) -- (wrong);
\draw[->] (hangs) -- (1,4) -- (human);
\draw[->] (wrong) -- (3,4) -- (human);
\end{tikzpicture}
\end{center}
\end{frame}

\begin{frame}
\frametitle{Dual Role of Model}
\begin{itemize}
\item Allows Human to Express Problem
\begin{itemize}
\item Close to Problem Domain
\item Constraints as Abstractions
\end{itemize}
\item Allows Solver to Execute
\begin{itemize}
\item Variables as Communication Mechanism
\item Constraints as Algorithms
\end{itemize}
\end{itemize}
\end{frame}

\begin{frame}
\frametitle{Modelling Frameworks}
\begin{itemize}
\item \href{https://www.minizinc.org/}{MiniZinc} (NICTA, Monash University, Australia)
\item \href{https://github.com/eomahony/Numberjack}{NumberJack} (Insight, Ireland) 
\item \href{https://www-users.york.ac.uk/peter.nightingale/savilerow//}{EssencePrime/SavilleRow} (UK) 
\item \href{https://cpmpy.readthedocs.io/en/latest/}{CPMpy} (KU Leuven)
\item Allow use of multiple back-end solvers
\item Compile model into variants for each solver
\item A priori solver independent model(CP, MIP, SAT)
\end{itemize}
\end{frame} 

\begin{frame}
\frametitle{Framework Process}
\begin{center}
\begin{tikzpicture}[xscale=2]
\node[shape=rectangle,fill=pantone127-4] (problem) at (2.5,5) {Problem};
\node (human) at (2.5,4) {Human};
\node[shape=rectangle,fill=pantone127-4] (model) at (2.5,3) {Model};
\node (compile) at (2.5,2) {Compile/Reformulate};
\node (solver1) at (1,1) {CP};
\node (solver2) at (2,1) {MIP};
\node (solver3) at (3,1) {SAT};
\node (solver4) at (4,1) {Other};
\node[shape=rectangle,fill=pantone157-8] (solution1) at (1,0) {Solution};
\node[shape=rectangle,fill=pantone157-8] (solution2) at (2,0) {Solution};
\node[shape=rectangle,fill=pantone157-8] (solution3) at (3,0) {Solution};
\node[shape=rectangle,fill=pantone157-8] (solution4) at (4,0) {Solution};
\draw[-] (problem) -- (human);
\draw[->] (human) -- (model);
\draw[-] (model) -- (compile);
\draw[-] (compile) -- (solver1);
\draw[-] (compile) -- (solver2);
\draw[-] (compile) -- (solver3);
\draw[-] (compile) -- (solver4);
\draw[->] (solver1) -- (solution1);
\draw[->] (solver2) -- (solution2);
\draw[->] (solver3) -- (solution3);
\draw[->] (solver4) -- (solution4);
\end{tikzpicture}
\end{center}
\end{frame}

% \begin{frame}
  % \frametitle{Do It Now!}
  % \begin{itemize}
  % \item Download and install Minizinc
  % \item \url{https://www.minizinc.org/}
  % \end{itemize}
% \end{frame}


\begin{frame}
\frametitle{Why use Puzzles as Examples?}
\begin{itemize}
\item Easy to understand the problem
\item Solvable by hand without specialized knowledge
\item Possible to compare automated to manual solving process
\end{itemize}

\begin{quote}
The puzzle, though inanimate, is presented as a solvable problem without lasting negative consequences, a very low-risk low-reward situation. By being a puzzle, the object is attempting to convince the user that it must be completed.
\end{quote}
{\tiny Source: \href{https://www.everydayrhetoric.com/2020/03/the-rhetoric-of-puzzles/}{Every Day Rhetoric}}
\end{frame}


\mode<all>{
\part{Basic Constraint Propagation}
}
\begin{frame}
\partpage
\end{frame}
\mode<all>{
\input{../sendmore/body}
}

\mode<all>{
\part{Global Constraints}
}
\begin{frame}
\partpage
\end{frame}
\mode<all>{
\input{../sudoku/body}
}

\clearpage
\mode<all>{
\part{Customizing Search}
}
\begin{frame}
\partpage
\end{frame}
\mode<all>{
\input{../nqueen/body}
}


%% \clearpage
%% \mode<all>{
%%   \part{Problem Decomposition}
%% }
%% \begin{frame}
%%   \partpage
%% \end{frame}
%% \mode<all>{
%%   \input{../party/body}
%%   }

%% \clearpage
%% \mode<all>{
%%   \part{More Global Constraints}
%% }
%% \begin{frame}
%%   \partpage
%% \end{frame}
%% \mode<all>{
%%   \input{../car/body}
%%   }

%% \clearpage
%% \mode<all>{
%%   \part{Methodology}
%% }
%% \begin{frame}
%%   \partpage
%% \end{frame}
%% \mode<all>{
%%   \input{../methodology/body}
%%   }

\clearpage
\mode<all>{
\part{What is missing?}
}
\begin{frame}
\partpage
\end{frame}

\begin{frame}
\frametitle{Many Specialized Topics}
\begin{itemize}
\item How to design efficient core engine
\item Hybrids with LP/MIP tools
\item Hybrids with SAT
\item Symmetry breaking
\item Use of MDD/BDD to encode sets of solutions
\item High level modelling tools
\item Debugging/visualization
\end{itemize}
\end{frame}

\begin{frame}
\frametitle{Reformulation}
\begin{itemize}
\item Just because the user has modelled it this way, it doesn't mean we have to solve it that way
\begin{itemize}
\item Replace some constraint(s) by other, equivalent constraints
\item Because we don't have that constraint in our system
\item For performance
\end{itemize}
\end{itemize}
\end{frame}

\begin{frame}
\frametitle{Learning}
\begin{itemize}
\item While solving the problem we can learn how to strengthen the model/search 
\begin{itemize}
\item Understand which constraints/method contribute to propagation and change schedule
\item Learn no-good constraints by explaining failure
\item Adapt search strategy based on search experience
\end{itemize}
\end{itemize}
\end{frame}

%% \begin{frame}
%% \frametitle{Refined Process}
%% \begin{center}
%% \begin{tikzpicture}
%% \node[shape=rectangle,fill=pantone127-4] (problem) at (3,7) {Problem};
%% \node (human) at (3,6) {Human};
%% \node[shape=rectangle,fill=pantone127-4] (model) at (3,5) {User Model};
%% \node (reformulation) at (3,4) {Reformulation};
%% \node[shape=rectangle,fill=pantone127-4] (implementation) at (3,3) {Implementation Model};
%% \node (solver) at (3,2) {Constraint Solver/Search};
%% \node[shape=rectangle,fill=pantone157-8] (solution) at (3,1) {Solution};
%% \draw[-] (problem) -- (human);
%% \draw[->] (human) -- (model);
%% \draw[-] (model) -- (reformulation);
%% \draw[->] (reformulation) -- (implementation);
%% \draw[-] (implementation) -- (solver);
%% \draw[->] (solver) -- (0,2) -- (0,3) -- (implementation);
%% \draw[->] (solver) -- (solution);
%% \end{tikzpicture}
%% \end{center}
%% \end{frame}

%\begin{frame}
%\frametitle{What is CP actually used for?}
%\url{http://hsimonis.wordpress.com}
%\includegraphics[width=10cm]{../introduction/images/blog}
%\end{frame}

%% \begin{frame}
%% \frametitle{Exercises}
%% \begin{itemize}
%% \item Install Constraint Solver of your Choice
%% \item MiniZinc (MiningZinc) will be used in later sessions
%% \item Run Sendmoremoney, Sudoku, Queens examples
%% \item Can you choose which propagator to use for alldifferent?
%% \item Can you use different search strategies?
%% \end{itemize}
%% \end{frame}

\begin{frame}
\frametitle{More Learning Resources}
\begin{itemize}
\item Survey of Methods, Resources, and Formats for Teaching Constraint Programming
\begin{itemize}
\item by Tejas Santanam, Helmut Simonis
\item \url{https://doi.org/10.48550/arXiv.2403.12717}
\item Based on survey of community for WTCP 2023
\item \url{https://hsimonis.github.io/WTCP2023/}
\end{itemize}
\end{itemize}
\end{frame}


}


%% \clearpage
%% \mode<all>{
%%   \part{Problem Decomposition}
%% }
%% \begin{frame}
%%   \partpage
%% \end{frame}
%% \mode<all>{
%%   \begin{frame}
\frametitle{Objectives}
\begin{itemize}
\item Overview of Core Constraint Programming
\item Three Main Concepts
\begin{itemize}
\item Constraint Propagation
\item Global Constraints
\item Customizing Search
\end{itemize}
%\item Get Some Experience with MiniZinc
\item Topics will be treated in more detail in later parts of the school
\item Based on Examples, not Formal Description
\end{itemize}
\end{frame}

\begin{frame}
\frametitle{Outline}
\begin{itemize}
\item Why Constraint Programming?
\item Constraint Propagation
\item Global Constraints
\item Customizing Search
%\item Choices in Modelling
%\item Decomposition
%\item More Global Constraints
%\item Methodology
%\item What is missing?
\end{itemize}
\end{frame}

% \begin{frame}
  % \frametitle{Using MiniZinc IDE}
  % \begin{itemize}
  % \item Developed in the Australian NICTA project
  % \item Maintained by Monash University
  % \item Modelling tool with multiple back-end solvers
  % \item Available from \url{https://www.minizinc.org/}
  % \end{itemize}
% \end{frame}


% \begin{frame}
% \frametitle{Examples in ECLiPSe}
% \begin{itemize}
% \item Open sourced constraint programming language
% \item Development goes back to 1985
% \item ECRC, ICL, IC-Parc, PTL, Cisco
% \item \url{https://eclipseclp.org/}
% \item Specialities
% \begin{itemize}
% \item Develop new solvers for specific domains
% \item Integration with MIP
% \end{itemize}
% \item Not included in bundled MiniZinc IDE
% \item Specialized visualization tools used here
  % \begin{itemize}
  % \item CP-Viz, Simonis et al. 2010
    % \end{itemize}
% \end{itemize}
% \end{frame}

\begin{frame}
\frametitle{Tutorial Based on ECLiPSe ELearning Course}
\begin{itemize}
\item Self-study course in constraint programming
\item Supported by Cisco Systems and Silicon Valley Community Foundation
\item Multi-media format, video lectures, slides, handout etc
\item {\small \url{https://eclipseclp.org/ELearning/index.html}}
\end{itemize}
\end{frame}

\begin{frame}
\frametitle{Also Part of CRT-AI Constraint Week}
\begin{itemize}
\item Annual one week course on CP and Optimization in Ireland
\item Part of national training program for PhD students in AI
\item \url{https://www.crt-ai.ie/}
\end{itemize}
\end{frame}



\begin{frame}
\frametitle{Constraint Programming - in a nutshell}
\begin{itemize}
\item Declarative description of problems with
\begin{itemize}
\item {\em Variables} which range over (finite) sets of values
\item {\em Constraints} over subsets of variables which restrict possible value combinations
\item A {\em solution} is a value assignment which satisfies all constraints
\end{itemize}

\item Constraint propagation/reasoning
\begin{itemize}
\item Removing inconsistent values for variables
\item Detect failure if constraint can not be satisfied
\item Interaction of constraints via shared variables
\item Incomplete
\end{itemize}

\item Search
\begin{itemize}
\item User controlled assignment of values to variables
\item Each step triggers constraint propagation 
\end{itemize}
\item Different domains require/allow different methods
\end{itemize}
\end{frame}

\begin{frame}
  \frametitle{Constraint Programming is Different}
  \begin{itemize}
  \item Declarative Programming
    \begin{itemize}
    \item Concentrate on what you want
      \item Not how to get there
      \item Program != Algorithm
      \item Program = Model
    \end{itemize}
    \item Applied to Combinatorial Problems
      \begin{itemize}
        \item No complete polynomial algorithms known (exist?)
        \item CP less ad-hoc than heuristics
        \item Models can evolve
  \end{itemize}
  \end{itemize}
  \end{frame}
    
\begin{frame}
  \frametitle{A Subtractive Process}
  \begin{textblock}{4}(8,-3)
    \includegraphics[width=4cm]{../introduction/images/stages}
  \end{textblock}
  \vfill
  \begin{quote}
    ``Oh, bosh, as Mr. Ruskin says. Sculpture, per se, is the simplest thing in the world. All you have to do is to take a big chunk of marble and a hammer and chisel, make up your mind what you are about to create and chip off all the marble you don't want.''-Paris Gaulois.
  \end{quote}
  
  {\tiny Source: \url{https://quoteinvestigator.com/2014/06/22/chip-away/}}
\end{frame}

\begin{frame}
\frametitle{Basic Process}
\begin{center}
\begin{tikzpicture}
\node[shape=rectangle,fill=pantone127-4] (problem) at (2,5) {Problem};
\node (human) at (2,4) {Human};
\node[shape=rectangle,fill=pantone127-4] (model) at (2,3) {Model};
\node (solver) at (2,2) {Constraint Solver/Search};
\node[shape=rectangle,fill=pantone157-8] (solution) at (2,1) {Solution};
\draw[-] (problem) -- (human);
\draw[->] (human) -- (model);
\draw[-] (model) -- (solver);
\draw[->] (solver) -- (solution);
\end{tikzpicture}
\end{center}
\end{frame}

\begin{frame}
\frametitle{More Realistic}
\begin{center}
\begin{tikzpicture}[xscale=4.0]
\node[shape=rectangle,fill=pantone127-4] (problem) at (2,5) {Problem};
\node (human) at (2,4) {Human};
\node[shape=rectangle,fill=pantone127-4] (model) at (2,3) {Model};
\node (solver) at (2,2) {Constraint Solver/Search};
\node[shape=rectangle,fill=pantone24-8] (hangs) at (1,1) {Hangs};
\node[shape=rectangle,fill=pantone157-8] (solution) at (2,1) {Solution};
\node[shape=rectangle,fill=pantone24-8] (wrong) at (3,1) {Wrong Solution};
\draw[-] (problem) -- (human);
\draw[->] (human) -- (model);
\draw[-] (model) -- (solver);
\draw[->] (solver) -- (hangs);
\draw[->] (solver) -- (solution);
\draw[->] (solver) -- (wrong);
\draw[->] (hangs) -- (1,4) -- (human);
\draw[->] (wrong) -- (3,4) -- (human);
\end{tikzpicture}
\end{center}
\end{frame}

\begin{frame}
\frametitle{Dual Role of Model}
\begin{itemize}
\item Allows Human to Express Problem
\begin{itemize}
\item Close to Problem Domain
\item Constraints as Abstractions
\end{itemize}
\item Allows Solver to Execute
\begin{itemize}
\item Variables as Communication Mechanism
\item Constraints as Algorithms
\end{itemize}
\end{itemize}
\end{frame}

\begin{frame}
\frametitle{Modelling Frameworks}
\begin{itemize}
\item \href{https://www.minizinc.org/}{MiniZinc} (NICTA, Monash University, Australia)
\item \href{https://github.com/eomahony/Numberjack}{NumberJack} (Insight, Ireland) 
\item \href{https://www-users.york.ac.uk/peter.nightingale/savilerow//}{EssencePrime/SavilleRow} (UK) 
\item \href{https://cpmpy.readthedocs.io/en/latest/}{CPMpy} (KU Leuven)
\item Allow use of multiple back-end solvers
\item Compile model into variants for each solver
\item A priori solver independent model(CP, MIP, SAT)
\end{itemize}
\end{frame} 

\begin{frame}
\frametitle{Framework Process}
\begin{center}
\begin{tikzpicture}[xscale=2]
\node[shape=rectangle,fill=pantone127-4] (problem) at (2.5,5) {Problem};
\node (human) at (2.5,4) {Human};
\node[shape=rectangle,fill=pantone127-4] (model) at (2.5,3) {Model};
\node (compile) at (2.5,2) {Compile/Reformulate};
\node (solver1) at (1,1) {CP};
\node (solver2) at (2,1) {MIP};
\node (solver3) at (3,1) {SAT};
\node (solver4) at (4,1) {Other};
\node[shape=rectangle,fill=pantone157-8] (solution1) at (1,0) {Solution};
\node[shape=rectangle,fill=pantone157-8] (solution2) at (2,0) {Solution};
\node[shape=rectangle,fill=pantone157-8] (solution3) at (3,0) {Solution};
\node[shape=rectangle,fill=pantone157-8] (solution4) at (4,0) {Solution};
\draw[-] (problem) -- (human);
\draw[->] (human) -- (model);
\draw[-] (model) -- (compile);
\draw[-] (compile) -- (solver1);
\draw[-] (compile) -- (solver2);
\draw[-] (compile) -- (solver3);
\draw[-] (compile) -- (solver4);
\draw[->] (solver1) -- (solution1);
\draw[->] (solver2) -- (solution2);
\draw[->] (solver3) -- (solution3);
\draw[->] (solver4) -- (solution4);
\end{tikzpicture}
\end{center}
\end{frame}

% \begin{frame}
  % \frametitle{Do It Now!}
  % \begin{itemize}
  % \item Download and install Minizinc
  % \item \url{https://www.minizinc.org/}
  % \end{itemize}
% \end{frame}


\begin{frame}
\frametitle{Why use Puzzles as Examples?}
\begin{itemize}
\item Easy to understand the problem
\item Solvable by hand without specialized knowledge
\item Possible to compare automated to manual solving process
\end{itemize}

\begin{quote}
The puzzle, though inanimate, is presented as a solvable problem without lasting negative consequences, a very low-risk low-reward situation. By being a puzzle, the object is attempting to convince the user that it must be completed.
\end{quote}
{\tiny Source: \href{https://www.everydayrhetoric.com/2020/03/the-rhetoric-of-puzzles/}{Every Day Rhetoric}}
\end{frame}


\mode<all>{
\part{Basic Constraint Propagation}
}
\begin{frame}
\partpage
\end{frame}
\mode<all>{
\input{../sendmore/body}
}

\mode<all>{
\part{Global Constraints}
}
\begin{frame}
\partpage
\end{frame}
\mode<all>{
\input{../sudoku/body}
}

\clearpage
\mode<all>{
\part{Customizing Search}
}
\begin{frame}
\partpage
\end{frame}
\mode<all>{
\input{../nqueen/body}
}


%% \clearpage
%% \mode<all>{
%%   \part{Problem Decomposition}
%% }
%% \begin{frame}
%%   \partpage
%% \end{frame}
%% \mode<all>{
%%   \input{../party/body}
%%   }

%% \clearpage
%% \mode<all>{
%%   \part{More Global Constraints}
%% }
%% \begin{frame}
%%   \partpage
%% \end{frame}
%% \mode<all>{
%%   \input{../car/body}
%%   }

%% \clearpage
%% \mode<all>{
%%   \part{Methodology}
%% }
%% \begin{frame}
%%   \partpage
%% \end{frame}
%% \mode<all>{
%%   \input{../methodology/body}
%%   }

\clearpage
\mode<all>{
\part{What is missing?}
}
\begin{frame}
\partpage
\end{frame}

\begin{frame}
\frametitle{Many Specialized Topics}
\begin{itemize}
\item How to design efficient core engine
\item Hybrids with LP/MIP tools
\item Hybrids with SAT
\item Symmetry breaking
\item Use of MDD/BDD to encode sets of solutions
\item High level modelling tools
\item Debugging/visualization
\end{itemize}
\end{frame}

\begin{frame}
\frametitle{Reformulation}
\begin{itemize}
\item Just because the user has modelled it this way, it doesn't mean we have to solve it that way
\begin{itemize}
\item Replace some constraint(s) by other, equivalent constraints
\item Because we don't have that constraint in our system
\item For performance
\end{itemize}
\end{itemize}
\end{frame}

\begin{frame}
\frametitle{Learning}
\begin{itemize}
\item While solving the problem we can learn how to strengthen the model/search 
\begin{itemize}
\item Understand which constraints/method contribute to propagation and change schedule
\item Learn no-good constraints by explaining failure
\item Adapt search strategy based on search experience
\end{itemize}
\end{itemize}
\end{frame}

%% \begin{frame}
%% \frametitle{Refined Process}
%% \begin{center}
%% \begin{tikzpicture}
%% \node[shape=rectangle,fill=pantone127-4] (problem) at (3,7) {Problem};
%% \node (human) at (3,6) {Human};
%% \node[shape=rectangle,fill=pantone127-4] (model) at (3,5) {User Model};
%% \node (reformulation) at (3,4) {Reformulation};
%% \node[shape=rectangle,fill=pantone127-4] (implementation) at (3,3) {Implementation Model};
%% \node (solver) at (3,2) {Constraint Solver/Search};
%% \node[shape=rectangle,fill=pantone157-8] (solution) at (3,1) {Solution};
%% \draw[-] (problem) -- (human);
%% \draw[->] (human) -- (model);
%% \draw[-] (model) -- (reformulation);
%% \draw[->] (reformulation) -- (implementation);
%% \draw[-] (implementation) -- (solver);
%% \draw[->] (solver) -- (0,2) -- (0,3) -- (implementation);
%% \draw[->] (solver) -- (solution);
%% \end{tikzpicture}
%% \end{center}
%% \end{frame}

%\begin{frame}
%\frametitle{What is CP actually used for?}
%\url{http://hsimonis.wordpress.com}
%\includegraphics[width=10cm]{../introduction/images/blog}
%\end{frame}

%% \begin{frame}
%% \frametitle{Exercises}
%% \begin{itemize}
%% \item Install Constraint Solver of your Choice
%% \item MiniZinc (MiningZinc) will be used in later sessions
%% \item Run Sendmoremoney, Sudoku, Queens examples
%% \item Can you choose which propagator to use for alldifferent?
%% \item Can you use different search strategies?
%% \end{itemize}
%% \end{frame}

\begin{frame}
\frametitle{More Learning Resources}
\begin{itemize}
\item Survey of Methods, Resources, and Formats for Teaching Constraint Programming
\begin{itemize}
\item by Tejas Santanam, Helmut Simonis
\item \url{https://doi.org/10.48550/arXiv.2403.12717}
\item Based on survey of community for WTCP 2023
\item \url{https://hsimonis.github.io/WTCP2023/}
\end{itemize}
\end{itemize}
\end{frame}


%%   }

%% \clearpage
%% \mode<all>{
%%   \part{More Global Constraints}
%% }
%% \begin{frame}
%%   \partpage
%% \end{frame}
%% \mode<all>{
%%   \begin{frame}
\frametitle{Objectives}
\begin{itemize}
\item Overview of Core Constraint Programming
\item Three Main Concepts
\begin{itemize}
\item Constraint Propagation
\item Global Constraints
\item Customizing Search
\end{itemize}
%\item Get Some Experience with MiniZinc
\item Topics will be treated in more detail in later parts of the school
\item Based on Examples, not Formal Description
\end{itemize}
\end{frame}

\begin{frame}
\frametitle{Outline}
\begin{itemize}
\item Why Constraint Programming?
\item Constraint Propagation
\item Global Constraints
\item Customizing Search
%\item Choices in Modelling
%\item Decomposition
%\item More Global Constraints
%\item Methodology
%\item What is missing?
\end{itemize}
\end{frame}

% \begin{frame}
  % \frametitle{Using MiniZinc IDE}
  % \begin{itemize}
  % \item Developed in the Australian NICTA project
  % \item Maintained by Monash University
  % \item Modelling tool with multiple back-end solvers
  % \item Available from \url{https://www.minizinc.org/}
  % \end{itemize}
% \end{frame}


% \begin{frame}
% \frametitle{Examples in ECLiPSe}
% \begin{itemize}
% \item Open sourced constraint programming language
% \item Development goes back to 1985
% \item ECRC, ICL, IC-Parc, PTL, Cisco
% \item \url{https://eclipseclp.org/}
% \item Specialities
% \begin{itemize}
% \item Develop new solvers for specific domains
% \item Integration with MIP
% \end{itemize}
% \item Not included in bundled MiniZinc IDE
% \item Specialized visualization tools used here
  % \begin{itemize}
  % \item CP-Viz, Simonis et al. 2010
    % \end{itemize}
% \end{itemize}
% \end{frame}

\begin{frame}
\frametitle{Tutorial Based on ECLiPSe ELearning Course}
\begin{itemize}
\item Self-study course in constraint programming
\item Supported by Cisco Systems and Silicon Valley Community Foundation
\item Multi-media format, video lectures, slides, handout etc
\item {\small \url{https://eclipseclp.org/ELearning/index.html}}
\end{itemize}
\end{frame}

\begin{frame}
\frametitle{Also Part of CRT-AI Constraint Week}
\begin{itemize}
\item Annual one week course on CP and Optimization in Ireland
\item Part of national training program for PhD students in AI
\item \url{https://www.crt-ai.ie/}
\end{itemize}
\end{frame}



\begin{frame}
\frametitle{Constraint Programming - in a nutshell}
\begin{itemize}
\item Declarative description of problems with
\begin{itemize}
\item {\em Variables} which range over (finite) sets of values
\item {\em Constraints} over subsets of variables which restrict possible value combinations
\item A {\em solution} is a value assignment which satisfies all constraints
\end{itemize}

\item Constraint propagation/reasoning
\begin{itemize}
\item Removing inconsistent values for variables
\item Detect failure if constraint can not be satisfied
\item Interaction of constraints via shared variables
\item Incomplete
\end{itemize}

\item Search
\begin{itemize}
\item User controlled assignment of values to variables
\item Each step triggers constraint propagation 
\end{itemize}
\item Different domains require/allow different methods
\end{itemize}
\end{frame}

\begin{frame}
  \frametitle{Constraint Programming is Different}
  \begin{itemize}
  \item Declarative Programming
    \begin{itemize}
    \item Concentrate on what you want
      \item Not how to get there
      \item Program != Algorithm
      \item Program = Model
    \end{itemize}
    \item Applied to Combinatorial Problems
      \begin{itemize}
        \item No complete polynomial algorithms known (exist?)
        \item CP less ad-hoc than heuristics
        \item Models can evolve
  \end{itemize}
  \end{itemize}
  \end{frame}
    
\begin{frame}
  \frametitle{A Subtractive Process}
  \begin{textblock}{4}(8,-3)
    \includegraphics[width=4cm]{../introduction/images/stages}
  \end{textblock}
  \vfill
  \begin{quote}
    ``Oh, bosh, as Mr. Ruskin says. Sculpture, per se, is the simplest thing in the world. All you have to do is to take a big chunk of marble and a hammer and chisel, make up your mind what you are about to create and chip off all the marble you don't want.''-Paris Gaulois.
  \end{quote}
  
  {\tiny Source: \url{https://quoteinvestigator.com/2014/06/22/chip-away/}}
\end{frame}

\begin{frame}
\frametitle{Basic Process}
\begin{center}
\begin{tikzpicture}
\node[shape=rectangle,fill=pantone127-4] (problem) at (2,5) {Problem};
\node (human) at (2,4) {Human};
\node[shape=rectangle,fill=pantone127-4] (model) at (2,3) {Model};
\node (solver) at (2,2) {Constraint Solver/Search};
\node[shape=rectangle,fill=pantone157-8] (solution) at (2,1) {Solution};
\draw[-] (problem) -- (human);
\draw[->] (human) -- (model);
\draw[-] (model) -- (solver);
\draw[->] (solver) -- (solution);
\end{tikzpicture}
\end{center}
\end{frame}

\begin{frame}
\frametitle{More Realistic}
\begin{center}
\begin{tikzpicture}[xscale=4.0]
\node[shape=rectangle,fill=pantone127-4] (problem) at (2,5) {Problem};
\node (human) at (2,4) {Human};
\node[shape=rectangle,fill=pantone127-4] (model) at (2,3) {Model};
\node (solver) at (2,2) {Constraint Solver/Search};
\node[shape=rectangle,fill=pantone24-8] (hangs) at (1,1) {Hangs};
\node[shape=rectangle,fill=pantone157-8] (solution) at (2,1) {Solution};
\node[shape=rectangle,fill=pantone24-8] (wrong) at (3,1) {Wrong Solution};
\draw[-] (problem) -- (human);
\draw[->] (human) -- (model);
\draw[-] (model) -- (solver);
\draw[->] (solver) -- (hangs);
\draw[->] (solver) -- (solution);
\draw[->] (solver) -- (wrong);
\draw[->] (hangs) -- (1,4) -- (human);
\draw[->] (wrong) -- (3,4) -- (human);
\end{tikzpicture}
\end{center}
\end{frame}

\begin{frame}
\frametitle{Dual Role of Model}
\begin{itemize}
\item Allows Human to Express Problem
\begin{itemize}
\item Close to Problem Domain
\item Constraints as Abstractions
\end{itemize}
\item Allows Solver to Execute
\begin{itemize}
\item Variables as Communication Mechanism
\item Constraints as Algorithms
\end{itemize}
\end{itemize}
\end{frame}

\begin{frame}
\frametitle{Modelling Frameworks}
\begin{itemize}
\item \href{https://www.minizinc.org/}{MiniZinc} (NICTA, Monash University, Australia)
\item \href{https://github.com/eomahony/Numberjack}{NumberJack} (Insight, Ireland) 
\item \href{https://www-users.york.ac.uk/peter.nightingale/savilerow//}{EssencePrime/SavilleRow} (UK) 
\item \href{https://cpmpy.readthedocs.io/en/latest/}{CPMpy} (KU Leuven)
\item Allow use of multiple back-end solvers
\item Compile model into variants for each solver
\item A priori solver independent model(CP, MIP, SAT)
\end{itemize}
\end{frame} 

\begin{frame}
\frametitle{Framework Process}
\begin{center}
\begin{tikzpicture}[xscale=2]
\node[shape=rectangle,fill=pantone127-4] (problem) at (2.5,5) {Problem};
\node (human) at (2.5,4) {Human};
\node[shape=rectangle,fill=pantone127-4] (model) at (2.5,3) {Model};
\node (compile) at (2.5,2) {Compile/Reformulate};
\node (solver1) at (1,1) {CP};
\node (solver2) at (2,1) {MIP};
\node (solver3) at (3,1) {SAT};
\node (solver4) at (4,1) {Other};
\node[shape=rectangle,fill=pantone157-8] (solution1) at (1,0) {Solution};
\node[shape=rectangle,fill=pantone157-8] (solution2) at (2,0) {Solution};
\node[shape=rectangle,fill=pantone157-8] (solution3) at (3,0) {Solution};
\node[shape=rectangle,fill=pantone157-8] (solution4) at (4,0) {Solution};
\draw[-] (problem) -- (human);
\draw[->] (human) -- (model);
\draw[-] (model) -- (compile);
\draw[-] (compile) -- (solver1);
\draw[-] (compile) -- (solver2);
\draw[-] (compile) -- (solver3);
\draw[-] (compile) -- (solver4);
\draw[->] (solver1) -- (solution1);
\draw[->] (solver2) -- (solution2);
\draw[->] (solver3) -- (solution3);
\draw[->] (solver4) -- (solution4);
\end{tikzpicture}
\end{center}
\end{frame}

% \begin{frame}
  % \frametitle{Do It Now!}
  % \begin{itemize}
  % \item Download and install Minizinc
  % \item \url{https://www.minizinc.org/}
  % \end{itemize}
% \end{frame}


\begin{frame}
\frametitle{Why use Puzzles as Examples?}
\begin{itemize}
\item Easy to understand the problem
\item Solvable by hand without specialized knowledge
\item Possible to compare automated to manual solving process
\end{itemize}

\begin{quote}
The puzzle, though inanimate, is presented as a solvable problem without lasting negative consequences, a very low-risk low-reward situation. By being a puzzle, the object is attempting to convince the user that it must be completed.
\end{quote}
{\tiny Source: \href{https://www.everydayrhetoric.com/2020/03/the-rhetoric-of-puzzles/}{Every Day Rhetoric}}
\end{frame}


\mode<all>{
\part{Basic Constraint Propagation}
}
\begin{frame}
\partpage
\end{frame}
\mode<all>{
\input{../sendmore/body}
}

\mode<all>{
\part{Global Constraints}
}
\begin{frame}
\partpage
\end{frame}
\mode<all>{
\input{../sudoku/body}
}

\clearpage
\mode<all>{
\part{Customizing Search}
}
\begin{frame}
\partpage
\end{frame}
\mode<all>{
\input{../nqueen/body}
}


%% \clearpage
%% \mode<all>{
%%   \part{Problem Decomposition}
%% }
%% \begin{frame}
%%   \partpage
%% \end{frame}
%% \mode<all>{
%%   \input{../party/body}
%%   }

%% \clearpage
%% \mode<all>{
%%   \part{More Global Constraints}
%% }
%% \begin{frame}
%%   \partpage
%% \end{frame}
%% \mode<all>{
%%   \input{../car/body}
%%   }

%% \clearpage
%% \mode<all>{
%%   \part{Methodology}
%% }
%% \begin{frame}
%%   \partpage
%% \end{frame}
%% \mode<all>{
%%   \input{../methodology/body}
%%   }

\clearpage
\mode<all>{
\part{What is missing?}
}
\begin{frame}
\partpage
\end{frame}

\begin{frame}
\frametitle{Many Specialized Topics}
\begin{itemize}
\item How to design efficient core engine
\item Hybrids with LP/MIP tools
\item Hybrids with SAT
\item Symmetry breaking
\item Use of MDD/BDD to encode sets of solutions
\item High level modelling tools
\item Debugging/visualization
\end{itemize}
\end{frame}

\begin{frame}
\frametitle{Reformulation}
\begin{itemize}
\item Just because the user has modelled it this way, it doesn't mean we have to solve it that way
\begin{itemize}
\item Replace some constraint(s) by other, equivalent constraints
\item Because we don't have that constraint in our system
\item For performance
\end{itemize}
\end{itemize}
\end{frame}

\begin{frame}
\frametitle{Learning}
\begin{itemize}
\item While solving the problem we can learn how to strengthen the model/search 
\begin{itemize}
\item Understand which constraints/method contribute to propagation and change schedule
\item Learn no-good constraints by explaining failure
\item Adapt search strategy based on search experience
\end{itemize}
\end{itemize}
\end{frame}

%% \begin{frame}
%% \frametitle{Refined Process}
%% \begin{center}
%% \begin{tikzpicture}
%% \node[shape=rectangle,fill=pantone127-4] (problem) at (3,7) {Problem};
%% \node (human) at (3,6) {Human};
%% \node[shape=rectangle,fill=pantone127-4] (model) at (3,5) {User Model};
%% \node (reformulation) at (3,4) {Reformulation};
%% \node[shape=rectangle,fill=pantone127-4] (implementation) at (3,3) {Implementation Model};
%% \node (solver) at (3,2) {Constraint Solver/Search};
%% \node[shape=rectangle,fill=pantone157-8] (solution) at (3,1) {Solution};
%% \draw[-] (problem) -- (human);
%% \draw[->] (human) -- (model);
%% \draw[-] (model) -- (reformulation);
%% \draw[->] (reformulation) -- (implementation);
%% \draw[-] (implementation) -- (solver);
%% \draw[->] (solver) -- (0,2) -- (0,3) -- (implementation);
%% \draw[->] (solver) -- (solution);
%% \end{tikzpicture}
%% \end{center}
%% \end{frame}

%\begin{frame}
%\frametitle{What is CP actually used for?}
%\url{http://hsimonis.wordpress.com}
%\includegraphics[width=10cm]{../introduction/images/blog}
%\end{frame}

%% \begin{frame}
%% \frametitle{Exercises}
%% \begin{itemize}
%% \item Install Constraint Solver of your Choice
%% \item MiniZinc (MiningZinc) will be used in later sessions
%% \item Run Sendmoremoney, Sudoku, Queens examples
%% \item Can you choose which propagator to use for alldifferent?
%% \item Can you use different search strategies?
%% \end{itemize}
%% \end{frame}

\begin{frame}
\frametitle{More Learning Resources}
\begin{itemize}
\item Survey of Methods, Resources, and Formats for Teaching Constraint Programming
\begin{itemize}
\item by Tejas Santanam, Helmut Simonis
\item \url{https://doi.org/10.48550/arXiv.2403.12717}
\item Based on survey of community for WTCP 2023
\item \url{https://hsimonis.github.io/WTCP2023/}
\end{itemize}
\end{itemize}
\end{frame}


%%   }

%% \clearpage
%% \mode<all>{
%%   \part{Methodology}
%% }
%% \begin{frame}
%%   \partpage
%% \end{frame}
%% \mode<all>{
%%   \begin{frame}
\frametitle{Objectives}
\begin{itemize}
\item Overview of Core Constraint Programming
\item Three Main Concepts
\begin{itemize}
\item Constraint Propagation
\item Global Constraints
\item Customizing Search
\end{itemize}
%\item Get Some Experience with MiniZinc
\item Topics will be treated in more detail in later parts of the school
\item Based on Examples, not Formal Description
\end{itemize}
\end{frame}

\begin{frame}
\frametitle{Outline}
\begin{itemize}
\item Why Constraint Programming?
\item Constraint Propagation
\item Global Constraints
\item Customizing Search
%\item Choices in Modelling
%\item Decomposition
%\item More Global Constraints
%\item Methodology
%\item What is missing?
\end{itemize}
\end{frame}

% \begin{frame}
  % \frametitle{Using MiniZinc IDE}
  % \begin{itemize}
  % \item Developed in the Australian NICTA project
  % \item Maintained by Monash University
  % \item Modelling tool with multiple back-end solvers
  % \item Available from \url{https://www.minizinc.org/}
  % \end{itemize}
% \end{frame}


% \begin{frame}
% \frametitle{Examples in ECLiPSe}
% \begin{itemize}
% \item Open sourced constraint programming language
% \item Development goes back to 1985
% \item ECRC, ICL, IC-Parc, PTL, Cisco
% \item \url{https://eclipseclp.org/}
% \item Specialities
% \begin{itemize}
% \item Develop new solvers for specific domains
% \item Integration with MIP
% \end{itemize}
% \item Not included in bundled MiniZinc IDE
% \item Specialized visualization tools used here
  % \begin{itemize}
  % \item CP-Viz, Simonis et al. 2010
    % \end{itemize}
% \end{itemize}
% \end{frame}

\begin{frame}
\frametitle{Tutorial Based on ECLiPSe ELearning Course}
\begin{itemize}
\item Self-study course in constraint programming
\item Supported by Cisco Systems and Silicon Valley Community Foundation
\item Multi-media format, video lectures, slides, handout etc
\item {\small \url{https://eclipseclp.org/ELearning/index.html}}
\end{itemize}
\end{frame}

\begin{frame}
\frametitle{Also Part of CRT-AI Constraint Week}
\begin{itemize}
\item Annual one week course on CP and Optimization in Ireland
\item Part of national training program for PhD students in AI
\item \url{https://www.crt-ai.ie/}
\end{itemize}
\end{frame}



\begin{frame}
\frametitle{Constraint Programming - in a nutshell}
\begin{itemize}
\item Declarative description of problems with
\begin{itemize}
\item {\em Variables} which range over (finite) sets of values
\item {\em Constraints} over subsets of variables which restrict possible value combinations
\item A {\em solution} is a value assignment which satisfies all constraints
\end{itemize}

\item Constraint propagation/reasoning
\begin{itemize}
\item Removing inconsistent values for variables
\item Detect failure if constraint can not be satisfied
\item Interaction of constraints via shared variables
\item Incomplete
\end{itemize}

\item Search
\begin{itemize}
\item User controlled assignment of values to variables
\item Each step triggers constraint propagation 
\end{itemize}
\item Different domains require/allow different methods
\end{itemize}
\end{frame}

\begin{frame}
  \frametitle{Constraint Programming is Different}
  \begin{itemize}
  \item Declarative Programming
    \begin{itemize}
    \item Concentrate on what you want
      \item Not how to get there
      \item Program != Algorithm
      \item Program = Model
    \end{itemize}
    \item Applied to Combinatorial Problems
      \begin{itemize}
        \item No complete polynomial algorithms known (exist?)
        \item CP less ad-hoc than heuristics
        \item Models can evolve
  \end{itemize}
  \end{itemize}
  \end{frame}
    
\begin{frame}
  \frametitle{A Subtractive Process}
  \begin{textblock}{4}(8,-3)
    \includegraphics[width=4cm]{../introduction/images/stages}
  \end{textblock}
  \vfill
  \begin{quote}
    ``Oh, bosh, as Mr. Ruskin says. Sculpture, per se, is the simplest thing in the world. All you have to do is to take a big chunk of marble and a hammer and chisel, make up your mind what you are about to create and chip off all the marble you don't want.''-Paris Gaulois.
  \end{quote}
  
  {\tiny Source: \url{https://quoteinvestigator.com/2014/06/22/chip-away/}}
\end{frame}

\begin{frame}
\frametitle{Basic Process}
\begin{center}
\begin{tikzpicture}
\node[shape=rectangle,fill=pantone127-4] (problem) at (2,5) {Problem};
\node (human) at (2,4) {Human};
\node[shape=rectangle,fill=pantone127-4] (model) at (2,3) {Model};
\node (solver) at (2,2) {Constraint Solver/Search};
\node[shape=rectangle,fill=pantone157-8] (solution) at (2,1) {Solution};
\draw[-] (problem) -- (human);
\draw[->] (human) -- (model);
\draw[-] (model) -- (solver);
\draw[->] (solver) -- (solution);
\end{tikzpicture}
\end{center}
\end{frame}

\begin{frame}
\frametitle{More Realistic}
\begin{center}
\begin{tikzpicture}[xscale=4.0]
\node[shape=rectangle,fill=pantone127-4] (problem) at (2,5) {Problem};
\node (human) at (2,4) {Human};
\node[shape=rectangle,fill=pantone127-4] (model) at (2,3) {Model};
\node (solver) at (2,2) {Constraint Solver/Search};
\node[shape=rectangle,fill=pantone24-8] (hangs) at (1,1) {Hangs};
\node[shape=rectangle,fill=pantone157-8] (solution) at (2,1) {Solution};
\node[shape=rectangle,fill=pantone24-8] (wrong) at (3,1) {Wrong Solution};
\draw[-] (problem) -- (human);
\draw[->] (human) -- (model);
\draw[-] (model) -- (solver);
\draw[->] (solver) -- (hangs);
\draw[->] (solver) -- (solution);
\draw[->] (solver) -- (wrong);
\draw[->] (hangs) -- (1,4) -- (human);
\draw[->] (wrong) -- (3,4) -- (human);
\end{tikzpicture}
\end{center}
\end{frame}

\begin{frame}
\frametitle{Dual Role of Model}
\begin{itemize}
\item Allows Human to Express Problem
\begin{itemize}
\item Close to Problem Domain
\item Constraints as Abstractions
\end{itemize}
\item Allows Solver to Execute
\begin{itemize}
\item Variables as Communication Mechanism
\item Constraints as Algorithms
\end{itemize}
\end{itemize}
\end{frame}

\begin{frame}
\frametitle{Modelling Frameworks}
\begin{itemize}
\item \href{https://www.minizinc.org/}{MiniZinc} (NICTA, Monash University, Australia)
\item \href{https://github.com/eomahony/Numberjack}{NumberJack} (Insight, Ireland) 
\item \href{https://www-users.york.ac.uk/peter.nightingale/savilerow//}{EssencePrime/SavilleRow} (UK) 
\item \href{https://cpmpy.readthedocs.io/en/latest/}{CPMpy} (KU Leuven)
\item Allow use of multiple back-end solvers
\item Compile model into variants for each solver
\item A priori solver independent model(CP, MIP, SAT)
\end{itemize}
\end{frame} 

\begin{frame}
\frametitle{Framework Process}
\begin{center}
\begin{tikzpicture}[xscale=2]
\node[shape=rectangle,fill=pantone127-4] (problem) at (2.5,5) {Problem};
\node (human) at (2.5,4) {Human};
\node[shape=rectangle,fill=pantone127-4] (model) at (2.5,3) {Model};
\node (compile) at (2.5,2) {Compile/Reformulate};
\node (solver1) at (1,1) {CP};
\node (solver2) at (2,1) {MIP};
\node (solver3) at (3,1) {SAT};
\node (solver4) at (4,1) {Other};
\node[shape=rectangle,fill=pantone157-8] (solution1) at (1,0) {Solution};
\node[shape=rectangle,fill=pantone157-8] (solution2) at (2,0) {Solution};
\node[shape=rectangle,fill=pantone157-8] (solution3) at (3,0) {Solution};
\node[shape=rectangle,fill=pantone157-8] (solution4) at (4,0) {Solution};
\draw[-] (problem) -- (human);
\draw[->] (human) -- (model);
\draw[-] (model) -- (compile);
\draw[-] (compile) -- (solver1);
\draw[-] (compile) -- (solver2);
\draw[-] (compile) -- (solver3);
\draw[-] (compile) -- (solver4);
\draw[->] (solver1) -- (solution1);
\draw[->] (solver2) -- (solution2);
\draw[->] (solver3) -- (solution3);
\draw[->] (solver4) -- (solution4);
\end{tikzpicture}
\end{center}
\end{frame}

% \begin{frame}
  % \frametitle{Do It Now!}
  % \begin{itemize}
  % \item Download and install Minizinc
  % \item \url{https://www.minizinc.org/}
  % \end{itemize}
% \end{frame}


\begin{frame}
\frametitle{Why use Puzzles as Examples?}
\begin{itemize}
\item Easy to understand the problem
\item Solvable by hand without specialized knowledge
\item Possible to compare automated to manual solving process
\end{itemize}

\begin{quote}
The puzzle, though inanimate, is presented as a solvable problem without lasting negative consequences, a very low-risk low-reward situation. By being a puzzle, the object is attempting to convince the user that it must be completed.
\end{quote}
{\tiny Source: \href{https://www.everydayrhetoric.com/2020/03/the-rhetoric-of-puzzles/}{Every Day Rhetoric}}
\end{frame}


\mode<all>{
\part{Basic Constraint Propagation}
}
\begin{frame}
\partpage
\end{frame}
\mode<all>{
\input{../sendmore/body}
}

\mode<all>{
\part{Global Constraints}
}
\begin{frame}
\partpage
\end{frame}
\mode<all>{
\input{../sudoku/body}
}

\clearpage
\mode<all>{
\part{Customizing Search}
}
\begin{frame}
\partpage
\end{frame}
\mode<all>{
\input{../nqueen/body}
}


%% \clearpage
%% \mode<all>{
%%   \part{Problem Decomposition}
%% }
%% \begin{frame}
%%   \partpage
%% \end{frame}
%% \mode<all>{
%%   \input{../party/body}
%%   }

%% \clearpage
%% \mode<all>{
%%   \part{More Global Constraints}
%% }
%% \begin{frame}
%%   \partpage
%% \end{frame}
%% \mode<all>{
%%   \input{../car/body}
%%   }

%% \clearpage
%% \mode<all>{
%%   \part{Methodology}
%% }
%% \begin{frame}
%%   \partpage
%% \end{frame}
%% \mode<all>{
%%   \input{../methodology/body}
%%   }

\clearpage
\mode<all>{
\part{What is missing?}
}
\begin{frame}
\partpage
\end{frame}

\begin{frame}
\frametitle{Many Specialized Topics}
\begin{itemize}
\item How to design efficient core engine
\item Hybrids with LP/MIP tools
\item Hybrids with SAT
\item Symmetry breaking
\item Use of MDD/BDD to encode sets of solutions
\item High level modelling tools
\item Debugging/visualization
\end{itemize}
\end{frame}

\begin{frame}
\frametitle{Reformulation}
\begin{itemize}
\item Just because the user has modelled it this way, it doesn't mean we have to solve it that way
\begin{itemize}
\item Replace some constraint(s) by other, equivalent constraints
\item Because we don't have that constraint in our system
\item For performance
\end{itemize}
\end{itemize}
\end{frame}

\begin{frame}
\frametitle{Learning}
\begin{itemize}
\item While solving the problem we can learn how to strengthen the model/search 
\begin{itemize}
\item Understand which constraints/method contribute to propagation and change schedule
\item Learn no-good constraints by explaining failure
\item Adapt search strategy based on search experience
\end{itemize}
\end{itemize}
\end{frame}

%% \begin{frame}
%% \frametitle{Refined Process}
%% \begin{center}
%% \begin{tikzpicture}
%% \node[shape=rectangle,fill=pantone127-4] (problem) at (3,7) {Problem};
%% \node (human) at (3,6) {Human};
%% \node[shape=rectangle,fill=pantone127-4] (model) at (3,5) {User Model};
%% \node (reformulation) at (3,4) {Reformulation};
%% \node[shape=rectangle,fill=pantone127-4] (implementation) at (3,3) {Implementation Model};
%% \node (solver) at (3,2) {Constraint Solver/Search};
%% \node[shape=rectangle,fill=pantone157-8] (solution) at (3,1) {Solution};
%% \draw[-] (problem) -- (human);
%% \draw[->] (human) -- (model);
%% \draw[-] (model) -- (reformulation);
%% \draw[->] (reformulation) -- (implementation);
%% \draw[-] (implementation) -- (solver);
%% \draw[->] (solver) -- (0,2) -- (0,3) -- (implementation);
%% \draw[->] (solver) -- (solution);
%% \end{tikzpicture}
%% \end{center}
%% \end{frame}

%\begin{frame}
%\frametitle{What is CP actually used for?}
%\url{http://hsimonis.wordpress.com}
%\includegraphics[width=10cm]{../introduction/images/blog}
%\end{frame}

%% \begin{frame}
%% \frametitle{Exercises}
%% \begin{itemize}
%% \item Install Constraint Solver of your Choice
%% \item MiniZinc (MiningZinc) will be used in later sessions
%% \item Run Sendmoremoney, Sudoku, Queens examples
%% \item Can you choose which propagator to use for alldifferent?
%% \item Can you use different search strategies?
%% \end{itemize}
%% \end{frame}

\begin{frame}
\frametitle{More Learning Resources}
\begin{itemize}
\item Survey of Methods, Resources, and Formats for Teaching Constraint Programming
\begin{itemize}
\item by Tejas Santanam, Helmut Simonis
\item \url{https://doi.org/10.48550/arXiv.2403.12717}
\item Based on survey of community for WTCP 2023
\item \url{https://hsimonis.github.io/WTCP2023/}
\end{itemize}
\end{itemize}
\end{frame}


%%   }

\clearpage
\mode<all>{
\part{What is missing?}
}
\begin{frame}
\partpage
\end{frame}

\begin{frame}
\frametitle{Many Specialized Topics}
\begin{itemize}
\item How to design efficient core engine
\item Hybrids with LP/MIP tools
\item Hybrids with SAT
\item Symmetry breaking
\item Use of MDD/BDD to encode sets of solutions
\item High level modelling tools
\item Debugging/visualization
\end{itemize}
\end{frame}

\begin{frame}
\frametitle{Reformulation}
\begin{itemize}
\item Just because the user has modelled it this way, it doesn't mean we have to solve it that way
\begin{itemize}
\item Replace some constraint(s) by other, equivalent constraints
\item Because we don't have that constraint in our system
\item For performance
\end{itemize}
\end{itemize}
\end{frame}

\begin{frame}
\frametitle{Learning}
\begin{itemize}
\item While solving the problem we can learn how to strengthen the model/search 
\begin{itemize}
\item Understand which constraints/method contribute to propagation and change schedule
\item Learn no-good constraints by explaining failure
\item Adapt search strategy based on search experience
\end{itemize}
\end{itemize}
\end{frame}

%% \begin{frame}
%% \frametitle{Refined Process}
%% \begin{center}
%% \begin{tikzpicture}
%% \node[shape=rectangle,fill=pantone127-4] (problem) at (3,7) {Problem};
%% \node (human) at (3,6) {Human};
%% \node[shape=rectangle,fill=pantone127-4] (model) at (3,5) {User Model};
%% \node (reformulation) at (3,4) {Reformulation};
%% \node[shape=rectangle,fill=pantone127-4] (implementation) at (3,3) {Implementation Model};
%% \node (solver) at (3,2) {Constraint Solver/Search};
%% \node[shape=rectangle,fill=pantone157-8] (solution) at (3,1) {Solution};
%% \draw[-] (problem) -- (human);
%% \draw[->] (human) -- (model);
%% \draw[-] (model) -- (reformulation);
%% \draw[->] (reformulation) -- (implementation);
%% \draw[-] (implementation) -- (solver);
%% \draw[->] (solver) -- (0,2) -- (0,3) -- (implementation);
%% \draw[->] (solver) -- (solution);
%% \end{tikzpicture}
%% \end{center}
%% \end{frame}

%\begin{frame}
%\frametitle{What is CP actually used for?}
%\url{http://hsimonis.wordpress.com}
%\includegraphics[width=10cm]{../introduction/images/blog}
%\end{frame}

%% \begin{frame}
%% \frametitle{Exercises}
%% \begin{itemize}
%% \item Install Constraint Solver of your Choice
%% \item MiniZinc (MiningZinc) will be used in later sessions
%% \item Run Sendmoremoney, Sudoku, Queens examples
%% \item Can you choose which propagator to use for alldifferent?
%% \item Can you use different search strategies?
%% \end{itemize}
%% \end{frame}

\begin{frame}
\frametitle{More Learning Resources}
\begin{itemize}
\item Survey of Methods, Resources, and Formats for Teaching Constraint Programming
\begin{itemize}
\item by Tejas Santanam, Helmut Simonis
\item \url{https://doi.org/10.48550/arXiv.2403.12717}
\item Based on survey of community for WTCP 2023
\item \url{https://hsimonis.github.io/WTCP2023/}
\end{itemize}
\end{itemize}
\end{frame}


}

\clearpage
\mode<all>{
\part{Customizing Search}
}
\begin{frame}
\partpage
\end{frame}
\mode<all>{
\begin{frame}
\frametitle{Objectives}
\begin{itemize}
\item Overview of Core Constraint Programming
\item Three Main Concepts
\begin{itemize}
\item Constraint Propagation
\item Global Constraints
\item Customizing Search
\end{itemize}
%\item Get Some Experience with MiniZinc
\item Topics will be treated in more detail in later parts of the school
\item Based on Examples, not Formal Description
\end{itemize}
\end{frame}

\begin{frame}
\frametitle{Outline}
\begin{itemize}
\item Why Constraint Programming?
\item Constraint Propagation
\item Global Constraints
\item Customizing Search
%\item Choices in Modelling
%\item Decomposition
%\item More Global Constraints
%\item Methodology
%\item What is missing?
\end{itemize}
\end{frame}

% \begin{frame}
  % \frametitle{Using MiniZinc IDE}
  % \begin{itemize}
  % \item Developed in the Australian NICTA project
  % \item Maintained by Monash University
  % \item Modelling tool with multiple back-end solvers
  % \item Available from \url{https://www.minizinc.org/}
  % \end{itemize}
% \end{frame}


% \begin{frame}
% \frametitle{Examples in ECLiPSe}
% \begin{itemize}
% \item Open sourced constraint programming language
% \item Development goes back to 1985
% \item ECRC, ICL, IC-Parc, PTL, Cisco
% \item \url{https://eclipseclp.org/}
% \item Specialities
% \begin{itemize}
% \item Develop new solvers for specific domains
% \item Integration with MIP
% \end{itemize}
% \item Not included in bundled MiniZinc IDE
% \item Specialized visualization tools used here
  % \begin{itemize}
  % \item CP-Viz, Simonis et al. 2010
    % \end{itemize}
% \end{itemize}
% \end{frame}

\begin{frame}
\frametitle{Tutorial Based on ECLiPSe ELearning Course}
\begin{itemize}
\item Self-study course in constraint programming
\item Supported by Cisco Systems and Silicon Valley Community Foundation
\item Multi-media format, video lectures, slides, handout etc
\item {\small \url{https://eclipseclp.org/ELearning/index.html}}
\end{itemize}
\end{frame}

\begin{frame}
\frametitle{Also Part of CRT-AI Constraint Week}
\begin{itemize}
\item Annual one week course on CP and Optimization in Ireland
\item Part of national training program for PhD students in AI
\item \url{https://www.crt-ai.ie/}
\end{itemize}
\end{frame}



\begin{frame}
\frametitle{Constraint Programming - in a nutshell}
\begin{itemize}
\item Declarative description of problems with
\begin{itemize}
\item {\em Variables} which range over (finite) sets of values
\item {\em Constraints} over subsets of variables which restrict possible value combinations
\item A {\em solution} is a value assignment which satisfies all constraints
\end{itemize}

\item Constraint propagation/reasoning
\begin{itemize}
\item Removing inconsistent values for variables
\item Detect failure if constraint can not be satisfied
\item Interaction of constraints via shared variables
\item Incomplete
\end{itemize}

\item Search
\begin{itemize}
\item User controlled assignment of values to variables
\item Each step triggers constraint propagation 
\end{itemize}
\item Different domains require/allow different methods
\end{itemize}
\end{frame}

\begin{frame}
  \frametitle{Constraint Programming is Different}
  \begin{itemize}
  \item Declarative Programming
    \begin{itemize}
    \item Concentrate on what you want
      \item Not how to get there
      \item Program != Algorithm
      \item Program = Model
    \end{itemize}
    \item Applied to Combinatorial Problems
      \begin{itemize}
        \item No complete polynomial algorithms known (exist?)
        \item CP less ad-hoc than heuristics
        \item Models can evolve
  \end{itemize}
  \end{itemize}
  \end{frame}
    
\begin{frame}
  \frametitle{A Subtractive Process}
  \begin{textblock}{4}(8,-3)
    \includegraphics[width=4cm]{../introduction/images/stages}
  \end{textblock}
  \vfill
  \begin{quote}
    ``Oh, bosh, as Mr. Ruskin says. Sculpture, per se, is the simplest thing in the world. All you have to do is to take a big chunk of marble and a hammer and chisel, make up your mind what you are about to create and chip off all the marble you don't want.''-Paris Gaulois.
  \end{quote}
  
  {\tiny Source: \url{https://quoteinvestigator.com/2014/06/22/chip-away/}}
\end{frame}

\begin{frame}
\frametitle{Basic Process}
\begin{center}
\begin{tikzpicture}
\node[shape=rectangle,fill=pantone127-4] (problem) at (2,5) {Problem};
\node (human) at (2,4) {Human};
\node[shape=rectangle,fill=pantone127-4] (model) at (2,3) {Model};
\node (solver) at (2,2) {Constraint Solver/Search};
\node[shape=rectangle,fill=pantone157-8] (solution) at (2,1) {Solution};
\draw[-] (problem) -- (human);
\draw[->] (human) -- (model);
\draw[-] (model) -- (solver);
\draw[->] (solver) -- (solution);
\end{tikzpicture}
\end{center}
\end{frame}

\begin{frame}
\frametitle{More Realistic}
\begin{center}
\begin{tikzpicture}[xscale=4.0]
\node[shape=rectangle,fill=pantone127-4] (problem) at (2,5) {Problem};
\node (human) at (2,4) {Human};
\node[shape=rectangle,fill=pantone127-4] (model) at (2,3) {Model};
\node (solver) at (2,2) {Constraint Solver/Search};
\node[shape=rectangle,fill=pantone24-8] (hangs) at (1,1) {Hangs};
\node[shape=rectangle,fill=pantone157-8] (solution) at (2,1) {Solution};
\node[shape=rectangle,fill=pantone24-8] (wrong) at (3,1) {Wrong Solution};
\draw[-] (problem) -- (human);
\draw[->] (human) -- (model);
\draw[-] (model) -- (solver);
\draw[->] (solver) -- (hangs);
\draw[->] (solver) -- (solution);
\draw[->] (solver) -- (wrong);
\draw[->] (hangs) -- (1,4) -- (human);
\draw[->] (wrong) -- (3,4) -- (human);
\end{tikzpicture}
\end{center}
\end{frame}

\begin{frame}
\frametitle{Dual Role of Model}
\begin{itemize}
\item Allows Human to Express Problem
\begin{itemize}
\item Close to Problem Domain
\item Constraints as Abstractions
\end{itemize}
\item Allows Solver to Execute
\begin{itemize}
\item Variables as Communication Mechanism
\item Constraints as Algorithms
\end{itemize}
\end{itemize}
\end{frame}

\begin{frame}
\frametitle{Modelling Frameworks}
\begin{itemize}
\item \href{https://www.minizinc.org/}{MiniZinc} (NICTA, Monash University, Australia)
\item \href{https://github.com/eomahony/Numberjack}{NumberJack} (Insight, Ireland) 
\item \href{https://www-users.york.ac.uk/peter.nightingale/savilerow//}{EssencePrime/SavilleRow} (UK) 
\item \href{https://cpmpy.readthedocs.io/en/latest/}{CPMpy} (KU Leuven)
\item Allow use of multiple back-end solvers
\item Compile model into variants for each solver
\item A priori solver independent model(CP, MIP, SAT)
\end{itemize}
\end{frame} 

\begin{frame}
\frametitle{Framework Process}
\begin{center}
\begin{tikzpicture}[xscale=2]
\node[shape=rectangle,fill=pantone127-4] (problem) at (2.5,5) {Problem};
\node (human) at (2.5,4) {Human};
\node[shape=rectangle,fill=pantone127-4] (model) at (2.5,3) {Model};
\node (compile) at (2.5,2) {Compile/Reformulate};
\node (solver1) at (1,1) {CP};
\node (solver2) at (2,1) {MIP};
\node (solver3) at (3,1) {SAT};
\node (solver4) at (4,1) {Other};
\node[shape=rectangle,fill=pantone157-8] (solution1) at (1,0) {Solution};
\node[shape=rectangle,fill=pantone157-8] (solution2) at (2,0) {Solution};
\node[shape=rectangle,fill=pantone157-8] (solution3) at (3,0) {Solution};
\node[shape=rectangle,fill=pantone157-8] (solution4) at (4,0) {Solution};
\draw[-] (problem) -- (human);
\draw[->] (human) -- (model);
\draw[-] (model) -- (compile);
\draw[-] (compile) -- (solver1);
\draw[-] (compile) -- (solver2);
\draw[-] (compile) -- (solver3);
\draw[-] (compile) -- (solver4);
\draw[->] (solver1) -- (solution1);
\draw[->] (solver2) -- (solution2);
\draw[->] (solver3) -- (solution3);
\draw[->] (solver4) -- (solution4);
\end{tikzpicture}
\end{center}
\end{frame}

% \begin{frame}
  % \frametitle{Do It Now!}
  % \begin{itemize}
  % \item Download and install Minizinc
  % \item \url{https://www.minizinc.org/}
  % \end{itemize}
% \end{frame}


\begin{frame}
\frametitle{Why use Puzzles as Examples?}
\begin{itemize}
\item Easy to understand the problem
\item Solvable by hand without specialized knowledge
\item Possible to compare automated to manual solving process
\end{itemize}

\begin{quote}
The puzzle, though inanimate, is presented as a solvable problem without lasting negative consequences, a very low-risk low-reward situation. By being a puzzle, the object is attempting to convince the user that it must be completed.
\end{quote}
{\tiny Source: \href{https://www.everydayrhetoric.com/2020/03/the-rhetoric-of-puzzles/}{Every Day Rhetoric}}
\end{frame}


\mode<all>{
\part{Basic Constraint Propagation}
}
\begin{frame}
\partpage
\end{frame}
\mode<all>{
\begin{frame}
\frametitle{Objectives}
\begin{itemize}
\item Overview of Core Constraint Programming
\item Three Main Concepts
\begin{itemize}
\item Constraint Propagation
\item Global Constraints
\item Customizing Search
\end{itemize}
%\item Get Some Experience with MiniZinc
\item Topics will be treated in more detail in later parts of the school
\item Based on Examples, not Formal Description
\end{itemize}
\end{frame}

\begin{frame}
\frametitle{Outline}
\begin{itemize}
\item Why Constraint Programming?
\item Constraint Propagation
\item Global Constraints
\item Customizing Search
%\item Choices in Modelling
%\item Decomposition
%\item More Global Constraints
%\item Methodology
%\item What is missing?
\end{itemize}
\end{frame}

% \begin{frame}
  % \frametitle{Using MiniZinc IDE}
  % \begin{itemize}
  % \item Developed in the Australian NICTA project
  % \item Maintained by Monash University
  % \item Modelling tool with multiple back-end solvers
  % \item Available from \url{https://www.minizinc.org/}
  % \end{itemize}
% \end{frame}


% \begin{frame}
% \frametitle{Examples in ECLiPSe}
% \begin{itemize}
% \item Open sourced constraint programming language
% \item Development goes back to 1985
% \item ECRC, ICL, IC-Parc, PTL, Cisco
% \item \url{https://eclipseclp.org/}
% \item Specialities
% \begin{itemize}
% \item Develop new solvers for specific domains
% \item Integration with MIP
% \end{itemize}
% \item Not included in bundled MiniZinc IDE
% \item Specialized visualization tools used here
  % \begin{itemize}
  % \item CP-Viz, Simonis et al. 2010
    % \end{itemize}
% \end{itemize}
% \end{frame}

\begin{frame}
\frametitle{Tutorial Based on ECLiPSe ELearning Course}
\begin{itemize}
\item Self-study course in constraint programming
\item Supported by Cisco Systems and Silicon Valley Community Foundation
\item Multi-media format, video lectures, slides, handout etc
\item {\small \url{https://eclipseclp.org/ELearning/index.html}}
\end{itemize}
\end{frame}

\begin{frame}
\frametitle{Also Part of CRT-AI Constraint Week}
\begin{itemize}
\item Annual one week course on CP and Optimization in Ireland
\item Part of national training program for PhD students in AI
\item \url{https://www.crt-ai.ie/}
\end{itemize}
\end{frame}



\begin{frame}
\frametitle{Constraint Programming - in a nutshell}
\begin{itemize}
\item Declarative description of problems with
\begin{itemize}
\item {\em Variables} which range over (finite) sets of values
\item {\em Constraints} over subsets of variables which restrict possible value combinations
\item A {\em solution} is a value assignment which satisfies all constraints
\end{itemize}

\item Constraint propagation/reasoning
\begin{itemize}
\item Removing inconsistent values for variables
\item Detect failure if constraint can not be satisfied
\item Interaction of constraints via shared variables
\item Incomplete
\end{itemize}

\item Search
\begin{itemize}
\item User controlled assignment of values to variables
\item Each step triggers constraint propagation 
\end{itemize}
\item Different domains require/allow different methods
\end{itemize}
\end{frame}

\begin{frame}
  \frametitle{Constraint Programming is Different}
  \begin{itemize}
  \item Declarative Programming
    \begin{itemize}
    \item Concentrate on what you want
      \item Not how to get there
      \item Program != Algorithm
      \item Program = Model
    \end{itemize}
    \item Applied to Combinatorial Problems
      \begin{itemize}
        \item No complete polynomial algorithms known (exist?)
        \item CP less ad-hoc than heuristics
        \item Models can evolve
  \end{itemize}
  \end{itemize}
  \end{frame}
    
\begin{frame}
  \frametitle{A Subtractive Process}
  \begin{textblock}{4}(8,-3)
    \includegraphics[width=4cm]{../introduction/images/stages}
  \end{textblock}
  \vfill
  \begin{quote}
    ``Oh, bosh, as Mr. Ruskin says. Sculpture, per se, is the simplest thing in the world. All you have to do is to take a big chunk of marble and a hammer and chisel, make up your mind what you are about to create and chip off all the marble you don't want.''-Paris Gaulois.
  \end{quote}
  
  {\tiny Source: \url{https://quoteinvestigator.com/2014/06/22/chip-away/}}
\end{frame}

\begin{frame}
\frametitle{Basic Process}
\begin{center}
\begin{tikzpicture}
\node[shape=rectangle,fill=pantone127-4] (problem) at (2,5) {Problem};
\node (human) at (2,4) {Human};
\node[shape=rectangle,fill=pantone127-4] (model) at (2,3) {Model};
\node (solver) at (2,2) {Constraint Solver/Search};
\node[shape=rectangle,fill=pantone157-8] (solution) at (2,1) {Solution};
\draw[-] (problem) -- (human);
\draw[->] (human) -- (model);
\draw[-] (model) -- (solver);
\draw[->] (solver) -- (solution);
\end{tikzpicture}
\end{center}
\end{frame}

\begin{frame}
\frametitle{More Realistic}
\begin{center}
\begin{tikzpicture}[xscale=4.0]
\node[shape=rectangle,fill=pantone127-4] (problem) at (2,5) {Problem};
\node (human) at (2,4) {Human};
\node[shape=rectangle,fill=pantone127-4] (model) at (2,3) {Model};
\node (solver) at (2,2) {Constraint Solver/Search};
\node[shape=rectangle,fill=pantone24-8] (hangs) at (1,1) {Hangs};
\node[shape=rectangle,fill=pantone157-8] (solution) at (2,1) {Solution};
\node[shape=rectangle,fill=pantone24-8] (wrong) at (3,1) {Wrong Solution};
\draw[-] (problem) -- (human);
\draw[->] (human) -- (model);
\draw[-] (model) -- (solver);
\draw[->] (solver) -- (hangs);
\draw[->] (solver) -- (solution);
\draw[->] (solver) -- (wrong);
\draw[->] (hangs) -- (1,4) -- (human);
\draw[->] (wrong) -- (3,4) -- (human);
\end{tikzpicture}
\end{center}
\end{frame}

\begin{frame}
\frametitle{Dual Role of Model}
\begin{itemize}
\item Allows Human to Express Problem
\begin{itemize}
\item Close to Problem Domain
\item Constraints as Abstractions
\end{itemize}
\item Allows Solver to Execute
\begin{itemize}
\item Variables as Communication Mechanism
\item Constraints as Algorithms
\end{itemize}
\end{itemize}
\end{frame}

\begin{frame}
\frametitle{Modelling Frameworks}
\begin{itemize}
\item \href{https://www.minizinc.org/}{MiniZinc} (NICTA, Monash University, Australia)
\item \href{https://github.com/eomahony/Numberjack}{NumberJack} (Insight, Ireland) 
\item \href{https://www-users.york.ac.uk/peter.nightingale/savilerow//}{EssencePrime/SavilleRow} (UK) 
\item \href{https://cpmpy.readthedocs.io/en/latest/}{CPMpy} (KU Leuven)
\item Allow use of multiple back-end solvers
\item Compile model into variants for each solver
\item A priori solver independent model(CP, MIP, SAT)
\end{itemize}
\end{frame} 

\begin{frame}
\frametitle{Framework Process}
\begin{center}
\begin{tikzpicture}[xscale=2]
\node[shape=rectangle,fill=pantone127-4] (problem) at (2.5,5) {Problem};
\node (human) at (2.5,4) {Human};
\node[shape=rectangle,fill=pantone127-4] (model) at (2.5,3) {Model};
\node (compile) at (2.5,2) {Compile/Reformulate};
\node (solver1) at (1,1) {CP};
\node (solver2) at (2,1) {MIP};
\node (solver3) at (3,1) {SAT};
\node (solver4) at (4,1) {Other};
\node[shape=rectangle,fill=pantone157-8] (solution1) at (1,0) {Solution};
\node[shape=rectangle,fill=pantone157-8] (solution2) at (2,0) {Solution};
\node[shape=rectangle,fill=pantone157-8] (solution3) at (3,0) {Solution};
\node[shape=rectangle,fill=pantone157-8] (solution4) at (4,0) {Solution};
\draw[-] (problem) -- (human);
\draw[->] (human) -- (model);
\draw[-] (model) -- (compile);
\draw[-] (compile) -- (solver1);
\draw[-] (compile) -- (solver2);
\draw[-] (compile) -- (solver3);
\draw[-] (compile) -- (solver4);
\draw[->] (solver1) -- (solution1);
\draw[->] (solver2) -- (solution2);
\draw[->] (solver3) -- (solution3);
\draw[->] (solver4) -- (solution4);
\end{tikzpicture}
\end{center}
\end{frame}

% \begin{frame}
  % \frametitle{Do It Now!}
  % \begin{itemize}
  % \item Download and install Minizinc
  % \item \url{https://www.minizinc.org/}
  % \end{itemize}
% \end{frame}


\begin{frame}
\frametitle{Why use Puzzles as Examples?}
\begin{itemize}
\item Easy to understand the problem
\item Solvable by hand without specialized knowledge
\item Possible to compare automated to manual solving process
\end{itemize}

\begin{quote}
The puzzle, though inanimate, is presented as a solvable problem without lasting negative consequences, a very low-risk low-reward situation. By being a puzzle, the object is attempting to convince the user that it must be completed.
\end{quote}
{\tiny Source: \href{https://www.everydayrhetoric.com/2020/03/the-rhetoric-of-puzzles/}{Every Day Rhetoric}}
\end{frame}


\mode<all>{
\part{Basic Constraint Propagation}
}
\begin{frame}
\partpage
\end{frame}
\mode<all>{
\input{../sendmore/body}
}

\mode<all>{
\part{Global Constraints}
}
\begin{frame}
\partpage
\end{frame}
\mode<all>{
\input{../sudoku/body}
}

\clearpage
\mode<all>{
\part{Customizing Search}
}
\begin{frame}
\partpage
\end{frame}
\mode<all>{
\input{../nqueen/body}
}


%% \clearpage
%% \mode<all>{
%%   \part{Problem Decomposition}
%% }
%% \begin{frame}
%%   \partpage
%% \end{frame}
%% \mode<all>{
%%   \input{../party/body}
%%   }

%% \clearpage
%% \mode<all>{
%%   \part{More Global Constraints}
%% }
%% \begin{frame}
%%   \partpage
%% \end{frame}
%% \mode<all>{
%%   \input{../car/body}
%%   }

%% \clearpage
%% \mode<all>{
%%   \part{Methodology}
%% }
%% \begin{frame}
%%   \partpage
%% \end{frame}
%% \mode<all>{
%%   \input{../methodology/body}
%%   }

\clearpage
\mode<all>{
\part{What is missing?}
}
\begin{frame}
\partpage
\end{frame}

\begin{frame}
\frametitle{Many Specialized Topics}
\begin{itemize}
\item How to design efficient core engine
\item Hybrids with LP/MIP tools
\item Hybrids with SAT
\item Symmetry breaking
\item Use of MDD/BDD to encode sets of solutions
\item High level modelling tools
\item Debugging/visualization
\end{itemize}
\end{frame}

\begin{frame}
\frametitle{Reformulation}
\begin{itemize}
\item Just because the user has modelled it this way, it doesn't mean we have to solve it that way
\begin{itemize}
\item Replace some constraint(s) by other, equivalent constraints
\item Because we don't have that constraint in our system
\item For performance
\end{itemize}
\end{itemize}
\end{frame}

\begin{frame}
\frametitle{Learning}
\begin{itemize}
\item While solving the problem we can learn how to strengthen the model/search 
\begin{itemize}
\item Understand which constraints/method contribute to propagation and change schedule
\item Learn no-good constraints by explaining failure
\item Adapt search strategy based on search experience
\end{itemize}
\end{itemize}
\end{frame}

%% \begin{frame}
%% \frametitle{Refined Process}
%% \begin{center}
%% \begin{tikzpicture}
%% \node[shape=rectangle,fill=pantone127-4] (problem) at (3,7) {Problem};
%% \node (human) at (3,6) {Human};
%% \node[shape=rectangle,fill=pantone127-4] (model) at (3,5) {User Model};
%% \node (reformulation) at (3,4) {Reformulation};
%% \node[shape=rectangle,fill=pantone127-4] (implementation) at (3,3) {Implementation Model};
%% \node (solver) at (3,2) {Constraint Solver/Search};
%% \node[shape=rectangle,fill=pantone157-8] (solution) at (3,1) {Solution};
%% \draw[-] (problem) -- (human);
%% \draw[->] (human) -- (model);
%% \draw[-] (model) -- (reformulation);
%% \draw[->] (reformulation) -- (implementation);
%% \draw[-] (implementation) -- (solver);
%% \draw[->] (solver) -- (0,2) -- (0,3) -- (implementation);
%% \draw[->] (solver) -- (solution);
%% \end{tikzpicture}
%% \end{center}
%% \end{frame}

%\begin{frame}
%\frametitle{What is CP actually used for?}
%\url{http://hsimonis.wordpress.com}
%\includegraphics[width=10cm]{../introduction/images/blog}
%\end{frame}

%% \begin{frame}
%% \frametitle{Exercises}
%% \begin{itemize}
%% \item Install Constraint Solver of your Choice
%% \item MiniZinc (MiningZinc) will be used in later sessions
%% \item Run Sendmoremoney, Sudoku, Queens examples
%% \item Can you choose which propagator to use for alldifferent?
%% \item Can you use different search strategies?
%% \end{itemize}
%% \end{frame}

\begin{frame}
\frametitle{More Learning Resources}
\begin{itemize}
\item Survey of Methods, Resources, and Formats for Teaching Constraint Programming
\begin{itemize}
\item by Tejas Santanam, Helmut Simonis
\item \url{https://doi.org/10.48550/arXiv.2403.12717}
\item Based on survey of community for WTCP 2023
\item \url{https://hsimonis.github.io/WTCP2023/}
\end{itemize}
\end{itemize}
\end{frame}


}

\mode<all>{
\part{Global Constraints}
}
\begin{frame}
\partpage
\end{frame}
\mode<all>{
\begin{frame}
\frametitle{Objectives}
\begin{itemize}
\item Overview of Core Constraint Programming
\item Three Main Concepts
\begin{itemize}
\item Constraint Propagation
\item Global Constraints
\item Customizing Search
\end{itemize}
%\item Get Some Experience with MiniZinc
\item Topics will be treated in more detail in later parts of the school
\item Based on Examples, not Formal Description
\end{itemize}
\end{frame}

\begin{frame}
\frametitle{Outline}
\begin{itemize}
\item Why Constraint Programming?
\item Constraint Propagation
\item Global Constraints
\item Customizing Search
%\item Choices in Modelling
%\item Decomposition
%\item More Global Constraints
%\item Methodology
%\item What is missing?
\end{itemize}
\end{frame}

% \begin{frame}
  % \frametitle{Using MiniZinc IDE}
  % \begin{itemize}
  % \item Developed in the Australian NICTA project
  % \item Maintained by Monash University
  % \item Modelling tool with multiple back-end solvers
  % \item Available from \url{https://www.minizinc.org/}
  % \end{itemize}
% \end{frame}


% \begin{frame}
% \frametitle{Examples in ECLiPSe}
% \begin{itemize}
% \item Open sourced constraint programming language
% \item Development goes back to 1985
% \item ECRC, ICL, IC-Parc, PTL, Cisco
% \item \url{https://eclipseclp.org/}
% \item Specialities
% \begin{itemize}
% \item Develop new solvers for specific domains
% \item Integration with MIP
% \end{itemize}
% \item Not included in bundled MiniZinc IDE
% \item Specialized visualization tools used here
  % \begin{itemize}
  % \item CP-Viz, Simonis et al. 2010
    % \end{itemize}
% \end{itemize}
% \end{frame}

\begin{frame}
\frametitle{Tutorial Based on ECLiPSe ELearning Course}
\begin{itemize}
\item Self-study course in constraint programming
\item Supported by Cisco Systems and Silicon Valley Community Foundation
\item Multi-media format, video lectures, slides, handout etc
\item {\small \url{https://eclipseclp.org/ELearning/index.html}}
\end{itemize}
\end{frame}

\begin{frame}
\frametitle{Also Part of CRT-AI Constraint Week}
\begin{itemize}
\item Annual one week course on CP and Optimization in Ireland
\item Part of national training program for PhD students in AI
\item \url{https://www.crt-ai.ie/}
\end{itemize}
\end{frame}



\begin{frame}
\frametitle{Constraint Programming - in a nutshell}
\begin{itemize}
\item Declarative description of problems with
\begin{itemize}
\item {\em Variables} which range over (finite) sets of values
\item {\em Constraints} over subsets of variables which restrict possible value combinations
\item A {\em solution} is a value assignment which satisfies all constraints
\end{itemize}

\item Constraint propagation/reasoning
\begin{itemize}
\item Removing inconsistent values for variables
\item Detect failure if constraint can not be satisfied
\item Interaction of constraints via shared variables
\item Incomplete
\end{itemize}

\item Search
\begin{itemize}
\item User controlled assignment of values to variables
\item Each step triggers constraint propagation 
\end{itemize}
\item Different domains require/allow different methods
\end{itemize}
\end{frame}

\begin{frame}
  \frametitle{Constraint Programming is Different}
  \begin{itemize}
  \item Declarative Programming
    \begin{itemize}
    \item Concentrate on what you want
      \item Not how to get there
      \item Program != Algorithm
      \item Program = Model
    \end{itemize}
    \item Applied to Combinatorial Problems
      \begin{itemize}
        \item No complete polynomial algorithms known (exist?)
        \item CP less ad-hoc than heuristics
        \item Models can evolve
  \end{itemize}
  \end{itemize}
  \end{frame}
    
\begin{frame}
  \frametitle{A Subtractive Process}
  \begin{textblock}{4}(8,-3)
    \includegraphics[width=4cm]{../introduction/images/stages}
  \end{textblock}
  \vfill
  \begin{quote}
    ``Oh, bosh, as Mr. Ruskin says. Sculpture, per se, is the simplest thing in the world. All you have to do is to take a big chunk of marble and a hammer and chisel, make up your mind what you are about to create and chip off all the marble you don't want.''-Paris Gaulois.
  \end{quote}
  
  {\tiny Source: \url{https://quoteinvestigator.com/2014/06/22/chip-away/}}
\end{frame}

\begin{frame}
\frametitle{Basic Process}
\begin{center}
\begin{tikzpicture}
\node[shape=rectangle,fill=pantone127-4] (problem) at (2,5) {Problem};
\node (human) at (2,4) {Human};
\node[shape=rectangle,fill=pantone127-4] (model) at (2,3) {Model};
\node (solver) at (2,2) {Constraint Solver/Search};
\node[shape=rectangle,fill=pantone157-8] (solution) at (2,1) {Solution};
\draw[-] (problem) -- (human);
\draw[->] (human) -- (model);
\draw[-] (model) -- (solver);
\draw[->] (solver) -- (solution);
\end{tikzpicture}
\end{center}
\end{frame}

\begin{frame}
\frametitle{More Realistic}
\begin{center}
\begin{tikzpicture}[xscale=4.0]
\node[shape=rectangle,fill=pantone127-4] (problem) at (2,5) {Problem};
\node (human) at (2,4) {Human};
\node[shape=rectangle,fill=pantone127-4] (model) at (2,3) {Model};
\node (solver) at (2,2) {Constraint Solver/Search};
\node[shape=rectangle,fill=pantone24-8] (hangs) at (1,1) {Hangs};
\node[shape=rectangle,fill=pantone157-8] (solution) at (2,1) {Solution};
\node[shape=rectangle,fill=pantone24-8] (wrong) at (3,1) {Wrong Solution};
\draw[-] (problem) -- (human);
\draw[->] (human) -- (model);
\draw[-] (model) -- (solver);
\draw[->] (solver) -- (hangs);
\draw[->] (solver) -- (solution);
\draw[->] (solver) -- (wrong);
\draw[->] (hangs) -- (1,4) -- (human);
\draw[->] (wrong) -- (3,4) -- (human);
\end{tikzpicture}
\end{center}
\end{frame}

\begin{frame}
\frametitle{Dual Role of Model}
\begin{itemize}
\item Allows Human to Express Problem
\begin{itemize}
\item Close to Problem Domain
\item Constraints as Abstractions
\end{itemize}
\item Allows Solver to Execute
\begin{itemize}
\item Variables as Communication Mechanism
\item Constraints as Algorithms
\end{itemize}
\end{itemize}
\end{frame}

\begin{frame}
\frametitle{Modelling Frameworks}
\begin{itemize}
\item \href{https://www.minizinc.org/}{MiniZinc} (NICTA, Monash University, Australia)
\item \href{https://github.com/eomahony/Numberjack}{NumberJack} (Insight, Ireland) 
\item \href{https://www-users.york.ac.uk/peter.nightingale/savilerow//}{EssencePrime/SavilleRow} (UK) 
\item \href{https://cpmpy.readthedocs.io/en/latest/}{CPMpy} (KU Leuven)
\item Allow use of multiple back-end solvers
\item Compile model into variants for each solver
\item A priori solver independent model(CP, MIP, SAT)
\end{itemize}
\end{frame} 

\begin{frame}
\frametitle{Framework Process}
\begin{center}
\begin{tikzpicture}[xscale=2]
\node[shape=rectangle,fill=pantone127-4] (problem) at (2.5,5) {Problem};
\node (human) at (2.5,4) {Human};
\node[shape=rectangle,fill=pantone127-4] (model) at (2.5,3) {Model};
\node (compile) at (2.5,2) {Compile/Reformulate};
\node (solver1) at (1,1) {CP};
\node (solver2) at (2,1) {MIP};
\node (solver3) at (3,1) {SAT};
\node (solver4) at (4,1) {Other};
\node[shape=rectangle,fill=pantone157-8] (solution1) at (1,0) {Solution};
\node[shape=rectangle,fill=pantone157-8] (solution2) at (2,0) {Solution};
\node[shape=rectangle,fill=pantone157-8] (solution3) at (3,0) {Solution};
\node[shape=rectangle,fill=pantone157-8] (solution4) at (4,0) {Solution};
\draw[-] (problem) -- (human);
\draw[->] (human) -- (model);
\draw[-] (model) -- (compile);
\draw[-] (compile) -- (solver1);
\draw[-] (compile) -- (solver2);
\draw[-] (compile) -- (solver3);
\draw[-] (compile) -- (solver4);
\draw[->] (solver1) -- (solution1);
\draw[->] (solver2) -- (solution2);
\draw[->] (solver3) -- (solution3);
\draw[->] (solver4) -- (solution4);
\end{tikzpicture}
\end{center}
\end{frame}

% \begin{frame}
  % \frametitle{Do It Now!}
  % \begin{itemize}
  % \item Download and install Minizinc
  % \item \url{https://www.minizinc.org/}
  % \end{itemize}
% \end{frame}


\begin{frame}
\frametitle{Why use Puzzles as Examples?}
\begin{itemize}
\item Easy to understand the problem
\item Solvable by hand without specialized knowledge
\item Possible to compare automated to manual solving process
\end{itemize}

\begin{quote}
The puzzle, though inanimate, is presented as a solvable problem without lasting negative consequences, a very low-risk low-reward situation. By being a puzzle, the object is attempting to convince the user that it must be completed.
\end{quote}
{\tiny Source: \href{https://www.everydayrhetoric.com/2020/03/the-rhetoric-of-puzzles/}{Every Day Rhetoric}}
\end{frame}


\mode<all>{
\part{Basic Constraint Propagation}
}
\begin{frame}
\partpage
\end{frame}
\mode<all>{
\input{../sendmore/body}
}

\mode<all>{
\part{Global Constraints}
}
\begin{frame}
\partpage
\end{frame}
\mode<all>{
\input{../sudoku/body}
}

\clearpage
\mode<all>{
\part{Customizing Search}
}
\begin{frame}
\partpage
\end{frame}
\mode<all>{
\input{../nqueen/body}
}


%% \clearpage
%% \mode<all>{
%%   \part{Problem Decomposition}
%% }
%% \begin{frame}
%%   \partpage
%% \end{frame}
%% \mode<all>{
%%   \input{../party/body}
%%   }

%% \clearpage
%% \mode<all>{
%%   \part{More Global Constraints}
%% }
%% \begin{frame}
%%   \partpage
%% \end{frame}
%% \mode<all>{
%%   \input{../car/body}
%%   }

%% \clearpage
%% \mode<all>{
%%   \part{Methodology}
%% }
%% \begin{frame}
%%   \partpage
%% \end{frame}
%% \mode<all>{
%%   \input{../methodology/body}
%%   }

\clearpage
\mode<all>{
\part{What is missing?}
}
\begin{frame}
\partpage
\end{frame}

\begin{frame}
\frametitle{Many Specialized Topics}
\begin{itemize}
\item How to design efficient core engine
\item Hybrids with LP/MIP tools
\item Hybrids with SAT
\item Symmetry breaking
\item Use of MDD/BDD to encode sets of solutions
\item High level modelling tools
\item Debugging/visualization
\end{itemize}
\end{frame}

\begin{frame}
\frametitle{Reformulation}
\begin{itemize}
\item Just because the user has modelled it this way, it doesn't mean we have to solve it that way
\begin{itemize}
\item Replace some constraint(s) by other, equivalent constraints
\item Because we don't have that constraint in our system
\item For performance
\end{itemize}
\end{itemize}
\end{frame}

\begin{frame}
\frametitle{Learning}
\begin{itemize}
\item While solving the problem we can learn how to strengthen the model/search 
\begin{itemize}
\item Understand which constraints/method contribute to propagation and change schedule
\item Learn no-good constraints by explaining failure
\item Adapt search strategy based on search experience
\end{itemize}
\end{itemize}
\end{frame}

%% \begin{frame}
%% \frametitle{Refined Process}
%% \begin{center}
%% \begin{tikzpicture}
%% \node[shape=rectangle,fill=pantone127-4] (problem) at (3,7) {Problem};
%% \node (human) at (3,6) {Human};
%% \node[shape=rectangle,fill=pantone127-4] (model) at (3,5) {User Model};
%% \node (reformulation) at (3,4) {Reformulation};
%% \node[shape=rectangle,fill=pantone127-4] (implementation) at (3,3) {Implementation Model};
%% \node (solver) at (3,2) {Constraint Solver/Search};
%% \node[shape=rectangle,fill=pantone157-8] (solution) at (3,1) {Solution};
%% \draw[-] (problem) -- (human);
%% \draw[->] (human) -- (model);
%% \draw[-] (model) -- (reformulation);
%% \draw[->] (reformulation) -- (implementation);
%% \draw[-] (implementation) -- (solver);
%% \draw[->] (solver) -- (0,2) -- (0,3) -- (implementation);
%% \draw[->] (solver) -- (solution);
%% \end{tikzpicture}
%% \end{center}
%% \end{frame}

%\begin{frame}
%\frametitle{What is CP actually used for?}
%\url{http://hsimonis.wordpress.com}
%\includegraphics[width=10cm]{../introduction/images/blog}
%\end{frame}

%% \begin{frame}
%% \frametitle{Exercises}
%% \begin{itemize}
%% \item Install Constraint Solver of your Choice
%% \item MiniZinc (MiningZinc) will be used in later sessions
%% \item Run Sendmoremoney, Sudoku, Queens examples
%% \item Can you choose which propagator to use for alldifferent?
%% \item Can you use different search strategies?
%% \end{itemize}
%% \end{frame}

\begin{frame}
\frametitle{More Learning Resources}
\begin{itemize}
\item Survey of Methods, Resources, and Formats for Teaching Constraint Programming
\begin{itemize}
\item by Tejas Santanam, Helmut Simonis
\item \url{https://doi.org/10.48550/arXiv.2403.12717}
\item Based on survey of community for WTCP 2023
\item \url{https://hsimonis.github.io/WTCP2023/}
\end{itemize}
\end{itemize}
\end{frame}


}

\clearpage
\mode<all>{
\part{Customizing Search}
}
\begin{frame}
\partpage
\end{frame}
\mode<all>{
\begin{frame}
\frametitle{Objectives}
\begin{itemize}
\item Overview of Core Constraint Programming
\item Three Main Concepts
\begin{itemize}
\item Constraint Propagation
\item Global Constraints
\item Customizing Search
\end{itemize}
%\item Get Some Experience with MiniZinc
\item Topics will be treated in more detail in later parts of the school
\item Based on Examples, not Formal Description
\end{itemize}
\end{frame}

\begin{frame}
\frametitle{Outline}
\begin{itemize}
\item Why Constraint Programming?
\item Constraint Propagation
\item Global Constraints
\item Customizing Search
%\item Choices in Modelling
%\item Decomposition
%\item More Global Constraints
%\item Methodology
%\item What is missing?
\end{itemize}
\end{frame}

% \begin{frame}
  % \frametitle{Using MiniZinc IDE}
  % \begin{itemize}
  % \item Developed in the Australian NICTA project
  % \item Maintained by Monash University
  % \item Modelling tool with multiple back-end solvers
  % \item Available from \url{https://www.minizinc.org/}
  % \end{itemize}
% \end{frame}


% \begin{frame}
% \frametitle{Examples in ECLiPSe}
% \begin{itemize}
% \item Open sourced constraint programming language
% \item Development goes back to 1985
% \item ECRC, ICL, IC-Parc, PTL, Cisco
% \item \url{https://eclipseclp.org/}
% \item Specialities
% \begin{itemize}
% \item Develop new solvers for specific domains
% \item Integration with MIP
% \end{itemize}
% \item Not included in bundled MiniZinc IDE
% \item Specialized visualization tools used here
  % \begin{itemize}
  % \item CP-Viz, Simonis et al. 2010
    % \end{itemize}
% \end{itemize}
% \end{frame}

\begin{frame}
\frametitle{Tutorial Based on ECLiPSe ELearning Course}
\begin{itemize}
\item Self-study course in constraint programming
\item Supported by Cisco Systems and Silicon Valley Community Foundation
\item Multi-media format, video lectures, slides, handout etc
\item {\small \url{https://eclipseclp.org/ELearning/index.html}}
\end{itemize}
\end{frame}

\begin{frame}
\frametitle{Also Part of CRT-AI Constraint Week}
\begin{itemize}
\item Annual one week course on CP and Optimization in Ireland
\item Part of national training program for PhD students in AI
\item \url{https://www.crt-ai.ie/}
\end{itemize}
\end{frame}



\begin{frame}
\frametitle{Constraint Programming - in a nutshell}
\begin{itemize}
\item Declarative description of problems with
\begin{itemize}
\item {\em Variables} which range over (finite) sets of values
\item {\em Constraints} over subsets of variables which restrict possible value combinations
\item A {\em solution} is a value assignment which satisfies all constraints
\end{itemize}

\item Constraint propagation/reasoning
\begin{itemize}
\item Removing inconsistent values for variables
\item Detect failure if constraint can not be satisfied
\item Interaction of constraints via shared variables
\item Incomplete
\end{itemize}

\item Search
\begin{itemize}
\item User controlled assignment of values to variables
\item Each step triggers constraint propagation 
\end{itemize}
\item Different domains require/allow different methods
\end{itemize}
\end{frame}

\begin{frame}
  \frametitle{Constraint Programming is Different}
  \begin{itemize}
  \item Declarative Programming
    \begin{itemize}
    \item Concentrate on what you want
      \item Not how to get there
      \item Program != Algorithm
      \item Program = Model
    \end{itemize}
    \item Applied to Combinatorial Problems
      \begin{itemize}
        \item No complete polynomial algorithms known (exist?)
        \item CP less ad-hoc than heuristics
        \item Models can evolve
  \end{itemize}
  \end{itemize}
  \end{frame}
    
\begin{frame}
  \frametitle{A Subtractive Process}
  \begin{textblock}{4}(8,-3)
    \includegraphics[width=4cm]{../introduction/images/stages}
  \end{textblock}
  \vfill
  \begin{quote}
    ``Oh, bosh, as Mr. Ruskin says. Sculpture, per se, is the simplest thing in the world. All you have to do is to take a big chunk of marble and a hammer and chisel, make up your mind what you are about to create and chip off all the marble you don't want.''-Paris Gaulois.
  \end{quote}
  
  {\tiny Source: \url{https://quoteinvestigator.com/2014/06/22/chip-away/}}
\end{frame}

\begin{frame}
\frametitle{Basic Process}
\begin{center}
\begin{tikzpicture}
\node[shape=rectangle,fill=pantone127-4] (problem) at (2,5) {Problem};
\node (human) at (2,4) {Human};
\node[shape=rectangle,fill=pantone127-4] (model) at (2,3) {Model};
\node (solver) at (2,2) {Constraint Solver/Search};
\node[shape=rectangle,fill=pantone157-8] (solution) at (2,1) {Solution};
\draw[-] (problem) -- (human);
\draw[->] (human) -- (model);
\draw[-] (model) -- (solver);
\draw[->] (solver) -- (solution);
\end{tikzpicture}
\end{center}
\end{frame}

\begin{frame}
\frametitle{More Realistic}
\begin{center}
\begin{tikzpicture}[xscale=4.0]
\node[shape=rectangle,fill=pantone127-4] (problem) at (2,5) {Problem};
\node (human) at (2,4) {Human};
\node[shape=rectangle,fill=pantone127-4] (model) at (2,3) {Model};
\node (solver) at (2,2) {Constraint Solver/Search};
\node[shape=rectangle,fill=pantone24-8] (hangs) at (1,1) {Hangs};
\node[shape=rectangle,fill=pantone157-8] (solution) at (2,1) {Solution};
\node[shape=rectangle,fill=pantone24-8] (wrong) at (3,1) {Wrong Solution};
\draw[-] (problem) -- (human);
\draw[->] (human) -- (model);
\draw[-] (model) -- (solver);
\draw[->] (solver) -- (hangs);
\draw[->] (solver) -- (solution);
\draw[->] (solver) -- (wrong);
\draw[->] (hangs) -- (1,4) -- (human);
\draw[->] (wrong) -- (3,4) -- (human);
\end{tikzpicture}
\end{center}
\end{frame}

\begin{frame}
\frametitle{Dual Role of Model}
\begin{itemize}
\item Allows Human to Express Problem
\begin{itemize}
\item Close to Problem Domain
\item Constraints as Abstractions
\end{itemize}
\item Allows Solver to Execute
\begin{itemize}
\item Variables as Communication Mechanism
\item Constraints as Algorithms
\end{itemize}
\end{itemize}
\end{frame}

\begin{frame}
\frametitle{Modelling Frameworks}
\begin{itemize}
\item \href{https://www.minizinc.org/}{MiniZinc} (NICTA, Monash University, Australia)
\item \href{https://github.com/eomahony/Numberjack}{NumberJack} (Insight, Ireland) 
\item \href{https://www-users.york.ac.uk/peter.nightingale/savilerow//}{EssencePrime/SavilleRow} (UK) 
\item \href{https://cpmpy.readthedocs.io/en/latest/}{CPMpy} (KU Leuven)
\item Allow use of multiple back-end solvers
\item Compile model into variants for each solver
\item A priori solver independent model(CP, MIP, SAT)
\end{itemize}
\end{frame} 

\begin{frame}
\frametitle{Framework Process}
\begin{center}
\begin{tikzpicture}[xscale=2]
\node[shape=rectangle,fill=pantone127-4] (problem) at (2.5,5) {Problem};
\node (human) at (2.5,4) {Human};
\node[shape=rectangle,fill=pantone127-4] (model) at (2.5,3) {Model};
\node (compile) at (2.5,2) {Compile/Reformulate};
\node (solver1) at (1,1) {CP};
\node (solver2) at (2,1) {MIP};
\node (solver3) at (3,1) {SAT};
\node (solver4) at (4,1) {Other};
\node[shape=rectangle,fill=pantone157-8] (solution1) at (1,0) {Solution};
\node[shape=rectangle,fill=pantone157-8] (solution2) at (2,0) {Solution};
\node[shape=rectangle,fill=pantone157-8] (solution3) at (3,0) {Solution};
\node[shape=rectangle,fill=pantone157-8] (solution4) at (4,0) {Solution};
\draw[-] (problem) -- (human);
\draw[->] (human) -- (model);
\draw[-] (model) -- (compile);
\draw[-] (compile) -- (solver1);
\draw[-] (compile) -- (solver2);
\draw[-] (compile) -- (solver3);
\draw[-] (compile) -- (solver4);
\draw[->] (solver1) -- (solution1);
\draw[->] (solver2) -- (solution2);
\draw[->] (solver3) -- (solution3);
\draw[->] (solver4) -- (solution4);
\end{tikzpicture}
\end{center}
\end{frame}

% \begin{frame}
  % \frametitle{Do It Now!}
  % \begin{itemize}
  % \item Download and install Minizinc
  % \item \url{https://www.minizinc.org/}
  % \end{itemize}
% \end{frame}


\begin{frame}
\frametitle{Why use Puzzles as Examples?}
\begin{itemize}
\item Easy to understand the problem
\item Solvable by hand without specialized knowledge
\item Possible to compare automated to manual solving process
\end{itemize}

\begin{quote}
The puzzle, though inanimate, is presented as a solvable problem without lasting negative consequences, a very low-risk low-reward situation. By being a puzzle, the object is attempting to convince the user that it must be completed.
\end{quote}
{\tiny Source: \href{https://www.everydayrhetoric.com/2020/03/the-rhetoric-of-puzzles/}{Every Day Rhetoric}}
\end{frame}


\mode<all>{
\part{Basic Constraint Propagation}
}
\begin{frame}
\partpage
\end{frame}
\mode<all>{
\input{../sendmore/body}
}

\mode<all>{
\part{Global Constraints}
}
\begin{frame}
\partpage
\end{frame}
\mode<all>{
\input{../sudoku/body}
}

\clearpage
\mode<all>{
\part{Customizing Search}
}
\begin{frame}
\partpage
\end{frame}
\mode<all>{
\input{../nqueen/body}
}


%% \clearpage
%% \mode<all>{
%%   \part{Problem Decomposition}
%% }
%% \begin{frame}
%%   \partpage
%% \end{frame}
%% \mode<all>{
%%   \input{../party/body}
%%   }

%% \clearpage
%% \mode<all>{
%%   \part{More Global Constraints}
%% }
%% \begin{frame}
%%   \partpage
%% \end{frame}
%% \mode<all>{
%%   \input{../car/body}
%%   }

%% \clearpage
%% \mode<all>{
%%   \part{Methodology}
%% }
%% \begin{frame}
%%   \partpage
%% \end{frame}
%% \mode<all>{
%%   \input{../methodology/body}
%%   }

\clearpage
\mode<all>{
\part{What is missing?}
}
\begin{frame}
\partpage
\end{frame}

\begin{frame}
\frametitle{Many Specialized Topics}
\begin{itemize}
\item How to design efficient core engine
\item Hybrids with LP/MIP tools
\item Hybrids with SAT
\item Symmetry breaking
\item Use of MDD/BDD to encode sets of solutions
\item High level modelling tools
\item Debugging/visualization
\end{itemize}
\end{frame}

\begin{frame}
\frametitle{Reformulation}
\begin{itemize}
\item Just because the user has modelled it this way, it doesn't mean we have to solve it that way
\begin{itemize}
\item Replace some constraint(s) by other, equivalent constraints
\item Because we don't have that constraint in our system
\item For performance
\end{itemize}
\end{itemize}
\end{frame}

\begin{frame}
\frametitle{Learning}
\begin{itemize}
\item While solving the problem we can learn how to strengthen the model/search 
\begin{itemize}
\item Understand which constraints/method contribute to propagation and change schedule
\item Learn no-good constraints by explaining failure
\item Adapt search strategy based on search experience
\end{itemize}
\end{itemize}
\end{frame}

%% \begin{frame}
%% \frametitle{Refined Process}
%% \begin{center}
%% \begin{tikzpicture}
%% \node[shape=rectangle,fill=pantone127-4] (problem) at (3,7) {Problem};
%% \node (human) at (3,6) {Human};
%% \node[shape=rectangle,fill=pantone127-4] (model) at (3,5) {User Model};
%% \node (reformulation) at (3,4) {Reformulation};
%% \node[shape=rectangle,fill=pantone127-4] (implementation) at (3,3) {Implementation Model};
%% \node (solver) at (3,2) {Constraint Solver/Search};
%% \node[shape=rectangle,fill=pantone157-8] (solution) at (3,1) {Solution};
%% \draw[-] (problem) -- (human);
%% \draw[->] (human) -- (model);
%% \draw[-] (model) -- (reformulation);
%% \draw[->] (reformulation) -- (implementation);
%% \draw[-] (implementation) -- (solver);
%% \draw[->] (solver) -- (0,2) -- (0,3) -- (implementation);
%% \draw[->] (solver) -- (solution);
%% \end{tikzpicture}
%% \end{center}
%% \end{frame}

%\begin{frame}
%\frametitle{What is CP actually used for?}
%\url{http://hsimonis.wordpress.com}
%\includegraphics[width=10cm]{../introduction/images/blog}
%\end{frame}

%% \begin{frame}
%% \frametitle{Exercises}
%% \begin{itemize}
%% \item Install Constraint Solver of your Choice
%% \item MiniZinc (MiningZinc) will be used in later sessions
%% \item Run Sendmoremoney, Sudoku, Queens examples
%% \item Can you choose which propagator to use for alldifferent?
%% \item Can you use different search strategies?
%% \end{itemize}
%% \end{frame}

\begin{frame}
\frametitle{More Learning Resources}
\begin{itemize}
\item Survey of Methods, Resources, and Formats for Teaching Constraint Programming
\begin{itemize}
\item by Tejas Santanam, Helmut Simonis
\item \url{https://doi.org/10.48550/arXiv.2403.12717}
\item Based on survey of community for WTCP 2023
\item \url{https://hsimonis.github.io/WTCP2023/}
\end{itemize}
\end{itemize}
\end{frame}


}


%% \clearpage
%% \mode<all>{
%%   \part{Problem Decomposition}
%% }
%% \begin{frame}
%%   \partpage
%% \end{frame}
%% \mode<all>{
%%   \begin{frame}
\frametitle{Objectives}
\begin{itemize}
\item Overview of Core Constraint Programming
\item Three Main Concepts
\begin{itemize}
\item Constraint Propagation
\item Global Constraints
\item Customizing Search
\end{itemize}
%\item Get Some Experience with MiniZinc
\item Topics will be treated in more detail in later parts of the school
\item Based on Examples, not Formal Description
\end{itemize}
\end{frame}

\begin{frame}
\frametitle{Outline}
\begin{itemize}
\item Why Constraint Programming?
\item Constraint Propagation
\item Global Constraints
\item Customizing Search
%\item Choices in Modelling
%\item Decomposition
%\item More Global Constraints
%\item Methodology
%\item What is missing?
\end{itemize}
\end{frame}

% \begin{frame}
  % \frametitle{Using MiniZinc IDE}
  % \begin{itemize}
  % \item Developed in the Australian NICTA project
  % \item Maintained by Monash University
  % \item Modelling tool with multiple back-end solvers
  % \item Available from \url{https://www.minizinc.org/}
  % \end{itemize}
% \end{frame}


% \begin{frame}
% \frametitle{Examples in ECLiPSe}
% \begin{itemize}
% \item Open sourced constraint programming language
% \item Development goes back to 1985
% \item ECRC, ICL, IC-Parc, PTL, Cisco
% \item \url{https://eclipseclp.org/}
% \item Specialities
% \begin{itemize}
% \item Develop new solvers for specific domains
% \item Integration with MIP
% \end{itemize}
% \item Not included in bundled MiniZinc IDE
% \item Specialized visualization tools used here
  % \begin{itemize}
  % \item CP-Viz, Simonis et al. 2010
    % \end{itemize}
% \end{itemize}
% \end{frame}

\begin{frame}
\frametitle{Tutorial Based on ECLiPSe ELearning Course}
\begin{itemize}
\item Self-study course in constraint programming
\item Supported by Cisco Systems and Silicon Valley Community Foundation
\item Multi-media format, video lectures, slides, handout etc
\item {\small \url{https://eclipseclp.org/ELearning/index.html}}
\end{itemize}
\end{frame}

\begin{frame}
\frametitle{Also Part of CRT-AI Constraint Week}
\begin{itemize}
\item Annual one week course on CP and Optimization in Ireland
\item Part of national training program for PhD students in AI
\item \url{https://www.crt-ai.ie/}
\end{itemize}
\end{frame}



\begin{frame}
\frametitle{Constraint Programming - in a nutshell}
\begin{itemize}
\item Declarative description of problems with
\begin{itemize}
\item {\em Variables} which range over (finite) sets of values
\item {\em Constraints} over subsets of variables which restrict possible value combinations
\item A {\em solution} is a value assignment which satisfies all constraints
\end{itemize}

\item Constraint propagation/reasoning
\begin{itemize}
\item Removing inconsistent values for variables
\item Detect failure if constraint can not be satisfied
\item Interaction of constraints via shared variables
\item Incomplete
\end{itemize}

\item Search
\begin{itemize}
\item User controlled assignment of values to variables
\item Each step triggers constraint propagation 
\end{itemize}
\item Different domains require/allow different methods
\end{itemize}
\end{frame}

\begin{frame}
  \frametitle{Constraint Programming is Different}
  \begin{itemize}
  \item Declarative Programming
    \begin{itemize}
    \item Concentrate on what you want
      \item Not how to get there
      \item Program != Algorithm
      \item Program = Model
    \end{itemize}
    \item Applied to Combinatorial Problems
      \begin{itemize}
        \item No complete polynomial algorithms known (exist?)
        \item CP less ad-hoc than heuristics
        \item Models can evolve
  \end{itemize}
  \end{itemize}
  \end{frame}
    
\begin{frame}
  \frametitle{A Subtractive Process}
  \begin{textblock}{4}(8,-3)
    \includegraphics[width=4cm]{../introduction/images/stages}
  \end{textblock}
  \vfill
  \begin{quote}
    ``Oh, bosh, as Mr. Ruskin says. Sculpture, per se, is the simplest thing in the world. All you have to do is to take a big chunk of marble and a hammer and chisel, make up your mind what you are about to create and chip off all the marble you don't want.''-Paris Gaulois.
  \end{quote}
  
  {\tiny Source: \url{https://quoteinvestigator.com/2014/06/22/chip-away/}}
\end{frame}

\begin{frame}
\frametitle{Basic Process}
\begin{center}
\begin{tikzpicture}
\node[shape=rectangle,fill=pantone127-4] (problem) at (2,5) {Problem};
\node (human) at (2,4) {Human};
\node[shape=rectangle,fill=pantone127-4] (model) at (2,3) {Model};
\node (solver) at (2,2) {Constraint Solver/Search};
\node[shape=rectangle,fill=pantone157-8] (solution) at (2,1) {Solution};
\draw[-] (problem) -- (human);
\draw[->] (human) -- (model);
\draw[-] (model) -- (solver);
\draw[->] (solver) -- (solution);
\end{tikzpicture}
\end{center}
\end{frame}

\begin{frame}
\frametitle{More Realistic}
\begin{center}
\begin{tikzpicture}[xscale=4.0]
\node[shape=rectangle,fill=pantone127-4] (problem) at (2,5) {Problem};
\node (human) at (2,4) {Human};
\node[shape=rectangle,fill=pantone127-4] (model) at (2,3) {Model};
\node (solver) at (2,2) {Constraint Solver/Search};
\node[shape=rectangle,fill=pantone24-8] (hangs) at (1,1) {Hangs};
\node[shape=rectangle,fill=pantone157-8] (solution) at (2,1) {Solution};
\node[shape=rectangle,fill=pantone24-8] (wrong) at (3,1) {Wrong Solution};
\draw[-] (problem) -- (human);
\draw[->] (human) -- (model);
\draw[-] (model) -- (solver);
\draw[->] (solver) -- (hangs);
\draw[->] (solver) -- (solution);
\draw[->] (solver) -- (wrong);
\draw[->] (hangs) -- (1,4) -- (human);
\draw[->] (wrong) -- (3,4) -- (human);
\end{tikzpicture}
\end{center}
\end{frame}

\begin{frame}
\frametitle{Dual Role of Model}
\begin{itemize}
\item Allows Human to Express Problem
\begin{itemize}
\item Close to Problem Domain
\item Constraints as Abstractions
\end{itemize}
\item Allows Solver to Execute
\begin{itemize}
\item Variables as Communication Mechanism
\item Constraints as Algorithms
\end{itemize}
\end{itemize}
\end{frame}

\begin{frame}
\frametitle{Modelling Frameworks}
\begin{itemize}
\item \href{https://www.minizinc.org/}{MiniZinc} (NICTA, Monash University, Australia)
\item \href{https://github.com/eomahony/Numberjack}{NumberJack} (Insight, Ireland) 
\item \href{https://www-users.york.ac.uk/peter.nightingale/savilerow//}{EssencePrime/SavilleRow} (UK) 
\item \href{https://cpmpy.readthedocs.io/en/latest/}{CPMpy} (KU Leuven)
\item Allow use of multiple back-end solvers
\item Compile model into variants for each solver
\item A priori solver independent model(CP, MIP, SAT)
\end{itemize}
\end{frame} 

\begin{frame}
\frametitle{Framework Process}
\begin{center}
\begin{tikzpicture}[xscale=2]
\node[shape=rectangle,fill=pantone127-4] (problem) at (2.5,5) {Problem};
\node (human) at (2.5,4) {Human};
\node[shape=rectangle,fill=pantone127-4] (model) at (2.5,3) {Model};
\node (compile) at (2.5,2) {Compile/Reformulate};
\node (solver1) at (1,1) {CP};
\node (solver2) at (2,1) {MIP};
\node (solver3) at (3,1) {SAT};
\node (solver4) at (4,1) {Other};
\node[shape=rectangle,fill=pantone157-8] (solution1) at (1,0) {Solution};
\node[shape=rectangle,fill=pantone157-8] (solution2) at (2,0) {Solution};
\node[shape=rectangle,fill=pantone157-8] (solution3) at (3,0) {Solution};
\node[shape=rectangle,fill=pantone157-8] (solution4) at (4,0) {Solution};
\draw[-] (problem) -- (human);
\draw[->] (human) -- (model);
\draw[-] (model) -- (compile);
\draw[-] (compile) -- (solver1);
\draw[-] (compile) -- (solver2);
\draw[-] (compile) -- (solver3);
\draw[-] (compile) -- (solver4);
\draw[->] (solver1) -- (solution1);
\draw[->] (solver2) -- (solution2);
\draw[->] (solver3) -- (solution3);
\draw[->] (solver4) -- (solution4);
\end{tikzpicture}
\end{center}
\end{frame}

% \begin{frame}
  % \frametitle{Do It Now!}
  % \begin{itemize}
  % \item Download and install Minizinc
  % \item \url{https://www.minizinc.org/}
  % \end{itemize}
% \end{frame}


\begin{frame}
\frametitle{Why use Puzzles as Examples?}
\begin{itemize}
\item Easy to understand the problem
\item Solvable by hand without specialized knowledge
\item Possible to compare automated to manual solving process
\end{itemize}

\begin{quote}
The puzzle, though inanimate, is presented as a solvable problem without lasting negative consequences, a very low-risk low-reward situation. By being a puzzle, the object is attempting to convince the user that it must be completed.
\end{quote}
{\tiny Source: \href{https://www.everydayrhetoric.com/2020/03/the-rhetoric-of-puzzles/}{Every Day Rhetoric}}
\end{frame}


\mode<all>{
\part{Basic Constraint Propagation}
}
\begin{frame}
\partpage
\end{frame}
\mode<all>{
\input{../sendmore/body}
}

\mode<all>{
\part{Global Constraints}
}
\begin{frame}
\partpage
\end{frame}
\mode<all>{
\input{../sudoku/body}
}

\clearpage
\mode<all>{
\part{Customizing Search}
}
\begin{frame}
\partpage
\end{frame}
\mode<all>{
\input{../nqueen/body}
}


%% \clearpage
%% \mode<all>{
%%   \part{Problem Decomposition}
%% }
%% \begin{frame}
%%   \partpage
%% \end{frame}
%% \mode<all>{
%%   \input{../party/body}
%%   }

%% \clearpage
%% \mode<all>{
%%   \part{More Global Constraints}
%% }
%% \begin{frame}
%%   \partpage
%% \end{frame}
%% \mode<all>{
%%   \input{../car/body}
%%   }

%% \clearpage
%% \mode<all>{
%%   \part{Methodology}
%% }
%% \begin{frame}
%%   \partpage
%% \end{frame}
%% \mode<all>{
%%   \input{../methodology/body}
%%   }

\clearpage
\mode<all>{
\part{What is missing?}
}
\begin{frame}
\partpage
\end{frame}

\begin{frame}
\frametitle{Many Specialized Topics}
\begin{itemize}
\item How to design efficient core engine
\item Hybrids with LP/MIP tools
\item Hybrids with SAT
\item Symmetry breaking
\item Use of MDD/BDD to encode sets of solutions
\item High level modelling tools
\item Debugging/visualization
\end{itemize}
\end{frame}

\begin{frame}
\frametitle{Reformulation}
\begin{itemize}
\item Just because the user has modelled it this way, it doesn't mean we have to solve it that way
\begin{itemize}
\item Replace some constraint(s) by other, equivalent constraints
\item Because we don't have that constraint in our system
\item For performance
\end{itemize}
\end{itemize}
\end{frame}

\begin{frame}
\frametitle{Learning}
\begin{itemize}
\item While solving the problem we can learn how to strengthen the model/search 
\begin{itemize}
\item Understand which constraints/method contribute to propagation and change schedule
\item Learn no-good constraints by explaining failure
\item Adapt search strategy based on search experience
\end{itemize}
\end{itemize}
\end{frame}

%% \begin{frame}
%% \frametitle{Refined Process}
%% \begin{center}
%% \begin{tikzpicture}
%% \node[shape=rectangle,fill=pantone127-4] (problem) at (3,7) {Problem};
%% \node (human) at (3,6) {Human};
%% \node[shape=rectangle,fill=pantone127-4] (model) at (3,5) {User Model};
%% \node (reformulation) at (3,4) {Reformulation};
%% \node[shape=rectangle,fill=pantone127-4] (implementation) at (3,3) {Implementation Model};
%% \node (solver) at (3,2) {Constraint Solver/Search};
%% \node[shape=rectangle,fill=pantone157-8] (solution) at (3,1) {Solution};
%% \draw[-] (problem) -- (human);
%% \draw[->] (human) -- (model);
%% \draw[-] (model) -- (reformulation);
%% \draw[->] (reformulation) -- (implementation);
%% \draw[-] (implementation) -- (solver);
%% \draw[->] (solver) -- (0,2) -- (0,3) -- (implementation);
%% \draw[->] (solver) -- (solution);
%% \end{tikzpicture}
%% \end{center}
%% \end{frame}

%\begin{frame}
%\frametitle{What is CP actually used for?}
%\url{http://hsimonis.wordpress.com}
%\includegraphics[width=10cm]{../introduction/images/blog}
%\end{frame}

%% \begin{frame}
%% \frametitle{Exercises}
%% \begin{itemize}
%% \item Install Constraint Solver of your Choice
%% \item MiniZinc (MiningZinc) will be used in later sessions
%% \item Run Sendmoremoney, Sudoku, Queens examples
%% \item Can you choose which propagator to use for alldifferent?
%% \item Can you use different search strategies?
%% \end{itemize}
%% \end{frame}

\begin{frame}
\frametitle{More Learning Resources}
\begin{itemize}
\item Survey of Methods, Resources, and Formats for Teaching Constraint Programming
\begin{itemize}
\item by Tejas Santanam, Helmut Simonis
\item \url{https://doi.org/10.48550/arXiv.2403.12717}
\item Based on survey of community for WTCP 2023
\item \url{https://hsimonis.github.io/WTCP2023/}
\end{itemize}
\end{itemize}
\end{frame}


%%   }

%% \clearpage
%% \mode<all>{
%%   \part{More Global Constraints}
%% }
%% \begin{frame}
%%   \partpage
%% \end{frame}
%% \mode<all>{
%%   \begin{frame}
\frametitle{Objectives}
\begin{itemize}
\item Overview of Core Constraint Programming
\item Three Main Concepts
\begin{itemize}
\item Constraint Propagation
\item Global Constraints
\item Customizing Search
\end{itemize}
%\item Get Some Experience with MiniZinc
\item Topics will be treated in more detail in later parts of the school
\item Based on Examples, not Formal Description
\end{itemize}
\end{frame}

\begin{frame}
\frametitle{Outline}
\begin{itemize}
\item Why Constraint Programming?
\item Constraint Propagation
\item Global Constraints
\item Customizing Search
%\item Choices in Modelling
%\item Decomposition
%\item More Global Constraints
%\item Methodology
%\item What is missing?
\end{itemize}
\end{frame}

% \begin{frame}
  % \frametitle{Using MiniZinc IDE}
  % \begin{itemize}
  % \item Developed in the Australian NICTA project
  % \item Maintained by Monash University
  % \item Modelling tool with multiple back-end solvers
  % \item Available from \url{https://www.minizinc.org/}
  % \end{itemize}
% \end{frame}


% \begin{frame}
% \frametitle{Examples in ECLiPSe}
% \begin{itemize}
% \item Open sourced constraint programming language
% \item Development goes back to 1985
% \item ECRC, ICL, IC-Parc, PTL, Cisco
% \item \url{https://eclipseclp.org/}
% \item Specialities
% \begin{itemize}
% \item Develop new solvers for specific domains
% \item Integration with MIP
% \end{itemize}
% \item Not included in bundled MiniZinc IDE
% \item Specialized visualization tools used here
  % \begin{itemize}
  % \item CP-Viz, Simonis et al. 2010
    % \end{itemize}
% \end{itemize}
% \end{frame}

\begin{frame}
\frametitle{Tutorial Based on ECLiPSe ELearning Course}
\begin{itemize}
\item Self-study course in constraint programming
\item Supported by Cisco Systems and Silicon Valley Community Foundation
\item Multi-media format, video lectures, slides, handout etc
\item {\small \url{https://eclipseclp.org/ELearning/index.html}}
\end{itemize}
\end{frame}

\begin{frame}
\frametitle{Also Part of CRT-AI Constraint Week}
\begin{itemize}
\item Annual one week course on CP and Optimization in Ireland
\item Part of national training program for PhD students in AI
\item \url{https://www.crt-ai.ie/}
\end{itemize}
\end{frame}



\begin{frame}
\frametitle{Constraint Programming - in a nutshell}
\begin{itemize}
\item Declarative description of problems with
\begin{itemize}
\item {\em Variables} which range over (finite) sets of values
\item {\em Constraints} over subsets of variables which restrict possible value combinations
\item A {\em solution} is a value assignment which satisfies all constraints
\end{itemize}

\item Constraint propagation/reasoning
\begin{itemize}
\item Removing inconsistent values for variables
\item Detect failure if constraint can not be satisfied
\item Interaction of constraints via shared variables
\item Incomplete
\end{itemize}

\item Search
\begin{itemize}
\item User controlled assignment of values to variables
\item Each step triggers constraint propagation 
\end{itemize}
\item Different domains require/allow different methods
\end{itemize}
\end{frame}

\begin{frame}
  \frametitle{Constraint Programming is Different}
  \begin{itemize}
  \item Declarative Programming
    \begin{itemize}
    \item Concentrate on what you want
      \item Not how to get there
      \item Program != Algorithm
      \item Program = Model
    \end{itemize}
    \item Applied to Combinatorial Problems
      \begin{itemize}
        \item No complete polynomial algorithms known (exist?)
        \item CP less ad-hoc than heuristics
        \item Models can evolve
  \end{itemize}
  \end{itemize}
  \end{frame}
    
\begin{frame}
  \frametitle{A Subtractive Process}
  \begin{textblock}{4}(8,-3)
    \includegraphics[width=4cm]{../introduction/images/stages}
  \end{textblock}
  \vfill
  \begin{quote}
    ``Oh, bosh, as Mr. Ruskin says. Sculpture, per se, is the simplest thing in the world. All you have to do is to take a big chunk of marble and a hammer and chisel, make up your mind what you are about to create and chip off all the marble you don't want.''-Paris Gaulois.
  \end{quote}
  
  {\tiny Source: \url{https://quoteinvestigator.com/2014/06/22/chip-away/}}
\end{frame}

\begin{frame}
\frametitle{Basic Process}
\begin{center}
\begin{tikzpicture}
\node[shape=rectangle,fill=pantone127-4] (problem) at (2,5) {Problem};
\node (human) at (2,4) {Human};
\node[shape=rectangle,fill=pantone127-4] (model) at (2,3) {Model};
\node (solver) at (2,2) {Constraint Solver/Search};
\node[shape=rectangle,fill=pantone157-8] (solution) at (2,1) {Solution};
\draw[-] (problem) -- (human);
\draw[->] (human) -- (model);
\draw[-] (model) -- (solver);
\draw[->] (solver) -- (solution);
\end{tikzpicture}
\end{center}
\end{frame}

\begin{frame}
\frametitle{More Realistic}
\begin{center}
\begin{tikzpicture}[xscale=4.0]
\node[shape=rectangle,fill=pantone127-4] (problem) at (2,5) {Problem};
\node (human) at (2,4) {Human};
\node[shape=rectangle,fill=pantone127-4] (model) at (2,3) {Model};
\node (solver) at (2,2) {Constraint Solver/Search};
\node[shape=rectangle,fill=pantone24-8] (hangs) at (1,1) {Hangs};
\node[shape=rectangle,fill=pantone157-8] (solution) at (2,1) {Solution};
\node[shape=rectangle,fill=pantone24-8] (wrong) at (3,1) {Wrong Solution};
\draw[-] (problem) -- (human);
\draw[->] (human) -- (model);
\draw[-] (model) -- (solver);
\draw[->] (solver) -- (hangs);
\draw[->] (solver) -- (solution);
\draw[->] (solver) -- (wrong);
\draw[->] (hangs) -- (1,4) -- (human);
\draw[->] (wrong) -- (3,4) -- (human);
\end{tikzpicture}
\end{center}
\end{frame}

\begin{frame}
\frametitle{Dual Role of Model}
\begin{itemize}
\item Allows Human to Express Problem
\begin{itemize}
\item Close to Problem Domain
\item Constraints as Abstractions
\end{itemize}
\item Allows Solver to Execute
\begin{itemize}
\item Variables as Communication Mechanism
\item Constraints as Algorithms
\end{itemize}
\end{itemize}
\end{frame}

\begin{frame}
\frametitle{Modelling Frameworks}
\begin{itemize}
\item \href{https://www.minizinc.org/}{MiniZinc} (NICTA, Monash University, Australia)
\item \href{https://github.com/eomahony/Numberjack}{NumberJack} (Insight, Ireland) 
\item \href{https://www-users.york.ac.uk/peter.nightingale/savilerow//}{EssencePrime/SavilleRow} (UK) 
\item \href{https://cpmpy.readthedocs.io/en/latest/}{CPMpy} (KU Leuven)
\item Allow use of multiple back-end solvers
\item Compile model into variants for each solver
\item A priori solver independent model(CP, MIP, SAT)
\end{itemize}
\end{frame} 

\begin{frame}
\frametitle{Framework Process}
\begin{center}
\begin{tikzpicture}[xscale=2]
\node[shape=rectangle,fill=pantone127-4] (problem) at (2.5,5) {Problem};
\node (human) at (2.5,4) {Human};
\node[shape=rectangle,fill=pantone127-4] (model) at (2.5,3) {Model};
\node (compile) at (2.5,2) {Compile/Reformulate};
\node (solver1) at (1,1) {CP};
\node (solver2) at (2,1) {MIP};
\node (solver3) at (3,1) {SAT};
\node (solver4) at (4,1) {Other};
\node[shape=rectangle,fill=pantone157-8] (solution1) at (1,0) {Solution};
\node[shape=rectangle,fill=pantone157-8] (solution2) at (2,0) {Solution};
\node[shape=rectangle,fill=pantone157-8] (solution3) at (3,0) {Solution};
\node[shape=rectangle,fill=pantone157-8] (solution4) at (4,0) {Solution};
\draw[-] (problem) -- (human);
\draw[->] (human) -- (model);
\draw[-] (model) -- (compile);
\draw[-] (compile) -- (solver1);
\draw[-] (compile) -- (solver2);
\draw[-] (compile) -- (solver3);
\draw[-] (compile) -- (solver4);
\draw[->] (solver1) -- (solution1);
\draw[->] (solver2) -- (solution2);
\draw[->] (solver3) -- (solution3);
\draw[->] (solver4) -- (solution4);
\end{tikzpicture}
\end{center}
\end{frame}

% \begin{frame}
  % \frametitle{Do It Now!}
  % \begin{itemize}
  % \item Download and install Minizinc
  % \item \url{https://www.minizinc.org/}
  % \end{itemize}
% \end{frame}


\begin{frame}
\frametitle{Why use Puzzles as Examples?}
\begin{itemize}
\item Easy to understand the problem
\item Solvable by hand without specialized knowledge
\item Possible to compare automated to manual solving process
\end{itemize}

\begin{quote}
The puzzle, though inanimate, is presented as a solvable problem without lasting negative consequences, a very low-risk low-reward situation. By being a puzzle, the object is attempting to convince the user that it must be completed.
\end{quote}
{\tiny Source: \href{https://www.everydayrhetoric.com/2020/03/the-rhetoric-of-puzzles/}{Every Day Rhetoric}}
\end{frame}


\mode<all>{
\part{Basic Constraint Propagation}
}
\begin{frame}
\partpage
\end{frame}
\mode<all>{
\input{../sendmore/body}
}

\mode<all>{
\part{Global Constraints}
}
\begin{frame}
\partpage
\end{frame}
\mode<all>{
\input{../sudoku/body}
}

\clearpage
\mode<all>{
\part{Customizing Search}
}
\begin{frame}
\partpage
\end{frame}
\mode<all>{
\input{../nqueen/body}
}


%% \clearpage
%% \mode<all>{
%%   \part{Problem Decomposition}
%% }
%% \begin{frame}
%%   \partpage
%% \end{frame}
%% \mode<all>{
%%   \input{../party/body}
%%   }

%% \clearpage
%% \mode<all>{
%%   \part{More Global Constraints}
%% }
%% \begin{frame}
%%   \partpage
%% \end{frame}
%% \mode<all>{
%%   \input{../car/body}
%%   }

%% \clearpage
%% \mode<all>{
%%   \part{Methodology}
%% }
%% \begin{frame}
%%   \partpage
%% \end{frame}
%% \mode<all>{
%%   \input{../methodology/body}
%%   }

\clearpage
\mode<all>{
\part{What is missing?}
}
\begin{frame}
\partpage
\end{frame}

\begin{frame}
\frametitle{Many Specialized Topics}
\begin{itemize}
\item How to design efficient core engine
\item Hybrids with LP/MIP tools
\item Hybrids with SAT
\item Symmetry breaking
\item Use of MDD/BDD to encode sets of solutions
\item High level modelling tools
\item Debugging/visualization
\end{itemize}
\end{frame}

\begin{frame}
\frametitle{Reformulation}
\begin{itemize}
\item Just because the user has modelled it this way, it doesn't mean we have to solve it that way
\begin{itemize}
\item Replace some constraint(s) by other, equivalent constraints
\item Because we don't have that constraint in our system
\item For performance
\end{itemize}
\end{itemize}
\end{frame}

\begin{frame}
\frametitle{Learning}
\begin{itemize}
\item While solving the problem we can learn how to strengthen the model/search 
\begin{itemize}
\item Understand which constraints/method contribute to propagation and change schedule
\item Learn no-good constraints by explaining failure
\item Adapt search strategy based on search experience
\end{itemize}
\end{itemize}
\end{frame}

%% \begin{frame}
%% \frametitle{Refined Process}
%% \begin{center}
%% \begin{tikzpicture}
%% \node[shape=rectangle,fill=pantone127-4] (problem) at (3,7) {Problem};
%% \node (human) at (3,6) {Human};
%% \node[shape=rectangle,fill=pantone127-4] (model) at (3,5) {User Model};
%% \node (reformulation) at (3,4) {Reformulation};
%% \node[shape=rectangle,fill=pantone127-4] (implementation) at (3,3) {Implementation Model};
%% \node (solver) at (3,2) {Constraint Solver/Search};
%% \node[shape=rectangle,fill=pantone157-8] (solution) at (3,1) {Solution};
%% \draw[-] (problem) -- (human);
%% \draw[->] (human) -- (model);
%% \draw[-] (model) -- (reformulation);
%% \draw[->] (reformulation) -- (implementation);
%% \draw[-] (implementation) -- (solver);
%% \draw[->] (solver) -- (0,2) -- (0,3) -- (implementation);
%% \draw[->] (solver) -- (solution);
%% \end{tikzpicture}
%% \end{center}
%% \end{frame}

%\begin{frame}
%\frametitle{What is CP actually used for?}
%\url{http://hsimonis.wordpress.com}
%\includegraphics[width=10cm]{../introduction/images/blog}
%\end{frame}

%% \begin{frame}
%% \frametitle{Exercises}
%% \begin{itemize}
%% \item Install Constraint Solver of your Choice
%% \item MiniZinc (MiningZinc) will be used in later sessions
%% \item Run Sendmoremoney, Sudoku, Queens examples
%% \item Can you choose which propagator to use for alldifferent?
%% \item Can you use different search strategies?
%% \end{itemize}
%% \end{frame}

\begin{frame}
\frametitle{More Learning Resources}
\begin{itemize}
\item Survey of Methods, Resources, and Formats for Teaching Constraint Programming
\begin{itemize}
\item by Tejas Santanam, Helmut Simonis
\item \url{https://doi.org/10.48550/arXiv.2403.12717}
\item Based on survey of community for WTCP 2023
\item \url{https://hsimonis.github.io/WTCP2023/}
\end{itemize}
\end{itemize}
\end{frame}


%%   }

%% \clearpage
%% \mode<all>{
%%   \part{Methodology}
%% }
%% \begin{frame}
%%   \partpage
%% \end{frame}
%% \mode<all>{
%%   \begin{frame}
\frametitle{Objectives}
\begin{itemize}
\item Overview of Core Constraint Programming
\item Three Main Concepts
\begin{itemize}
\item Constraint Propagation
\item Global Constraints
\item Customizing Search
\end{itemize}
%\item Get Some Experience with MiniZinc
\item Topics will be treated in more detail in later parts of the school
\item Based on Examples, not Formal Description
\end{itemize}
\end{frame}

\begin{frame}
\frametitle{Outline}
\begin{itemize}
\item Why Constraint Programming?
\item Constraint Propagation
\item Global Constraints
\item Customizing Search
%\item Choices in Modelling
%\item Decomposition
%\item More Global Constraints
%\item Methodology
%\item What is missing?
\end{itemize}
\end{frame}

% \begin{frame}
  % \frametitle{Using MiniZinc IDE}
  % \begin{itemize}
  % \item Developed in the Australian NICTA project
  % \item Maintained by Monash University
  % \item Modelling tool with multiple back-end solvers
  % \item Available from \url{https://www.minizinc.org/}
  % \end{itemize}
% \end{frame}


% \begin{frame}
% \frametitle{Examples in ECLiPSe}
% \begin{itemize}
% \item Open sourced constraint programming language
% \item Development goes back to 1985
% \item ECRC, ICL, IC-Parc, PTL, Cisco
% \item \url{https://eclipseclp.org/}
% \item Specialities
% \begin{itemize}
% \item Develop new solvers for specific domains
% \item Integration with MIP
% \end{itemize}
% \item Not included in bundled MiniZinc IDE
% \item Specialized visualization tools used here
  % \begin{itemize}
  % \item CP-Viz, Simonis et al. 2010
    % \end{itemize}
% \end{itemize}
% \end{frame}

\begin{frame}
\frametitle{Tutorial Based on ECLiPSe ELearning Course}
\begin{itemize}
\item Self-study course in constraint programming
\item Supported by Cisco Systems and Silicon Valley Community Foundation
\item Multi-media format, video lectures, slides, handout etc
\item {\small \url{https://eclipseclp.org/ELearning/index.html}}
\end{itemize}
\end{frame}

\begin{frame}
\frametitle{Also Part of CRT-AI Constraint Week}
\begin{itemize}
\item Annual one week course on CP and Optimization in Ireland
\item Part of national training program for PhD students in AI
\item \url{https://www.crt-ai.ie/}
\end{itemize}
\end{frame}



\begin{frame}
\frametitle{Constraint Programming - in a nutshell}
\begin{itemize}
\item Declarative description of problems with
\begin{itemize}
\item {\em Variables} which range over (finite) sets of values
\item {\em Constraints} over subsets of variables which restrict possible value combinations
\item A {\em solution} is a value assignment which satisfies all constraints
\end{itemize}

\item Constraint propagation/reasoning
\begin{itemize}
\item Removing inconsistent values for variables
\item Detect failure if constraint can not be satisfied
\item Interaction of constraints via shared variables
\item Incomplete
\end{itemize}

\item Search
\begin{itemize}
\item User controlled assignment of values to variables
\item Each step triggers constraint propagation 
\end{itemize}
\item Different domains require/allow different methods
\end{itemize}
\end{frame}

\begin{frame}
  \frametitle{Constraint Programming is Different}
  \begin{itemize}
  \item Declarative Programming
    \begin{itemize}
    \item Concentrate on what you want
      \item Not how to get there
      \item Program != Algorithm
      \item Program = Model
    \end{itemize}
    \item Applied to Combinatorial Problems
      \begin{itemize}
        \item No complete polynomial algorithms known (exist?)
        \item CP less ad-hoc than heuristics
        \item Models can evolve
  \end{itemize}
  \end{itemize}
  \end{frame}
    
\begin{frame}
  \frametitle{A Subtractive Process}
  \begin{textblock}{4}(8,-3)
    \includegraphics[width=4cm]{../introduction/images/stages}
  \end{textblock}
  \vfill
  \begin{quote}
    ``Oh, bosh, as Mr. Ruskin says. Sculpture, per se, is the simplest thing in the world. All you have to do is to take a big chunk of marble and a hammer and chisel, make up your mind what you are about to create and chip off all the marble you don't want.''-Paris Gaulois.
  \end{quote}
  
  {\tiny Source: \url{https://quoteinvestigator.com/2014/06/22/chip-away/}}
\end{frame}

\begin{frame}
\frametitle{Basic Process}
\begin{center}
\begin{tikzpicture}
\node[shape=rectangle,fill=pantone127-4] (problem) at (2,5) {Problem};
\node (human) at (2,4) {Human};
\node[shape=rectangle,fill=pantone127-4] (model) at (2,3) {Model};
\node (solver) at (2,2) {Constraint Solver/Search};
\node[shape=rectangle,fill=pantone157-8] (solution) at (2,1) {Solution};
\draw[-] (problem) -- (human);
\draw[->] (human) -- (model);
\draw[-] (model) -- (solver);
\draw[->] (solver) -- (solution);
\end{tikzpicture}
\end{center}
\end{frame}

\begin{frame}
\frametitle{More Realistic}
\begin{center}
\begin{tikzpicture}[xscale=4.0]
\node[shape=rectangle,fill=pantone127-4] (problem) at (2,5) {Problem};
\node (human) at (2,4) {Human};
\node[shape=rectangle,fill=pantone127-4] (model) at (2,3) {Model};
\node (solver) at (2,2) {Constraint Solver/Search};
\node[shape=rectangle,fill=pantone24-8] (hangs) at (1,1) {Hangs};
\node[shape=rectangle,fill=pantone157-8] (solution) at (2,1) {Solution};
\node[shape=rectangle,fill=pantone24-8] (wrong) at (3,1) {Wrong Solution};
\draw[-] (problem) -- (human);
\draw[->] (human) -- (model);
\draw[-] (model) -- (solver);
\draw[->] (solver) -- (hangs);
\draw[->] (solver) -- (solution);
\draw[->] (solver) -- (wrong);
\draw[->] (hangs) -- (1,4) -- (human);
\draw[->] (wrong) -- (3,4) -- (human);
\end{tikzpicture}
\end{center}
\end{frame}

\begin{frame}
\frametitle{Dual Role of Model}
\begin{itemize}
\item Allows Human to Express Problem
\begin{itemize}
\item Close to Problem Domain
\item Constraints as Abstractions
\end{itemize}
\item Allows Solver to Execute
\begin{itemize}
\item Variables as Communication Mechanism
\item Constraints as Algorithms
\end{itemize}
\end{itemize}
\end{frame}

\begin{frame}
\frametitle{Modelling Frameworks}
\begin{itemize}
\item \href{https://www.minizinc.org/}{MiniZinc} (NICTA, Monash University, Australia)
\item \href{https://github.com/eomahony/Numberjack}{NumberJack} (Insight, Ireland) 
\item \href{https://www-users.york.ac.uk/peter.nightingale/savilerow//}{EssencePrime/SavilleRow} (UK) 
\item \href{https://cpmpy.readthedocs.io/en/latest/}{CPMpy} (KU Leuven)
\item Allow use of multiple back-end solvers
\item Compile model into variants for each solver
\item A priori solver independent model(CP, MIP, SAT)
\end{itemize}
\end{frame} 

\begin{frame}
\frametitle{Framework Process}
\begin{center}
\begin{tikzpicture}[xscale=2]
\node[shape=rectangle,fill=pantone127-4] (problem) at (2.5,5) {Problem};
\node (human) at (2.5,4) {Human};
\node[shape=rectangle,fill=pantone127-4] (model) at (2.5,3) {Model};
\node (compile) at (2.5,2) {Compile/Reformulate};
\node (solver1) at (1,1) {CP};
\node (solver2) at (2,1) {MIP};
\node (solver3) at (3,1) {SAT};
\node (solver4) at (4,1) {Other};
\node[shape=rectangle,fill=pantone157-8] (solution1) at (1,0) {Solution};
\node[shape=rectangle,fill=pantone157-8] (solution2) at (2,0) {Solution};
\node[shape=rectangle,fill=pantone157-8] (solution3) at (3,0) {Solution};
\node[shape=rectangle,fill=pantone157-8] (solution4) at (4,0) {Solution};
\draw[-] (problem) -- (human);
\draw[->] (human) -- (model);
\draw[-] (model) -- (compile);
\draw[-] (compile) -- (solver1);
\draw[-] (compile) -- (solver2);
\draw[-] (compile) -- (solver3);
\draw[-] (compile) -- (solver4);
\draw[->] (solver1) -- (solution1);
\draw[->] (solver2) -- (solution2);
\draw[->] (solver3) -- (solution3);
\draw[->] (solver4) -- (solution4);
\end{tikzpicture}
\end{center}
\end{frame}

% \begin{frame}
  % \frametitle{Do It Now!}
  % \begin{itemize}
  % \item Download and install Minizinc
  % \item \url{https://www.minizinc.org/}
  % \end{itemize}
% \end{frame}


\begin{frame}
\frametitle{Why use Puzzles as Examples?}
\begin{itemize}
\item Easy to understand the problem
\item Solvable by hand without specialized knowledge
\item Possible to compare automated to manual solving process
\end{itemize}

\begin{quote}
The puzzle, though inanimate, is presented as a solvable problem without lasting negative consequences, a very low-risk low-reward situation. By being a puzzle, the object is attempting to convince the user that it must be completed.
\end{quote}
{\tiny Source: \href{https://www.everydayrhetoric.com/2020/03/the-rhetoric-of-puzzles/}{Every Day Rhetoric}}
\end{frame}


\mode<all>{
\part{Basic Constraint Propagation}
}
\begin{frame}
\partpage
\end{frame}
\mode<all>{
\input{../sendmore/body}
}

\mode<all>{
\part{Global Constraints}
}
\begin{frame}
\partpage
\end{frame}
\mode<all>{
\input{../sudoku/body}
}

\clearpage
\mode<all>{
\part{Customizing Search}
}
\begin{frame}
\partpage
\end{frame}
\mode<all>{
\input{../nqueen/body}
}


%% \clearpage
%% \mode<all>{
%%   \part{Problem Decomposition}
%% }
%% \begin{frame}
%%   \partpage
%% \end{frame}
%% \mode<all>{
%%   \input{../party/body}
%%   }

%% \clearpage
%% \mode<all>{
%%   \part{More Global Constraints}
%% }
%% \begin{frame}
%%   \partpage
%% \end{frame}
%% \mode<all>{
%%   \input{../car/body}
%%   }

%% \clearpage
%% \mode<all>{
%%   \part{Methodology}
%% }
%% \begin{frame}
%%   \partpage
%% \end{frame}
%% \mode<all>{
%%   \input{../methodology/body}
%%   }

\clearpage
\mode<all>{
\part{What is missing?}
}
\begin{frame}
\partpage
\end{frame}

\begin{frame}
\frametitle{Many Specialized Topics}
\begin{itemize}
\item How to design efficient core engine
\item Hybrids with LP/MIP tools
\item Hybrids with SAT
\item Symmetry breaking
\item Use of MDD/BDD to encode sets of solutions
\item High level modelling tools
\item Debugging/visualization
\end{itemize}
\end{frame}

\begin{frame}
\frametitle{Reformulation}
\begin{itemize}
\item Just because the user has modelled it this way, it doesn't mean we have to solve it that way
\begin{itemize}
\item Replace some constraint(s) by other, equivalent constraints
\item Because we don't have that constraint in our system
\item For performance
\end{itemize}
\end{itemize}
\end{frame}

\begin{frame}
\frametitle{Learning}
\begin{itemize}
\item While solving the problem we can learn how to strengthen the model/search 
\begin{itemize}
\item Understand which constraints/method contribute to propagation and change schedule
\item Learn no-good constraints by explaining failure
\item Adapt search strategy based on search experience
\end{itemize}
\end{itemize}
\end{frame}

%% \begin{frame}
%% \frametitle{Refined Process}
%% \begin{center}
%% \begin{tikzpicture}
%% \node[shape=rectangle,fill=pantone127-4] (problem) at (3,7) {Problem};
%% \node (human) at (3,6) {Human};
%% \node[shape=rectangle,fill=pantone127-4] (model) at (3,5) {User Model};
%% \node (reformulation) at (3,4) {Reformulation};
%% \node[shape=rectangle,fill=pantone127-4] (implementation) at (3,3) {Implementation Model};
%% \node (solver) at (3,2) {Constraint Solver/Search};
%% \node[shape=rectangle,fill=pantone157-8] (solution) at (3,1) {Solution};
%% \draw[-] (problem) -- (human);
%% \draw[->] (human) -- (model);
%% \draw[-] (model) -- (reformulation);
%% \draw[->] (reformulation) -- (implementation);
%% \draw[-] (implementation) -- (solver);
%% \draw[->] (solver) -- (0,2) -- (0,3) -- (implementation);
%% \draw[->] (solver) -- (solution);
%% \end{tikzpicture}
%% \end{center}
%% \end{frame}

%\begin{frame}
%\frametitle{What is CP actually used for?}
%\url{http://hsimonis.wordpress.com}
%\includegraphics[width=10cm]{../introduction/images/blog}
%\end{frame}

%% \begin{frame}
%% \frametitle{Exercises}
%% \begin{itemize}
%% \item Install Constraint Solver of your Choice
%% \item MiniZinc (MiningZinc) will be used in later sessions
%% \item Run Sendmoremoney, Sudoku, Queens examples
%% \item Can you choose which propagator to use for alldifferent?
%% \item Can you use different search strategies?
%% \end{itemize}
%% \end{frame}

\begin{frame}
\frametitle{More Learning Resources}
\begin{itemize}
\item Survey of Methods, Resources, and Formats for Teaching Constraint Programming
\begin{itemize}
\item by Tejas Santanam, Helmut Simonis
\item \url{https://doi.org/10.48550/arXiv.2403.12717}
\item Based on survey of community for WTCP 2023
\item \url{https://hsimonis.github.io/WTCP2023/}
\end{itemize}
\end{itemize}
\end{frame}


%%   }

\clearpage
\mode<all>{
\part{What is missing?}
}
\begin{frame}
\partpage
\end{frame}

\begin{frame}
\frametitle{Many Specialized Topics}
\begin{itemize}
\item How to design efficient core engine
\item Hybrids with LP/MIP tools
\item Hybrids with SAT
\item Symmetry breaking
\item Use of MDD/BDD to encode sets of solutions
\item High level modelling tools
\item Debugging/visualization
\end{itemize}
\end{frame}

\begin{frame}
\frametitle{Reformulation}
\begin{itemize}
\item Just because the user has modelled it this way, it doesn't mean we have to solve it that way
\begin{itemize}
\item Replace some constraint(s) by other, equivalent constraints
\item Because we don't have that constraint in our system
\item For performance
\end{itemize}
\end{itemize}
\end{frame}

\begin{frame}
\frametitle{Learning}
\begin{itemize}
\item While solving the problem we can learn how to strengthen the model/search 
\begin{itemize}
\item Understand which constraints/method contribute to propagation and change schedule
\item Learn no-good constraints by explaining failure
\item Adapt search strategy based on search experience
\end{itemize}
\end{itemize}
\end{frame}

%% \begin{frame}
%% \frametitle{Refined Process}
%% \begin{center}
%% \begin{tikzpicture}
%% \node[shape=rectangle,fill=pantone127-4] (problem) at (3,7) {Problem};
%% \node (human) at (3,6) {Human};
%% \node[shape=rectangle,fill=pantone127-4] (model) at (3,5) {User Model};
%% \node (reformulation) at (3,4) {Reformulation};
%% \node[shape=rectangle,fill=pantone127-4] (implementation) at (3,3) {Implementation Model};
%% \node (solver) at (3,2) {Constraint Solver/Search};
%% \node[shape=rectangle,fill=pantone157-8] (solution) at (3,1) {Solution};
%% \draw[-] (problem) -- (human);
%% \draw[->] (human) -- (model);
%% \draw[-] (model) -- (reformulation);
%% \draw[->] (reformulation) -- (implementation);
%% \draw[-] (implementation) -- (solver);
%% \draw[->] (solver) -- (0,2) -- (0,3) -- (implementation);
%% \draw[->] (solver) -- (solution);
%% \end{tikzpicture}
%% \end{center}
%% \end{frame}

%\begin{frame}
%\frametitle{What is CP actually used for?}
%\url{http://hsimonis.wordpress.com}
%\includegraphics[width=10cm]{../introduction/images/blog}
%\end{frame}

%% \begin{frame}
%% \frametitle{Exercises}
%% \begin{itemize}
%% \item Install Constraint Solver of your Choice
%% \item MiniZinc (MiningZinc) will be used in later sessions
%% \item Run Sendmoremoney, Sudoku, Queens examples
%% \item Can you choose which propagator to use for alldifferent?
%% \item Can you use different search strategies?
%% \end{itemize}
%% \end{frame}

\begin{frame}
\frametitle{More Learning Resources}
\begin{itemize}
\item Survey of Methods, Resources, and Formats for Teaching Constraint Programming
\begin{itemize}
\item by Tejas Santanam, Helmut Simonis
\item \url{https://doi.org/10.48550/arXiv.2403.12717}
\item Based on survey of community for WTCP 2023
\item \url{https://hsimonis.github.io/WTCP2023/}
\end{itemize}
\end{itemize}
\end{frame}


}


%% \clearpage
%% \mode<all>{
%%   \part{Problem Decomposition}
%% }
%% \begin{frame}
%%   \partpage
%% \end{frame}
%% \mode<all>{
%%   \begin{frame}
\frametitle{Objectives}
\begin{itemize}
\item Overview of Core Constraint Programming
\item Three Main Concepts
\begin{itemize}
\item Constraint Propagation
\item Global Constraints
\item Customizing Search
\end{itemize}
%\item Get Some Experience with MiniZinc
\item Topics will be treated in more detail in later parts of the school
\item Based on Examples, not Formal Description
\end{itemize}
\end{frame}

\begin{frame}
\frametitle{Outline}
\begin{itemize}
\item Why Constraint Programming?
\item Constraint Propagation
\item Global Constraints
\item Customizing Search
%\item Choices in Modelling
%\item Decomposition
%\item More Global Constraints
%\item Methodology
%\item What is missing?
\end{itemize}
\end{frame}

% \begin{frame}
  % \frametitle{Using MiniZinc IDE}
  % \begin{itemize}
  % \item Developed in the Australian NICTA project
  % \item Maintained by Monash University
  % \item Modelling tool with multiple back-end solvers
  % \item Available from \url{https://www.minizinc.org/}
  % \end{itemize}
% \end{frame}


% \begin{frame}
% \frametitle{Examples in ECLiPSe}
% \begin{itemize}
% \item Open sourced constraint programming language
% \item Development goes back to 1985
% \item ECRC, ICL, IC-Parc, PTL, Cisco
% \item \url{https://eclipseclp.org/}
% \item Specialities
% \begin{itemize}
% \item Develop new solvers for specific domains
% \item Integration with MIP
% \end{itemize}
% \item Not included in bundled MiniZinc IDE
% \item Specialized visualization tools used here
  % \begin{itemize}
  % \item CP-Viz, Simonis et al. 2010
    % \end{itemize}
% \end{itemize}
% \end{frame}

\begin{frame}
\frametitle{Tutorial Based on ECLiPSe ELearning Course}
\begin{itemize}
\item Self-study course in constraint programming
\item Supported by Cisco Systems and Silicon Valley Community Foundation
\item Multi-media format, video lectures, slides, handout etc
\item {\small \url{https://eclipseclp.org/ELearning/index.html}}
\end{itemize}
\end{frame}

\begin{frame}
\frametitle{Also Part of CRT-AI Constraint Week}
\begin{itemize}
\item Annual one week course on CP and Optimization in Ireland
\item Part of national training program for PhD students in AI
\item \url{https://www.crt-ai.ie/}
\end{itemize}
\end{frame}



\begin{frame}
\frametitle{Constraint Programming - in a nutshell}
\begin{itemize}
\item Declarative description of problems with
\begin{itemize}
\item {\em Variables} which range over (finite) sets of values
\item {\em Constraints} over subsets of variables which restrict possible value combinations
\item A {\em solution} is a value assignment which satisfies all constraints
\end{itemize}

\item Constraint propagation/reasoning
\begin{itemize}
\item Removing inconsistent values for variables
\item Detect failure if constraint can not be satisfied
\item Interaction of constraints via shared variables
\item Incomplete
\end{itemize}

\item Search
\begin{itemize}
\item User controlled assignment of values to variables
\item Each step triggers constraint propagation 
\end{itemize}
\item Different domains require/allow different methods
\end{itemize}
\end{frame}

\begin{frame}
  \frametitle{Constraint Programming is Different}
  \begin{itemize}
  \item Declarative Programming
    \begin{itemize}
    \item Concentrate on what you want
      \item Not how to get there
      \item Program != Algorithm
      \item Program = Model
    \end{itemize}
    \item Applied to Combinatorial Problems
      \begin{itemize}
        \item No complete polynomial algorithms known (exist?)
        \item CP less ad-hoc than heuristics
        \item Models can evolve
  \end{itemize}
  \end{itemize}
  \end{frame}
    
\begin{frame}
  \frametitle{A Subtractive Process}
  \begin{textblock}{4}(8,-3)
    \includegraphics[width=4cm]{../introduction/images/stages}
  \end{textblock}
  \vfill
  \begin{quote}
    ``Oh, bosh, as Mr. Ruskin says. Sculpture, per se, is the simplest thing in the world. All you have to do is to take a big chunk of marble and a hammer and chisel, make up your mind what you are about to create and chip off all the marble you don't want.''-Paris Gaulois.
  \end{quote}
  
  {\tiny Source: \url{https://quoteinvestigator.com/2014/06/22/chip-away/}}
\end{frame}

\begin{frame}
\frametitle{Basic Process}
\begin{center}
\begin{tikzpicture}
\node[shape=rectangle,fill=pantone127-4] (problem) at (2,5) {Problem};
\node (human) at (2,4) {Human};
\node[shape=rectangle,fill=pantone127-4] (model) at (2,3) {Model};
\node (solver) at (2,2) {Constraint Solver/Search};
\node[shape=rectangle,fill=pantone157-8] (solution) at (2,1) {Solution};
\draw[-] (problem) -- (human);
\draw[->] (human) -- (model);
\draw[-] (model) -- (solver);
\draw[->] (solver) -- (solution);
\end{tikzpicture}
\end{center}
\end{frame}

\begin{frame}
\frametitle{More Realistic}
\begin{center}
\begin{tikzpicture}[xscale=4.0]
\node[shape=rectangle,fill=pantone127-4] (problem) at (2,5) {Problem};
\node (human) at (2,4) {Human};
\node[shape=rectangle,fill=pantone127-4] (model) at (2,3) {Model};
\node (solver) at (2,2) {Constraint Solver/Search};
\node[shape=rectangle,fill=pantone24-8] (hangs) at (1,1) {Hangs};
\node[shape=rectangle,fill=pantone157-8] (solution) at (2,1) {Solution};
\node[shape=rectangle,fill=pantone24-8] (wrong) at (3,1) {Wrong Solution};
\draw[-] (problem) -- (human);
\draw[->] (human) -- (model);
\draw[-] (model) -- (solver);
\draw[->] (solver) -- (hangs);
\draw[->] (solver) -- (solution);
\draw[->] (solver) -- (wrong);
\draw[->] (hangs) -- (1,4) -- (human);
\draw[->] (wrong) -- (3,4) -- (human);
\end{tikzpicture}
\end{center}
\end{frame}

\begin{frame}
\frametitle{Dual Role of Model}
\begin{itemize}
\item Allows Human to Express Problem
\begin{itemize}
\item Close to Problem Domain
\item Constraints as Abstractions
\end{itemize}
\item Allows Solver to Execute
\begin{itemize}
\item Variables as Communication Mechanism
\item Constraints as Algorithms
\end{itemize}
\end{itemize}
\end{frame}

\begin{frame}
\frametitle{Modelling Frameworks}
\begin{itemize}
\item \href{https://www.minizinc.org/}{MiniZinc} (NICTA, Monash University, Australia)
\item \href{https://github.com/eomahony/Numberjack}{NumberJack} (Insight, Ireland) 
\item \href{https://www-users.york.ac.uk/peter.nightingale/savilerow//}{EssencePrime/SavilleRow} (UK) 
\item \href{https://cpmpy.readthedocs.io/en/latest/}{CPMpy} (KU Leuven)
\item Allow use of multiple back-end solvers
\item Compile model into variants for each solver
\item A priori solver independent model(CP, MIP, SAT)
\end{itemize}
\end{frame} 

\begin{frame}
\frametitle{Framework Process}
\begin{center}
\begin{tikzpicture}[xscale=2]
\node[shape=rectangle,fill=pantone127-4] (problem) at (2.5,5) {Problem};
\node (human) at (2.5,4) {Human};
\node[shape=rectangle,fill=pantone127-4] (model) at (2.5,3) {Model};
\node (compile) at (2.5,2) {Compile/Reformulate};
\node (solver1) at (1,1) {CP};
\node (solver2) at (2,1) {MIP};
\node (solver3) at (3,1) {SAT};
\node (solver4) at (4,1) {Other};
\node[shape=rectangle,fill=pantone157-8] (solution1) at (1,0) {Solution};
\node[shape=rectangle,fill=pantone157-8] (solution2) at (2,0) {Solution};
\node[shape=rectangle,fill=pantone157-8] (solution3) at (3,0) {Solution};
\node[shape=rectangle,fill=pantone157-8] (solution4) at (4,0) {Solution};
\draw[-] (problem) -- (human);
\draw[->] (human) -- (model);
\draw[-] (model) -- (compile);
\draw[-] (compile) -- (solver1);
\draw[-] (compile) -- (solver2);
\draw[-] (compile) -- (solver3);
\draw[-] (compile) -- (solver4);
\draw[->] (solver1) -- (solution1);
\draw[->] (solver2) -- (solution2);
\draw[->] (solver3) -- (solution3);
\draw[->] (solver4) -- (solution4);
\end{tikzpicture}
\end{center}
\end{frame}

% \begin{frame}
  % \frametitle{Do It Now!}
  % \begin{itemize}
  % \item Download and install Minizinc
  % \item \url{https://www.minizinc.org/}
  % \end{itemize}
% \end{frame}


\begin{frame}
\frametitle{Why use Puzzles as Examples?}
\begin{itemize}
\item Easy to understand the problem
\item Solvable by hand without specialized knowledge
\item Possible to compare automated to manual solving process
\end{itemize}

\begin{quote}
The puzzle, though inanimate, is presented as a solvable problem without lasting negative consequences, a very low-risk low-reward situation. By being a puzzle, the object is attempting to convince the user that it must be completed.
\end{quote}
{\tiny Source: \href{https://www.everydayrhetoric.com/2020/03/the-rhetoric-of-puzzles/}{Every Day Rhetoric}}
\end{frame}


\mode<all>{
\part{Basic Constraint Propagation}
}
\begin{frame}
\partpage
\end{frame}
\mode<all>{
\begin{frame}
\frametitle{Objectives}
\begin{itemize}
\item Overview of Core Constraint Programming
\item Three Main Concepts
\begin{itemize}
\item Constraint Propagation
\item Global Constraints
\item Customizing Search
\end{itemize}
%\item Get Some Experience with MiniZinc
\item Topics will be treated in more detail in later parts of the school
\item Based on Examples, not Formal Description
\end{itemize}
\end{frame}

\begin{frame}
\frametitle{Outline}
\begin{itemize}
\item Why Constraint Programming?
\item Constraint Propagation
\item Global Constraints
\item Customizing Search
%\item Choices in Modelling
%\item Decomposition
%\item More Global Constraints
%\item Methodology
%\item What is missing?
\end{itemize}
\end{frame}

% \begin{frame}
  % \frametitle{Using MiniZinc IDE}
  % \begin{itemize}
  % \item Developed in the Australian NICTA project
  % \item Maintained by Monash University
  % \item Modelling tool with multiple back-end solvers
  % \item Available from \url{https://www.minizinc.org/}
  % \end{itemize}
% \end{frame}


% \begin{frame}
% \frametitle{Examples in ECLiPSe}
% \begin{itemize}
% \item Open sourced constraint programming language
% \item Development goes back to 1985
% \item ECRC, ICL, IC-Parc, PTL, Cisco
% \item \url{https://eclipseclp.org/}
% \item Specialities
% \begin{itemize}
% \item Develop new solvers for specific domains
% \item Integration with MIP
% \end{itemize}
% \item Not included in bundled MiniZinc IDE
% \item Specialized visualization tools used here
  % \begin{itemize}
  % \item CP-Viz, Simonis et al. 2010
    % \end{itemize}
% \end{itemize}
% \end{frame}

\begin{frame}
\frametitle{Tutorial Based on ECLiPSe ELearning Course}
\begin{itemize}
\item Self-study course in constraint programming
\item Supported by Cisco Systems and Silicon Valley Community Foundation
\item Multi-media format, video lectures, slides, handout etc
\item {\small \url{https://eclipseclp.org/ELearning/index.html}}
\end{itemize}
\end{frame}

\begin{frame}
\frametitle{Also Part of CRT-AI Constraint Week}
\begin{itemize}
\item Annual one week course on CP and Optimization in Ireland
\item Part of national training program for PhD students in AI
\item \url{https://www.crt-ai.ie/}
\end{itemize}
\end{frame}



\begin{frame}
\frametitle{Constraint Programming - in a nutshell}
\begin{itemize}
\item Declarative description of problems with
\begin{itemize}
\item {\em Variables} which range over (finite) sets of values
\item {\em Constraints} over subsets of variables which restrict possible value combinations
\item A {\em solution} is a value assignment which satisfies all constraints
\end{itemize}

\item Constraint propagation/reasoning
\begin{itemize}
\item Removing inconsistent values for variables
\item Detect failure if constraint can not be satisfied
\item Interaction of constraints via shared variables
\item Incomplete
\end{itemize}

\item Search
\begin{itemize}
\item User controlled assignment of values to variables
\item Each step triggers constraint propagation 
\end{itemize}
\item Different domains require/allow different methods
\end{itemize}
\end{frame}

\begin{frame}
  \frametitle{Constraint Programming is Different}
  \begin{itemize}
  \item Declarative Programming
    \begin{itemize}
    \item Concentrate on what you want
      \item Not how to get there
      \item Program != Algorithm
      \item Program = Model
    \end{itemize}
    \item Applied to Combinatorial Problems
      \begin{itemize}
        \item No complete polynomial algorithms known (exist?)
        \item CP less ad-hoc than heuristics
        \item Models can evolve
  \end{itemize}
  \end{itemize}
  \end{frame}
    
\begin{frame}
  \frametitle{A Subtractive Process}
  \begin{textblock}{4}(8,-3)
    \includegraphics[width=4cm]{../introduction/images/stages}
  \end{textblock}
  \vfill
  \begin{quote}
    ``Oh, bosh, as Mr. Ruskin says. Sculpture, per se, is the simplest thing in the world. All you have to do is to take a big chunk of marble and a hammer and chisel, make up your mind what you are about to create and chip off all the marble you don't want.''-Paris Gaulois.
  \end{quote}
  
  {\tiny Source: \url{https://quoteinvestigator.com/2014/06/22/chip-away/}}
\end{frame}

\begin{frame}
\frametitle{Basic Process}
\begin{center}
\begin{tikzpicture}
\node[shape=rectangle,fill=pantone127-4] (problem) at (2,5) {Problem};
\node (human) at (2,4) {Human};
\node[shape=rectangle,fill=pantone127-4] (model) at (2,3) {Model};
\node (solver) at (2,2) {Constraint Solver/Search};
\node[shape=rectangle,fill=pantone157-8] (solution) at (2,1) {Solution};
\draw[-] (problem) -- (human);
\draw[->] (human) -- (model);
\draw[-] (model) -- (solver);
\draw[->] (solver) -- (solution);
\end{tikzpicture}
\end{center}
\end{frame}

\begin{frame}
\frametitle{More Realistic}
\begin{center}
\begin{tikzpicture}[xscale=4.0]
\node[shape=rectangle,fill=pantone127-4] (problem) at (2,5) {Problem};
\node (human) at (2,4) {Human};
\node[shape=rectangle,fill=pantone127-4] (model) at (2,3) {Model};
\node (solver) at (2,2) {Constraint Solver/Search};
\node[shape=rectangle,fill=pantone24-8] (hangs) at (1,1) {Hangs};
\node[shape=rectangle,fill=pantone157-8] (solution) at (2,1) {Solution};
\node[shape=rectangle,fill=pantone24-8] (wrong) at (3,1) {Wrong Solution};
\draw[-] (problem) -- (human);
\draw[->] (human) -- (model);
\draw[-] (model) -- (solver);
\draw[->] (solver) -- (hangs);
\draw[->] (solver) -- (solution);
\draw[->] (solver) -- (wrong);
\draw[->] (hangs) -- (1,4) -- (human);
\draw[->] (wrong) -- (3,4) -- (human);
\end{tikzpicture}
\end{center}
\end{frame}

\begin{frame}
\frametitle{Dual Role of Model}
\begin{itemize}
\item Allows Human to Express Problem
\begin{itemize}
\item Close to Problem Domain
\item Constraints as Abstractions
\end{itemize}
\item Allows Solver to Execute
\begin{itemize}
\item Variables as Communication Mechanism
\item Constraints as Algorithms
\end{itemize}
\end{itemize}
\end{frame}

\begin{frame}
\frametitle{Modelling Frameworks}
\begin{itemize}
\item \href{https://www.minizinc.org/}{MiniZinc} (NICTA, Monash University, Australia)
\item \href{https://github.com/eomahony/Numberjack}{NumberJack} (Insight, Ireland) 
\item \href{https://www-users.york.ac.uk/peter.nightingale/savilerow//}{EssencePrime/SavilleRow} (UK) 
\item \href{https://cpmpy.readthedocs.io/en/latest/}{CPMpy} (KU Leuven)
\item Allow use of multiple back-end solvers
\item Compile model into variants for each solver
\item A priori solver independent model(CP, MIP, SAT)
\end{itemize}
\end{frame} 

\begin{frame}
\frametitle{Framework Process}
\begin{center}
\begin{tikzpicture}[xscale=2]
\node[shape=rectangle,fill=pantone127-4] (problem) at (2.5,5) {Problem};
\node (human) at (2.5,4) {Human};
\node[shape=rectangle,fill=pantone127-4] (model) at (2.5,3) {Model};
\node (compile) at (2.5,2) {Compile/Reformulate};
\node (solver1) at (1,1) {CP};
\node (solver2) at (2,1) {MIP};
\node (solver3) at (3,1) {SAT};
\node (solver4) at (4,1) {Other};
\node[shape=rectangle,fill=pantone157-8] (solution1) at (1,0) {Solution};
\node[shape=rectangle,fill=pantone157-8] (solution2) at (2,0) {Solution};
\node[shape=rectangle,fill=pantone157-8] (solution3) at (3,0) {Solution};
\node[shape=rectangle,fill=pantone157-8] (solution4) at (4,0) {Solution};
\draw[-] (problem) -- (human);
\draw[->] (human) -- (model);
\draw[-] (model) -- (compile);
\draw[-] (compile) -- (solver1);
\draw[-] (compile) -- (solver2);
\draw[-] (compile) -- (solver3);
\draw[-] (compile) -- (solver4);
\draw[->] (solver1) -- (solution1);
\draw[->] (solver2) -- (solution2);
\draw[->] (solver3) -- (solution3);
\draw[->] (solver4) -- (solution4);
\end{tikzpicture}
\end{center}
\end{frame}

% \begin{frame}
  % \frametitle{Do It Now!}
  % \begin{itemize}
  % \item Download and install Minizinc
  % \item \url{https://www.minizinc.org/}
  % \end{itemize}
% \end{frame}


\begin{frame}
\frametitle{Why use Puzzles as Examples?}
\begin{itemize}
\item Easy to understand the problem
\item Solvable by hand without specialized knowledge
\item Possible to compare automated to manual solving process
\end{itemize}

\begin{quote}
The puzzle, though inanimate, is presented as a solvable problem without lasting negative consequences, a very low-risk low-reward situation. By being a puzzle, the object is attempting to convince the user that it must be completed.
\end{quote}
{\tiny Source: \href{https://www.everydayrhetoric.com/2020/03/the-rhetoric-of-puzzles/}{Every Day Rhetoric}}
\end{frame}


\mode<all>{
\part{Basic Constraint Propagation}
}
\begin{frame}
\partpage
\end{frame}
\mode<all>{
\input{../sendmore/body}
}

\mode<all>{
\part{Global Constraints}
}
\begin{frame}
\partpage
\end{frame}
\mode<all>{
\input{../sudoku/body}
}

\clearpage
\mode<all>{
\part{Customizing Search}
}
\begin{frame}
\partpage
\end{frame}
\mode<all>{
\input{../nqueen/body}
}


%% \clearpage
%% \mode<all>{
%%   \part{Problem Decomposition}
%% }
%% \begin{frame}
%%   \partpage
%% \end{frame}
%% \mode<all>{
%%   \input{../party/body}
%%   }

%% \clearpage
%% \mode<all>{
%%   \part{More Global Constraints}
%% }
%% \begin{frame}
%%   \partpage
%% \end{frame}
%% \mode<all>{
%%   \input{../car/body}
%%   }

%% \clearpage
%% \mode<all>{
%%   \part{Methodology}
%% }
%% \begin{frame}
%%   \partpage
%% \end{frame}
%% \mode<all>{
%%   \input{../methodology/body}
%%   }

\clearpage
\mode<all>{
\part{What is missing?}
}
\begin{frame}
\partpage
\end{frame}

\begin{frame}
\frametitle{Many Specialized Topics}
\begin{itemize}
\item How to design efficient core engine
\item Hybrids with LP/MIP tools
\item Hybrids with SAT
\item Symmetry breaking
\item Use of MDD/BDD to encode sets of solutions
\item High level modelling tools
\item Debugging/visualization
\end{itemize}
\end{frame}

\begin{frame}
\frametitle{Reformulation}
\begin{itemize}
\item Just because the user has modelled it this way, it doesn't mean we have to solve it that way
\begin{itemize}
\item Replace some constraint(s) by other, equivalent constraints
\item Because we don't have that constraint in our system
\item For performance
\end{itemize}
\end{itemize}
\end{frame}

\begin{frame}
\frametitle{Learning}
\begin{itemize}
\item While solving the problem we can learn how to strengthen the model/search 
\begin{itemize}
\item Understand which constraints/method contribute to propagation and change schedule
\item Learn no-good constraints by explaining failure
\item Adapt search strategy based on search experience
\end{itemize}
\end{itemize}
\end{frame}

%% \begin{frame}
%% \frametitle{Refined Process}
%% \begin{center}
%% \begin{tikzpicture}
%% \node[shape=rectangle,fill=pantone127-4] (problem) at (3,7) {Problem};
%% \node (human) at (3,6) {Human};
%% \node[shape=rectangle,fill=pantone127-4] (model) at (3,5) {User Model};
%% \node (reformulation) at (3,4) {Reformulation};
%% \node[shape=rectangle,fill=pantone127-4] (implementation) at (3,3) {Implementation Model};
%% \node (solver) at (3,2) {Constraint Solver/Search};
%% \node[shape=rectangle,fill=pantone157-8] (solution) at (3,1) {Solution};
%% \draw[-] (problem) -- (human);
%% \draw[->] (human) -- (model);
%% \draw[-] (model) -- (reformulation);
%% \draw[->] (reformulation) -- (implementation);
%% \draw[-] (implementation) -- (solver);
%% \draw[->] (solver) -- (0,2) -- (0,3) -- (implementation);
%% \draw[->] (solver) -- (solution);
%% \end{tikzpicture}
%% \end{center}
%% \end{frame}

%\begin{frame}
%\frametitle{What is CP actually used for?}
%\url{http://hsimonis.wordpress.com}
%\includegraphics[width=10cm]{../introduction/images/blog}
%\end{frame}

%% \begin{frame}
%% \frametitle{Exercises}
%% \begin{itemize}
%% \item Install Constraint Solver of your Choice
%% \item MiniZinc (MiningZinc) will be used in later sessions
%% \item Run Sendmoremoney, Sudoku, Queens examples
%% \item Can you choose which propagator to use for alldifferent?
%% \item Can you use different search strategies?
%% \end{itemize}
%% \end{frame}

\begin{frame}
\frametitle{More Learning Resources}
\begin{itemize}
\item Survey of Methods, Resources, and Formats for Teaching Constraint Programming
\begin{itemize}
\item by Tejas Santanam, Helmut Simonis
\item \url{https://doi.org/10.48550/arXiv.2403.12717}
\item Based on survey of community for WTCP 2023
\item \url{https://hsimonis.github.io/WTCP2023/}
\end{itemize}
\end{itemize}
\end{frame}


}

\mode<all>{
\part{Global Constraints}
}
\begin{frame}
\partpage
\end{frame}
\mode<all>{
\begin{frame}
\frametitle{Objectives}
\begin{itemize}
\item Overview of Core Constraint Programming
\item Three Main Concepts
\begin{itemize}
\item Constraint Propagation
\item Global Constraints
\item Customizing Search
\end{itemize}
%\item Get Some Experience with MiniZinc
\item Topics will be treated in more detail in later parts of the school
\item Based on Examples, not Formal Description
\end{itemize}
\end{frame}

\begin{frame}
\frametitle{Outline}
\begin{itemize}
\item Why Constraint Programming?
\item Constraint Propagation
\item Global Constraints
\item Customizing Search
%\item Choices in Modelling
%\item Decomposition
%\item More Global Constraints
%\item Methodology
%\item What is missing?
\end{itemize}
\end{frame}

% \begin{frame}
  % \frametitle{Using MiniZinc IDE}
  % \begin{itemize}
  % \item Developed in the Australian NICTA project
  % \item Maintained by Monash University
  % \item Modelling tool with multiple back-end solvers
  % \item Available from \url{https://www.minizinc.org/}
  % \end{itemize}
% \end{frame}


% \begin{frame}
% \frametitle{Examples in ECLiPSe}
% \begin{itemize}
% \item Open sourced constraint programming language
% \item Development goes back to 1985
% \item ECRC, ICL, IC-Parc, PTL, Cisco
% \item \url{https://eclipseclp.org/}
% \item Specialities
% \begin{itemize}
% \item Develop new solvers for specific domains
% \item Integration with MIP
% \end{itemize}
% \item Not included in bundled MiniZinc IDE
% \item Specialized visualization tools used here
  % \begin{itemize}
  % \item CP-Viz, Simonis et al. 2010
    % \end{itemize}
% \end{itemize}
% \end{frame}

\begin{frame}
\frametitle{Tutorial Based on ECLiPSe ELearning Course}
\begin{itemize}
\item Self-study course in constraint programming
\item Supported by Cisco Systems and Silicon Valley Community Foundation
\item Multi-media format, video lectures, slides, handout etc
\item {\small \url{https://eclipseclp.org/ELearning/index.html}}
\end{itemize}
\end{frame}

\begin{frame}
\frametitle{Also Part of CRT-AI Constraint Week}
\begin{itemize}
\item Annual one week course on CP and Optimization in Ireland
\item Part of national training program for PhD students in AI
\item \url{https://www.crt-ai.ie/}
\end{itemize}
\end{frame}



\begin{frame}
\frametitle{Constraint Programming - in a nutshell}
\begin{itemize}
\item Declarative description of problems with
\begin{itemize}
\item {\em Variables} which range over (finite) sets of values
\item {\em Constraints} over subsets of variables which restrict possible value combinations
\item A {\em solution} is a value assignment which satisfies all constraints
\end{itemize}

\item Constraint propagation/reasoning
\begin{itemize}
\item Removing inconsistent values for variables
\item Detect failure if constraint can not be satisfied
\item Interaction of constraints via shared variables
\item Incomplete
\end{itemize}

\item Search
\begin{itemize}
\item User controlled assignment of values to variables
\item Each step triggers constraint propagation 
\end{itemize}
\item Different domains require/allow different methods
\end{itemize}
\end{frame}

\begin{frame}
  \frametitle{Constraint Programming is Different}
  \begin{itemize}
  \item Declarative Programming
    \begin{itemize}
    \item Concentrate on what you want
      \item Not how to get there
      \item Program != Algorithm
      \item Program = Model
    \end{itemize}
    \item Applied to Combinatorial Problems
      \begin{itemize}
        \item No complete polynomial algorithms known (exist?)
        \item CP less ad-hoc than heuristics
        \item Models can evolve
  \end{itemize}
  \end{itemize}
  \end{frame}
    
\begin{frame}
  \frametitle{A Subtractive Process}
  \begin{textblock}{4}(8,-3)
    \includegraphics[width=4cm]{../introduction/images/stages}
  \end{textblock}
  \vfill
  \begin{quote}
    ``Oh, bosh, as Mr. Ruskin says. Sculpture, per se, is the simplest thing in the world. All you have to do is to take a big chunk of marble and a hammer and chisel, make up your mind what you are about to create and chip off all the marble you don't want.''-Paris Gaulois.
  \end{quote}
  
  {\tiny Source: \url{https://quoteinvestigator.com/2014/06/22/chip-away/}}
\end{frame}

\begin{frame}
\frametitle{Basic Process}
\begin{center}
\begin{tikzpicture}
\node[shape=rectangle,fill=pantone127-4] (problem) at (2,5) {Problem};
\node (human) at (2,4) {Human};
\node[shape=rectangle,fill=pantone127-4] (model) at (2,3) {Model};
\node (solver) at (2,2) {Constraint Solver/Search};
\node[shape=rectangle,fill=pantone157-8] (solution) at (2,1) {Solution};
\draw[-] (problem) -- (human);
\draw[->] (human) -- (model);
\draw[-] (model) -- (solver);
\draw[->] (solver) -- (solution);
\end{tikzpicture}
\end{center}
\end{frame}

\begin{frame}
\frametitle{More Realistic}
\begin{center}
\begin{tikzpicture}[xscale=4.0]
\node[shape=rectangle,fill=pantone127-4] (problem) at (2,5) {Problem};
\node (human) at (2,4) {Human};
\node[shape=rectangle,fill=pantone127-4] (model) at (2,3) {Model};
\node (solver) at (2,2) {Constraint Solver/Search};
\node[shape=rectangle,fill=pantone24-8] (hangs) at (1,1) {Hangs};
\node[shape=rectangle,fill=pantone157-8] (solution) at (2,1) {Solution};
\node[shape=rectangle,fill=pantone24-8] (wrong) at (3,1) {Wrong Solution};
\draw[-] (problem) -- (human);
\draw[->] (human) -- (model);
\draw[-] (model) -- (solver);
\draw[->] (solver) -- (hangs);
\draw[->] (solver) -- (solution);
\draw[->] (solver) -- (wrong);
\draw[->] (hangs) -- (1,4) -- (human);
\draw[->] (wrong) -- (3,4) -- (human);
\end{tikzpicture}
\end{center}
\end{frame}

\begin{frame}
\frametitle{Dual Role of Model}
\begin{itemize}
\item Allows Human to Express Problem
\begin{itemize}
\item Close to Problem Domain
\item Constraints as Abstractions
\end{itemize}
\item Allows Solver to Execute
\begin{itemize}
\item Variables as Communication Mechanism
\item Constraints as Algorithms
\end{itemize}
\end{itemize}
\end{frame}

\begin{frame}
\frametitle{Modelling Frameworks}
\begin{itemize}
\item \href{https://www.minizinc.org/}{MiniZinc} (NICTA, Monash University, Australia)
\item \href{https://github.com/eomahony/Numberjack}{NumberJack} (Insight, Ireland) 
\item \href{https://www-users.york.ac.uk/peter.nightingale/savilerow//}{EssencePrime/SavilleRow} (UK) 
\item \href{https://cpmpy.readthedocs.io/en/latest/}{CPMpy} (KU Leuven)
\item Allow use of multiple back-end solvers
\item Compile model into variants for each solver
\item A priori solver independent model(CP, MIP, SAT)
\end{itemize}
\end{frame} 

\begin{frame}
\frametitle{Framework Process}
\begin{center}
\begin{tikzpicture}[xscale=2]
\node[shape=rectangle,fill=pantone127-4] (problem) at (2.5,5) {Problem};
\node (human) at (2.5,4) {Human};
\node[shape=rectangle,fill=pantone127-4] (model) at (2.5,3) {Model};
\node (compile) at (2.5,2) {Compile/Reformulate};
\node (solver1) at (1,1) {CP};
\node (solver2) at (2,1) {MIP};
\node (solver3) at (3,1) {SAT};
\node (solver4) at (4,1) {Other};
\node[shape=rectangle,fill=pantone157-8] (solution1) at (1,0) {Solution};
\node[shape=rectangle,fill=pantone157-8] (solution2) at (2,0) {Solution};
\node[shape=rectangle,fill=pantone157-8] (solution3) at (3,0) {Solution};
\node[shape=rectangle,fill=pantone157-8] (solution4) at (4,0) {Solution};
\draw[-] (problem) -- (human);
\draw[->] (human) -- (model);
\draw[-] (model) -- (compile);
\draw[-] (compile) -- (solver1);
\draw[-] (compile) -- (solver2);
\draw[-] (compile) -- (solver3);
\draw[-] (compile) -- (solver4);
\draw[->] (solver1) -- (solution1);
\draw[->] (solver2) -- (solution2);
\draw[->] (solver3) -- (solution3);
\draw[->] (solver4) -- (solution4);
\end{tikzpicture}
\end{center}
\end{frame}

% \begin{frame}
  % \frametitle{Do It Now!}
  % \begin{itemize}
  % \item Download and install Minizinc
  % \item \url{https://www.minizinc.org/}
  % \end{itemize}
% \end{frame}


\begin{frame}
\frametitle{Why use Puzzles as Examples?}
\begin{itemize}
\item Easy to understand the problem
\item Solvable by hand without specialized knowledge
\item Possible to compare automated to manual solving process
\end{itemize}

\begin{quote}
The puzzle, though inanimate, is presented as a solvable problem without lasting negative consequences, a very low-risk low-reward situation. By being a puzzle, the object is attempting to convince the user that it must be completed.
\end{quote}
{\tiny Source: \href{https://www.everydayrhetoric.com/2020/03/the-rhetoric-of-puzzles/}{Every Day Rhetoric}}
\end{frame}


\mode<all>{
\part{Basic Constraint Propagation}
}
\begin{frame}
\partpage
\end{frame}
\mode<all>{
\input{../sendmore/body}
}

\mode<all>{
\part{Global Constraints}
}
\begin{frame}
\partpage
\end{frame}
\mode<all>{
\input{../sudoku/body}
}

\clearpage
\mode<all>{
\part{Customizing Search}
}
\begin{frame}
\partpage
\end{frame}
\mode<all>{
\input{../nqueen/body}
}


%% \clearpage
%% \mode<all>{
%%   \part{Problem Decomposition}
%% }
%% \begin{frame}
%%   \partpage
%% \end{frame}
%% \mode<all>{
%%   \input{../party/body}
%%   }

%% \clearpage
%% \mode<all>{
%%   \part{More Global Constraints}
%% }
%% \begin{frame}
%%   \partpage
%% \end{frame}
%% \mode<all>{
%%   \input{../car/body}
%%   }

%% \clearpage
%% \mode<all>{
%%   \part{Methodology}
%% }
%% \begin{frame}
%%   \partpage
%% \end{frame}
%% \mode<all>{
%%   \input{../methodology/body}
%%   }

\clearpage
\mode<all>{
\part{What is missing?}
}
\begin{frame}
\partpage
\end{frame}

\begin{frame}
\frametitle{Many Specialized Topics}
\begin{itemize}
\item How to design efficient core engine
\item Hybrids with LP/MIP tools
\item Hybrids with SAT
\item Symmetry breaking
\item Use of MDD/BDD to encode sets of solutions
\item High level modelling tools
\item Debugging/visualization
\end{itemize}
\end{frame}

\begin{frame}
\frametitle{Reformulation}
\begin{itemize}
\item Just because the user has modelled it this way, it doesn't mean we have to solve it that way
\begin{itemize}
\item Replace some constraint(s) by other, equivalent constraints
\item Because we don't have that constraint in our system
\item For performance
\end{itemize}
\end{itemize}
\end{frame}

\begin{frame}
\frametitle{Learning}
\begin{itemize}
\item While solving the problem we can learn how to strengthen the model/search 
\begin{itemize}
\item Understand which constraints/method contribute to propagation and change schedule
\item Learn no-good constraints by explaining failure
\item Adapt search strategy based on search experience
\end{itemize}
\end{itemize}
\end{frame}

%% \begin{frame}
%% \frametitle{Refined Process}
%% \begin{center}
%% \begin{tikzpicture}
%% \node[shape=rectangle,fill=pantone127-4] (problem) at (3,7) {Problem};
%% \node (human) at (3,6) {Human};
%% \node[shape=rectangle,fill=pantone127-4] (model) at (3,5) {User Model};
%% \node (reformulation) at (3,4) {Reformulation};
%% \node[shape=rectangle,fill=pantone127-4] (implementation) at (3,3) {Implementation Model};
%% \node (solver) at (3,2) {Constraint Solver/Search};
%% \node[shape=rectangle,fill=pantone157-8] (solution) at (3,1) {Solution};
%% \draw[-] (problem) -- (human);
%% \draw[->] (human) -- (model);
%% \draw[-] (model) -- (reformulation);
%% \draw[->] (reformulation) -- (implementation);
%% \draw[-] (implementation) -- (solver);
%% \draw[->] (solver) -- (0,2) -- (0,3) -- (implementation);
%% \draw[->] (solver) -- (solution);
%% \end{tikzpicture}
%% \end{center}
%% \end{frame}

%\begin{frame}
%\frametitle{What is CP actually used for?}
%\url{http://hsimonis.wordpress.com}
%\includegraphics[width=10cm]{../introduction/images/blog}
%\end{frame}

%% \begin{frame}
%% \frametitle{Exercises}
%% \begin{itemize}
%% \item Install Constraint Solver of your Choice
%% \item MiniZinc (MiningZinc) will be used in later sessions
%% \item Run Sendmoremoney, Sudoku, Queens examples
%% \item Can you choose which propagator to use for alldifferent?
%% \item Can you use different search strategies?
%% \end{itemize}
%% \end{frame}

\begin{frame}
\frametitle{More Learning Resources}
\begin{itemize}
\item Survey of Methods, Resources, and Formats for Teaching Constraint Programming
\begin{itemize}
\item by Tejas Santanam, Helmut Simonis
\item \url{https://doi.org/10.48550/arXiv.2403.12717}
\item Based on survey of community for WTCP 2023
\item \url{https://hsimonis.github.io/WTCP2023/}
\end{itemize}
\end{itemize}
\end{frame}


}

\clearpage
\mode<all>{
\part{Customizing Search}
}
\begin{frame}
\partpage
\end{frame}
\mode<all>{
\begin{frame}
\frametitle{Objectives}
\begin{itemize}
\item Overview of Core Constraint Programming
\item Three Main Concepts
\begin{itemize}
\item Constraint Propagation
\item Global Constraints
\item Customizing Search
\end{itemize}
%\item Get Some Experience with MiniZinc
\item Topics will be treated in more detail in later parts of the school
\item Based on Examples, not Formal Description
\end{itemize}
\end{frame}

\begin{frame}
\frametitle{Outline}
\begin{itemize}
\item Why Constraint Programming?
\item Constraint Propagation
\item Global Constraints
\item Customizing Search
%\item Choices in Modelling
%\item Decomposition
%\item More Global Constraints
%\item Methodology
%\item What is missing?
\end{itemize}
\end{frame}

% \begin{frame}
  % \frametitle{Using MiniZinc IDE}
  % \begin{itemize}
  % \item Developed in the Australian NICTA project
  % \item Maintained by Monash University
  % \item Modelling tool with multiple back-end solvers
  % \item Available from \url{https://www.minizinc.org/}
  % \end{itemize}
% \end{frame}


% \begin{frame}
% \frametitle{Examples in ECLiPSe}
% \begin{itemize}
% \item Open sourced constraint programming language
% \item Development goes back to 1985
% \item ECRC, ICL, IC-Parc, PTL, Cisco
% \item \url{https://eclipseclp.org/}
% \item Specialities
% \begin{itemize}
% \item Develop new solvers for specific domains
% \item Integration with MIP
% \end{itemize}
% \item Not included in bundled MiniZinc IDE
% \item Specialized visualization tools used here
  % \begin{itemize}
  % \item CP-Viz, Simonis et al. 2010
    % \end{itemize}
% \end{itemize}
% \end{frame}

\begin{frame}
\frametitle{Tutorial Based on ECLiPSe ELearning Course}
\begin{itemize}
\item Self-study course in constraint programming
\item Supported by Cisco Systems and Silicon Valley Community Foundation
\item Multi-media format, video lectures, slides, handout etc
\item {\small \url{https://eclipseclp.org/ELearning/index.html}}
\end{itemize}
\end{frame}

\begin{frame}
\frametitle{Also Part of CRT-AI Constraint Week}
\begin{itemize}
\item Annual one week course on CP and Optimization in Ireland
\item Part of national training program for PhD students in AI
\item \url{https://www.crt-ai.ie/}
\end{itemize}
\end{frame}



\begin{frame}
\frametitle{Constraint Programming - in a nutshell}
\begin{itemize}
\item Declarative description of problems with
\begin{itemize}
\item {\em Variables} which range over (finite) sets of values
\item {\em Constraints} over subsets of variables which restrict possible value combinations
\item A {\em solution} is a value assignment which satisfies all constraints
\end{itemize}

\item Constraint propagation/reasoning
\begin{itemize}
\item Removing inconsistent values for variables
\item Detect failure if constraint can not be satisfied
\item Interaction of constraints via shared variables
\item Incomplete
\end{itemize}

\item Search
\begin{itemize}
\item User controlled assignment of values to variables
\item Each step triggers constraint propagation 
\end{itemize}
\item Different domains require/allow different methods
\end{itemize}
\end{frame}

\begin{frame}
  \frametitle{Constraint Programming is Different}
  \begin{itemize}
  \item Declarative Programming
    \begin{itemize}
    \item Concentrate on what you want
      \item Not how to get there
      \item Program != Algorithm
      \item Program = Model
    \end{itemize}
    \item Applied to Combinatorial Problems
      \begin{itemize}
        \item No complete polynomial algorithms known (exist?)
        \item CP less ad-hoc than heuristics
        \item Models can evolve
  \end{itemize}
  \end{itemize}
  \end{frame}
    
\begin{frame}
  \frametitle{A Subtractive Process}
  \begin{textblock}{4}(8,-3)
    \includegraphics[width=4cm]{../introduction/images/stages}
  \end{textblock}
  \vfill
  \begin{quote}
    ``Oh, bosh, as Mr. Ruskin says. Sculpture, per se, is the simplest thing in the world. All you have to do is to take a big chunk of marble and a hammer and chisel, make up your mind what you are about to create and chip off all the marble you don't want.''-Paris Gaulois.
  \end{quote}
  
  {\tiny Source: \url{https://quoteinvestigator.com/2014/06/22/chip-away/}}
\end{frame}

\begin{frame}
\frametitle{Basic Process}
\begin{center}
\begin{tikzpicture}
\node[shape=rectangle,fill=pantone127-4] (problem) at (2,5) {Problem};
\node (human) at (2,4) {Human};
\node[shape=rectangle,fill=pantone127-4] (model) at (2,3) {Model};
\node (solver) at (2,2) {Constraint Solver/Search};
\node[shape=rectangle,fill=pantone157-8] (solution) at (2,1) {Solution};
\draw[-] (problem) -- (human);
\draw[->] (human) -- (model);
\draw[-] (model) -- (solver);
\draw[->] (solver) -- (solution);
\end{tikzpicture}
\end{center}
\end{frame}

\begin{frame}
\frametitle{More Realistic}
\begin{center}
\begin{tikzpicture}[xscale=4.0]
\node[shape=rectangle,fill=pantone127-4] (problem) at (2,5) {Problem};
\node (human) at (2,4) {Human};
\node[shape=rectangle,fill=pantone127-4] (model) at (2,3) {Model};
\node (solver) at (2,2) {Constraint Solver/Search};
\node[shape=rectangle,fill=pantone24-8] (hangs) at (1,1) {Hangs};
\node[shape=rectangle,fill=pantone157-8] (solution) at (2,1) {Solution};
\node[shape=rectangle,fill=pantone24-8] (wrong) at (3,1) {Wrong Solution};
\draw[-] (problem) -- (human);
\draw[->] (human) -- (model);
\draw[-] (model) -- (solver);
\draw[->] (solver) -- (hangs);
\draw[->] (solver) -- (solution);
\draw[->] (solver) -- (wrong);
\draw[->] (hangs) -- (1,4) -- (human);
\draw[->] (wrong) -- (3,4) -- (human);
\end{tikzpicture}
\end{center}
\end{frame}

\begin{frame}
\frametitle{Dual Role of Model}
\begin{itemize}
\item Allows Human to Express Problem
\begin{itemize}
\item Close to Problem Domain
\item Constraints as Abstractions
\end{itemize}
\item Allows Solver to Execute
\begin{itemize}
\item Variables as Communication Mechanism
\item Constraints as Algorithms
\end{itemize}
\end{itemize}
\end{frame}

\begin{frame}
\frametitle{Modelling Frameworks}
\begin{itemize}
\item \href{https://www.minizinc.org/}{MiniZinc} (NICTA, Monash University, Australia)
\item \href{https://github.com/eomahony/Numberjack}{NumberJack} (Insight, Ireland) 
\item \href{https://www-users.york.ac.uk/peter.nightingale/savilerow//}{EssencePrime/SavilleRow} (UK) 
\item \href{https://cpmpy.readthedocs.io/en/latest/}{CPMpy} (KU Leuven)
\item Allow use of multiple back-end solvers
\item Compile model into variants for each solver
\item A priori solver independent model(CP, MIP, SAT)
\end{itemize}
\end{frame} 

\begin{frame}
\frametitle{Framework Process}
\begin{center}
\begin{tikzpicture}[xscale=2]
\node[shape=rectangle,fill=pantone127-4] (problem) at (2.5,5) {Problem};
\node (human) at (2.5,4) {Human};
\node[shape=rectangle,fill=pantone127-4] (model) at (2.5,3) {Model};
\node (compile) at (2.5,2) {Compile/Reformulate};
\node (solver1) at (1,1) {CP};
\node (solver2) at (2,1) {MIP};
\node (solver3) at (3,1) {SAT};
\node (solver4) at (4,1) {Other};
\node[shape=rectangle,fill=pantone157-8] (solution1) at (1,0) {Solution};
\node[shape=rectangle,fill=pantone157-8] (solution2) at (2,0) {Solution};
\node[shape=rectangle,fill=pantone157-8] (solution3) at (3,0) {Solution};
\node[shape=rectangle,fill=pantone157-8] (solution4) at (4,0) {Solution};
\draw[-] (problem) -- (human);
\draw[->] (human) -- (model);
\draw[-] (model) -- (compile);
\draw[-] (compile) -- (solver1);
\draw[-] (compile) -- (solver2);
\draw[-] (compile) -- (solver3);
\draw[-] (compile) -- (solver4);
\draw[->] (solver1) -- (solution1);
\draw[->] (solver2) -- (solution2);
\draw[->] (solver3) -- (solution3);
\draw[->] (solver4) -- (solution4);
\end{tikzpicture}
\end{center}
\end{frame}

% \begin{frame}
  % \frametitle{Do It Now!}
  % \begin{itemize}
  % \item Download and install Minizinc
  % \item \url{https://www.minizinc.org/}
  % \end{itemize}
% \end{frame}


\begin{frame}
\frametitle{Why use Puzzles as Examples?}
\begin{itemize}
\item Easy to understand the problem
\item Solvable by hand without specialized knowledge
\item Possible to compare automated to manual solving process
\end{itemize}

\begin{quote}
The puzzle, though inanimate, is presented as a solvable problem without lasting negative consequences, a very low-risk low-reward situation. By being a puzzle, the object is attempting to convince the user that it must be completed.
\end{quote}
{\tiny Source: \href{https://www.everydayrhetoric.com/2020/03/the-rhetoric-of-puzzles/}{Every Day Rhetoric}}
\end{frame}


\mode<all>{
\part{Basic Constraint Propagation}
}
\begin{frame}
\partpage
\end{frame}
\mode<all>{
\input{../sendmore/body}
}

\mode<all>{
\part{Global Constraints}
}
\begin{frame}
\partpage
\end{frame}
\mode<all>{
\input{../sudoku/body}
}

\clearpage
\mode<all>{
\part{Customizing Search}
}
\begin{frame}
\partpage
\end{frame}
\mode<all>{
\input{../nqueen/body}
}


%% \clearpage
%% \mode<all>{
%%   \part{Problem Decomposition}
%% }
%% \begin{frame}
%%   \partpage
%% \end{frame}
%% \mode<all>{
%%   \input{../party/body}
%%   }

%% \clearpage
%% \mode<all>{
%%   \part{More Global Constraints}
%% }
%% \begin{frame}
%%   \partpage
%% \end{frame}
%% \mode<all>{
%%   \input{../car/body}
%%   }

%% \clearpage
%% \mode<all>{
%%   \part{Methodology}
%% }
%% \begin{frame}
%%   \partpage
%% \end{frame}
%% \mode<all>{
%%   \input{../methodology/body}
%%   }

\clearpage
\mode<all>{
\part{What is missing?}
}
\begin{frame}
\partpage
\end{frame}

\begin{frame}
\frametitle{Many Specialized Topics}
\begin{itemize}
\item How to design efficient core engine
\item Hybrids with LP/MIP tools
\item Hybrids with SAT
\item Symmetry breaking
\item Use of MDD/BDD to encode sets of solutions
\item High level modelling tools
\item Debugging/visualization
\end{itemize}
\end{frame}

\begin{frame}
\frametitle{Reformulation}
\begin{itemize}
\item Just because the user has modelled it this way, it doesn't mean we have to solve it that way
\begin{itemize}
\item Replace some constraint(s) by other, equivalent constraints
\item Because we don't have that constraint in our system
\item For performance
\end{itemize}
\end{itemize}
\end{frame}

\begin{frame}
\frametitle{Learning}
\begin{itemize}
\item While solving the problem we can learn how to strengthen the model/search 
\begin{itemize}
\item Understand which constraints/method contribute to propagation and change schedule
\item Learn no-good constraints by explaining failure
\item Adapt search strategy based on search experience
\end{itemize}
\end{itemize}
\end{frame}

%% \begin{frame}
%% \frametitle{Refined Process}
%% \begin{center}
%% \begin{tikzpicture}
%% \node[shape=rectangle,fill=pantone127-4] (problem) at (3,7) {Problem};
%% \node (human) at (3,6) {Human};
%% \node[shape=rectangle,fill=pantone127-4] (model) at (3,5) {User Model};
%% \node (reformulation) at (3,4) {Reformulation};
%% \node[shape=rectangle,fill=pantone127-4] (implementation) at (3,3) {Implementation Model};
%% \node (solver) at (3,2) {Constraint Solver/Search};
%% \node[shape=rectangle,fill=pantone157-8] (solution) at (3,1) {Solution};
%% \draw[-] (problem) -- (human);
%% \draw[->] (human) -- (model);
%% \draw[-] (model) -- (reformulation);
%% \draw[->] (reformulation) -- (implementation);
%% \draw[-] (implementation) -- (solver);
%% \draw[->] (solver) -- (0,2) -- (0,3) -- (implementation);
%% \draw[->] (solver) -- (solution);
%% \end{tikzpicture}
%% \end{center}
%% \end{frame}

%\begin{frame}
%\frametitle{What is CP actually used for?}
%\url{http://hsimonis.wordpress.com}
%\includegraphics[width=10cm]{../introduction/images/blog}
%\end{frame}

%% \begin{frame}
%% \frametitle{Exercises}
%% \begin{itemize}
%% \item Install Constraint Solver of your Choice
%% \item MiniZinc (MiningZinc) will be used in later sessions
%% \item Run Sendmoremoney, Sudoku, Queens examples
%% \item Can you choose which propagator to use for alldifferent?
%% \item Can you use different search strategies?
%% \end{itemize}
%% \end{frame}

\begin{frame}
\frametitle{More Learning Resources}
\begin{itemize}
\item Survey of Methods, Resources, and Formats for Teaching Constraint Programming
\begin{itemize}
\item by Tejas Santanam, Helmut Simonis
\item \url{https://doi.org/10.48550/arXiv.2403.12717}
\item Based on survey of community for WTCP 2023
\item \url{https://hsimonis.github.io/WTCP2023/}
\end{itemize}
\end{itemize}
\end{frame}


}


%% \clearpage
%% \mode<all>{
%%   \part{Problem Decomposition}
%% }
%% \begin{frame}
%%   \partpage
%% \end{frame}
%% \mode<all>{
%%   \begin{frame}
\frametitle{Objectives}
\begin{itemize}
\item Overview of Core Constraint Programming
\item Three Main Concepts
\begin{itemize}
\item Constraint Propagation
\item Global Constraints
\item Customizing Search
\end{itemize}
%\item Get Some Experience with MiniZinc
\item Topics will be treated in more detail in later parts of the school
\item Based on Examples, not Formal Description
\end{itemize}
\end{frame}

\begin{frame}
\frametitle{Outline}
\begin{itemize}
\item Why Constraint Programming?
\item Constraint Propagation
\item Global Constraints
\item Customizing Search
%\item Choices in Modelling
%\item Decomposition
%\item More Global Constraints
%\item Methodology
%\item What is missing?
\end{itemize}
\end{frame}

% \begin{frame}
  % \frametitle{Using MiniZinc IDE}
  % \begin{itemize}
  % \item Developed in the Australian NICTA project
  % \item Maintained by Monash University
  % \item Modelling tool with multiple back-end solvers
  % \item Available from \url{https://www.minizinc.org/}
  % \end{itemize}
% \end{frame}


% \begin{frame}
% \frametitle{Examples in ECLiPSe}
% \begin{itemize}
% \item Open sourced constraint programming language
% \item Development goes back to 1985
% \item ECRC, ICL, IC-Parc, PTL, Cisco
% \item \url{https://eclipseclp.org/}
% \item Specialities
% \begin{itemize}
% \item Develop new solvers for specific domains
% \item Integration with MIP
% \end{itemize}
% \item Not included in bundled MiniZinc IDE
% \item Specialized visualization tools used here
  % \begin{itemize}
  % \item CP-Viz, Simonis et al. 2010
    % \end{itemize}
% \end{itemize}
% \end{frame}

\begin{frame}
\frametitle{Tutorial Based on ECLiPSe ELearning Course}
\begin{itemize}
\item Self-study course in constraint programming
\item Supported by Cisco Systems and Silicon Valley Community Foundation
\item Multi-media format, video lectures, slides, handout etc
\item {\small \url{https://eclipseclp.org/ELearning/index.html}}
\end{itemize}
\end{frame}

\begin{frame}
\frametitle{Also Part of CRT-AI Constraint Week}
\begin{itemize}
\item Annual one week course on CP and Optimization in Ireland
\item Part of national training program for PhD students in AI
\item \url{https://www.crt-ai.ie/}
\end{itemize}
\end{frame}



\begin{frame}
\frametitle{Constraint Programming - in a nutshell}
\begin{itemize}
\item Declarative description of problems with
\begin{itemize}
\item {\em Variables} which range over (finite) sets of values
\item {\em Constraints} over subsets of variables which restrict possible value combinations
\item A {\em solution} is a value assignment which satisfies all constraints
\end{itemize}

\item Constraint propagation/reasoning
\begin{itemize}
\item Removing inconsistent values for variables
\item Detect failure if constraint can not be satisfied
\item Interaction of constraints via shared variables
\item Incomplete
\end{itemize}

\item Search
\begin{itemize}
\item User controlled assignment of values to variables
\item Each step triggers constraint propagation 
\end{itemize}
\item Different domains require/allow different methods
\end{itemize}
\end{frame}

\begin{frame}
  \frametitle{Constraint Programming is Different}
  \begin{itemize}
  \item Declarative Programming
    \begin{itemize}
    \item Concentrate on what you want
      \item Not how to get there
      \item Program != Algorithm
      \item Program = Model
    \end{itemize}
    \item Applied to Combinatorial Problems
      \begin{itemize}
        \item No complete polynomial algorithms known (exist?)
        \item CP less ad-hoc than heuristics
        \item Models can evolve
  \end{itemize}
  \end{itemize}
  \end{frame}
    
\begin{frame}
  \frametitle{A Subtractive Process}
  \begin{textblock}{4}(8,-3)
    \includegraphics[width=4cm]{../introduction/images/stages}
  \end{textblock}
  \vfill
  \begin{quote}
    ``Oh, bosh, as Mr. Ruskin says. Sculpture, per se, is the simplest thing in the world. All you have to do is to take a big chunk of marble and a hammer and chisel, make up your mind what you are about to create and chip off all the marble you don't want.''-Paris Gaulois.
  \end{quote}
  
  {\tiny Source: \url{https://quoteinvestigator.com/2014/06/22/chip-away/}}
\end{frame}

\begin{frame}
\frametitle{Basic Process}
\begin{center}
\begin{tikzpicture}
\node[shape=rectangle,fill=pantone127-4] (problem) at (2,5) {Problem};
\node (human) at (2,4) {Human};
\node[shape=rectangle,fill=pantone127-4] (model) at (2,3) {Model};
\node (solver) at (2,2) {Constraint Solver/Search};
\node[shape=rectangle,fill=pantone157-8] (solution) at (2,1) {Solution};
\draw[-] (problem) -- (human);
\draw[->] (human) -- (model);
\draw[-] (model) -- (solver);
\draw[->] (solver) -- (solution);
\end{tikzpicture}
\end{center}
\end{frame}

\begin{frame}
\frametitle{More Realistic}
\begin{center}
\begin{tikzpicture}[xscale=4.0]
\node[shape=rectangle,fill=pantone127-4] (problem) at (2,5) {Problem};
\node (human) at (2,4) {Human};
\node[shape=rectangle,fill=pantone127-4] (model) at (2,3) {Model};
\node (solver) at (2,2) {Constraint Solver/Search};
\node[shape=rectangle,fill=pantone24-8] (hangs) at (1,1) {Hangs};
\node[shape=rectangle,fill=pantone157-8] (solution) at (2,1) {Solution};
\node[shape=rectangle,fill=pantone24-8] (wrong) at (3,1) {Wrong Solution};
\draw[-] (problem) -- (human);
\draw[->] (human) -- (model);
\draw[-] (model) -- (solver);
\draw[->] (solver) -- (hangs);
\draw[->] (solver) -- (solution);
\draw[->] (solver) -- (wrong);
\draw[->] (hangs) -- (1,4) -- (human);
\draw[->] (wrong) -- (3,4) -- (human);
\end{tikzpicture}
\end{center}
\end{frame}

\begin{frame}
\frametitle{Dual Role of Model}
\begin{itemize}
\item Allows Human to Express Problem
\begin{itemize}
\item Close to Problem Domain
\item Constraints as Abstractions
\end{itemize}
\item Allows Solver to Execute
\begin{itemize}
\item Variables as Communication Mechanism
\item Constraints as Algorithms
\end{itemize}
\end{itemize}
\end{frame}

\begin{frame}
\frametitle{Modelling Frameworks}
\begin{itemize}
\item \href{https://www.minizinc.org/}{MiniZinc} (NICTA, Monash University, Australia)
\item \href{https://github.com/eomahony/Numberjack}{NumberJack} (Insight, Ireland) 
\item \href{https://www-users.york.ac.uk/peter.nightingale/savilerow//}{EssencePrime/SavilleRow} (UK) 
\item \href{https://cpmpy.readthedocs.io/en/latest/}{CPMpy} (KU Leuven)
\item Allow use of multiple back-end solvers
\item Compile model into variants for each solver
\item A priori solver independent model(CP, MIP, SAT)
\end{itemize}
\end{frame} 

\begin{frame}
\frametitle{Framework Process}
\begin{center}
\begin{tikzpicture}[xscale=2]
\node[shape=rectangle,fill=pantone127-4] (problem) at (2.5,5) {Problem};
\node (human) at (2.5,4) {Human};
\node[shape=rectangle,fill=pantone127-4] (model) at (2.5,3) {Model};
\node (compile) at (2.5,2) {Compile/Reformulate};
\node (solver1) at (1,1) {CP};
\node (solver2) at (2,1) {MIP};
\node (solver3) at (3,1) {SAT};
\node (solver4) at (4,1) {Other};
\node[shape=rectangle,fill=pantone157-8] (solution1) at (1,0) {Solution};
\node[shape=rectangle,fill=pantone157-8] (solution2) at (2,0) {Solution};
\node[shape=rectangle,fill=pantone157-8] (solution3) at (3,0) {Solution};
\node[shape=rectangle,fill=pantone157-8] (solution4) at (4,0) {Solution};
\draw[-] (problem) -- (human);
\draw[->] (human) -- (model);
\draw[-] (model) -- (compile);
\draw[-] (compile) -- (solver1);
\draw[-] (compile) -- (solver2);
\draw[-] (compile) -- (solver3);
\draw[-] (compile) -- (solver4);
\draw[->] (solver1) -- (solution1);
\draw[->] (solver2) -- (solution2);
\draw[->] (solver3) -- (solution3);
\draw[->] (solver4) -- (solution4);
\end{tikzpicture}
\end{center}
\end{frame}

% \begin{frame}
  % \frametitle{Do It Now!}
  % \begin{itemize}
  % \item Download and install Minizinc
  % \item \url{https://www.minizinc.org/}
  % \end{itemize}
% \end{frame}


\begin{frame}
\frametitle{Why use Puzzles as Examples?}
\begin{itemize}
\item Easy to understand the problem
\item Solvable by hand without specialized knowledge
\item Possible to compare automated to manual solving process
\end{itemize}

\begin{quote}
The puzzle, though inanimate, is presented as a solvable problem without lasting negative consequences, a very low-risk low-reward situation. By being a puzzle, the object is attempting to convince the user that it must be completed.
\end{quote}
{\tiny Source: \href{https://www.everydayrhetoric.com/2020/03/the-rhetoric-of-puzzles/}{Every Day Rhetoric}}
\end{frame}


\mode<all>{
\part{Basic Constraint Propagation}
}
\begin{frame}
\partpage
\end{frame}
\mode<all>{
\input{../sendmore/body}
}

\mode<all>{
\part{Global Constraints}
}
\begin{frame}
\partpage
\end{frame}
\mode<all>{
\input{../sudoku/body}
}

\clearpage
\mode<all>{
\part{Customizing Search}
}
\begin{frame}
\partpage
\end{frame}
\mode<all>{
\input{../nqueen/body}
}


%% \clearpage
%% \mode<all>{
%%   \part{Problem Decomposition}
%% }
%% \begin{frame}
%%   \partpage
%% \end{frame}
%% \mode<all>{
%%   \input{../party/body}
%%   }

%% \clearpage
%% \mode<all>{
%%   \part{More Global Constraints}
%% }
%% \begin{frame}
%%   \partpage
%% \end{frame}
%% \mode<all>{
%%   \input{../car/body}
%%   }

%% \clearpage
%% \mode<all>{
%%   \part{Methodology}
%% }
%% \begin{frame}
%%   \partpage
%% \end{frame}
%% \mode<all>{
%%   \input{../methodology/body}
%%   }

\clearpage
\mode<all>{
\part{What is missing?}
}
\begin{frame}
\partpage
\end{frame}

\begin{frame}
\frametitle{Many Specialized Topics}
\begin{itemize}
\item How to design efficient core engine
\item Hybrids with LP/MIP tools
\item Hybrids with SAT
\item Symmetry breaking
\item Use of MDD/BDD to encode sets of solutions
\item High level modelling tools
\item Debugging/visualization
\end{itemize}
\end{frame}

\begin{frame}
\frametitle{Reformulation}
\begin{itemize}
\item Just because the user has modelled it this way, it doesn't mean we have to solve it that way
\begin{itemize}
\item Replace some constraint(s) by other, equivalent constraints
\item Because we don't have that constraint in our system
\item For performance
\end{itemize}
\end{itemize}
\end{frame}

\begin{frame}
\frametitle{Learning}
\begin{itemize}
\item While solving the problem we can learn how to strengthen the model/search 
\begin{itemize}
\item Understand which constraints/method contribute to propagation and change schedule
\item Learn no-good constraints by explaining failure
\item Adapt search strategy based on search experience
\end{itemize}
\end{itemize}
\end{frame}

%% \begin{frame}
%% \frametitle{Refined Process}
%% \begin{center}
%% \begin{tikzpicture}
%% \node[shape=rectangle,fill=pantone127-4] (problem) at (3,7) {Problem};
%% \node (human) at (3,6) {Human};
%% \node[shape=rectangle,fill=pantone127-4] (model) at (3,5) {User Model};
%% \node (reformulation) at (3,4) {Reformulation};
%% \node[shape=rectangle,fill=pantone127-4] (implementation) at (3,3) {Implementation Model};
%% \node (solver) at (3,2) {Constraint Solver/Search};
%% \node[shape=rectangle,fill=pantone157-8] (solution) at (3,1) {Solution};
%% \draw[-] (problem) -- (human);
%% \draw[->] (human) -- (model);
%% \draw[-] (model) -- (reformulation);
%% \draw[->] (reformulation) -- (implementation);
%% \draw[-] (implementation) -- (solver);
%% \draw[->] (solver) -- (0,2) -- (0,3) -- (implementation);
%% \draw[->] (solver) -- (solution);
%% \end{tikzpicture}
%% \end{center}
%% \end{frame}

%\begin{frame}
%\frametitle{What is CP actually used for?}
%\url{http://hsimonis.wordpress.com}
%\includegraphics[width=10cm]{../introduction/images/blog}
%\end{frame}

%% \begin{frame}
%% \frametitle{Exercises}
%% \begin{itemize}
%% \item Install Constraint Solver of your Choice
%% \item MiniZinc (MiningZinc) will be used in later sessions
%% \item Run Sendmoremoney, Sudoku, Queens examples
%% \item Can you choose which propagator to use for alldifferent?
%% \item Can you use different search strategies?
%% \end{itemize}
%% \end{frame}

\begin{frame}
\frametitle{More Learning Resources}
\begin{itemize}
\item Survey of Methods, Resources, and Formats for Teaching Constraint Programming
\begin{itemize}
\item by Tejas Santanam, Helmut Simonis
\item \url{https://doi.org/10.48550/arXiv.2403.12717}
\item Based on survey of community for WTCP 2023
\item \url{https://hsimonis.github.io/WTCP2023/}
\end{itemize}
\end{itemize}
\end{frame}


%%   }

%% \clearpage
%% \mode<all>{
%%   \part{More Global Constraints}
%% }
%% \begin{frame}
%%   \partpage
%% \end{frame}
%% \mode<all>{
%%   \begin{frame}
\frametitle{Objectives}
\begin{itemize}
\item Overview of Core Constraint Programming
\item Three Main Concepts
\begin{itemize}
\item Constraint Propagation
\item Global Constraints
\item Customizing Search
\end{itemize}
%\item Get Some Experience with MiniZinc
\item Topics will be treated in more detail in later parts of the school
\item Based on Examples, not Formal Description
\end{itemize}
\end{frame}

\begin{frame}
\frametitle{Outline}
\begin{itemize}
\item Why Constraint Programming?
\item Constraint Propagation
\item Global Constraints
\item Customizing Search
%\item Choices in Modelling
%\item Decomposition
%\item More Global Constraints
%\item Methodology
%\item What is missing?
\end{itemize}
\end{frame}

% \begin{frame}
  % \frametitle{Using MiniZinc IDE}
  % \begin{itemize}
  % \item Developed in the Australian NICTA project
  % \item Maintained by Monash University
  % \item Modelling tool with multiple back-end solvers
  % \item Available from \url{https://www.minizinc.org/}
  % \end{itemize}
% \end{frame}


% \begin{frame}
% \frametitle{Examples in ECLiPSe}
% \begin{itemize}
% \item Open sourced constraint programming language
% \item Development goes back to 1985
% \item ECRC, ICL, IC-Parc, PTL, Cisco
% \item \url{https://eclipseclp.org/}
% \item Specialities
% \begin{itemize}
% \item Develop new solvers for specific domains
% \item Integration with MIP
% \end{itemize}
% \item Not included in bundled MiniZinc IDE
% \item Specialized visualization tools used here
  % \begin{itemize}
  % \item CP-Viz, Simonis et al. 2010
    % \end{itemize}
% \end{itemize}
% \end{frame}

\begin{frame}
\frametitle{Tutorial Based on ECLiPSe ELearning Course}
\begin{itemize}
\item Self-study course in constraint programming
\item Supported by Cisco Systems and Silicon Valley Community Foundation
\item Multi-media format, video lectures, slides, handout etc
\item {\small \url{https://eclipseclp.org/ELearning/index.html}}
\end{itemize}
\end{frame}

\begin{frame}
\frametitle{Also Part of CRT-AI Constraint Week}
\begin{itemize}
\item Annual one week course on CP and Optimization in Ireland
\item Part of national training program for PhD students in AI
\item \url{https://www.crt-ai.ie/}
\end{itemize}
\end{frame}



\begin{frame}
\frametitle{Constraint Programming - in a nutshell}
\begin{itemize}
\item Declarative description of problems with
\begin{itemize}
\item {\em Variables} which range over (finite) sets of values
\item {\em Constraints} over subsets of variables which restrict possible value combinations
\item A {\em solution} is a value assignment which satisfies all constraints
\end{itemize}

\item Constraint propagation/reasoning
\begin{itemize}
\item Removing inconsistent values for variables
\item Detect failure if constraint can not be satisfied
\item Interaction of constraints via shared variables
\item Incomplete
\end{itemize}

\item Search
\begin{itemize}
\item User controlled assignment of values to variables
\item Each step triggers constraint propagation 
\end{itemize}
\item Different domains require/allow different methods
\end{itemize}
\end{frame}

\begin{frame}
  \frametitle{Constraint Programming is Different}
  \begin{itemize}
  \item Declarative Programming
    \begin{itemize}
    \item Concentrate on what you want
      \item Not how to get there
      \item Program != Algorithm
      \item Program = Model
    \end{itemize}
    \item Applied to Combinatorial Problems
      \begin{itemize}
        \item No complete polynomial algorithms known (exist?)
        \item CP less ad-hoc than heuristics
        \item Models can evolve
  \end{itemize}
  \end{itemize}
  \end{frame}
    
\begin{frame}
  \frametitle{A Subtractive Process}
  \begin{textblock}{4}(8,-3)
    \includegraphics[width=4cm]{../introduction/images/stages}
  \end{textblock}
  \vfill
  \begin{quote}
    ``Oh, bosh, as Mr. Ruskin says. Sculpture, per se, is the simplest thing in the world. All you have to do is to take a big chunk of marble and a hammer and chisel, make up your mind what you are about to create and chip off all the marble you don't want.''-Paris Gaulois.
  \end{quote}
  
  {\tiny Source: \url{https://quoteinvestigator.com/2014/06/22/chip-away/}}
\end{frame}

\begin{frame}
\frametitle{Basic Process}
\begin{center}
\begin{tikzpicture}
\node[shape=rectangle,fill=pantone127-4] (problem) at (2,5) {Problem};
\node (human) at (2,4) {Human};
\node[shape=rectangle,fill=pantone127-4] (model) at (2,3) {Model};
\node (solver) at (2,2) {Constraint Solver/Search};
\node[shape=rectangle,fill=pantone157-8] (solution) at (2,1) {Solution};
\draw[-] (problem) -- (human);
\draw[->] (human) -- (model);
\draw[-] (model) -- (solver);
\draw[->] (solver) -- (solution);
\end{tikzpicture}
\end{center}
\end{frame}

\begin{frame}
\frametitle{More Realistic}
\begin{center}
\begin{tikzpicture}[xscale=4.0]
\node[shape=rectangle,fill=pantone127-4] (problem) at (2,5) {Problem};
\node (human) at (2,4) {Human};
\node[shape=rectangle,fill=pantone127-4] (model) at (2,3) {Model};
\node (solver) at (2,2) {Constraint Solver/Search};
\node[shape=rectangle,fill=pantone24-8] (hangs) at (1,1) {Hangs};
\node[shape=rectangle,fill=pantone157-8] (solution) at (2,1) {Solution};
\node[shape=rectangle,fill=pantone24-8] (wrong) at (3,1) {Wrong Solution};
\draw[-] (problem) -- (human);
\draw[->] (human) -- (model);
\draw[-] (model) -- (solver);
\draw[->] (solver) -- (hangs);
\draw[->] (solver) -- (solution);
\draw[->] (solver) -- (wrong);
\draw[->] (hangs) -- (1,4) -- (human);
\draw[->] (wrong) -- (3,4) -- (human);
\end{tikzpicture}
\end{center}
\end{frame}

\begin{frame}
\frametitle{Dual Role of Model}
\begin{itemize}
\item Allows Human to Express Problem
\begin{itemize}
\item Close to Problem Domain
\item Constraints as Abstractions
\end{itemize}
\item Allows Solver to Execute
\begin{itemize}
\item Variables as Communication Mechanism
\item Constraints as Algorithms
\end{itemize}
\end{itemize}
\end{frame}

\begin{frame}
\frametitle{Modelling Frameworks}
\begin{itemize}
\item \href{https://www.minizinc.org/}{MiniZinc} (NICTA, Monash University, Australia)
\item \href{https://github.com/eomahony/Numberjack}{NumberJack} (Insight, Ireland) 
\item \href{https://www-users.york.ac.uk/peter.nightingale/savilerow//}{EssencePrime/SavilleRow} (UK) 
\item \href{https://cpmpy.readthedocs.io/en/latest/}{CPMpy} (KU Leuven)
\item Allow use of multiple back-end solvers
\item Compile model into variants for each solver
\item A priori solver independent model(CP, MIP, SAT)
\end{itemize}
\end{frame} 

\begin{frame}
\frametitle{Framework Process}
\begin{center}
\begin{tikzpicture}[xscale=2]
\node[shape=rectangle,fill=pantone127-4] (problem) at (2.5,5) {Problem};
\node (human) at (2.5,4) {Human};
\node[shape=rectangle,fill=pantone127-4] (model) at (2.5,3) {Model};
\node (compile) at (2.5,2) {Compile/Reformulate};
\node (solver1) at (1,1) {CP};
\node (solver2) at (2,1) {MIP};
\node (solver3) at (3,1) {SAT};
\node (solver4) at (4,1) {Other};
\node[shape=rectangle,fill=pantone157-8] (solution1) at (1,0) {Solution};
\node[shape=rectangle,fill=pantone157-8] (solution2) at (2,0) {Solution};
\node[shape=rectangle,fill=pantone157-8] (solution3) at (3,0) {Solution};
\node[shape=rectangle,fill=pantone157-8] (solution4) at (4,0) {Solution};
\draw[-] (problem) -- (human);
\draw[->] (human) -- (model);
\draw[-] (model) -- (compile);
\draw[-] (compile) -- (solver1);
\draw[-] (compile) -- (solver2);
\draw[-] (compile) -- (solver3);
\draw[-] (compile) -- (solver4);
\draw[->] (solver1) -- (solution1);
\draw[->] (solver2) -- (solution2);
\draw[->] (solver3) -- (solution3);
\draw[->] (solver4) -- (solution4);
\end{tikzpicture}
\end{center}
\end{frame}

% \begin{frame}
  % \frametitle{Do It Now!}
  % \begin{itemize}
  % \item Download and install Minizinc
  % \item \url{https://www.minizinc.org/}
  % \end{itemize}
% \end{frame}


\begin{frame}
\frametitle{Why use Puzzles as Examples?}
\begin{itemize}
\item Easy to understand the problem
\item Solvable by hand without specialized knowledge
\item Possible to compare automated to manual solving process
\end{itemize}

\begin{quote}
The puzzle, though inanimate, is presented as a solvable problem without lasting negative consequences, a very low-risk low-reward situation. By being a puzzle, the object is attempting to convince the user that it must be completed.
\end{quote}
{\tiny Source: \href{https://www.everydayrhetoric.com/2020/03/the-rhetoric-of-puzzles/}{Every Day Rhetoric}}
\end{frame}


\mode<all>{
\part{Basic Constraint Propagation}
}
\begin{frame}
\partpage
\end{frame}
\mode<all>{
\input{../sendmore/body}
}

\mode<all>{
\part{Global Constraints}
}
\begin{frame}
\partpage
\end{frame}
\mode<all>{
\input{../sudoku/body}
}

\clearpage
\mode<all>{
\part{Customizing Search}
}
\begin{frame}
\partpage
\end{frame}
\mode<all>{
\input{../nqueen/body}
}


%% \clearpage
%% \mode<all>{
%%   \part{Problem Decomposition}
%% }
%% \begin{frame}
%%   \partpage
%% \end{frame}
%% \mode<all>{
%%   \input{../party/body}
%%   }

%% \clearpage
%% \mode<all>{
%%   \part{More Global Constraints}
%% }
%% \begin{frame}
%%   \partpage
%% \end{frame}
%% \mode<all>{
%%   \input{../car/body}
%%   }

%% \clearpage
%% \mode<all>{
%%   \part{Methodology}
%% }
%% \begin{frame}
%%   \partpage
%% \end{frame}
%% \mode<all>{
%%   \input{../methodology/body}
%%   }

\clearpage
\mode<all>{
\part{What is missing?}
}
\begin{frame}
\partpage
\end{frame}

\begin{frame}
\frametitle{Many Specialized Topics}
\begin{itemize}
\item How to design efficient core engine
\item Hybrids with LP/MIP tools
\item Hybrids with SAT
\item Symmetry breaking
\item Use of MDD/BDD to encode sets of solutions
\item High level modelling tools
\item Debugging/visualization
\end{itemize}
\end{frame}

\begin{frame}
\frametitle{Reformulation}
\begin{itemize}
\item Just because the user has modelled it this way, it doesn't mean we have to solve it that way
\begin{itemize}
\item Replace some constraint(s) by other, equivalent constraints
\item Because we don't have that constraint in our system
\item For performance
\end{itemize}
\end{itemize}
\end{frame}

\begin{frame}
\frametitle{Learning}
\begin{itemize}
\item While solving the problem we can learn how to strengthen the model/search 
\begin{itemize}
\item Understand which constraints/method contribute to propagation and change schedule
\item Learn no-good constraints by explaining failure
\item Adapt search strategy based on search experience
\end{itemize}
\end{itemize}
\end{frame}

%% \begin{frame}
%% \frametitle{Refined Process}
%% \begin{center}
%% \begin{tikzpicture}
%% \node[shape=rectangle,fill=pantone127-4] (problem) at (3,7) {Problem};
%% \node (human) at (3,6) {Human};
%% \node[shape=rectangle,fill=pantone127-4] (model) at (3,5) {User Model};
%% \node (reformulation) at (3,4) {Reformulation};
%% \node[shape=rectangle,fill=pantone127-4] (implementation) at (3,3) {Implementation Model};
%% \node (solver) at (3,2) {Constraint Solver/Search};
%% \node[shape=rectangle,fill=pantone157-8] (solution) at (3,1) {Solution};
%% \draw[-] (problem) -- (human);
%% \draw[->] (human) -- (model);
%% \draw[-] (model) -- (reformulation);
%% \draw[->] (reformulation) -- (implementation);
%% \draw[-] (implementation) -- (solver);
%% \draw[->] (solver) -- (0,2) -- (0,3) -- (implementation);
%% \draw[->] (solver) -- (solution);
%% \end{tikzpicture}
%% \end{center}
%% \end{frame}

%\begin{frame}
%\frametitle{What is CP actually used for?}
%\url{http://hsimonis.wordpress.com}
%\includegraphics[width=10cm]{../introduction/images/blog}
%\end{frame}

%% \begin{frame}
%% \frametitle{Exercises}
%% \begin{itemize}
%% \item Install Constraint Solver of your Choice
%% \item MiniZinc (MiningZinc) will be used in later sessions
%% \item Run Sendmoremoney, Sudoku, Queens examples
%% \item Can you choose which propagator to use for alldifferent?
%% \item Can you use different search strategies?
%% \end{itemize}
%% \end{frame}

\begin{frame}
\frametitle{More Learning Resources}
\begin{itemize}
\item Survey of Methods, Resources, and Formats for Teaching Constraint Programming
\begin{itemize}
\item by Tejas Santanam, Helmut Simonis
\item \url{https://doi.org/10.48550/arXiv.2403.12717}
\item Based on survey of community for WTCP 2023
\item \url{https://hsimonis.github.io/WTCP2023/}
\end{itemize}
\end{itemize}
\end{frame}


%%   }

%% \clearpage
%% \mode<all>{
%%   \part{Methodology}
%% }
%% \begin{frame}
%%   \partpage
%% \end{frame}
%% \mode<all>{
%%   \begin{frame}
\frametitle{Objectives}
\begin{itemize}
\item Overview of Core Constraint Programming
\item Three Main Concepts
\begin{itemize}
\item Constraint Propagation
\item Global Constraints
\item Customizing Search
\end{itemize}
%\item Get Some Experience with MiniZinc
\item Topics will be treated in more detail in later parts of the school
\item Based on Examples, not Formal Description
\end{itemize}
\end{frame}

\begin{frame}
\frametitle{Outline}
\begin{itemize}
\item Why Constraint Programming?
\item Constraint Propagation
\item Global Constraints
\item Customizing Search
%\item Choices in Modelling
%\item Decomposition
%\item More Global Constraints
%\item Methodology
%\item What is missing?
\end{itemize}
\end{frame}

% \begin{frame}
  % \frametitle{Using MiniZinc IDE}
  % \begin{itemize}
  % \item Developed in the Australian NICTA project
  % \item Maintained by Monash University
  % \item Modelling tool with multiple back-end solvers
  % \item Available from \url{https://www.minizinc.org/}
  % \end{itemize}
% \end{frame}


% \begin{frame}
% \frametitle{Examples in ECLiPSe}
% \begin{itemize}
% \item Open sourced constraint programming language
% \item Development goes back to 1985
% \item ECRC, ICL, IC-Parc, PTL, Cisco
% \item \url{https://eclipseclp.org/}
% \item Specialities
% \begin{itemize}
% \item Develop new solvers for specific domains
% \item Integration with MIP
% \end{itemize}
% \item Not included in bundled MiniZinc IDE
% \item Specialized visualization tools used here
  % \begin{itemize}
  % \item CP-Viz, Simonis et al. 2010
    % \end{itemize}
% \end{itemize}
% \end{frame}

\begin{frame}
\frametitle{Tutorial Based on ECLiPSe ELearning Course}
\begin{itemize}
\item Self-study course in constraint programming
\item Supported by Cisco Systems and Silicon Valley Community Foundation
\item Multi-media format, video lectures, slides, handout etc
\item {\small \url{https://eclipseclp.org/ELearning/index.html}}
\end{itemize}
\end{frame}

\begin{frame}
\frametitle{Also Part of CRT-AI Constraint Week}
\begin{itemize}
\item Annual one week course on CP and Optimization in Ireland
\item Part of national training program for PhD students in AI
\item \url{https://www.crt-ai.ie/}
\end{itemize}
\end{frame}



\begin{frame}
\frametitle{Constraint Programming - in a nutshell}
\begin{itemize}
\item Declarative description of problems with
\begin{itemize}
\item {\em Variables} which range over (finite) sets of values
\item {\em Constraints} over subsets of variables which restrict possible value combinations
\item A {\em solution} is a value assignment which satisfies all constraints
\end{itemize}

\item Constraint propagation/reasoning
\begin{itemize}
\item Removing inconsistent values for variables
\item Detect failure if constraint can not be satisfied
\item Interaction of constraints via shared variables
\item Incomplete
\end{itemize}

\item Search
\begin{itemize}
\item User controlled assignment of values to variables
\item Each step triggers constraint propagation 
\end{itemize}
\item Different domains require/allow different methods
\end{itemize}
\end{frame}

\begin{frame}
  \frametitle{Constraint Programming is Different}
  \begin{itemize}
  \item Declarative Programming
    \begin{itemize}
    \item Concentrate on what you want
      \item Not how to get there
      \item Program != Algorithm
      \item Program = Model
    \end{itemize}
    \item Applied to Combinatorial Problems
      \begin{itemize}
        \item No complete polynomial algorithms known (exist?)
        \item CP less ad-hoc than heuristics
        \item Models can evolve
  \end{itemize}
  \end{itemize}
  \end{frame}
    
\begin{frame}
  \frametitle{A Subtractive Process}
  \begin{textblock}{4}(8,-3)
    \includegraphics[width=4cm]{../introduction/images/stages}
  \end{textblock}
  \vfill
  \begin{quote}
    ``Oh, bosh, as Mr. Ruskin says. Sculpture, per se, is the simplest thing in the world. All you have to do is to take a big chunk of marble and a hammer and chisel, make up your mind what you are about to create and chip off all the marble you don't want.''-Paris Gaulois.
  \end{quote}
  
  {\tiny Source: \url{https://quoteinvestigator.com/2014/06/22/chip-away/}}
\end{frame}

\begin{frame}
\frametitle{Basic Process}
\begin{center}
\begin{tikzpicture}
\node[shape=rectangle,fill=pantone127-4] (problem) at (2,5) {Problem};
\node (human) at (2,4) {Human};
\node[shape=rectangle,fill=pantone127-4] (model) at (2,3) {Model};
\node (solver) at (2,2) {Constraint Solver/Search};
\node[shape=rectangle,fill=pantone157-8] (solution) at (2,1) {Solution};
\draw[-] (problem) -- (human);
\draw[->] (human) -- (model);
\draw[-] (model) -- (solver);
\draw[->] (solver) -- (solution);
\end{tikzpicture}
\end{center}
\end{frame}

\begin{frame}
\frametitle{More Realistic}
\begin{center}
\begin{tikzpicture}[xscale=4.0]
\node[shape=rectangle,fill=pantone127-4] (problem) at (2,5) {Problem};
\node (human) at (2,4) {Human};
\node[shape=rectangle,fill=pantone127-4] (model) at (2,3) {Model};
\node (solver) at (2,2) {Constraint Solver/Search};
\node[shape=rectangle,fill=pantone24-8] (hangs) at (1,1) {Hangs};
\node[shape=rectangle,fill=pantone157-8] (solution) at (2,1) {Solution};
\node[shape=rectangle,fill=pantone24-8] (wrong) at (3,1) {Wrong Solution};
\draw[-] (problem) -- (human);
\draw[->] (human) -- (model);
\draw[-] (model) -- (solver);
\draw[->] (solver) -- (hangs);
\draw[->] (solver) -- (solution);
\draw[->] (solver) -- (wrong);
\draw[->] (hangs) -- (1,4) -- (human);
\draw[->] (wrong) -- (3,4) -- (human);
\end{tikzpicture}
\end{center}
\end{frame}

\begin{frame}
\frametitle{Dual Role of Model}
\begin{itemize}
\item Allows Human to Express Problem
\begin{itemize}
\item Close to Problem Domain
\item Constraints as Abstractions
\end{itemize}
\item Allows Solver to Execute
\begin{itemize}
\item Variables as Communication Mechanism
\item Constraints as Algorithms
\end{itemize}
\end{itemize}
\end{frame}

\begin{frame}
\frametitle{Modelling Frameworks}
\begin{itemize}
\item \href{https://www.minizinc.org/}{MiniZinc} (NICTA, Monash University, Australia)
\item \href{https://github.com/eomahony/Numberjack}{NumberJack} (Insight, Ireland) 
\item \href{https://www-users.york.ac.uk/peter.nightingale/savilerow//}{EssencePrime/SavilleRow} (UK) 
\item \href{https://cpmpy.readthedocs.io/en/latest/}{CPMpy} (KU Leuven)
\item Allow use of multiple back-end solvers
\item Compile model into variants for each solver
\item A priori solver independent model(CP, MIP, SAT)
\end{itemize}
\end{frame} 

\begin{frame}
\frametitle{Framework Process}
\begin{center}
\begin{tikzpicture}[xscale=2]
\node[shape=rectangle,fill=pantone127-4] (problem) at (2.5,5) {Problem};
\node (human) at (2.5,4) {Human};
\node[shape=rectangle,fill=pantone127-4] (model) at (2.5,3) {Model};
\node (compile) at (2.5,2) {Compile/Reformulate};
\node (solver1) at (1,1) {CP};
\node (solver2) at (2,1) {MIP};
\node (solver3) at (3,1) {SAT};
\node (solver4) at (4,1) {Other};
\node[shape=rectangle,fill=pantone157-8] (solution1) at (1,0) {Solution};
\node[shape=rectangle,fill=pantone157-8] (solution2) at (2,0) {Solution};
\node[shape=rectangle,fill=pantone157-8] (solution3) at (3,0) {Solution};
\node[shape=rectangle,fill=pantone157-8] (solution4) at (4,0) {Solution};
\draw[-] (problem) -- (human);
\draw[->] (human) -- (model);
\draw[-] (model) -- (compile);
\draw[-] (compile) -- (solver1);
\draw[-] (compile) -- (solver2);
\draw[-] (compile) -- (solver3);
\draw[-] (compile) -- (solver4);
\draw[->] (solver1) -- (solution1);
\draw[->] (solver2) -- (solution2);
\draw[->] (solver3) -- (solution3);
\draw[->] (solver4) -- (solution4);
\end{tikzpicture}
\end{center}
\end{frame}

% \begin{frame}
  % \frametitle{Do It Now!}
  % \begin{itemize}
  % \item Download and install Minizinc
  % \item \url{https://www.minizinc.org/}
  % \end{itemize}
% \end{frame}


\begin{frame}
\frametitle{Why use Puzzles as Examples?}
\begin{itemize}
\item Easy to understand the problem
\item Solvable by hand without specialized knowledge
\item Possible to compare automated to manual solving process
\end{itemize}

\begin{quote}
The puzzle, though inanimate, is presented as a solvable problem without lasting negative consequences, a very low-risk low-reward situation. By being a puzzle, the object is attempting to convince the user that it must be completed.
\end{quote}
{\tiny Source: \href{https://www.everydayrhetoric.com/2020/03/the-rhetoric-of-puzzles/}{Every Day Rhetoric}}
\end{frame}


\mode<all>{
\part{Basic Constraint Propagation}
}
\begin{frame}
\partpage
\end{frame}
\mode<all>{
\input{../sendmore/body}
}

\mode<all>{
\part{Global Constraints}
}
\begin{frame}
\partpage
\end{frame}
\mode<all>{
\input{../sudoku/body}
}

\clearpage
\mode<all>{
\part{Customizing Search}
}
\begin{frame}
\partpage
\end{frame}
\mode<all>{
\input{../nqueen/body}
}


%% \clearpage
%% \mode<all>{
%%   \part{Problem Decomposition}
%% }
%% \begin{frame}
%%   \partpage
%% \end{frame}
%% \mode<all>{
%%   \input{../party/body}
%%   }

%% \clearpage
%% \mode<all>{
%%   \part{More Global Constraints}
%% }
%% \begin{frame}
%%   \partpage
%% \end{frame}
%% \mode<all>{
%%   \input{../car/body}
%%   }

%% \clearpage
%% \mode<all>{
%%   \part{Methodology}
%% }
%% \begin{frame}
%%   \partpage
%% \end{frame}
%% \mode<all>{
%%   \input{../methodology/body}
%%   }

\clearpage
\mode<all>{
\part{What is missing?}
}
\begin{frame}
\partpage
\end{frame}

\begin{frame}
\frametitle{Many Specialized Topics}
\begin{itemize}
\item How to design efficient core engine
\item Hybrids with LP/MIP tools
\item Hybrids with SAT
\item Symmetry breaking
\item Use of MDD/BDD to encode sets of solutions
\item High level modelling tools
\item Debugging/visualization
\end{itemize}
\end{frame}

\begin{frame}
\frametitle{Reformulation}
\begin{itemize}
\item Just because the user has modelled it this way, it doesn't mean we have to solve it that way
\begin{itemize}
\item Replace some constraint(s) by other, equivalent constraints
\item Because we don't have that constraint in our system
\item For performance
\end{itemize}
\end{itemize}
\end{frame}

\begin{frame}
\frametitle{Learning}
\begin{itemize}
\item While solving the problem we can learn how to strengthen the model/search 
\begin{itemize}
\item Understand which constraints/method contribute to propagation and change schedule
\item Learn no-good constraints by explaining failure
\item Adapt search strategy based on search experience
\end{itemize}
\end{itemize}
\end{frame}

%% \begin{frame}
%% \frametitle{Refined Process}
%% \begin{center}
%% \begin{tikzpicture}
%% \node[shape=rectangle,fill=pantone127-4] (problem) at (3,7) {Problem};
%% \node (human) at (3,6) {Human};
%% \node[shape=rectangle,fill=pantone127-4] (model) at (3,5) {User Model};
%% \node (reformulation) at (3,4) {Reformulation};
%% \node[shape=rectangle,fill=pantone127-4] (implementation) at (3,3) {Implementation Model};
%% \node (solver) at (3,2) {Constraint Solver/Search};
%% \node[shape=rectangle,fill=pantone157-8] (solution) at (3,1) {Solution};
%% \draw[-] (problem) -- (human);
%% \draw[->] (human) -- (model);
%% \draw[-] (model) -- (reformulation);
%% \draw[->] (reformulation) -- (implementation);
%% \draw[-] (implementation) -- (solver);
%% \draw[->] (solver) -- (0,2) -- (0,3) -- (implementation);
%% \draw[->] (solver) -- (solution);
%% \end{tikzpicture}
%% \end{center}
%% \end{frame}

%\begin{frame}
%\frametitle{What is CP actually used for?}
%\url{http://hsimonis.wordpress.com}
%\includegraphics[width=10cm]{../introduction/images/blog}
%\end{frame}

%% \begin{frame}
%% \frametitle{Exercises}
%% \begin{itemize}
%% \item Install Constraint Solver of your Choice
%% \item MiniZinc (MiningZinc) will be used in later sessions
%% \item Run Sendmoremoney, Sudoku, Queens examples
%% \item Can you choose which propagator to use for alldifferent?
%% \item Can you use different search strategies?
%% \end{itemize}
%% \end{frame}

\begin{frame}
\frametitle{More Learning Resources}
\begin{itemize}
\item Survey of Methods, Resources, and Formats for Teaching Constraint Programming
\begin{itemize}
\item by Tejas Santanam, Helmut Simonis
\item \url{https://doi.org/10.48550/arXiv.2403.12717}
\item Based on survey of community for WTCP 2023
\item \url{https://hsimonis.github.io/WTCP2023/}
\end{itemize}
\end{itemize}
\end{frame}


%%   }

\clearpage
\mode<all>{
\part{What is missing?}
}
\begin{frame}
\partpage
\end{frame}

\begin{frame}
\frametitle{Many Specialized Topics}
\begin{itemize}
\item How to design efficient core engine
\item Hybrids with LP/MIP tools
\item Hybrids with SAT
\item Symmetry breaking
\item Use of MDD/BDD to encode sets of solutions
\item High level modelling tools
\item Debugging/visualization
\end{itemize}
\end{frame}

\begin{frame}
\frametitle{Reformulation}
\begin{itemize}
\item Just because the user has modelled it this way, it doesn't mean we have to solve it that way
\begin{itemize}
\item Replace some constraint(s) by other, equivalent constraints
\item Because we don't have that constraint in our system
\item For performance
\end{itemize}
\end{itemize}
\end{frame}

\begin{frame}
\frametitle{Learning}
\begin{itemize}
\item While solving the problem we can learn how to strengthen the model/search 
\begin{itemize}
\item Understand which constraints/method contribute to propagation and change schedule
\item Learn no-good constraints by explaining failure
\item Adapt search strategy based on search experience
\end{itemize}
\end{itemize}
\end{frame}

%% \begin{frame}
%% \frametitle{Refined Process}
%% \begin{center}
%% \begin{tikzpicture}
%% \node[shape=rectangle,fill=pantone127-4] (problem) at (3,7) {Problem};
%% \node (human) at (3,6) {Human};
%% \node[shape=rectangle,fill=pantone127-4] (model) at (3,5) {User Model};
%% \node (reformulation) at (3,4) {Reformulation};
%% \node[shape=rectangle,fill=pantone127-4] (implementation) at (3,3) {Implementation Model};
%% \node (solver) at (3,2) {Constraint Solver/Search};
%% \node[shape=rectangle,fill=pantone157-8] (solution) at (3,1) {Solution};
%% \draw[-] (problem) -- (human);
%% \draw[->] (human) -- (model);
%% \draw[-] (model) -- (reformulation);
%% \draw[->] (reformulation) -- (implementation);
%% \draw[-] (implementation) -- (solver);
%% \draw[->] (solver) -- (0,2) -- (0,3) -- (implementation);
%% \draw[->] (solver) -- (solution);
%% \end{tikzpicture}
%% \end{center}
%% \end{frame}

%\begin{frame}
%\frametitle{What is CP actually used for?}
%\url{http://hsimonis.wordpress.com}
%\includegraphics[width=10cm]{../introduction/images/blog}
%\end{frame}

%% \begin{frame}
%% \frametitle{Exercises}
%% \begin{itemize}
%% \item Install Constraint Solver of your Choice
%% \item MiniZinc (MiningZinc) will be used in later sessions
%% \item Run Sendmoremoney, Sudoku, Queens examples
%% \item Can you choose which propagator to use for alldifferent?
%% \item Can you use different search strategies?
%% \end{itemize}
%% \end{frame}

\begin{frame}
\frametitle{More Learning Resources}
\begin{itemize}
\item Survey of Methods, Resources, and Formats for Teaching Constraint Programming
\begin{itemize}
\item by Tejas Santanam, Helmut Simonis
\item \url{https://doi.org/10.48550/arXiv.2403.12717}
\item Based on survey of community for WTCP 2023
\item \url{https://hsimonis.github.io/WTCP2023/}
\end{itemize}
\end{itemize}
\end{frame}


%%   }

%% \clearpage
%% \mode<all>{
%%   \part{More Global Constraints}
%% }
%% \begin{frame}
%%   \partpage
%% \end{frame}
%% \mode<all>{
%%   \begin{frame}
\frametitle{Objectives}
\begin{itemize}
\item Overview of Core Constraint Programming
\item Three Main Concepts
\begin{itemize}
\item Constraint Propagation
\item Global Constraints
\item Customizing Search
\end{itemize}
%\item Get Some Experience with MiniZinc
\item Topics will be treated in more detail in later parts of the school
\item Based on Examples, not Formal Description
\end{itemize}
\end{frame}

\begin{frame}
\frametitle{Outline}
\begin{itemize}
\item Why Constraint Programming?
\item Constraint Propagation
\item Global Constraints
\item Customizing Search
%\item Choices in Modelling
%\item Decomposition
%\item More Global Constraints
%\item Methodology
%\item What is missing?
\end{itemize}
\end{frame}

% \begin{frame}
  % \frametitle{Using MiniZinc IDE}
  % \begin{itemize}
  % \item Developed in the Australian NICTA project
  % \item Maintained by Monash University
  % \item Modelling tool with multiple back-end solvers
  % \item Available from \url{https://www.minizinc.org/}
  % \end{itemize}
% \end{frame}


% \begin{frame}
% \frametitle{Examples in ECLiPSe}
% \begin{itemize}
% \item Open sourced constraint programming language
% \item Development goes back to 1985
% \item ECRC, ICL, IC-Parc, PTL, Cisco
% \item \url{https://eclipseclp.org/}
% \item Specialities
% \begin{itemize}
% \item Develop new solvers for specific domains
% \item Integration with MIP
% \end{itemize}
% \item Not included in bundled MiniZinc IDE
% \item Specialized visualization tools used here
  % \begin{itemize}
  % \item CP-Viz, Simonis et al. 2010
    % \end{itemize}
% \end{itemize}
% \end{frame}

\begin{frame}
\frametitle{Tutorial Based on ECLiPSe ELearning Course}
\begin{itemize}
\item Self-study course in constraint programming
\item Supported by Cisco Systems and Silicon Valley Community Foundation
\item Multi-media format, video lectures, slides, handout etc
\item {\small \url{https://eclipseclp.org/ELearning/index.html}}
\end{itemize}
\end{frame}

\begin{frame}
\frametitle{Also Part of CRT-AI Constraint Week}
\begin{itemize}
\item Annual one week course on CP and Optimization in Ireland
\item Part of national training program for PhD students in AI
\item \url{https://www.crt-ai.ie/}
\end{itemize}
\end{frame}



\begin{frame}
\frametitle{Constraint Programming - in a nutshell}
\begin{itemize}
\item Declarative description of problems with
\begin{itemize}
\item {\em Variables} which range over (finite) sets of values
\item {\em Constraints} over subsets of variables which restrict possible value combinations
\item A {\em solution} is a value assignment which satisfies all constraints
\end{itemize}

\item Constraint propagation/reasoning
\begin{itemize}
\item Removing inconsistent values for variables
\item Detect failure if constraint can not be satisfied
\item Interaction of constraints via shared variables
\item Incomplete
\end{itemize}

\item Search
\begin{itemize}
\item User controlled assignment of values to variables
\item Each step triggers constraint propagation 
\end{itemize}
\item Different domains require/allow different methods
\end{itemize}
\end{frame}

\begin{frame}
  \frametitle{Constraint Programming is Different}
  \begin{itemize}
  \item Declarative Programming
    \begin{itemize}
    \item Concentrate on what you want
      \item Not how to get there
      \item Program != Algorithm
      \item Program = Model
    \end{itemize}
    \item Applied to Combinatorial Problems
      \begin{itemize}
        \item No complete polynomial algorithms known (exist?)
        \item CP less ad-hoc than heuristics
        \item Models can evolve
  \end{itemize}
  \end{itemize}
  \end{frame}
    
\begin{frame}
  \frametitle{A Subtractive Process}
  \begin{textblock}{4}(8,-3)
    \includegraphics[width=4cm]{../introduction/images/stages}
  \end{textblock}
  \vfill
  \begin{quote}
    ``Oh, bosh, as Mr. Ruskin says. Sculpture, per se, is the simplest thing in the world. All you have to do is to take a big chunk of marble and a hammer and chisel, make up your mind what you are about to create and chip off all the marble you don't want.''-Paris Gaulois.
  \end{quote}
  
  {\tiny Source: \url{https://quoteinvestigator.com/2014/06/22/chip-away/}}
\end{frame}

\begin{frame}
\frametitle{Basic Process}
\begin{center}
\begin{tikzpicture}
\node[shape=rectangle,fill=pantone127-4] (problem) at (2,5) {Problem};
\node (human) at (2,4) {Human};
\node[shape=rectangle,fill=pantone127-4] (model) at (2,3) {Model};
\node (solver) at (2,2) {Constraint Solver/Search};
\node[shape=rectangle,fill=pantone157-8] (solution) at (2,1) {Solution};
\draw[-] (problem) -- (human);
\draw[->] (human) -- (model);
\draw[-] (model) -- (solver);
\draw[->] (solver) -- (solution);
\end{tikzpicture}
\end{center}
\end{frame}

\begin{frame}
\frametitle{More Realistic}
\begin{center}
\begin{tikzpicture}[xscale=4.0]
\node[shape=rectangle,fill=pantone127-4] (problem) at (2,5) {Problem};
\node (human) at (2,4) {Human};
\node[shape=rectangle,fill=pantone127-4] (model) at (2,3) {Model};
\node (solver) at (2,2) {Constraint Solver/Search};
\node[shape=rectangle,fill=pantone24-8] (hangs) at (1,1) {Hangs};
\node[shape=rectangle,fill=pantone157-8] (solution) at (2,1) {Solution};
\node[shape=rectangle,fill=pantone24-8] (wrong) at (3,1) {Wrong Solution};
\draw[-] (problem) -- (human);
\draw[->] (human) -- (model);
\draw[-] (model) -- (solver);
\draw[->] (solver) -- (hangs);
\draw[->] (solver) -- (solution);
\draw[->] (solver) -- (wrong);
\draw[->] (hangs) -- (1,4) -- (human);
\draw[->] (wrong) -- (3,4) -- (human);
\end{tikzpicture}
\end{center}
\end{frame}

\begin{frame}
\frametitle{Dual Role of Model}
\begin{itemize}
\item Allows Human to Express Problem
\begin{itemize}
\item Close to Problem Domain
\item Constraints as Abstractions
\end{itemize}
\item Allows Solver to Execute
\begin{itemize}
\item Variables as Communication Mechanism
\item Constraints as Algorithms
\end{itemize}
\end{itemize}
\end{frame}

\begin{frame}
\frametitle{Modelling Frameworks}
\begin{itemize}
\item \href{https://www.minizinc.org/}{MiniZinc} (NICTA, Monash University, Australia)
\item \href{https://github.com/eomahony/Numberjack}{NumberJack} (Insight, Ireland) 
\item \href{https://www-users.york.ac.uk/peter.nightingale/savilerow//}{EssencePrime/SavilleRow} (UK) 
\item \href{https://cpmpy.readthedocs.io/en/latest/}{CPMpy} (KU Leuven)
\item Allow use of multiple back-end solvers
\item Compile model into variants for each solver
\item A priori solver independent model(CP, MIP, SAT)
\end{itemize}
\end{frame} 

\begin{frame}
\frametitle{Framework Process}
\begin{center}
\begin{tikzpicture}[xscale=2]
\node[shape=rectangle,fill=pantone127-4] (problem) at (2.5,5) {Problem};
\node (human) at (2.5,4) {Human};
\node[shape=rectangle,fill=pantone127-4] (model) at (2.5,3) {Model};
\node (compile) at (2.5,2) {Compile/Reformulate};
\node (solver1) at (1,1) {CP};
\node (solver2) at (2,1) {MIP};
\node (solver3) at (3,1) {SAT};
\node (solver4) at (4,1) {Other};
\node[shape=rectangle,fill=pantone157-8] (solution1) at (1,0) {Solution};
\node[shape=rectangle,fill=pantone157-8] (solution2) at (2,0) {Solution};
\node[shape=rectangle,fill=pantone157-8] (solution3) at (3,0) {Solution};
\node[shape=rectangle,fill=pantone157-8] (solution4) at (4,0) {Solution};
\draw[-] (problem) -- (human);
\draw[->] (human) -- (model);
\draw[-] (model) -- (compile);
\draw[-] (compile) -- (solver1);
\draw[-] (compile) -- (solver2);
\draw[-] (compile) -- (solver3);
\draw[-] (compile) -- (solver4);
\draw[->] (solver1) -- (solution1);
\draw[->] (solver2) -- (solution2);
\draw[->] (solver3) -- (solution3);
\draw[->] (solver4) -- (solution4);
\end{tikzpicture}
\end{center}
\end{frame}

% \begin{frame}
  % \frametitle{Do It Now!}
  % \begin{itemize}
  % \item Download and install Minizinc
  % \item \url{https://www.minizinc.org/}
  % \end{itemize}
% \end{frame}


\begin{frame}
\frametitle{Why use Puzzles as Examples?}
\begin{itemize}
\item Easy to understand the problem
\item Solvable by hand without specialized knowledge
\item Possible to compare automated to manual solving process
\end{itemize}

\begin{quote}
The puzzle, though inanimate, is presented as a solvable problem without lasting negative consequences, a very low-risk low-reward situation. By being a puzzle, the object is attempting to convince the user that it must be completed.
\end{quote}
{\tiny Source: \href{https://www.everydayrhetoric.com/2020/03/the-rhetoric-of-puzzles/}{Every Day Rhetoric}}
\end{frame}


\mode<all>{
\part{Basic Constraint Propagation}
}
\begin{frame}
\partpage
\end{frame}
\mode<all>{
\begin{frame}
\frametitle{Objectives}
\begin{itemize}
\item Overview of Core Constraint Programming
\item Three Main Concepts
\begin{itemize}
\item Constraint Propagation
\item Global Constraints
\item Customizing Search
\end{itemize}
%\item Get Some Experience with MiniZinc
\item Topics will be treated in more detail in later parts of the school
\item Based on Examples, not Formal Description
\end{itemize}
\end{frame}

\begin{frame}
\frametitle{Outline}
\begin{itemize}
\item Why Constraint Programming?
\item Constraint Propagation
\item Global Constraints
\item Customizing Search
%\item Choices in Modelling
%\item Decomposition
%\item More Global Constraints
%\item Methodology
%\item What is missing?
\end{itemize}
\end{frame}

% \begin{frame}
  % \frametitle{Using MiniZinc IDE}
  % \begin{itemize}
  % \item Developed in the Australian NICTA project
  % \item Maintained by Monash University
  % \item Modelling tool with multiple back-end solvers
  % \item Available from \url{https://www.minizinc.org/}
  % \end{itemize}
% \end{frame}


% \begin{frame}
% \frametitle{Examples in ECLiPSe}
% \begin{itemize}
% \item Open sourced constraint programming language
% \item Development goes back to 1985
% \item ECRC, ICL, IC-Parc, PTL, Cisco
% \item \url{https://eclipseclp.org/}
% \item Specialities
% \begin{itemize}
% \item Develop new solvers for specific domains
% \item Integration with MIP
% \end{itemize}
% \item Not included in bundled MiniZinc IDE
% \item Specialized visualization tools used here
  % \begin{itemize}
  % \item CP-Viz, Simonis et al. 2010
    % \end{itemize}
% \end{itemize}
% \end{frame}

\begin{frame}
\frametitle{Tutorial Based on ECLiPSe ELearning Course}
\begin{itemize}
\item Self-study course in constraint programming
\item Supported by Cisco Systems and Silicon Valley Community Foundation
\item Multi-media format, video lectures, slides, handout etc
\item {\small \url{https://eclipseclp.org/ELearning/index.html}}
\end{itemize}
\end{frame}

\begin{frame}
\frametitle{Also Part of CRT-AI Constraint Week}
\begin{itemize}
\item Annual one week course on CP and Optimization in Ireland
\item Part of national training program for PhD students in AI
\item \url{https://www.crt-ai.ie/}
\end{itemize}
\end{frame}



\begin{frame}
\frametitle{Constraint Programming - in a nutshell}
\begin{itemize}
\item Declarative description of problems with
\begin{itemize}
\item {\em Variables} which range over (finite) sets of values
\item {\em Constraints} over subsets of variables which restrict possible value combinations
\item A {\em solution} is a value assignment which satisfies all constraints
\end{itemize}

\item Constraint propagation/reasoning
\begin{itemize}
\item Removing inconsistent values for variables
\item Detect failure if constraint can not be satisfied
\item Interaction of constraints via shared variables
\item Incomplete
\end{itemize}

\item Search
\begin{itemize}
\item User controlled assignment of values to variables
\item Each step triggers constraint propagation 
\end{itemize}
\item Different domains require/allow different methods
\end{itemize}
\end{frame}

\begin{frame}
  \frametitle{Constraint Programming is Different}
  \begin{itemize}
  \item Declarative Programming
    \begin{itemize}
    \item Concentrate on what you want
      \item Not how to get there
      \item Program != Algorithm
      \item Program = Model
    \end{itemize}
    \item Applied to Combinatorial Problems
      \begin{itemize}
        \item No complete polynomial algorithms known (exist?)
        \item CP less ad-hoc than heuristics
        \item Models can evolve
  \end{itemize}
  \end{itemize}
  \end{frame}
    
\begin{frame}
  \frametitle{A Subtractive Process}
  \begin{textblock}{4}(8,-3)
    \includegraphics[width=4cm]{../introduction/images/stages}
  \end{textblock}
  \vfill
  \begin{quote}
    ``Oh, bosh, as Mr. Ruskin says. Sculpture, per se, is the simplest thing in the world. All you have to do is to take a big chunk of marble and a hammer and chisel, make up your mind what you are about to create and chip off all the marble you don't want.''-Paris Gaulois.
  \end{quote}
  
  {\tiny Source: \url{https://quoteinvestigator.com/2014/06/22/chip-away/}}
\end{frame}

\begin{frame}
\frametitle{Basic Process}
\begin{center}
\begin{tikzpicture}
\node[shape=rectangle,fill=pantone127-4] (problem) at (2,5) {Problem};
\node (human) at (2,4) {Human};
\node[shape=rectangle,fill=pantone127-4] (model) at (2,3) {Model};
\node (solver) at (2,2) {Constraint Solver/Search};
\node[shape=rectangle,fill=pantone157-8] (solution) at (2,1) {Solution};
\draw[-] (problem) -- (human);
\draw[->] (human) -- (model);
\draw[-] (model) -- (solver);
\draw[->] (solver) -- (solution);
\end{tikzpicture}
\end{center}
\end{frame}

\begin{frame}
\frametitle{More Realistic}
\begin{center}
\begin{tikzpicture}[xscale=4.0]
\node[shape=rectangle,fill=pantone127-4] (problem) at (2,5) {Problem};
\node (human) at (2,4) {Human};
\node[shape=rectangle,fill=pantone127-4] (model) at (2,3) {Model};
\node (solver) at (2,2) {Constraint Solver/Search};
\node[shape=rectangle,fill=pantone24-8] (hangs) at (1,1) {Hangs};
\node[shape=rectangle,fill=pantone157-8] (solution) at (2,1) {Solution};
\node[shape=rectangle,fill=pantone24-8] (wrong) at (3,1) {Wrong Solution};
\draw[-] (problem) -- (human);
\draw[->] (human) -- (model);
\draw[-] (model) -- (solver);
\draw[->] (solver) -- (hangs);
\draw[->] (solver) -- (solution);
\draw[->] (solver) -- (wrong);
\draw[->] (hangs) -- (1,4) -- (human);
\draw[->] (wrong) -- (3,4) -- (human);
\end{tikzpicture}
\end{center}
\end{frame}

\begin{frame}
\frametitle{Dual Role of Model}
\begin{itemize}
\item Allows Human to Express Problem
\begin{itemize}
\item Close to Problem Domain
\item Constraints as Abstractions
\end{itemize}
\item Allows Solver to Execute
\begin{itemize}
\item Variables as Communication Mechanism
\item Constraints as Algorithms
\end{itemize}
\end{itemize}
\end{frame}

\begin{frame}
\frametitle{Modelling Frameworks}
\begin{itemize}
\item \href{https://www.minizinc.org/}{MiniZinc} (NICTA, Monash University, Australia)
\item \href{https://github.com/eomahony/Numberjack}{NumberJack} (Insight, Ireland) 
\item \href{https://www-users.york.ac.uk/peter.nightingale/savilerow//}{EssencePrime/SavilleRow} (UK) 
\item \href{https://cpmpy.readthedocs.io/en/latest/}{CPMpy} (KU Leuven)
\item Allow use of multiple back-end solvers
\item Compile model into variants for each solver
\item A priori solver independent model(CP, MIP, SAT)
\end{itemize}
\end{frame} 

\begin{frame}
\frametitle{Framework Process}
\begin{center}
\begin{tikzpicture}[xscale=2]
\node[shape=rectangle,fill=pantone127-4] (problem) at (2.5,5) {Problem};
\node (human) at (2.5,4) {Human};
\node[shape=rectangle,fill=pantone127-4] (model) at (2.5,3) {Model};
\node (compile) at (2.5,2) {Compile/Reformulate};
\node (solver1) at (1,1) {CP};
\node (solver2) at (2,1) {MIP};
\node (solver3) at (3,1) {SAT};
\node (solver4) at (4,1) {Other};
\node[shape=rectangle,fill=pantone157-8] (solution1) at (1,0) {Solution};
\node[shape=rectangle,fill=pantone157-8] (solution2) at (2,0) {Solution};
\node[shape=rectangle,fill=pantone157-8] (solution3) at (3,0) {Solution};
\node[shape=rectangle,fill=pantone157-8] (solution4) at (4,0) {Solution};
\draw[-] (problem) -- (human);
\draw[->] (human) -- (model);
\draw[-] (model) -- (compile);
\draw[-] (compile) -- (solver1);
\draw[-] (compile) -- (solver2);
\draw[-] (compile) -- (solver3);
\draw[-] (compile) -- (solver4);
\draw[->] (solver1) -- (solution1);
\draw[->] (solver2) -- (solution2);
\draw[->] (solver3) -- (solution3);
\draw[->] (solver4) -- (solution4);
\end{tikzpicture}
\end{center}
\end{frame}

% \begin{frame}
  % \frametitle{Do It Now!}
  % \begin{itemize}
  % \item Download and install Minizinc
  % \item \url{https://www.minizinc.org/}
  % \end{itemize}
% \end{frame}


\begin{frame}
\frametitle{Why use Puzzles as Examples?}
\begin{itemize}
\item Easy to understand the problem
\item Solvable by hand without specialized knowledge
\item Possible to compare automated to manual solving process
\end{itemize}

\begin{quote}
The puzzle, though inanimate, is presented as a solvable problem without lasting negative consequences, a very low-risk low-reward situation. By being a puzzle, the object is attempting to convince the user that it must be completed.
\end{quote}
{\tiny Source: \href{https://www.everydayrhetoric.com/2020/03/the-rhetoric-of-puzzles/}{Every Day Rhetoric}}
\end{frame}


\mode<all>{
\part{Basic Constraint Propagation}
}
\begin{frame}
\partpage
\end{frame}
\mode<all>{
\input{../sendmore/body}
}

\mode<all>{
\part{Global Constraints}
}
\begin{frame}
\partpage
\end{frame}
\mode<all>{
\input{../sudoku/body}
}

\clearpage
\mode<all>{
\part{Customizing Search}
}
\begin{frame}
\partpage
\end{frame}
\mode<all>{
\input{../nqueen/body}
}


%% \clearpage
%% \mode<all>{
%%   \part{Problem Decomposition}
%% }
%% \begin{frame}
%%   \partpage
%% \end{frame}
%% \mode<all>{
%%   \input{../party/body}
%%   }

%% \clearpage
%% \mode<all>{
%%   \part{More Global Constraints}
%% }
%% \begin{frame}
%%   \partpage
%% \end{frame}
%% \mode<all>{
%%   \input{../car/body}
%%   }

%% \clearpage
%% \mode<all>{
%%   \part{Methodology}
%% }
%% \begin{frame}
%%   \partpage
%% \end{frame}
%% \mode<all>{
%%   \input{../methodology/body}
%%   }

\clearpage
\mode<all>{
\part{What is missing?}
}
\begin{frame}
\partpage
\end{frame}

\begin{frame}
\frametitle{Many Specialized Topics}
\begin{itemize}
\item How to design efficient core engine
\item Hybrids with LP/MIP tools
\item Hybrids with SAT
\item Symmetry breaking
\item Use of MDD/BDD to encode sets of solutions
\item High level modelling tools
\item Debugging/visualization
\end{itemize}
\end{frame}

\begin{frame}
\frametitle{Reformulation}
\begin{itemize}
\item Just because the user has modelled it this way, it doesn't mean we have to solve it that way
\begin{itemize}
\item Replace some constraint(s) by other, equivalent constraints
\item Because we don't have that constraint in our system
\item For performance
\end{itemize}
\end{itemize}
\end{frame}

\begin{frame}
\frametitle{Learning}
\begin{itemize}
\item While solving the problem we can learn how to strengthen the model/search 
\begin{itemize}
\item Understand which constraints/method contribute to propagation and change schedule
\item Learn no-good constraints by explaining failure
\item Adapt search strategy based on search experience
\end{itemize}
\end{itemize}
\end{frame}

%% \begin{frame}
%% \frametitle{Refined Process}
%% \begin{center}
%% \begin{tikzpicture}
%% \node[shape=rectangle,fill=pantone127-4] (problem) at (3,7) {Problem};
%% \node (human) at (3,6) {Human};
%% \node[shape=rectangle,fill=pantone127-4] (model) at (3,5) {User Model};
%% \node (reformulation) at (3,4) {Reformulation};
%% \node[shape=rectangle,fill=pantone127-4] (implementation) at (3,3) {Implementation Model};
%% \node (solver) at (3,2) {Constraint Solver/Search};
%% \node[shape=rectangle,fill=pantone157-8] (solution) at (3,1) {Solution};
%% \draw[-] (problem) -- (human);
%% \draw[->] (human) -- (model);
%% \draw[-] (model) -- (reformulation);
%% \draw[->] (reformulation) -- (implementation);
%% \draw[-] (implementation) -- (solver);
%% \draw[->] (solver) -- (0,2) -- (0,3) -- (implementation);
%% \draw[->] (solver) -- (solution);
%% \end{tikzpicture}
%% \end{center}
%% \end{frame}

%\begin{frame}
%\frametitle{What is CP actually used for?}
%\url{http://hsimonis.wordpress.com}
%\includegraphics[width=10cm]{../introduction/images/blog}
%\end{frame}

%% \begin{frame}
%% \frametitle{Exercises}
%% \begin{itemize}
%% \item Install Constraint Solver of your Choice
%% \item MiniZinc (MiningZinc) will be used in later sessions
%% \item Run Sendmoremoney, Sudoku, Queens examples
%% \item Can you choose which propagator to use for alldifferent?
%% \item Can you use different search strategies?
%% \end{itemize}
%% \end{frame}

\begin{frame}
\frametitle{More Learning Resources}
\begin{itemize}
\item Survey of Methods, Resources, and Formats for Teaching Constraint Programming
\begin{itemize}
\item by Tejas Santanam, Helmut Simonis
\item \url{https://doi.org/10.48550/arXiv.2403.12717}
\item Based on survey of community for WTCP 2023
\item \url{https://hsimonis.github.io/WTCP2023/}
\end{itemize}
\end{itemize}
\end{frame}


}

\mode<all>{
\part{Global Constraints}
}
\begin{frame}
\partpage
\end{frame}
\mode<all>{
\begin{frame}
\frametitle{Objectives}
\begin{itemize}
\item Overview of Core Constraint Programming
\item Three Main Concepts
\begin{itemize}
\item Constraint Propagation
\item Global Constraints
\item Customizing Search
\end{itemize}
%\item Get Some Experience with MiniZinc
\item Topics will be treated in more detail in later parts of the school
\item Based on Examples, not Formal Description
\end{itemize}
\end{frame}

\begin{frame}
\frametitle{Outline}
\begin{itemize}
\item Why Constraint Programming?
\item Constraint Propagation
\item Global Constraints
\item Customizing Search
%\item Choices in Modelling
%\item Decomposition
%\item More Global Constraints
%\item Methodology
%\item What is missing?
\end{itemize}
\end{frame}

% \begin{frame}
  % \frametitle{Using MiniZinc IDE}
  % \begin{itemize}
  % \item Developed in the Australian NICTA project
  % \item Maintained by Monash University
  % \item Modelling tool with multiple back-end solvers
  % \item Available from \url{https://www.minizinc.org/}
  % \end{itemize}
% \end{frame}


% \begin{frame}
% \frametitle{Examples in ECLiPSe}
% \begin{itemize}
% \item Open sourced constraint programming language
% \item Development goes back to 1985
% \item ECRC, ICL, IC-Parc, PTL, Cisco
% \item \url{https://eclipseclp.org/}
% \item Specialities
% \begin{itemize}
% \item Develop new solvers for specific domains
% \item Integration with MIP
% \end{itemize}
% \item Not included in bundled MiniZinc IDE
% \item Specialized visualization tools used here
  % \begin{itemize}
  % \item CP-Viz, Simonis et al. 2010
    % \end{itemize}
% \end{itemize}
% \end{frame}

\begin{frame}
\frametitle{Tutorial Based on ECLiPSe ELearning Course}
\begin{itemize}
\item Self-study course in constraint programming
\item Supported by Cisco Systems and Silicon Valley Community Foundation
\item Multi-media format, video lectures, slides, handout etc
\item {\small \url{https://eclipseclp.org/ELearning/index.html}}
\end{itemize}
\end{frame}

\begin{frame}
\frametitle{Also Part of CRT-AI Constraint Week}
\begin{itemize}
\item Annual one week course on CP and Optimization in Ireland
\item Part of national training program for PhD students in AI
\item \url{https://www.crt-ai.ie/}
\end{itemize}
\end{frame}



\begin{frame}
\frametitle{Constraint Programming - in a nutshell}
\begin{itemize}
\item Declarative description of problems with
\begin{itemize}
\item {\em Variables} which range over (finite) sets of values
\item {\em Constraints} over subsets of variables which restrict possible value combinations
\item A {\em solution} is a value assignment which satisfies all constraints
\end{itemize}

\item Constraint propagation/reasoning
\begin{itemize}
\item Removing inconsistent values for variables
\item Detect failure if constraint can not be satisfied
\item Interaction of constraints via shared variables
\item Incomplete
\end{itemize}

\item Search
\begin{itemize}
\item User controlled assignment of values to variables
\item Each step triggers constraint propagation 
\end{itemize}
\item Different domains require/allow different methods
\end{itemize}
\end{frame}

\begin{frame}
  \frametitle{Constraint Programming is Different}
  \begin{itemize}
  \item Declarative Programming
    \begin{itemize}
    \item Concentrate on what you want
      \item Not how to get there
      \item Program != Algorithm
      \item Program = Model
    \end{itemize}
    \item Applied to Combinatorial Problems
      \begin{itemize}
        \item No complete polynomial algorithms known (exist?)
        \item CP less ad-hoc than heuristics
        \item Models can evolve
  \end{itemize}
  \end{itemize}
  \end{frame}
    
\begin{frame}
  \frametitle{A Subtractive Process}
  \begin{textblock}{4}(8,-3)
    \includegraphics[width=4cm]{../introduction/images/stages}
  \end{textblock}
  \vfill
  \begin{quote}
    ``Oh, bosh, as Mr. Ruskin says. Sculpture, per se, is the simplest thing in the world. All you have to do is to take a big chunk of marble and a hammer and chisel, make up your mind what you are about to create and chip off all the marble you don't want.''-Paris Gaulois.
  \end{quote}
  
  {\tiny Source: \url{https://quoteinvestigator.com/2014/06/22/chip-away/}}
\end{frame}

\begin{frame}
\frametitle{Basic Process}
\begin{center}
\begin{tikzpicture}
\node[shape=rectangle,fill=pantone127-4] (problem) at (2,5) {Problem};
\node (human) at (2,4) {Human};
\node[shape=rectangle,fill=pantone127-4] (model) at (2,3) {Model};
\node (solver) at (2,2) {Constraint Solver/Search};
\node[shape=rectangle,fill=pantone157-8] (solution) at (2,1) {Solution};
\draw[-] (problem) -- (human);
\draw[->] (human) -- (model);
\draw[-] (model) -- (solver);
\draw[->] (solver) -- (solution);
\end{tikzpicture}
\end{center}
\end{frame}

\begin{frame}
\frametitle{More Realistic}
\begin{center}
\begin{tikzpicture}[xscale=4.0]
\node[shape=rectangle,fill=pantone127-4] (problem) at (2,5) {Problem};
\node (human) at (2,4) {Human};
\node[shape=rectangle,fill=pantone127-4] (model) at (2,3) {Model};
\node (solver) at (2,2) {Constraint Solver/Search};
\node[shape=rectangle,fill=pantone24-8] (hangs) at (1,1) {Hangs};
\node[shape=rectangle,fill=pantone157-8] (solution) at (2,1) {Solution};
\node[shape=rectangle,fill=pantone24-8] (wrong) at (3,1) {Wrong Solution};
\draw[-] (problem) -- (human);
\draw[->] (human) -- (model);
\draw[-] (model) -- (solver);
\draw[->] (solver) -- (hangs);
\draw[->] (solver) -- (solution);
\draw[->] (solver) -- (wrong);
\draw[->] (hangs) -- (1,4) -- (human);
\draw[->] (wrong) -- (3,4) -- (human);
\end{tikzpicture}
\end{center}
\end{frame}

\begin{frame}
\frametitle{Dual Role of Model}
\begin{itemize}
\item Allows Human to Express Problem
\begin{itemize}
\item Close to Problem Domain
\item Constraints as Abstractions
\end{itemize}
\item Allows Solver to Execute
\begin{itemize}
\item Variables as Communication Mechanism
\item Constraints as Algorithms
\end{itemize}
\end{itemize}
\end{frame}

\begin{frame}
\frametitle{Modelling Frameworks}
\begin{itemize}
\item \href{https://www.minizinc.org/}{MiniZinc} (NICTA, Monash University, Australia)
\item \href{https://github.com/eomahony/Numberjack}{NumberJack} (Insight, Ireland) 
\item \href{https://www-users.york.ac.uk/peter.nightingale/savilerow//}{EssencePrime/SavilleRow} (UK) 
\item \href{https://cpmpy.readthedocs.io/en/latest/}{CPMpy} (KU Leuven)
\item Allow use of multiple back-end solvers
\item Compile model into variants for each solver
\item A priori solver independent model(CP, MIP, SAT)
\end{itemize}
\end{frame} 

\begin{frame}
\frametitle{Framework Process}
\begin{center}
\begin{tikzpicture}[xscale=2]
\node[shape=rectangle,fill=pantone127-4] (problem) at (2.5,5) {Problem};
\node (human) at (2.5,4) {Human};
\node[shape=rectangle,fill=pantone127-4] (model) at (2.5,3) {Model};
\node (compile) at (2.5,2) {Compile/Reformulate};
\node (solver1) at (1,1) {CP};
\node (solver2) at (2,1) {MIP};
\node (solver3) at (3,1) {SAT};
\node (solver4) at (4,1) {Other};
\node[shape=rectangle,fill=pantone157-8] (solution1) at (1,0) {Solution};
\node[shape=rectangle,fill=pantone157-8] (solution2) at (2,0) {Solution};
\node[shape=rectangle,fill=pantone157-8] (solution3) at (3,0) {Solution};
\node[shape=rectangle,fill=pantone157-8] (solution4) at (4,0) {Solution};
\draw[-] (problem) -- (human);
\draw[->] (human) -- (model);
\draw[-] (model) -- (compile);
\draw[-] (compile) -- (solver1);
\draw[-] (compile) -- (solver2);
\draw[-] (compile) -- (solver3);
\draw[-] (compile) -- (solver4);
\draw[->] (solver1) -- (solution1);
\draw[->] (solver2) -- (solution2);
\draw[->] (solver3) -- (solution3);
\draw[->] (solver4) -- (solution4);
\end{tikzpicture}
\end{center}
\end{frame}

% \begin{frame}
  % \frametitle{Do It Now!}
  % \begin{itemize}
  % \item Download and install Minizinc
  % \item \url{https://www.minizinc.org/}
  % \end{itemize}
% \end{frame}


\begin{frame}
\frametitle{Why use Puzzles as Examples?}
\begin{itemize}
\item Easy to understand the problem
\item Solvable by hand without specialized knowledge
\item Possible to compare automated to manual solving process
\end{itemize}

\begin{quote}
The puzzle, though inanimate, is presented as a solvable problem without lasting negative consequences, a very low-risk low-reward situation. By being a puzzle, the object is attempting to convince the user that it must be completed.
\end{quote}
{\tiny Source: \href{https://www.everydayrhetoric.com/2020/03/the-rhetoric-of-puzzles/}{Every Day Rhetoric}}
\end{frame}


\mode<all>{
\part{Basic Constraint Propagation}
}
\begin{frame}
\partpage
\end{frame}
\mode<all>{
\input{../sendmore/body}
}

\mode<all>{
\part{Global Constraints}
}
\begin{frame}
\partpage
\end{frame}
\mode<all>{
\input{../sudoku/body}
}

\clearpage
\mode<all>{
\part{Customizing Search}
}
\begin{frame}
\partpage
\end{frame}
\mode<all>{
\input{../nqueen/body}
}


%% \clearpage
%% \mode<all>{
%%   \part{Problem Decomposition}
%% }
%% \begin{frame}
%%   \partpage
%% \end{frame}
%% \mode<all>{
%%   \input{../party/body}
%%   }

%% \clearpage
%% \mode<all>{
%%   \part{More Global Constraints}
%% }
%% \begin{frame}
%%   \partpage
%% \end{frame}
%% \mode<all>{
%%   \input{../car/body}
%%   }

%% \clearpage
%% \mode<all>{
%%   \part{Methodology}
%% }
%% \begin{frame}
%%   \partpage
%% \end{frame}
%% \mode<all>{
%%   \input{../methodology/body}
%%   }

\clearpage
\mode<all>{
\part{What is missing?}
}
\begin{frame}
\partpage
\end{frame}

\begin{frame}
\frametitle{Many Specialized Topics}
\begin{itemize}
\item How to design efficient core engine
\item Hybrids with LP/MIP tools
\item Hybrids with SAT
\item Symmetry breaking
\item Use of MDD/BDD to encode sets of solutions
\item High level modelling tools
\item Debugging/visualization
\end{itemize}
\end{frame}

\begin{frame}
\frametitle{Reformulation}
\begin{itemize}
\item Just because the user has modelled it this way, it doesn't mean we have to solve it that way
\begin{itemize}
\item Replace some constraint(s) by other, equivalent constraints
\item Because we don't have that constraint in our system
\item For performance
\end{itemize}
\end{itemize}
\end{frame}

\begin{frame}
\frametitle{Learning}
\begin{itemize}
\item While solving the problem we can learn how to strengthen the model/search 
\begin{itemize}
\item Understand which constraints/method contribute to propagation and change schedule
\item Learn no-good constraints by explaining failure
\item Adapt search strategy based on search experience
\end{itemize}
\end{itemize}
\end{frame}

%% \begin{frame}
%% \frametitle{Refined Process}
%% \begin{center}
%% \begin{tikzpicture}
%% \node[shape=rectangle,fill=pantone127-4] (problem) at (3,7) {Problem};
%% \node (human) at (3,6) {Human};
%% \node[shape=rectangle,fill=pantone127-4] (model) at (3,5) {User Model};
%% \node (reformulation) at (3,4) {Reformulation};
%% \node[shape=rectangle,fill=pantone127-4] (implementation) at (3,3) {Implementation Model};
%% \node (solver) at (3,2) {Constraint Solver/Search};
%% \node[shape=rectangle,fill=pantone157-8] (solution) at (3,1) {Solution};
%% \draw[-] (problem) -- (human);
%% \draw[->] (human) -- (model);
%% \draw[-] (model) -- (reformulation);
%% \draw[->] (reformulation) -- (implementation);
%% \draw[-] (implementation) -- (solver);
%% \draw[->] (solver) -- (0,2) -- (0,3) -- (implementation);
%% \draw[->] (solver) -- (solution);
%% \end{tikzpicture}
%% \end{center}
%% \end{frame}

%\begin{frame}
%\frametitle{What is CP actually used for?}
%\url{http://hsimonis.wordpress.com}
%\includegraphics[width=10cm]{../introduction/images/blog}
%\end{frame}

%% \begin{frame}
%% \frametitle{Exercises}
%% \begin{itemize}
%% \item Install Constraint Solver of your Choice
%% \item MiniZinc (MiningZinc) will be used in later sessions
%% \item Run Sendmoremoney, Sudoku, Queens examples
%% \item Can you choose which propagator to use for alldifferent?
%% \item Can you use different search strategies?
%% \end{itemize}
%% \end{frame}

\begin{frame}
\frametitle{More Learning Resources}
\begin{itemize}
\item Survey of Methods, Resources, and Formats for Teaching Constraint Programming
\begin{itemize}
\item by Tejas Santanam, Helmut Simonis
\item \url{https://doi.org/10.48550/arXiv.2403.12717}
\item Based on survey of community for WTCP 2023
\item \url{https://hsimonis.github.io/WTCP2023/}
\end{itemize}
\end{itemize}
\end{frame}


}

\clearpage
\mode<all>{
\part{Customizing Search}
}
\begin{frame}
\partpage
\end{frame}
\mode<all>{
\begin{frame}
\frametitle{Objectives}
\begin{itemize}
\item Overview of Core Constraint Programming
\item Three Main Concepts
\begin{itemize}
\item Constraint Propagation
\item Global Constraints
\item Customizing Search
\end{itemize}
%\item Get Some Experience with MiniZinc
\item Topics will be treated in more detail in later parts of the school
\item Based on Examples, not Formal Description
\end{itemize}
\end{frame}

\begin{frame}
\frametitle{Outline}
\begin{itemize}
\item Why Constraint Programming?
\item Constraint Propagation
\item Global Constraints
\item Customizing Search
%\item Choices in Modelling
%\item Decomposition
%\item More Global Constraints
%\item Methodology
%\item What is missing?
\end{itemize}
\end{frame}

% \begin{frame}
  % \frametitle{Using MiniZinc IDE}
  % \begin{itemize}
  % \item Developed in the Australian NICTA project
  % \item Maintained by Monash University
  % \item Modelling tool with multiple back-end solvers
  % \item Available from \url{https://www.minizinc.org/}
  % \end{itemize}
% \end{frame}


% \begin{frame}
% \frametitle{Examples in ECLiPSe}
% \begin{itemize}
% \item Open sourced constraint programming language
% \item Development goes back to 1985
% \item ECRC, ICL, IC-Parc, PTL, Cisco
% \item \url{https://eclipseclp.org/}
% \item Specialities
% \begin{itemize}
% \item Develop new solvers for specific domains
% \item Integration with MIP
% \end{itemize}
% \item Not included in bundled MiniZinc IDE
% \item Specialized visualization tools used here
  % \begin{itemize}
  % \item CP-Viz, Simonis et al. 2010
    % \end{itemize}
% \end{itemize}
% \end{frame}

\begin{frame}
\frametitle{Tutorial Based on ECLiPSe ELearning Course}
\begin{itemize}
\item Self-study course in constraint programming
\item Supported by Cisco Systems and Silicon Valley Community Foundation
\item Multi-media format, video lectures, slides, handout etc
\item {\small \url{https://eclipseclp.org/ELearning/index.html}}
\end{itemize}
\end{frame}

\begin{frame}
\frametitle{Also Part of CRT-AI Constraint Week}
\begin{itemize}
\item Annual one week course on CP and Optimization in Ireland
\item Part of national training program for PhD students in AI
\item \url{https://www.crt-ai.ie/}
\end{itemize}
\end{frame}



\begin{frame}
\frametitle{Constraint Programming - in a nutshell}
\begin{itemize}
\item Declarative description of problems with
\begin{itemize}
\item {\em Variables} which range over (finite) sets of values
\item {\em Constraints} over subsets of variables which restrict possible value combinations
\item A {\em solution} is a value assignment which satisfies all constraints
\end{itemize}

\item Constraint propagation/reasoning
\begin{itemize}
\item Removing inconsistent values for variables
\item Detect failure if constraint can not be satisfied
\item Interaction of constraints via shared variables
\item Incomplete
\end{itemize}

\item Search
\begin{itemize}
\item User controlled assignment of values to variables
\item Each step triggers constraint propagation 
\end{itemize}
\item Different domains require/allow different methods
\end{itemize}
\end{frame}

\begin{frame}
  \frametitle{Constraint Programming is Different}
  \begin{itemize}
  \item Declarative Programming
    \begin{itemize}
    \item Concentrate on what you want
      \item Not how to get there
      \item Program != Algorithm
      \item Program = Model
    \end{itemize}
    \item Applied to Combinatorial Problems
      \begin{itemize}
        \item No complete polynomial algorithms known (exist?)
        \item CP less ad-hoc than heuristics
        \item Models can evolve
  \end{itemize}
  \end{itemize}
  \end{frame}
    
\begin{frame}
  \frametitle{A Subtractive Process}
  \begin{textblock}{4}(8,-3)
    \includegraphics[width=4cm]{../introduction/images/stages}
  \end{textblock}
  \vfill
  \begin{quote}
    ``Oh, bosh, as Mr. Ruskin says. Sculpture, per se, is the simplest thing in the world. All you have to do is to take a big chunk of marble and a hammer and chisel, make up your mind what you are about to create and chip off all the marble you don't want.''-Paris Gaulois.
  \end{quote}
  
  {\tiny Source: \url{https://quoteinvestigator.com/2014/06/22/chip-away/}}
\end{frame}

\begin{frame}
\frametitle{Basic Process}
\begin{center}
\begin{tikzpicture}
\node[shape=rectangle,fill=pantone127-4] (problem) at (2,5) {Problem};
\node (human) at (2,4) {Human};
\node[shape=rectangle,fill=pantone127-4] (model) at (2,3) {Model};
\node (solver) at (2,2) {Constraint Solver/Search};
\node[shape=rectangle,fill=pantone157-8] (solution) at (2,1) {Solution};
\draw[-] (problem) -- (human);
\draw[->] (human) -- (model);
\draw[-] (model) -- (solver);
\draw[->] (solver) -- (solution);
\end{tikzpicture}
\end{center}
\end{frame}

\begin{frame}
\frametitle{More Realistic}
\begin{center}
\begin{tikzpicture}[xscale=4.0]
\node[shape=rectangle,fill=pantone127-4] (problem) at (2,5) {Problem};
\node (human) at (2,4) {Human};
\node[shape=rectangle,fill=pantone127-4] (model) at (2,3) {Model};
\node (solver) at (2,2) {Constraint Solver/Search};
\node[shape=rectangle,fill=pantone24-8] (hangs) at (1,1) {Hangs};
\node[shape=rectangle,fill=pantone157-8] (solution) at (2,1) {Solution};
\node[shape=rectangle,fill=pantone24-8] (wrong) at (3,1) {Wrong Solution};
\draw[-] (problem) -- (human);
\draw[->] (human) -- (model);
\draw[-] (model) -- (solver);
\draw[->] (solver) -- (hangs);
\draw[->] (solver) -- (solution);
\draw[->] (solver) -- (wrong);
\draw[->] (hangs) -- (1,4) -- (human);
\draw[->] (wrong) -- (3,4) -- (human);
\end{tikzpicture}
\end{center}
\end{frame}

\begin{frame}
\frametitle{Dual Role of Model}
\begin{itemize}
\item Allows Human to Express Problem
\begin{itemize}
\item Close to Problem Domain
\item Constraints as Abstractions
\end{itemize}
\item Allows Solver to Execute
\begin{itemize}
\item Variables as Communication Mechanism
\item Constraints as Algorithms
\end{itemize}
\end{itemize}
\end{frame}

\begin{frame}
\frametitle{Modelling Frameworks}
\begin{itemize}
\item \href{https://www.minizinc.org/}{MiniZinc} (NICTA, Monash University, Australia)
\item \href{https://github.com/eomahony/Numberjack}{NumberJack} (Insight, Ireland) 
\item \href{https://www-users.york.ac.uk/peter.nightingale/savilerow//}{EssencePrime/SavilleRow} (UK) 
\item \href{https://cpmpy.readthedocs.io/en/latest/}{CPMpy} (KU Leuven)
\item Allow use of multiple back-end solvers
\item Compile model into variants for each solver
\item A priori solver independent model(CP, MIP, SAT)
\end{itemize}
\end{frame} 

\begin{frame}
\frametitle{Framework Process}
\begin{center}
\begin{tikzpicture}[xscale=2]
\node[shape=rectangle,fill=pantone127-4] (problem) at (2.5,5) {Problem};
\node (human) at (2.5,4) {Human};
\node[shape=rectangle,fill=pantone127-4] (model) at (2.5,3) {Model};
\node (compile) at (2.5,2) {Compile/Reformulate};
\node (solver1) at (1,1) {CP};
\node (solver2) at (2,1) {MIP};
\node (solver3) at (3,1) {SAT};
\node (solver4) at (4,1) {Other};
\node[shape=rectangle,fill=pantone157-8] (solution1) at (1,0) {Solution};
\node[shape=rectangle,fill=pantone157-8] (solution2) at (2,0) {Solution};
\node[shape=rectangle,fill=pantone157-8] (solution3) at (3,0) {Solution};
\node[shape=rectangle,fill=pantone157-8] (solution4) at (4,0) {Solution};
\draw[-] (problem) -- (human);
\draw[->] (human) -- (model);
\draw[-] (model) -- (compile);
\draw[-] (compile) -- (solver1);
\draw[-] (compile) -- (solver2);
\draw[-] (compile) -- (solver3);
\draw[-] (compile) -- (solver4);
\draw[->] (solver1) -- (solution1);
\draw[->] (solver2) -- (solution2);
\draw[->] (solver3) -- (solution3);
\draw[->] (solver4) -- (solution4);
\end{tikzpicture}
\end{center}
\end{frame}

% \begin{frame}
  % \frametitle{Do It Now!}
  % \begin{itemize}
  % \item Download and install Minizinc
  % \item \url{https://www.minizinc.org/}
  % \end{itemize}
% \end{frame}


\begin{frame}
\frametitle{Why use Puzzles as Examples?}
\begin{itemize}
\item Easy to understand the problem
\item Solvable by hand without specialized knowledge
\item Possible to compare automated to manual solving process
\end{itemize}

\begin{quote}
The puzzle, though inanimate, is presented as a solvable problem without lasting negative consequences, a very low-risk low-reward situation. By being a puzzle, the object is attempting to convince the user that it must be completed.
\end{quote}
{\tiny Source: \href{https://www.everydayrhetoric.com/2020/03/the-rhetoric-of-puzzles/}{Every Day Rhetoric}}
\end{frame}


\mode<all>{
\part{Basic Constraint Propagation}
}
\begin{frame}
\partpage
\end{frame}
\mode<all>{
\input{../sendmore/body}
}

\mode<all>{
\part{Global Constraints}
}
\begin{frame}
\partpage
\end{frame}
\mode<all>{
\input{../sudoku/body}
}

\clearpage
\mode<all>{
\part{Customizing Search}
}
\begin{frame}
\partpage
\end{frame}
\mode<all>{
\input{../nqueen/body}
}


%% \clearpage
%% \mode<all>{
%%   \part{Problem Decomposition}
%% }
%% \begin{frame}
%%   \partpage
%% \end{frame}
%% \mode<all>{
%%   \input{../party/body}
%%   }

%% \clearpage
%% \mode<all>{
%%   \part{More Global Constraints}
%% }
%% \begin{frame}
%%   \partpage
%% \end{frame}
%% \mode<all>{
%%   \input{../car/body}
%%   }

%% \clearpage
%% \mode<all>{
%%   \part{Methodology}
%% }
%% \begin{frame}
%%   \partpage
%% \end{frame}
%% \mode<all>{
%%   \input{../methodology/body}
%%   }

\clearpage
\mode<all>{
\part{What is missing?}
}
\begin{frame}
\partpage
\end{frame}

\begin{frame}
\frametitle{Many Specialized Topics}
\begin{itemize}
\item How to design efficient core engine
\item Hybrids with LP/MIP tools
\item Hybrids with SAT
\item Symmetry breaking
\item Use of MDD/BDD to encode sets of solutions
\item High level modelling tools
\item Debugging/visualization
\end{itemize}
\end{frame}

\begin{frame}
\frametitle{Reformulation}
\begin{itemize}
\item Just because the user has modelled it this way, it doesn't mean we have to solve it that way
\begin{itemize}
\item Replace some constraint(s) by other, equivalent constraints
\item Because we don't have that constraint in our system
\item For performance
\end{itemize}
\end{itemize}
\end{frame}

\begin{frame}
\frametitle{Learning}
\begin{itemize}
\item While solving the problem we can learn how to strengthen the model/search 
\begin{itemize}
\item Understand which constraints/method contribute to propagation and change schedule
\item Learn no-good constraints by explaining failure
\item Adapt search strategy based on search experience
\end{itemize}
\end{itemize}
\end{frame}

%% \begin{frame}
%% \frametitle{Refined Process}
%% \begin{center}
%% \begin{tikzpicture}
%% \node[shape=rectangle,fill=pantone127-4] (problem) at (3,7) {Problem};
%% \node (human) at (3,6) {Human};
%% \node[shape=rectangle,fill=pantone127-4] (model) at (3,5) {User Model};
%% \node (reformulation) at (3,4) {Reformulation};
%% \node[shape=rectangle,fill=pantone127-4] (implementation) at (3,3) {Implementation Model};
%% \node (solver) at (3,2) {Constraint Solver/Search};
%% \node[shape=rectangle,fill=pantone157-8] (solution) at (3,1) {Solution};
%% \draw[-] (problem) -- (human);
%% \draw[->] (human) -- (model);
%% \draw[-] (model) -- (reformulation);
%% \draw[->] (reformulation) -- (implementation);
%% \draw[-] (implementation) -- (solver);
%% \draw[->] (solver) -- (0,2) -- (0,3) -- (implementation);
%% \draw[->] (solver) -- (solution);
%% \end{tikzpicture}
%% \end{center}
%% \end{frame}

%\begin{frame}
%\frametitle{What is CP actually used for?}
%\url{http://hsimonis.wordpress.com}
%\includegraphics[width=10cm]{../introduction/images/blog}
%\end{frame}

%% \begin{frame}
%% \frametitle{Exercises}
%% \begin{itemize}
%% \item Install Constraint Solver of your Choice
%% \item MiniZinc (MiningZinc) will be used in later sessions
%% \item Run Sendmoremoney, Sudoku, Queens examples
%% \item Can you choose which propagator to use for alldifferent?
%% \item Can you use different search strategies?
%% \end{itemize}
%% \end{frame}

\begin{frame}
\frametitle{More Learning Resources}
\begin{itemize}
\item Survey of Methods, Resources, and Formats for Teaching Constraint Programming
\begin{itemize}
\item by Tejas Santanam, Helmut Simonis
\item \url{https://doi.org/10.48550/arXiv.2403.12717}
\item Based on survey of community for WTCP 2023
\item \url{https://hsimonis.github.io/WTCP2023/}
\end{itemize}
\end{itemize}
\end{frame}


}


%% \clearpage
%% \mode<all>{
%%   \part{Problem Decomposition}
%% }
%% \begin{frame}
%%   \partpage
%% \end{frame}
%% \mode<all>{
%%   \begin{frame}
\frametitle{Objectives}
\begin{itemize}
\item Overview of Core Constraint Programming
\item Three Main Concepts
\begin{itemize}
\item Constraint Propagation
\item Global Constraints
\item Customizing Search
\end{itemize}
%\item Get Some Experience with MiniZinc
\item Topics will be treated in more detail in later parts of the school
\item Based on Examples, not Formal Description
\end{itemize}
\end{frame}

\begin{frame}
\frametitle{Outline}
\begin{itemize}
\item Why Constraint Programming?
\item Constraint Propagation
\item Global Constraints
\item Customizing Search
%\item Choices in Modelling
%\item Decomposition
%\item More Global Constraints
%\item Methodology
%\item What is missing?
\end{itemize}
\end{frame}

% \begin{frame}
  % \frametitle{Using MiniZinc IDE}
  % \begin{itemize}
  % \item Developed in the Australian NICTA project
  % \item Maintained by Monash University
  % \item Modelling tool with multiple back-end solvers
  % \item Available from \url{https://www.minizinc.org/}
  % \end{itemize}
% \end{frame}


% \begin{frame}
% \frametitle{Examples in ECLiPSe}
% \begin{itemize}
% \item Open sourced constraint programming language
% \item Development goes back to 1985
% \item ECRC, ICL, IC-Parc, PTL, Cisco
% \item \url{https://eclipseclp.org/}
% \item Specialities
% \begin{itemize}
% \item Develop new solvers for specific domains
% \item Integration with MIP
% \end{itemize}
% \item Not included in bundled MiniZinc IDE
% \item Specialized visualization tools used here
  % \begin{itemize}
  % \item CP-Viz, Simonis et al. 2010
    % \end{itemize}
% \end{itemize}
% \end{frame}

\begin{frame}
\frametitle{Tutorial Based on ECLiPSe ELearning Course}
\begin{itemize}
\item Self-study course in constraint programming
\item Supported by Cisco Systems and Silicon Valley Community Foundation
\item Multi-media format, video lectures, slides, handout etc
\item {\small \url{https://eclipseclp.org/ELearning/index.html}}
\end{itemize}
\end{frame}

\begin{frame}
\frametitle{Also Part of CRT-AI Constraint Week}
\begin{itemize}
\item Annual one week course on CP and Optimization in Ireland
\item Part of national training program for PhD students in AI
\item \url{https://www.crt-ai.ie/}
\end{itemize}
\end{frame}



\begin{frame}
\frametitle{Constraint Programming - in a nutshell}
\begin{itemize}
\item Declarative description of problems with
\begin{itemize}
\item {\em Variables} which range over (finite) sets of values
\item {\em Constraints} over subsets of variables which restrict possible value combinations
\item A {\em solution} is a value assignment which satisfies all constraints
\end{itemize}

\item Constraint propagation/reasoning
\begin{itemize}
\item Removing inconsistent values for variables
\item Detect failure if constraint can not be satisfied
\item Interaction of constraints via shared variables
\item Incomplete
\end{itemize}

\item Search
\begin{itemize}
\item User controlled assignment of values to variables
\item Each step triggers constraint propagation 
\end{itemize}
\item Different domains require/allow different methods
\end{itemize}
\end{frame}

\begin{frame}
  \frametitle{Constraint Programming is Different}
  \begin{itemize}
  \item Declarative Programming
    \begin{itemize}
    \item Concentrate on what you want
      \item Not how to get there
      \item Program != Algorithm
      \item Program = Model
    \end{itemize}
    \item Applied to Combinatorial Problems
      \begin{itemize}
        \item No complete polynomial algorithms known (exist?)
        \item CP less ad-hoc than heuristics
        \item Models can evolve
  \end{itemize}
  \end{itemize}
  \end{frame}
    
\begin{frame}
  \frametitle{A Subtractive Process}
  \begin{textblock}{4}(8,-3)
    \includegraphics[width=4cm]{../introduction/images/stages}
  \end{textblock}
  \vfill
  \begin{quote}
    ``Oh, bosh, as Mr. Ruskin says. Sculpture, per se, is the simplest thing in the world. All you have to do is to take a big chunk of marble and a hammer and chisel, make up your mind what you are about to create and chip off all the marble you don't want.''-Paris Gaulois.
  \end{quote}
  
  {\tiny Source: \url{https://quoteinvestigator.com/2014/06/22/chip-away/}}
\end{frame}

\begin{frame}
\frametitle{Basic Process}
\begin{center}
\begin{tikzpicture}
\node[shape=rectangle,fill=pantone127-4] (problem) at (2,5) {Problem};
\node (human) at (2,4) {Human};
\node[shape=rectangle,fill=pantone127-4] (model) at (2,3) {Model};
\node (solver) at (2,2) {Constraint Solver/Search};
\node[shape=rectangle,fill=pantone157-8] (solution) at (2,1) {Solution};
\draw[-] (problem) -- (human);
\draw[->] (human) -- (model);
\draw[-] (model) -- (solver);
\draw[->] (solver) -- (solution);
\end{tikzpicture}
\end{center}
\end{frame}

\begin{frame}
\frametitle{More Realistic}
\begin{center}
\begin{tikzpicture}[xscale=4.0]
\node[shape=rectangle,fill=pantone127-4] (problem) at (2,5) {Problem};
\node (human) at (2,4) {Human};
\node[shape=rectangle,fill=pantone127-4] (model) at (2,3) {Model};
\node (solver) at (2,2) {Constraint Solver/Search};
\node[shape=rectangle,fill=pantone24-8] (hangs) at (1,1) {Hangs};
\node[shape=rectangle,fill=pantone157-8] (solution) at (2,1) {Solution};
\node[shape=rectangle,fill=pantone24-8] (wrong) at (3,1) {Wrong Solution};
\draw[-] (problem) -- (human);
\draw[->] (human) -- (model);
\draw[-] (model) -- (solver);
\draw[->] (solver) -- (hangs);
\draw[->] (solver) -- (solution);
\draw[->] (solver) -- (wrong);
\draw[->] (hangs) -- (1,4) -- (human);
\draw[->] (wrong) -- (3,4) -- (human);
\end{tikzpicture}
\end{center}
\end{frame}

\begin{frame}
\frametitle{Dual Role of Model}
\begin{itemize}
\item Allows Human to Express Problem
\begin{itemize}
\item Close to Problem Domain
\item Constraints as Abstractions
\end{itemize}
\item Allows Solver to Execute
\begin{itemize}
\item Variables as Communication Mechanism
\item Constraints as Algorithms
\end{itemize}
\end{itemize}
\end{frame}

\begin{frame}
\frametitle{Modelling Frameworks}
\begin{itemize}
\item \href{https://www.minizinc.org/}{MiniZinc} (NICTA, Monash University, Australia)
\item \href{https://github.com/eomahony/Numberjack}{NumberJack} (Insight, Ireland) 
\item \href{https://www-users.york.ac.uk/peter.nightingale/savilerow//}{EssencePrime/SavilleRow} (UK) 
\item \href{https://cpmpy.readthedocs.io/en/latest/}{CPMpy} (KU Leuven)
\item Allow use of multiple back-end solvers
\item Compile model into variants for each solver
\item A priori solver independent model(CP, MIP, SAT)
\end{itemize}
\end{frame} 

\begin{frame}
\frametitle{Framework Process}
\begin{center}
\begin{tikzpicture}[xscale=2]
\node[shape=rectangle,fill=pantone127-4] (problem) at (2.5,5) {Problem};
\node (human) at (2.5,4) {Human};
\node[shape=rectangle,fill=pantone127-4] (model) at (2.5,3) {Model};
\node (compile) at (2.5,2) {Compile/Reformulate};
\node (solver1) at (1,1) {CP};
\node (solver2) at (2,1) {MIP};
\node (solver3) at (3,1) {SAT};
\node (solver4) at (4,1) {Other};
\node[shape=rectangle,fill=pantone157-8] (solution1) at (1,0) {Solution};
\node[shape=rectangle,fill=pantone157-8] (solution2) at (2,0) {Solution};
\node[shape=rectangle,fill=pantone157-8] (solution3) at (3,0) {Solution};
\node[shape=rectangle,fill=pantone157-8] (solution4) at (4,0) {Solution};
\draw[-] (problem) -- (human);
\draw[->] (human) -- (model);
\draw[-] (model) -- (compile);
\draw[-] (compile) -- (solver1);
\draw[-] (compile) -- (solver2);
\draw[-] (compile) -- (solver3);
\draw[-] (compile) -- (solver4);
\draw[->] (solver1) -- (solution1);
\draw[->] (solver2) -- (solution2);
\draw[->] (solver3) -- (solution3);
\draw[->] (solver4) -- (solution4);
\end{tikzpicture}
\end{center}
\end{frame}

% \begin{frame}
  % \frametitle{Do It Now!}
  % \begin{itemize}
  % \item Download and install Minizinc
  % \item \url{https://www.minizinc.org/}
  % \end{itemize}
% \end{frame}


\begin{frame}
\frametitle{Why use Puzzles as Examples?}
\begin{itemize}
\item Easy to understand the problem
\item Solvable by hand without specialized knowledge
\item Possible to compare automated to manual solving process
\end{itemize}

\begin{quote}
The puzzle, though inanimate, is presented as a solvable problem without lasting negative consequences, a very low-risk low-reward situation. By being a puzzle, the object is attempting to convince the user that it must be completed.
\end{quote}
{\tiny Source: \href{https://www.everydayrhetoric.com/2020/03/the-rhetoric-of-puzzles/}{Every Day Rhetoric}}
\end{frame}


\mode<all>{
\part{Basic Constraint Propagation}
}
\begin{frame}
\partpage
\end{frame}
\mode<all>{
\input{../sendmore/body}
}

\mode<all>{
\part{Global Constraints}
}
\begin{frame}
\partpage
\end{frame}
\mode<all>{
\input{../sudoku/body}
}

\clearpage
\mode<all>{
\part{Customizing Search}
}
\begin{frame}
\partpage
\end{frame}
\mode<all>{
\input{../nqueen/body}
}


%% \clearpage
%% \mode<all>{
%%   \part{Problem Decomposition}
%% }
%% \begin{frame}
%%   \partpage
%% \end{frame}
%% \mode<all>{
%%   \input{../party/body}
%%   }

%% \clearpage
%% \mode<all>{
%%   \part{More Global Constraints}
%% }
%% \begin{frame}
%%   \partpage
%% \end{frame}
%% \mode<all>{
%%   \input{../car/body}
%%   }

%% \clearpage
%% \mode<all>{
%%   \part{Methodology}
%% }
%% \begin{frame}
%%   \partpage
%% \end{frame}
%% \mode<all>{
%%   \input{../methodology/body}
%%   }

\clearpage
\mode<all>{
\part{What is missing?}
}
\begin{frame}
\partpage
\end{frame}

\begin{frame}
\frametitle{Many Specialized Topics}
\begin{itemize}
\item How to design efficient core engine
\item Hybrids with LP/MIP tools
\item Hybrids with SAT
\item Symmetry breaking
\item Use of MDD/BDD to encode sets of solutions
\item High level modelling tools
\item Debugging/visualization
\end{itemize}
\end{frame}

\begin{frame}
\frametitle{Reformulation}
\begin{itemize}
\item Just because the user has modelled it this way, it doesn't mean we have to solve it that way
\begin{itemize}
\item Replace some constraint(s) by other, equivalent constraints
\item Because we don't have that constraint in our system
\item For performance
\end{itemize}
\end{itemize}
\end{frame}

\begin{frame}
\frametitle{Learning}
\begin{itemize}
\item While solving the problem we can learn how to strengthen the model/search 
\begin{itemize}
\item Understand which constraints/method contribute to propagation and change schedule
\item Learn no-good constraints by explaining failure
\item Adapt search strategy based on search experience
\end{itemize}
\end{itemize}
\end{frame}

%% \begin{frame}
%% \frametitle{Refined Process}
%% \begin{center}
%% \begin{tikzpicture}
%% \node[shape=rectangle,fill=pantone127-4] (problem) at (3,7) {Problem};
%% \node (human) at (3,6) {Human};
%% \node[shape=rectangle,fill=pantone127-4] (model) at (3,5) {User Model};
%% \node (reformulation) at (3,4) {Reformulation};
%% \node[shape=rectangle,fill=pantone127-4] (implementation) at (3,3) {Implementation Model};
%% \node (solver) at (3,2) {Constraint Solver/Search};
%% \node[shape=rectangle,fill=pantone157-8] (solution) at (3,1) {Solution};
%% \draw[-] (problem) -- (human);
%% \draw[->] (human) -- (model);
%% \draw[-] (model) -- (reformulation);
%% \draw[->] (reformulation) -- (implementation);
%% \draw[-] (implementation) -- (solver);
%% \draw[->] (solver) -- (0,2) -- (0,3) -- (implementation);
%% \draw[->] (solver) -- (solution);
%% \end{tikzpicture}
%% \end{center}
%% \end{frame}

%\begin{frame}
%\frametitle{What is CP actually used for?}
%\url{http://hsimonis.wordpress.com}
%\includegraphics[width=10cm]{../introduction/images/blog}
%\end{frame}

%% \begin{frame}
%% \frametitle{Exercises}
%% \begin{itemize}
%% \item Install Constraint Solver of your Choice
%% \item MiniZinc (MiningZinc) will be used in later sessions
%% \item Run Sendmoremoney, Sudoku, Queens examples
%% \item Can you choose which propagator to use for alldifferent?
%% \item Can you use different search strategies?
%% \end{itemize}
%% \end{frame}

\begin{frame}
\frametitle{More Learning Resources}
\begin{itemize}
\item Survey of Methods, Resources, and Formats for Teaching Constraint Programming
\begin{itemize}
\item by Tejas Santanam, Helmut Simonis
\item \url{https://doi.org/10.48550/arXiv.2403.12717}
\item Based on survey of community for WTCP 2023
\item \url{https://hsimonis.github.io/WTCP2023/}
\end{itemize}
\end{itemize}
\end{frame}


%%   }

%% \clearpage
%% \mode<all>{
%%   \part{More Global Constraints}
%% }
%% \begin{frame}
%%   \partpage
%% \end{frame}
%% \mode<all>{
%%   \begin{frame}
\frametitle{Objectives}
\begin{itemize}
\item Overview of Core Constraint Programming
\item Three Main Concepts
\begin{itemize}
\item Constraint Propagation
\item Global Constraints
\item Customizing Search
\end{itemize}
%\item Get Some Experience with MiniZinc
\item Topics will be treated in more detail in later parts of the school
\item Based on Examples, not Formal Description
\end{itemize}
\end{frame}

\begin{frame}
\frametitle{Outline}
\begin{itemize}
\item Why Constraint Programming?
\item Constraint Propagation
\item Global Constraints
\item Customizing Search
%\item Choices in Modelling
%\item Decomposition
%\item More Global Constraints
%\item Methodology
%\item What is missing?
\end{itemize}
\end{frame}

% \begin{frame}
  % \frametitle{Using MiniZinc IDE}
  % \begin{itemize}
  % \item Developed in the Australian NICTA project
  % \item Maintained by Monash University
  % \item Modelling tool with multiple back-end solvers
  % \item Available from \url{https://www.minizinc.org/}
  % \end{itemize}
% \end{frame}


% \begin{frame}
% \frametitle{Examples in ECLiPSe}
% \begin{itemize}
% \item Open sourced constraint programming language
% \item Development goes back to 1985
% \item ECRC, ICL, IC-Parc, PTL, Cisco
% \item \url{https://eclipseclp.org/}
% \item Specialities
% \begin{itemize}
% \item Develop new solvers for specific domains
% \item Integration with MIP
% \end{itemize}
% \item Not included in bundled MiniZinc IDE
% \item Specialized visualization tools used here
  % \begin{itemize}
  % \item CP-Viz, Simonis et al. 2010
    % \end{itemize}
% \end{itemize}
% \end{frame}

\begin{frame}
\frametitle{Tutorial Based on ECLiPSe ELearning Course}
\begin{itemize}
\item Self-study course in constraint programming
\item Supported by Cisco Systems and Silicon Valley Community Foundation
\item Multi-media format, video lectures, slides, handout etc
\item {\small \url{https://eclipseclp.org/ELearning/index.html}}
\end{itemize}
\end{frame}

\begin{frame}
\frametitle{Also Part of CRT-AI Constraint Week}
\begin{itemize}
\item Annual one week course on CP and Optimization in Ireland
\item Part of national training program for PhD students in AI
\item \url{https://www.crt-ai.ie/}
\end{itemize}
\end{frame}



\begin{frame}
\frametitle{Constraint Programming - in a nutshell}
\begin{itemize}
\item Declarative description of problems with
\begin{itemize}
\item {\em Variables} which range over (finite) sets of values
\item {\em Constraints} over subsets of variables which restrict possible value combinations
\item A {\em solution} is a value assignment which satisfies all constraints
\end{itemize}

\item Constraint propagation/reasoning
\begin{itemize}
\item Removing inconsistent values for variables
\item Detect failure if constraint can not be satisfied
\item Interaction of constraints via shared variables
\item Incomplete
\end{itemize}

\item Search
\begin{itemize}
\item User controlled assignment of values to variables
\item Each step triggers constraint propagation 
\end{itemize}
\item Different domains require/allow different methods
\end{itemize}
\end{frame}

\begin{frame}
  \frametitle{Constraint Programming is Different}
  \begin{itemize}
  \item Declarative Programming
    \begin{itemize}
    \item Concentrate on what you want
      \item Not how to get there
      \item Program != Algorithm
      \item Program = Model
    \end{itemize}
    \item Applied to Combinatorial Problems
      \begin{itemize}
        \item No complete polynomial algorithms known (exist?)
        \item CP less ad-hoc than heuristics
        \item Models can evolve
  \end{itemize}
  \end{itemize}
  \end{frame}
    
\begin{frame}
  \frametitle{A Subtractive Process}
  \begin{textblock}{4}(8,-3)
    \includegraphics[width=4cm]{../introduction/images/stages}
  \end{textblock}
  \vfill
  \begin{quote}
    ``Oh, bosh, as Mr. Ruskin says. Sculpture, per se, is the simplest thing in the world. All you have to do is to take a big chunk of marble and a hammer and chisel, make up your mind what you are about to create and chip off all the marble you don't want.''-Paris Gaulois.
  \end{quote}
  
  {\tiny Source: \url{https://quoteinvestigator.com/2014/06/22/chip-away/}}
\end{frame}

\begin{frame}
\frametitle{Basic Process}
\begin{center}
\begin{tikzpicture}
\node[shape=rectangle,fill=pantone127-4] (problem) at (2,5) {Problem};
\node (human) at (2,4) {Human};
\node[shape=rectangle,fill=pantone127-4] (model) at (2,3) {Model};
\node (solver) at (2,2) {Constraint Solver/Search};
\node[shape=rectangle,fill=pantone157-8] (solution) at (2,1) {Solution};
\draw[-] (problem) -- (human);
\draw[->] (human) -- (model);
\draw[-] (model) -- (solver);
\draw[->] (solver) -- (solution);
\end{tikzpicture}
\end{center}
\end{frame}

\begin{frame}
\frametitle{More Realistic}
\begin{center}
\begin{tikzpicture}[xscale=4.0]
\node[shape=rectangle,fill=pantone127-4] (problem) at (2,5) {Problem};
\node (human) at (2,4) {Human};
\node[shape=rectangle,fill=pantone127-4] (model) at (2,3) {Model};
\node (solver) at (2,2) {Constraint Solver/Search};
\node[shape=rectangle,fill=pantone24-8] (hangs) at (1,1) {Hangs};
\node[shape=rectangle,fill=pantone157-8] (solution) at (2,1) {Solution};
\node[shape=rectangle,fill=pantone24-8] (wrong) at (3,1) {Wrong Solution};
\draw[-] (problem) -- (human);
\draw[->] (human) -- (model);
\draw[-] (model) -- (solver);
\draw[->] (solver) -- (hangs);
\draw[->] (solver) -- (solution);
\draw[->] (solver) -- (wrong);
\draw[->] (hangs) -- (1,4) -- (human);
\draw[->] (wrong) -- (3,4) -- (human);
\end{tikzpicture}
\end{center}
\end{frame}

\begin{frame}
\frametitle{Dual Role of Model}
\begin{itemize}
\item Allows Human to Express Problem
\begin{itemize}
\item Close to Problem Domain
\item Constraints as Abstractions
\end{itemize}
\item Allows Solver to Execute
\begin{itemize}
\item Variables as Communication Mechanism
\item Constraints as Algorithms
\end{itemize}
\end{itemize}
\end{frame}

\begin{frame}
\frametitle{Modelling Frameworks}
\begin{itemize}
\item \href{https://www.minizinc.org/}{MiniZinc} (NICTA, Monash University, Australia)
\item \href{https://github.com/eomahony/Numberjack}{NumberJack} (Insight, Ireland) 
\item \href{https://www-users.york.ac.uk/peter.nightingale/savilerow//}{EssencePrime/SavilleRow} (UK) 
\item \href{https://cpmpy.readthedocs.io/en/latest/}{CPMpy} (KU Leuven)
\item Allow use of multiple back-end solvers
\item Compile model into variants for each solver
\item A priori solver independent model(CP, MIP, SAT)
\end{itemize}
\end{frame} 

\begin{frame}
\frametitle{Framework Process}
\begin{center}
\begin{tikzpicture}[xscale=2]
\node[shape=rectangle,fill=pantone127-4] (problem) at (2.5,5) {Problem};
\node (human) at (2.5,4) {Human};
\node[shape=rectangle,fill=pantone127-4] (model) at (2.5,3) {Model};
\node (compile) at (2.5,2) {Compile/Reformulate};
\node (solver1) at (1,1) {CP};
\node (solver2) at (2,1) {MIP};
\node (solver3) at (3,1) {SAT};
\node (solver4) at (4,1) {Other};
\node[shape=rectangle,fill=pantone157-8] (solution1) at (1,0) {Solution};
\node[shape=rectangle,fill=pantone157-8] (solution2) at (2,0) {Solution};
\node[shape=rectangle,fill=pantone157-8] (solution3) at (3,0) {Solution};
\node[shape=rectangle,fill=pantone157-8] (solution4) at (4,0) {Solution};
\draw[-] (problem) -- (human);
\draw[->] (human) -- (model);
\draw[-] (model) -- (compile);
\draw[-] (compile) -- (solver1);
\draw[-] (compile) -- (solver2);
\draw[-] (compile) -- (solver3);
\draw[-] (compile) -- (solver4);
\draw[->] (solver1) -- (solution1);
\draw[->] (solver2) -- (solution2);
\draw[->] (solver3) -- (solution3);
\draw[->] (solver4) -- (solution4);
\end{tikzpicture}
\end{center}
\end{frame}

% \begin{frame}
  % \frametitle{Do It Now!}
  % \begin{itemize}
  % \item Download and install Minizinc
  % \item \url{https://www.minizinc.org/}
  % \end{itemize}
% \end{frame}


\begin{frame}
\frametitle{Why use Puzzles as Examples?}
\begin{itemize}
\item Easy to understand the problem
\item Solvable by hand without specialized knowledge
\item Possible to compare automated to manual solving process
\end{itemize}

\begin{quote}
The puzzle, though inanimate, is presented as a solvable problem without lasting negative consequences, a very low-risk low-reward situation. By being a puzzle, the object is attempting to convince the user that it must be completed.
\end{quote}
{\tiny Source: \href{https://www.everydayrhetoric.com/2020/03/the-rhetoric-of-puzzles/}{Every Day Rhetoric}}
\end{frame}


\mode<all>{
\part{Basic Constraint Propagation}
}
\begin{frame}
\partpage
\end{frame}
\mode<all>{
\input{../sendmore/body}
}

\mode<all>{
\part{Global Constraints}
}
\begin{frame}
\partpage
\end{frame}
\mode<all>{
\input{../sudoku/body}
}

\clearpage
\mode<all>{
\part{Customizing Search}
}
\begin{frame}
\partpage
\end{frame}
\mode<all>{
\input{../nqueen/body}
}


%% \clearpage
%% \mode<all>{
%%   \part{Problem Decomposition}
%% }
%% \begin{frame}
%%   \partpage
%% \end{frame}
%% \mode<all>{
%%   \input{../party/body}
%%   }

%% \clearpage
%% \mode<all>{
%%   \part{More Global Constraints}
%% }
%% \begin{frame}
%%   \partpage
%% \end{frame}
%% \mode<all>{
%%   \input{../car/body}
%%   }

%% \clearpage
%% \mode<all>{
%%   \part{Methodology}
%% }
%% \begin{frame}
%%   \partpage
%% \end{frame}
%% \mode<all>{
%%   \input{../methodology/body}
%%   }

\clearpage
\mode<all>{
\part{What is missing?}
}
\begin{frame}
\partpage
\end{frame}

\begin{frame}
\frametitle{Many Specialized Topics}
\begin{itemize}
\item How to design efficient core engine
\item Hybrids with LP/MIP tools
\item Hybrids with SAT
\item Symmetry breaking
\item Use of MDD/BDD to encode sets of solutions
\item High level modelling tools
\item Debugging/visualization
\end{itemize}
\end{frame}

\begin{frame}
\frametitle{Reformulation}
\begin{itemize}
\item Just because the user has modelled it this way, it doesn't mean we have to solve it that way
\begin{itemize}
\item Replace some constraint(s) by other, equivalent constraints
\item Because we don't have that constraint in our system
\item For performance
\end{itemize}
\end{itemize}
\end{frame}

\begin{frame}
\frametitle{Learning}
\begin{itemize}
\item While solving the problem we can learn how to strengthen the model/search 
\begin{itemize}
\item Understand which constraints/method contribute to propagation and change schedule
\item Learn no-good constraints by explaining failure
\item Adapt search strategy based on search experience
\end{itemize}
\end{itemize}
\end{frame}

%% \begin{frame}
%% \frametitle{Refined Process}
%% \begin{center}
%% \begin{tikzpicture}
%% \node[shape=rectangle,fill=pantone127-4] (problem) at (3,7) {Problem};
%% \node (human) at (3,6) {Human};
%% \node[shape=rectangle,fill=pantone127-4] (model) at (3,5) {User Model};
%% \node (reformulation) at (3,4) {Reformulation};
%% \node[shape=rectangle,fill=pantone127-4] (implementation) at (3,3) {Implementation Model};
%% \node (solver) at (3,2) {Constraint Solver/Search};
%% \node[shape=rectangle,fill=pantone157-8] (solution) at (3,1) {Solution};
%% \draw[-] (problem) -- (human);
%% \draw[->] (human) -- (model);
%% \draw[-] (model) -- (reformulation);
%% \draw[->] (reformulation) -- (implementation);
%% \draw[-] (implementation) -- (solver);
%% \draw[->] (solver) -- (0,2) -- (0,3) -- (implementation);
%% \draw[->] (solver) -- (solution);
%% \end{tikzpicture}
%% \end{center}
%% \end{frame}

%\begin{frame}
%\frametitle{What is CP actually used for?}
%\url{http://hsimonis.wordpress.com}
%\includegraphics[width=10cm]{../introduction/images/blog}
%\end{frame}

%% \begin{frame}
%% \frametitle{Exercises}
%% \begin{itemize}
%% \item Install Constraint Solver of your Choice
%% \item MiniZinc (MiningZinc) will be used in later sessions
%% \item Run Sendmoremoney, Sudoku, Queens examples
%% \item Can you choose which propagator to use for alldifferent?
%% \item Can you use different search strategies?
%% \end{itemize}
%% \end{frame}

\begin{frame}
\frametitle{More Learning Resources}
\begin{itemize}
\item Survey of Methods, Resources, and Formats for Teaching Constraint Programming
\begin{itemize}
\item by Tejas Santanam, Helmut Simonis
\item \url{https://doi.org/10.48550/arXiv.2403.12717}
\item Based on survey of community for WTCP 2023
\item \url{https://hsimonis.github.io/WTCP2023/}
\end{itemize}
\end{itemize}
\end{frame}


%%   }

%% \clearpage
%% \mode<all>{
%%   \part{Methodology}
%% }
%% \begin{frame}
%%   \partpage
%% \end{frame}
%% \mode<all>{
%%   \begin{frame}
\frametitle{Objectives}
\begin{itemize}
\item Overview of Core Constraint Programming
\item Three Main Concepts
\begin{itemize}
\item Constraint Propagation
\item Global Constraints
\item Customizing Search
\end{itemize}
%\item Get Some Experience with MiniZinc
\item Topics will be treated in more detail in later parts of the school
\item Based on Examples, not Formal Description
\end{itemize}
\end{frame}

\begin{frame}
\frametitle{Outline}
\begin{itemize}
\item Why Constraint Programming?
\item Constraint Propagation
\item Global Constraints
\item Customizing Search
%\item Choices in Modelling
%\item Decomposition
%\item More Global Constraints
%\item Methodology
%\item What is missing?
\end{itemize}
\end{frame}

% \begin{frame}
  % \frametitle{Using MiniZinc IDE}
  % \begin{itemize}
  % \item Developed in the Australian NICTA project
  % \item Maintained by Monash University
  % \item Modelling tool with multiple back-end solvers
  % \item Available from \url{https://www.minizinc.org/}
  % \end{itemize}
% \end{frame}


% \begin{frame}
% \frametitle{Examples in ECLiPSe}
% \begin{itemize}
% \item Open sourced constraint programming language
% \item Development goes back to 1985
% \item ECRC, ICL, IC-Parc, PTL, Cisco
% \item \url{https://eclipseclp.org/}
% \item Specialities
% \begin{itemize}
% \item Develop new solvers for specific domains
% \item Integration with MIP
% \end{itemize}
% \item Not included in bundled MiniZinc IDE
% \item Specialized visualization tools used here
  % \begin{itemize}
  % \item CP-Viz, Simonis et al. 2010
    % \end{itemize}
% \end{itemize}
% \end{frame}

\begin{frame}
\frametitle{Tutorial Based on ECLiPSe ELearning Course}
\begin{itemize}
\item Self-study course in constraint programming
\item Supported by Cisco Systems and Silicon Valley Community Foundation
\item Multi-media format, video lectures, slides, handout etc
\item {\small \url{https://eclipseclp.org/ELearning/index.html}}
\end{itemize}
\end{frame}

\begin{frame}
\frametitle{Also Part of CRT-AI Constraint Week}
\begin{itemize}
\item Annual one week course on CP and Optimization in Ireland
\item Part of national training program for PhD students in AI
\item \url{https://www.crt-ai.ie/}
\end{itemize}
\end{frame}



\begin{frame}
\frametitle{Constraint Programming - in a nutshell}
\begin{itemize}
\item Declarative description of problems with
\begin{itemize}
\item {\em Variables} which range over (finite) sets of values
\item {\em Constraints} over subsets of variables which restrict possible value combinations
\item A {\em solution} is a value assignment which satisfies all constraints
\end{itemize}

\item Constraint propagation/reasoning
\begin{itemize}
\item Removing inconsistent values for variables
\item Detect failure if constraint can not be satisfied
\item Interaction of constraints via shared variables
\item Incomplete
\end{itemize}

\item Search
\begin{itemize}
\item User controlled assignment of values to variables
\item Each step triggers constraint propagation 
\end{itemize}
\item Different domains require/allow different methods
\end{itemize}
\end{frame}

\begin{frame}
  \frametitle{Constraint Programming is Different}
  \begin{itemize}
  \item Declarative Programming
    \begin{itemize}
    \item Concentrate on what you want
      \item Not how to get there
      \item Program != Algorithm
      \item Program = Model
    \end{itemize}
    \item Applied to Combinatorial Problems
      \begin{itemize}
        \item No complete polynomial algorithms known (exist?)
        \item CP less ad-hoc than heuristics
        \item Models can evolve
  \end{itemize}
  \end{itemize}
  \end{frame}
    
\begin{frame}
  \frametitle{A Subtractive Process}
  \begin{textblock}{4}(8,-3)
    \includegraphics[width=4cm]{../introduction/images/stages}
  \end{textblock}
  \vfill
  \begin{quote}
    ``Oh, bosh, as Mr. Ruskin says. Sculpture, per se, is the simplest thing in the world. All you have to do is to take a big chunk of marble and a hammer and chisel, make up your mind what you are about to create and chip off all the marble you don't want.''-Paris Gaulois.
  \end{quote}
  
  {\tiny Source: \url{https://quoteinvestigator.com/2014/06/22/chip-away/}}
\end{frame}

\begin{frame}
\frametitle{Basic Process}
\begin{center}
\begin{tikzpicture}
\node[shape=rectangle,fill=pantone127-4] (problem) at (2,5) {Problem};
\node (human) at (2,4) {Human};
\node[shape=rectangle,fill=pantone127-4] (model) at (2,3) {Model};
\node (solver) at (2,2) {Constraint Solver/Search};
\node[shape=rectangle,fill=pantone157-8] (solution) at (2,1) {Solution};
\draw[-] (problem) -- (human);
\draw[->] (human) -- (model);
\draw[-] (model) -- (solver);
\draw[->] (solver) -- (solution);
\end{tikzpicture}
\end{center}
\end{frame}

\begin{frame}
\frametitle{More Realistic}
\begin{center}
\begin{tikzpicture}[xscale=4.0]
\node[shape=rectangle,fill=pantone127-4] (problem) at (2,5) {Problem};
\node (human) at (2,4) {Human};
\node[shape=rectangle,fill=pantone127-4] (model) at (2,3) {Model};
\node (solver) at (2,2) {Constraint Solver/Search};
\node[shape=rectangle,fill=pantone24-8] (hangs) at (1,1) {Hangs};
\node[shape=rectangle,fill=pantone157-8] (solution) at (2,1) {Solution};
\node[shape=rectangle,fill=pantone24-8] (wrong) at (3,1) {Wrong Solution};
\draw[-] (problem) -- (human);
\draw[->] (human) -- (model);
\draw[-] (model) -- (solver);
\draw[->] (solver) -- (hangs);
\draw[->] (solver) -- (solution);
\draw[->] (solver) -- (wrong);
\draw[->] (hangs) -- (1,4) -- (human);
\draw[->] (wrong) -- (3,4) -- (human);
\end{tikzpicture}
\end{center}
\end{frame}

\begin{frame}
\frametitle{Dual Role of Model}
\begin{itemize}
\item Allows Human to Express Problem
\begin{itemize}
\item Close to Problem Domain
\item Constraints as Abstractions
\end{itemize}
\item Allows Solver to Execute
\begin{itemize}
\item Variables as Communication Mechanism
\item Constraints as Algorithms
\end{itemize}
\end{itemize}
\end{frame}

\begin{frame}
\frametitle{Modelling Frameworks}
\begin{itemize}
\item \href{https://www.minizinc.org/}{MiniZinc} (NICTA, Monash University, Australia)
\item \href{https://github.com/eomahony/Numberjack}{NumberJack} (Insight, Ireland) 
\item \href{https://www-users.york.ac.uk/peter.nightingale/savilerow//}{EssencePrime/SavilleRow} (UK) 
\item \href{https://cpmpy.readthedocs.io/en/latest/}{CPMpy} (KU Leuven)
\item Allow use of multiple back-end solvers
\item Compile model into variants for each solver
\item A priori solver independent model(CP, MIP, SAT)
\end{itemize}
\end{frame} 

\begin{frame}
\frametitle{Framework Process}
\begin{center}
\begin{tikzpicture}[xscale=2]
\node[shape=rectangle,fill=pantone127-4] (problem) at (2.5,5) {Problem};
\node (human) at (2.5,4) {Human};
\node[shape=rectangle,fill=pantone127-4] (model) at (2.5,3) {Model};
\node (compile) at (2.5,2) {Compile/Reformulate};
\node (solver1) at (1,1) {CP};
\node (solver2) at (2,1) {MIP};
\node (solver3) at (3,1) {SAT};
\node (solver4) at (4,1) {Other};
\node[shape=rectangle,fill=pantone157-8] (solution1) at (1,0) {Solution};
\node[shape=rectangle,fill=pantone157-8] (solution2) at (2,0) {Solution};
\node[shape=rectangle,fill=pantone157-8] (solution3) at (3,0) {Solution};
\node[shape=rectangle,fill=pantone157-8] (solution4) at (4,0) {Solution};
\draw[-] (problem) -- (human);
\draw[->] (human) -- (model);
\draw[-] (model) -- (compile);
\draw[-] (compile) -- (solver1);
\draw[-] (compile) -- (solver2);
\draw[-] (compile) -- (solver3);
\draw[-] (compile) -- (solver4);
\draw[->] (solver1) -- (solution1);
\draw[->] (solver2) -- (solution2);
\draw[->] (solver3) -- (solution3);
\draw[->] (solver4) -- (solution4);
\end{tikzpicture}
\end{center}
\end{frame}

% \begin{frame}
  % \frametitle{Do It Now!}
  % \begin{itemize}
  % \item Download and install Minizinc
  % \item \url{https://www.minizinc.org/}
  % \end{itemize}
% \end{frame}


\begin{frame}
\frametitle{Why use Puzzles as Examples?}
\begin{itemize}
\item Easy to understand the problem
\item Solvable by hand without specialized knowledge
\item Possible to compare automated to manual solving process
\end{itemize}

\begin{quote}
The puzzle, though inanimate, is presented as a solvable problem without lasting negative consequences, a very low-risk low-reward situation. By being a puzzle, the object is attempting to convince the user that it must be completed.
\end{quote}
{\tiny Source: \href{https://www.everydayrhetoric.com/2020/03/the-rhetoric-of-puzzles/}{Every Day Rhetoric}}
\end{frame}


\mode<all>{
\part{Basic Constraint Propagation}
}
\begin{frame}
\partpage
\end{frame}
\mode<all>{
\input{../sendmore/body}
}

\mode<all>{
\part{Global Constraints}
}
\begin{frame}
\partpage
\end{frame}
\mode<all>{
\input{../sudoku/body}
}

\clearpage
\mode<all>{
\part{Customizing Search}
}
\begin{frame}
\partpage
\end{frame}
\mode<all>{
\input{../nqueen/body}
}


%% \clearpage
%% \mode<all>{
%%   \part{Problem Decomposition}
%% }
%% \begin{frame}
%%   \partpage
%% \end{frame}
%% \mode<all>{
%%   \input{../party/body}
%%   }

%% \clearpage
%% \mode<all>{
%%   \part{More Global Constraints}
%% }
%% \begin{frame}
%%   \partpage
%% \end{frame}
%% \mode<all>{
%%   \input{../car/body}
%%   }

%% \clearpage
%% \mode<all>{
%%   \part{Methodology}
%% }
%% \begin{frame}
%%   \partpage
%% \end{frame}
%% \mode<all>{
%%   \input{../methodology/body}
%%   }

\clearpage
\mode<all>{
\part{What is missing?}
}
\begin{frame}
\partpage
\end{frame}

\begin{frame}
\frametitle{Many Specialized Topics}
\begin{itemize}
\item How to design efficient core engine
\item Hybrids with LP/MIP tools
\item Hybrids with SAT
\item Symmetry breaking
\item Use of MDD/BDD to encode sets of solutions
\item High level modelling tools
\item Debugging/visualization
\end{itemize}
\end{frame}

\begin{frame}
\frametitle{Reformulation}
\begin{itemize}
\item Just because the user has modelled it this way, it doesn't mean we have to solve it that way
\begin{itemize}
\item Replace some constraint(s) by other, equivalent constraints
\item Because we don't have that constraint in our system
\item For performance
\end{itemize}
\end{itemize}
\end{frame}

\begin{frame}
\frametitle{Learning}
\begin{itemize}
\item While solving the problem we can learn how to strengthen the model/search 
\begin{itemize}
\item Understand which constraints/method contribute to propagation and change schedule
\item Learn no-good constraints by explaining failure
\item Adapt search strategy based on search experience
\end{itemize}
\end{itemize}
\end{frame}

%% \begin{frame}
%% \frametitle{Refined Process}
%% \begin{center}
%% \begin{tikzpicture}
%% \node[shape=rectangle,fill=pantone127-4] (problem) at (3,7) {Problem};
%% \node (human) at (3,6) {Human};
%% \node[shape=rectangle,fill=pantone127-4] (model) at (3,5) {User Model};
%% \node (reformulation) at (3,4) {Reformulation};
%% \node[shape=rectangle,fill=pantone127-4] (implementation) at (3,3) {Implementation Model};
%% \node (solver) at (3,2) {Constraint Solver/Search};
%% \node[shape=rectangle,fill=pantone157-8] (solution) at (3,1) {Solution};
%% \draw[-] (problem) -- (human);
%% \draw[->] (human) -- (model);
%% \draw[-] (model) -- (reformulation);
%% \draw[->] (reformulation) -- (implementation);
%% \draw[-] (implementation) -- (solver);
%% \draw[->] (solver) -- (0,2) -- (0,3) -- (implementation);
%% \draw[->] (solver) -- (solution);
%% \end{tikzpicture}
%% \end{center}
%% \end{frame}

%\begin{frame}
%\frametitle{What is CP actually used for?}
%\url{http://hsimonis.wordpress.com}
%\includegraphics[width=10cm]{../introduction/images/blog}
%\end{frame}

%% \begin{frame}
%% \frametitle{Exercises}
%% \begin{itemize}
%% \item Install Constraint Solver of your Choice
%% \item MiniZinc (MiningZinc) will be used in later sessions
%% \item Run Sendmoremoney, Sudoku, Queens examples
%% \item Can you choose which propagator to use for alldifferent?
%% \item Can you use different search strategies?
%% \end{itemize}
%% \end{frame}

\begin{frame}
\frametitle{More Learning Resources}
\begin{itemize}
\item Survey of Methods, Resources, and Formats for Teaching Constraint Programming
\begin{itemize}
\item by Tejas Santanam, Helmut Simonis
\item \url{https://doi.org/10.48550/arXiv.2403.12717}
\item Based on survey of community for WTCP 2023
\item \url{https://hsimonis.github.io/WTCP2023/}
\end{itemize}
\end{itemize}
\end{frame}


%%   }

\clearpage
\mode<all>{
\part{What is missing?}
}
\begin{frame}
\partpage
\end{frame}

\begin{frame}
\frametitle{Many Specialized Topics}
\begin{itemize}
\item How to design efficient core engine
\item Hybrids with LP/MIP tools
\item Hybrids with SAT
\item Symmetry breaking
\item Use of MDD/BDD to encode sets of solutions
\item High level modelling tools
\item Debugging/visualization
\end{itemize}
\end{frame}

\begin{frame}
\frametitle{Reformulation}
\begin{itemize}
\item Just because the user has modelled it this way, it doesn't mean we have to solve it that way
\begin{itemize}
\item Replace some constraint(s) by other, equivalent constraints
\item Because we don't have that constraint in our system
\item For performance
\end{itemize}
\end{itemize}
\end{frame}

\begin{frame}
\frametitle{Learning}
\begin{itemize}
\item While solving the problem we can learn how to strengthen the model/search 
\begin{itemize}
\item Understand which constraints/method contribute to propagation and change schedule
\item Learn no-good constraints by explaining failure
\item Adapt search strategy based on search experience
\end{itemize}
\end{itemize}
\end{frame}

%% \begin{frame}
%% \frametitle{Refined Process}
%% \begin{center}
%% \begin{tikzpicture}
%% \node[shape=rectangle,fill=pantone127-4] (problem) at (3,7) {Problem};
%% \node (human) at (3,6) {Human};
%% \node[shape=rectangle,fill=pantone127-4] (model) at (3,5) {User Model};
%% \node (reformulation) at (3,4) {Reformulation};
%% \node[shape=rectangle,fill=pantone127-4] (implementation) at (3,3) {Implementation Model};
%% \node (solver) at (3,2) {Constraint Solver/Search};
%% \node[shape=rectangle,fill=pantone157-8] (solution) at (3,1) {Solution};
%% \draw[-] (problem) -- (human);
%% \draw[->] (human) -- (model);
%% \draw[-] (model) -- (reformulation);
%% \draw[->] (reformulation) -- (implementation);
%% \draw[-] (implementation) -- (solver);
%% \draw[->] (solver) -- (0,2) -- (0,3) -- (implementation);
%% \draw[->] (solver) -- (solution);
%% \end{tikzpicture}
%% \end{center}
%% \end{frame}

%\begin{frame}
%\frametitle{What is CP actually used for?}
%\url{http://hsimonis.wordpress.com}
%\includegraphics[width=10cm]{../introduction/images/blog}
%\end{frame}

%% \begin{frame}
%% \frametitle{Exercises}
%% \begin{itemize}
%% \item Install Constraint Solver of your Choice
%% \item MiniZinc (MiningZinc) will be used in later sessions
%% \item Run Sendmoremoney, Sudoku, Queens examples
%% \item Can you choose which propagator to use for alldifferent?
%% \item Can you use different search strategies?
%% \end{itemize}
%% \end{frame}

\begin{frame}
\frametitle{More Learning Resources}
\begin{itemize}
\item Survey of Methods, Resources, and Formats for Teaching Constraint Programming
\begin{itemize}
\item by Tejas Santanam, Helmut Simonis
\item \url{https://doi.org/10.48550/arXiv.2403.12717}
\item Based on survey of community for WTCP 2023
\item \url{https://hsimonis.github.io/WTCP2023/}
\end{itemize}
\end{itemize}
\end{frame}


%%   }

%% \clearpage
%% \mode<all>{
%%   \part{Methodology}
%% }
%% \begin{frame}
%%   \partpage
%% \end{frame}
%% \mode<all>{
%%   \begin{frame}
\frametitle{Objectives}
\begin{itemize}
\item Overview of Core Constraint Programming
\item Three Main Concepts
\begin{itemize}
\item Constraint Propagation
\item Global Constraints
\item Customizing Search
\end{itemize}
%\item Get Some Experience with MiniZinc
\item Topics will be treated in more detail in later parts of the school
\item Based on Examples, not Formal Description
\end{itemize}
\end{frame}

\begin{frame}
\frametitle{Outline}
\begin{itemize}
\item Why Constraint Programming?
\item Constraint Propagation
\item Global Constraints
\item Customizing Search
%\item Choices in Modelling
%\item Decomposition
%\item More Global Constraints
%\item Methodology
%\item What is missing?
\end{itemize}
\end{frame}

% \begin{frame}
  % \frametitle{Using MiniZinc IDE}
  % \begin{itemize}
  % \item Developed in the Australian NICTA project
  % \item Maintained by Monash University
  % \item Modelling tool with multiple back-end solvers
  % \item Available from \url{https://www.minizinc.org/}
  % \end{itemize}
% \end{frame}


% \begin{frame}
% \frametitle{Examples in ECLiPSe}
% \begin{itemize}
% \item Open sourced constraint programming language
% \item Development goes back to 1985
% \item ECRC, ICL, IC-Parc, PTL, Cisco
% \item \url{https://eclipseclp.org/}
% \item Specialities
% \begin{itemize}
% \item Develop new solvers for specific domains
% \item Integration with MIP
% \end{itemize}
% \item Not included in bundled MiniZinc IDE
% \item Specialized visualization tools used here
  % \begin{itemize}
  % \item CP-Viz, Simonis et al. 2010
    % \end{itemize}
% \end{itemize}
% \end{frame}

\begin{frame}
\frametitle{Tutorial Based on ECLiPSe ELearning Course}
\begin{itemize}
\item Self-study course in constraint programming
\item Supported by Cisco Systems and Silicon Valley Community Foundation
\item Multi-media format, video lectures, slides, handout etc
\item {\small \url{https://eclipseclp.org/ELearning/index.html}}
\end{itemize}
\end{frame}

\begin{frame}
\frametitle{Also Part of CRT-AI Constraint Week}
\begin{itemize}
\item Annual one week course on CP and Optimization in Ireland
\item Part of national training program for PhD students in AI
\item \url{https://www.crt-ai.ie/}
\end{itemize}
\end{frame}



\begin{frame}
\frametitle{Constraint Programming - in a nutshell}
\begin{itemize}
\item Declarative description of problems with
\begin{itemize}
\item {\em Variables} which range over (finite) sets of values
\item {\em Constraints} over subsets of variables which restrict possible value combinations
\item A {\em solution} is a value assignment which satisfies all constraints
\end{itemize}

\item Constraint propagation/reasoning
\begin{itemize}
\item Removing inconsistent values for variables
\item Detect failure if constraint can not be satisfied
\item Interaction of constraints via shared variables
\item Incomplete
\end{itemize}

\item Search
\begin{itemize}
\item User controlled assignment of values to variables
\item Each step triggers constraint propagation 
\end{itemize}
\item Different domains require/allow different methods
\end{itemize}
\end{frame}

\begin{frame}
  \frametitle{Constraint Programming is Different}
  \begin{itemize}
  \item Declarative Programming
    \begin{itemize}
    \item Concentrate on what you want
      \item Not how to get there
      \item Program != Algorithm
      \item Program = Model
    \end{itemize}
    \item Applied to Combinatorial Problems
      \begin{itemize}
        \item No complete polynomial algorithms known (exist?)
        \item CP less ad-hoc than heuristics
        \item Models can evolve
  \end{itemize}
  \end{itemize}
  \end{frame}
    
\begin{frame}
  \frametitle{A Subtractive Process}
  \begin{textblock}{4}(8,-3)
    \includegraphics[width=4cm]{../introduction/images/stages}
  \end{textblock}
  \vfill
  \begin{quote}
    ``Oh, bosh, as Mr. Ruskin says. Sculpture, per se, is the simplest thing in the world. All you have to do is to take a big chunk of marble and a hammer and chisel, make up your mind what you are about to create and chip off all the marble you don't want.''-Paris Gaulois.
  \end{quote}
  
  {\tiny Source: \url{https://quoteinvestigator.com/2014/06/22/chip-away/}}
\end{frame}

\begin{frame}
\frametitle{Basic Process}
\begin{center}
\begin{tikzpicture}
\node[shape=rectangle,fill=pantone127-4] (problem) at (2,5) {Problem};
\node (human) at (2,4) {Human};
\node[shape=rectangle,fill=pantone127-4] (model) at (2,3) {Model};
\node (solver) at (2,2) {Constraint Solver/Search};
\node[shape=rectangle,fill=pantone157-8] (solution) at (2,1) {Solution};
\draw[-] (problem) -- (human);
\draw[->] (human) -- (model);
\draw[-] (model) -- (solver);
\draw[->] (solver) -- (solution);
\end{tikzpicture}
\end{center}
\end{frame}

\begin{frame}
\frametitle{More Realistic}
\begin{center}
\begin{tikzpicture}[xscale=4.0]
\node[shape=rectangle,fill=pantone127-4] (problem) at (2,5) {Problem};
\node (human) at (2,4) {Human};
\node[shape=rectangle,fill=pantone127-4] (model) at (2,3) {Model};
\node (solver) at (2,2) {Constraint Solver/Search};
\node[shape=rectangle,fill=pantone24-8] (hangs) at (1,1) {Hangs};
\node[shape=rectangle,fill=pantone157-8] (solution) at (2,1) {Solution};
\node[shape=rectangle,fill=pantone24-8] (wrong) at (3,1) {Wrong Solution};
\draw[-] (problem) -- (human);
\draw[->] (human) -- (model);
\draw[-] (model) -- (solver);
\draw[->] (solver) -- (hangs);
\draw[->] (solver) -- (solution);
\draw[->] (solver) -- (wrong);
\draw[->] (hangs) -- (1,4) -- (human);
\draw[->] (wrong) -- (3,4) -- (human);
\end{tikzpicture}
\end{center}
\end{frame}

\begin{frame}
\frametitle{Dual Role of Model}
\begin{itemize}
\item Allows Human to Express Problem
\begin{itemize}
\item Close to Problem Domain
\item Constraints as Abstractions
\end{itemize}
\item Allows Solver to Execute
\begin{itemize}
\item Variables as Communication Mechanism
\item Constraints as Algorithms
\end{itemize}
\end{itemize}
\end{frame}

\begin{frame}
\frametitle{Modelling Frameworks}
\begin{itemize}
\item \href{https://www.minizinc.org/}{MiniZinc} (NICTA, Monash University, Australia)
\item \href{https://github.com/eomahony/Numberjack}{NumberJack} (Insight, Ireland) 
\item \href{https://www-users.york.ac.uk/peter.nightingale/savilerow//}{EssencePrime/SavilleRow} (UK) 
\item \href{https://cpmpy.readthedocs.io/en/latest/}{CPMpy} (KU Leuven)
\item Allow use of multiple back-end solvers
\item Compile model into variants for each solver
\item A priori solver independent model(CP, MIP, SAT)
\end{itemize}
\end{frame} 

\begin{frame}
\frametitle{Framework Process}
\begin{center}
\begin{tikzpicture}[xscale=2]
\node[shape=rectangle,fill=pantone127-4] (problem) at (2.5,5) {Problem};
\node (human) at (2.5,4) {Human};
\node[shape=rectangle,fill=pantone127-4] (model) at (2.5,3) {Model};
\node (compile) at (2.5,2) {Compile/Reformulate};
\node (solver1) at (1,1) {CP};
\node (solver2) at (2,1) {MIP};
\node (solver3) at (3,1) {SAT};
\node (solver4) at (4,1) {Other};
\node[shape=rectangle,fill=pantone157-8] (solution1) at (1,0) {Solution};
\node[shape=rectangle,fill=pantone157-8] (solution2) at (2,0) {Solution};
\node[shape=rectangle,fill=pantone157-8] (solution3) at (3,0) {Solution};
\node[shape=rectangle,fill=pantone157-8] (solution4) at (4,0) {Solution};
\draw[-] (problem) -- (human);
\draw[->] (human) -- (model);
\draw[-] (model) -- (compile);
\draw[-] (compile) -- (solver1);
\draw[-] (compile) -- (solver2);
\draw[-] (compile) -- (solver3);
\draw[-] (compile) -- (solver4);
\draw[->] (solver1) -- (solution1);
\draw[->] (solver2) -- (solution2);
\draw[->] (solver3) -- (solution3);
\draw[->] (solver4) -- (solution4);
\end{tikzpicture}
\end{center}
\end{frame}

% \begin{frame}
  % \frametitle{Do It Now!}
  % \begin{itemize}
  % \item Download and install Minizinc
  % \item \url{https://www.minizinc.org/}
  % \end{itemize}
% \end{frame}


\begin{frame}
\frametitle{Why use Puzzles as Examples?}
\begin{itemize}
\item Easy to understand the problem
\item Solvable by hand without specialized knowledge
\item Possible to compare automated to manual solving process
\end{itemize}

\begin{quote}
The puzzle, though inanimate, is presented as a solvable problem without lasting negative consequences, a very low-risk low-reward situation. By being a puzzle, the object is attempting to convince the user that it must be completed.
\end{quote}
{\tiny Source: \href{https://www.everydayrhetoric.com/2020/03/the-rhetoric-of-puzzles/}{Every Day Rhetoric}}
\end{frame}


\mode<all>{
\part{Basic Constraint Propagation}
}
\begin{frame}
\partpage
\end{frame}
\mode<all>{
\begin{frame}
\frametitle{Objectives}
\begin{itemize}
\item Overview of Core Constraint Programming
\item Three Main Concepts
\begin{itemize}
\item Constraint Propagation
\item Global Constraints
\item Customizing Search
\end{itemize}
%\item Get Some Experience with MiniZinc
\item Topics will be treated in more detail in later parts of the school
\item Based on Examples, not Formal Description
\end{itemize}
\end{frame}

\begin{frame}
\frametitle{Outline}
\begin{itemize}
\item Why Constraint Programming?
\item Constraint Propagation
\item Global Constraints
\item Customizing Search
%\item Choices in Modelling
%\item Decomposition
%\item More Global Constraints
%\item Methodology
%\item What is missing?
\end{itemize}
\end{frame}

% \begin{frame}
  % \frametitle{Using MiniZinc IDE}
  % \begin{itemize}
  % \item Developed in the Australian NICTA project
  % \item Maintained by Monash University
  % \item Modelling tool with multiple back-end solvers
  % \item Available from \url{https://www.minizinc.org/}
  % \end{itemize}
% \end{frame}


% \begin{frame}
% \frametitle{Examples in ECLiPSe}
% \begin{itemize}
% \item Open sourced constraint programming language
% \item Development goes back to 1985
% \item ECRC, ICL, IC-Parc, PTL, Cisco
% \item \url{https://eclipseclp.org/}
% \item Specialities
% \begin{itemize}
% \item Develop new solvers for specific domains
% \item Integration with MIP
% \end{itemize}
% \item Not included in bundled MiniZinc IDE
% \item Specialized visualization tools used here
  % \begin{itemize}
  % \item CP-Viz, Simonis et al. 2010
    % \end{itemize}
% \end{itemize}
% \end{frame}

\begin{frame}
\frametitle{Tutorial Based on ECLiPSe ELearning Course}
\begin{itemize}
\item Self-study course in constraint programming
\item Supported by Cisco Systems and Silicon Valley Community Foundation
\item Multi-media format, video lectures, slides, handout etc
\item {\small \url{https://eclipseclp.org/ELearning/index.html}}
\end{itemize}
\end{frame}

\begin{frame}
\frametitle{Also Part of CRT-AI Constraint Week}
\begin{itemize}
\item Annual one week course on CP and Optimization in Ireland
\item Part of national training program for PhD students in AI
\item \url{https://www.crt-ai.ie/}
\end{itemize}
\end{frame}



\begin{frame}
\frametitle{Constraint Programming - in a nutshell}
\begin{itemize}
\item Declarative description of problems with
\begin{itemize}
\item {\em Variables} which range over (finite) sets of values
\item {\em Constraints} over subsets of variables which restrict possible value combinations
\item A {\em solution} is a value assignment which satisfies all constraints
\end{itemize}

\item Constraint propagation/reasoning
\begin{itemize}
\item Removing inconsistent values for variables
\item Detect failure if constraint can not be satisfied
\item Interaction of constraints via shared variables
\item Incomplete
\end{itemize}

\item Search
\begin{itemize}
\item User controlled assignment of values to variables
\item Each step triggers constraint propagation 
\end{itemize}
\item Different domains require/allow different methods
\end{itemize}
\end{frame}

\begin{frame}
  \frametitle{Constraint Programming is Different}
  \begin{itemize}
  \item Declarative Programming
    \begin{itemize}
    \item Concentrate on what you want
      \item Not how to get there
      \item Program != Algorithm
      \item Program = Model
    \end{itemize}
    \item Applied to Combinatorial Problems
      \begin{itemize}
        \item No complete polynomial algorithms known (exist?)
        \item CP less ad-hoc than heuristics
        \item Models can evolve
  \end{itemize}
  \end{itemize}
  \end{frame}
    
\begin{frame}
  \frametitle{A Subtractive Process}
  \begin{textblock}{4}(8,-3)
    \includegraphics[width=4cm]{../introduction/images/stages}
  \end{textblock}
  \vfill
  \begin{quote}
    ``Oh, bosh, as Mr. Ruskin says. Sculpture, per se, is the simplest thing in the world. All you have to do is to take a big chunk of marble and a hammer and chisel, make up your mind what you are about to create and chip off all the marble you don't want.''-Paris Gaulois.
  \end{quote}
  
  {\tiny Source: \url{https://quoteinvestigator.com/2014/06/22/chip-away/}}
\end{frame}

\begin{frame}
\frametitle{Basic Process}
\begin{center}
\begin{tikzpicture}
\node[shape=rectangle,fill=pantone127-4] (problem) at (2,5) {Problem};
\node (human) at (2,4) {Human};
\node[shape=rectangle,fill=pantone127-4] (model) at (2,3) {Model};
\node (solver) at (2,2) {Constraint Solver/Search};
\node[shape=rectangle,fill=pantone157-8] (solution) at (2,1) {Solution};
\draw[-] (problem) -- (human);
\draw[->] (human) -- (model);
\draw[-] (model) -- (solver);
\draw[->] (solver) -- (solution);
\end{tikzpicture}
\end{center}
\end{frame}

\begin{frame}
\frametitle{More Realistic}
\begin{center}
\begin{tikzpicture}[xscale=4.0]
\node[shape=rectangle,fill=pantone127-4] (problem) at (2,5) {Problem};
\node (human) at (2,4) {Human};
\node[shape=rectangle,fill=pantone127-4] (model) at (2,3) {Model};
\node (solver) at (2,2) {Constraint Solver/Search};
\node[shape=rectangle,fill=pantone24-8] (hangs) at (1,1) {Hangs};
\node[shape=rectangle,fill=pantone157-8] (solution) at (2,1) {Solution};
\node[shape=rectangle,fill=pantone24-8] (wrong) at (3,1) {Wrong Solution};
\draw[-] (problem) -- (human);
\draw[->] (human) -- (model);
\draw[-] (model) -- (solver);
\draw[->] (solver) -- (hangs);
\draw[->] (solver) -- (solution);
\draw[->] (solver) -- (wrong);
\draw[->] (hangs) -- (1,4) -- (human);
\draw[->] (wrong) -- (3,4) -- (human);
\end{tikzpicture}
\end{center}
\end{frame}

\begin{frame}
\frametitle{Dual Role of Model}
\begin{itemize}
\item Allows Human to Express Problem
\begin{itemize}
\item Close to Problem Domain
\item Constraints as Abstractions
\end{itemize}
\item Allows Solver to Execute
\begin{itemize}
\item Variables as Communication Mechanism
\item Constraints as Algorithms
\end{itemize}
\end{itemize}
\end{frame}

\begin{frame}
\frametitle{Modelling Frameworks}
\begin{itemize}
\item \href{https://www.minizinc.org/}{MiniZinc} (NICTA, Monash University, Australia)
\item \href{https://github.com/eomahony/Numberjack}{NumberJack} (Insight, Ireland) 
\item \href{https://www-users.york.ac.uk/peter.nightingale/savilerow//}{EssencePrime/SavilleRow} (UK) 
\item \href{https://cpmpy.readthedocs.io/en/latest/}{CPMpy} (KU Leuven)
\item Allow use of multiple back-end solvers
\item Compile model into variants for each solver
\item A priori solver independent model(CP, MIP, SAT)
\end{itemize}
\end{frame} 

\begin{frame}
\frametitle{Framework Process}
\begin{center}
\begin{tikzpicture}[xscale=2]
\node[shape=rectangle,fill=pantone127-4] (problem) at (2.5,5) {Problem};
\node (human) at (2.5,4) {Human};
\node[shape=rectangle,fill=pantone127-4] (model) at (2.5,3) {Model};
\node (compile) at (2.5,2) {Compile/Reformulate};
\node (solver1) at (1,1) {CP};
\node (solver2) at (2,1) {MIP};
\node (solver3) at (3,1) {SAT};
\node (solver4) at (4,1) {Other};
\node[shape=rectangle,fill=pantone157-8] (solution1) at (1,0) {Solution};
\node[shape=rectangle,fill=pantone157-8] (solution2) at (2,0) {Solution};
\node[shape=rectangle,fill=pantone157-8] (solution3) at (3,0) {Solution};
\node[shape=rectangle,fill=pantone157-8] (solution4) at (4,0) {Solution};
\draw[-] (problem) -- (human);
\draw[->] (human) -- (model);
\draw[-] (model) -- (compile);
\draw[-] (compile) -- (solver1);
\draw[-] (compile) -- (solver2);
\draw[-] (compile) -- (solver3);
\draw[-] (compile) -- (solver4);
\draw[->] (solver1) -- (solution1);
\draw[->] (solver2) -- (solution2);
\draw[->] (solver3) -- (solution3);
\draw[->] (solver4) -- (solution4);
\end{tikzpicture}
\end{center}
\end{frame}

% \begin{frame}
  % \frametitle{Do It Now!}
  % \begin{itemize}
  % \item Download and install Minizinc
  % \item \url{https://www.minizinc.org/}
  % \end{itemize}
% \end{frame}


\begin{frame}
\frametitle{Why use Puzzles as Examples?}
\begin{itemize}
\item Easy to understand the problem
\item Solvable by hand without specialized knowledge
\item Possible to compare automated to manual solving process
\end{itemize}

\begin{quote}
The puzzle, though inanimate, is presented as a solvable problem without lasting negative consequences, a very low-risk low-reward situation. By being a puzzle, the object is attempting to convince the user that it must be completed.
\end{quote}
{\tiny Source: \href{https://www.everydayrhetoric.com/2020/03/the-rhetoric-of-puzzles/}{Every Day Rhetoric}}
\end{frame}


\mode<all>{
\part{Basic Constraint Propagation}
}
\begin{frame}
\partpage
\end{frame}
\mode<all>{
\input{../sendmore/body}
}

\mode<all>{
\part{Global Constraints}
}
\begin{frame}
\partpage
\end{frame}
\mode<all>{
\input{../sudoku/body}
}

\clearpage
\mode<all>{
\part{Customizing Search}
}
\begin{frame}
\partpage
\end{frame}
\mode<all>{
\input{../nqueen/body}
}


%% \clearpage
%% \mode<all>{
%%   \part{Problem Decomposition}
%% }
%% \begin{frame}
%%   \partpage
%% \end{frame}
%% \mode<all>{
%%   \input{../party/body}
%%   }

%% \clearpage
%% \mode<all>{
%%   \part{More Global Constraints}
%% }
%% \begin{frame}
%%   \partpage
%% \end{frame}
%% \mode<all>{
%%   \input{../car/body}
%%   }

%% \clearpage
%% \mode<all>{
%%   \part{Methodology}
%% }
%% \begin{frame}
%%   \partpage
%% \end{frame}
%% \mode<all>{
%%   \input{../methodology/body}
%%   }

\clearpage
\mode<all>{
\part{What is missing?}
}
\begin{frame}
\partpage
\end{frame}

\begin{frame}
\frametitle{Many Specialized Topics}
\begin{itemize}
\item How to design efficient core engine
\item Hybrids with LP/MIP tools
\item Hybrids with SAT
\item Symmetry breaking
\item Use of MDD/BDD to encode sets of solutions
\item High level modelling tools
\item Debugging/visualization
\end{itemize}
\end{frame}

\begin{frame}
\frametitle{Reformulation}
\begin{itemize}
\item Just because the user has modelled it this way, it doesn't mean we have to solve it that way
\begin{itemize}
\item Replace some constraint(s) by other, equivalent constraints
\item Because we don't have that constraint in our system
\item For performance
\end{itemize}
\end{itemize}
\end{frame}

\begin{frame}
\frametitle{Learning}
\begin{itemize}
\item While solving the problem we can learn how to strengthen the model/search 
\begin{itemize}
\item Understand which constraints/method contribute to propagation and change schedule
\item Learn no-good constraints by explaining failure
\item Adapt search strategy based on search experience
\end{itemize}
\end{itemize}
\end{frame}

%% \begin{frame}
%% \frametitle{Refined Process}
%% \begin{center}
%% \begin{tikzpicture}
%% \node[shape=rectangle,fill=pantone127-4] (problem) at (3,7) {Problem};
%% \node (human) at (3,6) {Human};
%% \node[shape=rectangle,fill=pantone127-4] (model) at (3,5) {User Model};
%% \node (reformulation) at (3,4) {Reformulation};
%% \node[shape=rectangle,fill=pantone127-4] (implementation) at (3,3) {Implementation Model};
%% \node (solver) at (3,2) {Constraint Solver/Search};
%% \node[shape=rectangle,fill=pantone157-8] (solution) at (3,1) {Solution};
%% \draw[-] (problem) -- (human);
%% \draw[->] (human) -- (model);
%% \draw[-] (model) -- (reformulation);
%% \draw[->] (reformulation) -- (implementation);
%% \draw[-] (implementation) -- (solver);
%% \draw[->] (solver) -- (0,2) -- (0,3) -- (implementation);
%% \draw[->] (solver) -- (solution);
%% \end{tikzpicture}
%% \end{center}
%% \end{frame}

%\begin{frame}
%\frametitle{What is CP actually used for?}
%\url{http://hsimonis.wordpress.com}
%\includegraphics[width=10cm]{../introduction/images/blog}
%\end{frame}

%% \begin{frame}
%% \frametitle{Exercises}
%% \begin{itemize}
%% \item Install Constraint Solver of your Choice
%% \item MiniZinc (MiningZinc) will be used in later sessions
%% \item Run Sendmoremoney, Sudoku, Queens examples
%% \item Can you choose which propagator to use for alldifferent?
%% \item Can you use different search strategies?
%% \end{itemize}
%% \end{frame}

\begin{frame}
\frametitle{More Learning Resources}
\begin{itemize}
\item Survey of Methods, Resources, and Formats for Teaching Constraint Programming
\begin{itemize}
\item by Tejas Santanam, Helmut Simonis
\item \url{https://doi.org/10.48550/arXiv.2403.12717}
\item Based on survey of community for WTCP 2023
\item \url{https://hsimonis.github.io/WTCP2023/}
\end{itemize}
\end{itemize}
\end{frame}


}

\mode<all>{
\part{Global Constraints}
}
\begin{frame}
\partpage
\end{frame}
\mode<all>{
\begin{frame}
\frametitle{Objectives}
\begin{itemize}
\item Overview of Core Constraint Programming
\item Three Main Concepts
\begin{itemize}
\item Constraint Propagation
\item Global Constraints
\item Customizing Search
\end{itemize}
%\item Get Some Experience with MiniZinc
\item Topics will be treated in more detail in later parts of the school
\item Based on Examples, not Formal Description
\end{itemize}
\end{frame}

\begin{frame}
\frametitle{Outline}
\begin{itemize}
\item Why Constraint Programming?
\item Constraint Propagation
\item Global Constraints
\item Customizing Search
%\item Choices in Modelling
%\item Decomposition
%\item More Global Constraints
%\item Methodology
%\item What is missing?
\end{itemize}
\end{frame}

% \begin{frame}
  % \frametitle{Using MiniZinc IDE}
  % \begin{itemize}
  % \item Developed in the Australian NICTA project
  % \item Maintained by Monash University
  % \item Modelling tool with multiple back-end solvers
  % \item Available from \url{https://www.minizinc.org/}
  % \end{itemize}
% \end{frame}


% \begin{frame}
% \frametitle{Examples in ECLiPSe}
% \begin{itemize}
% \item Open sourced constraint programming language
% \item Development goes back to 1985
% \item ECRC, ICL, IC-Parc, PTL, Cisco
% \item \url{https://eclipseclp.org/}
% \item Specialities
% \begin{itemize}
% \item Develop new solvers for specific domains
% \item Integration with MIP
% \end{itemize}
% \item Not included in bundled MiniZinc IDE
% \item Specialized visualization tools used here
  % \begin{itemize}
  % \item CP-Viz, Simonis et al. 2010
    % \end{itemize}
% \end{itemize}
% \end{frame}

\begin{frame}
\frametitle{Tutorial Based on ECLiPSe ELearning Course}
\begin{itemize}
\item Self-study course in constraint programming
\item Supported by Cisco Systems and Silicon Valley Community Foundation
\item Multi-media format, video lectures, slides, handout etc
\item {\small \url{https://eclipseclp.org/ELearning/index.html}}
\end{itemize}
\end{frame}

\begin{frame}
\frametitle{Also Part of CRT-AI Constraint Week}
\begin{itemize}
\item Annual one week course on CP and Optimization in Ireland
\item Part of national training program for PhD students in AI
\item \url{https://www.crt-ai.ie/}
\end{itemize}
\end{frame}



\begin{frame}
\frametitle{Constraint Programming - in a nutshell}
\begin{itemize}
\item Declarative description of problems with
\begin{itemize}
\item {\em Variables} which range over (finite) sets of values
\item {\em Constraints} over subsets of variables which restrict possible value combinations
\item A {\em solution} is a value assignment which satisfies all constraints
\end{itemize}

\item Constraint propagation/reasoning
\begin{itemize}
\item Removing inconsistent values for variables
\item Detect failure if constraint can not be satisfied
\item Interaction of constraints via shared variables
\item Incomplete
\end{itemize}

\item Search
\begin{itemize}
\item User controlled assignment of values to variables
\item Each step triggers constraint propagation 
\end{itemize}
\item Different domains require/allow different methods
\end{itemize}
\end{frame}

\begin{frame}
  \frametitle{Constraint Programming is Different}
  \begin{itemize}
  \item Declarative Programming
    \begin{itemize}
    \item Concentrate on what you want
      \item Not how to get there
      \item Program != Algorithm
      \item Program = Model
    \end{itemize}
    \item Applied to Combinatorial Problems
      \begin{itemize}
        \item No complete polynomial algorithms known (exist?)
        \item CP less ad-hoc than heuristics
        \item Models can evolve
  \end{itemize}
  \end{itemize}
  \end{frame}
    
\begin{frame}
  \frametitle{A Subtractive Process}
  \begin{textblock}{4}(8,-3)
    \includegraphics[width=4cm]{../introduction/images/stages}
  \end{textblock}
  \vfill
  \begin{quote}
    ``Oh, bosh, as Mr. Ruskin says. Sculpture, per se, is the simplest thing in the world. All you have to do is to take a big chunk of marble and a hammer and chisel, make up your mind what you are about to create and chip off all the marble you don't want.''-Paris Gaulois.
  \end{quote}
  
  {\tiny Source: \url{https://quoteinvestigator.com/2014/06/22/chip-away/}}
\end{frame}

\begin{frame}
\frametitle{Basic Process}
\begin{center}
\begin{tikzpicture}
\node[shape=rectangle,fill=pantone127-4] (problem) at (2,5) {Problem};
\node (human) at (2,4) {Human};
\node[shape=rectangle,fill=pantone127-4] (model) at (2,3) {Model};
\node (solver) at (2,2) {Constraint Solver/Search};
\node[shape=rectangle,fill=pantone157-8] (solution) at (2,1) {Solution};
\draw[-] (problem) -- (human);
\draw[->] (human) -- (model);
\draw[-] (model) -- (solver);
\draw[->] (solver) -- (solution);
\end{tikzpicture}
\end{center}
\end{frame}

\begin{frame}
\frametitle{More Realistic}
\begin{center}
\begin{tikzpicture}[xscale=4.0]
\node[shape=rectangle,fill=pantone127-4] (problem) at (2,5) {Problem};
\node (human) at (2,4) {Human};
\node[shape=rectangle,fill=pantone127-4] (model) at (2,3) {Model};
\node (solver) at (2,2) {Constraint Solver/Search};
\node[shape=rectangle,fill=pantone24-8] (hangs) at (1,1) {Hangs};
\node[shape=rectangle,fill=pantone157-8] (solution) at (2,1) {Solution};
\node[shape=rectangle,fill=pantone24-8] (wrong) at (3,1) {Wrong Solution};
\draw[-] (problem) -- (human);
\draw[->] (human) -- (model);
\draw[-] (model) -- (solver);
\draw[->] (solver) -- (hangs);
\draw[->] (solver) -- (solution);
\draw[->] (solver) -- (wrong);
\draw[->] (hangs) -- (1,4) -- (human);
\draw[->] (wrong) -- (3,4) -- (human);
\end{tikzpicture}
\end{center}
\end{frame}

\begin{frame}
\frametitle{Dual Role of Model}
\begin{itemize}
\item Allows Human to Express Problem
\begin{itemize}
\item Close to Problem Domain
\item Constraints as Abstractions
\end{itemize}
\item Allows Solver to Execute
\begin{itemize}
\item Variables as Communication Mechanism
\item Constraints as Algorithms
\end{itemize}
\end{itemize}
\end{frame}

\begin{frame}
\frametitle{Modelling Frameworks}
\begin{itemize}
\item \href{https://www.minizinc.org/}{MiniZinc} (NICTA, Monash University, Australia)
\item \href{https://github.com/eomahony/Numberjack}{NumberJack} (Insight, Ireland) 
\item \href{https://www-users.york.ac.uk/peter.nightingale/savilerow//}{EssencePrime/SavilleRow} (UK) 
\item \href{https://cpmpy.readthedocs.io/en/latest/}{CPMpy} (KU Leuven)
\item Allow use of multiple back-end solvers
\item Compile model into variants for each solver
\item A priori solver independent model(CP, MIP, SAT)
\end{itemize}
\end{frame} 

\begin{frame}
\frametitle{Framework Process}
\begin{center}
\begin{tikzpicture}[xscale=2]
\node[shape=rectangle,fill=pantone127-4] (problem) at (2.5,5) {Problem};
\node (human) at (2.5,4) {Human};
\node[shape=rectangle,fill=pantone127-4] (model) at (2.5,3) {Model};
\node (compile) at (2.5,2) {Compile/Reformulate};
\node (solver1) at (1,1) {CP};
\node (solver2) at (2,1) {MIP};
\node (solver3) at (3,1) {SAT};
\node (solver4) at (4,1) {Other};
\node[shape=rectangle,fill=pantone157-8] (solution1) at (1,0) {Solution};
\node[shape=rectangle,fill=pantone157-8] (solution2) at (2,0) {Solution};
\node[shape=rectangle,fill=pantone157-8] (solution3) at (3,0) {Solution};
\node[shape=rectangle,fill=pantone157-8] (solution4) at (4,0) {Solution};
\draw[-] (problem) -- (human);
\draw[->] (human) -- (model);
\draw[-] (model) -- (compile);
\draw[-] (compile) -- (solver1);
\draw[-] (compile) -- (solver2);
\draw[-] (compile) -- (solver3);
\draw[-] (compile) -- (solver4);
\draw[->] (solver1) -- (solution1);
\draw[->] (solver2) -- (solution2);
\draw[->] (solver3) -- (solution3);
\draw[->] (solver4) -- (solution4);
\end{tikzpicture}
\end{center}
\end{frame}

% \begin{frame}
  % \frametitle{Do It Now!}
  % \begin{itemize}
  % \item Download and install Minizinc
  % \item \url{https://www.minizinc.org/}
  % \end{itemize}
% \end{frame}


\begin{frame}
\frametitle{Why use Puzzles as Examples?}
\begin{itemize}
\item Easy to understand the problem
\item Solvable by hand without specialized knowledge
\item Possible to compare automated to manual solving process
\end{itemize}

\begin{quote}
The puzzle, though inanimate, is presented as a solvable problem without lasting negative consequences, a very low-risk low-reward situation. By being a puzzle, the object is attempting to convince the user that it must be completed.
\end{quote}
{\tiny Source: \href{https://www.everydayrhetoric.com/2020/03/the-rhetoric-of-puzzles/}{Every Day Rhetoric}}
\end{frame}


\mode<all>{
\part{Basic Constraint Propagation}
}
\begin{frame}
\partpage
\end{frame}
\mode<all>{
\input{../sendmore/body}
}

\mode<all>{
\part{Global Constraints}
}
\begin{frame}
\partpage
\end{frame}
\mode<all>{
\input{../sudoku/body}
}

\clearpage
\mode<all>{
\part{Customizing Search}
}
\begin{frame}
\partpage
\end{frame}
\mode<all>{
\input{../nqueen/body}
}


%% \clearpage
%% \mode<all>{
%%   \part{Problem Decomposition}
%% }
%% \begin{frame}
%%   \partpage
%% \end{frame}
%% \mode<all>{
%%   \input{../party/body}
%%   }

%% \clearpage
%% \mode<all>{
%%   \part{More Global Constraints}
%% }
%% \begin{frame}
%%   \partpage
%% \end{frame}
%% \mode<all>{
%%   \input{../car/body}
%%   }

%% \clearpage
%% \mode<all>{
%%   \part{Methodology}
%% }
%% \begin{frame}
%%   \partpage
%% \end{frame}
%% \mode<all>{
%%   \input{../methodology/body}
%%   }

\clearpage
\mode<all>{
\part{What is missing?}
}
\begin{frame}
\partpage
\end{frame}

\begin{frame}
\frametitle{Many Specialized Topics}
\begin{itemize}
\item How to design efficient core engine
\item Hybrids with LP/MIP tools
\item Hybrids with SAT
\item Symmetry breaking
\item Use of MDD/BDD to encode sets of solutions
\item High level modelling tools
\item Debugging/visualization
\end{itemize}
\end{frame}

\begin{frame}
\frametitle{Reformulation}
\begin{itemize}
\item Just because the user has modelled it this way, it doesn't mean we have to solve it that way
\begin{itemize}
\item Replace some constraint(s) by other, equivalent constraints
\item Because we don't have that constraint in our system
\item For performance
\end{itemize}
\end{itemize}
\end{frame}

\begin{frame}
\frametitle{Learning}
\begin{itemize}
\item While solving the problem we can learn how to strengthen the model/search 
\begin{itemize}
\item Understand which constraints/method contribute to propagation and change schedule
\item Learn no-good constraints by explaining failure
\item Adapt search strategy based on search experience
\end{itemize}
\end{itemize}
\end{frame}

%% \begin{frame}
%% \frametitle{Refined Process}
%% \begin{center}
%% \begin{tikzpicture}
%% \node[shape=rectangle,fill=pantone127-4] (problem) at (3,7) {Problem};
%% \node (human) at (3,6) {Human};
%% \node[shape=rectangle,fill=pantone127-4] (model) at (3,5) {User Model};
%% \node (reformulation) at (3,4) {Reformulation};
%% \node[shape=rectangle,fill=pantone127-4] (implementation) at (3,3) {Implementation Model};
%% \node (solver) at (3,2) {Constraint Solver/Search};
%% \node[shape=rectangle,fill=pantone157-8] (solution) at (3,1) {Solution};
%% \draw[-] (problem) -- (human);
%% \draw[->] (human) -- (model);
%% \draw[-] (model) -- (reformulation);
%% \draw[->] (reformulation) -- (implementation);
%% \draw[-] (implementation) -- (solver);
%% \draw[->] (solver) -- (0,2) -- (0,3) -- (implementation);
%% \draw[->] (solver) -- (solution);
%% \end{tikzpicture}
%% \end{center}
%% \end{frame}

%\begin{frame}
%\frametitle{What is CP actually used for?}
%\url{http://hsimonis.wordpress.com}
%\includegraphics[width=10cm]{../introduction/images/blog}
%\end{frame}

%% \begin{frame}
%% \frametitle{Exercises}
%% \begin{itemize}
%% \item Install Constraint Solver of your Choice
%% \item MiniZinc (MiningZinc) will be used in later sessions
%% \item Run Sendmoremoney, Sudoku, Queens examples
%% \item Can you choose which propagator to use for alldifferent?
%% \item Can you use different search strategies?
%% \end{itemize}
%% \end{frame}

\begin{frame}
\frametitle{More Learning Resources}
\begin{itemize}
\item Survey of Methods, Resources, and Formats for Teaching Constraint Programming
\begin{itemize}
\item by Tejas Santanam, Helmut Simonis
\item \url{https://doi.org/10.48550/arXiv.2403.12717}
\item Based on survey of community for WTCP 2023
\item \url{https://hsimonis.github.io/WTCP2023/}
\end{itemize}
\end{itemize}
\end{frame}


}

\clearpage
\mode<all>{
\part{Customizing Search}
}
\begin{frame}
\partpage
\end{frame}
\mode<all>{
\begin{frame}
\frametitle{Objectives}
\begin{itemize}
\item Overview of Core Constraint Programming
\item Three Main Concepts
\begin{itemize}
\item Constraint Propagation
\item Global Constraints
\item Customizing Search
\end{itemize}
%\item Get Some Experience with MiniZinc
\item Topics will be treated in more detail in later parts of the school
\item Based on Examples, not Formal Description
\end{itemize}
\end{frame}

\begin{frame}
\frametitle{Outline}
\begin{itemize}
\item Why Constraint Programming?
\item Constraint Propagation
\item Global Constraints
\item Customizing Search
%\item Choices in Modelling
%\item Decomposition
%\item More Global Constraints
%\item Methodology
%\item What is missing?
\end{itemize}
\end{frame}

% \begin{frame}
  % \frametitle{Using MiniZinc IDE}
  % \begin{itemize}
  % \item Developed in the Australian NICTA project
  % \item Maintained by Monash University
  % \item Modelling tool with multiple back-end solvers
  % \item Available from \url{https://www.minizinc.org/}
  % \end{itemize}
% \end{frame}


% \begin{frame}
% \frametitle{Examples in ECLiPSe}
% \begin{itemize}
% \item Open sourced constraint programming language
% \item Development goes back to 1985
% \item ECRC, ICL, IC-Parc, PTL, Cisco
% \item \url{https://eclipseclp.org/}
% \item Specialities
% \begin{itemize}
% \item Develop new solvers for specific domains
% \item Integration with MIP
% \end{itemize}
% \item Not included in bundled MiniZinc IDE
% \item Specialized visualization tools used here
  % \begin{itemize}
  % \item CP-Viz, Simonis et al. 2010
    % \end{itemize}
% \end{itemize}
% \end{frame}

\begin{frame}
\frametitle{Tutorial Based on ECLiPSe ELearning Course}
\begin{itemize}
\item Self-study course in constraint programming
\item Supported by Cisco Systems and Silicon Valley Community Foundation
\item Multi-media format, video lectures, slides, handout etc
\item {\small \url{https://eclipseclp.org/ELearning/index.html}}
\end{itemize}
\end{frame}

\begin{frame}
\frametitle{Also Part of CRT-AI Constraint Week}
\begin{itemize}
\item Annual one week course on CP and Optimization in Ireland
\item Part of national training program for PhD students in AI
\item \url{https://www.crt-ai.ie/}
\end{itemize}
\end{frame}



\begin{frame}
\frametitle{Constraint Programming - in a nutshell}
\begin{itemize}
\item Declarative description of problems with
\begin{itemize}
\item {\em Variables} which range over (finite) sets of values
\item {\em Constraints} over subsets of variables which restrict possible value combinations
\item A {\em solution} is a value assignment which satisfies all constraints
\end{itemize}

\item Constraint propagation/reasoning
\begin{itemize}
\item Removing inconsistent values for variables
\item Detect failure if constraint can not be satisfied
\item Interaction of constraints via shared variables
\item Incomplete
\end{itemize}

\item Search
\begin{itemize}
\item User controlled assignment of values to variables
\item Each step triggers constraint propagation 
\end{itemize}
\item Different domains require/allow different methods
\end{itemize}
\end{frame}

\begin{frame}
  \frametitle{Constraint Programming is Different}
  \begin{itemize}
  \item Declarative Programming
    \begin{itemize}
    \item Concentrate on what you want
      \item Not how to get there
      \item Program != Algorithm
      \item Program = Model
    \end{itemize}
    \item Applied to Combinatorial Problems
      \begin{itemize}
        \item No complete polynomial algorithms known (exist?)
        \item CP less ad-hoc than heuristics
        \item Models can evolve
  \end{itemize}
  \end{itemize}
  \end{frame}
    
\begin{frame}
  \frametitle{A Subtractive Process}
  \begin{textblock}{4}(8,-3)
    \includegraphics[width=4cm]{../introduction/images/stages}
  \end{textblock}
  \vfill
  \begin{quote}
    ``Oh, bosh, as Mr. Ruskin says. Sculpture, per se, is the simplest thing in the world. All you have to do is to take a big chunk of marble and a hammer and chisel, make up your mind what you are about to create and chip off all the marble you don't want.''-Paris Gaulois.
  \end{quote}
  
  {\tiny Source: \url{https://quoteinvestigator.com/2014/06/22/chip-away/}}
\end{frame}

\begin{frame}
\frametitle{Basic Process}
\begin{center}
\begin{tikzpicture}
\node[shape=rectangle,fill=pantone127-4] (problem) at (2,5) {Problem};
\node (human) at (2,4) {Human};
\node[shape=rectangle,fill=pantone127-4] (model) at (2,3) {Model};
\node (solver) at (2,2) {Constraint Solver/Search};
\node[shape=rectangle,fill=pantone157-8] (solution) at (2,1) {Solution};
\draw[-] (problem) -- (human);
\draw[->] (human) -- (model);
\draw[-] (model) -- (solver);
\draw[->] (solver) -- (solution);
\end{tikzpicture}
\end{center}
\end{frame}

\begin{frame}
\frametitle{More Realistic}
\begin{center}
\begin{tikzpicture}[xscale=4.0]
\node[shape=rectangle,fill=pantone127-4] (problem) at (2,5) {Problem};
\node (human) at (2,4) {Human};
\node[shape=rectangle,fill=pantone127-4] (model) at (2,3) {Model};
\node (solver) at (2,2) {Constraint Solver/Search};
\node[shape=rectangle,fill=pantone24-8] (hangs) at (1,1) {Hangs};
\node[shape=rectangle,fill=pantone157-8] (solution) at (2,1) {Solution};
\node[shape=rectangle,fill=pantone24-8] (wrong) at (3,1) {Wrong Solution};
\draw[-] (problem) -- (human);
\draw[->] (human) -- (model);
\draw[-] (model) -- (solver);
\draw[->] (solver) -- (hangs);
\draw[->] (solver) -- (solution);
\draw[->] (solver) -- (wrong);
\draw[->] (hangs) -- (1,4) -- (human);
\draw[->] (wrong) -- (3,4) -- (human);
\end{tikzpicture}
\end{center}
\end{frame}

\begin{frame}
\frametitle{Dual Role of Model}
\begin{itemize}
\item Allows Human to Express Problem
\begin{itemize}
\item Close to Problem Domain
\item Constraints as Abstractions
\end{itemize}
\item Allows Solver to Execute
\begin{itemize}
\item Variables as Communication Mechanism
\item Constraints as Algorithms
\end{itemize}
\end{itemize}
\end{frame}

\begin{frame}
\frametitle{Modelling Frameworks}
\begin{itemize}
\item \href{https://www.minizinc.org/}{MiniZinc} (NICTA, Monash University, Australia)
\item \href{https://github.com/eomahony/Numberjack}{NumberJack} (Insight, Ireland) 
\item \href{https://www-users.york.ac.uk/peter.nightingale/savilerow//}{EssencePrime/SavilleRow} (UK) 
\item \href{https://cpmpy.readthedocs.io/en/latest/}{CPMpy} (KU Leuven)
\item Allow use of multiple back-end solvers
\item Compile model into variants for each solver
\item A priori solver independent model(CP, MIP, SAT)
\end{itemize}
\end{frame} 

\begin{frame}
\frametitle{Framework Process}
\begin{center}
\begin{tikzpicture}[xscale=2]
\node[shape=rectangle,fill=pantone127-4] (problem) at (2.5,5) {Problem};
\node (human) at (2.5,4) {Human};
\node[shape=rectangle,fill=pantone127-4] (model) at (2.5,3) {Model};
\node (compile) at (2.5,2) {Compile/Reformulate};
\node (solver1) at (1,1) {CP};
\node (solver2) at (2,1) {MIP};
\node (solver3) at (3,1) {SAT};
\node (solver4) at (4,1) {Other};
\node[shape=rectangle,fill=pantone157-8] (solution1) at (1,0) {Solution};
\node[shape=rectangle,fill=pantone157-8] (solution2) at (2,0) {Solution};
\node[shape=rectangle,fill=pantone157-8] (solution3) at (3,0) {Solution};
\node[shape=rectangle,fill=pantone157-8] (solution4) at (4,0) {Solution};
\draw[-] (problem) -- (human);
\draw[->] (human) -- (model);
\draw[-] (model) -- (compile);
\draw[-] (compile) -- (solver1);
\draw[-] (compile) -- (solver2);
\draw[-] (compile) -- (solver3);
\draw[-] (compile) -- (solver4);
\draw[->] (solver1) -- (solution1);
\draw[->] (solver2) -- (solution2);
\draw[->] (solver3) -- (solution3);
\draw[->] (solver4) -- (solution4);
\end{tikzpicture}
\end{center}
\end{frame}

% \begin{frame}
  % \frametitle{Do It Now!}
  % \begin{itemize}
  % \item Download and install Minizinc
  % \item \url{https://www.minizinc.org/}
  % \end{itemize}
% \end{frame}


\begin{frame}
\frametitle{Why use Puzzles as Examples?}
\begin{itemize}
\item Easy to understand the problem
\item Solvable by hand without specialized knowledge
\item Possible to compare automated to manual solving process
\end{itemize}

\begin{quote}
The puzzle, though inanimate, is presented as a solvable problem without lasting negative consequences, a very low-risk low-reward situation. By being a puzzle, the object is attempting to convince the user that it must be completed.
\end{quote}
{\tiny Source: \href{https://www.everydayrhetoric.com/2020/03/the-rhetoric-of-puzzles/}{Every Day Rhetoric}}
\end{frame}


\mode<all>{
\part{Basic Constraint Propagation}
}
\begin{frame}
\partpage
\end{frame}
\mode<all>{
\input{../sendmore/body}
}

\mode<all>{
\part{Global Constraints}
}
\begin{frame}
\partpage
\end{frame}
\mode<all>{
\input{../sudoku/body}
}

\clearpage
\mode<all>{
\part{Customizing Search}
}
\begin{frame}
\partpage
\end{frame}
\mode<all>{
\input{../nqueen/body}
}


%% \clearpage
%% \mode<all>{
%%   \part{Problem Decomposition}
%% }
%% \begin{frame}
%%   \partpage
%% \end{frame}
%% \mode<all>{
%%   \input{../party/body}
%%   }

%% \clearpage
%% \mode<all>{
%%   \part{More Global Constraints}
%% }
%% \begin{frame}
%%   \partpage
%% \end{frame}
%% \mode<all>{
%%   \input{../car/body}
%%   }

%% \clearpage
%% \mode<all>{
%%   \part{Methodology}
%% }
%% \begin{frame}
%%   \partpage
%% \end{frame}
%% \mode<all>{
%%   \input{../methodology/body}
%%   }

\clearpage
\mode<all>{
\part{What is missing?}
}
\begin{frame}
\partpage
\end{frame}

\begin{frame}
\frametitle{Many Specialized Topics}
\begin{itemize}
\item How to design efficient core engine
\item Hybrids with LP/MIP tools
\item Hybrids with SAT
\item Symmetry breaking
\item Use of MDD/BDD to encode sets of solutions
\item High level modelling tools
\item Debugging/visualization
\end{itemize}
\end{frame}

\begin{frame}
\frametitle{Reformulation}
\begin{itemize}
\item Just because the user has modelled it this way, it doesn't mean we have to solve it that way
\begin{itemize}
\item Replace some constraint(s) by other, equivalent constraints
\item Because we don't have that constraint in our system
\item For performance
\end{itemize}
\end{itemize}
\end{frame}

\begin{frame}
\frametitle{Learning}
\begin{itemize}
\item While solving the problem we can learn how to strengthen the model/search 
\begin{itemize}
\item Understand which constraints/method contribute to propagation and change schedule
\item Learn no-good constraints by explaining failure
\item Adapt search strategy based on search experience
\end{itemize}
\end{itemize}
\end{frame}

%% \begin{frame}
%% \frametitle{Refined Process}
%% \begin{center}
%% \begin{tikzpicture}
%% \node[shape=rectangle,fill=pantone127-4] (problem) at (3,7) {Problem};
%% \node (human) at (3,6) {Human};
%% \node[shape=rectangle,fill=pantone127-4] (model) at (3,5) {User Model};
%% \node (reformulation) at (3,4) {Reformulation};
%% \node[shape=rectangle,fill=pantone127-4] (implementation) at (3,3) {Implementation Model};
%% \node (solver) at (3,2) {Constraint Solver/Search};
%% \node[shape=rectangle,fill=pantone157-8] (solution) at (3,1) {Solution};
%% \draw[-] (problem) -- (human);
%% \draw[->] (human) -- (model);
%% \draw[-] (model) -- (reformulation);
%% \draw[->] (reformulation) -- (implementation);
%% \draw[-] (implementation) -- (solver);
%% \draw[->] (solver) -- (0,2) -- (0,3) -- (implementation);
%% \draw[->] (solver) -- (solution);
%% \end{tikzpicture}
%% \end{center}
%% \end{frame}

%\begin{frame}
%\frametitle{What is CP actually used for?}
%\url{http://hsimonis.wordpress.com}
%\includegraphics[width=10cm]{../introduction/images/blog}
%\end{frame}

%% \begin{frame}
%% \frametitle{Exercises}
%% \begin{itemize}
%% \item Install Constraint Solver of your Choice
%% \item MiniZinc (MiningZinc) will be used in later sessions
%% \item Run Sendmoremoney, Sudoku, Queens examples
%% \item Can you choose which propagator to use for alldifferent?
%% \item Can you use different search strategies?
%% \end{itemize}
%% \end{frame}

\begin{frame}
\frametitle{More Learning Resources}
\begin{itemize}
\item Survey of Methods, Resources, and Formats for Teaching Constraint Programming
\begin{itemize}
\item by Tejas Santanam, Helmut Simonis
\item \url{https://doi.org/10.48550/arXiv.2403.12717}
\item Based on survey of community for WTCP 2023
\item \url{https://hsimonis.github.io/WTCP2023/}
\end{itemize}
\end{itemize}
\end{frame}


}


%% \clearpage
%% \mode<all>{
%%   \part{Problem Decomposition}
%% }
%% \begin{frame}
%%   \partpage
%% \end{frame}
%% \mode<all>{
%%   \begin{frame}
\frametitle{Objectives}
\begin{itemize}
\item Overview of Core Constraint Programming
\item Three Main Concepts
\begin{itemize}
\item Constraint Propagation
\item Global Constraints
\item Customizing Search
\end{itemize}
%\item Get Some Experience with MiniZinc
\item Topics will be treated in more detail in later parts of the school
\item Based on Examples, not Formal Description
\end{itemize}
\end{frame}

\begin{frame}
\frametitle{Outline}
\begin{itemize}
\item Why Constraint Programming?
\item Constraint Propagation
\item Global Constraints
\item Customizing Search
%\item Choices in Modelling
%\item Decomposition
%\item More Global Constraints
%\item Methodology
%\item What is missing?
\end{itemize}
\end{frame}

% \begin{frame}
  % \frametitle{Using MiniZinc IDE}
  % \begin{itemize}
  % \item Developed in the Australian NICTA project
  % \item Maintained by Monash University
  % \item Modelling tool with multiple back-end solvers
  % \item Available from \url{https://www.minizinc.org/}
  % \end{itemize}
% \end{frame}


% \begin{frame}
% \frametitle{Examples in ECLiPSe}
% \begin{itemize}
% \item Open sourced constraint programming language
% \item Development goes back to 1985
% \item ECRC, ICL, IC-Parc, PTL, Cisco
% \item \url{https://eclipseclp.org/}
% \item Specialities
% \begin{itemize}
% \item Develop new solvers for specific domains
% \item Integration with MIP
% \end{itemize}
% \item Not included in bundled MiniZinc IDE
% \item Specialized visualization tools used here
  % \begin{itemize}
  % \item CP-Viz, Simonis et al. 2010
    % \end{itemize}
% \end{itemize}
% \end{frame}

\begin{frame}
\frametitle{Tutorial Based on ECLiPSe ELearning Course}
\begin{itemize}
\item Self-study course in constraint programming
\item Supported by Cisco Systems and Silicon Valley Community Foundation
\item Multi-media format, video lectures, slides, handout etc
\item {\small \url{https://eclipseclp.org/ELearning/index.html}}
\end{itemize}
\end{frame}

\begin{frame}
\frametitle{Also Part of CRT-AI Constraint Week}
\begin{itemize}
\item Annual one week course on CP and Optimization in Ireland
\item Part of national training program for PhD students in AI
\item \url{https://www.crt-ai.ie/}
\end{itemize}
\end{frame}



\begin{frame}
\frametitle{Constraint Programming - in a nutshell}
\begin{itemize}
\item Declarative description of problems with
\begin{itemize}
\item {\em Variables} which range over (finite) sets of values
\item {\em Constraints} over subsets of variables which restrict possible value combinations
\item A {\em solution} is a value assignment which satisfies all constraints
\end{itemize}

\item Constraint propagation/reasoning
\begin{itemize}
\item Removing inconsistent values for variables
\item Detect failure if constraint can not be satisfied
\item Interaction of constraints via shared variables
\item Incomplete
\end{itemize}

\item Search
\begin{itemize}
\item User controlled assignment of values to variables
\item Each step triggers constraint propagation 
\end{itemize}
\item Different domains require/allow different methods
\end{itemize}
\end{frame}

\begin{frame}
  \frametitle{Constraint Programming is Different}
  \begin{itemize}
  \item Declarative Programming
    \begin{itemize}
    \item Concentrate on what you want
      \item Not how to get there
      \item Program != Algorithm
      \item Program = Model
    \end{itemize}
    \item Applied to Combinatorial Problems
      \begin{itemize}
        \item No complete polynomial algorithms known (exist?)
        \item CP less ad-hoc than heuristics
        \item Models can evolve
  \end{itemize}
  \end{itemize}
  \end{frame}
    
\begin{frame}
  \frametitle{A Subtractive Process}
  \begin{textblock}{4}(8,-3)
    \includegraphics[width=4cm]{../introduction/images/stages}
  \end{textblock}
  \vfill
  \begin{quote}
    ``Oh, bosh, as Mr. Ruskin says. Sculpture, per se, is the simplest thing in the world. All you have to do is to take a big chunk of marble and a hammer and chisel, make up your mind what you are about to create and chip off all the marble you don't want.''-Paris Gaulois.
  \end{quote}
  
  {\tiny Source: \url{https://quoteinvestigator.com/2014/06/22/chip-away/}}
\end{frame}

\begin{frame}
\frametitle{Basic Process}
\begin{center}
\begin{tikzpicture}
\node[shape=rectangle,fill=pantone127-4] (problem) at (2,5) {Problem};
\node (human) at (2,4) {Human};
\node[shape=rectangle,fill=pantone127-4] (model) at (2,3) {Model};
\node (solver) at (2,2) {Constraint Solver/Search};
\node[shape=rectangle,fill=pantone157-8] (solution) at (2,1) {Solution};
\draw[-] (problem) -- (human);
\draw[->] (human) -- (model);
\draw[-] (model) -- (solver);
\draw[->] (solver) -- (solution);
\end{tikzpicture}
\end{center}
\end{frame}

\begin{frame}
\frametitle{More Realistic}
\begin{center}
\begin{tikzpicture}[xscale=4.0]
\node[shape=rectangle,fill=pantone127-4] (problem) at (2,5) {Problem};
\node (human) at (2,4) {Human};
\node[shape=rectangle,fill=pantone127-4] (model) at (2,3) {Model};
\node (solver) at (2,2) {Constraint Solver/Search};
\node[shape=rectangle,fill=pantone24-8] (hangs) at (1,1) {Hangs};
\node[shape=rectangle,fill=pantone157-8] (solution) at (2,1) {Solution};
\node[shape=rectangle,fill=pantone24-8] (wrong) at (3,1) {Wrong Solution};
\draw[-] (problem) -- (human);
\draw[->] (human) -- (model);
\draw[-] (model) -- (solver);
\draw[->] (solver) -- (hangs);
\draw[->] (solver) -- (solution);
\draw[->] (solver) -- (wrong);
\draw[->] (hangs) -- (1,4) -- (human);
\draw[->] (wrong) -- (3,4) -- (human);
\end{tikzpicture}
\end{center}
\end{frame}

\begin{frame}
\frametitle{Dual Role of Model}
\begin{itemize}
\item Allows Human to Express Problem
\begin{itemize}
\item Close to Problem Domain
\item Constraints as Abstractions
\end{itemize}
\item Allows Solver to Execute
\begin{itemize}
\item Variables as Communication Mechanism
\item Constraints as Algorithms
\end{itemize}
\end{itemize}
\end{frame}

\begin{frame}
\frametitle{Modelling Frameworks}
\begin{itemize}
\item \href{https://www.minizinc.org/}{MiniZinc} (NICTA, Monash University, Australia)
\item \href{https://github.com/eomahony/Numberjack}{NumberJack} (Insight, Ireland) 
\item \href{https://www-users.york.ac.uk/peter.nightingale/savilerow//}{EssencePrime/SavilleRow} (UK) 
\item \href{https://cpmpy.readthedocs.io/en/latest/}{CPMpy} (KU Leuven)
\item Allow use of multiple back-end solvers
\item Compile model into variants for each solver
\item A priori solver independent model(CP, MIP, SAT)
\end{itemize}
\end{frame} 

\begin{frame}
\frametitle{Framework Process}
\begin{center}
\begin{tikzpicture}[xscale=2]
\node[shape=rectangle,fill=pantone127-4] (problem) at (2.5,5) {Problem};
\node (human) at (2.5,4) {Human};
\node[shape=rectangle,fill=pantone127-4] (model) at (2.5,3) {Model};
\node (compile) at (2.5,2) {Compile/Reformulate};
\node (solver1) at (1,1) {CP};
\node (solver2) at (2,1) {MIP};
\node (solver3) at (3,1) {SAT};
\node (solver4) at (4,1) {Other};
\node[shape=rectangle,fill=pantone157-8] (solution1) at (1,0) {Solution};
\node[shape=rectangle,fill=pantone157-8] (solution2) at (2,0) {Solution};
\node[shape=rectangle,fill=pantone157-8] (solution3) at (3,0) {Solution};
\node[shape=rectangle,fill=pantone157-8] (solution4) at (4,0) {Solution};
\draw[-] (problem) -- (human);
\draw[->] (human) -- (model);
\draw[-] (model) -- (compile);
\draw[-] (compile) -- (solver1);
\draw[-] (compile) -- (solver2);
\draw[-] (compile) -- (solver3);
\draw[-] (compile) -- (solver4);
\draw[->] (solver1) -- (solution1);
\draw[->] (solver2) -- (solution2);
\draw[->] (solver3) -- (solution3);
\draw[->] (solver4) -- (solution4);
\end{tikzpicture}
\end{center}
\end{frame}

% \begin{frame}
  % \frametitle{Do It Now!}
  % \begin{itemize}
  % \item Download and install Minizinc
  % \item \url{https://www.minizinc.org/}
  % \end{itemize}
% \end{frame}


\begin{frame}
\frametitle{Why use Puzzles as Examples?}
\begin{itemize}
\item Easy to understand the problem
\item Solvable by hand without specialized knowledge
\item Possible to compare automated to manual solving process
\end{itemize}

\begin{quote}
The puzzle, though inanimate, is presented as a solvable problem without lasting negative consequences, a very low-risk low-reward situation. By being a puzzle, the object is attempting to convince the user that it must be completed.
\end{quote}
{\tiny Source: \href{https://www.everydayrhetoric.com/2020/03/the-rhetoric-of-puzzles/}{Every Day Rhetoric}}
\end{frame}


\mode<all>{
\part{Basic Constraint Propagation}
}
\begin{frame}
\partpage
\end{frame}
\mode<all>{
\input{../sendmore/body}
}

\mode<all>{
\part{Global Constraints}
}
\begin{frame}
\partpage
\end{frame}
\mode<all>{
\input{../sudoku/body}
}

\clearpage
\mode<all>{
\part{Customizing Search}
}
\begin{frame}
\partpage
\end{frame}
\mode<all>{
\input{../nqueen/body}
}


%% \clearpage
%% \mode<all>{
%%   \part{Problem Decomposition}
%% }
%% \begin{frame}
%%   \partpage
%% \end{frame}
%% \mode<all>{
%%   \input{../party/body}
%%   }

%% \clearpage
%% \mode<all>{
%%   \part{More Global Constraints}
%% }
%% \begin{frame}
%%   \partpage
%% \end{frame}
%% \mode<all>{
%%   \input{../car/body}
%%   }

%% \clearpage
%% \mode<all>{
%%   \part{Methodology}
%% }
%% \begin{frame}
%%   \partpage
%% \end{frame}
%% \mode<all>{
%%   \input{../methodology/body}
%%   }

\clearpage
\mode<all>{
\part{What is missing?}
}
\begin{frame}
\partpage
\end{frame}

\begin{frame}
\frametitle{Many Specialized Topics}
\begin{itemize}
\item How to design efficient core engine
\item Hybrids with LP/MIP tools
\item Hybrids with SAT
\item Symmetry breaking
\item Use of MDD/BDD to encode sets of solutions
\item High level modelling tools
\item Debugging/visualization
\end{itemize}
\end{frame}

\begin{frame}
\frametitle{Reformulation}
\begin{itemize}
\item Just because the user has modelled it this way, it doesn't mean we have to solve it that way
\begin{itemize}
\item Replace some constraint(s) by other, equivalent constraints
\item Because we don't have that constraint in our system
\item For performance
\end{itemize}
\end{itemize}
\end{frame}

\begin{frame}
\frametitle{Learning}
\begin{itemize}
\item While solving the problem we can learn how to strengthen the model/search 
\begin{itemize}
\item Understand which constraints/method contribute to propagation and change schedule
\item Learn no-good constraints by explaining failure
\item Adapt search strategy based on search experience
\end{itemize}
\end{itemize}
\end{frame}

%% \begin{frame}
%% \frametitle{Refined Process}
%% \begin{center}
%% \begin{tikzpicture}
%% \node[shape=rectangle,fill=pantone127-4] (problem) at (3,7) {Problem};
%% \node (human) at (3,6) {Human};
%% \node[shape=rectangle,fill=pantone127-4] (model) at (3,5) {User Model};
%% \node (reformulation) at (3,4) {Reformulation};
%% \node[shape=rectangle,fill=pantone127-4] (implementation) at (3,3) {Implementation Model};
%% \node (solver) at (3,2) {Constraint Solver/Search};
%% \node[shape=rectangle,fill=pantone157-8] (solution) at (3,1) {Solution};
%% \draw[-] (problem) -- (human);
%% \draw[->] (human) -- (model);
%% \draw[-] (model) -- (reformulation);
%% \draw[->] (reformulation) -- (implementation);
%% \draw[-] (implementation) -- (solver);
%% \draw[->] (solver) -- (0,2) -- (0,3) -- (implementation);
%% \draw[->] (solver) -- (solution);
%% \end{tikzpicture}
%% \end{center}
%% \end{frame}

%\begin{frame}
%\frametitle{What is CP actually used for?}
%\url{http://hsimonis.wordpress.com}
%\includegraphics[width=10cm]{../introduction/images/blog}
%\end{frame}

%% \begin{frame}
%% \frametitle{Exercises}
%% \begin{itemize}
%% \item Install Constraint Solver of your Choice
%% \item MiniZinc (MiningZinc) will be used in later sessions
%% \item Run Sendmoremoney, Sudoku, Queens examples
%% \item Can you choose which propagator to use for alldifferent?
%% \item Can you use different search strategies?
%% \end{itemize}
%% \end{frame}

\begin{frame}
\frametitle{More Learning Resources}
\begin{itemize}
\item Survey of Methods, Resources, and Formats for Teaching Constraint Programming
\begin{itemize}
\item by Tejas Santanam, Helmut Simonis
\item \url{https://doi.org/10.48550/arXiv.2403.12717}
\item Based on survey of community for WTCP 2023
\item \url{https://hsimonis.github.io/WTCP2023/}
\end{itemize}
\end{itemize}
\end{frame}


%%   }

%% \clearpage
%% \mode<all>{
%%   \part{More Global Constraints}
%% }
%% \begin{frame}
%%   \partpage
%% \end{frame}
%% \mode<all>{
%%   \begin{frame}
\frametitle{Objectives}
\begin{itemize}
\item Overview of Core Constraint Programming
\item Three Main Concepts
\begin{itemize}
\item Constraint Propagation
\item Global Constraints
\item Customizing Search
\end{itemize}
%\item Get Some Experience with MiniZinc
\item Topics will be treated in more detail in later parts of the school
\item Based on Examples, not Formal Description
\end{itemize}
\end{frame}

\begin{frame}
\frametitle{Outline}
\begin{itemize}
\item Why Constraint Programming?
\item Constraint Propagation
\item Global Constraints
\item Customizing Search
%\item Choices in Modelling
%\item Decomposition
%\item More Global Constraints
%\item Methodology
%\item What is missing?
\end{itemize}
\end{frame}

% \begin{frame}
  % \frametitle{Using MiniZinc IDE}
  % \begin{itemize}
  % \item Developed in the Australian NICTA project
  % \item Maintained by Monash University
  % \item Modelling tool with multiple back-end solvers
  % \item Available from \url{https://www.minizinc.org/}
  % \end{itemize}
% \end{frame}


% \begin{frame}
% \frametitle{Examples in ECLiPSe}
% \begin{itemize}
% \item Open sourced constraint programming language
% \item Development goes back to 1985
% \item ECRC, ICL, IC-Parc, PTL, Cisco
% \item \url{https://eclipseclp.org/}
% \item Specialities
% \begin{itemize}
% \item Develop new solvers for specific domains
% \item Integration with MIP
% \end{itemize}
% \item Not included in bundled MiniZinc IDE
% \item Specialized visualization tools used here
  % \begin{itemize}
  % \item CP-Viz, Simonis et al. 2010
    % \end{itemize}
% \end{itemize}
% \end{frame}

\begin{frame}
\frametitle{Tutorial Based on ECLiPSe ELearning Course}
\begin{itemize}
\item Self-study course in constraint programming
\item Supported by Cisco Systems and Silicon Valley Community Foundation
\item Multi-media format, video lectures, slides, handout etc
\item {\small \url{https://eclipseclp.org/ELearning/index.html}}
\end{itemize}
\end{frame}

\begin{frame}
\frametitle{Also Part of CRT-AI Constraint Week}
\begin{itemize}
\item Annual one week course on CP and Optimization in Ireland
\item Part of national training program for PhD students in AI
\item \url{https://www.crt-ai.ie/}
\end{itemize}
\end{frame}



\begin{frame}
\frametitle{Constraint Programming - in a nutshell}
\begin{itemize}
\item Declarative description of problems with
\begin{itemize}
\item {\em Variables} which range over (finite) sets of values
\item {\em Constraints} over subsets of variables which restrict possible value combinations
\item A {\em solution} is a value assignment which satisfies all constraints
\end{itemize}

\item Constraint propagation/reasoning
\begin{itemize}
\item Removing inconsistent values for variables
\item Detect failure if constraint can not be satisfied
\item Interaction of constraints via shared variables
\item Incomplete
\end{itemize}

\item Search
\begin{itemize}
\item User controlled assignment of values to variables
\item Each step triggers constraint propagation 
\end{itemize}
\item Different domains require/allow different methods
\end{itemize}
\end{frame}

\begin{frame}
  \frametitle{Constraint Programming is Different}
  \begin{itemize}
  \item Declarative Programming
    \begin{itemize}
    \item Concentrate on what you want
      \item Not how to get there
      \item Program != Algorithm
      \item Program = Model
    \end{itemize}
    \item Applied to Combinatorial Problems
      \begin{itemize}
        \item No complete polynomial algorithms known (exist?)
        \item CP less ad-hoc than heuristics
        \item Models can evolve
  \end{itemize}
  \end{itemize}
  \end{frame}
    
\begin{frame}
  \frametitle{A Subtractive Process}
  \begin{textblock}{4}(8,-3)
    \includegraphics[width=4cm]{../introduction/images/stages}
  \end{textblock}
  \vfill
  \begin{quote}
    ``Oh, bosh, as Mr. Ruskin says. Sculpture, per se, is the simplest thing in the world. All you have to do is to take a big chunk of marble and a hammer and chisel, make up your mind what you are about to create and chip off all the marble you don't want.''-Paris Gaulois.
  \end{quote}
  
  {\tiny Source: \url{https://quoteinvestigator.com/2014/06/22/chip-away/}}
\end{frame}

\begin{frame}
\frametitle{Basic Process}
\begin{center}
\begin{tikzpicture}
\node[shape=rectangle,fill=pantone127-4] (problem) at (2,5) {Problem};
\node (human) at (2,4) {Human};
\node[shape=rectangle,fill=pantone127-4] (model) at (2,3) {Model};
\node (solver) at (2,2) {Constraint Solver/Search};
\node[shape=rectangle,fill=pantone157-8] (solution) at (2,1) {Solution};
\draw[-] (problem) -- (human);
\draw[->] (human) -- (model);
\draw[-] (model) -- (solver);
\draw[->] (solver) -- (solution);
\end{tikzpicture}
\end{center}
\end{frame}

\begin{frame}
\frametitle{More Realistic}
\begin{center}
\begin{tikzpicture}[xscale=4.0]
\node[shape=rectangle,fill=pantone127-4] (problem) at (2,5) {Problem};
\node (human) at (2,4) {Human};
\node[shape=rectangle,fill=pantone127-4] (model) at (2,3) {Model};
\node (solver) at (2,2) {Constraint Solver/Search};
\node[shape=rectangle,fill=pantone24-8] (hangs) at (1,1) {Hangs};
\node[shape=rectangle,fill=pantone157-8] (solution) at (2,1) {Solution};
\node[shape=rectangle,fill=pantone24-8] (wrong) at (3,1) {Wrong Solution};
\draw[-] (problem) -- (human);
\draw[->] (human) -- (model);
\draw[-] (model) -- (solver);
\draw[->] (solver) -- (hangs);
\draw[->] (solver) -- (solution);
\draw[->] (solver) -- (wrong);
\draw[->] (hangs) -- (1,4) -- (human);
\draw[->] (wrong) -- (3,4) -- (human);
\end{tikzpicture}
\end{center}
\end{frame}

\begin{frame}
\frametitle{Dual Role of Model}
\begin{itemize}
\item Allows Human to Express Problem
\begin{itemize}
\item Close to Problem Domain
\item Constraints as Abstractions
\end{itemize}
\item Allows Solver to Execute
\begin{itemize}
\item Variables as Communication Mechanism
\item Constraints as Algorithms
\end{itemize}
\end{itemize}
\end{frame}

\begin{frame}
\frametitle{Modelling Frameworks}
\begin{itemize}
\item \href{https://www.minizinc.org/}{MiniZinc} (NICTA, Monash University, Australia)
\item \href{https://github.com/eomahony/Numberjack}{NumberJack} (Insight, Ireland) 
\item \href{https://www-users.york.ac.uk/peter.nightingale/savilerow//}{EssencePrime/SavilleRow} (UK) 
\item \href{https://cpmpy.readthedocs.io/en/latest/}{CPMpy} (KU Leuven)
\item Allow use of multiple back-end solvers
\item Compile model into variants for each solver
\item A priori solver independent model(CP, MIP, SAT)
\end{itemize}
\end{frame} 

\begin{frame}
\frametitle{Framework Process}
\begin{center}
\begin{tikzpicture}[xscale=2]
\node[shape=rectangle,fill=pantone127-4] (problem) at (2.5,5) {Problem};
\node (human) at (2.5,4) {Human};
\node[shape=rectangle,fill=pantone127-4] (model) at (2.5,3) {Model};
\node (compile) at (2.5,2) {Compile/Reformulate};
\node (solver1) at (1,1) {CP};
\node (solver2) at (2,1) {MIP};
\node (solver3) at (3,1) {SAT};
\node (solver4) at (4,1) {Other};
\node[shape=rectangle,fill=pantone157-8] (solution1) at (1,0) {Solution};
\node[shape=rectangle,fill=pantone157-8] (solution2) at (2,0) {Solution};
\node[shape=rectangle,fill=pantone157-8] (solution3) at (3,0) {Solution};
\node[shape=rectangle,fill=pantone157-8] (solution4) at (4,0) {Solution};
\draw[-] (problem) -- (human);
\draw[->] (human) -- (model);
\draw[-] (model) -- (compile);
\draw[-] (compile) -- (solver1);
\draw[-] (compile) -- (solver2);
\draw[-] (compile) -- (solver3);
\draw[-] (compile) -- (solver4);
\draw[->] (solver1) -- (solution1);
\draw[->] (solver2) -- (solution2);
\draw[->] (solver3) -- (solution3);
\draw[->] (solver4) -- (solution4);
\end{tikzpicture}
\end{center}
\end{frame}

% \begin{frame}
  % \frametitle{Do It Now!}
  % \begin{itemize}
  % \item Download and install Minizinc
  % \item \url{https://www.minizinc.org/}
  % \end{itemize}
% \end{frame}


\begin{frame}
\frametitle{Why use Puzzles as Examples?}
\begin{itemize}
\item Easy to understand the problem
\item Solvable by hand without specialized knowledge
\item Possible to compare automated to manual solving process
\end{itemize}

\begin{quote}
The puzzle, though inanimate, is presented as a solvable problem without lasting negative consequences, a very low-risk low-reward situation. By being a puzzle, the object is attempting to convince the user that it must be completed.
\end{quote}
{\tiny Source: \href{https://www.everydayrhetoric.com/2020/03/the-rhetoric-of-puzzles/}{Every Day Rhetoric}}
\end{frame}


\mode<all>{
\part{Basic Constraint Propagation}
}
\begin{frame}
\partpage
\end{frame}
\mode<all>{
\input{../sendmore/body}
}

\mode<all>{
\part{Global Constraints}
}
\begin{frame}
\partpage
\end{frame}
\mode<all>{
\input{../sudoku/body}
}

\clearpage
\mode<all>{
\part{Customizing Search}
}
\begin{frame}
\partpage
\end{frame}
\mode<all>{
\input{../nqueen/body}
}


%% \clearpage
%% \mode<all>{
%%   \part{Problem Decomposition}
%% }
%% \begin{frame}
%%   \partpage
%% \end{frame}
%% \mode<all>{
%%   \input{../party/body}
%%   }

%% \clearpage
%% \mode<all>{
%%   \part{More Global Constraints}
%% }
%% \begin{frame}
%%   \partpage
%% \end{frame}
%% \mode<all>{
%%   \input{../car/body}
%%   }

%% \clearpage
%% \mode<all>{
%%   \part{Methodology}
%% }
%% \begin{frame}
%%   \partpage
%% \end{frame}
%% \mode<all>{
%%   \input{../methodology/body}
%%   }

\clearpage
\mode<all>{
\part{What is missing?}
}
\begin{frame}
\partpage
\end{frame}

\begin{frame}
\frametitle{Many Specialized Topics}
\begin{itemize}
\item How to design efficient core engine
\item Hybrids with LP/MIP tools
\item Hybrids with SAT
\item Symmetry breaking
\item Use of MDD/BDD to encode sets of solutions
\item High level modelling tools
\item Debugging/visualization
\end{itemize}
\end{frame}

\begin{frame}
\frametitle{Reformulation}
\begin{itemize}
\item Just because the user has modelled it this way, it doesn't mean we have to solve it that way
\begin{itemize}
\item Replace some constraint(s) by other, equivalent constraints
\item Because we don't have that constraint in our system
\item For performance
\end{itemize}
\end{itemize}
\end{frame}

\begin{frame}
\frametitle{Learning}
\begin{itemize}
\item While solving the problem we can learn how to strengthen the model/search 
\begin{itemize}
\item Understand which constraints/method contribute to propagation and change schedule
\item Learn no-good constraints by explaining failure
\item Adapt search strategy based on search experience
\end{itemize}
\end{itemize}
\end{frame}

%% \begin{frame}
%% \frametitle{Refined Process}
%% \begin{center}
%% \begin{tikzpicture}
%% \node[shape=rectangle,fill=pantone127-4] (problem) at (3,7) {Problem};
%% \node (human) at (3,6) {Human};
%% \node[shape=rectangle,fill=pantone127-4] (model) at (3,5) {User Model};
%% \node (reformulation) at (3,4) {Reformulation};
%% \node[shape=rectangle,fill=pantone127-4] (implementation) at (3,3) {Implementation Model};
%% \node (solver) at (3,2) {Constraint Solver/Search};
%% \node[shape=rectangle,fill=pantone157-8] (solution) at (3,1) {Solution};
%% \draw[-] (problem) -- (human);
%% \draw[->] (human) -- (model);
%% \draw[-] (model) -- (reformulation);
%% \draw[->] (reformulation) -- (implementation);
%% \draw[-] (implementation) -- (solver);
%% \draw[->] (solver) -- (0,2) -- (0,3) -- (implementation);
%% \draw[->] (solver) -- (solution);
%% \end{tikzpicture}
%% \end{center}
%% \end{frame}

%\begin{frame}
%\frametitle{What is CP actually used for?}
%\url{http://hsimonis.wordpress.com}
%\includegraphics[width=10cm]{../introduction/images/blog}
%\end{frame}

%% \begin{frame}
%% \frametitle{Exercises}
%% \begin{itemize}
%% \item Install Constraint Solver of your Choice
%% \item MiniZinc (MiningZinc) will be used in later sessions
%% \item Run Sendmoremoney, Sudoku, Queens examples
%% \item Can you choose which propagator to use for alldifferent?
%% \item Can you use different search strategies?
%% \end{itemize}
%% \end{frame}

\begin{frame}
\frametitle{More Learning Resources}
\begin{itemize}
\item Survey of Methods, Resources, and Formats for Teaching Constraint Programming
\begin{itemize}
\item by Tejas Santanam, Helmut Simonis
\item \url{https://doi.org/10.48550/arXiv.2403.12717}
\item Based on survey of community for WTCP 2023
\item \url{https://hsimonis.github.io/WTCP2023/}
\end{itemize}
\end{itemize}
\end{frame}


%%   }

%% \clearpage
%% \mode<all>{
%%   \part{Methodology}
%% }
%% \begin{frame}
%%   \partpage
%% \end{frame}
%% \mode<all>{
%%   \begin{frame}
\frametitle{Objectives}
\begin{itemize}
\item Overview of Core Constraint Programming
\item Three Main Concepts
\begin{itemize}
\item Constraint Propagation
\item Global Constraints
\item Customizing Search
\end{itemize}
%\item Get Some Experience with MiniZinc
\item Topics will be treated in more detail in later parts of the school
\item Based on Examples, not Formal Description
\end{itemize}
\end{frame}

\begin{frame}
\frametitle{Outline}
\begin{itemize}
\item Why Constraint Programming?
\item Constraint Propagation
\item Global Constraints
\item Customizing Search
%\item Choices in Modelling
%\item Decomposition
%\item More Global Constraints
%\item Methodology
%\item What is missing?
\end{itemize}
\end{frame}

% \begin{frame}
  % \frametitle{Using MiniZinc IDE}
  % \begin{itemize}
  % \item Developed in the Australian NICTA project
  % \item Maintained by Monash University
  % \item Modelling tool with multiple back-end solvers
  % \item Available from \url{https://www.minizinc.org/}
  % \end{itemize}
% \end{frame}


% \begin{frame}
% \frametitle{Examples in ECLiPSe}
% \begin{itemize}
% \item Open sourced constraint programming language
% \item Development goes back to 1985
% \item ECRC, ICL, IC-Parc, PTL, Cisco
% \item \url{https://eclipseclp.org/}
% \item Specialities
% \begin{itemize}
% \item Develop new solvers for specific domains
% \item Integration with MIP
% \end{itemize}
% \item Not included in bundled MiniZinc IDE
% \item Specialized visualization tools used here
  % \begin{itemize}
  % \item CP-Viz, Simonis et al. 2010
    % \end{itemize}
% \end{itemize}
% \end{frame}

\begin{frame}
\frametitle{Tutorial Based on ECLiPSe ELearning Course}
\begin{itemize}
\item Self-study course in constraint programming
\item Supported by Cisco Systems and Silicon Valley Community Foundation
\item Multi-media format, video lectures, slides, handout etc
\item {\small \url{https://eclipseclp.org/ELearning/index.html}}
\end{itemize}
\end{frame}

\begin{frame}
\frametitle{Also Part of CRT-AI Constraint Week}
\begin{itemize}
\item Annual one week course on CP and Optimization in Ireland
\item Part of national training program for PhD students in AI
\item \url{https://www.crt-ai.ie/}
\end{itemize}
\end{frame}



\begin{frame}
\frametitle{Constraint Programming - in a nutshell}
\begin{itemize}
\item Declarative description of problems with
\begin{itemize}
\item {\em Variables} which range over (finite) sets of values
\item {\em Constraints} over subsets of variables which restrict possible value combinations
\item A {\em solution} is a value assignment which satisfies all constraints
\end{itemize}

\item Constraint propagation/reasoning
\begin{itemize}
\item Removing inconsistent values for variables
\item Detect failure if constraint can not be satisfied
\item Interaction of constraints via shared variables
\item Incomplete
\end{itemize}

\item Search
\begin{itemize}
\item User controlled assignment of values to variables
\item Each step triggers constraint propagation 
\end{itemize}
\item Different domains require/allow different methods
\end{itemize}
\end{frame}

\begin{frame}
  \frametitle{Constraint Programming is Different}
  \begin{itemize}
  \item Declarative Programming
    \begin{itemize}
    \item Concentrate on what you want
      \item Not how to get there
      \item Program != Algorithm
      \item Program = Model
    \end{itemize}
    \item Applied to Combinatorial Problems
      \begin{itemize}
        \item No complete polynomial algorithms known (exist?)
        \item CP less ad-hoc than heuristics
        \item Models can evolve
  \end{itemize}
  \end{itemize}
  \end{frame}
    
\begin{frame}
  \frametitle{A Subtractive Process}
  \begin{textblock}{4}(8,-3)
    \includegraphics[width=4cm]{../introduction/images/stages}
  \end{textblock}
  \vfill
  \begin{quote}
    ``Oh, bosh, as Mr. Ruskin says. Sculpture, per se, is the simplest thing in the world. All you have to do is to take a big chunk of marble and a hammer and chisel, make up your mind what you are about to create and chip off all the marble you don't want.''-Paris Gaulois.
  \end{quote}
  
  {\tiny Source: \url{https://quoteinvestigator.com/2014/06/22/chip-away/}}
\end{frame}

\begin{frame}
\frametitle{Basic Process}
\begin{center}
\begin{tikzpicture}
\node[shape=rectangle,fill=pantone127-4] (problem) at (2,5) {Problem};
\node (human) at (2,4) {Human};
\node[shape=rectangle,fill=pantone127-4] (model) at (2,3) {Model};
\node (solver) at (2,2) {Constraint Solver/Search};
\node[shape=rectangle,fill=pantone157-8] (solution) at (2,1) {Solution};
\draw[-] (problem) -- (human);
\draw[->] (human) -- (model);
\draw[-] (model) -- (solver);
\draw[->] (solver) -- (solution);
\end{tikzpicture}
\end{center}
\end{frame}

\begin{frame}
\frametitle{More Realistic}
\begin{center}
\begin{tikzpicture}[xscale=4.0]
\node[shape=rectangle,fill=pantone127-4] (problem) at (2,5) {Problem};
\node (human) at (2,4) {Human};
\node[shape=rectangle,fill=pantone127-4] (model) at (2,3) {Model};
\node (solver) at (2,2) {Constraint Solver/Search};
\node[shape=rectangle,fill=pantone24-8] (hangs) at (1,1) {Hangs};
\node[shape=rectangle,fill=pantone157-8] (solution) at (2,1) {Solution};
\node[shape=rectangle,fill=pantone24-8] (wrong) at (3,1) {Wrong Solution};
\draw[-] (problem) -- (human);
\draw[->] (human) -- (model);
\draw[-] (model) -- (solver);
\draw[->] (solver) -- (hangs);
\draw[->] (solver) -- (solution);
\draw[->] (solver) -- (wrong);
\draw[->] (hangs) -- (1,4) -- (human);
\draw[->] (wrong) -- (3,4) -- (human);
\end{tikzpicture}
\end{center}
\end{frame}

\begin{frame}
\frametitle{Dual Role of Model}
\begin{itemize}
\item Allows Human to Express Problem
\begin{itemize}
\item Close to Problem Domain
\item Constraints as Abstractions
\end{itemize}
\item Allows Solver to Execute
\begin{itemize}
\item Variables as Communication Mechanism
\item Constraints as Algorithms
\end{itemize}
\end{itemize}
\end{frame}

\begin{frame}
\frametitle{Modelling Frameworks}
\begin{itemize}
\item \href{https://www.minizinc.org/}{MiniZinc} (NICTA, Monash University, Australia)
\item \href{https://github.com/eomahony/Numberjack}{NumberJack} (Insight, Ireland) 
\item \href{https://www-users.york.ac.uk/peter.nightingale/savilerow//}{EssencePrime/SavilleRow} (UK) 
\item \href{https://cpmpy.readthedocs.io/en/latest/}{CPMpy} (KU Leuven)
\item Allow use of multiple back-end solvers
\item Compile model into variants for each solver
\item A priori solver independent model(CP, MIP, SAT)
\end{itemize}
\end{frame} 

\begin{frame}
\frametitle{Framework Process}
\begin{center}
\begin{tikzpicture}[xscale=2]
\node[shape=rectangle,fill=pantone127-4] (problem) at (2.5,5) {Problem};
\node (human) at (2.5,4) {Human};
\node[shape=rectangle,fill=pantone127-4] (model) at (2.5,3) {Model};
\node (compile) at (2.5,2) {Compile/Reformulate};
\node (solver1) at (1,1) {CP};
\node (solver2) at (2,1) {MIP};
\node (solver3) at (3,1) {SAT};
\node (solver4) at (4,1) {Other};
\node[shape=rectangle,fill=pantone157-8] (solution1) at (1,0) {Solution};
\node[shape=rectangle,fill=pantone157-8] (solution2) at (2,0) {Solution};
\node[shape=rectangle,fill=pantone157-8] (solution3) at (3,0) {Solution};
\node[shape=rectangle,fill=pantone157-8] (solution4) at (4,0) {Solution};
\draw[-] (problem) -- (human);
\draw[->] (human) -- (model);
\draw[-] (model) -- (compile);
\draw[-] (compile) -- (solver1);
\draw[-] (compile) -- (solver2);
\draw[-] (compile) -- (solver3);
\draw[-] (compile) -- (solver4);
\draw[->] (solver1) -- (solution1);
\draw[->] (solver2) -- (solution2);
\draw[->] (solver3) -- (solution3);
\draw[->] (solver4) -- (solution4);
\end{tikzpicture}
\end{center}
\end{frame}

% \begin{frame}
  % \frametitle{Do It Now!}
  % \begin{itemize}
  % \item Download and install Minizinc
  % \item \url{https://www.minizinc.org/}
  % \end{itemize}
% \end{frame}


\begin{frame}
\frametitle{Why use Puzzles as Examples?}
\begin{itemize}
\item Easy to understand the problem
\item Solvable by hand without specialized knowledge
\item Possible to compare automated to manual solving process
\end{itemize}

\begin{quote}
The puzzle, though inanimate, is presented as a solvable problem without lasting negative consequences, a very low-risk low-reward situation. By being a puzzle, the object is attempting to convince the user that it must be completed.
\end{quote}
{\tiny Source: \href{https://www.everydayrhetoric.com/2020/03/the-rhetoric-of-puzzles/}{Every Day Rhetoric}}
\end{frame}


\mode<all>{
\part{Basic Constraint Propagation}
}
\begin{frame}
\partpage
\end{frame}
\mode<all>{
\input{../sendmore/body}
}

\mode<all>{
\part{Global Constraints}
}
\begin{frame}
\partpage
\end{frame}
\mode<all>{
\input{../sudoku/body}
}

\clearpage
\mode<all>{
\part{Customizing Search}
}
\begin{frame}
\partpage
\end{frame}
\mode<all>{
\input{../nqueen/body}
}


%% \clearpage
%% \mode<all>{
%%   \part{Problem Decomposition}
%% }
%% \begin{frame}
%%   \partpage
%% \end{frame}
%% \mode<all>{
%%   \input{../party/body}
%%   }

%% \clearpage
%% \mode<all>{
%%   \part{More Global Constraints}
%% }
%% \begin{frame}
%%   \partpage
%% \end{frame}
%% \mode<all>{
%%   \input{../car/body}
%%   }

%% \clearpage
%% \mode<all>{
%%   \part{Methodology}
%% }
%% \begin{frame}
%%   \partpage
%% \end{frame}
%% \mode<all>{
%%   \input{../methodology/body}
%%   }

\clearpage
\mode<all>{
\part{What is missing?}
}
\begin{frame}
\partpage
\end{frame}

\begin{frame}
\frametitle{Many Specialized Topics}
\begin{itemize}
\item How to design efficient core engine
\item Hybrids with LP/MIP tools
\item Hybrids with SAT
\item Symmetry breaking
\item Use of MDD/BDD to encode sets of solutions
\item High level modelling tools
\item Debugging/visualization
\end{itemize}
\end{frame}

\begin{frame}
\frametitle{Reformulation}
\begin{itemize}
\item Just because the user has modelled it this way, it doesn't mean we have to solve it that way
\begin{itemize}
\item Replace some constraint(s) by other, equivalent constraints
\item Because we don't have that constraint in our system
\item For performance
\end{itemize}
\end{itemize}
\end{frame}

\begin{frame}
\frametitle{Learning}
\begin{itemize}
\item While solving the problem we can learn how to strengthen the model/search 
\begin{itemize}
\item Understand which constraints/method contribute to propagation and change schedule
\item Learn no-good constraints by explaining failure
\item Adapt search strategy based on search experience
\end{itemize}
\end{itemize}
\end{frame}

%% \begin{frame}
%% \frametitle{Refined Process}
%% \begin{center}
%% \begin{tikzpicture}
%% \node[shape=rectangle,fill=pantone127-4] (problem) at (3,7) {Problem};
%% \node (human) at (3,6) {Human};
%% \node[shape=rectangle,fill=pantone127-4] (model) at (3,5) {User Model};
%% \node (reformulation) at (3,4) {Reformulation};
%% \node[shape=rectangle,fill=pantone127-4] (implementation) at (3,3) {Implementation Model};
%% \node (solver) at (3,2) {Constraint Solver/Search};
%% \node[shape=rectangle,fill=pantone157-8] (solution) at (3,1) {Solution};
%% \draw[-] (problem) -- (human);
%% \draw[->] (human) -- (model);
%% \draw[-] (model) -- (reformulation);
%% \draw[->] (reformulation) -- (implementation);
%% \draw[-] (implementation) -- (solver);
%% \draw[->] (solver) -- (0,2) -- (0,3) -- (implementation);
%% \draw[->] (solver) -- (solution);
%% \end{tikzpicture}
%% \end{center}
%% \end{frame}

%\begin{frame}
%\frametitle{What is CP actually used for?}
%\url{http://hsimonis.wordpress.com}
%\includegraphics[width=10cm]{../introduction/images/blog}
%\end{frame}

%% \begin{frame}
%% \frametitle{Exercises}
%% \begin{itemize}
%% \item Install Constraint Solver of your Choice
%% \item MiniZinc (MiningZinc) will be used in later sessions
%% \item Run Sendmoremoney, Sudoku, Queens examples
%% \item Can you choose which propagator to use for alldifferent?
%% \item Can you use different search strategies?
%% \end{itemize}
%% \end{frame}

\begin{frame}
\frametitle{More Learning Resources}
\begin{itemize}
\item Survey of Methods, Resources, and Formats for Teaching Constraint Programming
\begin{itemize}
\item by Tejas Santanam, Helmut Simonis
\item \url{https://doi.org/10.48550/arXiv.2403.12717}
\item Based on survey of community for WTCP 2023
\item \url{https://hsimonis.github.io/WTCP2023/}
\end{itemize}
\end{itemize}
\end{frame}


%%   }

\clearpage
\mode<all>{
\part{What is missing?}
}
\begin{frame}
\partpage
\end{frame}

\begin{frame}
\frametitle{Many Specialized Topics}
\begin{itemize}
\item How to design efficient core engine
\item Hybrids with LP/MIP tools
\item Hybrids with SAT
\item Symmetry breaking
\item Use of MDD/BDD to encode sets of solutions
\item High level modelling tools
\item Debugging/visualization
\end{itemize}
\end{frame}

\begin{frame}
\frametitle{Reformulation}
\begin{itemize}
\item Just because the user has modelled it this way, it doesn't mean we have to solve it that way
\begin{itemize}
\item Replace some constraint(s) by other, equivalent constraints
\item Because we don't have that constraint in our system
\item For performance
\end{itemize}
\end{itemize}
\end{frame}

\begin{frame}
\frametitle{Learning}
\begin{itemize}
\item While solving the problem we can learn how to strengthen the model/search 
\begin{itemize}
\item Understand which constraints/method contribute to propagation and change schedule
\item Learn no-good constraints by explaining failure
\item Adapt search strategy based on search experience
\end{itemize}
\end{itemize}
\end{frame}

%% \begin{frame}
%% \frametitle{Refined Process}
%% \begin{center}
%% \begin{tikzpicture}
%% \node[shape=rectangle,fill=pantone127-4] (problem) at (3,7) {Problem};
%% \node (human) at (3,6) {Human};
%% \node[shape=rectangle,fill=pantone127-4] (model) at (3,5) {User Model};
%% \node (reformulation) at (3,4) {Reformulation};
%% \node[shape=rectangle,fill=pantone127-4] (implementation) at (3,3) {Implementation Model};
%% \node (solver) at (3,2) {Constraint Solver/Search};
%% \node[shape=rectangle,fill=pantone157-8] (solution) at (3,1) {Solution};
%% \draw[-] (problem) -- (human);
%% \draw[->] (human) -- (model);
%% \draw[-] (model) -- (reformulation);
%% \draw[->] (reformulation) -- (implementation);
%% \draw[-] (implementation) -- (solver);
%% \draw[->] (solver) -- (0,2) -- (0,3) -- (implementation);
%% \draw[->] (solver) -- (solution);
%% \end{tikzpicture}
%% \end{center}
%% \end{frame}

%\begin{frame}
%\frametitle{What is CP actually used for?}
%\url{http://hsimonis.wordpress.com}
%\includegraphics[width=10cm]{../introduction/images/blog}
%\end{frame}

%% \begin{frame}
%% \frametitle{Exercises}
%% \begin{itemize}
%% \item Install Constraint Solver of your Choice
%% \item MiniZinc (MiningZinc) will be used in later sessions
%% \item Run Sendmoremoney, Sudoku, Queens examples
%% \item Can you choose which propagator to use for alldifferent?
%% \item Can you use different search strategies?
%% \end{itemize}
%% \end{frame}

\begin{frame}
\frametitle{More Learning Resources}
\begin{itemize}
\item Survey of Methods, Resources, and Formats for Teaching Constraint Programming
\begin{itemize}
\item by Tejas Santanam, Helmut Simonis
\item \url{https://doi.org/10.48550/arXiv.2403.12717}
\item Based on survey of community for WTCP 2023
\item \url{https://hsimonis.github.io/WTCP2023/}
\end{itemize}
\end{itemize}
\end{frame}


%%   }

\clearpage
\mode<all>{
\part{What is missing?}
}
\begin{frame}
\partpage
\end{frame}

\begin{frame}
\frametitle{Many Specialized Topics}
\begin{itemize}
\item How to design efficient core engine
\item Hybrids with LP/MIP tools
\item Hybrids with SAT
\item Symmetry breaking
\item Use of MDD/BDD to encode sets of solutions
\item High level modelling tools
\item Debugging/visualization
\end{itemize}
\end{frame}

\begin{frame}
\frametitle{Reformulation}
\begin{itemize}
\item Just because the user has modelled it this way, it doesn't mean we have to solve it that way
\begin{itemize}
\item Replace some constraint(s) by other, equivalent constraints
\item Because we don't have that constraint in our system
\item For performance
\end{itemize}
\end{itemize}
\end{frame}

\begin{frame}
\frametitle{Learning}
\begin{itemize}
\item While solving the problem we can learn how to strengthen the model/search 
\begin{itemize}
\item Understand which constraints/method contribute to propagation and change schedule
\item Learn no-good constraints by explaining failure
\item Adapt search strategy based on search experience
\end{itemize}
\end{itemize}
\end{frame}

%% \begin{frame}
%% \frametitle{Refined Process}
%% \begin{center}
%% \begin{tikzpicture}
%% \node[shape=rectangle,fill=pantone127-4] (problem) at (3,7) {Problem};
%% \node (human) at (3,6) {Human};
%% \node[shape=rectangle,fill=pantone127-4] (model) at (3,5) {User Model};
%% \node (reformulation) at (3,4) {Reformulation};
%% \node[shape=rectangle,fill=pantone127-4] (implementation) at (3,3) {Implementation Model};
%% \node (solver) at (3,2) {Constraint Solver/Search};
%% \node[shape=rectangle,fill=pantone157-8] (solution) at (3,1) {Solution};
%% \draw[-] (problem) -- (human);
%% \draw[->] (human) -- (model);
%% \draw[-] (model) -- (reformulation);
%% \draw[->] (reformulation) -- (implementation);
%% \draw[-] (implementation) -- (solver);
%% \draw[->] (solver) -- (0,2) -- (0,3) -- (implementation);
%% \draw[->] (solver) -- (solution);
%% \end{tikzpicture}
%% \end{center}
%% \end{frame}

%\begin{frame}
%\frametitle{What is CP actually used for?}
%\url{http://hsimonis.wordpress.com}
%\includegraphics[width=10cm]{../introduction/images/blog}
%\end{frame}

%% \begin{frame}
%% \frametitle{Exercises}
%% \begin{itemize}
%% \item Install Constraint Solver of your Choice
%% \item MiniZinc (MiningZinc) will be used in later sessions
%% \item Run Sendmoremoney, Sudoku, Queens examples
%% \item Can you choose which propagator to use for alldifferent?
%% \item Can you use different search strategies?
%% \end{itemize}
%% \end{frame}

\begin{frame}
\frametitle{More Learning Resources}
\begin{itemize}
\item Survey of Methods, Resources, and Formats for Teaching Constraint Programming
\begin{itemize}
\item by Tejas Santanam, Helmut Simonis
\item \url{https://doi.org/10.48550/arXiv.2403.12717}
\item Based on survey of community for WTCP 2023
\item \url{https://hsimonis.github.io/WTCP2023/}
\end{itemize}
\end{itemize}
\end{frame}


}

\mode<all>{
\part{Global Constraints}
}
\begin{frame}
\partpage
\end{frame}
\mode<all>{
\begin{frame}
\frametitle{Objectives}
\begin{itemize}
\item Overview of Core Constraint Programming
\item Three Main Concepts
\begin{itemize}
\item Constraint Propagation
\item Global Constraints
\item Customizing Search
\end{itemize}
%\item Get Some Experience with MiniZinc
\item Topics will be treated in more detail in later parts of the school
\item Based on Examples, not Formal Description
\end{itemize}
\end{frame}

\begin{frame}
\frametitle{Outline}
\begin{itemize}
\item Why Constraint Programming?
\item Constraint Propagation
\item Global Constraints
\item Customizing Search
%\item Choices in Modelling
%\item Decomposition
%\item More Global Constraints
%\item Methodology
%\item What is missing?
\end{itemize}
\end{frame}

% \begin{frame}
  % \frametitle{Using MiniZinc IDE}
  % \begin{itemize}
  % \item Developed in the Australian NICTA project
  % \item Maintained by Monash University
  % \item Modelling tool with multiple back-end solvers
  % \item Available from \url{https://www.minizinc.org/}
  % \end{itemize}
% \end{frame}


% \begin{frame}
% \frametitle{Examples in ECLiPSe}
% \begin{itemize}
% \item Open sourced constraint programming language
% \item Development goes back to 1985
% \item ECRC, ICL, IC-Parc, PTL, Cisco
% \item \url{https://eclipseclp.org/}
% \item Specialities
% \begin{itemize}
% \item Develop new solvers for specific domains
% \item Integration with MIP
% \end{itemize}
% \item Not included in bundled MiniZinc IDE
% \item Specialized visualization tools used here
  % \begin{itemize}
  % \item CP-Viz, Simonis et al. 2010
    % \end{itemize}
% \end{itemize}
% \end{frame}

\begin{frame}
\frametitle{Tutorial Based on ECLiPSe ELearning Course}
\begin{itemize}
\item Self-study course in constraint programming
\item Supported by Cisco Systems and Silicon Valley Community Foundation
\item Multi-media format, video lectures, slides, handout etc
\item {\small \url{https://eclipseclp.org/ELearning/index.html}}
\end{itemize}
\end{frame}

\begin{frame}
\frametitle{Also Part of CRT-AI Constraint Week}
\begin{itemize}
\item Annual one week course on CP and Optimization in Ireland
\item Part of national training program for PhD students in AI
\item \url{https://www.crt-ai.ie/}
\end{itemize}
\end{frame}



\begin{frame}
\frametitle{Constraint Programming - in a nutshell}
\begin{itemize}
\item Declarative description of problems with
\begin{itemize}
\item {\em Variables} which range over (finite) sets of values
\item {\em Constraints} over subsets of variables which restrict possible value combinations
\item A {\em solution} is a value assignment which satisfies all constraints
\end{itemize}

\item Constraint propagation/reasoning
\begin{itemize}
\item Removing inconsistent values for variables
\item Detect failure if constraint can not be satisfied
\item Interaction of constraints via shared variables
\item Incomplete
\end{itemize}

\item Search
\begin{itemize}
\item User controlled assignment of values to variables
\item Each step triggers constraint propagation 
\end{itemize}
\item Different domains require/allow different methods
\end{itemize}
\end{frame}

\begin{frame}
  \frametitle{Constraint Programming is Different}
  \begin{itemize}
  \item Declarative Programming
    \begin{itemize}
    \item Concentrate on what you want
      \item Not how to get there
      \item Program != Algorithm
      \item Program = Model
    \end{itemize}
    \item Applied to Combinatorial Problems
      \begin{itemize}
        \item No complete polynomial algorithms known (exist?)
        \item CP less ad-hoc than heuristics
        \item Models can evolve
  \end{itemize}
  \end{itemize}
  \end{frame}
    
\begin{frame}
  \frametitle{A Subtractive Process}
  \begin{textblock}{4}(8,-3)
    \includegraphics[width=4cm]{../introduction/images/stages}
  \end{textblock}
  \vfill
  \begin{quote}
    ``Oh, bosh, as Mr. Ruskin says. Sculpture, per se, is the simplest thing in the world. All you have to do is to take a big chunk of marble and a hammer and chisel, make up your mind what you are about to create and chip off all the marble you don't want.''-Paris Gaulois.
  \end{quote}
  
  {\tiny Source: \url{https://quoteinvestigator.com/2014/06/22/chip-away/}}
\end{frame}

\begin{frame}
\frametitle{Basic Process}
\begin{center}
\begin{tikzpicture}
\node[shape=rectangle,fill=pantone127-4] (problem) at (2,5) {Problem};
\node (human) at (2,4) {Human};
\node[shape=rectangle,fill=pantone127-4] (model) at (2,3) {Model};
\node (solver) at (2,2) {Constraint Solver/Search};
\node[shape=rectangle,fill=pantone157-8] (solution) at (2,1) {Solution};
\draw[-] (problem) -- (human);
\draw[->] (human) -- (model);
\draw[-] (model) -- (solver);
\draw[->] (solver) -- (solution);
\end{tikzpicture}
\end{center}
\end{frame}

\begin{frame}
\frametitle{More Realistic}
\begin{center}
\begin{tikzpicture}[xscale=4.0]
\node[shape=rectangle,fill=pantone127-4] (problem) at (2,5) {Problem};
\node (human) at (2,4) {Human};
\node[shape=rectangle,fill=pantone127-4] (model) at (2,3) {Model};
\node (solver) at (2,2) {Constraint Solver/Search};
\node[shape=rectangle,fill=pantone24-8] (hangs) at (1,1) {Hangs};
\node[shape=rectangle,fill=pantone157-8] (solution) at (2,1) {Solution};
\node[shape=rectangle,fill=pantone24-8] (wrong) at (3,1) {Wrong Solution};
\draw[-] (problem) -- (human);
\draw[->] (human) -- (model);
\draw[-] (model) -- (solver);
\draw[->] (solver) -- (hangs);
\draw[->] (solver) -- (solution);
\draw[->] (solver) -- (wrong);
\draw[->] (hangs) -- (1,4) -- (human);
\draw[->] (wrong) -- (3,4) -- (human);
\end{tikzpicture}
\end{center}
\end{frame}

\begin{frame}
\frametitle{Dual Role of Model}
\begin{itemize}
\item Allows Human to Express Problem
\begin{itemize}
\item Close to Problem Domain
\item Constraints as Abstractions
\end{itemize}
\item Allows Solver to Execute
\begin{itemize}
\item Variables as Communication Mechanism
\item Constraints as Algorithms
\end{itemize}
\end{itemize}
\end{frame}

\begin{frame}
\frametitle{Modelling Frameworks}
\begin{itemize}
\item \href{https://www.minizinc.org/}{MiniZinc} (NICTA, Monash University, Australia)
\item \href{https://github.com/eomahony/Numberjack}{NumberJack} (Insight, Ireland) 
\item \href{https://www-users.york.ac.uk/peter.nightingale/savilerow//}{EssencePrime/SavilleRow} (UK) 
\item \href{https://cpmpy.readthedocs.io/en/latest/}{CPMpy} (KU Leuven)
\item Allow use of multiple back-end solvers
\item Compile model into variants for each solver
\item A priori solver independent model(CP, MIP, SAT)
\end{itemize}
\end{frame} 

\begin{frame}
\frametitle{Framework Process}
\begin{center}
\begin{tikzpicture}[xscale=2]
\node[shape=rectangle,fill=pantone127-4] (problem) at (2.5,5) {Problem};
\node (human) at (2.5,4) {Human};
\node[shape=rectangle,fill=pantone127-4] (model) at (2.5,3) {Model};
\node (compile) at (2.5,2) {Compile/Reformulate};
\node (solver1) at (1,1) {CP};
\node (solver2) at (2,1) {MIP};
\node (solver3) at (3,1) {SAT};
\node (solver4) at (4,1) {Other};
\node[shape=rectangle,fill=pantone157-8] (solution1) at (1,0) {Solution};
\node[shape=rectangle,fill=pantone157-8] (solution2) at (2,0) {Solution};
\node[shape=rectangle,fill=pantone157-8] (solution3) at (3,0) {Solution};
\node[shape=rectangle,fill=pantone157-8] (solution4) at (4,0) {Solution};
\draw[-] (problem) -- (human);
\draw[->] (human) -- (model);
\draw[-] (model) -- (compile);
\draw[-] (compile) -- (solver1);
\draw[-] (compile) -- (solver2);
\draw[-] (compile) -- (solver3);
\draw[-] (compile) -- (solver4);
\draw[->] (solver1) -- (solution1);
\draw[->] (solver2) -- (solution2);
\draw[->] (solver3) -- (solution3);
\draw[->] (solver4) -- (solution4);
\end{tikzpicture}
\end{center}
\end{frame}

% \begin{frame}
  % \frametitle{Do It Now!}
  % \begin{itemize}
  % \item Download and install Minizinc
  % \item \url{https://www.minizinc.org/}
  % \end{itemize}
% \end{frame}


\begin{frame}
\frametitle{Why use Puzzles as Examples?}
\begin{itemize}
\item Easy to understand the problem
\item Solvable by hand without specialized knowledge
\item Possible to compare automated to manual solving process
\end{itemize}

\begin{quote}
The puzzle, though inanimate, is presented as a solvable problem without lasting negative consequences, a very low-risk low-reward situation. By being a puzzle, the object is attempting to convince the user that it must be completed.
\end{quote}
{\tiny Source: \href{https://www.everydayrhetoric.com/2020/03/the-rhetoric-of-puzzles/}{Every Day Rhetoric}}
\end{frame}


\mode<all>{
\part{Basic Constraint Propagation}
}
\begin{frame}
\partpage
\end{frame}
\mode<all>{
\begin{frame}
\frametitle{Objectives}
\begin{itemize}
\item Overview of Core Constraint Programming
\item Three Main Concepts
\begin{itemize}
\item Constraint Propagation
\item Global Constraints
\item Customizing Search
\end{itemize}
%\item Get Some Experience with MiniZinc
\item Topics will be treated in more detail in later parts of the school
\item Based on Examples, not Formal Description
\end{itemize}
\end{frame}

\begin{frame}
\frametitle{Outline}
\begin{itemize}
\item Why Constraint Programming?
\item Constraint Propagation
\item Global Constraints
\item Customizing Search
%\item Choices in Modelling
%\item Decomposition
%\item More Global Constraints
%\item Methodology
%\item What is missing?
\end{itemize}
\end{frame}

% \begin{frame}
  % \frametitle{Using MiniZinc IDE}
  % \begin{itemize}
  % \item Developed in the Australian NICTA project
  % \item Maintained by Monash University
  % \item Modelling tool with multiple back-end solvers
  % \item Available from \url{https://www.minizinc.org/}
  % \end{itemize}
% \end{frame}


% \begin{frame}
% \frametitle{Examples in ECLiPSe}
% \begin{itemize}
% \item Open sourced constraint programming language
% \item Development goes back to 1985
% \item ECRC, ICL, IC-Parc, PTL, Cisco
% \item \url{https://eclipseclp.org/}
% \item Specialities
% \begin{itemize}
% \item Develop new solvers for specific domains
% \item Integration with MIP
% \end{itemize}
% \item Not included in bundled MiniZinc IDE
% \item Specialized visualization tools used here
  % \begin{itemize}
  % \item CP-Viz, Simonis et al. 2010
    % \end{itemize}
% \end{itemize}
% \end{frame}

\begin{frame}
\frametitle{Tutorial Based on ECLiPSe ELearning Course}
\begin{itemize}
\item Self-study course in constraint programming
\item Supported by Cisco Systems and Silicon Valley Community Foundation
\item Multi-media format, video lectures, slides, handout etc
\item {\small \url{https://eclipseclp.org/ELearning/index.html}}
\end{itemize}
\end{frame}

\begin{frame}
\frametitle{Also Part of CRT-AI Constraint Week}
\begin{itemize}
\item Annual one week course on CP and Optimization in Ireland
\item Part of national training program for PhD students in AI
\item \url{https://www.crt-ai.ie/}
\end{itemize}
\end{frame}



\begin{frame}
\frametitle{Constraint Programming - in a nutshell}
\begin{itemize}
\item Declarative description of problems with
\begin{itemize}
\item {\em Variables} which range over (finite) sets of values
\item {\em Constraints} over subsets of variables which restrict possible value combinations
\item A {\em solution} is a value assignment which satisfies all constraints
\end{itemize}

\item Constraint propagation/reasoning
\begin{itemize}
\item Removing inconsistent values for variables
\item Detect failure if constraint can not be satisfied
\item Interaction of constraints via shared variables
\item Incomplete
\end{itemize}

\item Search
\begin{itemize}
\item User controlled assignment of values to variables
\item Each step triggers constraint propagation 
\end{itemize}
\item Different domains require/allow different methods
\end{itemize}
\end{frame}

\begin{frame}
  \frametitle{Constraint Programming is Different}
  \begin{itemize}
  \item Declarative Programming
    \begin{itemize}
    \item Concentrate on what you want
      \item Not how to get there
      \item Program != Algorithm
      \item Program = Model
    \end{itemize}
    \item Applied to Combinatorial Problems
      \begin{itemize}
        \item No complete polynomial algorithms known (exist?)
        \item CP less ad-hoc than heuristics
        \item Models can evolve
  \end{itemize}
  \end{itemize}
  \end{frame}
    
\begin{frame}
  \frametitle{A Subtractive Process}
  \begin{textblock}{4}(8,-3)
    \includegraphics[width=4cm]{../introduction/images/stages}
  \end{textblock}
  \vfill
  \begin{quote}
    ``Oh, bosh, as Mr. Ruskin says. Sculpture, per se, is the simplest thing in the world. All you have to do is to take a big chunk of marble and a hammer and chisel, make up your mind what you are about to create and chip off all the marble you don't want.''-Paris Gaulois.
  \end{quote}
  
  {\tiny Source: \url{https://quoteinvestigator.com/2014/06/22/chip-away/}}
\end{frame}

\begin{frame}
\frametitle{Basic Process}
\begin{center}
\begin{tikzpicture}
\node[shape=rectangle,fill=pantone127-4] (problem) at (2,5) {Problem};
\node (human) at (2,4) {Human};
\node[shape=rectangle,fill=pantone127-4] (model) at (2,3) {Model};
\node (solver) at (2,2) {Constraint Solver/Search};
\node[shape=rectangle,fill=pantone157-8] (solution) at (2,1) {Solution};
\draw[-] (problem) -- (human);
\draw[->] (human) -- (model);
\draw[-] (model) -- (solver);
\draw[->] (solver) -- (solution);
\end{tikzpicture}
\end{center}
\end{frame}

\begin{frame}
\frametitle{More Realistic}
\begin{center}
\begin{tikzpicture}[xscale=4.0]
\node[shape=rectangle,fill=pantone127-4] (problem) at (2,5) {Problem};
\node (human) at (2,4) {Human};
\node[shape=rectangle,fill=pantone127-4] (model) at (2,3) {Model};
\node (solver) at (2,2) {Constraint Solver/Search};
\node[shape=rectangle,fill=pantone24-8] (hangs) at (1,1) {Hangs};
\node[shape=rectangle,fill=pantone157-8] (solution) at (2,1) {Solution};
\node[shape=rectangle,fill=pantone24-8] (wrong) at (3,1) {Wrong Solution};
\draw[-] (problem) -- (human);
\draw[->] (human) -- (model);
\draw[-] (model) -- (solver);
\draw[->] (solver) -- (hangs);
\draw[->] (solver) -- (solution);
\draw[->] (solver) -- (wrong);
\draw[->] (hangs) -- (1,4) -- (human);
\draw[->] (wrong) -- (3,4) -- (human);
\end{tikzpicture}
\end{center}
\end{frame}

\begin{frame}
\frametitle{Dual Role of Model}
\begin{itemize}
\item Allows Human to Express Problem
\begin{itemize}
\item Close to Problem Domain
\item Constraints as Abstractions
\end{itemize}
\item Allows Solver to Execute
\begin{itemize}
\item Variables as Communication Mechanism
\item Constraints as Algorithms
\end{itemize}
\end{itemize}
\end{frame}

\begin{frame}
\frametitle{Modelling Frameworks}
\begin{itemize}
\item \href{https://www.minizinc.org/}{MiniZinc} (NICTA, Monash University, Australia)
\item \href{https://github.com/eomahony/Numberjack}{NumberJack} (Insight, Ireland) 
\item \href{https://www-users.york.ac.uk/peter.nightingale/savilerow//}{EssencePrime/SavilleRow} (UK) 
\item \href{https://cpmpy.readthedocs.io/en/latest/}{CPMpy} (KU Leuven)
\item Allow use of multiple back-end solvers
\item Compile model into variants for each solver
\item A priori solver independent model(CP, MIP, SAT)
\end{itemize}
\end{frame} 

\begin{frame}
\frametitle{Framework Process}
\begin{center}
\begin{tikzpicture}[xscale=2]
\node[shape=rectangle,fill=pantone127-4] (problem) at (2.5,5) {Problem};
\node (human) at (2.5,4) {Human};
\node[shape=rectangle,fill=pantone127-4] (model) at (2.5,3) {Model};
\node (compile) at (2.5,2) {Compile/Reformulate};
\node (solver1) at (1,1) {CP};
\node (solver2) at (2,1) {MIP};
\node (solver3) at (3,1) {SAT};
\node (solver4) at (4,1) {Other};
\node[shape=rectangle,fill=pantone157-8] (solution1) at (1,0) {Solution};
\node[shape=rectangle,fill=pantone157-8] (solution2) at (2,0) {Solution};
\node[shape=rectangle,fill=pantone157-8] (solution3) at (3,0) {Solution};
\node[shape=rectangle,fill=pantone157-8] (solution4) at (4,0) {Solution};
\draw[-] (problem) -- (human);
\draw[->] (human) -- (model);
\draw[-] (model) -- (compile);
\draw[-] (compile) -- (solver1);
\draw[-] (compile) -- (solver2);
\draw[-] (compile) -- (solver3);
\draw[-] (compile) -- (solver4);
\draw[->] (solver1) -- (solution1);
\draw[->] (solver2) -- (solution2);
\draw[->] (solver3) -- (solution3);
\draw[->] (solver4) -- (solution4);
\end{tikzpicture}
\end{center}
\end{frame}

% \begin{frame}
  % \frametitle{Do It Now!}
  % \begin{itemize}
  % \item Download and install Minizinc
  % \item \url{https://www.minizinc.org/}
  % \end{itemize}
% \end{frame}


\begin{frame}
\frametitle{Why use Puzzles as Examples?}
\begin{itemize}
\item Easy to understand the problem
\item Solvable by hand without specialized knowledge
\item Possible to compare automated to manual solving process
\end{itemize}

\begin{quote}
The puzzle, though inanimate, is presented as a solvable problem without lasting negative consequences, a very low-risk low-reward situation. By being a puzzle, the object is attempting to convince the user that it must be completed.
\end{quote}
{\tiny Source: \href{https://www.everydayrhetoric.com/2020/03/the-rhetoric-of-puzzles/}{Every Day Rhetoric}}
\end{frame}


\mode<all>{
\part{Basic Constraint Propagation}
}
\begin{frame}
\partpage
\end{frame}
\mode<all>{
\begin{frame}
\frametitle{Objectives}
\begin{itemize}
\item Overview of Core Constraint Programming
\item Three Main Concepts
\begin{itemize}
\item Constraint Propagation
\item Global Constraints
\item Customizing Search
\end{itemize}
%\item Get Some Experience with MiniZinc
\item Topics will be treated in more detail in later parts of the school
\item Based on Examples, not Formal Description
\end{itemize}
\end{frame}

\begin{frame}
\frametitle{Outline}
\begin{itemize}
\item Why Constraint Programming?
\item Constraint Propagation
\item Global Constraints
\item Customizing Search
%\item Choices in Modelling
%\item Decomposition
%\item More Global Constraints
%\item Methodology
%\item What is missing?
\end{itemize}
\end{frame}

% \begin{frame}
  % \frametitle{Using MiniZinc IDE}
  % \begin{itemize}
  % \item Developed in the Australian NICTA project
  % \item Maintained by Monash University
  % \item Modelling tool with multiple back-end solvers
  % \item Available from \url{https://www.minizinc.org/}
  % \end{itemize}
% \end{frame}


% \begin{frame}
% \frametitle{Examples in ECLiPSe}
% \begin{itemize}
% \item Open sourced constraint programming language
% \item Development goes back to 1985
% \item ECRC, ICL, IC-Parc, PTL, Cisco
% \item \url{https://eclipseclp.org/}
% \item Specialities
% \begin{itemize}
% \item Develop new solvers for specific domains
% \item Integration with MIP
% \end{itemize}
% \item Not included in bundled MiniZinc IDE
% \item Specialized visualization tools used here
  % \begin{itemize}
  % \item CP-Viz, Simonis et al. 2010
    % \end{itemize}
% \end{itemize}
% \end{frame}

\begin{frame}
\frametitle{Tutorial Based on ECLiPSe ELearning Course}
\begin{itemize}
\item Self-study course in constraint programming
\item Supported by Cisco Systems and Silicon Valley Community Foundation
\item Multi-media format, video lectures, slides, handout etc
\item {\small \url{https://eclipseclp.org/ELearning/index.html}}
\end{itemize}
\end{frame}

\begin{frame}
\frametitle{Also Part of CRT-AI Constraint Week}
\begin{itemize}
\item Annual one week course on CP and Optimization in Ireland
\item Part of national training program for PhD students in AI
\item \url{https://www.crt-ai.ie/}
\end{itemize}
\end{frame}



\begin{frame}
\frametitle{Constraint Programming - in a nutshell}
\begin{itemize}
\item Declarative description of problems with
\begin{itemize}
\item {\em Variables} which range over (finite) sets of values
\item {\em Constraints} over subsets of variables which restrict possible value combinations
\item A {\em solution} is a value assignment which satisfies all constraints
\end{itemize}

\item Constraint propagation/reasoning
\begin{itemize}
\item Removing inconsistent values for variables
\item Detect failure if constraint can not be satisfied
\item Interaction of constraints via shared variables
\item Incomplete
\end{itemize}

\item Search
\begin{itemize}
\item User controlled assignment of values to variables
\item Each step triggers constraint propagation 
\end{itemize}
\item Different domains require/allow different methods
\end{itemize}
\end{frame}

\begin{frame}
  \frametitle{Constraint Programming is Different}
  \begin{itemize}
  \item Declarative Programming
    \begin{itemize}
    \item Concentrate on what you want
      \item Not how to get there
      \item Program != Algorithm
      \item Program = Model
    \end{itemize}
    \item Applied to Combinatorial Problems
      \begin{itemize}
        \item No complete polynomial algorithms known (exist?)
        \item CP less ad-hoc than heuristics
        \item Models can evolve
  \end{itemize}
  \end{itemize}
  \end{frame}
    
\begin{frame}
  \frametitle{A Subtractive Process}
  \begin{textblock}{4}(8,-3)
    \includegraphics[width=4cm]{../introduction/images/stages}
  \end{textblock}
  \vfill
  \begin{quote}
    ``Oh, bosh, as Mr. Ruskin says. Sculpture, per se, is the simplest thing in the world. All you have to do is to take a big chunk of marble and a hammer and chisel, make up your mind what you are about to create and chip off all the marble you don't want.''-Paris Gaulois.
  \end{quote}
  
  {\tiny Source: \url{https://quoteinvestigator.com/2014/06/22/chip-away/}}
\end{frame}

\begin{frame}
\frametitle{Basic Process}
\begin{center}
\begin{tikzpicture}
\node[shape=rectangle,fill=pantone127-4] (problem) at (2,5) {Problem};
\node (human) at (2,4) {Human};
\node[shape=rectangle,fill=pantone127-4] (model) at (2,3) {Model};
\node (solver) at (2,2) {Constraint Solver/Search};
\node[shape=rectangle,fill=pantone157-8] (solution) at (2,1) {Solution};
\draw[-] (problem) -- (human);
\draw[->] (human) -- (model);
\draw[-] (model) -- (solver);
\draw[->] (solver) -- (solution);
\end{tikzpicture}
\end{center}
\end{frame}

\begin{frame}
\frametitle{More Realistic}
\begin{center}
\begin{tikzpicture}[xscale=4.0]
\node[shape=rectangle,fill=pantone127-4] (problem) at (2,5) {Problem};
\node (human) at (2,4) {Human};
\node[shape=rectangle,fill=pantone127-4] (model) at (2,3) {Model};
\node (solver) at (2,2) {Constraint Solver/Search};
\node[shape=rectangle,fill=pantone24-8] (hangs) at (1,1) {Hangs};
\node[shape=rectangle,fill=pantone157-8] (solution) at (2,1) {Solution};
\node[shape=rectangle,fill=pantone24-8] (wrong) at (3,1) {Wrong Solution};
\draw[-] (problem) -- (human);
\draw[->] (human) -- (model);
\draw[-] (model) -- (solver);
\draw[->] (solver) -- (hangs);
\draw[->] (solver) -- (solution);
\draw[->] (solver) -- (wrong);
\draw[->] (hangs) -- (1,4) -- (human);
\draw[->] (wrong) -- (3,4) -- (human);
\end{tikzpicture}
\end{center}
\end{frame}

\begin{frame}
\frametitle{Dual Role of Model}
\begin{itemize}
\item Allows Human to Express Problem
\begin{itemize}
\item Close to Problem Domain
\item Constraints as Abstractions
\end{itemize}
\item Allows Solver to Execute
\begin{itemize}
\item Variables as Communication Mechanism
\item Constraints as Algorithms
\end{itemize}
\end{itemize}
\end{frame}

\begin{frame}
\frametitle{Modelling Frameworks}
\begin{itemize}
\item \href{https://www.minizinc.org/}{MiniZinc} (NICTA, Monash University, Australia)
\item \href{https://github.com/eomahony/Numberjack}{NumberJack} (Insight, Ireland) 
\item \href{https://www-users.york.ac.uk/peter.nightingale/savilerow//}{EssencePrime/SavilleRow} (UK) 
\item \href{https://cpmpy.readthedocs.io/en/latest/}{CPMpy} (KU Leuven)
\item Allow use of multiple back-end solvers
\item Compile model into variants for each solver
\item A priori solver independent model(CP, MIP, SAT)
\end{itemize}
\end{frame} 

\begin{frame}
\frametitle{Framework Process}
\begin{center}
\begin{tikzpicture}[xscale=2]
\node[shape=rectangle,fill=pantone127-4] (problem) at (2.5,5) {Problem};
\node (human) at (2.5,4) {Human};
\node[shape=rectangle,fill=pantone127-4] (model) at (2.5,3) {Model};
\node (compile) at (2.5,2) {Compile/Reformulate};
\node (solver1) at (1,1) {CP};
\node (solver2) at (2,1) {MIP};
\node (solver3) at (3,1) {SAT};
\node (solver4) at (4,1) {Other};
\node[shape=rectangle,fill=pantone157-8] (solution1) at (1,0) {Solution};
\node[shape=rectangle,fill=pantone157-8] (solution2) at (2,0) {Solution};
\node[shape=rectangle,fill=pantone157-8] (solution3) at (3,0) {Solution};
\node[shape=rectangle,fill=pantone157-8] (solution4) at (4,0) {Solution};
\draw[-] (problem) -- (human);
\draw[->] (human) -- (model);
\draw[-] (model) -- (compile);
\draw[-] (compile) -- (solver1);
\draw[-] (compile) -- (solver2);
\draw[-] (compile) -- (solver3);
\draw[-] (compile) -- (solver4);
\draw[->] (solver1) -- (solution1);
\draw[->] (solver2) -- (solution2);
\draw[->] (solver3) -- (solution3);
\draw[->] (solver4) -- (solution4);
\end{tikzpicture}
\end{center}
\end{frame}

% \begin{frame}
  % \frametitle{Do It Now!}
  % \begin{itemize}
  % \item Download and install Minizinc
  % \item \url{https://www.minizinc.org/}
  % \end{itemize}
% \end{frame}


\begin{frame}
\frametitle{Why use Puzzles as Examples?}
\begin{itemize}
\item Easy to understand the problem
\item Solvable by hand without specialized knowledge
\item Possible to compare automated to manual solving process
\end{itemize}

\begin{quote}
The puzzle, though inanimate, is presented as a solvable problem without lasting negative consequences, a very low-risk low-reward situation. By being a puzzle, the object is attempting to convince the user that it must be completed.
\end{quote}
{\tiny Source: \href{https://www.everydayrhetoric.com/2020/03/the-rhetoric-of-puzzles/}{Every Day Rhetoric}}
\end{frame}


\mode<all>{
\part{Basic Constraint Propagation}
}
\begin{frame}
\partpage
\end{frame}
\mode<all>{
\input{../sendmore/body}
}

\mode<all>{
\part{Global Constraints}
}
\begin{frame}
\partpage
\end{frame}
\mode<all>{
\input{../sudoku/body}
}

\clearpage
\mode<all>{
\part{Customizing Search}
}
\begin{frame}
\partpage
\end{frame}
\mode<all>{
\input{../nqueen/body}
}


%% \clearpage
%% \mode<all>{
%%   \part{Problem Decomposition}
%% }
%% \begin{frame}
%%   \partpage
%% \end{frame}
%% \mode<all>{
%%   \input{../party/body}
%%   }

%% \clearpage
%% \mode<all>{
%%   \part{More Global Constraints}
%% }
%% \begin{frame}
%%   \partpage
%% \end{frame}
%% \mode<all>{
%%   \input{../car/body}
%%   }

%% \clearpage
%% \mode<all>{
%%   \part{Methodology}
%% }
%% \begin{frame}
%%   \partpage
%% \end{frame}
%% \mode<all>{
%%   \input{../methodology/body}
%%   }

\clearpage
\mode<all>{
\part{What is missing?}
}
\begin{frame}
\partpage
\end{frame}

\begin{frame}
\frametitle{Many Specialized Topics}
\begin{itemize}
\item How to design efficient core engine
\item Hybrids with LP/MIP tools
\item Hybrids with SAT
\item Symmetry breaking
\item Use of MDD/BDD to encode sets of solutions
\item High level modelling tools
\item Debugging/visualization
\end{itemize}
\end{frame}

\begin{frame}
\frametitle{Reformulation}
\begin{itemize}
\item Just because the user has modelled it this way, it doesn't mean we have to solve it that way
\begin{itemize}
\item Replace some constraint(s) by other, equivalent constraints
\item Because we don't have that constraint in our system
\item For performance
\end{itemize}
\end{itemize}
\end{frame}

\begin{frame}
\frametitle{Learning}
\begin{itemize}
\item While solving the problem we can learn how to strengthen the model/search 
\begin{itemize}
\item Understand which constraints/method contribute to propagation and change schedule
\item Learn no-good constraints by explaining failure
\item Adapt search strategy based on search experience
\end{itemize}
\end{itemize}
\end{frame}

%% \begin{frame}
%% \frametitle{Refined Process}
%% \begin{center}
%% \begin{tikzpicture}
%% \node[shape=rectangle,fill=pantone127-4] (problem) at (3,7) {Problem};
%% \node (human) at (3,6) {Human};
%% \node[shape=rectangle,fill=pantone127-4] (model) at (3,5) {User Model};
%% \node (reformulation) at (3,4) {Reformulation};
%% \node[shape=rectangle,fill=pantone127-4] (implementation) at (3,3) {Implementation Model};
%% \node (solver) at (3,2) {Constraint Solver/Search};
%% \node[shape=rectangle,fill=pantone157-8] (solution) at (3,1) {Solution};
%% \draw[-] (problem) -- (human);
%% \draw[->] (human) -- (model);
%% \draw[-] (model) -- (reformulation);
%% \draw[->] (reformulation) -- (implementation);
%% \draw[-] (implementation) -- (solver);
%% \draw[->] (solver) -- (0,2) -- (0,3) -- (implementation);
%% \draw[->] (solver) -- (solution);
%% \end{tikzpicture}
%% \end{center}
%% \end{frame}

%\begin{frame}
%\frametitle{What is CP actually used for?}
%\url{http://hsimonis.wordpress.com}
%\includegraphics[width=10cm]{../introduction/images/blog}
%\end{frame}

%% \begin{frame}
%% \frametitle{Exercises}
%% \begin{itemize}
%% \item Install Constraint Solver of your Choice
%% \item MiniZinc (MiningZinc) will be used in later sessions
%% \item Run Sendmoremoney, Sudoku, Queens examples
%% \item Can you choose which propagator to use for alldifferent?
%% \item Can you use different search strategies?
%% \end{itemize}
%% \end{frame}

\begin{frame}
\frametitle{More Learning Resources}
\begin{itemize}
\item Survey of Methods, Resources, and Formats for Teaching Constraint Programming
\begin{itemize}
\item by Tejas Santanam, Helmut Simonis
\item \url{https://doi.org/10.48550/arXiv.2403.12717}
\item Based on survey of community for WTCP 2023
\item \url{https://hsimonis.github.io/WTCP2023/}
\end{itemize}
\end{itemize}
\end{frame}


}

\mode<all>{
\part{Global Constraints}
}
\begin{frame}
\partpage
\end{frame}
\mode<all>{
\begin{frame}
\frametitle{Objectives}
\begin{itemize}
\item Overview of Core Constraint Programming
\item Three Main Concepts
\begin{itemize}
\item Constraint Propagation
\item Global Constraints
\item Customizing Search
\end{itemize}
%\item Get Some Experience with MiniZinc
\item Topics will be treated in more detail in later parts of the school
\item Based on Examples, not Formal Description
\end{itemize}
\end{frame}

\begin{frame}
\frametitle{Outline}
\begin{itemize}
\item Why Constraint Programming?
\item Constraint Propagation
\item Global Constraints
\item Customizing Search
%\item Choices in Modelling
%\item Decomposition
%\item More Global Constraints
%\item Methodology
%\item What is missing?
\end{itemize}
\end{frame}

% \begin{frame}
  % \frametitle{Using MiniZinc IDE}
  % \begin{itemize}
  % \item Developed in the Australian NICTA project
  % \item Maintained by Monash University
  % \item Modelling tool with multiple back-end solvers
  % \item Available from \url{https://www.minizinc.org/}
  % \end{itemize}
% \end{frame}


% \begin{frame}
% \frametitle{Examples in ECLiPSe}
% \begin{itemize}
% \item Open sourced constraint programming language
% \item Development goes back to 1985
% \item ECRC, ICL, IC-Parc, PTL, Cisco
% \item \url{https://eclipseclp.org/}
% \item Specialities
% \begin{itemize}
% \item Develop new solvers for specific domains
% \item Integration with MIP
% \end{itemize}
% \item Not included in bundled MiniZinc IDE
% \item Specialized visualization tools used here
  % \begin{itemize}
  % \item CP-Viz, Simonis et al. 2010
    % \end{itemize}
% \end{itemize}
% \end{frame}

\begin{frame}
\frametitle{Tutorial Based on ECLiPSe ELearning Course}
\begin{itemize}
\item Self-study course in constraint programming
\item Supported by Cisco Systems and Silicon Valley Community Foundation
\item Multi-media format, video lectures, slides, handout etc
\item {\small \url{https://eclipseclp.org/ELearning/index.html}}
\end{itemize}
\end{frame}

\begin{frame}
\frametitle{Also Part of CRT-AI Constraint Week}
\begin{itemize}
\item Annual one week course on CP and Optimization in Ireland
\item Part of national training program for PhD students in AI
\item \url{https://www.crt-ai.ie/}
\end{itemize}
\end{frame}



\begin{frame}
\frametitle{Constraint Programming - in a nutshell}
\begin{itemize}
\item Declarative description of problems with
\begin{itemize}
\item {\em Variables} which range over (finite) sets of values
\item {\em Constraints} over subsets of variables which restrict possible value combinations
\item A {\em solution} is a value assignment which satisfies all constraints
\end{itemize}

\item Constraint propagation/reasoning
\begin{itemize}
\item Removing inconsistent values for variables
\item Detect failure if constraint can not be satisfied
\item Interaction of constraints via shared variables
\item Incomplete
\end{itemize}

\item Search
\begin{itemize}
\item User controlled assignment of values to variables
\item Each step triggers constraint propagation 
\end{itemize}
\item Different domains require/allow different methods
\end{itemize}
\end{frame}

\begin{frame}
  \frametitle{Constraint Programming is Different}
  \begin{itemize}
  \item Declarative Programming
    \begin{itemize}
    \item Concentrate on what you want
      \item Not how to get there
      \item Program != Algorithm
      \item Program = Model
    \end{itemize}
    \item Applied to Combinatorial Problems
      \begin{itemize}
        \item No complete polynomial algorithms known (exist?)
        \item CP less ad-hoc than heuristics
        \item Models can evolve
  \end{itemize}
  \end{itemize}
  \end{frame}
    
\begin{frame}
  \frametitle{A Subtractive Process}
  \begin{textblock}{4}(8,-3)
    \includegraphics[width=4cm]{../introduction/images/stages}
  \end{textblock}
  \vfill
  \begin{quote}
    ``Oh, bosh, as Mr. Ruskin says. Sculpture, per se, is the simplest thing in the world. All you have to do is to take a big chunk of marble and a hammer and chisel, make up your mind what you are about to create and chip off all the marble you don't want.''-Paris Gaulois.
  \end{quote}
  
  {\tiny Source: \url{https://quoteinvestigator.com/2014/06/22/chip-away/}}
\end{frame}

\begin{frame}
\frametitle{Basic Process}
\begin{center}
\begin{tikzpicture}
\node[shape=rectangle,fill=pantone127-4] (problem) at (2,5) {Problem};
\node (human) at (2,4) {Human};
\node[shape=rectangle,fill=pantone127-4] (model) at (2,3) {Model};
\node (solver) at (2,2) {Constraint Solver/Search};
\node[shape=rectangle,fill=pantone157-8] (solution) at (2,1) {Solution};
\draw[-] (problem) -- (human);
\draw[->] (human) -- (model);
\draw[-] (model) -- (solver);
\draw[->] (solver) -- (solution);
\end{tikzpicture}
\end{center}
\end{frame}

\begin{frame}
\frametitle{More Realistic}
\begin{center}
\begin{tikzpicture}[xscale=4.0]
\node[shape=rectangle,fill=pantone127-4] (problem) at (2,5) {Problem};
\node (human) at (2,4) {Human};
\node[shape=rectangle,fill=pantone127-4] (model) at (2,3) {Model};
\node (solver) at (2,2) {Constraint Solver/Search};
\node[shape=rectangle,fill=pantone24-8] (hangs) at (1,1) {Hangs};
\node[shape=rectangle,fill=pantone157-8] (solution) at (2,1) {Solution};
\node[shape=rectangle,fill=pantone24-8] (wrong) at (3,1) {Wrong Solution};
\draw[-] (problem) -- (human);
\draw[->] (human) -- (model);
\draw[-] (model) -- (solver);
\draw[->] (solver) -- (hangs);
\draw[->] (solver) -- (solution);
\draw[->] (solver) -- (wrong);
\draw[->] (hangs) -- (1,4) -- (human);
\draw[->] (wrong) -- (3,4) -- (human);
\end{tikzpicture}
\end{center}
\end{frame}

\begin{frame}
\frametitle{Dual Role of Model}
\begin{itemize}
\item Allows Human to Express Problem
\begin{itemize}
\item Close to Problem Domain
\item Constraints as Abstractions
\end{itemize}
\item Allows Solver to Execute
\begin{itemize}
\item Variables as Communication Mechanism
\item Constraints as Algorithms
\end{itemize}
\end{itemize}
\end{frame}

\begin{frame}
\frametitle{Modelling Frameworks}
\begin{itemize}
\item \href{https://www.minizinc.org/}{MiniZinc} (NICTA, Monash University, Australia)
\item \href{https://github.com/eomahony/Numberjack}{NumberJack} (Insight, Ireland) 
\item \href{https://www-users.york.ac.uk/peter.nightingale/savilerow//}{EssencePrime/SavilleRow} (UK) 
\item \href{https://cpmpy.readthedocs.io/en/latest/}{CPMpy} (KU Leuven)
\item Allow use of multiple back-end solvers
\item Compile model into variants for each solver
\item A priori solver independent model(CP, MIP, SAT)
\end{itemize}
\end{frame} 

\begin{frame}
\frametitle{Framework Process}
\begin{center}
\begin{tikzpicture}[xscale=2]
\node[shape=rectangle,fill=pantone127-4] (problem) at (2.5,5) {Problem};
\node (human) at (2.5,4) {Human};
\node[shape=rectangle,fill=pantone127-4] (model) at (2.5,3) {Model};
\node (compile) at (2.5,2) {Compile/Reformulate};
\node (solver1) at (1,1) {CP};
\node (solver2) at (2,1) {MIP};
\node (solver3) at (3,1) {SAT};
\node (solver4) at (4,1) {Other};
\node[shape=rectangle,fill=pantone157-8] (solution1) at (1,0) {Solution};
\node[shape=rectangle,fill=pantone157-8] (solution2) at (2,0) {Solution};
\node[shape=rectangle,fill=pantone157-8] (solution3) at (3,0) {Solution};
\node[shape=rectangle,fill=pantone157-8] (solution4) at (4,0) {Solution};
\draw[-] (problem) -- (human);
\draw[->] (human) -- (model);
\draw[-] (model) -- (compile);
\draw[-] (compile) -- (solver1);
\draw[-] (compile) -- (solver2);
\draw[-] (compile) -- (solver3);
\draw[-] (compile) -- (solver4);
\draw[->] (solver1) -- (solution1);
\draw[->] (solver2) -- (solution2);
\draw[->] (solver3) -- (solution3);
\draw[->] (solver4) -- (solution4);
\end{tikzpicture}
\end{center}
\end{frame}

% \begin{frame}
  % \frametitle{Do It Now!}
  % \begin{itemize}
  % \item Download and install Minizinc
  % \item \url{https://www.minizinc.org/}
  % \end{itemize}
% \end{frame}


\begin{frame}
\frametitle{Why use Puzzles as Examples?}
\begin{itemize}
\item Easy to understand the problem
\item Solvable by hand without specialized knowledge
\item Possible to compare automated to manual solving process
\end{itemize}

\begin{quote}
The puzzle, though inanimate, is presented as a solvable problem without lasting negative consequences, a very low-risk low-reward situation. By being a puzzle, the object is attempting to convince the user that it must be completed.
\end{quote}
{\tiny Source: \href{https://www.everydayrhetoric.com/2020/03/the-rhetoric-of-puzzles/}{Every Day Rhetoric}}
\end{frame}


\mode<all>{
\part{Basic Constraint Propagation}
}
\begin{frame}
\partpage
\end{frame}
\mode<all>{
\input{../sendmore/body}
}

\mode<all>{
\part{Global Constraints}
}
\begin{frame}
\partpage
\end{frame}
\mode<all>{
\input{../sudoku/body}
}

\clearpage
\mode<all>{
\part{Customizing Search}
}
\begin{frame}
\partpage
\end{frame}
\mode<all>{
\input{../nqueen/body}
}


%% \clearpage
%% \mode<all>{
%%   \part{Problem Decomposition}
%% }
%% \begin{frame}
%%   \partpage
%% \end{frame}
%% \mode<all>{
%%   \input{../party/body}
%%   }

%% \clearpage
%% \mode<all>{
%%   \part{More Global Constraints}
%% }
%% \begin{frame}
%%   \partpage
%% \end{frame}
%% \mode<all>{
%%   \input{../car/body}
%%   }

%% \clearpage
%% \mode<all>{
%%   \part{Methodology}
%% }
%% \begin{frame}
%%   \partpage
%% \end{frame}
%% \mode<all>{
%%   \input{../methodology/body}
%%   }

\clearpage
\mode<all>{
\part{What is missing?}
}
\begin{frame}
\partpage
\end{frame}

\begin{frame}
\frametitle{Many Specialized Topics}
\begin{itemize}
\item How to design efficient core engine
\item Hybrids with LP/MIP tools
\item Hybrids with SAT
\item Symmetry breaking
\item Use of MDD/BDD to encode sets of solutions
\item High level modelling tools
\item Debugging/visualization
\end{itemize}
\end{frame}

\begin{frame}
\frametitle{Reformulation}
\begin{itemize}
\item Just because the user has modelled it this way, it doesn't mean we have to solve it that way
\begin{itemize}
\item Replace some constraint(s) by other, equivalent constraints
\item Because we don't have that constraint in our system
\item For performance
\end{itemize}
\end{itemize}
\end{frame}

\begin{frame}
\frametitle{Learning}
\begin{itemize}
\item While solving the problem we can learn how to strengthen the model/search 
\begin{itemize}
\item Understand which constraints/method contribute to propagation and change schedule
\item Learn no-good constraints by explaining failure
\item Adapt search strategy based on search experience
\end{itemize}
\end{itemize}
\end{frame}

%% \begin{frame}
%% \frametitle{Refined Process}
%% \begin{center}
%% \begin{tikzpicture}
%% \node[shape=rectangle,fill=pantone127-4] (problem) at (3,7) {Problem};
%% \node (human) at (3,6) {Human};
%% \node[shape=rectangle,fill=pantone127-4] (model) at (3,5) {User Model};
%% \node (reformulation) at (3,4) {Reformulation};
%% \node[shape=rectangle,fill=pantone127-4] (implementation) at (3,3) {Implementation Model};
%% \node (solver) at (3,2) {Constraint Solver/Search};
%% \node[shape=rectangle,fill=pantone157-8] (solution) at (3,1) {Solution};
%% \draw[-] (problem) -- (human);
%% \draw[->] (human) -- (model);
%% \draw[-] (model) -- (reformulation);
%% \draw[->] (reformulation) -- (implementation);
%% \draw[-] (implementation) -- (solver);
%% \draw[->] (solver) -- (0,2) -- (0,3) -- (implementation);
%% \draw[->] (solver) -- (solution);
%% \end{tikzpicture}
%% \end{center}
%% \end{frame}

%\begin{frame}
%\frametitle{What is CP actually used for?}
%\url{http://hsimonis.wordpress.com}
%\includegraphics[width=10cm]{../introduction/images/blog}
%\end{frame}

%% \begin{frame}
%% \frametitle{Exercises}
%% \begin{itemize}
%% \item Install Constraint Solver of your Choice
%% \item MiniZinc (MiningZinc) will be used in later sessions
%% \item Run Sendmoremoney, Sudoku, Queens examples
%% \item Can you choose which propagator to use for alldifferent?
%% \item Can you use different search strategies?
%% \end{itemize}
%% \end{frame}

\begin{frame}
\frametitle{More Learning Resources}
\begin{itemize}
\item Survey of Methods, Resources, and Formats for Teaching Constraint Programming
\begin{itemize}
\item by Tejas Santanam, Helmut Simonis
\item \url{https://doi.org/10.48550/arXiv.2403.12717}
\item Based on survey of community for WTCP 2023
\item \url{https://hsimonis.github.io/WTCP2023/}
\end{itemize}
\end{itemize}
\end{frame}


}

\clearpage
\mode<all>{
\part{Customizing Search}
}
\begin{frame}
\partpage
\end{frame}
\mode<all>{
\begin{frame}
\frametitle{Objectives}
\begin{itemize}
\item Overview of Core Constraint Programming
\item Three Main Concepts
\begin{itemize}
\item Constraint Propagation
\item Global Constraints
\item Customizing Search
\end{itemize}
%\item Get Some Experience with MiniZinc
\item Topics will be treated in more detail in later parts of the school
\item Based on Examples, not Formal Description
\end{itemize}
\end{frame}

\begin{frame}
\frametitle{Outline}
\begin{itemize}
\item Why Constraint Programming?
\item Constraint Propagation
\item Global Constraints
\item Customizing Search
%\item Choices in Modelling
%\item Decomposition
%\item More Global Constraints
%\item Methodology
%\item What is missing?
\end{itemize}
\end{frame}

% \begin{frame}
  % \frametitle{Using MiniZinc IDE}
  % \begin{itemize}
  % \item Developed in the Australian NICTA project
  % \item Maintained by Monash University
  % \item Modelling tool with multiple back-end solvers
  % \item Available from \url{https://www.minizinc.org/}
  % \end{itemize}
% \end{frame}


% \begin{frame}
% \frametitle{Examples in ECLiPSe}
% \begin{itemize}
% \item Open sourced constraint programming language
% \item Development goes back to 1985
% \item ECRC, ICL, IC-Parc, PTL, Cisco
% \item \url{https://eclipseclp.org/}
% \item Specialities
% \begin{itemize}
% \item Develop new solvers for specific domains
% \item Integration with MIP
% \end{itemize}
% \item Not included in bundled MiniZinc IDE
% \item Specialized visualization tools used here
  % \begin{itemize}
  % \item CP-Viz, Simonis et al. 2010
    % \end{itemize}
% \end{itemize}
% \end{frame}

\begin{frame}
\frametitle{Tutorial Based on ECLiPSe ELearning Course}
\begin{itemize}
\item Self-study course in constraint programming
\item Supported by Cisco Systems and Silicon Valley Community Foundation
\item Multi-media format, video lectures, slides, handout etc
\item {\small \url{https://eclipseclp.org/ELearning/index.html}}
\end{itemize}
\end{frame}

\begin{frame}
\frametitle{Also Part of CRT-AI Constraint Week}
\begin{itemize}
\item Annual one week course on CP and Optimization in Ireland
\item Part of national training program for PhD students in AI
\item \url{https://www.crt-ai.ie/}
\end{itemize}
\end{frame}



\begin{frame}
\frametitle{Constraint Programming - in a nutshell}
\begin{itemize}
\item Declarative description of problems with
\begin{itemize}
\item {\em Variables} which range over (finite) sets of values
\item {\em Constraints} over subsets of variables which restrict possible value combinations
\item A {\em solution} is a value assignment which satisfies all constraints
\end{itemize}

\item Constraint propagation/reasoning
\begin{itemize}
\item Removing inconsistent values for variables
\item Detect failure if constraint can not be satisfied
\item Interaction of constraints via shared variables
\item Incomplete
\end{itemize}

\item Search
\begin{itemize}
\item User controlled assignment of values to variables
\item Each step triggers constraint propagation 
\end{itemize}
\item Different domains require/allow different methods
\end{itemize}
\end{frame}

\begin{frame}
  \frametitle{Constraint Programming is Different}
  \begin{itemize}
  \item Declarative Programming
    \begin{itemize}
    \item Concentrate on what you want
      \item Not how to get there
      \item Program != Algorithm
      \item Program = Model
    \end{itemize}
    \item Applied to Combinatorial Problems
      \begin{itemize}
        \item No complete polynomial algorithms known (exist?)
        \item CP less ad-hoc than heuristics
        \item Models can evolve
  \end{itemize}
  \end{itemize}
  \end{frame}
    
\begin{frame}
  \frametitle{A Subtractive Process}
  \begin{textblock}{4}(8,-3)
    \includegraphics[width=4cm]{../introduction/images/stages}
  \end{textblock}
  \vfill
  \begin{quote}
    ``Oh, bosh, as Mr. Ruskin says. Sculpture, per se, is the simplest thing in the world. All you have to do is to take a big chunk of marble and a hammer and chisel, make up your mind what you are about to create and chip off all the marble you don't want.''-Paris Gaulois.
  \end{quote}
  
  {\tiny Source: \url{https://quoteinvestigator.com/2014/06/22/chip-away/}}
\end{frame}

\begin{frame}
\frametitle{Basic Process}
\begin{center}
\begin{tikzpicture}
\node[shape=rectangle,fill=pantone127-4] (problem) at (2,5) {Problem};
\node (human) at (2,4) {Human};
\node[shape=rectangle,fill=pantone127-4] (model) at (2,3) {Model};
\node (solver) at (2,2) {Constraint Solver/Search};
\node[shape=rectangle,fill=pantone157-8] (solution) at (2,1) {Solution};
\draw[-] (problem) -- (human);
\draw[->] (human) -- (model);
\draw[-] (model) -- (solver);
\draw[->] (solver) -- (solution);
\end{tikzpicture}
\end{center}
\end{frame}

\begin{frame}
\frametitle{More Realistic}
\begin{center}
\begin{tikzpicture}[xscale=4.0]
\node[shape=rectangle,fill=pantone127-4] (problem) at (2,5) {Problem};
\node (human) at (2,4) {Human};
\node[shape=rectangle,fill=pantone127-4] (model) at (2,3) {Model};
\node (solver) at (2,2) {Constraint Solver/Search};
\node[shape=rectangle,fill=pantone24-8] (hangs) at (1,1) {Hangs};
\node[shape=rectangle,fill=pantone157-8] (solution) at (2,1) {Solution};
\node[shape=rectangle,fill=pantone24-8] (wrong) at (3,1) {Wrong Solution};
\draw[-] (problem) -- (human);
\draw[->] (human) -- (model);
\draw[-] (model) -- (solver);
\draw[->] (solver) -- (hangs);
\draw[->] (solver) -- (solution);
\draw[->] (solver) -- (wrong);
\draw[->] (hangs) -- (1,4) -- (human);
\draw[->] (wrong) -- (3,4) -- (human);
\end{tikzpicture}
\end{center}
\end{frame}

\begin{frame}
\frametitle{Dual Role of Model}
\begin{itemize}
\item Allows Human to Express Problem
\begin{itemize}
\item Close to Problem Domain
\item Constraints as Abstractions
\end{itemize}
\item Allows Solver to Execute
\begin{itemize}
\item Variables as Communication Mechanism
\item Constraints as Algorithms
\end{itemize}
\end{itemize}
\end{frame}

\begin{frame}
\frametitle{Modelling Frameworks}
\begin{itemize}
\item \href{https://www.minizinc.org/}{MiniZinc} (NICTA, Monash University, Australia)
\item \href{https://github.com/eomahony/Numberjack}{NumberJack} (Insight, Ireland) 
\item \href{https://www-users.york.ac.uk/peter.nightingale/savilerow//}{EssencePrime/SavilleRow} (UK) 
\item \href{https://cpmpy.readthedocs.io/en/latest/}{CPMpy} (KU Leuven)
\item Allow use of multiple back-end solvers
\item Compile model into variants for each solver
\item A priori solver independent model(CP, MIP, SAT)
\end{itemize}
\end{frame} 

\begin{frame}
\frametitle{Framework Process}
\begin{center}
\begin{tikzpicture}[xscale=2]
\node[shape=rectangle,fill=pantone127-4] (problem) at (2.5,5) {Problem};
\node (human) at (2.5,4) {Human};
\node[shape=rectangle,fill=pantone127-4] (model) at (2.5,3) {Model};
\node (compile) at (2.5,2) {Compile/Reformulate};
\node (solver1) at (1,1) {CP};
\node (solver2) at (2,1) {MIP};
\node (solver3) at (3,1) {SAT};
\node (solver4) at (4,1) {Other};
\node[shape=rectangle,fill=pantone157-8] (solution1) at (1,0) {Solution};
\node[shape=rectangle,fill=pantone157-8] (solution2) at (2,0) {Solution};
\node[shape=rectangle,fill=pantone157-8] (solution3) at (3,0) {Solution};
\node[shape=rectangle,fill=pantone157-8] (solution4) at (4,0) {Solution};
\draw[-] (problem) -- (human);
\draw[->] (human) -- (model);
\draw[-] (model) -- (compile);
\draw[-] (compile) -- (solver1);
\draw[-] (compile) -- (solver2);
\draw[-] (compile) -- (solver3);
\draw[-] (compile) -- (solver4);
\draw[->] (solver1) -- (solution1);
\draw[->] (solver2) -- (solution2);
\draw[->] (solver3) -- (solution3);
\draw[->] (solver4) -- (solution4);
\end{tikzpicture}
\end{center}
\end{frame}

% \begin{frame}
  % \frametitle{Do It Now!}
  % \begin{itemize}
  % \item Download and install Minizinc
  % \item \url{https://www.minizinc.org/}
  % \end{itemize}
% \end{frame}


\begin{frame}
\frametitle{Why use Puzzles as Examples?}
\begin{itemize}
\item Easy to understand the problem
\item Solvable by hand without specialized knowledge
\item Possible to compare automated to manual solving process
\end{itemize}

\begin{quote}
The puzzle, though inanimate, is presented as a solvable problem without lasting negative consequences, a very low-risk low-reward situation. By being a puzzle, the object is attempting to convince the user that it must be completed.
\end{quote}
{\tiny Source: \href{https://www.everydayrhetoric.com/2020/03/the-rhetoric-of-puzzles/}{Every Day Rhetoric}}
\end{frame}


\mode<all>{
\part{Basic Constraint Propagation}
}
\begin{frame}
\partpage
\end{frame}
\mode<all>{
\input{../sendmore/body}
}

\mode<all>{
\part{Global Constraints}
}
\begin{frame}
\partpage
\end{frame}
\mode<all>{
\input{../sudoku/body}
}

\clearpage
\mode<all>{
\part{Customizing Search}
}
\begin{frame}
\partpage
\end{frame}
\mode<all>{
\input{../nqueen/body}
}


%% \clearpage
%% \mode<all>{
%%   \part{Problem Decomposition}
%% }
%% \begin{frame}
%%   \partpage
%% \end{frame}
%% \mode<all>{
%%   \input{../party/body}
%%   }

%% \clearpage
%% \mode<all>{
%%   \part{More Global Constraints}
%% }
%% \begin{frame}
%%   \partpage
%% \end{frame}
%% \mode<all>{
%%   \input{../car/body}
%%   }

%% \clearpage
%% \mode<all>{
%%   \part{Methodology}
%% }
%% \begin{frame}
%%   \partpage
%% \end{frame}
%% \mode<all>{
%%   \input{../methodology/body}
%%   }

\clearpage
\mode<all>{
\part{What is missing?}
}
\begin{frame}
\partpage
\end{frame}

\begin{frame}
\frametitle{Many Specialized Topics}
\begin{itemize}
\item How to design efficient core engine
\item Hybrids with LP/MIP tools
\item Hybrids with SAT
\item Symmetry breaking
\item Use of MDD/BDD to encode sets of solutions
\item High level modelling tools
\item Debugging/visualization
\end{itemize}
\end{frame}

\begin{frame}
\frametitle{Reformulation}
\begin{itemize}
\item Just because the user has modelled it this way, it doesn't mean we have to solve it that way
\begin{itemize}
\item Replace some constraint(s) by other, equivalent constraints
\item Because we don't have that constraint in our system
\item For performance
\end{itemize}
\end{itemize}
\end{frame}

\begin{frame}
\frametitle{Learning}
\begin{itemize}
\item While solving the problem we can learn how to strengthen the model/search 
\begin{itemize}
\item Understand which constraints/method contribute to propagation and change schedule
\item Learn no-good constraints by explaining failure
\item Adapt search strategy based on search experience
\end{itemize}
\end{itemize}
\end{frame}

%% \begin{frame}
%% \frametitle{Refined Process}
%% \begin{center}
%% \begin{tikzpicture}
%% \node[shape=rectangle,fill=pantone127-4] (problem) at (3,7) {Problem};
%% \node (human) at (3,6) {Human};
%% \node[shape=rectangle,fill=pantone127-4] (model) at (3,5) {User Model};
%% \node (reformulation) at (3,4) {Reformulation};
%% \node[shape=rectangle,fill=pantone127-4] (implementation) at (3,3) {Implementation Model};
%% \node (solver) at (3,2) {Constraint Solver/Search};
%% \node[shape=rectangle,fill=pantone157-8] (solution) at (3,1) {Solution};
%% \draw[-] (problem) -- (human);
%% \draw[->] (human) -- (model);
%% \draw[-] (model) -- (reformulation);
%% \draw[->] (reformulation) -- (implementation);
%% \draw[-] (implementation) -- (solver);
%% \draw[->] (solver) -- (0,2) -- (0,3) -- (implementation);
%% \draw[->] (solver) -- (solution);
%% \end{tikzpicture}
%% \end{center}
%% \end{frame}

%\begin{frame}
%\frametitle{What is CP actually used for?}
%\url{http://hsimonis.wordpress.com}
%\includegraphics[width=10cm]{../introduction/images/blog}
%\end{frame}

%% \begin{frame}
%% \frametitle{Exercises}
%% \begin{itemize}
%% \item Install Constraint Solver of your Choice
%% \item MiniZinc (MiningZinc) will be used in later sessions
%% \item Run Sendmoremoney, Sudoku, Queens examples
%% \item Can you choose which propagator to use for alldifferent?
%% \item Can you use different search strategies?
%% \end{itemize}
%% \end{frame}

\begin{frame}
\frametitle{More Learning Resources}
\begin{itemize}
\item Survey of Methods, Resources, and Formats for Teaching Constraint Programming
\begin{itemize}
\item by Tejas Santanam, Helmut Simonis
\item \url{https://doi.org/10.48550/arXiv.2403.12717}
\item Based on survey of community for WTCP 2023
\item \url{https://hsimonis.github.io/WTCP2023/}
\end{itemize}
\end{itemize}
\end{frame}


}


%% \clearpage
%% \mode<all>{
%%   \part{Problem Decomposition}
%% }
%% \begin{frame}
%%   \partpage
%% \end{frame}
%% \mode<all>{
%%   \begin{frame}
\frametitle{Objectives}
\begin{itemize}
\item Overview of Core Constraint Programming
\item Three Main Concepts
\begin{itemize}
\item Constraint Propagation
\item Global Constraints
\item Customizing Search
\end{itemize}
%\item Get Some Experience with MiniZinc
\item Topics will be treated in more detail in later parts of the school
\item Based on Examples, not Formal Description
\end{itemize}
\end{frame}

\begin{frame}
\frametitle{Outline}
\begin{itemize}
\item Why Constraint Programming?
\item Constraint Propagation
\item Global Constraints
\item Customizing Search
%\item Choices in Modelling
%\item Decomposition
%\item More Global Constraints
%\item Methodology
%\item What is missing?
\end{itemize}
\end{frame}

% \begin{frame}
  % \frametitle{Using MiniZinc IDE}
  % \begin{itemize}
  % \item Developed in the Australian NICTA project
  % \item Maintained by Monash University
  % \item Modelling tool with multiple back-end solvers
  % \item Available from \url{https://www.minizinc.org/}
  % \end{itemize}
% \end{frame}


% \begin{frame}
% \frametitle{Examples in ECLiPSe}
% \begin{itemize}
% \item Open sourced constraint programming language
% \item Development goes back to 1985
% \item ECRC, ICL, IC-Parc, PTL, Cisco
% \item \url{https://eclipseclp.org/}
% \item Specialities
% \begin{itemize}
% \item Develop new solvers for specific domains
% \item Integration with MIP
% \end{itemize}
% \item Not included in bundled MiniZinc IDE
% \item Specialized visualization tools used here
  % \begin{itemize}
  % \item CP-Viz, Simonis et al. 2010
    % \end{itemize}
% \end{itemize}
% \end{frame}

\begin{frame}
\frametitle{Tutorial Based on ECLiPSe ELearning Course}
\begin{itemize}
\item Self-study course in constraint programming
\item Supported by Cisco Systems and Silicon Valley Community Foundation
\item Multi-media format, video lectures, slides, handout etc
\item {\small \url{https://eclipseclp.org/ELearning/index.html}}
\end{itemize}
\end{frame}

\begin{frame}
\frametitle{Also Part of CRT-AI Constraint Week}
\begin{itemize}
\item Annual one week course on CP and Optimization in Ireland
\item Part of national training program for PhD students in AI
\item \url{https://www.crt-ai.ie/}
\end{itemize}
\end{frame}



\begin{frame}
\frametitle{Constraint Programming - in a nutshell}
\begin{itemize}
\item Declarative description of problems with
\begin{itemize}
\item {\em Variables} which range over (finite) sets of values
\item {\em Constraints} over subsets of variables which restrict possible value combinations
\item A {\em solution} is a value assignment which satisfies all constraints
\end{itemize}

\item Constraint propagation/reasoning
\begin{itemize}
\item Removing inconsistent values for variables
\item Detect failure if constraint can not be satisfied
\item Interaction of constraints via shared variables
\item Incomplete
\end{itemize}

\item Search
\begin{itemize}
\item User controlled assignment of values to variables
\item Each step triggers constraint propagation 
\end{itemize}
\item Different domains require/allow different methods
\end{itemize}
\end{frame}

\begin{frame}
  \frametitle{Constraint Programming is Different}
  \begin{itemize}
  \item Declarative Programming
    \begin{itemize}
    \item Concentrate on what you want
      \item Not how to get there
      \item Program != Algorithm
      \item Program = Model
    \end{itemize}
    \item Applied to Combinatorial Problems
      \begin{itemize}
        \item No complete polynomial algorithms known (exist?)
        \item CP less ad-hoc than heuristics
        \item Models can evolve
  \end{itemize}
  \end{itemize}
  \end{frame}
    
\begin{frame}
  \frametitle{A Subtractive Process}
  \begin{textblock}{4}(8,-3)
    \includegraphics[width=4cm]{../introduction/images/stages}
  \end{textblock}
  \vfill
  \begin{quote}
    ``Oh, bosh, as Mr. Ruskin says. Sculpture, per se, is the simplest thing in the world. All you have to do is to take a big chunk of marble and a hammer and chisel, make up your mind what you are about to create and chip off all the marble you don't want.''-Paris Gaulois.
  \end{quote}
  
  {\tiny Source: \url{https://quoteinvestigator.com/2014/06/22/chip-away/}}
\end{frame}

\begin{frame}
\frametitle{Basic Process}
\begin{center}
\begin{tikzpicture}
\node[shape=rectangle,fill=pantone127-4] (problem) at (2,5) {Problem};
\node (human) at (2,4) {Human};
\node[shape=rectangle,fill=pantone127-4] (model) at (2,3) {Model};
\node (solver) at (2,2) {Constraint Solver/Search};
\node[shape=rectangle,fill=pantone157-8] (solution) at (2,1) {Solution};
\draw[-] (problem) -- (human);
\draw[->] (human) -- (model);
\draw[-] (model) -- (solver);
\draw[->] (solver) -- (solution);
\end{tikzpicture}
\end{center}
\end{frame}

\begin{frame}
\frametitle{More Realistic}
\begin{center}
\begin{tikzpicture}[xscale=4.0]
\node[shape=rectangle,fill=pantone127-4] (problem) at (2,5) {Problem};
\node (human) at (2,4) {Human};
\node[shape=rectangle,fill=pantone127-4] (model) at (2,3) {Model};
\node (solver) at (2,2) {Constraint Solver/Search};
\node[shape=rectangle,fill=pantone24-8] (hangs) at (1,1) {Hangs};
\node[shape=rectangle,fill=pantone157-8] (solution) at (2,1) {Solution};
\node[shape=rectangle,fill=pantone24-8] (wrong) at (3,1) {Wrong Solution};
\draw[-] (problem) -- (human);
\draw[->] (human) -- (model);
\draw[-] (model) -- (solver);
\draw[->] (solver) -- (hangs);
\draw[->] (solver) -- (solution);
\draw[->] (solver) -- (wrong);
\draw[->] (hangs) -- (1,4) -- (human);
\draw[->] (wrong) -- (3,4) -- (human);
\end{tikzpicture}
\end{center}
\end{frame}

\begin{frame}
\frametitle{Dual Role of Model}
\begin{itemize}
\item Allows Human to Express Problem
\begin{itemize}
\item Close to Problem Domain
\item Constraints as Abstractions
\end{itemize}
\item Allows Solver to Execute
\begin{itemize}
\item Variables as Communication Mechanism
\item Constraints as Algorithms
\end{itemize}
\end{itemize}
\end{frame}

\begin{frame}
\frametitle{Modelling Frameworks}
\begin{itemize}
\item \href{https://www.minizinc.org/}{MiniZinc} (NICTA, Monash University, Australia)
\item \href{https://github.com/eomahony/Numberjack}{NumberJack} (Insight, Ireland) 
\item \href{https://www-users.york.ac.uk/peter.nightingale/savilerow//}{EssencePrime/SavilleRow} (UK) 
\item \href{https://cpmpy.readthedocs.io/en/latest/}{CPMpy} (KU Leuven)
\item Allow use of multiple back-end solvers
\item Compile model into variants for each solver
\item A priori solver independent model(CP, MIP, SAT)
\end{itemize}
\end{frame} 

\begin{frame}
\frametitle{Framework Process}
\begin{center}
\begin{tikzpicture}[xscale=2]
\node[shape=rectangle,fill=pantone127-4] (problem) at (2.5,5) {Problem};
\node (human) at (2.5,4) {Human};
\node[shape=rectangle,fill=pantone127-4] (model) at (2.5,3) {Model};
\node (compile) at (2.5,2) {Compile/Reformulate};
\node (solver1) at (1,1) {CP};
\node (solver2) at (2,1) {MIP};
\node (solver3) at (3,1) {SAT};
\node (solver4) at (4,1) {Other};
\node[shape=rectangle,fill=pantone157-8] (solution1) at (1,0) {Solution};
\node[shape=rectangle,fill=pantone157-8] (solution2) at (2,0) {Solution};
\node[shape=rectangle,fill=pantone157-8] (solution3) at (3,0) {Solution};
\node[shape=rectangle,fill=pantone157-8] (solution4) at (4,0) {Solution};
\draw[-] (problem) -- (human);
\draw[->] (human) -- (model);
\draw[-] (model) -- (compile);
\draw[-] (compile) -- (solver1);
\draw[-] (compile) -- (solver2);
\draw[-] (compile) -- (solver3);
\draw[-] (compile) -- (solver4);
\draw[->] (solver1) -- (solution1);
\draw[->] (solver2) -- (solution2);
\draw[->] (solver3) -- (solution3);
\draw[->] (solver4) -- (solution4);
\end{tikzpicture}
\end{center}
\end{frame}

% \begin{frame}
  % \frametitle{Do It Now!}
  % \begin{itemize}
  % \item Download and install Minizinc
  % \item \url{https://www.minizinc.org/}
  % \end{itemize}
% \end{frame}


\begin{frame}
\frametitle{Why use Puzzles as Examples?}
\begin{itemize}
\item Easy to understand the problem
\item Solvable by hand without specialized knowledge
\item Possible to compare automated to manual solving process
\end{itemize}

\begin{quote}
The puzzle, though inanimate, is presented as a solvable problem without lasting negative consequences, a very low-risk low-reward situation. By being a puzzle, the object is attempting to convince the user that it must be completed.
\end{quote}
{\tiny Source: \href{https://www.everydayrhetoric.com/2020/03/the-rhetoric-of-puzzles/}{Every Day Rhetoric}}
\end{frame}


\mode<all>{
\part{Basic Constraint Propagation}
}
\begin{frame}
\partpage
\end{frame}
\mode<all>{
\input{../sendmore/body}
}

\mode<all>{
\part{Global Constraints}
}
\begin{frame}
\partpage
\end{frame}
\mode<all>{
\input{../sudoku/body}
}

\clearpage
\mode<all>{
\part{Customizing Search}
}
\begin{frame}
\partpage
\end{frame}
\mode<all>{
\input{../nqueen/body}
}


%% \clearpage
%% \mode<all>{
%%   \part{Problem Decomposition}
%% }
%% \begin{frame}
%%   \partpage
%% \end{frame}
%% \mode<all>{
%%   \input{../party/body}
%%   }

%% \clearpage
%% \mode<all>{
%%   \part{More Global Constraints}
%% }
%% \begin{frame}
%%   \partpage
%% \end{frame}
%% \mode<all>{
%%   \input{../car/body}
%%   }

%% \clearpage
%% \mode<all>{
%%   \part{Methodology}
%% }
%% \begin{frame}
%%   \partpage
%% \end{frame}
%% \mode<all>{
%%   \input{../methodology/body}
%%   }

\clearpage
\mode<all>{
\part{What is missing?}
}
\begin{frame}
\partpage
\end{frame}

\begin{frame}
\frametitle{Many Specialized Topics}
\begin{itemize}
\item How to design efficient core engine
\item Hybrids with LP/MIP tools
\item Hybrids with SAT
\item Symmetry breaking
\item Use of MDD/BDD to encode sets of solutions
\item High level modelling tools
\item Debugging/visualization
\end{itemize}
\end{frame}

\begin{frame}
\frametitle{Reformulation}
\begin{itemize}
\item Just because the user has modelled it this way, it doesn't mean we have to solve it that way
\begin{itemize}
\item Replace some constraint(s) by other, equivalent constraints
\item Because we don't have that constraint in our system
\item For performance
\end{itemize}
\end{itemize}
\end{frame}

\begin{frame}
\frametitle{Learning}
\begin{itemize}
\item While solving the problem we can learn how to strengthen the model/search 
\begin{itemize}
\item Understand which constraints/method contribute to propagation and change schedule
\item Learn no-good constraints by explaining failure
\item Adapt search strategy based on search experience
\end{itemize}
\end{itemize}
\end{frame}

%% \begin{frame}
%% \frametitle{Refined Process}
%% \begin{center}
%% \begin{tikzpicture}
%% \node[shape=rectangle,fill=pantone127-4] (problem) at (3,7) {Problem};
%% \node (human) at (3,6) {Human};
%% \node[shape=rectangle,fill=pantone127-4] (model) at (3,5) {User Model};
%% \node (reformulation) at (3,4) {Reformulation};
%% \node[shape=rectangle,fill=pantone127-4] (implementation) at (3,3) {Implementation Model};
%% \node (solver) at (3,2) {Constraint Solver/Search};
%% \node[shape=rectangle,fill=pantone157-8] (solution) at (3,1) {Solution};
%% \draw[-] (problem) -- (human);
%% \draw[->] (human) -- (model);
%% \draw[-] (model) -- (reformulation);
%% \draw[->] (reformulation) -- (implementation);
%% \draw[-] (implementation) -- (solver);
%% \draw[->] (solver) -- (0,2) -- (0,3) -- (implementation);
%% \draw[->] (solver) -- (solution);
%% \end{tikzpicture}
%% \end{center}
%% \end{frame}

%\begin{frame}
%\frametitle{What is CP actually used for?}
%\url{http://hsimonis.wordpress.com}
%\includegraphics[width=10cm]{../introduction/images/blog}
%\end{frame}

%% \begin{frame}
%% \frametitle{Exercises}
%% \begin{itemize}
%% \item Install Constraint Solver of your Choice
%% \item MiniZinc (MiningZinc) will be used in later sessions
%% \item Run Sendmoremoney, Sudoku, Queens examples
%% \item Can you choose which propagator to use for alldifferent?
%% \item Can you use different search strategies?
%% \end{itemize}
%% \end{frame}

\begin{frame}
\frametitle{More Learning Resources}
\begin{itemize}
\item Survey of Methods, Resources, and Formats for Teaching Constraint Programming
\begin{itemize}
\item by Tejas Santanam, Helmut Simonis
\item \url{https://doi.org/10.48550/arXiv.2403.12717}
\item Based on survey of community for WTCP 2023
\item \url{https://hsimonis.github.io/WTCP2023/}
\end{itemize}
\end{itemize}
\end{frame}


%%   }

%% \clearpage
%% \mode<all>{
%%   \part{More Global Constraints}
%% }
%% \begin{frame}
%%   \partpage
%% \end{frame}
%% \mode<all>{
%%   \begin{frame}
\frametitle{Objectives}
\begin{itemize}
\item Overview of Core Constraint Programming
\item Three Main Concepts
\begin{itemize}
\item Constraint Propagation
\item Global Constraints
\item Customizing Search
\end{itemize}
%\item Get Some Experience with MiniZinc
\item Topics will be treated in more detail in later parts of the school
\item Based on Examples, not Formal Description
\end{itemize}
\end{frame}

\begin{frame}
\frametitle{Outline}
\begin{itemize}
\item Why Constraint Programming?
\item Constraint Propagation
\item Global Constraints
\item Customizing Search
%\item Choices in Modelling
%\item Decomposition
%\item More Global Constraints
%\item Methodology
%\item What is missing?
\end{itemize}
\end{frame}

% \begin{frame}
  % \frametitle{Using MiniZinc IDE}
  % \begin{itemize}
  % \item Developed in the Australian NICTA project
  % \item Maintained by Monash University
  % \item Modelling tool with multiple back-end solvers
  % \item Available from \url{https://www.minizinc.org/}
  % \end{itemize}
% \end{frame}


% \begin{frame}
% \frametitle{Examples in ECLiPSe}
% \begin{itemize}
% \item Open sourced constraint programming language
% \item Development goes back to 1985
% \item ECRC, ICL, IC-Parc, PTL, Cisco
% \item \url{https://eclipseclp.org/}
% \item Specialities
% \begin{itemize}
% \item Develop new solvers for specific domains
% \item Integration with MIP
% \end{itemize}
% \item Not included in bundled MiniZinc IDE
% \item Specialized visualization tools used here
  % \begin{itemize}
  % \item CP-Viz, Simonis et al. 2010
    % \end{itemize}
% \end{itemize}
% \end{frame}

\begin{frame}
\frametitle{Tutorial Based on ECLiPSe ELearning Course}
\begin{itemize}
\item Self-study course in constraint programming
\item Supported by Cisco Systems and Silicon Valley Community Foundation
\item Multi-media format, video lectures, slides, handout etc
\item {\small \url{https://eclipseclp.org/ELearning/index.html}}
\end{itemize}
\end{frame}

\begin{frame}
\frametitle{Also Part of CRT-AI Constraint Week}
\begin{itemize}
\item Annual one week course on CP and Optimization in Ireland
\item Part of national training program for PhD students in AI
\item \url{https://www.crt-ai.ie/}
\end{itemize}
\end{frame}



\begin{frame}
\frametitle{Constraint Programming - in a nutshell}
\begin{itemize}
\item Declarative description of problems with
\begin{itemize}
\item {\em Variables} which range over (finite) sets of values
\item {\em Constraints} over subsets of variables which restrict possible value combinations
\item A {\em solution} is a value assignment which satisfies all constraints
\end{itemize}

\item Constraint propagation/reasoning
\begin{itemize}
\item Removing inconsistent values for variables
\item Detect failure if constraint can not be satisfied
\item Interaction of constraints via shared variables
\item Incomplete
\end{itemize}

\item Search
\begin{itemize}
\item User controlled assignment of values to variables
\item Each step triggers constraint propagation 
\end{itemize}
\item Different domains require/allow different methods
\end{itemize}
\end{frame}

\begin{frame}
  \frametitle{Constraint Programming is Different}
  \begin{itemize}
  \item Declarative Programming
    \begin{itemize}
    \item Concentrate on what you want
      \item Not how to get there
      \item Program != Algorithm
      \item Program = Model
    \end{itemize}
    \item Applied to Combinatorial Problems
      \begin{itemize}
        \item No complete polynomial algorithms known (exist?)
        \item CP less ad-hoc than heuristics
        \item Models can evolve
  \end{itemize}
  \end{itemize}
  \end{frame}
    
\begin{frame}
  \frametitle{A Subtractive Process}
  \begin{textblock}{4}(8,-3)
    \includegraphics[width=4cm]{../introduction/images/stages}
  \end{textblock}
  \vfill
  \begin{quote}
    ``Oh, bosh, as Mr. Ruskin says. Sculpture, per se, is the simplest thing in the world. All you have to do is to take a big chunk of marble and a hammer and chisel, make up your mind what you are about to create and chip off all the marble you don't want.''-Paris Gaulois.
  \end{quote}
  
  {\tiny Source: \url{https://quoteinvestigator.com/2014/06/22/chip-away/}}
\end{frame}

\begin{frame}
\frametitle{Basic Process}
\begin{center}
\begin{tikzpicture}
\node[shape=rectangle,fill=pantone127-4] (problem) at (2,5) {Problem};
\node (human) at (2,4) {Human};
\node[shape=rectangle,fill=pantone127-4] (model) at (2,3) {Model};
\node (solver) at (2,2) {Constraint Solver/Search};
\node[shape=rectangle,fill=pantone157-8] (solution) at (2,1) {Solution};
\draw[-] (problem) -- (human);
\draw[->] (human) -- (model);
\draw[-] (model) -- (solver);
\draw[->] (solver) -- (solution);
\end{tikzpicture}
\end{center}
\end{frame}

\begin{frame}
\frametitle{More Realistic}
\begin{center}
\begin{tikzpicture}[xscale=4.0]
\node[shape=rectangle,fill=pantone127-4] (problem) at (2,5) {Problem};
\node (human) at (2,4) {Human};
\node[shape=rectangle,fill=pantone127-4] (model) at (2,3) {Model};
\node (solver) at (2,2) {Constraint Solver/Search};
\node[shape=rectangle,fill=pantone24-8] (hangs) at (1,1) {Hangs};
\node[shape=rectangle,fill=pantone157-8] (solution) at (2,1) {Solution};
\node[shape=rectangle,fill=pantone24-8] (wrong) at (3,1) {Wrong Solution};
\draw[-] (problem) -- (human);
\draw[->] (human) -- (model);
\draw[-] (model) -- (solver);
\draw[->] (solver) -- (hangs);
\draw[->] (solver) -- (solution);
\draw[->] (solver) -- (wrong);
\draw[->] (hangs) -- (1,4) -- (human);
\draw[->] (wrong) -- (3,4) -- (human);
\end{tikzpicture}
\end{center}
\end{frame}

\begin{frame}
\frametitle{Dual Role of Model}
\begin{itemize}
\item Allows Human to Express Problem
\begin{itemize}
\item Close to Problem Domain
\item Constraints as Abstractions
\end{itemize}
\item Allows Solver to Execute
\begin{itemize}
\item Variables as Communication Mechanism
\item Constraints as Algorithms
\end{itemize}
\end{itemize}
\end{frame}

\begin{frame}
\frametitle{Modelling Frameworks}
\begin{itemize}
\item \href{https://www.minizinc.org/}{MiniZinc} (NICTA, Monash University, Australia)
\item \href{https://github.com/eomahony/Numberjack}{NumberJack} (Insight, Ireland) 
\item \href{https://www-users.york.ac.uk/peter.nightingale/savilerow//}{EssencePrime/SavilleRow} (UK) 
\item \href{https://cpmpy.readthedocs.io/en/latest/}{CPMpy} (KU Leuven)
\item Allow use of multiple back-end solvers
\item Compile model into variants for each solver
\item A priori solver independent model(CP, MIP, SAT)
\end{itemize}
\end{frame} 

\begin{frame}
\frametitle{Framework Process}
\begin{center}
\begin{tikzpicture}[xscale=2]
\node[shape=rectangle,fill=pantone127-4] (problem) at (2.5,5) {Problem};
\node (human) at (2.5,4) {Human};
\node[shape=rectangle,fill=pantone127-4] (model) at (2.5,3) {Model};
\node (compile) at (2.5,2) {Compile/Reformulate};
\node (solver1) at (1,1) {CP};
\node (solver2) at (2,1) {MIP};
\node (solver3) at (3,1) {SAT};
\node (solver4) at (4,1) {Other};
\node[shape=rectangle,fill=pantone157-8] (solution1) at (1,0) {Solution};
\node[shape=rectangle,fill=pantone157-8] (solution2) at (2,0) {Solution};
\node[shape=rectangle,fill=pantone157-8] (solution3) at (3,0) {Solution};
\node[shape=rectangle,fill=pantone157-8] (solution4) at (4,0) {Solution};
\draw[-] (problem) -- (human);
\draw[->] (human) -- (model);
\draw[-] (model) -- (compile);
\draw[-] (compile) -- (solver1);
\draw[-] (compile) -- (solver2);
\draw[-] (compile) -- (solver3);
\draw[-] (compile) -- (solver4);
\draw[->] (solver1) -- (solution1);
\draw[->] (solver2) -- (solution2);
\draw[->] (solver3) -- (solution3);
\draw[->] (solver4) -- (solution4);
\end{tikzpicture}
\end{center}
\end{frame}

% \begin{frame}
  % \frametitle{Do It Now!}
  % \begin{itemize}
  % \item Download and install Minizinc
  % \item \url{https://www.minizinc.org/}
  % \end{itemize}
% \end{frame}


\begin{frame}
\frametitle{Why use Puzzles as Examples?}
\begin{itemize}
\item Easy to understand the problem
\item Solvable by hand without specialized knowledge
\item Possible to compare automated to manual solving process
\end{itemize}

\begin{quote}
The puzzle, though inanimate, is presented as a solvable problem without lasting negative consequences, a very low-risk low-reward situation. By being a puzzle, the object is attempting to convince the user that it must be completed.
\end{quote}
{\tiny Source: \href{https://www.everydayrhetoric.com/2020/03/the-rhetoric-of-puzzles/}{Every Day Rhetoric}}
\end{frame}


\mode<all>{
\part{Basic Constraint Propagation}
}
\begin{frame}
\partpage
\end{frame}
\mode<all>{
\input{../sendmore/body}
}

\mode<all>{
\part{Global Constraints}
}
\begin{frame}
\partpage
\end{frame}
\mode<all>{
\input{../sudoku/body}
}

\clearpage
\mode<all>{
\part{Customizing Search}
}
\begin{frame}
\partpage
\end{frame}
\mode<all>{
\input{../nqueen/body}
}


%% \clearpage
%% \mode<all>{
%%   \part{Problem Decomposition}
%% }
%% \begin{frame}
%%   \partpage
%% \end{frame}
%% \mode<all>{
%%   \input{../party/body}
%%   }

%% \clearpage
%% \mode<all>{
%%   \part{More Global Constraints}
%% }
%% \begin{frame}
%%   \partpage
%% \end{frame}
%% \mode<all>{
%%   \input{../car/body}
%%   }

%% \clearpage
%% \mode<all>{
%%   \part{Methodology}
%% }
%% \begin{frame}
%%   \partpage
%% \end{frame}
%% \mode<all>{
%%   \input{../methodology/body}
%%   }

\clearpage
\mode<all>{
\part{What is missing?}
}
\begin{frame}
\partpage
\end{frame}

\begin{frame}
\frametitle{Many Specialized Topics}
\begin{itemize}
\item How to design efficient core engine
\item Hybrids with LP/MIP tools
\item Hybrids with SAT
\item Symmetry breaking
\item Use of MDD/BDD to encode sets of solutions
\item High level modelling tools
\item Debugging/visualization
\end{itemize}
\end{frame}

\begin{frame}
\frametitle{Reformulation}
\begin{itemize}
\item Just because the user has modelled it this way, it doesn't mean we have to solve it that way
\begin{itemize}
\item Replace some constraint(s) by other, equivalent constraints
\item Because we don't have that constraint in our system
\item For performance
\end{itemize}
\end{itemize}
\end{frame}

\begin{frame}
\frametitle{Learning}
\begin{itemize}
\item While solving the problem we can learn how to strengthen the model/search 
\begin{itemize}
\item Understand which constraints/method contribute to propagation and change schedule
\item Learn no-good constraints by explaining failure
\item Adapt search strategy based on search experience
\end{itemize}
\end{itemize}
\end{frame}

%% \begin{frame}
%% \frametitle{Refined Process}
%% \begin{center}
%% \begin{tikzpicture}
%% \node[shape=rectangle,fill=pantone127-4] (problem) at (3,7) {Problem};
%% \node (human) at (3,6) {Human};
%% \node[shape=rectangle,fill=pantone127-4] (model) at (3,5) {User Model};
%% \node (reformulation) at (3,4) {Reformulation};
%% \node[shape=rectangle,fill=pantone127-4] (implementation) at (3,3) {Implementation Model};
%% \node (solver) at (3,2) {Constraint Solver/Search};
%% \node[shape=rectangle,fill=pantone157-8] (solution) at (3,1) {Solution};
%% \draw[-] (problem) -- (human);
%% \draw[->] (human) -- (model);
%% \draw[-] (model) -- (reformulation);
%% \draw[->] (reformulation) -- (implementation);
%% \draw[-] (implementation) -- (solver);
%% \draw[->] (solver) -- (0,2) -- (0,3) -- (implementation);
%% \draw[->] (solver) -- (solution);
%% \end{tikzpicture}
%% \end{center}
%% \end{frame}

%\begin{frame}
%\frametitle{What is CP actually used for?}
%\url{http://hsimonis.wordpress.com}
%\includegraphics[width=10cm]{../introduction/images/blog}
%\end{frame}

%% \begin{frame}
%% \frametitle{Exercises}
%% \begin{itemize}
%% \item Install Constraint Solver of your Choice
%% \item MiniZinc (MiningZinc) will be used in later sessions
%% \item Run Sendmoremoney, Sudoku, Queens examples
%% \item Can you choose which propagator to use for alldifferent?
%% \item Can you use different search strategies?
%% \end{itemize}
%% \end{frame}

\begin{frame}
\frametitle{More Learning Resources}
\begin{itemize}
\item Survey of Methods, Resources, and Formats for Teaching Constraint Programming
\begin{itemize}
\item by Tejas Santanam, Helmut Simonis
\item \url{https://doi.org/10.48550/arXiv.2403.12717}
\item Based on survey of community for WTCP 2023
\item \url{https://hsimonis.github.io/WTCP2023/}
\end{itemize}
\end{itemize}
\end{frame}


%%   }

%% \clearpage
%% \mode<all>{
%%   \part{Methodology}
%% }
%% \begin{frame}
%%   \partpage
%% \end{frame}
%% \mode<all>{
%%   \begin{frame}
\frametitle{Objectives}
\begin{itemize}
\item Overview of Core Constraint Programming
\item Three Main Concepts
\begin{itemize}
\item Constraint Propagation
\item Global Constraints
\item Customizing Search
\end{itemize}
%\item Get Some Experience with MiniZinc
\item Topics will be treated in more detail in later parts of the school
\item Based on Examples, not Formal Description
\end{itemize}
\end{frame}

\begin{frame}
\frametitle{Outline}
\begin{itemize}
\item Why Constraint Programming?
\item Constraint Propagation
\item Global Constraints
\item Customizing Search
%\item Choices in Modelling
%\item Decomposition
%\item More Global Constraints
%\item Methodology
%\item What is missing?
\end{itemize}
\end{frame}

% \begin{frame}
  % \frametitle{Using MiniZinc IDE}
  % \begin{itemize}
  % \item Developed in the Australian NICTA project
  % \item Maintained by Monash University
  % \item Modelling tool with multiple back-end solvers
  % \item Available from \url{https://www.minizinc.org/}
  % \end{itemize}
% \end{frame}


% \begin{frame}
% \frametitle{Examples in ECLiPSe}
% \begin{itemize}
% \item Open sourced constraint programming language
% \item Development goes back to 1985
% \item ECRC, ICL, IC-Parc, PTL, Cisco
% \item \url{https://eclipseclp.org/}
% \item Specialities
% \begin{itemize}
% \item Develop new solvers for specific domains
% \item Integration with MIP
% \end{itemize}
% \item Not included in bundled MiniZinc IDE
% \item Specialized visualization tools used here
  % \begin{itemize}
  % \item CP-Viz, Simonis et al. 2010
    % \end{itemize}
% \end{itemize}
% \end{frame}

\begin{frame}
\frametitle{Tutorial Based on ECLiPSe ELearning Course}
\begin{itemize}
\item Self-study course in constraint programming
\item Supported by Cisco Systems and Silicon Valley Community Foundation
\item Multi-media format, video lectures, slides, handout etc
\item {\small \url{https://eclipseclp.org/ELearning/index.html}}
\end{itemize}
\end{frame}

\begin{frame}
\frametitle{Also Part of CRT-AI Constraint Week}
\begin{itemize}
\item Annual one week course on CP and Optimization in Ireland
\item Part of national training program for PhD students in AI
\item \url{https://www.crt-ai.ie/}
\end{itemize}
\end{frame}



\begin{frame}
\frametitle{Constraint Programming - in a nutshell}
\begin{itemize}
\item Declarative description of problems with
\begin{itemize}
\item {\em Variables} which range over (finite) sets of values
\item {\em Constraints} over subsets of variables which restrict possible value combinations
\item A {\em solution} is a value assignment which satisfies all constraints
\end{itemize}

\item Constraint propagation/reasoning
\begin{itemize}
\item Removing inconsistent values for variables
\item Detect failure if constraint can not be satisfied
\item Interaction of constraints via shared variables
\item Incomplete
\end{itemize}

\item Search
\begin{itemize}
\item User controlled assignment of values to variables
\item Each step triggers constraint propagation 
\end{itemize}
\item Different domains require/allow different methods
\end{itemize}
\end{frame}

\begin{frame}
  \frametitle{Constraint Programming is Different}
  \begin{itemize}
  \item Declarative Programming
    \begin{itemize}
    \item Concentrate on what you want
      \item Not how to get there
      \item Program != Algorithm
      \item Program = Model
    \end{itemize}
    \item Applied to Combinatorial Problems
      \begin{itemize}
        \item No complete polynomial algorithms known (exist?)
        \item CP less ad-hoc than heuristics
        \item Models can evolve
  \end{itemize}
  \end{itemize}
  \end{frame}
    
\begin{frame}
  \frametitle{A Subtractive Process}
  \begin{textblock}{4}(8,-3)
    \includegraphics[width=4cm]{../introduction/images/stages}
  \end{textblock}
  \vfill
  \begin{quote}
    ``Oh, bosh, as Mr. Ruskin says. Sculpture, per se, is the simplest thing in the world. All you have to do is to take a big chunk of marble and a hammer and chisel, make up your mind what you are about to create and chip off all the marble you don't want.''-Paris Gaulois.
  \end{quote}
  
  {\tiny Source: \url{https://quoteinvestigator.com/2014/06/22/chip-away/}}
\end{frame}

\begin{frame}
\frametitle{Basic Process}
\begin{center}
\begin{tikzpicture}
\node[shape=rectangle,fill=pantone127-4] (problem) at (2,5) {Problem};
\node (human) at (2,4) {Human};
\node[shape=rectangle,fill=pantone127-4] (model) at (2,3) {Model};
\node (solver) at (2,2) {Constraint Solver/Search};
\node[shape=rectangle,fill=pantone157-8] (solution) at (2,1) {Solution};
\draw[-] (problem) -- (human);
\draw[->] (human) -- (model);
\draw[-] (model) -- (solver);
\draw[->] (solver) -- (solution);
\end{tikzpicture}
\end{center}
\end{frame}

\begin{frame}
\frametitle{More Realistic}
\begin{center}
\begin{tikzpicture}[xscale=4.0]
\node[shape=rectangle,fill=pantone127-4] (problem) at (2,5) {Problem};
\node (human) at (2,4) {Human};
\node[shape=rectangle,fill=pantone127-4] (model) at (2,3) {Model};
\node (solver) at (2,2) {Constraint Solver/Search};
\node[shape=rectangle,fill=pantone24-8] (hangs) at (1,1) {Hangs};
\node[shape=rectangle,fill=pantone157-8] (solution) at (2,1) {Solution};
\node[shape=rectangle,fill=pantone24-8] (wrong) at (3,1) {Wrong Solution};
\draw[-] (problem) -- (human);
\draw[->] (human) -- (model);
\draw[-] (model) -- (solver);
\draw[->] (solver) -- (hangs);
\draw[->] (solver) -- (solution);
\draw[->] (solver) -- (wrong);
\draw[->] (hangs) -- (1,4) -- (human);
\draw[->] (wrong) -- (3,4) -- (human);
\end{tikzpicture}
\end{center}
\end{frame}

\begin{frame}
\frametitle{Dual Role of Model}
\begin{itemize}
\item Allows Human to Express Problem
\begin{itemize}
\item Close to Problem Domain
\item Constraints as Abstractions
\end{itemize}
\item Allows Solver to Execute
\begin{itemize}
\item Variables as Communication Mechanism
\item Constraints as Algorithms
\end{itemize}
\end{itemize}
\end{frame}

\begin{frame}
\frametitle{Modelling Frameworks}
\begin{itemize}
\item \href{https://www.minizinc.org/}{MiniZinc} (NICTA, Monash University, Australia)
\item \href{https://github.com/eomahony/Numberjack}{NumberJack} (Insight, Ireland) 
\item \href{https://www-users.york.ac.uk/peter.nightingale/savilerow//}{EssencePrime/SavilleRow} (UK) 
\item \href{https://cpmpy.readthedocs.io/en/latest/}{CPMpy} (KU Leuven)
\item Allow use of multiple back-end solvers
\item Compile model into variants for each solver
\item A priori solver independent model(CP, MIP, SAT)
\end{itemize}
\end{frame} 

\begin{frame}
\frametitle{Framework Process}
\begin{center}
\begin{tikzpicture}[xscale=2]
\node[shape=rectangle,fill=pantone127-4] (problem) at (2.5,5) {Problem};
\node (human) at (2.5,4) {Human};
\node[shape=rectangle,fill=pantone127-4] (model) at (2.5,3) {Model};
\node (compile) at (2.5,2) {Compile/Reformulate};
\node (solver1) at (1,1) {CP};
\node (solver2) at (2,1) {MIP};
\node (solver3) at (3,1) {SAT};
\node (solver4) at (4,1) {Other};
\node[shape=rectangle,fill=pantone157-8] (solution1) at (1,0) {Solution};
\node[shape=rectangle,fill=pantone157-8] (solution2) at (2,0) {Solution};
\node[shape=rectangle,fill=pantone157-8] (solution3) at (3,0) {Solution};
\node[shape=rectangle,fill=pantone157-8] (solution4) at (4,0) {Solution};
\draw[-] (problem) -- (human);
\draw[->] (human) -- (model);
\draw[-] (model) -- (compile);
\draw[-] (compile) -- (solver1);
\draw[-] (compile) -- (solver2);
\draw[-] (compile) -- (solver3);
\draw[-] (compile) -- (solver4);
\draw[->] (solver1) -- (solution1);
\draw[->] (solver2) -- (solution2);
\draw[->] (solver3) -- (solution3);
\draw[->] (solver4) -- (solution4);
\end{tikzpicture}
\end{center}
\end{frame}

% \begin{frame}
  % \frametitle{Do It Now!}
  % \begin{itemize}
  % \item Download and install Minizinc
  % \item \url{https://www.minizinc.org/}
  % \end{itemize}
% \end{frame}


\begin{frame}
\frametitle{Why use Puzzles as Examples?}
\begin{itemize}
\item Easy to understand the problem
\item Solvable by hand without specialized knowledge
\item Possible to compare automated to manual solving process
\end{itemize}

\begin{quote}
The puzzle, though inanimate, is presented as a solvable problem without lasting negative consequences, a very low-risk low-reward situation. By being a puzzle, the object is attempting to convince the user that it must be completed.
\end{quote}
{\tiny Source: \href{https://www.everydayrhetoric.com/2020/03/the-rhetoric-of-puzzles/}{Every Day Rhetoric}}
\end{frame}


\mode<all>{
\part{Basic Constraint Propagation}
}
\begin{frame}
\partpage
\end{frame}
\mode<all>{
\input{../sendmore/body}
}

\mode<all>{
\part{Global Constraints}
}
\begin{frame}
\partpage
\end{frame}
\mode<all>{
\input{../sudoku/body}
}

\clearpage
\mode<all>{
\part{Customizing Search}
}
\begin{frame}
\partpage
\end{frame}
\mode<all>{
\input{../nqueen/body}
}


%% \clearpage
%% \mode<all>{
%%   \part{Problem Decomposition}
%% }
%% \begin{frame}
%%   \partpage
%% \end{frame}
%% \mode<all>{
%%   \input{../party/body}
%%   }

%% \clearpage
%% \mode<all>{
%%   \part{More Global Constraints}
%% }
%% \begin{frame}
%%   \partpage
%% \end{frame}
%% \mode<all>{
%%   \input{../car/body}
%%   }

%% \clearpage
%% \mode<all>{
%%   \part{Methodology}
%% }
%% \begin{frame}
%%   \partpage
%% \end{frame}
%% \mode<all>{
%%   \input{../methodology/body}
%%   }

\clearpage
\mode<all>{
\part{What is missing?}
}
\begin{frame}
\partpage
\end{frame}

\begin{frame}
\frametitle{Many Specialized Topics}
\begin{itemize}
\item How to design efficient core engine
\item Hybrids with LP/MIP tools
\item Hybrids with SAT
\item Symmetry breaking
\item Use of MDD/BDD to encode sets of solutions
\item High level modelling tools
\item Debugging/visualization
\end{itemize}
\end{frame}

\begin{frame}
\frametitle{Reformulation}
\begin{itemize}
\item Just because the user has modelled it this way, it doesn't mean we have to solve it that way
\begin{itemize}
\item Replace some constraint(s) by other, equivalent constraints
\item Because we don't have that constraint in our system
\item For performance
\end{itemize}
\end{itemize}
\end{frame}

\begin{frame}
\frametitle{Learning}
\begin{itemize}
\item While solving the problem we can learn how to strengthen the model/search 
\begin{itemize}
\item Understand which constraints/method contribute to propagation and change schedule
\item Learn no-good constraints by explaining failure
\item Adapt search strategy based on search experience
\end{itemize}
\end{itemize}
\end{frame}

%% \begin{frame}
%% \frametitle{Refined Process}
%% \begin{center}
%% \begin{tikzpicture}
%% \node[shape=rectangle,fill=pantone127-4] (problem) at (3,7) {Problem};
%% \node (human) at (3,6) {Human};
%% \node[shape=rectangle,fill=pantone127-4] (model) at (3,5) {User Model};
%% \node (reformulation) at (3,4) {Reformulation};
%% \node[shape=rectangle,fill=pantone127-4] (implementation) at (3,3) {Implementation Model};
%% \node (solver) at (3,2) {Constraint Solver/Search};
%% \node[shape=rectangle,fill=pantone157-8] (solution) at (3,1) {Solution};
%% \draw[-] (problem) -- (human);
%% \draw[->] (human) -- (model);
%% \draw[-] (model) -- (reformulation);
%% \draw[->] (reformulation) -- (implementation);
%% \draw[-] (implementation) -- (solver);
%% \draw[->] (solver) -- (0,2) -- (0,3) -- (implementation);
%% \draw[->] (solver) -- (solution);
%% \end{tikzpicture}
%% \end{center}
%% \end{frame}

%\begin{frame}
%\frametitle{What is CP actually used for?}
%\url{http://hsimonis.wordpress.com}
%\includegraphics[width=10cm]{../introduction/images/blog}
%\end{frame}

%% \begin{frame}
%% \frametitle{Exercises}
%% \begin{itemize}
%% \item Install Constraint Solver of your Choice
%% \item MiniZinc (MiningZinc) will be used in later sessions
%% \item Run Sendmoremoney, Sudoku, Queens examples
%% \item Can you choose which propagator to use for alldifferent?
%% \item Can you use different search strategies?
%% \end{itemize}
%% \end{frame}

\begin{frame}
\frametitle{More Learning Resources}
\begin{itemize}
\item Survey of Methods, Resources, and Formats for Teaching Constraint Programming
\begin{itemize}
\item by Tejas Santanam, Helmut Simonis
\item \url{https://doi.org/10.48550/arXiv.2403.12717}
\item Based on survey of community for WTCP 2023
\item \url{https://hsimonis.github.io/WTCP2023/}
\end{itemize}
\end{itemize}
\end{frame}


%%   }

\clearpage
\mode<all>{
\part{What is missing?}
}
\begin{frame}
\partpage
\end{frame}

\begin{frame}
\frametitle{Many Specialized Topics}
\begin{itemize}
\item How to design efficient core engine
\item Hybrids with LP/MIP tools
\item Hybrids with SAT
\item Symmetry breaking
\item Use of MDD/BDD to encode sets of solutions
\item High level modelling tools
\item Debugging/visualization
\end{itemize}
\end{frame}

\begin{frame}
\frametitle{Reformulation}
\begin{itemize}
\item Just because the user has modelled it this way, it doesn't mean we have to solve it that way
\begin{itemize}
\item Replace some constraint(s) by other, equivalent constraints
\item Because we don't have that constraint in our system
\item For performance
\end{itemize}
\end{itemize}
\end{frame}

\begin{frame}
\frametitle{Learning}
\begin{itemize}
\item While solving the problem we can learn how to strengthen the model/search 
\begin{itemize}
\item Understand which constraints/method contribute to propagation and change schedule
\item Learn no-good constraints by explaining failure
\item Adapt search strategy based on search experience
\end{itemize}
\end{itemize}
\end{frame}

%% \begin{frame}
%% \frametitle{Refined Process}
%% \begin{center}
%% \begin{tikzpicture}
%% \node[shape=rectangle,fill=pantone127-4] (problem) at (3,7) {Problem};
%% \node (human) at (3,6) {Human};
%% \node[shape=rectangle,fill=pantone127-4] (model) at (3,5) {User Model};
%% \node (reformulation) at (3,4) {Reformulation};
%% \node[shape=rectangle,fill=pantone127-4] (implementation) at (3,3) {Implementation Model};
%% \node (solver) at (3,2) {Constraint Solver/Search};
%% \node[shape=rectangle,fill=pantone157-8] (solution) at (3,1) {Solution};
%% \draw[-] (problem) -- (human);
%% \draw[->] (human) -- (model);
%% \draw[-] (model) -- (reformulation);
%% \draw[->] (reformulation) -- (implementation);
%% \draw[-] (implementation) -- (solver);
%% \draw[->] (solver) -- (0,2) -- (0,3) -- (implementation);
%% \draw[->] (solver) -- (solution);
%% \end{tikzpicture}
%% \end{center}
%% \end{frame}

%\begin{frame}
%\frametitle{What is CP actually used for?}
%\url{http://hsimonis.wordpress.com}
%\includegraphics[width=10cm]{../introduction/images/blog}
%\end{frame}

%% \begin{frame}
%% \frametitle{Exercises}
%% \begin{itemize}
%% \item Install Constraint Solver of your Choice
%% \item MiniZinc (MiningZinc) will be used in later sessions
%% \item Run Sendmoremoney, Sudoku, Queens examples
%% \item Can you choose which propagator to use for alldifferent?
%% \item Can you use different search strategies?
%% \end{itemize}
%% \end{frame}

\begin{frame}
\frametitle{More Learning Resources}
\begin{itemize}
\item Survey of Methods, Resources, and Formats for Teaching Constraint Programming
\begin{itemize}
\item by Tejas Santanam, Helmut Simonis
\item \url{https://doi.org/10.48550/arXiv.2403.12717}
\item Based on survey of community for WTCP 2023
\item \url{https://hsimonis.github.io/WTCP2023/}
\end{itemize}
\end{itemize}
\end{frame}


}

\mode<all>{
\part{Global Constraints}
}
\begin{frame}
\partpage
\end{frame}
\mode<all>{
\begin{frame}
\frametitle{Objectives}
\begin{itemize}
\item Overview of Core Constraint Programming
\item Three Main Concepts
\begin{itemize}
\item Constraint Propagation
\item Global Constraints
\item Customizing Search
\end{itemize}
%\item Get Some Experience with MiniZinc
\item Topics will be treated in more detail in later parts of the school
\item Based on Examples, not Formal Description
\end{itemize}
\end{frame}

\begin{frame}
\frametitle{Outline}
\begin{itemize}
\item Why Constraint Programming?
\item Constraint Propagation
\item Global Constraints
\item Customizing Search
%\item Choices in Modelling
%\item Decomposition
%\item More Global Constraints
%\item Methodology
%\item What is missing?
\end{itemize}
\end{frame}

% \begin{frame}
  % \frametitle{Using MiniZinc IDE}
  % \begin{itemize}
  % \item Developed in the Australian NICTA project
  % \item Maintained by Monash University
  % \item Modelling tool with multiple back-end solvers
  % \item Available from \url{https://www.minizinc.org/}
  % \end{itemize}
% \end{frame}


% \begin{frame}
% \frametitle{Examples in ECLiPSe}
% \begin{itemize}
% \item Open sourced constraint programming language
% \item Development goes back to 1985
% \item ECRC, ICL, IC-Parc, PTL, Cisco
% \item \url{https://eclipseclp.org/}
% \item Specialities
% \begin{itemize}
% \item Develop new solvers for specific domains
% \item Integration with MIP
% \end{itemize}
% \item Not included in bundled MiniZinc IDE
% \item Specialized visualization tools used here
  % \begin{itemize}
  % \item CP-Viz, Simonis et al. 2010
    % \end{itemize}
% \end{itemize}
% \end{frame}

\begin{frame}
\frametitle{Tutorial Based on ECLiPSe ELearning Course}
\begin{itemize}
\item Self-study course in constraint programming
\item Supported by Cisco Systems and Silicon Valley Community Foundation
\item Multi-media format, video lectures, slides, handout etc
\item {\small \url{https://eclipseclp.org/ELearning/index.html}}
\end{itemize}
\end{frame}

\begin{frame}
\frametitle{Also Part of CRT-AI Constraint Week}
\begin{itemize}
\item Annual one week course on CP and Optimization in Ireland
\item Part of national training program for PhD students in AI
\item \url{https://www.crt-ai.ie/}
\end{itemize}
\end{frame}



\begin{frame}
\frametitle{Constraint Programming - in a nutshell}
\begin{itemize}
\item Declarative description of problems with
\begin{itemize}
\item {\em Variables} which range over (finite) sets of values
\item {\em Constraints} over subsets of variables which restrict possible value combinations
\item A {\em solution} is a value assignment which satisfies all constraints
\end{itemize}

\item Constraint propagation/reasoning
\begin{itemize}
\item Removing inconsistent values for variables
\item Detect failure if constraint can not be satisfied
\item Interaction of constraints via shared variables
\item Incomplete
\end{itemize}

\item Search
\begin{itemize}
\item User controlled assignment of values to variables
\item Each step triggers constraint propagation 
\end{itemize}
\item Different domains require/allow different methods
\end{itemize}
\end{frame}

\begin{frame}
  \frametitle{Constraint Programming is Different}
  \begin{itemize}
  \item Declarative Programming
    \begin{itemize}
    \item Concentrate on what you want
      \item Not how to get there
      \item Program != Algorithm
      \item Program = Model
    \end{itemize}
    \item Applied to Combinatorial Problems
      \begin{itemize}
        \item No complete polynomial algorithms known (exist?)
        \item CP less ad-hoc than heuristics
        \item Models can evolve
  \end{itemize}
  \end{itemize}
  \end{frame}
    
\begin{frame}
  \frametitle{A Subtractive Process}
  \begin{textblock}{4}(8,-3)
    \includegraphics[width=4cm]{../introduction/images/stages}
  \end{textblock}
  \vfill
  \begin{quote}
    ``Oh, bosh, as Mr. Ruskin says. Sculpture, per se, is the simplest thing in the world. All you have to do is to take a big chunk of marble and a hammer and chisel, make up your mind what you are about to create and chip off all the marble you don't want.''-Paris Gaulois.
  \end{quote}
  
  {\tiny Source: \url{https://quoteinvestigator.com/2014/06/22/chip-away/}}
\end{frame}

\begin{frame}
\frametitle{Basic Process}
\begin{center}
\begin{tikzpicture}
\node[shape=rectangle,fill=pantone127-4] (problem) at (2,5) {Problem};
\node (human) at (2,4) {Human};
\node[shape=rectangle,fill=pantone127-4] (model) at (2,3) {Model};
\node (solver) at (2,2) {Constraint Solver/Search};
\node[shape=rectangle,fill=pantone157-8] (solution) at (2,1) {Solution};
\draw[-] (problem) -- (human);
\draw[->] (human) -- (model);
\draw[-] (model) -- (solver);
\draw[->] (solver) -- (solution);
\end{tikzpicture}
\end{center}
\end{frame}

\begin{frame}
\frametitle{More Realistic}
\begin{center}
\begin{tikzpicture}[xscale=4.0]
\node[shape=rectangle,fill=pantone127-4] (problem) at (2,5) {Problem};
\node (human) at (2,4) {Human};
\node[shape=rectangle,fill=pantone127-4] (model) at (2,3) {Model};
\node (solver) at (2,2) {Constraint Solver/Search};
\node[shape=rectangle,fill=pantone24-8] (hangs) at (1,1) {Hangs};
\node[shape=rectangle,fill=pantone157-8] (solution) at (2,1) {Solution};
\node[shape=rectangle,fill=pantone24-8] (wrong) at (3,1) {Wrong Solution};
\draw[-] (problem) -- (human);
\draw[->] (human) -- (model);
\draw[-] (model) -- (solver);
\draw[->] (solver) -- (hangs);
\draw[->] (solver) -- (solution);
\draw[->] (solver) -- (wrong);
\draw[->] (hangs) -- (1,4) -- (human);
\draw[->] (wrong) -- (3,4) -- (human);
\end{tikzpicture}
\end{center}
\end{frame}

\begin{frame}
\frametitle{Dual Role of Model}
\begin{itemize}
\item Allows Human to Express Problem
\begin{itemize}
\item Close to Problem Domain
\item Constraints as Abstractions
\end{itemize}
\item Allows Solver to Execute
\begin{itemize}
\item Variables as Communication Mechanism
\item Constraints as Algorithms
\end{itemize}
\end{itemize}
\end{frame}

\begin{frame}
\frametitle{Modelling Frameworks}
\begin{itemize}
\item \href{https://www.minizinc.org/}{MiniZinc} (NICTA, Monash University, Australia)
\item \href{https://github.com/eomahony/Numberjack}{NumberJack} (Insight, Ireland) 
\item \href{https://www-users.york.ac.uk/peter.nightingale/savilerow//}{EssencePrime/SavilleRow} (UK) 
\item \href{https://cpmpy.readthedocs.io/en/latest/}{CPMpy} (KU Leuven)
\item Allow use of multiple back-end solvers
\item Compile model into variants for each solver
\item A priori solver independent model(CP, MIP, SAT)
\end{itemize}
\end{frame} 

\begin{frame}
\frametitle{Framework Process}
\begin{center}
\begin{tikzpicture}[xscale=2]
\node[shape=rectangle,fill=pantone127-4] (problem) at (2.5,5) {Problem};
\node (human) at (2.5,4) {Human};
\node[shape=rectangle,fill=pantone127-4] (model) at (2.5,3) {Model};
\node (compile) at (2.5,2) {Compile/Reformulate};
\node (solver1) at (1,1) {CP};
\node (solver2) at (2,1) {MIP};
\node (solver3) at (3,1) {SAT};
\node (solver4) at (4,1) {Other};
\node[shape=rectangle,fill=pantone157-8] (solution1) at (1,0) {Solution};
\node[shape=rectangle,fill=pantone157-8] (solution2) at (2,0) {Solution};
\node[shape=rectangle,fill=pantone157-8] (solution3) at (3,0) {Solution};
\node[shape=rectangle,fill=pantone157-8] (solution4) at (4,0) {Solution};
\draw[-] (problem) -- (human);
\draw[->] (human) -- (model);
\draw[-] (model) -- (compile);
\draw[-] (compile) -- (solver1);
\draw[-] (compile) -- (solver2);
\draw[-] (compile) -- (solver3);
\draw[-] (compile) -- (solver4);
\draw[->] (solver1) -- (solution1);
\draw[->] (solver2) -- (solution2);
\draw[->] (solver3) -- (solution3);
\draw[->] (solver4) -- (solution4);
\end{tikzpicture}
\end{center}
\end{frame}

% \begin{frame}
  % \frametitle{Do It Now!}
  % \begin{itemize}
  % \item Download and install Minizinc
  % \item \url{https://www.minizinc.org/}
  % \end{itemize}
% \end{frame}


\begin{frame}
\frametitle{Why use Puzzles as Examples?}
\begin{itemize}
\item Easy to understand the problem
\item Solvable by hand without specialized knowledge
\item Possible to compare automated to manual solving process
\end{itemize}

\begin{quote}
The puzzle, though inanimate, is presented as a solvable problem without lasting negative consequences, a very low-risk low-reward situation. By being a puzzle, the object is attempting to convince the user that it must be completed.
\end{quote}
{\tiny Source: \href{https://www.everydayrhetoric.com/2020/03/the-rhetoric-of-puzzles/}{Every Day Rhetoric}}
\end{frame}


\mode<all>{
\part{Basic Constraint Propagation}
}
\begin{frame}
\partpage
\end{frame}
\mode<all>{
\begin{frame}
\frametitle{Objectives}
\begin{itemize}
\item Overview of Core Constraint Programming
\item Three Main Concepts
\begin{itemize}
\item Constraint Propagation
\item Global Constraints
\item Customizing Search
\end{itemize}
%\item Get Some Experience with MiniZinc
\item Topics will be treated in more detail in later parts of the school
\item Based on Examples, not Formal Description
\end{itemize}
\end{frame}

\begin{frame}
\frametitle{Outline}
\begin{itemize}
\item Why Constraint Programming?
\item Constraint Propagation
\item Global Constraints
\item Customizing Search
%\item Choices in Modelling
%\item Decomposition
%\item More Global Constraints
%\item Methodology
%\item What is missing?
\end{itemize}
\end{frame}

% \begin{frame}
  % \frametitle{Using MiniZinc IDE}
  % \begin{itemize}
  % \item Developed in the Australian NICTA project
  % \item Maintained by Monash University
  % \item Modelling tool with multiple back-end solvers
  % \item Available from \url{https://www.minizinc.org/}
  % \end{itemize}
% \end{frame}


% \begin{frame}
% \frametitle{Examples in ECLiPSe}
% \begin{itemize}
% \item Open sourced constraint programming language
% \item Development goes back to 1985
% \item ECRC, ICL, IC-Parc, PTL, Cisco
% \item \url{https://eclipseclp.org/}
% \item Specialities
% \begin{itemize}
% \item Develop new solvers for specific domains
% \item Integration with MIP
% \end{itemize}
% \item Not included in bundled MiniZinc IDE
% \item Specialized visualization tools used here
  % \begin{itemize}
  % \item CP-Viz, Simonis et al. 2010
    % \end{itemize}
% \end{itemize}
% \end{frame}

\begin{frame}
\frametitle{Tutorial Based on ECLiPSe ELearning Course}
\begin{itemize}
\item Self-study course in constraint programming
\item Supported by Cisco Systems and Silicon Valley Community Foundation
\item Multi-media format, video lectures, slides, handout etc
\item {\small \url{https://eclipseclp.org/ELearning/index.html}}
\end{itemize}
\end{frame}

\begin{frame}
\frametitle{Also Part of CRT-AI Constraint Week}
\begin{itemize}
\item Annual one week course on CP and Optimization in Ireland
\item Part of national training program for PhD students in AI
\item \url{https://www.crt-ai.ie/}
\end{itemize}
\end{frame}



\begin{frame}
\frametitle{Constraint Programming - in a nutshell}
\begin{itemize}
\item Declarative description of problems with
\begin{itemize}
\item {\em Variables} which range over (finite) sets of values
\item {\em Constraints} over subsets of variables which restrict possible value combinations
\item A {\em solution} is a value assignment which satisfies all constraints
\end{itemize}

\item Constraint propagation/reasoning
\begin{itemize}
\item Removing inconsistent values for variables
\item Detect failure if constraint can not be satisfied
\item Interaction of constraints via shared variables
\item Incomplete
\end{itemize}

\item Search
\begin{itemize}
\item User controlled assignment of values to variables
\item Each step triggers constraint propagation 
\end{itemize}
\item Different domains require/allow different methods
\end{itemize}
\end{frame}

\begin{frame}
  \frametitle{Constraint Programming is Different}
  \begin{itemize}
  \item Declarative Programming
    \begin{itemize}
    \item Concentrate on what you want
      \item Not how to get there
      \item Program != Algorithm
      \item Program = Model
    \end{itemize}
    \item Applied to Combinatorial Problems
      \begin{itemize}
        \item No complete polynomial algorithms known (exist?)
        \item CP less ad-hoc than heuristics
        \item Models can evolve
  \end{itemize}
  \end{itemize}
  \end{frame}
    
\begin{frame}
  \frametitle{A Subtractive Process}
  \begin{textblock}{4}(8,-3)
    \includegraphics[width=4cm]{../introduction/images/stages}
  \end{textblock}
  \vfill
  \begin{quote}
    ``Oh, bosh, as Mr. Ruskin says. Sculpture, per se, is the simplest thing in the world. All you have to do is to take a big chunk of marble and a hammer and chisel, make up your mind what you are about to create and chip off all the marble you don't want.''-Paris Gaulois.
  \end{quote}
  
  {\tiny Source: \url{https://quoteinvestigator.com/2014/06/22/chip-away/}}
\end{frame}

\begin{frame}
\frametitle{Basic Process}
\begin{center}
\begin{tikzpicture}
\node[shape=rectangle,fill=pantone127-4] (problem) at (2,5) {Problem};
\node (human) at (2,4) {Human};
\node[shape=rectangle,fill=pantone127-4] (model) at (2,3) {Model};
\node (solver) at (2,2) {Constraint Solver/Search};
\node[shape=rectangle,fill=pantone157-8] (solution) at (2,1) {Solution};
\draw[-] (problem) -- (human);
\draw[->] (human) -- (model);
\draw[-] (model) -- (solver);
\draw[->] (solver) -- (solution);
\end{tikzpicture}
\end{center}
\end{frame}

\begin{frame}
\frametitle{More Realistic}
\begin{center}
\begin{tikzpicture}[xscale=4.0]
\node[shape=rectangle,fill=pantone127-4] (problem) at (2,5) {Problem};
\node (human) at (2,4) {Human};
\node[shape=rectangle,fill=pantone127-4] (model) at (2,3) {Model};
\node (solver) at (2,2) {Constraint Solver/Search};
\node[shape=rectangle,fill=pantone24-8] (hangs) at (1,1) {Hangs};
\node[shape=rectangle,fill=pantone157-8] (solution) at (2,1) {Solution};
\node[shape=rectangle,fill=pantone24-8] (wrong) at (3,1) {Wrong Solution};
\draw[-] (problem) -- (human);
\draw[->] (human) -- (model);
\draw[-] (model) -- (solver);
\draw[->] (solver) -- (hangs);
\draw[->] (solver) -- (solution);
\draw[->] (solver) -- (wrong);
\draw[->] (hangs) -- (1,4) -- (human);
\draw[->] (wrong) -- (3,4) -- (human);
\end{tikzpicture}
\end{center}
\end{frame}

\begin{frame}
\frametitle{Dual Role of Model}
\begin{itemize}
\item Allows Human to Express Problem
\begin{itemize}
\item Close to Problem Domain
\item Constraints as Abstractions
\end{itemize}
\item Allows Solver to Execute
\begin{itemize}
\item Variables as Communication Mechanism
\item Constraints as Algorithms
\end{itemize}
\end{itemize}
\end{frame}

\begin{frame}
\frametitle{Modelling Frameworks}
\begin{itemize}
\item \href{https://www.minizinc.org/}{MiniZinc} (NICTA, Monash University, Australia)
\item \href{https://github.com/eomahony/Numberjack}{NumberJack} (Insight, Ireland) 
\item \href{https://www-users.york.ac.uk/peter.nightingale/savilerow//}{EssencePrime/SavilleRow} (UK) 
\item \href{https://cpmpy.readthedocs.io/en/latest/}{CPMpy} (KU Leuven)
\item Allow use of multiple back-end solvers
\item Compile model into variants for each solver
\item A priori solver independent model(CP, MIP, SAT)
\end{itemize}
\end{frame} 

\begin{frame}
\frametitle{Framework Process}
\begin{center}
\begin{tikzpicture}[xscale=2]
\node[shape=rectangle,fill=pantone127-4] (problem) at (2.5,5) {Problem};
\node (human) at (2.5,4) {Human};
\node[shape=rectangle,fill=pantone127-4] (model) at (2.5,3) {Model};
\node (compile) at (2.5,2) {Compile/Reformulate};
\node (solver1) at (1,1) {CP};
\node (solver2) at (2,1) {MIP};
\node (solver3) at (3,1) {SAT};
\node (solver4) at (4,1) {Other};
\node[shape=rectangle,fill=pantone157-8] (solution1) at (1,0) {Solution};
\node[shape=rectangle,fill=pantone157-8] (solution2) at (2,0) {Solution};
\node[shape=rectangle,fill=pantone157-8] (solution3) at (3,0) {Solution};
\node[shape=rectangle,fill=pantone157-8] (solution4) at (4,0) {Solution};
\draw[-] (problem) -- (human);
\draw[->] (human) -- (model);
\draw[-] (model) -- (compile);
\draw[-] (compile) -- (solver1);
\draw[-] (compile) -- (solver2);
\draw[-] (compile) -- (solver3);
\draw[-] (compile) -- (solver4);
\draw[->] (solver1) -- (solution1);
\draw[->] (solver2) -- (solution2);
\draw[->] (solver3) -- (solution3);
\draw[->] (solver4) -- (solution4);
\end{tikzpicture}
\end{center}
\end{frame}

% \begin{frame}
  % \frametitle{Do It Now!}
  % \begin{itemize}
  % \item Download and install Minizinc
  % \item \url{https://www.minizinc.org/}
  % \end{itemize}
% \end{frame}


\begin{frame}
\frametitle{Why use Puzzles as Examples?}
\begin{itemize}
\item Easy to understand the problem
\item Solvable by hand without specialized knowledge
\item Possible to compare automated to manual solving process
\end{itemize}

\begin{quote}
The puzzle, though inanimate, is presented as a solvable problem without lasting negative consequences, a very low-risk low-reward situation. By being a puzzle, the object is attempting to convince the user that it must be completed.
\end{quote}
{\tiny Source: \href{https://www.everydayrhetoric.com/2020/03/the-rhetoric-of-puzzles/}{Every Day Rhetoric}}
\end{frame}


\mode<all>{
\part{Basic Constraint Propagation}
}
\begin{frame}
\partpage
\end{frame}
\mode<all>{
\input{../sendmore/body}
}

\mode<all>{
\part{Global Constraints}
}
\begin{frame}
\partpage
\end{frame}
\mode<all>{
\input{../sudoku/body}
}

\clearpage
\mode<all>{
\part{Customizing Search}
}
\begin{frame}
\partpage
\end{frame}
\mode<all>{
\input{../nqueen/body}
}


%% \clearpage
%% \mode<all>{
%%   \part{Problem Decomposition}
%% }
%% \begin{frame}
%%   \partpage
%% \end{frame}
%% \mode<all>{
%%   \input{../party/body}
%%   }

%% \clearpage
%% \mode<all>{
%%   \part{More Global Constraints}
%% }
%% \begin{frame}
%%   \partpage
%% \end{frame}
%% \mode<all>{
%%   \input{../car/body}
%%   }

%% \clearpage
%% \mode<all>{
%%   \part{Methodology}
%% }
%% \begin{frame}
%%   \partpage
%% \end{frame}
%% \mode<all>{
%%   \input{../methodology/body}
%%   }

\clearpage
\mode<all>{
\part{What is missing?}
}
\begin{frame}
\partpage
\end{frame}

\begin{frame}
\frametitle{Many Specialized Topics}
\begin{itemize}
\item How to design efficient core engine
\item Hybrids with LP/MIP tools
\item Hybrids with SAT
\item Symmetry breaking
\item Use of MDD/BDD to encode sets of solutions
\item High level modelling tools
\item Debugging/visualization
\end{itemize}
\end{frame}

\begin{frame}
\frametitle{Reformulation}
\begin{itemize}
\item Just because the user has modelled it this way, it doesn't mean we have to solve it that way
\begin{itemize}
\item Replace some constraint(s) by other, equivalent constraints
\item Because we don't have that constraint in our system
\item For performance
\end{itemize}
\end{itemize}
\end{frame}

\begin{frame}
\frametitle{Learning}
\begin{itemize}
\item While solving the problem we can learn how to strengthen the model/search 
\begin{itemize}
\item Understand which constraints/method contribute to propagation and change schedule
\item Learn no-good constraints by explaining failure
\item Adapt search strategy based on search experience
\end{itemize}
\end{itemize}
\end{frame}

%% \begin{frame}
%% \frametitle{Refined Process}
%% \begin{center}
%% \begin{tikzpicture}
%% \node[shape=rectangle,fill=pantone127-4] (problem) at (3,7) {Problem};
%% \node (human) at (3,6) {Human};
%% \node[shape=rectangle,fill=pantone127-4] (model) at (3,5) {User Model};
%% \node (reformulation) at (3,4) {Reformulation};
%% \node[shape=rectangle,fill=pantone127-4] (implementation) at (3,3) {Implementation Model};
%% \node (solver) at (3,2) {Constraint Solver/Search};
%% \node[shape=rectangle,fill=pantone157-8] (solution) at (3,1) {Solution};
%% \draw[-] (problem) -- (human);
%% \draw[->] (human) -- (model);
%% \draw[-] (model) -- (reformulation);
%% \draw[->] (reformulation) -- (implementation);
%% \draw[-] (implementation) -- (solver);
%% \draw[->] (solver) -- (0,2) -- (0,3) -- (implementation);
%% \draw[->] (solver) -- (solution);
%% \end{tikzpicture}
%% \end{center}
%% \end{frame}

%\begin{frame}
%\frametitle{What is CP actually used for?}
%\url{http://hsimonis.wordpress.com}
%\includegraphics[width=10cm]{../introduction/images/blog}
%\end{frame}

%% \begin{frame}
%% \frametitle{Exercises}
%% \begin{itemize}
%% \item Install Constraint Solver of your Choice
%% \item MiniZinc (MiningZinc) will be used in later sessions
%% \item Run Sendmoremoney, Sudoku, Queens examples
%% \item Can you choose which propagator to use for alldifferent?
%% \item Can you use different search strategies?
%% \end{itemize}
%% \end{frame}

\begin{frame}
\frametitle{More Learning Resources}
\begin{itemize}
\item Survey of Methods, Resources, and Formats for Teaching Constraint Programming
\begin{itemize}
\item by Tejas Santanam, Helmut Simonis
\item \url{https://doi.org/10.48550/arXiv.2403.12717}
\item Based on survey of community for WTCP 2023
\item \url{https://hsimonis.github.io/WTCP2023/}
\end{itemize}
\end{itemize}
\end{frame}


}

\mode<all>{
\part{Global Constraints}
}
\begin{frame}
\partpage
\end{frame}
\mode<all>{
\begin{frame}
\frametitle{Objectives}
\begin{itemize}
\item Overview of Core Constraint Programming
\item Three Main Concepts
\begin{itemize}
\item Constraint Propagation
\item Global Constraints
\item Customizing Search
\end{itemize}
%\item Get Some Experience with MiniZinc
\item Topics will be treated in more detail in later parts of the school
\item Based on Examples, not Formal Description
\end{itemize}
\end{frame}

\begin{frame}
\frametitle{Outline}
\begin{itemize}
\item Why Constraint Programming?
\item Constraint Propagation
\item Global Constraints
\item Customizing Search
%\item Choices in Modelling
%\item Decomposition
%\item More Global Constraints
%\item Methodology
%\item What is missing?
\end{itemize}
\end{frame}

% \begin{frame}
  % \frametitle{Using MiniZinc IDE}
  % \begin{itemize}
  % \item Developed in the Australian NICTA project
  % \item Maintained by Monash University
  % \item Modelling tool with multiple back-end solvers
  % \item Available from \url{https://www.minizinc.org/}
  % \end{itemize}
% \end{frame}


% \begin{frame}
% \frametitle{Examples in ECLiPSe}
% \begin{itemize}
% \item Open sourced constraint programming language
% \item Development goes back to 1985
% \item ECRC, ICL, IC-Parc, PTL, Cisco
% \item \url{https://eclipseclp.org/}
% \item Specialities
% \begin{itemize}
% \item Develop new solvers for specific domains
% \item Integration with MIP
% \end{itemize}
% \item Not included in bundled MiniZinc IDE
% \item Specialized visualization tools used here
  % \begin{itemize}
  % \item CP-Viz, Simonis et al. 2010
    % \end{itemize}
% \end{itemize}
% \end{frame}

\begin{frame}
\frametitle{Tutorial Based on ECLiPSe ELearning Course}
\begin{itemize}
\item Self-study course in constraint programming
\item Supported by Cisco Systems and Silicon Valley Community Foundation
\item Multi-media format, video lectures, slides, handout etc
\item {\small \url{https://eclipseclp.org/ELearning/index.html}}
\end{itemize}
\end{frame}

\begin{frame}
\frametitle{Also Part of CRT-AI Constraint Week}
\begin{itemize}
\item Annual one week course on CP and Optimization in Ireland
\item Part of national training program for PhD students in AI
\item \url{https://www.crt-ai.ie/}
\end{itemize}
\end{frame}



\begin{frame}
\frametitle{Constraint Programming - in a nutshell}
\begin{itemize}
\item Declarative description of problems with
\begin{itemize}
\item {\em Variables} which range over (finite) sets of values
\item {\em Constraints} over subsets of variables which restrict possible value combinations
\item A {\em solution} is a value assignment which satisfies all constraints
\end{itemize}

\item Constraint propagation/reasoning
\begin{itemize}
\item Removing inconsistent values for variables
\item Detect failure if constraint can not be satisfied
\item Interaction of constraints via shared variables
\item Incomplete
\end{itemize}

\item Search
\begin{itemize}
\item User controlled assignment of values to variables
\item Each step triggers constraint propagation 
\end{itemize}
\item Different domains require/allow different methods
\end{itemize}
\end{frame}

\begin{frame}
  \frametitle{Constraint Programming is Different}
  \begin{itemize}
  \item Declarative Programming
    \begin{itemize}
    \item Concentrate on what you want
      \item Not how to get there
      \item Program != Algorithm
      \item Program = Model
    \end{itemize}
    \item Applied to Combinatorial Problems
      \begin{itemize}
        \item No complete polynomial algorithms known (exist?)
        \item CP less ad-hoc than heuristics
        \item Models can evolve
  \end{itemize}
  \end{itemize}
  \end{frame}
    
\begin{frame}
  \frametitle{A Subtractive Process}
  \begin{textblock}{4}(8,-3)
    \includegraphics[width=4cm]{../introduction/images/stages}
  \end{textblock}
  \vfill
  \begin{quote}
    ``Oh, bosh, as Mr. Ruskin says. Sculpture, per se, is the simplest thing in the world. All you have to do is to take a big chunk of marble and a hammer and chisel, make up your mind what you are about to create and chip off all the marble you don't want.''-Paris Gaulois.
  \end{quote}
  
  {\tiny Source: \url{https://quoteinvestigator.com/2014/06/22/chip-away/}}
\end{frame}

\begin{frame}
\frametitle{Basic Process}
\begin{center}
\begin{tikzpicture}
\node[shape=rectangle,fill=pantone127-4] (problem) at (2,5) {Problem};
\node (human) at (2,4) {Human};
\node[shape=rectangle,fill=pantone127-4] (model) at (2,3) {Model};
\node (solver) at (2,2) {Constraint Solver/Search};
\node[shape=rectangle,fill=pantone157-8] (solution) at (2,1) {Solution};
\draw[-] (problem) -- (human);
\draw[->] (human) -- (model);
\draw[-] (model) -- (solver);
\draw[->] (solver) -- (solution);
\end{tikzpicture}
\end{center}
\end{frame}

\begin{frame}
\frametitle{More Realistic}
\begin{center}
\begin{tikzpicture}[xscale=4.0]
\node[shape=rectangle,fill=pantone127-4] (problem) at (2,5) {Problem};
\node (human) at (2,4) {Human};
\node[shape=rectangle,fill=pantone127-4] (model) at (2,3) {Model};
\node (solver) at (2,2) {Constraint Solver/Search};
\node[shape=rectangle,fill=pantone24-8] (hangs) at (1,1) {Hangs};
\node[shape=rectangle,fill=pantone157-8] (solution) at (2,1) {Solution};
\node[shape=rectangle,fill=pantone24-8] (wrong) at (3,1) {Wrong Solution};
\draw[-] (problem) -- (human);
\draw[->] (human) -- (model);
\draw[-] (model) -- (solver);
\draw[->] (solver) -- (hangs);
\draw[->] (solver) -- (solution);
\draw[->] (solver) -- (wrong);
\draw[->] (hangs) -- (1,4) -- (human);
\draw[->] (wrong) -- (3,4) -- (human);
\end{tikzpicture}
\end{center}
\end{frame}

\begin{frame}
\frametitle{Dual Role of Model}
\begin{itemize}
\item Allows Human to Express Problem
\begin{itemize}
\item Close to Problem Domain
\item Constraints as Abstractions
\end{itemize}
\item Allows Solver to Execute
\begin{itemize}
\item Variables as Communication Mechanism
\item Constraints as Algorithms
\end{itemize}
\end{itemize}
\end{frame}

\begin{frame}
\frametitle{Modelling Frameworks}
\begin{itemize}
\item \href{https://www.minizinc.org/}{MiniZinc} (NICTA, Monash University, Australia)
\item \href{https://github.com/eomahony/Numberjack}{NumberJack} (Insight, Ireland) 
\item \href{https://www-users.york.ac.uk/peter.nightingale/savilerow//}{EssencePrime/SavilleRow} (UK) 
\item \href{https://cpmpy.readthedocs.io/en/latest/}{CPMpy} (KU Leuven)
\item Allow use of multiple back-end solvers
\item Compile model into variants for each solver
\item A priori solver independent model(CP, MIP, SAT)
\end{itemize}
\end{frame} 

\begin{frame}
\frametitle{Framework Process}
\begin{center}
\begin{tikzpicture}[xscale=2]
\node[shape=rectangle,fill=pantone127-4] (problem) at (2.5,5) {Problem};
\node (human) at (2.5,4) {Human};
\node[shape=rectangle,fill=pantone127-4] (model) at (2.5,3) {Model};
\node (compile) at (2.5,2) {Compile/Reformulate};
\node (solver1) at (1,1) {CP};
\node (solver2) at (2,1) {MIP};
\node (solver3) at (3,1) {SAT};
\node (solver4) at (4,1) {Other};
\node[shape=rectangle,fill=pantone157-8] (solution1) at (1,0) {Solution};
\node[shape=rectangle,fill=pantone157-8] (solution2) at (2,0) {Solution};
\node[shape=rectangle,fill=pantone157-8] (solution3) at (3,0) {Solution};
\node[shape=rectangle,fill=pantone157-8] (solution4) at (4,0) {Solution};
\draw[-] (problem) -- (human);
\draw[->] (human) -- (model);
\draw[-] (model) -- (compile);
\draw[-] (compile) -- (solver1);
\draw[-] (compile) -- (solver2);
\draw[-] (compile) -- (solver3);
\draw[-] (compile) -- (solver4);
\draw[->] (solver1) -- (solution1);
\draw[->] (solver2) -- (solution2);
\draw[->] (solver3) -- (solution3);
\draw[->] (solver4) -- (solution4);
\end{tikzpicture}
\end{center}
\end{frame}

% \begin{frame}
  % \frametitle{Do It Now!}
  % \begin{itemize}
  % \item Download and install Minizinc
  % \item \url{https://www.minizinc.org/}
  % \end{itemize}
% \end{frame}


\begin{frame}
\frametitle{Why use Puzzles as Examples?}
\begin{itemize}
\item Easy to understand the problem
\item Solvable by hand without specialized knowledge
\item Possible to compare automated to manual solving process
\end{itemize}

\begin{quote}
The puzzle, though inanimate, is presented as a solvable problem without lasting negative consequences, a very low-risk low-reward situation. By being a puzzle, the object is attempting to convince the user that it must be completed.
\end{quote}
{\tiny Source: \href{https://www.everydayrhetoric.com/2020/03/the-rhetoric-of-puzzles/}{Every Day Rhetoric}}
\end{frame}


\mode<all>{
\part{Basic Constraint Propagation}
}
\begin{frame}
\partpage
\end{frame}
\mode<all>{
\input{../sendmore/body}
}

\mode<all>{
\part{Global Constraints}
}
\begin{frame}
\partpage
\end{frame}
\mode<all>{
\input{../sudoku/body}
}

\clearpage
\mode<all>{
\part{Customizing Search}
}
\begin{frame}
\partpage
\end{frame}
\mode<all>{
\input{../nqueen/body}
}


%% \clearpage
%% \mode<all>{
%%   \part{Problem Decomposition}
%% }
%% \begin{frame}
%%   \partpage
%% \end{frame}
%% \mode<all>{
%%   \input{../party/body}
%%   }

%% \clearpage
%% \mode<all>{
%%   \part{More Global Constraints}
%% }
%% \begin{frame}
%%   \partpage
%% \end{frame}
%% \mode<all>{
%%   \input{../car/body}
%%   }

%% \clearpage
%% \mode<all>{
%%   \part{Methodology}
%% }
%% \begin{frame}
%%   \partpage
%% \end{frame}
%% \mode<all>{
%%   \input{../methodology/body}
%%   }

\clearpage
\mode<all>{
\part{What is missing?}
}
\begin{frame}
\partpage
\end{frame}

\begin{frame}
\frametitle{Many Specialized Topics}
\begin{itemize}
\item How to design efficient core engine
\item Hybrids with LP/MIP tools
\item Hybrids with SAT
\item Symmetry breaking
\item Use of MDD/BDD to encode sets of solutions
\item High level modelling tools
\item Debugging/visualization
\end{itemize}
\end{frame}

\begin{frame}
\frametitle{Reformulation}
\begin{itemize}
\item Just because the user has modelled it this way, it doesn't mean we have to solve it that way
\begin{itemize}
\item Replace some constraint(s) by other, equivalent constraints
\item Because we don't have that constraint in our system
\item For performance
\end{itemize}
\end{itemize}
\end{frame}

\begin{frame}
\frametitle{Learning}
\begin{itemize}
\item While solving the problem we can learn how to strengthen the model/search 
\begin{itemize}
\item Understand which constraints/method contribute to propagation and change schedule
\item Learn no-good constraints by explaining failure
\item Adapt search strategy based on search experience
\end{itemize}
\end{itemize}
\end{frame}

%% \begin{frame}
%% \frametitle{Refined Process}
%% \begin{center}
%% \begin{tikzpicture}
%% \node[shape=rectangle,fill=pantone127-4] (problem) at (3,7) {Problem};
%% \node (human) at (3,6) {Human};
%% \node[shape=rectangle,fill=pantone127-4] (model) at (3,5) {User Model};
%% \node (reformulation) at (3,4) {Reformulation};
%% \node[shape=rectangle,fill=pantone127-4] (implementation) at (3,3) {Implementation Model};
%% \node (solver) at (3,2) {Constraint Solver/Search};
%% \node[shape=rectangle,fill=pantone157-8] (solution) at (3,1) {Solution};
%% \draw[-] (problem) -- (human);
%% \draw[->] (human) -- (model);
%% \draw[-] (model) -- (reformulation);
%% \draw[->] (reformulation) -- (implementation);
%% \draw[-] (implementation) -- (solver);
%% \draw[->] (solver) -- (0,2) -- (0,3) -- (implementation);
%% \draw[->] (solver) -- (solution);
%% \end{tikzpicture}
%% \end{center}
%% \end{frame}

%\begin{frame}
%\frametitle{What is CP actually used for?}
%\url{http://hsimonis.wordpress.com}
%\includegraphics[width=10cm]{../introduction/images/blog}
%\end{frame}

%% \begin{frame}
%% \frametitle{Exercises}
%% \begin{itemize}
%% \item Install Constraint Solver of your Choice
%% \item MiniZinc (MiningZinc) will be used in later sessions
%% \item Run Sendmoremoney, Sudoku, Queens examples
%% \item Can you choose which propagator to use for alldifferent?
%% \item Can you use different search strategies?
%% \end{itemize}
%% \end{frame}

\begin{frame}
\frametitle{More Learning Resources}
\begin{itemize}
\item Survey of Methods, Resources, and Formats for Teaching Constraint Programming
\begin{itemize}
\item by Tejas Santanam, Helmut Simonis
\item \url{https://doi.org/10.48550/arXiv.2403.12717}
\item Based on survey of community for WTCP 2023
\item \url{https://hsimonis.github.io/WTCP2023/}
\end{itemize}
\end{itemize}
\end{frame}


}

\clearpage
\mode<all>{
\part{Customizing Search}
}
\begin{frame}
\partpage
\end{frame}
\mode<all>{
\begin{frame}
\frametitle{Objectives}
\begin{itemize}
\item Overview of Core Constraint Programming
\item Three Main Concepts
\begin{itemize}
\item Constraint Propagation
\item Global Constraints
\item Customizing Search
\end{itemize}
%\item Get Some Experience with MiniZinc
\item Topics will be treated in more detail in later parts of the school
\item Based on Examples, not Formal Description
\end{itemize}
\end{frame}

\begin{frame}
\frametitle{Outline}
\begin{itemize}
\item Why Constraint Programming?
\item Constraint Propagation
\item Global Constraints
\item Customizing Search
%\item Choices in Modelling
%\item Decomposition
%\item More Global Constraints
%\item Methodology
%\item What is missing?
\end{itemize}
\end{frame}

% \begin{frame}
  % \frametitle{Using MiniZinc IDE}
  % \begin{itemize}
  % \item Developed in the Australian NICTA project
  % \item Maintained by Monash University
  % \item Modelling tool with multiple back-end solvers
  % \item Available from \url{https://www.minizinc.org/}
  % \end{itemize}
% \end{frame}


% \begin{frame}
% \frametitle{Examples in ECLiPSe}
% \begin{itemize}
% \item Open sourced constraint programming language
% \item Development goes back to 1985
% \item ECRC, ICL, IC-Parc, PTL, Cisco
% \item \url{https://eclipseclp.org/}
% \item Specialities
% \begin{itemize}
% \item Develop new solvers for specific domains
% \item Integration with MIP
% \end{itemize}
% \item Not included in bundled MiniZinc IDE
% \item Specialized visualization tools used here
  % \begin{itemize}
  % \item CP-Viz, Simonis et al. 2010
    % \end{itemize}
% \end{itemize}
% \end{frame}

\begin{frame}
\frametitle{Tutorial Based on ECLiPSe ELearning Course}
\begin{itemize}
\item Self-study course in constraint programming
\item Supported by Cisco Systems and Silicon Valley Community Foundation
\item Multi-media format, video lectures, slides, handout etc
\item {\small \url{https://eclipseclp.org/ELearning/index.html}}
\end{itemize}
\end{frame}

\begin{frame}
\frametitle{Also Part of CRT-AI Constraint Week}
\begin{itemize}
\item Annual one week course on CP and Optimization in Ireland
\item Part of national training program for PhD students in AI
\item \url{https://www.crt-ai.ie/}
\end{itemize}
\end{frame}



\begin{frame}
\frametitle{Constraint Programming - in a nutshell}
\begin{itemize}
\item Declarative description of problems with
\begin{itemize}
\item {\em Variables} which range over (finite) sets of values
\item {\em Constraints} over subsets of variables which restrict possible value combinations
\item A {\em solution} is a value assignment which satisfies all constraints
\end{itemize}

\item Constraint propagation/reasoning
\begin{itemize}
\item Removing inconsistent values for variables
\item Detect failure if constraint can not be satisfied
\item Interaction of constraints via shared variables
\item Incomplete
\end{itemize}

\item Search
\begin{itemize}
\item User controlled assignment of values to variables
\item Each step triggers constraint propagation 
\end{itemize}
\item Different domains require/allow different methods
\end{itemize}
\end{frame}

\begin{frame}
  \frametitle{Constraint Programming is Different}
  \begin{itemize}
  \item Declarative Programming
    \begin{itemize}
    \item Concentrate on what you want
      \item Not how to get there
      \item Program != Algorithm
      \item Program = Model
    \end{itemize}
    \item Applied to Combinatorial Problems
      \begin{itemize}
        \item No complete polynomial algorithms known (exist?)
        \item CP less ad-hoc than heuristics
        \item Models can evolve
  \end{itemize}
  \end{itemize}
  \end{frame}
    
\begin{frame}
  \frametitle{A Subtractive Process}
  \begin{textblock}{4}(8,-3)
    \includegraphics[width=4cm]{../introduction/images/stages}
  \end{textblock}
  \vfill
  \begin{quote}
    ``Oh, bosh, as Mr. Ruskin says. Sculpture, per se, is the simplest thing in the world. All you have to do is to take a big chunk of marble and a hammer and chisel, make up your mind what you are about to create and chip off all the marble you don't want.''-Paris Gaulois.
  \end{quote}
  
  {\tiny Source: \url{https://quoteinvestigator.com/2014/06/22/chip-away/}}
\end{frame}

\begin{frame}
\frametitle{Basic Process}
\begin{center}
\begin{tikzpicture}
\node[shape=rectangle,fill=pantone127-4] (problem) at (2,5) {Problem};
\node (human) at (2,4) {Human};
\node[shape=rectangle,fill=pantone127-4] (model) at (2,3) {Model};
\node (solver) at (2,2) {Constraint Solver/Search};
\node[shape=rectangle,fill=pantone157-8] (solution) at (2,1) {Solution};
\draw[-] (problem) -- (human);
\draw[->] (human) -- (model);
\draw[-] (model) -- (solver);
\draw[->] (solver) -- (solution);
\end{tikzpicture}
\end{center}
\end{frame}

\begin{frame}
\frametitle{More Realistic}
\begin{center}
\begin{tikzpicture}[xscale=4.0]
\node[shape=rectangle,fill=pantone127-4] (problem) at (2,5) {Problem};
\node (human) at (2,4) {Human};
\node[shape=rectangle,fill=pantone127-4] (model) at (2,3) {Model};
\node (solver) at (2,2) {Constraint Solver/Search};
\node[shape=rectangle,fill=pantone24-8] (hangs) at (1,1) {Hangs};
\node[shape=rectangle,fill=pantone157-8] (solution) at (2,1) {Solution};
\node[shape=rectangle,fill=pantone24-8] (wrong) at (3,1) {Wrong Solution};
\draw[-] (problem) -- (human);
\draw[->] (human) -- (model);
\draw[-] (model) -- (solver);
\draw[->] (solver) -- (hangs);
\draw[->] (solver) -- (solution);
\draw[->] (solver) -- (wrong);
\draw[->] (hangs) -- (1,4) -- (human);
\draw[->] (wrong) -- (3,4) -- (human);
\end{tikzpicture}
\end{center}
\end{frame}

\begin{frame}
\frametitle{Dual Role of Model}
\begin{itemize}
\item Allows Human to Express Problem
\begin{itemize}
\item Close to Problem Domain
\item Constraints as Abstractions
\end{itemize}
\item Allows Solver to Execute
\begin{itemize}
\item Variables as Communication Mechanism
\item Constraints as Algorithms
\end{itemize}
\end{itemize}
\end{frame}

\begin{frame}
\frametitle{Modelling Frameworks}
\begin{itemize}
\item \href{https://www.minizinc.org/}{MiniZinc} (NICTA, Monash University, Australia)
\item \href{https://github.com/eomahony/Numberjack}{NumberJack} (Insight, Ireland) 
\item \href{https://www-users.york.ac.uk/peter.nightingale/savilerow//}{EssencePrime/SavilleRow} (UK) 
\item \href{https://cpmpy.readthedocs.io/en/latest/}{CPMpy} (KU Leuven)
\item Allow use of multiple back-end solvers
\item Compile model into variants for each solver
\item A priori solver independent model(CP, MIP, SAT)
\end{itemize}
\end{frame} 

\begin{frame}
\frametitle{Framework Process}
\begin{center}
\begin{tikzpicture}[xscale=2]
\node[shape=rectangle,fill=pantone127-4] (problem) at (2.5,5) {Problem};
\node (human) at (2.5,4) {Human};
\node[shape=rectangle,fill=pantone127-4] (model) at (2.5,3) {Model};
\node (compile) at (2.5,2) {Compile/Reformulate};
\node (solver1) at (1,1) {CP};
\node (solver2) at (2,1) {MIP};
\node (solver3) at (3,1) {SAT};
\node (solver4) at (4,1) {Other};
\node[shape=rectangle,fill=pantone157-8] (solution1) at (1,0) {Solution};
\node[shape=rectangle,fill=pantone157-8] (solution2) at (2,0) {Solution};
\node[shape=rectangle,fill=pantone157-8] (solution3) at (3,0) {Solution};
\node[shape=rectangle,fill=pantone157-8] (solution4) at (4,0) {Solution};
\draw[-] (problem) -- (human);
\draw[->] (human) -- (model);
\draw[-] (model) -- (compile);
\draw[-] (compile) -- (solver1);
\draw[-] (compile) -- (solver2);
\draw[-] (compile) -- (solver3);
\draw[-] (compile) -- (solver4);
\draw[->] (solver1) -- (solution1);
\draw[->] (solver2) -- (solution2);
\draw[->] (solver3) -- (solution3);
\draw[->] (solver4) -- (solution4);
\end{tikzpicture}
\end{center}
\end{frame}

% \begin{frame}
  % \frametitle{Do It Now!}
  % \begin{itemize}
  % \item Download and install Minizinc
  % \item \url{https://www.minizinc.org/}
  % \end{itemize}
% \end{frame}


\begin{frame}
\frametitle{Why use Puzzles as Examples?}
\begin{itemize}
\item Easy to understand the problem
\item Solvable by hand without specialized knowledge
\item Possible to compare automated to manual solving process
\end{itemize}

\begin{quote}
The puzzle, though inanimate, is presented as a solvable problem without lasting negative consequences, a very low-risk low-reward situation. By being a puzzle, the object is attempting to convince the user that it must be completed.
\end{quote}
{\tiny Source: \href{https://www.everydayrhetoric.com/2020/03/the-rhetoric-of-puzzles/}{Every Day Rhetoric}}
\end{frame}


\mode<all>{
\part{Basic Constraint Propagation}
}
\begin{frame}
\partpage
\end{frame}
\mode<all>{
\input{../sendmore/body}
}

\mode<all>{
\part{Global Constraints}
}
\begin{frame}
\partpage
\end{frame}
\mode<all>{
\input{../sudoku/body}
}

\clearpage
\mode<all>{
\part{Customizing Search}
}
\begin{frame}
\partpage
\end{frame}
\mode<all>{
\input{../nqueen/body}
}


%% \clearpage
%% \mode<all>{
%%   \part{Problem Decomposition}
%% }
%% \begin{frame}
%%   \partpage
%% \end{frame}
%% \mode<all>{
%%   \input{../party/body}
%%   }

%% \clearpage
%% \mode<all>{
%%   \part{More Global Constraints}
%% }
%% \begin{frame}
%%   \partpage
%% \end{frame}
%% \mode<all>{
%%   \input{../car/body}
%%   }

%% \clearpage
%% \mode<all>{
%%   \part{Methodology}
%% }
%% \begin{frame}
%%   \partpage
%% \end{frame}
%% \mode<all>{
%%   \input{../methodology/body}
%%   }

\clearpage
\mode<all>{
\part{What is missing?}
}
\begin{frame}
\partpage
\end{frame}

\begin{frame}
\frametitle{Many Specialized Topics}
\begin{itemize}
\item How to design efficient core engine
\item Hybrids with LP/MIP tools
\item Hybrids with SAT
\item Symmetry breaking
\item Use of MDD/BDD to encode sets of solutions
\item High level modelling tools
\item Debugging/visualization
\end{itemize}
\end{frame}

\begin{frame}
\frametitle{Reformulation}
\begin{itemize}
\item Just because the user has modelled it this way, it doesn't mean we have to solve it that way
\begin{itemize}
\item Replace some constraint(s) by other, equivalent constraints
\item Because we don't have that constraint in our system
\item For performance
\end{itemize}
\end{itemize}
\end{frame}

\begin{frame}
\frametitle{Learning}
\begin{itemize}
\item While solving the problem we can learn how to strengthen the model/search 
\begin{itemize}
\item Understand which constraints/method contribute to propagation and change schedule
\item Learn no-good constraints by explaining failure
\item Adapt search strategy based on search experience
\end{itemize}
\end{itemize}
\end{frame}

%% \begin{frame}
%% \frametitle{Refined Process}
%% \begin{center}
%% \begin{tikzpicture}
%% \node[shape=rectangle,fill=pantone127-4] (problem) at (3,7) {Problem};
%% \node (human) at (3,6) {Human};
%% \node[shape=rectangle,fill=pantone127-4] (model) at (3,5) {User Model};
%% \node (reformulation) at (3,4) {Reformulation};
%% \node[shape=rectangle,fill=pantone127-4] (implementation) at (3,3) {Implementation Model};
%% \node (solver) at (3,2) {Constraint Solver/Search};
%% \node[shape=rectangle,fill=pantone157-8] (solution) at (3,1) {Solution};
%% \draw[-] (problem) -- (human);
%% \draw[->] (human) -- (model);
%% \draw[-] (model) -- (reformulation);
%% \draw[->] (reformulation) -- (implementation);
%% \draw[-] (implementation) -- (solver);
%% \draw[->] (solver) -- (0,2) -- (0,3) -- (implementation);
%% \draw[->] (solver) -- (solution);
%% \end{tikzpicture}
%% \end{center}
%% \end{frame}

%\begin{frame}
%\frametitle{What is CP actually used for?}
%\url{http://hsimonis.wordpress.com}
%\includegraphics[width=10cm]{../introduction/images/blog}
%\end{frame}

%% \begin{frame}
%% \frametitle{Exercises}
%% \begin{itemize}
%% \item Install Constraint Solver of your Choice
%% \item MiniZinc (MiningZinc) will be used in later sessions
%% \item Run Sendmoremoney, Sudoku, Queens examples
%% \item Can you choose which propagator to use for alldifferent?
%% \item Can you use different search strategies?
%% \end{itemize}
%% \end{frame}

\begin{frame}
\frametitle{More Learning Resources}
\begin{itemize}
\item Survey of Methods, Resources, and Formats for Teaching Constraint Programming
\begin{itemize}
\item by Tejas Santanam, Helmut Simonis
\item \url{https://doi.org/10.48550/arXiv.2403.12717}
\item Based on survey of community for WTCP 2023
\item \url{https://hsimonis.github.io/WTCP2023/}
\end{itemize}
\end{itemize}
\end{frame}


}


%% \clearpage
%% \mode<all>{
%%   \part{Problem Decomposition}
%% }
%% \begin{frame}
%%   \partpage
%% \end{frame}
%% \mode<all>{
%%   \begin{frame}
\frametitle{Objectives}
\begin{itemize}
\item Overview of Core Constraint Programming
\item Three Main Concepts
\begin{itemize}
\item Constraint Propagation
\item Global Constraints
\item Customizing Search
\end{itemize}
%\item Get Some Experience with MiniZinc
\item Topics will be treated in more detail in later parts of the school
\item Based on Examples, not Formal Description
\end{itemize}
\end{frame}

\begin{frame}
\frametitle{Outline}
\begin{itemize}
\item Why Constraint Programming?
\item Constraint Propagation
\item Global Constraints
\item Customizing Search
%\item Choices in Modelling
%\item Decomposition
%\item More Global Constraints
%\item Methodology
%\item What is missing?
\end{itemize}
\end{frame}

% \begin{frame}
  % \frametitle{Using MiniZinc IDE}
  % \begin{itemize}
  % \item Developed in the Australian NICTA project
  % \item Maintained by Monash University
  % \item Modelling tool with multiple back-end solvers
  % \item Available from \url{https://www.minizinc.org/}
  % \end{itemize}
% \end{frame}


% \begin{frame}
% \frametitle{Examples in ECLiPSe}
% \begin{itemize}
% \item Open sourced constraint programming language
% \item Development goes back to 1985
% \item ECRC, ICL, IC-Parc, PTL, Cisco
% \item \url{https://eclipseclp.org/}
% \item Specialities
% \begin{itemize}
% \item Develop new solvers for specific domains
% \item Integration with MIP
% \end{itemize}
% \item Not included in bundled MiniZinc IDE
% \item Specialized visualization tools used here
  % \begin{itemize}
  % \item CP-Viz, Simonis et al. 2010
    % \end{itemize}
% \end{itemize}
% \end{frame}

\begin{frame}
\frametitle{Tutorial Based on ECLiPSe ELearning Course}
\begin{itemize}
\item Self-study course in constraint programming
\item Supported by Cisco Systems and Silicon Valley Community Foundation
\item Multi-media format, video lectures, slides, handout etc
\item {\small \url{https://eclipseclp.org/ELearning/index.html}}
\end{itemize}
\end{frame}

\begin{frame}
\frametitle{Also Part of CRT-AI Constraint Week}
\begin{itemize}
\item Annual one week course on CP and Optimization in Ireland
\item Part of national training program for PhD students in AI
\item \url{https://www.crt-ai.ie/}
\end{itemize}
\end{frame}



\begin{frame}
\frametitle{Constraint Programming - in a nutshell}
\begin{itemize}
\item Declarative description of problems with
\begin{itemize}
\item {\em Variables} which range over (finite) sets of values
\item {\em Constraints} over subsets of variables which restrict possible value combinations
\item A {\em solution} is a value assignment which satisfies all constraints
\end{itemize}

\item Constraint propagation/reasoning
\begin{itemize}
\item Removing inconsistent values for variables
\item Detect failure if constraint can not be satisfied
\item Interaction of constraints via shared variables
\item Incomplete
\end{itemize}

\item Search
\begin{itemize}
\item User controlled assignment of values to variables
\item Each step triggers constraint propagation 
\end{itemize}
\item Different domains require/allow different methods
\end{itemize}
\end{frame}

\begin{frame}
  \frametitle{Constraint Programming is Different}
  \begin{itemize}
  \item Declarative Programming
    \begin{itemize}
    \item Concentrate on what you want
      \item Not how to get there
      \item Program != Algorithm
      \item Program = Model
    \end{itemize}
    \item Applied to Combinatorial Problems
      \begin{itemize}
        \item No complete polynomial algorithms known (exist?)
        \item CP less ad-hoc than heuristics
        \item Models can evolve
  \end{itemize}
  \end{itemize}
  \end{frame}
    
\begin{frame}
  \frametitle{A Subtractive Process}
  \begin{textblock}{4}(8,-3)
    \includegraphics[width=4cm]{../introduction/images/stages}
  \end{textblock}
  \vfill
  \begin{quote}
    ``Oh, bosh, as Mr. Ruskin says. Sculpture, per se, is the simplest thing in the world. All you have to do is to take a big chunk of marble and a hammer and chisel, make up your mind what you are about to create and chip off all the marble you don't want.''-Paris Gaulois.
  \end{quote}
  
  {\tiny Source: \url{https://quoteinvestigator.com/2014/06/22/chip-away/}}
\end{frame}

\begin{frame}
\frametitle{Basic Process}
\begin{center}
\begin{tikzpicture}
\node[shape=rectangle,fill=pantone127-4] (problem) at (2,5) {Problem};
\node (human) at (2,4) {Human};
\node[shape=rectangle,fill=pantone127-4] (model) at (2,3) {Model};
\node (solver) at (2,2) {Constraint Solver/Search};
\node[shape=rectangle,fill=pantone157-8] (solution) at (2,1) {Solution};
\draw[-] (problem) -- (human);
\draw[->] (human) -- (model);
\draw[-] (model) -- (solver);
\draw[->] (solver) -- (solution);
\end{tikzpicture}
\end{center}
\end{frame}

\begin{frame}
\frametitle{More Realistic}
\begin{center}
\begin{tikzpicture}[xscale=4.0]
\node[shape=rectangle,fill=pantone127-4] (problem) at (2,5) {Problem};
\node (human) at (2,4) {Human};
\node[shape=rectangle,fill=pantone127-4] (model) at (2,3) {Model};
\node (solver) at (2,2) {Constraint Solver/Search};
\node[shape=rectangle,fill=pantone24-8] (hangs) at (1,1) {Hangs};
\node[shape=rectangle,fill=pantone157-8] (solution) at (2,1) {Solution};
\node[shape=rectangle,fill=pantone24-8] (wrong) at (3,1) {Wrong Solution};
\draw[-] (problem) -- (human);
\draw[->] (human) -- (model);
\draw[-] (model) -- (solver);
\draw[->] (solver) -- (hangs);
\draw[->] (solver) -- (solution);
\draw[->] (solver) -- (wrong);
\draw[->] (hangs) -- (1,4) -- (human);
\draw[->] (wrong) -- (3,4) -- (human);
\end{tikzpicture}
\end{center}
\end{frame}

\begin{frame}
\frametitle{Dual Role of Model}
\begin{itemize}
\item Allows Human to Express Problem
\begin{itemize}
\item Close to Problem Domain
\item Constraints as Abstractions
\end{itemize}
\item Allows Solver to Execute
\begin{itemize}
\item Variables as Communication Mechanism
\item Constraints as Algorithms
\end{itemize}
\end{itemize}
\end{frame}

\begin{frame}
\frametitle{Modelling Frameworks}
\begin{itemize}
\item \href{https://www.minizinc.org/}{MiniZinc} (NICTA, Monash University, Australia)
\item \href{https://github.com/eomahony/Numberjack}{NumberJack} (Insight, Ireland) 
\item \href{https://www-users.york.ac.uk/peter.nightingale/savilerow//}{EssencePrime/SavilleRow} (UK) 
\item \href{https://cpmpy.readthedocs.io/en/latest/}{CPMpy} (KU Leuven)
\item Allow use of multiple back-end solvers
\item Compile model into variants for each solver
\item A priori solver independent model(CP, MIP, SAT)
\end{itemize}
\end{frame} 

\begin{frame}
\frametitle{Framework Process}
\begin{center}
\begin{tikzpicture}[xscale=2]
\node[shape=rectangle,fill=pantone127-4] (problem) at (2.5,5) {Problem};
\node (human) at (2.5,4) {Human};
\node[shape=rectangle,fill=pantone127-4] (model) at (2.5,3) {Model};
\node (compile) at (2.5,2) {Compile/Reformulate};
\node (solver1) at (1,1) {CP};
\node (solver2) at (2,1) {MIP};
\node (solver3) at (3,1) {SAT};
\node (solver4) at (4,1) {Other};
\node[shape=rectangle,fill=pantone157-8] (solution1) at (1,0) {Solution};
\node[shape=rectangle,fill=pantone157-8] (solution2) at (2,0) {Solution};
\node[shape=rectangle,fill=pantone157-8] (solution3) at (3,0) {Solution};
\node[shape=rectangle,fill=pantone157-8] (solution4) at (4,0) {Solution};
\draw[-] (problem) -- (human);
\draw[->] (human) -- (model);
\draw[-] (model) -- (compile);
\draw[-] (compile) -- (solver1);
\draw[-] (compile) -- (solver2);
\draw[-] (compile) -- (solver3);
\draw[-] (compile) -- (solver4);
\draw[->] (solver1) -- (solution1);
\draw[->] (solver2) -- (solution2);
\draw[->] (solver3) -- (solution3);
\draw[->] (solver4) -- (solution4);
\end{tikzpicture}
\end{center}
\end{frame}

% \begin{frame}
  % \frametitle{Do It Now!}
  % \begin{itemize}
  % \item Download and install Minizinc
  % \item \url{https://www.minizinc.org/}
  % \end{itemize}
% \end{frame}


\begin{frame}
\frametitle{Why use Puzzles as Examples?}
\begin{itemize}
\item Easy to understand the problem
\item Solvable by hand without specialized knowledge
\item Possible to compare automated to manual solving process
\end{itemize}

\begin{quote}
The puzzle, though inanimate, is presented as a solvable problem without lasting negative consequences, a very low-risk low-reward situation. By being a puzzle, the object is attempting to convince the user that it must be completed.
\end{quote}
{\tiny Source: \href{https://www.everydayrhetoric.com/2020/03/the-rhetoric-of-puzzles/}{Every Day Rhetoric}}
\end{frame}


\mode<all>{
\part{Basic Constraint Propagation}
}
\begin{frame}
\partpage
\end{frame}
\mode<all>{
\input{../sendmore/body}
}

\mode<all>{
\part{Global Constraints}
}
\begin{frame}
\partpage
\end{frame}
\mode<all>{
\input{../sudoku/body}
}

\clearpage
\mode<all>{
\part{Customizing Search}
}
\begin{frame}
\partpage
\end{frame}
\mode<all>{
\input{../nqueen/body}
}


%% \clearpage
%% \mode<all>{
%%   \part{Problem Decomposition}
%% }
%% \begin{frame}
%%   \partpage
%% \end{frame}
%% \mode<all>{
%%   \input{../party/body}
%%   }

%% \clearpage
%% \mode<all>{
%%   \part{More Global Constraints}
%% }
%% \begin{frame}
%%   \partpage
%% \end{frame}
%% \mode<all>{
%%   \input{../car/body}
%%   }

%% \clearpage
%% \mode<all>{
%%   \part{Methodology}
%% }
%% \begin{frame}
%%   \partpage
%% \end{frame}
%% \mode<all>{
%%   \input{../methodology/body}
%%   }

\clearpage
\mode<all>{
\part{What is missing?}
}
\begin{frame}
\partpage
\end{frame}

\begin{frame}
\frametitle{Many Specialized Topics}
\begin{itemize}
\item How to design efficient core engine
\item Hybrids with LP/MIP tools
\item Hybrids with SAT
\item Symmetry breaking
\item Use of MDD/BDD to encode sets of solutions
\item High level modelling tools
\item Debugging/visualization
\end{itemize}
\end{frame}

\begin{frame}
\frametitle{Reformulation}
\begin{itemize}
\item Just because the user has modelled it this way, it doesn't mean we have to solve it that way
\begin{itemize}
\item Replace some constraint(s) by other, equivalent constraints
\item Because we don't have that constraint in our system
\item For performance
\end{itemize}
\end{itemize}
\end{frame}

\begin{frame}
\frametitle{Learning}
\begin{itemize}
\item While solving the problem we can learn how to strengthen the model/search 
\begin{itemize}
\item Understand which constraints/method contribute to propagation and change schedule
\item Learn no-good constraints by explaining failure
\item Adapt search strategy based on search experience
\end{itemize}
\end{itemize}
\end{frame}

%% \begin{frame}
%% \frametitle{Refined Process}
%% \begin{center}
%% \begin{tikzpicture}
%% \node[shape=rectangle,fill=pantone127-4] (problem) at (3,7) {Problem};
%% \node (human) at (3,6) {Human};
%% \node[shape=rectangle,fill=pantone127-4] (model) at (3,5) {User Model};
%% \node (reformulation) at (3,4) {Reformulation};
%% \node[shape=rectangle,fill=pantone127-4] (implementation) at (3,3) {Implementation Model};
%% \node (solver) at (3,2) {Constraint Solver/Search};
%% \node[shape=rectangle,fill=pantone157-8] (solution) at (3,1) {Solution};
%% \draw[-] (problem) -- (human);
%% \draw[->] (human) -- (model);
%% \draw[-] (model) -- (reformulation);
%% \draw[->] (reformulation) -- (implementation);
%% \draw[-] (implementation) -- (solver);
%% \draw[->] (solver) -- (0,2) -- (0,3) -- (implementation);
%% \draw[->] (solver) -- (solution);
%% \end{tikzpicture}
%% \end{center}
%% \end{frame}

%\begin{frame}
%\frametitle{What is CP actually used for?}
%\url{http://hsimonis.wordpress.com}
%\includegraphics[width=10cm]{../introduction/images/blog}
%\end{frame}

%% \begin{frame}
%% \frametitle{Exercises}
%% \begin{itemize}
%% \item Install Constraint Solver of your Choice
%% \item MiniZinc (MiningZinc) will be used in later sessions
%% \item Run Sendmoremoney, Sudoku, Queens examples
%% \item Can you choose which propagator to use for alldifferent?
%% \item Can you use different search strategies?
%% \end{itemize}
%% \end{frame}

\begin{frame}
\frametitle{More Learning Resources}
\begin{itemize}
\item Survey of Methods, Resources, and Formats for Teaching Constraint Programming
\begin{itemize}
\item by Tejas Santanam, Helmut Simonis
\item \url{https://doi.org/10.48550/arXiv.2403.12717}
\item Based on survey of community for WTCP 2023
\item \url{https://hsimonis.github.io/WTCP2023/}
\end{itemize}
\end{itemize}
\end{frame}


%%   }

%% \clearpage
%% \mode<all>{
%%   \part{More Global Constraints}
%% }
%% \begin{frame}
%%   \partpage
%% \end{frame}
%% \mode<all>{
%%   \begin{frame}
\frametitle{Objectives}
\begin{itemize}
\item Overview of Core Constraint Programming
\item Three Main Concepts
\begin{itemize}
\item Constraint Propagation
\item Global Constraints
\item Customizing Search
\end{itemize}
%\item Get Some Experience with MiniZinc
\item Topics will be treated in more detail in later parts of the school
\item Based on Examples, not Formal Description
\end{itemize}
\end{frame}

\begin{frame}
\frametitle{Outline}
\begin{itemize}
\item Why Constraint Programming?
\item Constraint Propagation
\item Global Constraints
\item Customizing Search
%\item Choices in Modelling
%\item Decomposition
%\item More Global Constraints
%\item Methodology
%\item What is missing?
\end{itemize}
\end{frame}

% \begin{frame}
  % \frametitle{Using MiniZinc IDE}
  % \begin{itemize}
  % \item Developed in the Australian NICTA project
  % \item Maintained by Monash University
  % \item Modelling tool with multiple back-end solvers
  % \item Available from \url{https://www.minizinc.org/}
  % \end{itemize}
% \end{frame}


% \begin{frame}
% \frametitle{Examples in ECLiPSe}
% \begin{itemize}
% \item Open sourced constraint programming language
% \item Development goes back to 1985
% \item ECRC, ICL, IC-Parc, PTL, Cisco
% \item \url{https://eclipseclp.org/}
% \item Specialities
% \begin{itemize}
% \item Develop new solvers for specific domains
% \item Integration with MIP
% \end{itemize}
% \item Not included in bundled MiniZinc IDE
% \item Specialized visualization tools used here
  % \begin{itemize}
  % \item CP-Viz, Simonis et al. 2010
    % \end{itemize}
% \end{itemize}
% \end{frame}

\begin{frame}
\frametitle{Tutorial Based on ECLiPSe ELearning Course}
\begin{itemize}
\item Self-study course in constraint programming
\item Supported by Cisco Systems and Silicon Valley Community Foundation
\item Multi-media format, video lectures, slides, handout etc
\item {\small \url{https://eclipseclp.org/ELearning/index.html}}
\end{itemize}
\end{frame}

\begin{frame}
\frametitle{Also Part of CRT-AI Constraint Week}
\begin{itemize}
\item Annual one week course on CP and Optimization in Ireland
\item Part of national training program for PhD students in AI
\item \url{https://www.crt-ai.ie/}
\end{itemize}
\end{frame}



\begin{frame}
\frametitle{Constraint Programming - in a nutshell}
\begin{itemize}
\item Declarative description of problems with
\begin{itemize}
\item {\em Variables} which range over (finite) sets of values
\item {\em Constraints} over subsets of variables which restrict possible value combinations
\item A {\em solution} is a value assignment which satisfies all constraints
\end{itemize}

\item Constraint propagation/reasoning
\begin{itemize}
\item Removing inconsistent values for variables
\item Detect failure if constraint can not be satisfied
\item Interaction of constraints via shared variables
\item Incomplete
\end{itemize}

\item Search
\begin{itemize}
\item User controlled assignment of values to variables
\item Each step triggers constraint propagation 
\end{itemize}
\item Different domains require/allow different methods
\end{itemize}
\end{frame}

\begin{frame}
  \frametitle{Constraint Programming is Different}
  \begin{itemize}
  \item Declarative Programming
    \begin{itemize}
    \item Concentrate on what you want
      \item Not how to get there
      \item Program != Algorithm
      \item Program = Model
    \end{itemize}
    \item Applied to Combinatorial Problems
      \begin{itemize}
        \item No complete polynomial algorithms known (exist?)
        \item CP less ad-hoc than heuristics
        \item Models can evolve
  \end{itemize}
  \end{itemize}
  \end{frame}
    
\begin{frame}
  \frametitle{A Subtractive Process}
  \begin{textblock}{4}(8,-3)
    \includegraphics[width=4cm]{../introduction/images/stages}
  \end{textblock}
  \vfill
  \begin{quote}
    ``Oh, bosh, as Mr. Ruskin says. Sculpture, per se, is the simplest thing in the world. All you have to do is to take a big chunk of marble and a hammer and chisel, make up your mind what you are about to create and chip off all the marble you don't want.''-Paris Gaulois.
  \end{quote}
  
  {\tiny Source: \url{https://quoteinvestigator.com/2014/06/22/chip-away/}}
\end{frame}

\begin{frame}
\frametitle{Basic Process}
\begin{center}
\begin{tikzpicture}
\node[shape=rectangle,fill=pantone127-4] (problem) at (2,5) {Problem};
\node (human) at (2,4) {Human};
\node[shape=rectangle,fill=pantone127-4] (model) at (2,3) {Model};
\node (solver) at (2,2) {Constraint Solver/Search};
\node[shape=rectangle,fill=pantone157-8] (solution) at (2,1) {Solution};
\draw[-] (problem) -- (human);
\draw[->] (human) -- (model);
\draw[-] (model) -- (solver);
\draw[->] (solver) -- (solution);
\end{tikzpicture}
\end{center}
\end{frame}

\begin{frame}
\frametitle{More Realistic}
\begin{center}
\begin{tikzpicture}[xscale=4.0]
\node[shape=rectangle,fill=pantone127-4] (problem) at (2,5) {Problem};
\node (human) at (2,4) {Human};
\node[shape=rectangle,fill=pantone127-4] (model) at (2,3) {Model};
\node (solver) at (2,2) {Constraint Solver/Search};
\node[shape=rectangle,fill=pantone24-8] (hangs) at (1,1) {Hangs};
\node[shape=rectangle,fill=pantone157-8] (solution) at (2,1) {Solution};
\node[shape=rectangle,fill=pantone24-8] (wrong) at (3,1) {Wrong Solution};
\draw[-] (problem) -- (human);
\draw[->] (human) -- (model);
\draw[-] (model) -- (solver);
\draw[->] (solver) -- (hangs);
\draw[->] (solver) -- (solution);
\draw[->] (solver) -- (wrong);
\draw[->] (hangs) -- (1,4) -- (human);
\draw[->] (wrong) -- (3,4) -- (human);
\end{tikzpicture}
\end{center}
\end{frame}

\begin{frame}
\frametitle{Dual Role of Model}
\begin{itemize}
\item Allows Human to Express Problem
\begin{itemize}
\item Close to Problem Domain
\item Constraints as Abstractions
\end{itemize}
\item Allows Solver to Execute
\begin{itemize}
\item Variables as Communication Mechanism
\item Constraints as Algorithms
\end{itemize}
\end{itemize}
\end{frame}

\begin{frame}
\frametitle{Modelling Frameworks}
\begin{itemize}
\item \href{https://www.minizinc.org/}{MiniZinc} (NICTA, Monash University, Australia)
\item \href{https://github.com/eomahony/Numberjack}{NumberJack} (Insight, Ireland) 
\item \href{https://www-users.york.ac.uk/peter.nightingale/savilerow//}{EssencePrime/SavilleRow} (UK) 
\item \href{https://cpmpy.readthedocs.io/en/latest/}{CPMpy} (KU Leuven)
\item Allow use of multiple back-end solvers
\item Compile model into variants for each solver
\item A priori solver independent model(CP, MIP, SAT)
\end{itemize}
\end{frame} 

\begin{frame}
\frametitle{Framework Process}
\begin{center}
\begin{tikzpicture}[xscale=2]
\node[shape=rectangle,fill=pantone127-4] (problem) at (2.5,5) {Problem};
\node (human) at (2.5,4) {Human};
\node[shape=rectangle,fill=pantone127-4] (model) at (2.5,3) {Model};
\node (compile) at (2.5,2) {Compile/Reformulate};
\node (solver1) at (1,1) {CP};
\node (solver2) at (2,1) {MIP};
\node (solver3) at (3,1) {SAT};
\node (solver4) at (4,1) {Other};
\node[shape=rectangle,fill=pantone157-8] (solution1) at (1,0) {Solution};
\node[shape=rectangle,fill=pantone157-8] (solution2) at (2,0) {Solution};
\node[shape=rectangle,fill=pantone157-8] (solution3) at (3,0) {Solution};
\node[shape=rectangle,fill=pantone157-8] (solution4) at (4,0) {Solution};
\draw[-] (problem) -- (human);
\draw[->] (human) -- (model);
\draw[-] (model) -- (compile);
\draw[-] (compile) -- (solver1);
\draw[-] (compile) -- (solver2);
\draw[-] (compile) -- (solver3);
\draw[-] (compile) -- (solver4);
\draw[->] (solver1) -- (solution1);
\draw[->] (solver2) -- (solution2);
\draw[->] (solver3) -- (solution3);
\draw[->] (solver4) -- (solution4);
\end{tikzpicture}
\end{center}
\end{frame}

% \begin{frame}
  % \frametitle{Do It Now!}
  % \begin{itemize}
  % \item Download and install Minizinc
  % \item \url{https://www.minizinc.org/}
  % \end{itemize}
% \end{frame}


\begin{frame}
\frametitle{Why use Puzzles as Examples?}
\begin{itemize}
\item Easy to understand the problem
\item Solvable by hand without specialized knowledge
\item Possible to compare automated to manual solving process
\end{itemize}

\begin{quote}
The puzzle, though inanimate, is presented as a solvable problem without lasting negative consequences, a very low-risk low-reward situation. By being a puzzle, the object is attempting to convince the user that it must be completed.
\end{quote}
{\tiny Source: \href{https://www.everydayrhetoric.com/2020/03/the-rhetoric-of-puzzles/}{Every Day Rhetoric}}
\end{frame}


\mode<all>{
\part{Basic Constraint Propagation}
}
\begin{frame}
\partpage
\end{frame}
\mode<all>{
\input{../sendmore/body}
}

\mode<all>{
\part{Global Constraints}
}
\begin{frame}
\partpage
\end{frame}
\mode<all>{
\input{../sudoku/body}
}

\clearpage
\mode<all>{
\part{Customizing Search}
}
\begin{frame}
\partpage
\end{frame}
\mode<all>{
\input{../nqueen/body}
}


%% \clearpage
%% \mode<all>{
%%   \part{Problem Decomposition}
%% }
%% \begin{frame}
%%   \partpage
%% \end{frame}
%% \mode<all>{
%%   \input{../party/body}
%%   }

%% \clearpage
%% \mode<all>{
%%   \part{More Global Constraints}
%% }
%% \begin{frame}
%%   \partpage
%% \end{frame}
%% \mode<all>{
%%   \input{../car/body}
%%   }

%% \clearpage
%% \mode<all>{
%%   \part{Methodology}
%% }
%% \begin{frame}
%%   \partpage
%% \end{frame}
%% \mode<all>{
%%   \input{../methodology/body}
%%   }

\clearpage
\mode<all>{
\part{What is missing?}
}
\begin{frame}
\partpage
\end{frame}

\begin{frame}
\frametitle{Many Specialized Topics}
\begin{itemize}
\item How to design efficient core engine
\item Hybrids with LP/MIP tools
\item Hybrids with SAT
\item Symmetry breaking
\item Use of MDD/BDD to encode sets of solutions
\item High level modelling tools
\item Debugging/visualization
\end{itemize}
\end{frame}

\begin{frame}
\frametitle{Reformulation}
\begin{itemize}
\item Just because the user has modelled it this way, it doesn't mean we have to solve it that way
\begin{itemize}
\item Replace some constraint(s) by other, equivalent constraints
\item Because we don't have that constraint in our system
\item For performance
\end{itemize}
\end{itemize}
\end{frame}

\begin{frame}
\frametitle{Learning}
\begin{itemize}
\item While solving the problem we can learn how to strengthen the model/search 
\begin{itemize}
\item Understand which constraints/method contribute to propagation and change schedule
\item Learn no-good constraints by explaining failure
\item Adapt search strategy based on search experience
\end{itemize}
\end{itemize}
\end{frame}

%% \begin{frame}
%% \frametitle{Refined Process}
%% \begin{center}
%% \begin{tikzpicture}
%% \node[shape=rectangle,fill=pantone127-4] (problem) at (3,7) {Problem};
%% \node (human) at (3,6) {Human};
%% \node[shape=rectangle,fill=pantone127-4] (model) at (3,5) {User Model};
%% \node (reformulation) at (3,4) {Reformulation};
%% \node[shape=rectangle,fill=pantone127-4] (implementation) at (3,3) {Implementation Model};
%% \node (solver) at (3,2) {Constraint Solver/Search};
%% \node[shape=rectangle,fill=pantone157-8] (solution) at (3,1) {Solution};
%% \draw[-] (problem) -- (human);
%% \draw[->] (human) -- (model);
%% \draw[-] (model) -- (reformulation);
%% \draw[->] (reformulation) -- (implementation);
%% \draw[-] (implementation) -- (solver);
%% \draw[->] (solver) -- (0,2) -- (0,3) -- (implementation);
%% \draw[->] (solver) -- (solution);
%% \end{tikzpicture}
%% \end{center}
%% \end{frame}

%\begin{frame}
%\frametitle{What is CP actually used for?}
%\url{http://hsimonis.wordpress.com}
%\includegraphics[width=10cm]{../introduction/images/blog}
%\end{frame}

%% \begin{frame}
%% \frametitle{Exercises}
%% \begin{itemize}
%% \item Install Constraint Solver of your Choice
%% \item MiniZinc (MiningZinc) will be used in later sessions
%% \item Run Sendmoremoney, Sudoku, Queens examples
%% \item Can you choose which propagator to use for alldifferent?
%% \item Can you use different search strategies?
%% \end{itemize}
%% \end{frame}

\begin{frame}
\frametitle{More Learning Resources}
\begin{itemize}
\item Survey of Methods, Resources, and Formats for Teaching Constraint Programming
\begin{itemize}
\item by Tejas Santanam, Helmut Simonis
\item \url{https://doi.org/10.48550/arXiv.2403.12717}
\item Based on survey of community for WTCP 2023
\item \url{https://hsimonis.github.io/WTCP2023/}
\end{itemize}
\end{itemize}
\end{frame}


%%   }

%% \clearpage
%% \mode<all>{
%%   \part{Methodology}
%% }
%% \begin{frame}
%%   \partpage
%% \end{frame}
%% \mode<all>{
%%   \begin{frame}
\frametitle{Objectives}
\begin{itemize}
\item Overview of Core Constraint Programming
\item Three Main Concepts
\begin{itemize}
\item Constraint Propagation
\item Global Constraints
\item Customizing Search
\end{itemize}
%\item Get Some Experience with MiniZinc
\item Topics will be treated in more detail in later parts of the school
\item Based on Examples, not Formal Description
\end{itemize}
\end{frame}

\begin{frame}
\frametitle{Outline}
\begin{itemize}
\item Why Constraint Programming?
\item Constraint Propagation
\item Global Constraints
\item Customizing Search
%\item Choices in Modelling
%\item Decomposition
%\item More Global Constraints
%\item Methodology
%\item What is missing?
\end{itemize}
\end{frame}

% \begin{frame}
  % \frametitle{Using MiniZinc IDE}
  % \begin{itemize}
  % \item Developed in the Australian NICTA project
  % \item Maintained by Monash University
  % \item Modelling tool with multiple back-end solvers
  % \item Available from \url{https://www.minizinc.org/}
  % \end{itemize}
% \end{frame}


% \begin{frame}
% \frametitle{Examples in ECLiPSe}
% \begin{itemize}
% \item Open sourced constraint programming language
% \item Development goes back to 1985
% \item ECRC, ICL, IC-Parc, PTL, Cisco
% \item \url{https://eclipseclp.org/}
% \item Specialities
% \begin{itemize}
% \item Develop new solvers for specific domains
% \item Integration with MIP
% \end{itemize}
% \item Not included in bundled MiniZinc IDE
% \item Specialized visualization tools used here
  % \begin{itemize}
  % \item CP-Viz, Simonis et al. 2010
    % \end{itemize}
% \end{itemize}
% \end{frame}

\begin{frame}
\frametitle{Tutorial Based on ECLiPSe ELearning Course}
\begin{itemize}
\item Self-study course in constraint programming
\item Supported by Cisco Systems and Silicon Valley Community Foundation
\item Multi-media format, video lectures, slides, handout etc
\item {\small \url{https://eclipseclp.org/ELearning/index.html}}
\end{itemize}
\end{frame}

\begin{frame}
\frametitle{Also Part of CRT-AI Constraint Week}
\begin{itemize}
\item Annual one week course on CP and Optimization in Ireland
\item Part of national training program for PhD students in AI
\item \url{https://www.crt-ai.ie/}
\end{itemize}
\end{frame}



\begin{frame}
\frametitle{Constraint Programming - in a nutshell}
\begin{itemize}
\item Declarative description of problems with
\begin{itemize}
\item {\em Variables} which range over (finite) sets of values
\item {\em Constraints} over subsets of variables which restrict possible value combinations
\item A {\em solution} is a value assignment which satisfies all constraints
\end{itemize}

\item Constraint propagation/reasoning
\begin{itemize}
\item Removing inconsistent values for variables
\item Detect failure if constraint can not be satisfied
\item Interaction of constraints via shared variables
\item Incomplete
\end{itemize}

\item Search
\begin{itemize}
\item User controlled assignment of values to variables
\item Each step triggers constraint propagation 
\end{itemize}
\item Different domains require/allow different methods
\end{itemize}
\end{frame}

\begin{frame}
  \frametitle{Constraint Programming is Different}
  \begin{itemize}
  \item Declarative Programming
    \begin{itemize}
    \item Concentrate on what you want
      \item Not how to get there
      \item Program != Algorithm
      \item Program = Model
    \end{itemize}
    \item Applied to Combinatorial Problems
      \begin{itemize}
        \item No complete polynomial algorithms known (exist?)
        \item CP less ad-hoc than heuristics
        \item Models can evolve
  \end{itemize}
  \end{itemize}
  \end{frame}
    
\begin{frame}
  \frametitle{A Subtractive Process}
  \begin{textblock}{4}(8,-3)
    \includegraphics[width=4cm]{../introduction/images/stages}
  \end{textblock}
  \vfill
  \begin{quote}
    ``Oh, bosh, as Mr. Ruskin says. Sculpture, per se, is the simplest thing in the world. All you have to do is to take a big chunk of marble and a hammer and chisel, make up your mind what you are about to create and chip off all the marble you don't want.''-Paris Gaulois.
  \end{quote}
  
  {\tiny Source: \url{https://quoteinvestigator.com/2014/06/22/chip-away/}}
\end{frame}

\begin{frame}
\frametitle{Basic Process}
\begin{center}
\begin{tikzpicture}
\node[shape=rectangle,fill=pantone127-4] (problem) at (2,5) {Problem};
\node (human) at (2,4) {Human};
\node[shape=rectangle,fill=pantone127-4] (model) at (2,3) {Model};
\node (solver) at (2,2) {Constraint Solver/Search};
\node[shape=rectangle,fill=pantone157-8] (solution) at (2,1) {Solution};
\draw[-] (problem) -- (human);
\draw[->] (human) -- (model);
\draw[-] (model) -- (solver);
\draw[->] (solver) -- (solution);
\end{tikzpicture}
\end{center}
\end{frame}

\begin{frame}
\frametitle{More Realistic}
\begin{center}
\begin{tikzpicture}[xscale=4.0]
\node[shape=rectangle,fill=pantone127-4] (problem) at (2,5) {Problem};
\node (human) at (2,4) {Human};
\node[shape=rectangle,fill=pantone127-4] (model) at (2,3) {Model};
\node (solver) at (2,2) {Constraint Solver/Search};
\node[shape=rectangle,fill=pantone24-8] (hangs) at (1,1) {Hangs};
\node[shape=rectangle,fill=pantone157-8] (solution) at (2,1) {Solution};
\node[shape=rectangle,fill=pantone24-8] (wrong) at (3,1) {Wrong Solution};
\draw[-] (problem) -- (human);
\draw[->] (human) -- (model);
\draw[-] (model) -- (solver);
\draw[->] (solver) -- (hangs);
\draw[->] (solver) -- (solution);
\draw[->] (solver) -- (wrong);
\draw[->] (hangs) -- (1,4) -- (human);
\draw[->] (wrong) -- (3,4) -- (human);
\end{tikzpicture}
\end{center}
\end{frame}

\begin{frame}
\frametitle{Dual Role of Model}
\begin{itemize}
\item Allows Human to Express Problem
\begin{itemize}
\item Close to Problem Domain
\item Constraints as Abstractions
\end{itemize}
\item Allows Solver to Execute
\begin{itemize}
\item Variables as Communication Mechanism
\item Constraints as Algorithms
\end{itemize}
\end{itemize}
\end{frame}

\begin{frame}
\frametitle{Modelling Frameworks}
\begin{itemize}
\item \href{https://www.minizinc.org/}{MiniZinc} (NICTA, Monash University, Australia)
\item \href{https://github.com/eomahony/Numberjack}{NumberJack} (Insight, Ireland) 
\item \href{https://www-users.york.ac.uk/peter.nightingale/savilerow//}{EssencePrime/SavilleRow} (UK) 
\item \href{https://cpmpy.readthedocs.io/en/latest/}{CPMpy} (KU Leuven)
\item Allow use of multiple back-end solvers
\item Compile model into variants for each solver
\item A priori solver independent model(CP, MIP, SAT)
\end{itemize}
\end{frame} 

\begin{frame}
\frametitle{Framework Process}
\begin{center}
\begin{tikzpicture}[xscale=2]
\node[shape=rectangle,fill=pantone127-4] (problem) at (2.5,5) {Problem};
\node (human) at (2.5,4) {Human};
\node[shape=rectangle,fill=pantone127-4] (model) at (2.5,3) {Model};
\node (compile) at (2.5,2) {Compile/Reformulate};
\node (solver1) at (1,1) {CP};
\node (solver2) at (2,1) {MIP};
\node (solver3) at (3,1) {SAT};
\node (solver4) at (4,1) {Other};
\node[shape=rectangle,fill=pantone157-8] (solution1) at (1,0) {Solution};
\node[shape=rectangle,fill=pantone157-8] (solution2) at (2,0) {Solution};
\node[shape=rectangle,fill=pantone157-8] (solution3) at (3,0) {Solution};
\node[shape=rectangle,fill=pantone157-8] (solution4) at (4,0) {Solution};
\draw[-] (problem) -- (human);
\draw[->] (human) -- (model);
\draw[-] (model) -- (compile);
\draw[-] (compile) -- (solver1);
\draw[-] (compile) -- (solver2);
\draw[-] (compile) -- (solver3);
\draw[-] (compile) -- (solver4);
\draw[->] (solver1) -- (solution1);
\draw[->] (solver2) -- (solution2);
\draw[->] (solver3) -- (solution3);
\draw[->] (solver4) -- (solution4);
\end{tikzpicture}
\end{center}
\end{frame}

% \begin{frame}
  % \frametitle{Do It Now!}
  % \begin{itemize}
  % \item Download and install Minizinc
  % \item \url{https://www.minizinc.org/}
  % \end{itemize}
% \end{frame}


\begin{frame}
\frametitle{Why use Puzzles as Examples?}
\begin{itemize}
\item Easy to understand the problem
\item Solvable by hand without specialized knowledge
\item Possible to compare automated to manual solving process
\end{itemize}

\begin{quote}
The puzzle, though inanimate, is presented as a solvable problem without lasting negative consequences, a very low-risk low-reward situation. By being a puzzle, the object is attempting to convince the user that it must be completed.
\end{quote}
{\tiny Source: \href{https://www.everydayrhetoric.com/2020/03/the-rhetoric-of-puzzles/}{Every Day Rhetoric}}
\end{frame}


\mode<all>{
\part{Basic Constraint Propagation}
}
\begin{frame}
\partpage
\end{frame}
\mode<all>{
\input{../sendmore/body}
}

\mode<all>{
\part{Global Constraints}
}
\begin{frame}
\partpage
\end{frame}
\mode<all>{
\input{../sudoku/body}
}

\clearpage
\mode<all>{
\part{Customizing Search}
}
\begin{frame}
\partpage
\end{frame}
\mode<all>{
\input{../nqueen/body}
}


%% \clearpage
%% \mode<all>{
%%   \part{Problem Decomposition}
%% }
%% \begin{frame}
%%   \partpage
%% \end{frame}
%% \mode<all>{
%%   \input{../party/body}
%%   }

%% \clearpage
%% \mode<all>{
%%   \part{More Global Constraints}
%% }
%% \begin{frame}
%%   \partpage
%% \end{frame}
%% \mode<all>{
%%   \input{../car/body}
%%   }

%% \clearpage
%% \mode<all>{
%%   \part{Methodology}
%% }
%% \begin{frame}
%%   \partpage
%% \end{frame}
%% \mode<all>{
%%   \input{../methodology/body}
%%   }

\clearpage
\mode<all>{
\part{What is missing?}
}
\begin{frame}
\partpage
\end{frame}

\begin{frame}
\frametitle{Many Specialized Topics}
\begin{itemize}
\item How to design efficient core engine
\item Hybrids with LP/MIP tools
\item Hybrids with SAT
\item Symmetry breaking
\item Use of MDD/BDD to encode sets of solutions
\item High level modelling tools
\item Debugging/visualization
\end{itemize}
\end{frame}

\begin{frame}
\frametitle{Reformulation}
\begin{itemize}
\item Just because the user has modelled it this way, it doesn't mean we have to solve it that way
\begin{itemize}
\item Replace some constraint(s) by other, equivalent constraints
\item Because we don't have that constraint in our system
\item For performance
\end{itemize}
\end{itemize}
\end{frame}

\begin{frame}
\frametitle{Learning}
\begin{itemize}
\item While solving the problem we can learn how to strengthen the model/search 
\begin{itemize}
\item Understand which constraints/method contribute to propagation and change schedule
\item Learn no-good constraints by explaining failure
\item Adapt search strategy based on search experience
\end{itemize}
\end{itemize}
\end{frame}

%% \begin{frame}
%% \frametitle{Refined Process}
%% \begin{center}
%% \begin{tikzpicture}
%% \node[shape=rectangle,fill=pantone127-4] (problem) at (3,7) {Problem};
%% \node (human) at (3,6) {Human};
%% \node[shape=rectangle,fill=pantone127-4] (model) at (3,5) {User Model};
%% \node (reformulation) at (3,4) {Reformulation};
%% \node[shape=rectangle,fill=pantone127-4] (implementation) at (3,3) {Implementation Model};
%% \node (solver) at (3,2) {Constraint Solver/Search};
%% \node[shape=rectangle,fill=pantone157-8] (solution) at (3,1) {Solution};
%% \draw[-] (problem) -- (human);
%% \draw[->] (human) -- (model);
%% \draw[-] (model) -- (reformulation);
%% \draw[->] (reformulation) -- (implementation);
%% \draw[-] (implementation) -- (solver);
%% \draw[->] (solver) -- (0,2) -- (0,3) -- (implementation);
%% \draw[->] (solver) -- (solution);
%% \end{tikzpicture}
%% \end{center}
%% \end{frame}

%\begin{frame}
%\frametitle{What is CP actually used for?}
%\url{http://hsimonis.wordpress.com}
%\includegraphics[width=10cm]{../introduction/images/blog}
%\end{frame}

%% \begin{frame}
%% \frametitle{Exercises}
%% \begin{itemize}
%% \item Install Constraint Solver of your Choice
%% \item MiniZinc (MiningZinc) will be used in later sessions
%% \item Run Sendmoremoney, Sudoku, Queens examples
%% \item Can you choose which propagator to use for alldifferent?
%% \item Can you use different search strategies?
%% \end{itemize}
%% \end{frame}

\begin{frame}
\frametitle{More Learning Resources}
\begin{itemize}
\item Survey of Methods, Resources, and Formats for Teaching Constraint Programming
\begin{itemize}
\item by Tejas Santanam, Helmut Simonis
\item \url{https://doi.org/10.48550/arXiv.2403.12717}
\item Based on survey of community for WTCP 2023
\item \url{https://hsimonis.github.io/WTCP2023/}
\end{itemize}
\end{itemize}
\end{frame}


%%   }

\clearpage
\mode<all>{
\part{What is missing?}
}
\begin{frame}
\partpage
\end{frame}

\begin{frame}
\frametitle{Many Specialized Topics}
\begin{itemize}
\item How to design efficient core engine
\item Hybrids with LP/MIP tools
\item Hybrids with SAT
\item Symmetry breaking
\item Use of MDD/BDD to encode sets of solutions
\item High level modelling tools
\item Debugging/visualization
\end{itemize}
\end{frame}

\begin{frame}
\frametitle{Reformulation}
\begin{itemize}
\item Just because the user has modelled it this way, it doesn't mean we have to solve it that way
\begin{itemize}
\item Replace some constraint(s) by other, equivalent constraints
\item Because we don't have that constraint in our system
\item For performance
\end{itemize}
\end{itemize}
\end{frame}

\begin{frame}
\frametitle{Learning}
\begin{itemize}
\item While solving the problem we can learn how to strengthen the model/search 
\begin{itemize}
\item Understand which constraints/method contribute to propagation and change schedule
\item Learn no-good constraints by explaining failure
\item Adapt search strategy based on search experience
\end{itemize}
\end{itemize}
\end{frame}

%% \begin{frame}
%% \frametitle{Refined Process}
%% \begin{center}
%% \begin{tikzpicture}
%% \node[shape=rectangle,fill=pantone127-4] (problem) at (3,7) {Problem};
%% \node (human) at (3,6) {Human};
%% \node[shape=rectangle,fill=pantone127-4] (model) at (3,5) {User Model};
%% \node (reformulation) at (3,4) {Reformulation};
%% \node[shape=rectangle,fill=pantone127-4] (implementation) at (3,3) {Implementation Model};
%% \node (solver) at (3,2) {Constraint Solver/Search};
%% \node[shape=rectangle,fill=pantone157-8] (solution) at (3,1) {Solution};
%% \draw[-] (problem) -- (human);
%% \draw[->] (human) -- (model);
%% \draw[-] (model) -- (reformulation);
%% \draw[->] (reformulation) -- (implementation);
%% \draw[-] (implementation) -- (solver);
%% \draw[->] (solver) -- (0,2) -- (0,3) -- (implementation);
%% \draw[->] (solver) -- (solution);
%% \end{tikzpicture}
%% \end{center}
%% \end{frame}

%\begin{frame}
%\frametitle{What is CP actually used for?}
%\url{http://hsimonis.wordpress.com}
%\includegraphics[width=10cm]{../introduction/images/blog}
%\end{frame}

%% \begin{frame}
%% \frametitle{Exercises}
%% \begin{itemize}
%% \item Install Constraint Solver of your Choice
%% \item MiniZinc (MiningZinc) will be used in later sessions
%% \item Run Sendmoremoney, Sudoku, Queens examples
%% \item Can you choose which propagator to use for alldifferent?
%% \item Can you use different search strategies?
%% \end{itemize}
%% \end{frame}

\begin{frame}
\frametitle{More Learning Resources}
\begin{itemize}
\item Survey of Methods, Resources, and Formats for Teaching Constraint Programming
\begin{itemize}
\item by Tejas Santanam, Helmut Simonis
\item \url{https://doi.org/10.48550/arXiv.2403.12717}
\item Based on survey of community for WTCP 2023
\item \url{https://hsimonis.github.io/WTCP2023/}
\end{itemize}
\end{itemize}
\end{frame}


}

\clearpage
\mode<all>{
\part{Customizing Search}
}
\begin{frame}
\partpage
\end{frame}
\mode<all>{
\begin{frame}
\frametitle{Objectives}
\begin{itemize}
\item Overview of Core Constraint Programming
\item Three Main Concepts
\begin{itemize}
\item Constraint Propagation
\item Global Constraints
\item Customizing Search
\end{itemize}
%\item Get Some Experience with MiniZinc
\item Topics will be treated in more detail in later parts of the school
\item Based on Examples, not Formal Description
\end{itemize}
\end{frame}

\begin{frame}
\frametitle{Outline}
\begin{itemize}
\item Why Constraint Programming?
\item Constraint Propagation
\item Global Constraints
\item Customizing Search
%\item Choices in Modelling
%\item Decomposition
%\item More Global Constraints
%\item Methodology
%\item What is missing?
\end{itemize}
\end{frame}

% \begin{frame}
  % \frametitle{Using MiniZinc IDE}
  % \begin{itemize}
  % \item Developed in the Australian NICTA project
  % \item Maintained by Monash University
  % \item Modelling tool with multiple back-end solvers
  % \item Available from \url{https://www.minizinc.org/}
  % \end{itemize}
% \end{frame}


% \begin{frame}
% \frametitle{Examples in ECLiPSe}
% \begin{itemize}
% \item Open sourced constraint programming language
% \item Development goes back to 1985
% \item ECRC, ICL, IC-Parc, PTL, Cisco
% \item \url{https://eclipseclp.org/}
% \item Specialities
% \begin{itemize}
% \item Develop new solvers for specific domains
% \item Integration with MIP
% \end{itemize}
% \item Not included in bundled MiniZinc IDE
% \item Specialized visualization tools used here
  % \begin{itemize}
  % \item CP-Viz, Simonis et al. 2010
    % \end{itemize}
% \end{itemize}
% \end{frame}

\begin{frame}
\frametitle{Tutorial Based on ECLiPSe ELearning Course}
\begin{itemize}
\item Self-study course in constraint programming
\item Supported by Cisco Systems and Silicon Valley Community Foundation
\item Multi-media format, video lectures, slides, handout etc
\item {\small \url{https://eclipseclp.org/ELearning/index.html}}
\end{itemize}
\end{frame}

\begin{frame}
\frametitle{Also Part of CRT-AI Constraint Week}
\begin{itemize}
\item Annual one week course on CP and Optimization in Ireland
\item Part of national training program for PhD students in AI
\item \url{https://www.crt-ai.ie/}
\end{itemize}
\end{frame}



\begin{frame}
\frametitle{Constraint Programming - in a nutshell}
\begin{itemize}
\item Declarative description of problems with
\begin{itemize}
\item {\em Variables} which range over (finite) sets of values
\item {\em Constraints} over subsets of variables which restrict possible value combinations
\item A {\em solution} is a value assignment which satisfies all constraints
\end{itemize}

\item Constraint propagation/reasoning
\begin{itemize}
\item Removing inconsistent values for variables
\item Detect failure if constraint can not be satisfied
\item Interaction of constraints via shared variables
\item Incomplete
\end{itemize}

\item Search
\begin{itemize}
\item User controlled assignment of values to variables
\item Each step triggers constraint propagation 
\end{itemize}
\item Different domains require/allow different methods
\end{itemize}
\end{frame}

\begin{frame}
  \frametitle{Constraint Programming is Different}
  \begin{itemize}
  \item Declarative Programming
    \begin{itemize}
    \item Concentrate on what you want
      \item Not how to get there
      \item Program != Algorithm
      \item Program = Model
    \end{itemize}
    \item Applied to Combinatorial Problems
      \begin{itemize}
        \item No complete polynomial algorithms known (exist?)
        \item CP less ad-hoc than heuristics
        \item Models can evolve
  \end{itemize}
  \end{itemize}
  \end{frame}
    
\begin{frame}
  \frametitle{A Subtractive Process}
  \begin{textblock}{4}(8,-3)
    \includegraphics[width=4cm]{../introduction/images/stages}
  \end{textblock}
  \vfill
  \begin{quote}
    ``Oh, bosh, as Mr. Ruskin says. Sculpture, per se, is the simplest thing in the world. All you have to do is to take a big chunk of marble and a hammer and chisel, make up your mind what you are about to create and chip off all the marble you don't want.''-Paris Gaulois.
  \end{quote}
  
  {\tiny Source: \url{https://quoteinvestigator.com/2014/06/22/chip-away/}}
\end{frame}

\begin{frame}
\frametitle{Basic Process}
\begin{center}
\begin{tikzpicture}
\node[shape=rectangle,fill=pantone127-4] (problem) at (2,5) {Problem};
\node (human) at (2,4) {Human};
\node[shape=rectangle,fill=pantone127-4] (model) at (2,3) {Model};
\node (solver) at (2,2) {Constraint Solver/Search};
\node[shape=rectangle,fill=pantone157-8] (solution) at (2,1) {Solution};
\draw[-] (problem) -- (human);
\draw[->] (human) -- (model);
\draw[-] (model) -- (solver);
\draw[->] (solver) -- (solution);
\end{tikzpicture}
\end{center}
\end{frame}

\begin{frame}
\frametitle{More Realistic}
\begin{center}
\begin{tikzpicture}[xscale=4.0]
\node[shape=rectangle,fill=pantone127-4] (problem) at (2,5) {Problem};
\node (human) at (2,4) {Human};
\node[shape=rectangle,fill=pantone127-4] (model) at (2,3) {Model};
\node (solver) at (2,2) {Constraint Solver/Search};
\node[shape=rectangle,fill=pantone24-8] (hangs) at (1,1) {Hangs};
\node[shape=rectangle,fill=pantone157-8] (solution) at (2,1) {Solution};
\node[shape=rectangle,fill=pantone24-8] (wrong) at (3,1) {Wrong Solution};
\draw[-] (problem) -- (human);
\draw[->] (human) -- (model);
\draw[-] (model) -- (solver);
\draw[->] (solver) -- (hangs);
\draw[->] (solver) -- (solution);
\draw[->] (solver) -- (wrong);
\draw[->] (hangs) -- (1,4) -- (human);
\draw[->] (wrong) -- (3,4) -- (human);
\end{tikzpicture}
\end{center}
\end{frame}

\begin{frame}
\frametitle{Dual Role of Model}
\begin{itemize}
\item Allows Human to Express Problem
\begin{itemize}
\item Close to Problem Domain
\item Constraints as Abstractions
\end{itemize}
\item Allows Solver to Execute
\begin{itemize}
\item Variables as Communication Mechanism
\item Constraints as Algorithms
\end{itemize}
\end{itemize}
\end{frame}

\begin{frame}
\frametitle{Modelling Frameworks}
\begin{itemize}
\item \href{https://www.minizinc.org/}{MiniZinc} (NICTA, Monash University, Australia)
\item \href{https://github.com/eomahony/Numberjack}{NumberJack} (Insight, Ireland) 
\item \href{https://www-users.york.ac.uk/peter.nightingale/savilerow//}{EssencePrime/SavilleRow} (UK) 
\item \href{https://cpmpy.readthedocs.io/en/latest/}{CPMpy} (KU Leuven)
\item Allow use of multiple back-end solvers
\item Compile model into variants for each solver
\item A priori solver independent model(CP, MIP, SAT)
\end{itemize}
\end{frame} 

\begin{frame}
\frametitle{Framework Process}
\begin{center}
\begin{tikzpicture}[xscale=2]
\node[shape=rectangle,fill=pantone127-4] (problem) at (2.5,5) {Problem};
\node (human) at (2.5,4) {Human};
\node[shape=rectangle,fill=pantone127-4] (model) at (2.5,3) {Model};
\node (compile) at (2.5,2) {Compile/Reformulate};
\node (solver1) at (1,1) {CP};
\node (solver2) at (2,1) {MIP};
\node (solver3) at (3,1) {SAT};
\node (solver4) at (4,1) {Other};
\node[shape=rectangle,fill=pantone157-8] (solution1) at (1,0) {Solution};
\node[shape=rectangle,fill=pantone157-8] (solution2) at (2,0) {Solution};
\node[shape=rectangle,fill=pantone157-8] (solution3) at (3,0) {Solution};
\node[shape=rectangle,fill=pantone157-8] (solution4) at (4,0) {Solution};
\draw[-] (problem) -- (human);
\draw[->] (human) -- (model);
\draw[-] (model) -- (compile);
\draw[-] (compile) -- (solver1);
\draw[-] (compile) -- (solver2);
\draw[-] (compile) -- (solver3);
\draw[-] (compile) -- (solver4);
\draw[->] (solver1) -- (solution1);
\draw[->] (solver2) -- (solution2);
\draw[->] (solver3) -- (solution3);
\draw[->] (solver4) -- (solution4);
\end{tikzpicture}
\end{center}
\end{frame}

% \begin{frame}
  % \frametitle{Do It Now!}
  % \begin{itemize}
  % \item Download and install Minizinc
  % \item \url{https://www.minizinc.org/}
  % \end{itemize}
% \end{frame}


\begin{frame}
\frametitle{Why use Puzzles as Examples?}
\begin{itemize}
\item Easy to understand the problem
\item Solvable by hand without specialized knowledge
\item Possible to compare automated to manual solving process
\end{itemize}

\begin{quote}
The puzzle, though inanimate, is presented as a solvable problem without lasting negative consequences, a very low-risk low-reward situation. By being a puzzle, the object is attempting to convince the user that it must be completed.
\end{quote}
{\tiny Source: \href{https://www.everydayrhetoric.com/2020/03/the-rhetoric-of-puzzles/}{Every Day Rhetoric}}
\end{frame}


\mode<all>{
\part{Basic Constraint Propagation}
}
\begin{frame}
\partpage
\end{frame}
\mode<all>{
\begin{frame}
\frametitle{Objectives}
\begin{itemize}
\item Overview of Core Constraint Programming
\item Three Main Concepts
\begin{itemize}
\item Constraint Propagation
\item Global Constraints
\item Customizing Search
\end{itemize}
%\item Get Some Experience with MiniZinc
\item Topics will be treated in more detail in later parts of the school
\item Based on Examples, not Formal Description
\end{itemize}
\end{frame}

\begin{frame}
\frametitle{Outline}
\begin{itemize}
\item Why Constraint Programming?
\item Constraint Propagation
\item Global Constraints
\item Customizing Search
%\item Choices in Modelling
%\item Decomposition
%\item More Global Constraints
%\item Methodology
%\item What is missing?
\end{itemize}
\end{frame}

% \begin{frame}
  % \frametitle{Using MiniZinc IDE}
  % \begin{itemize}
  % \item Developed in the Australian NICTA project
  % \item Maintained by Monash University
  % \item Modelling tool with multiple back-end solvers
  % \item Available from \url{https://www.minizinc.org/}
  % \end{itemize}
% \end{frame}


% \begin{frame}
% \frametitle{Examples in ECLiPSe}
% \begin{itemize}
% \item Open sourced constraint programming language
% \item Development goes back to 1985
% \item ECRC, ICL, IC-Parc, PTL, Cisco
% \item \url{https://eclipseclp.org/}
% \item Specialities
% \begin{itemize}
% \item Develop new solvers for specific domains
% \item Integration with MIP
% \end{itemize}
% \item Not included in bundled MiniZinc IDE
% \item Specialized visualization tools used here
  % \begin{itemize}
  % \item CP-Viz, Simonis et al. 2010
    % \end{itemize}
% \end{itemize}
% \end{frame}

\begin{frame}
\frametitle{Tutorial Based on ECLiPSe ELearning Course}
\begin{itemize}
\item Self-study course in constraint programming
\item Supported by Cisco Systems and Silicon Valley Community Foundation
\item Multi-media format, video lectures, slides, handout etc
\item {\small \url{https://eclipseclp.org/ELearning/index.html}}
\end{itemize}
\end{frame}

\begin{frame}
\frametitle{Also Part of CRT-AI Constraint Week}
\begin{itemize}
\item Annual one week course on CP and Optimization in Ireland
\item Part of national training program for PhD students in AI
\item \url{https://www.crt-ai.ie/}
\end{itemize}
\end{frame}



\begin{frame}
\frametitle{Constraint Programming - in a nutshell}
\begin{itemize}
\item Declarative description of problems with
\begin{itemize}
\item {\em Variables} which range over (finite) sets of values
\item {\em Constraints} over subsets of variables which restrict possible value combinations
\item A {\em solution} is a value assignment which satisfies all constraints
\end{itemize}

\item Constraint propagation/reasoning
\begin{itemize}
\item Removing inconsistent values for variables
\item Detect failure if constraint can not be satisfied
\item Interaction of constraints via shared variables
\item Incomplete
\end{itemize}

\item Search
\begin{itemize}
\item User controlled assignment of values to variables
\item Each step triggers constraint propagation 
\end{itemize}
\item Different domains require/allow different methods
\end{itemize}
\end{frame}

\begin{frame}
  \frametitle{Constraint Programming is Different}
  \begin{itemize}
  \item Declarative Programming
    \begin{itemize}
    \item Concentrate on what you want
      \item Not how to get there
      \item Program != Algorithm
      \item Program = Model
    \end{itemize}
    \item Applied to Combinatorial Problems
      \begin{itemize}
        \item No complete polynomial algorithms known (exist?)
        \item CP less ad-hoc than heuristics
        \item Models can evolve
  \end{itemize}
  \end{itemize}
  \end{frame}
    
\begin{frame}
  \frametitle{A Subtractive Process}
  \begin{textblock}{4}(8,-3)
    \includegraphics[width=4cm]{../introduction/images/stages}
  \end{textblock}
  \vfill
  \begin{quote}
    ``Oh, bosh, as Mr. Ruskin says. Sculpture, per se, is the simplest thing in the world. All you have to do is to take a big chunk of marble and a hammer and chisel, make up your mind what you are about to create and chip off all the marble you don't want.''-Paris Gaulois.
  \end{quote}
  
  {\tiny Source: \url{https://quoteinvestigator.com/2014/06/22/chip-away/}}
\end{frame}

\begin{frame}
\frametitle{Basic Process}
\begin{center}
\begin{tikzpicture}
\node[shape=rectangle,fill=pantone127-4] (problem) at (2,5) {Problem};
\node (human) at (2,4) {Human};
\node[shape=rectangle,fill=pantone127-4] (model) at (2,3) {Model};
\node (solver) at (2,2) {Constraint Solver/Search};
\node[shape=rectangle,fill=pantone157-8] (solution) at (2,1) {Solution};
\draw[-] (problem) -- (human);
\draw[->] (human) -- (model);
\draw[-] (model) -- (solver);
\draw[->] (solver) -- (solution);
\end{tikzpicture}
\end{center}
\end{frame}

\begin{frame}
\frametitle{More Realistic}
\begin{center}
\begin{tikzpicture}[xscale=4.0]
\node[shape=rectangle,fill=pantone127-4] (problem) at (2,5) {Problem};
\node (human) at (2,4) {Human};
\node[shape=rectangle,fill=pantone127-4] (model) at (2,3) {Model};
\node (solver) at (2,2) {Constraint Solver/Search};
\node[shape=rectangle,fill=pantone24-8] (hangs) at (1,1) {Hangs};
\node[shape=rectangle,fill=pantone157-8] (solution) at (2,1) {Solution};
\node[shape=rectangle,fill=pantone24-8] (wrong) at (3,1) {Wrong Solution};
\draw[-] (problem) -- (human);
\draw[->] (human) -- (model);
\draw[-] (model) -- (solver);
\draw[->] (solver) -- (hangs);
\draw[->] (solver) -- (solution);
\draw[->] (solver) -- (wrong);
\draw[->] (hangs) -- (1,4) -- (human);
\draw[->] (wrong) -- (3,4) -- (human);
\end{tikzpicture}
\end{center}
\end{frame}

\begin{frame}
\frametitle{Dual Role of Model}
\begin{itemize}
\item Allows Human to Express Problem
\begin{itemize}
\item Close to Problem Domain
\item Constraints as Abstractions
\end{itemize}
\item Allows Solver to Execute
\begin{itemize}
\item Variables as Communication Mechanism
\item Constraints as Algorithms
\end{itemize}
\end{itemize}
\end{frame}

\begin{frame}
\frametitle{Modelling Frameworks}
\begin{itemize}
\item \href{https://www.minizinc.org/}{MiniZinc} (NICTA, Monash University, Australia)
\item \href{https://github.com/eomahony/Numberjack}{NumberJack} (Insight, Ireland) 
\item \href{https://www-users.york.ac.uk/peter.nightingale/savilerow//}{EssencePrime/SavilleRow} (UK) 
\item \href{https://cpmpy.readthedocs.io/en/latest/}{CPMpy} (KU Leuven)
\item Allow use of multiple back-end solvers
\item Compile model into variants for each solver
\item A priori solver independent model(CP, MIP, SAT)
\end{itemize}
\end{frame} 

\begin{frame}
\frametitle{Framework Process}
\begin{center}
\begin{tikzpicture}[xscale=2]
\node[shape=rectangle,fill=pantone127-4] (problem) at (2.5,5) {Problem};
\node (human) at (2.5,4) {Human};
\node[shape=rectangle,fill=pantone127-4] (model) at (2.5,3) {Model};
\node (compile) at (2.5,2) {Compile/Reformulate};
\node (solver1) at (1,1) {CP};
\node (solver2) at (2,1) {MIP};
\node (solver3) at (3,1) {SAT};
\node (solver4) at (4,1) {Other};
\node[shape=rectangle,fill=pantone157-8] (solution1) at (1,0) {Solution};
\node[shape=rectangle,fill=pantone157-8] (solution2) at (2,0) {Solution};
\node[shape=rectangle,fill=pantone157-8] (solution3) at (3,0) {Solution};
\node[shape=rectangle,fill=pantone157-8] (solution4) at (4,0) {Solution};
\draw[-] (problem) -- (human);
\draw[->] (human) -- (model);
\draw[-] (model) -- (compile);
\draw[-] (compile) -- (solver1);
\draw[-] (compile) -- (solver2);
\draw[-] (compile) -- (solver3);
\draw[-] (compile) -- (solver4);
\draw[->] (solver1) -- (solution1);
\draw[->] (solver2) -- (solution2);
\draw[->] (solver3) -- (solution3);
\draw[->] (solver4) -- (solution4);
\end{tikzpicture}
\end{center}
\end{frame}

% \begin{frame}
  % \frametitle{Do It Now!}
  % \begin{itemize}
  % \item Download and install Minizinc
  % \item \url{https://www.minizinc.org/}
  % \end{itemize}
% \end{frame}


\begin{frame}
\frametitle{Why use Puzzles as Examples?}
\begin{itemize}
\item Easy to understand the problem
\item Solvable by hand without specialized knowledge
\item Possible to compare automated to manual solving process
\end{itemize}

\begin{quote}
The puzzle, though inanimate, is presented as a solvable problem without lasting negative consequences, a very low-risk low-reward situation. By being a puzzle, the object is attempting to convince the user that it must be completed.
\end{quote}
{\tiny Source: \href{https://www.everydayrhetoric.com/2020/03/the-rhetoric-of-puzzles/}{Every Day Rhetoric}}
\end{frame}


\mode<all>{
\part{Basic Constraint Propagation}
}
\begin{frame}
\partpage
\end{frame}
\mode<all>{
\input{../sendmore/body}
}

\mode<all>{
\part{Global Constraints}
}
\begin{frame}
\partpage
\end{frame}
\mode<all>{
\input{../sudoku/body}
}

\clearpage
\mode<all>{
\part{Customizing Search}
}
\begin{frame}
\partpage
\end{frame}
\mode<all>{
\input{../nqueen/body}
}


%% \clearpage
%% \mode<all>{
%%   \part{Problem Decomposition}
%% }
%% \begin{frame}
%%   \partpage
%% \end{frame}
%% \mode<all>{
%%   \input{../party/body}
%%   }

%% \clearpage
%% \mode<all>{
%%   \part{More Global Constraints}
%% }
%% \begin{frame}
%%   \partpage
%% \end{frame}
%% \mode<all>{
%%   \input{../car/body}
%%   }

%% \clearpage
%% \mode<all>{
%%   \part{Methodology}
%% }
%% \begin{frame}
%%   \partpage
%% \end{frame}
%% \mode<all>{
%%   \input{../methodology/body}
%%   }

\clearpage
\mode<all>{
\part{What is missing?}
}
\begin{frame}
\partpage
\end{frame}

\begin{frame}
\frametitle{Many Specialized Topics}
\begin{itemize}
\item How to design efficient core engine
\item Hybrids with LP/MIP tools
\item Hybrids with SAT
\item Symmetry breaking
\item Use of MDD/BDD to encode sets of solutions
\item High level modelling tools
\item Debugging/visualization
\end{itemize}
\end{frame}

\begin{frame}
\frametitle{Reformulation}
\begin{itemize}
\item Just because the user has modelled it this way, it doesn't mean we have to solve it that way
\begin{itemize}
\item Replace some constraint(s) by other, equivalent constraints
\item Because we don't have that constraint in our system
\item For performance
\end{itemize}
\end{itemize}
\end{frame}

\begin{frame}
\frametitle{Learning}
\begin{itemize}
\item While solving the problem we can learn how to strengthen the model/search 
\begin{itemize}
\item Understand which constraints/method contribute to propagation and change schedule
\item Learn no-good constraints by explaining failure
\item Adapt search strategy based on search experience
\end{itemize}
\end{itemize}
\end{frame}

%% \begin{frame}
%% \frametitle{Refined Process}
%% \begin{center}
%% \begin{tikzpicture}
%% \node[shape=rectangle,fill=pantone127-4] (problem) at (3,7) {Problem};
%% \node (human) at (3,6) {Human};
%% \node[shape=rectangle,fill=pantone127-4] (model) at (3,5) {User Model};
%% \node (reformulation) at (3,4) {Reformulation};
%% \node[shape=rectangle,fill=pantone127-4] (implementation) at (3,3) {Implementation Model};
%% \node (solver) at (3,2) {Constraint Solver/Search};
%% \node[shape=rectangle,fill=pantone157-8] (solution) at (3,1) {Solution};
%% \draw[-] (problem) -- (human);
%% \draw[->] (human) -- (model);
%% \draw[-] (model) -- (reformulation);
%% \draw[->] (reformulation) -- (implementation);
%% \draw[-] (implementation) -- (solver);
%% \draw[->] (solver) -- (0,2) -- (0,3) -- (implementation);
%% \draw[->] (solver) -- (solution);
%% \end{tikzpicture}
%% \end{center}
%% \end{frame}

%\begin{frame}
%\frametitle{What is CP actually used for?}
%\url{http://hsimonis.wordpress.com}
%\includegraphics[width=10cm]{../introduction/images/blog}
%\end{frame}

%% \begin{frame}
%% \frametitle{Exercises}
%% \begin{itemize}
%% \item Install Constraint Solver of your Choice
%% \item MiniZinc (MiningZinc) will be used in later sessions
%% \item Run Sendmoremoney, Sudoku, Queens examples
%% \item Can you choose which propagator to use for alldifferent?
%% \item Can you use different search strategies?
%% \end{itemize}
%% \end{frame}

\begin{frame}
\frametitle{More Learning Resources}
\begin{itemize}
\item Survey of Methods, Resources, and Formats for Teaching Constraint Programming
\begin{itemize}
\item by Tejas Santanam, Helmut Simonis
\item \url{https://doi.org/10.48550/arXiv.2403.12717}
\item Based on survey of community for WTCP 2023
\item \url{https://hsimonis.github.io/WTCP2023/}
\end{itemize}
\end{itemize}
\end{frame}


}

\mode<all>{
\part{Global Constraints}
}
\begin{frame}
\partpage
\end{frame}
\mode<all>{
\begin{frame}
\frametitle{Objectives}
\begin{itemize}
\item Overview of Core Constraint Programming
\item Three Main Concepts
\begin{itemize}
\item Constraint Propagation
\item Global Constraints
\item Customizing Search
\end{itemize}
%\item Get Some Experience with MiniZinc
\item Topics will be treated in more detail in later parts of the school
\item Based on Examples, not Formal Description
\end{itemize}
\end{frame}

\begin{frame}
\frametitle{Outline}
\begin{itemize}
\item Why Constraint Programming?
\item Constraint Propagation
\item Global Constraints
\item Customizing Search
%\item Choices in Modelling
%\item Decomposition
%\item More Global Constraints
%\item Methodology
%\item What is missing?
\end{itemize}
\end{frame}

% \begin{frame}
  % \frametitle{Using MiniZinc IDE}
  % \begin{itemize}
  % \item Developed in the Australian NICTA project
  % \item Maintained by Monash University
  % \item Modelling tool with multiple back-end solvers
  % \item Available from \url{https://www.minizinc.org/}
  % \end{itemize}
% \end{frame}


% \begin{frame}
% \frametitle{Examples in ECLiPSe}
% \begin{itemize}
% \item Open sourced constraint programming language
% \item Development goes back to 1985
% \item ECRC, ICL, IC-Parc, PTL, Cisco
% \item \url{https://eclipseclp.org/}
% \item Specialities
% \begin{itemize}
% \item Develop new solvers for specific domains
% \item Integration with MIP
% \end{itemize}
% \item Not included in bundled MiniZinc IDE
% \item Specialized visualization tools used here
  % \begin{itemize}
  % \item CP-Viz, Simonis et al. 2010
    % \end{itemize}
% \end{itemize}
% \end{frame}

\begin{frame}
\frametitle{Tutorial Based on ECLiPSe ELearning Course}
\begin{itemize}
\item Self-study course in constraint programming
\item Supported by Cisco Systems and Silicon Valley Community Foundation
\item Multi-media format, video lectures, slides, handout etc
\item {\small \url{https://eclipseclp.org/ELearning/index.html}}
\end{itemize}
\end{frame}

\begin{frame}
\frametitle{Also Part of CRT-AI Constraint Week}
\begin{itemize}
\item Annual one week course on CP and Optimization in Ireland
\item Part of national training program for PhD students in AI
\item \url{https://www.crt-ai.ie/}
\end{itemize}
\end{frame}



\begin{frame}
\frametitle{Constraint Programming - in a nutshell}
\begin{itemize}
\item Declarative description of problems with
\begin{itemize}
\item {\em Variables} which range over (finite) sets of values
\item {\em Constraints} over subsets of variables which restrict possible value combinations
\item A {\em solution} is a value assignment which satisfies all constraints
\end{itemize}

\item Constraint propagation/reasoning
\begin{itemize}
\item Removing inconsistent values for variables
\item Detect failure if constraint can not be satisfied
\item Interaction of constraints via shared variables
\item Incomplete
\end{itemize}

\item Search
\begin{itemize}
\item User controlled assignment of values to variables
\item Each step triggers constraint propagation 
\end{itemize}
\item Different domains require/allow different methods
\end{itemize}
\end{frame}

\begin{frame}
  \frametitle{Constraint Programming is Different}
  \begin{itemize}
  \item Declarative Programming
    \begin{itemize}
    \item Concentrate on what you want
      \item Not how to get there
      \item Program != Algorithm
      \item Program = Model
    \end{itemize}
    \item Applied to Combinatorial Problems
      \begin{itemize}
        \item No complete polynomial algorithms known (exist?)
        \item CP less ad-hoc than heuristics
        \item Models can evolve
  \end{itemize}
  \end{itemize}
  \end{frame}
    
\begin{frame}
  \frametitle{A Subtractive Process}
  \begin{textblock}{4}(8,-3)
    \includegraphics[width=4cm]{../introduction/images/stages}
  \end{textblock}
  \vfill
  \begin{quote}
    ``Oh, bosh, as Mr. Ruskin says. Sculpture, per se, is the simplest thing in the world. All you have to do is to take a big chunk of marble and a hammer and chisel, make up your mind what you are about to create and chip off all the marble you don't want.''-Paris Gaulois.
  \end{quote}
  
  {\tiny Source: \url{https://quoteinvestigator.com/2014/06/22/chip-away/}}
\end{frame}

\begin{frame}
\frametitle{Basic Process}
\begin{center}
\begin{tikzpicture}
\node[shape=rectangle,fill=pantone127-4] (problem) at (2,5) {Problem};
\node (human) at (2,4) {Human};
\node[shape=rectangle,fill=pantone127-4] (model) at (2,3) {Model};
\node (solver) at (2,2) {Constraint Solver/Search};
\node[shape=rectangle,fill=pantone157-8] (solution) at (2,1) {Solution};
\draw[-] (problem) -- (human);
\draw[->] (human) -- (model);
\draw[-] (model) -- (solver);
\draw[->] (solver) -- (solution);
\end{tikzpicture}
\end{center}
\end{frame}

\begin{frame}
\frametitle{More Realistic}
\begin{center}
\begin{tikzpicture}[xscale=4.0]
\node[shape=rectangle,fill=pantone127-4] (problem) at (2,5) {Problem};
\node (human) at (2,4) {Human};
\node[shape=rectangle,fill=pantone127-4] (model) at (2,3) {Model};
\node (solver) at (2,2) {Constraint Solver/Search};
\node[shape=rectangle,fill=pantone24-8] (hangs) at (1,1) {Hangs};
\node[shape=rectangle,fill=pantone157-8] (solution) at (2,1) {Solution};
\node[shape=rectangle,fill=pantone24-8] (wrong) at (3,1) {Wrong Solution};
\draw[-] (problem) -- (human);
\draw[->] (human) -- (model);
\draw[-] (model) -- (solver);
\draw[->] (solver) -- (hangs);
\draw[->] (solver) -- (solution);
\draw[->] (solver) -- (wrong);
\draw[->] (hangs) -- (1,4) -- (human);
\draw[->] (wrong) -- (3,4) -- (human);
\end{tikzpicture}
\end{center}
\end{frame}

\begin{frame}
\frametitle{Dual Role of Model}
\begin{itemize}
\item Allows Human to Express Problem
\begin{itemize}
\item Close to Problem Domain
\item Constraints as Abstractions
\end{itemize}
\item Allows Solver to Execute
\begin{itemize}
\item Variables as Communication Mechanism
\item Constraints as Algorithms
\end{itemize}
\end{itemize}
\end{frame}

\begin{frame}
\frametitle{Modelling Frameworks}
\begin{itemize}
\item \href{https://www.minizinc.org/}{MiniZinc} (NICTA, Monash University, Australia)
\item \href{https://github.com/eomahony/Numberjack}{NumberJack} (Insight, Ireland) 
\item \href{https://www-users.york.ac.uk/peter.nightingale/savilerow//}{EssencePrime/SavilleRow} (UK) 
\item \href{https://cpmpy.readthedocs.io/en/latest/}{CPMpy} (KU Leuven)
\item Allow use of multiple back-end solvers
\item Compile model into variants for each solver
\item A priori solver independent model(CP, MIP, SAT)
\end{itemize}
\end{frame} 

\begin{frame}
\frametitle{Framework Process}
\begin{center}
\begin{tikzpicture}[xscale=2]
\node[shape=rectangle,fill=pantone127-4] (problem) at (2.5,5) {Problem};
\node (human) at (2.5,4) {Human};
\node[shape=rectangle,fill=pantone127-4] (model) at (2.5,3) {Model};
\node (compile) at (2.5,2) {Compile/Reformulate};
\node (solver1) at (1,1) {CP};
\node (solver2) at (2,1) {MIP};
\node (solver3) at (3,1) {SAT};
\node (solver4) at (4,1) {Other};
\node[shape=rectangle,fill=pantone157-8] (solution1) at (1,0) {Solution};
\node[shape=rectangle,fill=pantone157-8] (solution2) at (2,0) {Solution};
\node[shape=rectangle,fill=pantone157-8] (solution3) at (3,0) {Solution};
\node[shape=rectangle,fill=pantone157-8] (solution4) at (4,0) {Solution};
\draw[-] (problem) -- (human);
\draw[->] (human) -- (model);
\draw[-] (model) -- (compile);
\draw[-] (compile) -- (solver1);
\draw[-] (compile) -- (solver2);
\draw[-] (compile) -- (solver3);
\draw[-] (compile) -- (solver4);
\draw[->] (solver1) -- (solution1);
\draw[->] (solver2) -- (solution2);
\draw[->] (solver3) -- (solution3);
\draw[->] (solver4) -- (solution4);
\end{tikzpicture}
\end{center}
\end{frame}

% \begin{frame}
  % \frametitle{Do It Now!}
  % \begin{itemize}
  % \item Download and install Minizinc
  % \item \url{https://www.minizinc.org/}
  % \end{itemize}
% \end{frame}


\begin{frame}
\frametitle{Why use Puzzles as Examples?}
\begin{itemize}
\item Easy to understand the problem
\item Solvable by hand without specialized knowledge
\item Possible to compare automated to manual solving process
\end{itemize}

\begin{quote}
The puzzle, though inanimate, is presented as a solvable problem without lasting negative consequences, a very low-risk low-reward situation. By being a puzzle, the object is attempting to convince the user that it must be completed.
\end{quote}
{\tiny Source: \href{https://www.everydayrhetoric.com/2020/03/the-rhetoric-of-puzzles/}{Every Day Rhetoric}}
\end{frame}


\mode<all>{
\part{Basic Constraint Propagation}
}
\begin{frame}
\partpage
\end{frame}
\mode<all>{
\input{../sendmore/body}
}

\mode<all>{
\part{Global Constraints}
}
\begin{frame}
\partpage
\end{frame}
\mode<all>{
\input{../sudoku/body}
}

\clearpage
\mode<all>{
\part{Customizing Search}
}
\begin{frame}
\partpage
\end{frame}
\mode<all>{
\input{../nqueen/body}
}


%% \clearpage
%% \mode<all>{
%%   \part{Problem Decomposition}
%% }
%% \begin{frame}
%%   \partpage
%% \end{frame}
%% \mode<all>{
%%   \input{../party/body}
%%   }

%% \clearpage
%% \mode<all>{
%%   \part{More Global Constraints}
%% }
%% \begin{frame}
%%   \partpage
%% \end{frame}
%% \mode<all>{
%%   \input{../car/body}
%%   }

%% \clearpage
%% \mode<all>{
%%   \part{Methodology}
%% }
%% \begin{frame}
%%   \partpage
%% \end{frame}
%% \mode<all>{
%%   \input{../methodology/body}
%%   }

\clearpage
\mode<all>{
\part{What is missing?}
}
\begin{frame}
\partpage
\end{frame}

\begin{frame}
\frametitle{Many Specialized Topics}
\begin{itemize}
\item How to design efficient core engine
\item Hybrids with LP/MIP tools
\item Hybrids with SAT
\item Symmetry breaking
\item Use of MDD/BDD to encode sets of solutions
\item High level modelling tools
\item Debugging/visualization
\end{itemize}
\end{frame}

\begin{frame}
\frametitle{Reformulation}
\begin{itemize}
\item Just because the user has modelled it this way, it doesn't mean we have to solve it that way
\begin{itemize}
\item Replace some constraint(s) by other, equivalent constraints
\item Because we don't have that constraint in our system
\item For performance
\end{itemize}
\end{itemize}
\end{frame}

\begin{frame}
\frametitle{Learning}
\begin{itemize}
\item While solving the problem we can learn how to strengthen the model/search 
\begin{itemize}
\item Understand which constraints/method contribute to propagation and change schedule
\item Learn no-good constraints by explaining failure
\item Adapt search strategy based on search experience
\end{itemize}
\end{itemize}
\end{frame}

%% \begin{frame}
%% \frametitle{Refined Process}
%% \begin{center}
%% \begin{tikzpicture}
%% \node[shape=rectangle,fill=pantone127-4] (problem) at (3,7) {Problem};
%% \node (human) at (3,6) {Human};
%% \node[shape=rectangle,fill=pantone127-4] (model) at (3,5) {User Model};
%% \node (reformulation) at (3,4) {Reformulation};
%% \node[shape=rectangle,fill=pantone127-4] (implementation) at (3,3) {Implementation Model};
%% \node (solver) at (3,2) {Constraint Solver/Search};
%% \node[shape=rectangle,fill=pantone157-8] (solution) at (3,1) {Solution};
%% \draw[-] (problem) -- (human);
%% \draw[->] (human) -- (model);
%% \draw[-] (model) -- (reformulation);
%% \draw[->] (reformulation) -- (implementation);
%% \draw[-] (implementation) -- (solver);
%% \draw[->] (solver) -- (0,2) -- (0,3) -- (implementation);
%% \draw[->] (solver) -- (solution);
%% \end{tikzpicture}
%% \end{center}
%% \end{frame}

%\begin{frame}
%\frametitle{What is CP actually used for?}
%\url{http://hsimonis.wordpress.com}
%\includegraphics[width=10cm]{../introduction/images/blog}
%\end{frame}

%% \begin{frame}
%% \frametitle{Exercises}
%% \begin{itemize}
%% \item Install Constraint Solver of your Choice
%% \item MiniZinc (MiningZinc) will be used in later sessions
%% \item Run Sendmoremoney, Sudoku, Queens examples
%% \item Can you choose which propagator to use for alldifferent?
%% \item Can you use different search strategies?
%% \end{itemize}
%% \end{frame}

\begin{frame}
\frametitle{More Learning Resources}
\begin{itemize}
\item Survey of Methods, Resources, and Formats for Teaching Constraint Programming
\begin{itemize}
\item by Tejas Santanam, Helmut Simonis
\item \url{https://doi.org/10.48550/arXiv.2403.12717}
\item Based on survey of community for WTCP 2023
\item \url{https://hsimonis.github.io/WTCP2023/}
\end{itemize}
\end{itemize}
\end{frame}


}

\clearpage
\mode<all>{
\part{Customizing Search}
}
\begin{frame}
\partpage
\end{frame}
\mode<all>{
\begin{frame}
\frametitle{Objectives}
\begin{itemize}
\item Overview of Core Constraint Programming
\item Three Main Concepts
\begin{itemize}
\item Constraint Propagation
\item Global Constraints
\item Customizing Search
\end{itemize}
%\item Get Some Experience with MiniZinc
\item Topics will be treated in more detail in later parts of the school
\item Based on Examples, not Formal Description
\end{itemize}
\end{frame}

\begin{frame}
\frametitle{Outline}
\begin{itemize}
\item Why Constraint Programming?
\item Constraint Propagation
\item Global Constraints
\item Customizing Search
%\item Choices in Modelling
%\item Decomposition
%\item More Global Constraints
%\item Methodology
%\item What is missing?
\end{itemize}
\end{frame}

% \begin{frame}
  % \frametitle{Using MiniZinc IDE}
  % \begin{itemize}
  % \item Developed in the Australian NICTA project
  % \item Maintained by Monash University
  % \item Modelling tool with multiple back-end solvers
  % \item Available from \url{https://www.minizinc.org/}
  % \end{itemize}
% \end{frame}


% \begin{frame}
% \frametitle{Examples in ECLiPSe}
% \begin{itemize}
% \item Open sourced constraint programming language
% \item Development goes back to 1985
% \item ECRC, ICL, IC-Parc, PTL, Cisco
% \item \url{https://eclipseclp.org/}
% \item Specialities
% \begin{itemize}
% \item Develop new solvers for specific domains
% \item Integration with MIP
% \end{itemize}
% \item Not included in bundled MiniZinc IDE
% \item Specialized visualization tools used here
  % \begin{itemize}
  % \item CP-Viz, Simonis et al. 2010
    % \end{itemize}
% \end{itemize}
% \end{frame}

\begin{frame}
\frametitle{Tutorial Based on ECLiPSe ELearning Course}
\begin{itemize}
\item Self-study course in constraint programming
\item Supported by Cisco Systems and Silicon Valley Community Foundation
\item Multi-media format, video lectures, slides, handout etc
\item {\small \url{https://eclipseclp.org/ELearning/index.html}}
\end{itemize}
\end{frame}

\begin{frame}
\frametitle{Also Part of CRT-AI Constraint Week}
\begin{itemize}
\item Annual one week course on CP and Optimization in Ireland
\item Part of national training program for PhD students in AI
\item \url{https://www.crt-ai.ie/}
\end{itemize}
\end{frame}



\begin{frame}
\frametitle{Constraint Programming - in a nutshell}
\begin{itemize}
\item Declarative description of problems with
\begin{itemize}
\item {\em Variables} which range over (finite) sets of values
\item {\em Constraints} over subsets of variables which restrict possible value combinations
\item A {\em solution} is a value assignment which satisfies all constraints
\end{itemize}

\item Constraint propagation/reasoning
\begin{itemize}
\item Removing inconsistent values for variables
\item Detect failure if constraint can not be satisfied
\item Interaction of constraints via shared variables
\item Incomplete
\end{itemize}

\item Search
\begin{itemize}
\item User controlled assignment of values to variables
\item Each step triggers constraint propagation 
\end{itemize}
\item Different domains require/allow different methods
\end{itemize}
\end{frame}

\begin{frame}
  \frametitle{Constraint Programming is Different}
  \begin{itemize}
  \item Declarative Programming
    \begin{itemize}
    \item Concentrate on what you want
      \item Not how to get there
      \item Program != Algorithm
      \item Program = Model
    \end{itemize}
    \item Applied to Combinatorial Problems
      \begin{itemize}
        \item No complete polynomial algorithms known (exist?)
        \item CP less ad-hoc than heuristics
        \item Models can evolve
  \end{itemize}
  \end{itemize}
  \end{frame}
    
\begin{frame}
  \frametitle{A Subtractive Process}
  \begin{textblock}{4}(8,-3)
    \includegraphics[width=4cm]{../introduction/images/stages}
  \end{textblock}
  \vfill
  \begin{quote}
    ``Oh, bosh, as Mr. Ruskin says. Sculpture, per se, is the simplest thing in the world. All you have to do is to take a big chunk of marble and a hammer and chisel, make up your mind what you are about to create and chip off all the marble you don't want.''-Paris Gaulois.
  \end{quote}
  
  {\tiny Source: \url{https://quoteinvestigator.com/2014/06/22/chip-away/}}
\end{frame}

\begin{frame}
\frametitle{Basic Process}
\begin{center}
\begin{tikzpicture}
\node[shape=rectangle,fill=pantone127-4] (problem) at (2,5) {Problem};
\node (human) at (2,4) {Human};
\node[shape=rectangle,fill=pantone127-4] (model) at (2,3) {Model};
\node (solver) at (2,2) {Constraint Solver/Search};
\node[shape=rectangle,fill=pantone157-8] (solution) at (2,1) {Solution};
\draw[-] (problem) -- (human);
\draw[->] (human) -- (model);
\draw[-] (model) -- (solver);
\draw[->] (solver) -- (solution);
\end{tikzpicture}
\end{center}
\end{frame}

\begin{frame}
\frametitle{More Realistic}
\begin{center}
\begin{tikzpicture}[xscale=4.0]
\node[shape=rectangle,fill=pantone127-4] (problem) at (2,5) {Problem};
\node (human) at (2,4) {Human};
\node[shape=rectangle,fill=pantone127-4] (model) at (2,3) {Model};
\node (solver) at (2,2) {Constraint Solver/Search};
\node[shape=rectangle,fill=pantone24-8] (hangs) at (1,1) {Hangs};
\node[shape=rectangle,fill=pantone157-8] (solution) at (2,1) {Solution};
\node[shape=rectangle,fill=pantone24-8] (wrong) at (3,1) {Wrong Solution};
\draw[-] (problem) -- (human);
\draw[->] (human) -- (model);
\draw[-] (model) -- (solver);
\draw[->] (solver) -- (hangs);
\draw[->] (solver) -- (solution);
\draw[->] (solver) -- (wrong);
\draw[->] (hangs) -- (1,4) -- (human);
\draw[->] (wrong) -- (3,4) -- (human);
\end{tikzpicture}
\end{center}
\end{frame}

\begin{frame}
\frametitle{Dual Role of Model}
\begin{itemize}
\item Allows Human to Express Problem
\begin{itemize}
\item Close to Problem Domain
\item Constraints as Abstractions
\end{itemize}
\item Allows Solver to Execute
\begin{itemize}
\item Variables as Communication Mechanism
\item Constraints as Algorithms
\end{itemize}
\end{itemize}
\end{frame}

\begin{frame}
\frametitle{Modelling Frameworks}
\begin{itemize}
\item \href{https://www.minizinc.org/}{MiniZinc} (NICTA, Monash University, Australia)
\item \href{https://github.com/eomahony/Numberjack}{NumberJack} (Insight, Ireland) 
\item \href{https://www-users.york.ac.uk/peter.nightingale/savilerow//}{EssencePrime/SavilleRow} (UK) 
\item \href{https://cpmpy.readthedocs.io/en/latest/}{CPMpy} (KU Leuven)
\item Allow use of multiple back-end solvers
\item Compile model into variants for each solver
\item A priori solver independent model(CP, MIP, SAT)
\end{itemize}
\end{frame} 

\begin{frame}
\frametitle{Framework Process}
\begin{center}
\begin{tikzpicture}[xscale=2]
\node[shape=rectangle,fill=pantone127-4] (problem) at (2.5,5) {Problem};
\node (human) at (2.5,4) {Human};
\node[shape=rectangle,fill=pantone127-4] (model) at (2.5,3) {Model};
\node (compile) at (2.5,2) {Compile/Reformulate};
\node (solver1) at (1,1) {CP};
\node (solver2) at (2,1) {MIP};
\node (solver3) at (3,1) {SAT};
\node (solver4) at (4,1) {Other};
\node[shape=rectangle,fill=pantone157-8] (solution1) at (1,0) {Solution};
\node[shape=rectangle,fill=pantone157-8] (solution2) at (2,0) {Solution};
\node[shape=rectangle,fill=pantone157-8] (solution3) at (3,0) {Solution};
\node[shape=rectangle,fill=pantone157-8] (solution4) at (4,0) {Solution};
\draw[-] (problem) -- (human);
\draw[->] (human) -- (model);
\draw[-] (model) -- (compile);
\draw[-] (compile) -- (solver1);
\draw[-] (compile) -- (solver2);
\draw[-] (compile) -- (solver3);
\draw[-] (compile) -- (solver4);
\draw[->] (solver1) -- (solution1);
\draw[->] (solver2) -- (solution2);
\draw[->] (solver3) -- (solution3);
\draw[->] (solver4) -- (solution4);
\end{tikzpicture}
\end{center}
\end{frame}

% \begin{frame}
  % \frametitle{Do It Now!}
  % \begin{itemize}
  % \item Download and install Minizinc
  % \item \url{https://www.minizinc.org/}
  % \end{itemize}
% \end{frame}


\begin{frame}
\frametitle{Why use Puzzles as Examples?}
\begin{itemize}
\item Easy to understand the problem
\item Solvable by hand without specialized knowledge
\item Possible to compare automated to manual solving process
\end{itemize}

\begin{quote}
The puzzle, though inanimate, is presented as a solvable problem without lasting negative consequences, a very low-risk low-reward situation. By being a puzzle, the object is attempting to convince the user that it must be completed.
\end{quote}
{\tiny Source: \href{https://www.everydayrhetoric.com/2020/03/the-rhetoric-of-puzzles/}{Every Day Rhetoric}}
\end{frame}


\mode<all>{
\part{Basic Constraint Propagation}
}
\begin{frame}
\partpage
\end{frame}
\mode<all>{
\input{../sendmore/body}
}

\mode<all>{
\part{Global Constraints}
}
\begin{frame}
\partpage
\end{frame}
\mode<all>{
\input{../sudoku/body}
}

\clearpage
\mode<all>{
\part{Customizing Search}
}
\begin{frame}
\partpage
\end{frame}
\mode<all>{
\input{../nqueen/body}
}


%% \clearpage
%% \mode<all>{
%%   \part{Problem Decomposition}
%% }
%% \begin{frame}
%%   \partpage
%% \end{frame}
%% \mode<all>{
%%   \input{../party/body}
%%   }

%% \clearpage
%% \mode<all>{
%%   \part{More Global Constraints}
%% }
%% \begin{frame}
%%   \partpage
%% \end{frame}
%% \mode<all>{
%%   \input{../car/body}
%%   }

%% \clearpage
%% \mode<all>{
%%   \part{Methodology}
%% }
%% \begin{frame}
%%   \partpage
%% \end{frame}
%% \mode<all>{
%%   \input{../methodology/body}
%%   }

\clearpage
\mode<all>{
\part{What is missing?}
}
\begin{frame}
\partpage
\end{frame}

\begin{frame}
\frametitle{Many Specialized Topics}
\begin{itemize}
\item How to design efficient core engine
\item Hybrids with LP/MIP tools
\item Hybrids with SAT
\item Symmetry breaking
\item Use of MDD/BDD to encode sets of solutions
\item High level modelling tools
\item Debugging/visualization
\end{itemize}
\end{frame}

\begin{frame}
\frametitle{Reformulation}
\begin{itemize}
\item Just because the user has modelled it this way, it doesn't mean we have to solve it that way
\begin{itemize}
\item Replace some constraint(s) by other, equivalent constraints
\item Because we don't have that constraint in our system
\item For performance
\end{itemize}
\end{itemize}
\end{frame}

\begin{frame}
\frametitle{Learning}
\begin{itemize}
\item While solving the problem we can learn how to strengthen the model/search 
\begin{itemize}
\item Understand which constraints/method contribute to propagation and change schedule
\item Learn no-good constraints by explaining failure
\item Adapt search strategy based on search experience
\end{itemize}
\end{itemize}
\end{frame}

%% \begin{frame}
%% \frametitle{Refined Process}
%% \begin{center}
%% \begin{tikzpicture}
%% \node[shape=rectangle,fill=pantone127-4] (problem) at (3,7) {Problem};
%% \node (human) at (3,6) {Human};
%% \node[shape=rectangle,fill=pantone127-4] (model) at (3,5) {User Model};
%% \node (reformulation) at (3,4) {Reformulation};
%% \node[shape=rectangle,fill=pantone127-4] (implementation) at (3,3) {Implementation Model};
%% \node (solver) at (3,2) {Constraint Solver/Search};
%% \node[shape=rectangle,fill=pantone157-8] (solution) at (3,1) {Solution};
%% \draw[-] (problem) -- (human);
%% \draw[->] (human) -- (model);
%% \draw[-] (model) -- (reformulation);
%% \draw[->] (reformulation) -- (implementation);
%% \draw[-] (implementation) -- (solver);
%% \draw[->] (solver) -- (0,2) -- (0,3) -- (implementation);
%% \draw[->] (solver) -- (solution);
%% \end{tikzpicture}
%% \end{center}
%% \end{frame}

%\begin{frame}
%\frametitle{What is CP actually used for?}
%\url{http://hsimonis.wordpress.com}
%\includegraphics[width=10cm]{../introduction/images/blog}
%\end{frame}

%% \begin{frame}
%% \frametitle{Exercises}
%% \begin{itemize}
%% \item Install Constraint Solver of your Choice
%% \item MiniZinc (MiningZinc) will be used in later sessions
%% \item Run Sendmoremoney, Sudoku, Queens examples
%% \item Can you choose which propagator to use for alldifferent?
%% \item Can you use different search strategies?
%% \end{itemize}
%% \end{frame}

\begin{frame}
\frametitle{More Learning Resources}
\begin{itemize}
\item Survey of Methods, Resources, and Formats for Teaching Constraint Programming
\begin{itemize}
\item by Tejas Santanam, Helmut Simonis
\item \url{https://doi.org/10.48550/arXiv.2403.12717}
\item Based on survey of community for WTCP 2023
\item \url{https://hsimonis.github.io/WTCP2023/}
\end{itemize}
\end{itemize}
\end{frame}


}


%% \clearpage
%% \mode<all>{
%%   \part{Problem Decomposition}
%% }
%% \begin{frame}
%%   \partpage
%% \end{frame}
%% \mode<all>{
%%   \begin{frame}
\frametitle{Objectives}
\begin{itemize}
\item Overview of Core Constraint Programming
\item Three Main Concepts
\begin{itemize}
\item Constraint Propagation
\item Global Constraints
\item Customizing Search
\end{itemize}
%\item Get Some Experience with MiniZinc
\item Topics will be treated in more detail in later parts of the school
\item Based on Examples, not Formal Description
\end{itemize}
\end{frame}

\begin{frame}
\frametitle{Outline}
\begin{itemize}
\item Why Constraint Programming?
\item Constraint Propagation
\item Global Constraints
\item Customizing Search
%\item Choices in Modelling
%\item Decomposition
%\item More Global Constraints
%\item Methodology
%\item What is missing?
\end{itemize}
\end{frame}

% \begin{frame}
  % \frametitle{Using MiniZinc IDE}
  % \begin{itemize}
  % \item Developed in the Australian NICTA project
  % \item Maintained by Monash University
  % \item Modelling tool with multiple back-end solvers
  % \item Available from \url{https://www.minizinc.org/}
  % \end{itemize}
% \end{frame}


% \begin{frame}
% \frametitle{Examples in ECLiPSe}
% \begin{itemize}
% \item Open sourced constraint programming language
% \item Development goes back to 1985
% \item ECRC, ICL, IC-Parc, PTL, Cisco
% \item \url{https://eclipseclp.org/}
% \item Specialities
% \begin{itemize}
% \item Develop new solvers for specific domains
% \item Integration with MIP
% \end{itemize}
% \item Not included in bundled MiniZinc IDE
% \item Specialized visualization tools used here
  % \begin{itemize}
  % \item CP-Viz, Simonis et al. 2010
    % \end{itemize}
% \end{itemize}
% \end{frame}

\begin{frame}
\frametitle{Tutorial Based on ECLiPSe ELearning Course}
\begin{itemize}
\item Self-study course in constraint programming
\item Supported by Cisco Systems and Silicon Valley Community Foundation
\item Multi-media format, video lectures, slides, handout etc
\item {\small \url{https://eclipseclp.org/ELearning/index.html}}
\end{itemize}
\end{frame}

\begin{frame}
\frametitle{Also Part of CRT-AI Constraint Week}
\begin{itemize}
\item Annual one week course on CP and Optimization in Ireland
\item Part of national training program for PhD students in AI
\item \url{https://www.crt-ai.ie/}
\end{itemize}
\end{frame}



\begin{frame}
\frametitle{Constraint Programming - in a nutshell}
\begin{itemize}
\item Declarative description of problems with
\begin{itemize}
\item {\em Variables} which range over (finite) sets of values
\item {\em Constraints} over subsets of variables which restrict possible value combinations
\item A {\em solution} is a value assignment which satisfies all constraints
\end{itemize}

\item Constraint propagation/reasoning
\begin{itemize}
\item Removing inconsistent values for variables
\item Detect failure if constraint can not be satisfied
\item Interaction of constraints via shared variables
\item Incomplete
\end{itemize}

\item Search
\begin{itemize}
\item User controlled assignment of values to variables
\item Each step triggers constraint propagation 
\end{itemize}
\item Different domains require/allow different methods
\end{itemize}
\end{frame}

\begin{frame}
  \frametitle{Constraint Programming is Different}
  \begin{itemize}
  \item Declarative Programming
    \begin{itemize}
    \item Concentrate on what you want
      \item Not how to get there
      \item Program != Algorithm
      \item Program = Model
    \end{itemize}
    \item Applied to Combinatorial Problems
      \begin{itemize}
        \item No complete polynomial algorithms known (exist?)
        \item CP less ad-hoc than heuristics
        \item Models can evolve
  \end{itemize}
  \end{itemize}
  \end{frame}
    
\begin{frame}
  \frametitle{A Subtractive Process}
  \begin{textblock}{4}(8,-3)
    \includegraphics[width=4cm]{../introduction/images/stages}
  \end{textblock}
  \vfill
  \begin{quote}
    ``Oh, bosh, as Mr. Ruskin says. Sculpture, per se, is the simplest thing in the world. All you have to do is to take a big chunk of marble and a hammer and chisel, make up your mind what you are about to create and chip off all the marble you don't want.''-Paris Gaulois.
  \end{quote}
  
  {\tiny Source: \url{https://quoteinvestigator.com/2014/06/22/chip-away/}}
\end{frame}

\begin{frame}
\frametitle{Basic Process}
\begin{center}
\begin{tikzpicture}
\node[shape=rectangle,fill=pantone127-4] (problem) at (2,5) {Problem};
\node (human) at (2,4) {Human};
\node[shape=rectangle,fill=pantone127-4] (model) at (2,3) {Model};
\node (solver) at (2,2) {Constraint Solver/Search};
\node[shape=rectangle,fill=pantone157-8] (solution) at (2,1) {Solution};
\draw[-] (problem) -- (human);
\draw[->] (human) -- (model);
\draw[-] (model) -- (solver);
\draw[->] (solver) -- (solution);
\end{tikzpicture}
\end{center}
\end{frame}

\begin{frame}
\frametitle{More Realistic}
\begin{center}
\begin{tikzpicture}[xscale=4.0]
\node[shape=rectangle,fill=pantone127-4] (problem) at (2,5) {Problem};
\node (human) at (2,4) {Human};
\node[shape=rectangle,fill=pantone127-4] (model) at (2,3) {Model};
\node (solver) at (2,2) {Constraint Solver/Search};
\node[shape=rectangle,fill=pantone24-8] (hangs) at (1,1) {Hangs};
\node[shape=rectangle,fill=pantone157-8] (solution) at (2,1) {Solution};
\node[shape=rectangle,fill=pantone24-8] (wrong) at (3,1) {Wrong Solution};
\draw[-] (problem) -- (human);
\draw[->] (human) -- (model);
\draw[-] (model) -- (solver);
\draw[->] (solver) -- (hangs);
\draw[->] (solver) -- (solution);
\draw[->] (solver) -- (wrong);
\draw[->] (hangs) -- (1,4) -- (human);
\draw[->] (wrong) -- (3,4) -- (human);
\end{tikzpicture}
\end{center}
\end{frame}

\begin{frame}
\frametitle{Dual Role of Model}
\begin{itemize}
\item Allows Human to Express Problem
\begin{itemize}
\item Close to Problem Domain
\item Constraints as Abstractions
\end{itemize}
\item Allows Solver to Execute
\begin{itemize}
\item Variables as Communication Mechanism
\item Constraints as Algorithms
\end{itemize}
\end{itemize}
\end{frame}

\begin{frame}
\frametitle{Modelling Frameworks}
\begin{itemize}
\item \href{https://www.minizinc.org/}{MiniZinc} (NICTA, Monash University, Australia)
\item \href{https://github.com/eomahony/Numberjack}{NumberJack} (Insight, Ireland) 
\item \href{https://www-users.york.ac.uk/peter.nightingale/savilerow//}{EssencePrime/SavilleRow} (UK) 
\item \href{https://cpmpy.readthedocs.io/en/latest/}{CPMpy} (KU Leuven)
\item Allow use of multiple back-end solvers
\item Compile model into variants for each solver
\item A priori solver independent model(CP, MIP, SAT)
\end{itemize}
\end{frame} 

\begin{frame}
\frametitle{Framework Process}
\begin{center}
\begin{tikzpicture}[xscale=2]
\node[shape=rectangle,fill=pantone127-4] (problem) at (2.5,5) {Problem};
\node (human) at (2.5,4) {Human};
\node[shape=rectangle,fill=pantone127-4] (model) at (2.5,3) {Model};
\node (compile) at (2.5,2) {Compile/Reformulate};
\node (solver1) at (1,1) {CP};
\node (solver2) at (2,1) {MIP};
\node (solver3) at (3,1) {SAT};
\node (solver4) at (4,1) {Other};
\node[shape=rectangle,fill=pantone157-8] (solution1) at (1,0) {Solution};
\node[shape=rectangle,fill=pantone157-8] (solution2) at (2,0) {Solution};
\node[shape=rectangle,fill=pantone157-8] (solution3) at (3,0) {Solution};
\node[shape=rectangle,fill=pantone157-8] (solution4) at (4,0) {Solution};
\draw[-] (problem) -- (human);
\draw[->] (human) -- (model);
\draw[-] (model) -- (compile);
\draw[-] (compile) -- (solver1);
\draw[-] (compile) -- (solver2);
\draw[-] (compile) -- (solver3);
\draw[-] (compile) -- (solver4);
\draw[->] (solver1) -- (solution1);
\draw[->] (solver2) -- (solution2);
\draw[->] (solver3) -- (solution3);
\draw[->] (solver4) -- (solution4);
\end{tikzpicture}
\end{center}
\end{frame}

% \begin{frame}
  % \frametitle{Do It Now!}
  % \begin{itemize}
  % \item Download and install Minizinc
  % \item \url{https://www.minizinc.org/}
  % \end{itemize}
% \end{frame}


\begin{frame}
\frametitle{Why use Puzzles as Examples?}
\begin{itemize}
\item Easy to understand the problem
\item Solvable by hand without specialized knowledge
\item Possible to compare automated to manual solving process
\end{itemize}

\begin{quote}
The puzzle, though inanimate, is presented as a solvable problem without lasting negative consequences, a very low-risk low-reward situation. By being a puzzle, the object is attempting to convince the user that it must be completed.
\end{quote}
{\tiny Source: \href{https://www.everydayrhetoric.com/2020/03/the-rhetoric-of-puzzles/}{Every Day Rhetoric}}
\end{frame}


\mode<all>{
\part{Basic Constraint Propagation}
}
\begin{frame}
\partpage
\end{frame}
\mode<all>{
\input{../sendmore/body}
}

\mode<all>{
\part{Global Constraints}
}
\begin{frame}
\partpage
\end{frame}
\mode<all>{
\input{../sudoku/body}
}

\clearpage
\mode<all>{
\part{Customizing Search}
}
\begin{frame}
\partpage
\end{frame}
\mode<all>{
\input{../nqueen/body}
}


%% \clearpage
%% \mode<all>{
%%   \part{Problem Decomposition}
%% }
%% \begin{frame}
%%   \partpage
%% \end{frame}
%% \mode<all>{
%%   \input{../party/body}
%%   }

%% \clearpage
%% \mode<all>{
%%   \part{More Global Constraints}
%% }
%% \begin{frame}
%%   \partpage
%% \end{frame}
%% \mode<all>{
%%   \input{../car/body}
%%   }

%% \clearpage
%% \mode<all>{
%%   \part{Methodology}
%% }
%% \begin{frame}
%%   \partpage
%% \end{frame}
%% \mode<all>{
%%   \input{../methodology/body}
%%   }

\clearpage
\mode<all>{
\part{What is missing?}
}
\begin{frame}
\partpage
\end{frame}

\begin{frame}
\frametitle{Many Specialized Topics}
\begin{itemize}
\item How to design efficient core engine
\item Hybrids with LP/MIP tools
\item Hybrids with SAT
\item Symmetry breaking
\item Use of MDD/BDD to encode sets of solutions
\item High level modelling tools
\item Debugging/visualization
\end{itemize}
\end{frame}

\begin{frame}
\frametitle{Reformulation}
\begin{itemize}
\item Just because the user has modelled it this way, it doesn't mean we have to solve it that way
\begin{itemize}
\item Replace some constraint(s) by other, equivalent constraints
\item Because we don't have that constraint in our system
\item For performance
\end{itemize}
\end{itemize}
\end{frame}

\begin{frame}
\frametitle{Learning}
\begin{itemize}
\item While solving the problem we can learn how to strengthen the model/search 
\begin{itemize}
\item Understand which constraints/method contribute to propagation and change schedule
\item Learn no-good constraints by explaining failure
\item Adapt search strategy based on search experience
\end{itemize}
\end{itemize}
\end{frame}

%% \begin{frame}
%% \frametitle{Refined Process}
%% \begin{center}
%% \begin{tikzpicture}
%% \node[shape=rectangle,fill=pantone127-4] (problem) at (3,7) {Problem};
%% \node (human) at (3,6) {Human};
%% \node[shape=rectangle,fill=pantone127-4] (model) at (3,5) {User Model};
%% \node (reformulation) at (3,4) {Reformulation};
%% \node[shape=rectangle,fill=pantone127-4] (implementation) at (3,3) {Implementation Model};
%% \node (solver) at (3,2) {Constraint Solver/Search};
%% \node[shape=rectangle,fill=pantone157-8] (solution) at (3,1) {Solution};
%% \draw[-] (problem) -- (human);
%% \draw[->] (human) -- (model);
%% \draw[-] (model) -- (reformulation);
%% \draw[->] (reformulation) -- (implementation);
%% \draw[-] (implementation) -- (solver);
%% \draw[->] (solver) -- (0,2) -- (0,3) -- (implementation);
%% \draw[->] (solver) -- (solution);
%% \end{tikzpicture}
%% \end{center}
%% \end{frame}

%\begin{frame}
%\frametitle{What is CP actually used for?}
%\url{http://hsimonis.wordpress.com}
%\includegraphics[width=10cm]{../introduction/images/blog}
%\end{frame}

%% \begin{frame}
%% \frametitle{Exercises}
%% \begin{itemize}
%% \item Install Constraint Solver of your Choice
%% \item MiniZinc (MiningZinc) will be used in later sessions
%% \item Run Sendmoremoney, Sudoku, Queens examples
%% \item Can you choose which propagator to use for alldifferent?
%% \item Can you use different search strategies?
%% \end{itemize}
%% \end{frame}

\begin{frame}
\frametitle{More Learning Resources}
\begin{itemize}
\item Survey of Methods, Resources, and Formats for Teaching Constraint Programming
\begin{itemize}
\item by Tejas Santanam, Helmut Simonis
\item \url{https://doi.org/10.48550/arXiv.2403.12717}
\item Based on survey of community for WTCP 2023
\item \url{https://hsimonis.github.io/WTCP2023/}
\end{itemize}
\end{itemize}
\end{frame}


%%   }

%% \clearpage
%% \mode<all>{
%%   \part{More Global Constraints}
%% }
%% \begin{frame}
%%   \partpage
%% \end{frame}
%% \mode<all>{
%%   \begin{frame}
\frametitle{Objectives}
\begin{itemize}
\item Overview of Core Constraint Programming
\item Three Main Concepts
\begin{itemize}
\item Constraint Propagation
\item Global Constraints
\item Customizing Search
\end{itemize}
%\item Get Some Experience with MiniZinc
\item Topics will be treated in more detail in later parts of the school
\item Based on Examples, not Formal Description
\end{itemize}
\end{frame}

\begin{frame}
\frametitle{Outline}
\begin{itemize}
\item Why Constraint Programming?
\item Constraint Propagation
\item Global Constraints
\item Customizing Search
%\item Choices in Modelling
%\item Decomposition
%\item More Global Constraints
%\item Methodology
%\item What is missing?
\end{itemize}
\end{frame}

% \begin{frame}
  % \frametitle{Using MiniZinc IDE}
  % \begin{itemize}
  % \item Developed in the Australian NICTA project
  % \item Maintained by Monash University
  % \item Modelling tool with multiple back-end solvers
  % \item Available from \url{https://www.minizinc.org/}
  % \end{itemize}
% \end{frame}


% \begin{frame}
% \frametitle{Examples in ECLiPSe}
% \begin{itemize}
% \item Open sourced constraint programming language
% \item Development goes back to 1985
% \item ECRC, ICL, IC-Parc, PTL, Cisco
% \item \url{https://eclipseclp.org/}
% \item Specialities
% \begin{itemize}
% \item Develop new solvers for specific domains
% \item Integration with MIP
% \end{itemize}
% \item Not included in bundled MiniZinc IDE
% \item Specialized visualization tools used here
  % \begin{itemize}
  % \item CP-Viz, Simonis et al. 2010
    % \end{itemize}
% \end{itemize}
% \end{frame}

\begin{frame}
\frametitle{Tutorial Based on ECLiPSe ELearning Course}
\begin{itemize}
\item Self-study course in constraint programming
\item Supported by Cisco Systems and Silicon Valley Community Foundation
\item Multi-media format, video lectures, slides, handout etc
\item {\small \url{https://eclipseclp.org/ELearning/index.html}}
\end{itemize}
\end{frame}

\begin{frame}
\frametitle{Also Part of CRT-AI Constraint Week}
\begin{itemize}
\item Annual one week course on CP and Optimization in Ireland
\item Part of national training program for PhD students in AI
\item \url{https://www.crt-ai.ie/}
\end{itemize}
\end{frame}



\begin{frame}
\frametitle{Constraint Programming - in a nutshell}
\begin{itemize}
\item Declarative description of problems with
\begin{itemize}
\item {\em Variables} which range over (finite) sets of values
\item {\em Constraints} over subsets of variables which restrict possible value combinations
\item A {\em solution} is a value assignment which satisfies all constraints
\end{itemize}

\item Constraint propagation/reasoning
\begin{itemize}
\item Removing inconsistent values for variables
\item Detect failure if constraint can not be satisfied
\item Interaction of constraints via shared variables
\item Incomplete
\end{itemize}

\item Search
\begin{itemize}
\item User controlled assignment of values to variables
\item Each step triggers constraint propagation 
\end{itemize}
\item Different domains require/allow different methods
\end{itemize}
\end{frame}

\begin{frame}
  \frametitle{Constraint Programming is Different}
  \begin{itemize}
  \item Declarative Programming
    \begin{itemize}
    \item Concentrate on what you want
      \item Not how to get there
      \item Program != Algorithm
      \item Program = Model
    \end{itemize}
    \item Applied to Combinatorial Problems
      \begin{itemize}
        \item No complete polynomial algorithms known (exist?)
        \item CP less ad-hoc than heuristics
        \item Models can evolve
  \end{itemize}
  \end{itemize}
  \end{frame}
    
\begin{frame}
  \frametitle{A Subtractive Process}
  \begin{textblock}{4}(8,-3)
    \includegraphics[width=4cm]{../introduction/images/stages}
  \end{textblock}
  \vfill
  \begin{quote}
    ``Oh, bosh, as Mr. Ruskin says. Sculpture, per se, is the simplest thing in the world. All you have to do is to take a big chunk of marble and a hammer and chisel, make up your mind what you are about to create and chip off all the marble you don't want.''-Paris Gaulois.
  \end{quote}
  
  {\tiny Source: \url{https://quoteinvestigator.com/2014/06/22/chip-away/}}
\end{frame}

\begin{frame}
\frametitle{Basic Process}
\begin{center}
\begin{tikzpicture}
\node[shape=rectangle,fill=pantone127-4] (problem) at (2,5) {Problem};
\node (human) at (2,4) {Human};
\node[shape=rectangle,fill=pantone127-4] (model) at (2,3) {Model};
\node (solver) at (2,2) {Constraint Solver/Search};
\node[shape=rectangle,fill=pantone157-8] (solution) at (2,1) {Solution};
\draw[-] (problem) -- (human);
\draw[->] (human) -- (model);
\draw[-] (model) -- (solver);
\draw[->] (solver) -- (solution);
\end{tikzpicture}
\end{center}
\end{frame}

\begin{frame}
\frametitle{More Realistic}
\begin{center}
\begin{tikzpicture}[xscale=4.0]
\node[shape=rectangle,fill=pantone127-4] (problem) at (2,5) {Problem};
\node (human) at (2,4) {Human};
\node[shape=rectangle,fill=pantone127-4] (model) at (2,3) {Model};
\node (solver) at (2,2) {Constraint Solver/Search};
\node[shape=rectangle,fill=pantone24-8] (hangs) at (1,1) {Hangs};
\node[shape=rectangle,fill=pantone157-8] (solution) at (2,1) {Solution};
\node[shape=rectangle,fill=pantone24-8] (wrong) at (3,1) {Wrong Solution};
\draw[-] (problem) -- (human);
\draw[->] (human) -- (model);
\draw[-] (model) -- (solver);
\draw[->] (solver) -- (hangs);
\draw[->] (solver) -- (solution);
\draw[->] (solver) -- (wrong);
\draw[->] (hangs) -- (1,4) -- (human);
\draw[->] (wrong) -- (3,4) -- (human);
\end{tikzpicture}
\end{center}
\end{frame}

\begin{frame}
\frametitle{Dual Role of Model}
\begin{itemize}
\item Allows Human to Express Problem
\begin{itemize}
\item Close to Problem Domain
\item Constraints as Abstractions
\end{itemize}
\item Allows Solver to Execute
\begin{itemize}
\item Variables as Communication Mechanism
\item Constraints as Algorithms
\end{itemize}
\end{itemize}
\end{frame}

\begin{frame}
\frametitle{Modelling Frameworks}
\begin{itemize}
\item \href{https://www.minizinc.org/}{MiniZinc} (NICTA, Monash University, Australia)
\item \href{https://github.com/eomahony/Numberjack}{NumberJack} (Insight, Ireland) 
\item \href{https://www-users.york.ac.uk/peter.nightingale/savilerow//}{EssencePrime/SavilleRow} (UK) 
\item \href{https://cpmpy.readthedocs.io/en/latest/}{CPMpy} (KU Leuven)
\item Allow use of multiple back-end solvers
\item Compile model into variants for each solver
\item A priori solver independent model(CP, MIP, SAT)
\end{itemize}
\end{frame} 

\begin{frame}
\frametitle{Framework Process}
\begin{center}
\begin{tikzpicture}[xscale=2]
\node[shape=rectangle,fill=pantone127-4] (problem) at (2.5,5) {Problem};
\node (human) at (2.5,4) {Human};
\node[shape=rectangle,fill=pantone127-4] (model) at (2.5,3) {Model};
\node (compile) at (2.5,2) {Compile/Reformulate};
\node (solver1) at (1,1) {CP};
\node (solver2) at (2,1) {MIP};
\node (solver3) at (3,1) {SAT};
\node (solver4) at (4,1) {Other};
\node[shape=rectangle,fill=pantone157-8] (solution1) at (1,0) {Solution};
\node[shape=rectangle,fill=pantone157-8] (solution2) at (2,0) {Solution};
\node[shape=rectangle,fill=pantone157-8] (solution3) at (3,0) {Solution};
\node[shape=rectangle,fill=pantone157-8] (solution4) at (4,0) {Solution};
\draw[-] (problem) -- (human);
\draw[->] (human) -- (model);
\draw[-] (model) -- (compile);
\draw[-] (compile) -- (solver1);
\draw[-] (compile) -- (solver2);
\draw[-] (compile) -- (solver3);
\draw[-] (compile) -- (solver4);
\draw[->] (solver1) -- (solution1);
\draw[->] (solver2) -- (solution2);
\draw[->] (solver3) -- (solution3);
\draw[->] (solver4) -- (solution4);
\end{tikzpicture}
\end{center}
\end{frame}

% \begin{frame}
  % \frametitle{Do It Now!}
  % \begin{itemize}
  % \item Download and install Minizinc
  % \item \url{https://www.minizinc.org/}
  % \end{itemize}
% \end{frame}


\begin{frame}
\frametitle{Why use Puzzles as Examples?}
\begin{itemize}
\item Easy to understand the problem
\item Solvable by hand without specialized knowledge
\item Possible to compare automated to manual solving process
\end{itemize}

\begin{quote}
The puzzle, though inanimate, is presented as a solvable problem without lasting negative consequences, a very low-risk low-reward situation. By being a puzzle, the object is attempting to convince the user that it must be completed.
\end{quote}
{\tiny Source: \href{https://www.everydayrhetoric.com/2020/03/the-rhetoric-of-puzzles/}{Every Day Rhetoric}}
\end{frame}


\mode<all>{
\part{Basic Constraint Propagation}
}
\begin{frame}
\partpage
\end{frame}
\mode<all>{
\input{../sendmore/body}
}

\mode<all>{
\part{Global Constraints}
}
\begin{frame}
\partpage
\end{frame}
\mode<all>{
\input{../sudoku/body}
}

\clearpage
\mode<all>{
\part{Customizing Search}
}
\begin{frame}
\partpage
\end{frame}
\mode<all>{
\input{../nqueen/body}
}


%% \clearpage
%% \mode<all>{
%%   \part{Problem Decomposition}
%% }
%% \begin{frame}
%%   \partpage
%% \end{frame}
%% \mode<all>{
%%   \input{../party/body}
%%   }

%% \clearpage
%% \mode<all>{
%%   \part{More Global Constraints}
%% }
%% \begin{frame}
%%   \partpage
%% \end{frame}
%% \mode<all>{
%%   \input{../car/body}
%%   }

%% \clearpage
%% \mode<all>{
%%   \part{Methodology}
%% }
%% \begin{frame}
%%   \partpage
%% \end{frame}
%% \mode<all>{
%%   \input{../methodology/body}
%%   }

\clearpage
\mode<all>{
\part{What is missing?}
}
\begin{frame}
\partpage
\end{frame}

\begin{frame}
\frametitle{Many Specialized Topics}
\begin{itemize}
\item How to design efficient core engine
\item Hybrids with LP/MIP tools
\item Hybrids with SAT
\item Symmetry breaking
\item Use of MDD/BDD to encode sets of solutions
\item High level modelling tools
\item Debugging/visualization
\end{itemize}
\end{frame}

\begin{frame}
\frametitle{Reformulation}
\begin{itemize}
\item Just because the user has modelled it this way, it doesn't mean we have to solve it that way
\begin{itemize}
\item Replace some constraint(s) by other, equivalent constraints
\item Because we don't have that constraint in our system
\item For performance
\end{itemize}
\end{itemize}
\end{frame}

\begin{frame}
\frametitle{Learning}
\begin{itemize}
\item While solving the problem we can learn how to strengthen the model/search 
\begin{itemize}
\item Understand which constraints/method contribute to propagation and change schedule
\item Learn no-good constraints by explaining failure
\item Adapt search strategy based on search experience
\end{itemize}
\end{itemize}
\end{frame}

%% \begin{frame}
%% \frametitle{Refined Process}
%% \begin{center}
%% \begin{tikzpicture}
%% \node[shape=rectangle,fill=pantone127-4] (problem) at (3,7) {Problem};
%% \node (human) at (3,6) {Human};
%% \node[shape=rectangle,fill=pantone127-4] (model) at (3,5) {User Model};
%% \node (reformulation) at (3,4) {Reformulation};
%% \node[shape=rectangle,fill=pantone127-4] (implementation) at (3,3) {Implementation Model};
%% \node (solver) at (3,2) {Constraint Solver/Search};
%% \node[shape=rectangle,fill=pantone157-8] (solution) at (3,1) {Solution};
%% \draw[-] (problem) -- (human);
%% \draw[->] (human) -- (model);
%% \draw[-] (model) -- (reformulation);
%% \draw[->] (reformulation) -- (implementation);
%% \draw[-] (implementation) -- (solver);
%% \draw[->] (solver) -- (0,2) -- (0,3) -- (implementation);
%% \draw[->] (solver) -- (solution);
%% \end{tikzpicture}
%% \end{center}
%% \end{frame}

%\begin{frame}
%\frametitle{What is CP actually used for?}
%\url{http://hsimonis.wordpress.com}
%\includegraphics[width=10cm]{../introduction/images/blog}
%\end{frame}

%% \begin{frame}
%% \frametitle{Exercises}
%% \begin{itemize}
%% \item Install Constraint Solver of your Choice
%% \item MiniZinc (MiningZinc) will be used in later sessions
%% \item Run Sendmoremoney, Sudoku, Queens examples
%% \item Can you choose which propagator to use for alldifferent?
%% \item Can you use different search strategies?
%% \end{itemize}
%% \end{frame}

\begin{frame}
\frametitle{More Learning Resources}
\begin{itemize}
\item Survey of Methods, Resources, and Formats for Teaching Constraint Programming
\begin{itemize}
\item by Tejas Santanam, Helmut Simonis
\item \url{https://doi.org/10.48550/arXiv.2403.12717}
\item Based on survey of community for WTCP 2023
\item \url{https://hsimonis.github.io/WTCP2023/}
\end{itemize}
\end{itemize}
\end{frame}


%%   }

%% \clearpage
%% \mode<all>{
%%   \part{Methodology}
%% }
%% \begin{frame}
%%   \partpage
%% \end{frame}
%% \mode<all>{
%%   \begin{frame}
\frametitle{Objectives}
\begin{itemize}
\item Overview of Core Constraint Programming
\item Three Main Concepts
\begin{itemize}
\item Constraint Propagation
\item Global Constraints
\item Customizing Search
\end{itemize}
%\item Get Some Experience with MiniZinc
\item Topics will be treated in more detail in later parts of the school
\item Based on Examples, not Formal Description
\end{itemize}
\end{frame}

\begin{frame}
\frametitle{Outline}
\begin{itemize}
\item Why Constraint Programming?
\item Constraint Propagation
\item Global Constraints
\item Customizing Search
%\item Choices in Modelling
%\item Decomposition
%\item More Global Constraints
%\item Methodology
%\item What is missing?
\end{itemize}
\end{frame}

% \begin{frame}
  % \frametitle{Using MiniZinc IDE}
  % \begin{itemize}
  % \item Developed in the Australian NICTA project
  % \item Maintained by Monash University
  % \item Modelling tool with multiple back-end solvers
  % \item Available from \url{https://www.minizinc.org/}
  % \end{itemize}
% \end{frame}


% \begin{frame}
% \frametitle{Examples in ECLiPSe}
% \begin{itemize}
% \item Open sourced constraint programming language
% \item Development goes back to 1985
% \item ECRC, ICL, IC-Parc, PTL, Cisco
% \item \url{https://eclipseclp.org/}
% \item Specialities
% \begin{itemize}
% \item Develop new solvers for specific domains
% \item Integration with MIP
% \end{itemize}
% \item Not included in bundled MiniZinc IDE
% \item Specialized visualization tools used here
  % \begin{itemize}
  % \item CP-Viz, Simonis et al. 2010
    % \end{itemize}
% \end{itemize}
% \end{frame}

\begin{frame}
\frametitle{Tutorial Based on ECLiPSe ELearning Course}
\begin{itemize}
\item Self-study course in constraint programming
\item Supported by Cisco Systems and Silicon Valley Community Foundation
\item Multi-media format, video lectures, slides, handout etc
\item {\small \url{https://eclipseclp.org/ELearning/index.html}}
\end{itemize}
\end{frame}

\begin{frame}
\frametitle{Also Part of CRT-AI Constraint Week}
\begin{itemize}
\item Annual one week course on CP and Optimization in Ireland
\item Part of national training program for PhD students in AI
\item \url{https://www.crt-ai.ie/}
\end{itemize}
\end{frame}



\begin{frame}
\frametitle{Constraint Programming - in a nutshell}
\begin{itemize}
\item Declarative description of problems with
\begin{itemize}
\item {\em Variables} which range over (finite) sets of values
\item {\em Constraints} over subsets of variables which restrict possible value combinations
\item A {\em solution} is a value assignment which satisfies all constraints
\end{itemize}

\item Constraint propagation/reasoning
\begin{itemize}
\item Removing inconsistent values for variables
\item Detect failure if constraint can not be satisfied
\item Interaction of constraints via shared variables
\item Incomplete
\end{itemize}

\item Search
\begin{itemize}
\item User controlled assignment of values to variables
\item Each step triggers constraint propagation 
\end{itemize}
\item Different domains require/allow different methods
\end{itemize}
\end{frame}

\begin{frame}
  \frametitle{Constraint Programming is Different}
  \begin{itemize}
  \item Declarative Programming
    \begin{itemize}
    \item Concentrate on what you want
      \item Not how to get there
      \item Program != Algorithm
      \item Program = Model
    \end{itemize}
    \item Applied to Combinatorial Problems
      \begin{itemize}
        \item No complete polynomial algorithms known (exist?)
        \item CP less ad-hoc than heuristics
        \item Models can evolve
  \end{itemize}
  \end{itemize}
  \end{frame}
    
\begin{frame}
  \frametitle{A Subtractive Process}
  \begin{textblock}{4}(8,-3)
    \includegraphics[width=4cm]{../introduction/images/stages}
  \end{textblock}
  \vfill
  \begin{quote}
    ``Oh, bosh, as Mr. Ruskin says. Sculpture, per se, is the simplest thing in the world. All you have to do is to take a big chunk of marble and a hammer and chisel, make up your mind what you are about to create and chip off all the marble you don't want.''-Paris Gaulois.
  \end{quote}
  
  {\tiny Source: \url{https://quoteinvestigator.com/2014/06/22/chip-away/}}
\end{frame}

\begin{frame}
\frametitle{Basic Process}
\begin{center}
\begin{tikzpicture}
\node[shape=rectangle,fill=pantone127-4] (problem) at (2,5) {Problem};
\node (human) at (2,4) {Human};
\node[shape=rectangle,fill=pantone127-4] (model) at (2,3) {Model};
\node (solver) at (2,2) {Constraint Solver/Search};
\node[shape=rectangle,fill=pantone157-8] (solution) at (2,1) {Solution};
\draw[-] (problem) -- (human);
\draw[->] (human) -- (model);
\draw[-] (model) -- (solver);
\draw[->] (solver) -- (solution);
\end{tikzpicture}
\end{center}
\end{frame}

\begin{frame}
\frametitle{More Realistic}
\begin{center}
\begin{tikzpicture}[xscale=4.0]
\node[shape=rectangle,fill=pantone127-4] (problem) at (2,5) {Problem};
\node (human) at (2,4) {Human};
\node[shape=rectangle,fill=pantone127-4] (model) at (2,3) {Model};
\node (solver) at (2,2) {Constraint Solver/Search};
\node[shape=rectangle,fill=pantone24-8] (hangs) at (1,1) {Hangs};
\node[shape=rectangle,fill=pantone157-8] (solution) at (2,1) {Solution};
\node[shape=rectangle,fill=pantone24-8] (wrong) at (3,1) {Wrong Solution};
\draw[-] (problem) -- (human);
\draw[->] (human) -- (model);
\draw[-] (model) -- (solver);
\draw[->] (solver) -- (hangs);
\draw[->] (solver) -- (solution);
\draw[->] (solver) -- (wrong);
\draw[->] (hangs) -- (1,4) -- (human);
\draw[->] (wrong) -- (3,4) -- (human);
\end{tikzpicture}
\end{center}
\end{frame}

\begin{frame}
\frametitle{Dual Role of Model}
\begin{itemize}
\item Allows Human to Express Problem
\begin{itemize}
\item Close to Problem Domain
\item Constraints as Abstractions
\end{itemize}
\item Allows Solver to Execute
\begin{itemize}
\item Variables as Communication Mechanism
\item Constraints as Algorithms
\end{itemize}
\end{itemize}
\end{frame}

\begin{frame}
\frametitle{Modelling Frameworks}
\begin{itemize}
\item \href{https://www.minizinc.org/}{MiniZinc} (NICTA, Monash University, Australia)
\item \href{https://github.com/eomahony/Numberjack}{NumberJack} (Insight, Ireland) 
\item \href{https://www-users.york.ac.uk/peter.nightingale/savilerow//}{EssencePrime/SavilleRow} (UK) 
\item \href{https://cpmpy.readthedocs.io/en/latest/}{CPMpy} (KU Leuven)
\item Allow use of multiple back-end solvers
\item Compile model into variants for each solver
\item A priori solver independent model(CP, MIP, SAT)
\end{itemize}
\end{frame} 

\begin{frame}
\frametitle{Framework Process}
\begin{center}
\begin{tikzpicture}[xscale=2]
\node[shape=rectangle,fill=pantone127-4] (problem) at (2.5,5) {Problem};
\node (human) at (2.5,4) {Human};
\node[shape=rectangle,fill=pantone127-4] (model) at (2.5,3) {Model};
\node (compile) at (2.5,2) {Compile/Reformulate};
\node (solver1) at (1,1) {CP};
\node (solver2) at (2,1) {MIP};
\node (solver3) at (3,1) {SAT};
\node (solver4) at (4,1) {Other};
\node[shape=rectangle,fill=pantone157-8] (solution1) at (1,0) {Solution};
\node[shape=rectangle,fill=pantone157-8] (solution2) at (2,0) {Solution};
\node[shape=rectangle,fill=pantone157-8] (solution3) at (3,0) {Solution};
\node[shape=rectangle,fill=pantone157-8] (solution4) at (4,0) {Solution};
\draw[-] (problem) -- (human);
\draw[->] (human) -- (model);
\draw[-] (model) -- (compile);
\draw[-] (compile) -- (solver1);
\draw[-] (compile) -- (solver2);
\draw[-] (compile) -- (solver3);
\draw[-] (compile) -- (solver4);
\draw[->] (solver1) -- (solution1);
\draw[->] (solver2) -- (solution2);
\draw[->] (solver3) -- (solution3);
\draw[->] (solver4) -- (solution4);
\end{tikzpicture}
\end{center}
\end{frame}

% \begin{frame}
  % \frametitle{Do It Now!}
  % \begin{itemize}
  % \item Download and install Minizinc
  % \item \url{https://www.minizinc.org/}
  % \end{itemize}
% \end{frame}


\begin{frame}
\frametitle{Why use Puzzles as Examples?}
\begin{itemize}
\item Easy to understand the problem
\item Solvable by hand without specialized knowledge
\item Possible to compare automated to manual solving process
\end{itemize}

\begin{quote}
The puzzle, though inanimate, is presented as a solvable problem without lasting negative consequences, a very low-risk low-reward situation. By being a puzzle, the object is attempting to convince the user that it must be completed.
\end{quote}
{\tiny Source: \href{https://www.everydayrhetoric.com/2020/03/the-rhetoric-of-puzzles/}{Every Day Rhetoric}}
\end{frame}


\mode<all>{
\part{Basic Constraint Propagation}
}
\begin{frame}
\partpage
\end{frame}
\mode<all>{
\input{../sendmore/body}
}

\mode<all>{
\part{Global Constraints}
}
\begin{frame}
\partpage
\end{frame}
\mode<all>{
\input{../sudoku/body}
}

\clearpage
\mode<all>{
\part{Customizing Search}
}
\begin{frame}
\partpage
\end{frame}
\mode<all>{
\input{../nqueen/body}
}


%% \clearpage
%% \mode<all>{
%%   \part{Problem Decomposition}
%% }
%% \begin{frame}
%%   \partpage
%% \end{frame}
%% \mode<all>{
%%   \input{../party/body}
%%   }

%% \clearpage
%% \mode<all>{
%%   \part{More Global Constraints}
%% }
%% \begin{frame}
%%   \partpage
%% \end{frame}
%% \mode<all>{
%%   \input{../car/body}
%%   }

%% \clearpage
%% \mode<all>{
%%   \part{Methodology}
%% }
%% \begin{frame}
%%   \partpage
%% \end{frame}
%% \mode<all>{
%%   \input{../methodology/body}
%%   }

\clearpage
\mode<all>{
\part{What is missing?}
}
\begin{frame}
\partpage
\end{frame}

\begin{frame}
\frametitle{Many Specialized Topics}
\begin{itemize}
\item How to design efficient core engine
\item Hybrids with LP/MIP tools
\item Hybrids with SAT
\item Symmetry breaking
\item Use of MDD/BDD to encode sets of solutions
\item High level modelling tools
\item Debugging/visualization
\end{itemize}
\end{frame}

\begin{frame}
\frametitle{Reformulation}
\begin{itemize}
\item Just because the user has modelled it this way, it doesn't mean we have to solve it that way
\begin{itemize}
\item Replace some constraint(s) by other, equivalent constraints
\item Because we don't have that constraint in our system
\item For performance
\end{itemize}
\end{itemize}
\end{frame}

\begin{frame}
\frametitle{Learning}
\begin{itemize}
\item While solving the problem we can learn how to strengthen the model/search 
\begin{itemize}
\item Understand which constraints/method contribute to propagation and change schedule
\item Learn no-good constraints by explaining failure
\item Adapt search strategy based on search experience
\end{itemize}
\end{itemize}
\end{frame}

%% \begin{frame}
%% \frametitle{Refined Process}
%% \begin{center}
%% \begin{tikzpicture}
%% \node[shape=rectangle,fill=pantone127-4] (problem) at (3,7) {Problem};
%% \node (human) at (3,6) {Human};
%% \node[shape=rectangle,fill=pantone127-4] (model) at (3,5) {User Model};
%% \node (reformulation) at (3,4) {Reformulation};
%% \node[shape=rectangle,fill=pantone127-4] (implementation) at (3,3) {Implementation Model};
%% \node (solver) at (3,2) {Constraint Solver/Search};
%% \node[shape=rectangle,fill=pantone157-8] (solution) at (3,1) {Solution};
%% \draw[-] (problem) -- (human);
%% \draw[->] (human) -- (model);
%% \draw[-] (model) -- (reformulation);
%% \draw[->] (reformulation) -- (implementation);
%% \draw[-] (implementation) -- (solver);
%% \draw[->] (solver) -- (0,2) -- (0,3) -- (implementation);
%% \draw[->] (solver) -- (solution);
%% \end{tikzpicture}
%% \end{center}
%% \end{frame}

%\begin{frame}
%\frametitle{What is CP actually used for?}
%\url{http://hsimonis.wordpress.com}
%\includegraphics[width=10cm]{../introduction/images/blog}
%\end{frame}

%% \begin{frame}
%% \frametitle{Exercises}
%% \begin{itemize}
%% \item Install Constraint Solver of your Choice
%% \item MiniZinc (MiningZinc) will be used in later sessions
%% \item Run Sendmoremoney, Sudoku, Queens examples
%% \item Can you choose which propagator to use for alldifferent?
%% \item Can you use different search strategies?
%% \end{itemize}
%% \end{frame}

\begin{frame}
\frametitle{More Learning Resources}
\begin{itemize}
\item Survey of Methods, Resources, and Formats for Teaching Constraint Programming
\begin{itemize}
\item by Tejas Santanam, Helmut Simonis
\item \url{https://doi.org/10.48550/arXiv.2403.12717}
\item Based on survey of community for WTCP 2023
\item \url{https://hsimonis.github.io/WTCP2023/}
\end{itemize}
\end{itemize}
\end{frame}


%%   }

\clearpage
\mode<all>{
\part{What is missing?}
}
\begin{frame}
\partpage
\end{frame}

\begin{frame}
\frametitle{Many Specialized Topics}
\begin{itemize}
\item How to design efficient core engine
\item Hybrids with LP/MIP tools
\item Hybrids with SAT
\item Symmetry breaking
\item Use of MDD/BDD to encode sets of solutions
\item High level modelling tools
\item Debugging/visualization
\end{itemize}
\end{frame}

\begin{frame}
\frametitle{Reformulation}
\begin{itemize}
\item Just because the user has modelled it this way, it doesn't mean we have to solve it that way
\begin{itemize}
\item Replace some constraint(s) by other, equivalent constraints
\item Because we don't have that constraint in our system
\item For performance
\end{itemize}
\end{itemize}
\end{frame}

\begin{frame}
\frametitle{Learning}
\begin{itemize}
\item While solving the problem we can learn how to strengthen the model/search 
\begin{itemize}
\item Understand which constraints/method contribute to propagation and change schedule
\item Learn no-good constraints by explaining failure
\item Adapt search strategy based on search experience
\end{itemize}
\end{itemize}
\end{frame}

%% \begin{frame}
%% \frametitle{Refined Process}
%% \begin{center}
%% \begin{tikzpicture}
%% \node[shape=rectangle,fill=pantone127-4] (problem) at (3,7) {Problem};
%% \node (human) at (3,6) {Human};
%% \node[shape=rectangle,fill=pantone127-4] (model) at (3,5) {User Model};
%% \node (reformulation) at (3,4) {Reformulation};
%% \node[shape=rectangle,fill=pantone127-4] (implementation) at (3,3) {Implementation Model};
%% \node (solver) at (3,2) {Constraint Solver/Search};
%% \node[shape=rectangle,fill=pantone157-8] (solution) at (3,1) {Solution};
%% \draw[-] (problem) -- (human);
%% \draw[->] (human) -- (model);
%% \draw[-] (model) -- (reformulation);
%% \draw[->] (reformulation) -- (implementation);
%% \draw[-] (implementation) -- (solver);
%% \draw[->] (solver) -- (0,2) -- (0,3) -- (implementation);
%% \draw[->] (solver) -- (solution);
%% \end{tikzpicture}
%% \end{center}
%% \end{frame}

%\begin{frame}
%\frametitle{What is CP actually used for?}
%\url{http://hsimonis.wordpress.com}
%\includegraphics[width=10cm]{../introduction/images/blog}
%\end{frame}

%% \begin{frame}
%% \frametitle{Exercises}
%% \begin{itemize}
%% \item Install Constraint Solver of your Choice
%% \item MiniZinc (MiningZinc) will be used in later sessions
%% \item Run Sendmoremoney, Sudoku, Queens examples
%% \item Can you choose which propagator to use for alldifferent?
%% \item Can you use different search strategies?
%% \end{itemize}
%% \end{frame}

\begin{frame}
\frametitle{More Learning Resources}
\begin{itemize}
\item Survey of Methods, Resources, and Formats for Teaching Constraint Programming
\begin{itemize}
\item by Tejas Santanam, Helmut Simonis
\item \url{https://doi.org/10.48550/arXiv.2403.12717}
\item Based on survey of community for WTCP 2023
\item \url{https://hsimonis.github.io/WTCP2023/}
\end{itemize}
\end{itemize}
\end{frame}


}


%% \clearpage
%% \mode<all>{
%%   \part{Problem Decomposition}
%% }
%% \begin{frame}
%%   \partpage
%% \end{frame}
%% \mode<all>{
%%   \begin{frame}
\frametitle{Objectives}
\begin{itemize}
\item Overview of Core Constraint Programming
\item Three Main Concepts
\begin{itemize}
\item Constraint Propagation
\item Global Constraints
\item Customizing Search
\end{itemize}
%\item Get Some Experience with MiniZinc
\item Topics will be treated in more detail in later parts of the school
\item Based on Examples, not Formal Description
\end{itemize}
\end{frame}

\begin{frame}
\frametitle{Outline}
\begin{itemize}
\item Why Constraint Programming?
\item Constraint Propagation
\item Global Constraints
\item Customizing Search
%\item Choices in Modelling
%\item Decomposition
%\item More Global Constraints
%\item Methodology
%\item What is missing?
\end{itemize}
\end{frame}

% \begin{frame}
  % \frametitle{Using MiniZinc IDE}
  % \begin{itemize}
  % \item Developed in the Australian NICTA project
  % \item Maintained by Monash University
  % \item Modelling tool with multiple back-end solvers
  % \item Available from \url{https://www.minizinc.org/}
  % \end{itemize}
% \end{frame}


% \begin{frame}
% \frametitle{Examples in ECLiPSe}
% \begin{itemize}
% \item Open sourced constraint programming language
% \item Development goes back to 1985
% \item ECRC, ICL, IC-Parc, PTL, Cisco
% \item \url{https://eclipseclp.org/}
% \item Specialities
% \begin{itemize}
% \item Develop new solvers for specific domains
% \item Integration with MIP
% \end{itemize}
% \item Not included in bundled MiniZinc IDE
% \item Specialized visualization tools used here
  % \begin{itemize}
  % \item CP-Viz, Simonis et al. 2010
    % \end{itemize}
% \end{itemize}
% \end{frame}

\begin{frame}
\frametitle{Tutorial Based on ECLiPSe ELearning Course}
\begin{itemize}
\item Self-study course in constraint programming
\item Supported by Cisco Systems and Silicon Valley Community Foundation
\item Multi-media format, video lectures, slides, handout etc
\item {\small \url{https://eclipseclp.org/ELearning/index.html}}
\end{itemize}
\end{frame}

\begin{frame}
\frametitle{Also Part of CRT-AI Constraint Week}
\begin{itemize}
\item Annual one week course on CP and Optimization in Ireland
\item Part of national training program for PhD students in AI
\item \url{https://www.crt-ai.ie/}
\end{itemize}
\end{frame}



\begin{frame}
\frametitle{Constraint Programming - in a nutshell}
\begin{itemize}
\item Declarative description of problems with
\begin{itemize}
\item {\em Variables} which range over (finite) sets of values
\item {\em Constraints} over subsets of variables which restrict possible value combinations
\item A {\em solution} is a value assignment which satisfies all constraints
\end{itemize}

\item Constraint propagation/reasoning
\begin{itemize}
\item Removing inconsistent values for variables
\item Detect failure if constraint can not be satisfied
\item Interaction of constraints via shared variables
\item Incomplete
\end{itemize}

\item Search
\begin{itemize}
\item User controlled assignment of values to variables
\item Each step triggers constraint propagation 
\end{itemize}
\item Different domains require/allow different methods
\end{itemize}
\end{frame}

\begin{frame}
  \frametitle{Constraint Programming is Different}
  \begin{itemize}
  \item Declarative Programming
    \begin{itemize}
    \item Concentrate on what you want
      \item Not how to get there
      \item Program != Algorithm
      \item Program = Model
    \end{itemize}
    \item Applied to Combinatorial Problems
      \begin{itemize}
        \item No complete polynomial algorithms known (exist?)
        \item CP less ad-hoc than heuristics
        \item Models can evolve
  \end{itemize}
  \end{itemize}
  \end{frame}
    
\begin{frame}
  \frametitle{A Subtractive Process}
  \begin{textblock}{4}(8,-3)
    \includegraphics[width=4cm]{../introduction/images/stages}
  \end{textblock}
  \vfill
  \begin{quote}
    ``Oh, bosh, as Mr. Ruskin says. Sculpture, per se, is the simplest thing in the world. All you have to do is to take a big chunk of marble and a hammer and chisel, make up your mind what you are about to create and chip off all the marble you don't want.''-Paris Gaulois.
  \end{quote}
  
  {\tiny Source: \url{https://quoteinvestigator.com/2014/06/22/chip-away/}}
\end{frame}

\begin{frame}
\frametitle{Basic Process}
\begin{center}
\begin{tikzpicture}
\node[shape=rectangle,fill=pantone127-4] (problem) at (2,5) {Problem};
\node (human) at (2,4) {Human};
\node[shape=rectangle,fill=pantone127-4] (model) at (2,3) {Model};
\node (solver) at (2,2) {Constraint Solver/Search};
\node[shape=rectangle,fill=pantone157-8] (solution) at (2,1) {Solution};
\draw[-] (problem) -- (human);
\draw[->] (human) -- (model);
\draw[-] (model) -- (solver);
\draw[->] (solver) -- (solution);
\end{tikzpicture}
\end{center}
\end{frame}

\begin{frame}
\frametitle{More Realistic}
\begin{center}
\begin{tikzpicture}[xscale=4.0]
\node[shape=rectangle,fill=pantone127-4] (problem) at (2,5) {Problem};
\node (human) at (2,4) {Human};
\node[shape=rectangle,fill=pantone127-4] (model) at (2,3) {Model};
\node (solver) at (2,2) {Constraint Solver/Search};
\node[shape=rectangle,fill=pantone24-8] (hangs) at (1,1) {Hangs};
\node[shape=rectangle,fill=pantone157-8] (solution) at (2,1) {Solution};
\node[shape=rectangle,fill=pantone24-8] (wrong) at (3,1) {Wrong Solution};
\draw[-] (problem) -- (human);
\draw[->] (human) -- (model);
\draw[-] (model) -- (solver);
\draw[->] (solver) -- (hangs);
\draw[->] (solver) -- (solution);
\draw[->] (solver) -- (wrong);
\draw[->] (hangs) -- (1,4) -- (human);
\draw[->] (wrong) -- (3,4) -- (human);
\end{tikzpicture}
\end{center}
\end{frame}

\begin{frame}
\frametitle{Dual Role of Model}
\begin{itemize}
\item Allows Human to Express Problem
\begin{itemize}
\item Close to Problem Domain
\item Constraints as Abstractions
\end{itemize}
\item Allows Solver to Execute
\begin{itemize}
\item Variables as Communication Mechanism
\item Constraints as Algorithms
\end{itemize}
\end{itemize}
\end{frame}

\begin{frame}
\frametitle{Modelling Frameworks}
\begin{itemize}
\item \href{https://www.minizinc.org/}{MiniZinc} (NICTA, Monash University, Australia)
\item \href{https://github.com/eomahony/Numberjack}{NumberJack} (Insight, Ireland) 
\item \href{https://www-users.york.ac.uk/peter.nightingale/savilerow//}{EssencePrime/SavilleRow} (UK) 
\item \href{https://cpmpy.readthedocs.io/en/latest/}{CPMpy} (KU Leuven)
\item Allow use of multiple back-end solvers
\item Compile model into variants for each solver
\item A priori solver independent model(CP, MIP, SAT)
\end{itemize}
\end{frame} 

\begin{frame}
\frametitle{Framework Process}
\begin{center}
\begin{tikzpicture}[xscale=2]
\node[shape=rectangle,fill=pantone127-4] (problem) at (2.5,5) {Problem};
\node (human) at (2.5,4) {Human};
\node[shape=rectangle,fill=pantone127-4] (model) at (2.5,3) {Model};
\node (compile) at (2.5,2) {Compile/Reformulate};
\node (solver1) at (1,1) {CP};
\node (solver2) at (2,1) {MIP};
\node (solver3) at (3,1) {SAT};
\node (solver4) at (4,1) {Other};
\node[shape=rectangle,fill=pantone157-8] (solution1) at (1,0) {Solution};
\node[shape=rectangle,fill=pantone157-8] (solution2) at (2,0) {Solution};
\node[shape=rectangle,fill=pantone157-8] (solution3) at (3,0) {Solution};
\node[shape=rectangle,fill=pantone157-8] (solution4) at (4,0) {Solution};
\draw[-] (problem) -- (human);
\draw[->] (human) -- (model);
\draw[-] (model) -- (compile);
\draw[-] (compile) -- (solver1);
\draw[-] (compile) -- (solver2);
\draw[-] (compile) -- (solver3);
\draw[-] (compile) -- (solver4);
\draw[->] (solver1) -- (solution1);
\draw[->] (solver2) -- (solution2);
\draw[->] (solver3) -- (solution3);
\draw[->] (solver4) -- (solution4);
\end{tikzpicture}
\end{center}
\end{frame}

% \begin{frame}
  % \frametitle{Do It Now!}
  % \begin{itemize}
  % \item Download and install Minizinc
  % \item \url{https://www.minizinc.org/}
  % \end{itemize}
% \end{frame}


\begin{frame}
\frametitle{Why use Puzzles as Examples?}
\begin{itemize}
\item Easy to understand the problem
\item Solvable by hand without specialized knowledge
\item Possible to compare automated to manual solving process
\end{itemize}

\begin{quote}
The puzzle, though inanimate, is presented as a solvable problem without lasting negative consequences, a very low-risk low-reward situation. By being a puzzle, the object is attempting to convince the user that it must be completed.
\end{quote}
{\tiny Source: \href{https://www.everydayrhetoric.com/2020/03/the-rhetoric-of-puzzles/}{Every Day Rhetoric}}
\end{frame}


\mode<all>{
\part{Basic Constraint Propagation}
}
\begin{frame}
\partpage
\end{frame}
\mode<all>{
\begin{frame}
\frametitle{Objectives}
\begin{itemize}
\item Overview of Core Constraint Programming
\item Three Main Concepts
\begin{itemize}
\item Constraint Propagation
\item Global Constraints
\item Customizing Search
\end{itemize}
%\item Get Some Experience with MiniZinc
\item Topics will be treated in more detail in later parts of the school
\item Based on Examples, not Formal Description
\end{itemize}
\end{frame}

\begin{frame}
\frametitle{Outline}
\begin{itemize}
\item Why Constraint Programming?
\item Constraint Propagation
\item Global Constraints
\item Customizing Search
%\item Choices in Modelling
%\item Decomposition
%\item More Global Constraints
%\item Methodology
%\item What is missing?
\end{itemize}
\end{frame}

% \begin{frame}
  % \frametitle{Using MiniZinc IDE}
  % \begin{itemize}
  % \item Developed in the Australian NICTA project
  % \item Maintained by Monash University
  % \item Modelling tool with multiple back-end solvers
  % \item Available from \url{https://www.minizinc.org/}
  % \end{itemize}
% \end{frame}


% \begin{frame}
% \frametitle{Examples in ECLiPSe}
% \begin{itemize}
% \item Open sourced constraint programming language
% \item Development goes back to 1985
% \item ECRC, ICL, IC-Parc, PTL, Cisco
% \item \url{https://eclipseclp.org/}
% \item Specialities
% \begin{itemize}
% \item Develop new solvers for specific domains
% \item Integration with MIP
% \end{itemize}
% \item Not included in bundled MiniZinc IDE
% \item Specialized visualization tools used here
  % \begin{itemize}
  % \item CP-Viz, Simonis et al. 2010
    % \end{itemize}
% \end{itemize}
% \end{frame}

\begin{frame}
\frametitle{Tutorial Based on ECLiPSe ELearning Course}
\begin{itemize}
\item Self-study course in constraint programming
\item Supported by Cisco Systems and Silicon Valley Community Foundation
\item Multi-media format, video lectures, slides, handout etc
\item {\small \url{https://eclipseclp.org/ELearning/index.html}}
\end{itemize}
\end{frame}

\begin{frame}
\frametitle{Also Part of CRT-AI Constraint Week}
\begin{itemize}
\item Annual one week course on CP and Optimization in Ireland
\item Part of national training program for PhD students in AI
\item \url{https://www.crt-ai.ie/}
\end{itemize}
\end{frame}



\begin{frame}
\frametitle{Constraint Programming - in a nutshell}
\begin{itemize}
\item Declarative description of problems with
\begin{itemize}
\item {\em Variables} which range over (finite) sets of values
\item {\em Constraints} over subsets of variables which restrict possible value combinations
\item A {\em solution} is a value assignment which satisfies all constraints
\end{itemize}

\item Constraint propagation/reasoning
\begin{itemize}
\item Removing inconsistent values for variables
\item Detect failure if constraint can not be satisfied
\item Interaction of constraints via shared variables
\item Incomplete
\end{itemize}

\item Search
\begin{itemize}
\item User controlled assignment of values to variables
\item Each step triggers constraint propagation 
\end{itemize}
\item Different domains require/allow different methods
\end{itemize}
\end{frame}

\begin{frame}
  \frametitle{Constraint Programming is Different}
  \begin{itemize}
  \item Declarative Programming
    \begin{itemize}
    \item Concentrate on what you want
      \item Not how to get there
      \item Program != Algorithm
      \item Program = Model
    \end{itemize}
    \item Applied to Combinatorial Problems
      \begin{itemize}
        \item No complete polynomial algorithms known (exist?)
        \item CP less ad-hoc than heuristics
        \item Models can evolve
  \end{itemize}
  \end{itemize}
  \end{frame}
    
\begin{frame}
  \frametitle{A Subtractive Process}
  \begin{textblock}{4}(8,-3)
    \includegraphics[width=4cm]{../introduction/images/stages}
  \end{textblock}
  \vfill
  \begin{quote}
    ``Oh, bosh, as Mr. Ruskin says. Sculpture, per se, is the simplest thing in the world. All you have to do is to take a big chunk of marble and a hammer and chisel, make up your mind what you are about to create and chip off all the marble you don't want.''-Paris Gaulois.
  \end{quote}
  
  {\tiny Source: \url{https://quoteinvestigator.com/2014/06/22/chip-away/}}
\end{frame}

\begin{frame}
\frametitle{Basic Process}
\begin{center}
\begin{tikzpicture}
\node[shape=rectangle,fill=pantone127-4] (problem) at (2,5) {Problem};
\node (human) at (2,4) {Human};
\node[shape=rectangle,fill=pantone127-4] (model) at (2,3) {Model};
\node (solver) at (2,2) {Constraint Solver/Search};
\node[shape=rectangle,fill=pantone157-8] (solution) at (2,1) {Solution};
\draw[-] (problem) -- (human);
\draw[->] (human) -- (model);
\draw[-] (model) -- (solver);
\draw[->] (solver) -- (solution);
\end{tikzpicture}
\end{center}
\end{frame}

\begin{frame}
\frametitle{More Realistic}
\begin{center}
\begin{tikzpicture}[xscale=4.0]
\node[shape=rectangle,fill=pantone127-4] (problem) at (2,5) {Problem};
\node (human) at (2,4) {Human};
\node[shape=rectangle,fill=pantone127-4] (model) at (2,3) {Model};
\node (solver) at (2,2) {Constraint Solver/Search};
\node[shape=rectangle,fill=pantone24-8] (hangs) at (1,1) {Hangs};
\node[shape=rectangle,fill=pantone157-8] (solution) at (2,1) {Solution};
\node[shape=rectangle,fill=pantone24-8] (wrong) at (3,1) {Wrong Solution};
\draw[-] (problem) -- (human);
\draw[->] (human) -- (model);
\draw[-] (model) -- (solver);
\draw[->] (solver) -- (hangs);
\draw[->] (solver) -- (solution);
\draw[->] (solver) -- (wrong);
\draw[->] (hangs) -- (1,4) -- (human);
\draw[->] (wrong) -- (3,4) -- (human);
\end{tikzpicture}
\end{center}
\end{frame}

\begin{frame}
\frametitle{Dual Role of Model}
\begin{itemize}
\item Allows Human to Express Problem
\begin{itemize}
\item Close to Problem Domain
\item Constraints as Abstractions
\end{itemize}
\item Allows Solver to Execute
\begin{itemize}
\item Variables as Communication Mechanism
\item Constraints as Algorithms
\end{itemize}
\end{itemize}
\end{frame}

\begin{frame}
\frametitle{Modelling Frameworks}
\begin{itemize}
\item \href{https://www.minizinc.org/}{MiniZinc} (NICTA, Monash University, Australia)
\item \href{https://github.com/eomahony/Numberjack}{NumberJack} (Insight, Ireland) 
\item \href{https://www-users.york.ac.uk/peter.nightingale/savilerow//}{EssencePrime/SavilleRow} (UK) 
\item \href{https://cpmpy.readthedocs.io/en/latest/}{CPMpy} (KU Leuven)
\item Allow use of multiple back-end solvers
\item Compile model into variants for each solver
\item A priori solver independent model(CP, MIP, SAT)
\end{itemize}
\end{frame} 

\begin{frame}
\frametitle{Framework Process}
\begin{center}
\begin{tikzpicture}[xscale=2]
\node[shape=rectangle,fill=pantone127-4] (problem) at (2.5,5) {Problem};
\node (human) at (2.5,4) {Human};
\node[shape=rectangle,fill=pantone127-4] (model) at (2.5,3) {Model};
\node (compile) at (2.5,2) {Compile/Reformulate};
\node (solver1) at (1,1) {CP};
\node (solver2) at (2,1) {MIP};
\node (solver3) at (3,1) {SAT};
\node (solver4) at (4,1) {Other};
\node[shape=rectangle,fill=pantone157-8] (solution1) at (1,0) {Solution};
\node[shape=rectangle,fill=pantone157-8] (solution2) at (2,0) {Solution};
\node[shape=rectangle,fill=pantone157-8] (solution3) at (3,0) {Solution};
\node[shape=rectangle,fill=pantone157-8] (solution4) at (4,0) {Solution};
\draw[-] (problem) -- (human);
\draw[->] (human) -- (model);
\draw[-] (model) -- (compile);
\draw[-] (compile) -- (solver1);
\draw[-] (compile) -- (solver2);
\draw[-] (compile) -- (solver3);
\draw[-] (compile) -- (solver4);
\draw[->] (solver1) -- (solution1);
\draw[->] (solver2) -- (solution2);
\draw[->] (solver3) -- (solution3);
\draw[->] (solver4) -- (solution4);
\end{tikzpicture}
\end{center}
\end{frame}

% \begin{frame}
  % \frametitle{Do It Now!}
  % \begin{itemize}
  % \item Download and install Minizinc
  % \item \url{https://www.minizinc.org/}
  % \end{itemize}
% \end{frame}


\begin{frame}
\frametitle{Why use Puzzles as Examples?}
\begin{itemize}
\item Easy to understand the problem
\item Solvable by hand without specialized knowledge
\item Possible to compare automated to manual solving process
\end{itemize}

\begin{quote}
The puzzle, though inanimate, is presented as a solvable problem without lasting negative consequences, a very low-risk low-reward situation. By being a puzzle, the object is attempting to convince the user that it must be completed.
\end{quote}
{\tiny Source: \href{https://www.everydayrhetoric.com/2020/03/the-rhetoric-of-puzzles/}{Every Day Rhetoric}}
\end{frame}


\mode<all>{
\part{Basic Constraint Propagation}
}
\begin{frame}
\partpage
\end{frame}
\mode<all>{
\input{../sendmore/body}
}

\mode<all>{
\part{Global Constraints}
}
\begin{frame}
\partpage
\end{frame}
\mode<all>{
\input{../sudoku/body}
}

\clearpage
\mode<all>{
\part{Customizing Search}
}
\begin{frame}
\partpage
\end{frame}
\mode<all>{
\input{../nqueen/body}
}


%% \clearpage
%% \mode<all>{
%%   \part{Problem Decomposition}
%% }
%% \begin{frame}
%%   \partpage
%% \end{frame}
%% \mode<all>{
%%   \input{../party/body}
%%   }

%% \clearpage
%% \mode<all>{
%%   \part{More Global Constraints}
%% }
%% \begin{frame}
%%   \partpage
%% \end{frame}
%% \mode<all>{
%%   \input{../car/body}
%%   }

%% \clearpage
%% \mode<all>{
%%   \part{Methodology}
%% }
%% \begin{frame}
%%   \partpage
%% \end{frame}
%% \mode<all>{
%%   \input{../methodology/body}
%%   }

\clearpage
\mode<all>{
\part{What is missing?}
}
\begin{frame}
\partpage
\end{frame}

\begin{frame}
\frametitle{Many Specialized Topics}
\begin{itemize}
\item How to design efficient core engine
\item Hybrids with LP/MIP tools
\item Hybrids with SAT
\item Symmetry breaking
\item Use of MDD/BDD to encode sets of solutions
\item High level modelling tools
\item Debugging/visualization
\end{itemize}
\end{frame}

\begin{frame}
\frametitle{Reformulation}
\begin{itemize}
\item Just because the user has modelled it this way, it doesn't mean we have to solve it that way
\begin{itemize}
\item Replace some constraint(s) by other, equivalent constraints
\item Because we don't have that constraint in our system
\item For performance
\end{itemize}
\end{itemize}
\end{frame}

\begin{frame}
\frametitle{Learning}
\begin{itemize}
\item While solving the problem we can learn how to strengthen the model/search 
\begin{itemize}
\item Understand which constraints/method contribute to propagation and change schedule
\item Learn no-good constraints by explaining failure
\item Adapt search strategy based on search experience
\end{itemize}
\end{itemize}
\end{frame}

%% \begin{frame}
%% \frametitle{Refined Process}
%% \begin{center}
%% \begin{tikzpicture}
%% \node[shape=rectangle,fill=pantone127-4] (problem) at (3,7) {Problem};
%% \node (human) at (3,6) {Human};
%% \node[shape=rectangle,fill=pantone127-4] (model) at (3,5) {User Model};
%% \node (reformulation) at (3,4) {Reformulation};
%% \node[shape=rectangle,fill=pantone127-4] (implementation) at (3,3) {Implementation Model};
%% \node (solver) at (3,2) {Constraint Solver/Search};
%% \node[shape=rectangle,fill=pantone157-8] (solution) at (3,1) {Solution};
%% \draw[-] (problem) -- (human);
%% \draw[->] (human) -- (model);
%% \draw[-] (model) -- (reformulation);
%% \draw[->] (reformulation) -- (implementation);
%% \draw[-] (implementation) -- (solver);
%% \draw[->] (solver) -- (0,2) -- (0,3) -- (implementation);
%% \draw[->] (solver) -- (solution);
%% \end{tikzpicture}
%% \end{center}
%% \end{frame}

%\begin{frame}
%\frametitle{What is CP actually used for?}
%\url{http://hsimonis.wordpress.com}
%\includegraphics[width=10cm]{../introduction/images/blog}
%\end{frame}

%% \begin{frame}
%% \frametitle{Exercises}
%% \begin{itemize}
%% \item Install Constraint Solver of your Choice
%% \item MiniZinc (MiningZinc) will be used in later sessions
%% \item Run Sendmoremoney, Sudoku, Queens examples
%% \item Can you choose which propagator to use for alldifferent?
%% \item Can you use different search strategies?
%% \end{itemize}
%% \end{frame}

\begin{frame}
\frametitle{More Learning Resources}
\begin{itemize}
\item Survey of Methods, Resources, and Formats for Teaching Constraint Programming
\begin{itemize}
\item by Tejas Santanam, Helmut Simonis
\item \url{https://doi.org/10.48550/arXiv.2403.12717}
\item Based on survey of community for WTCP 2023
\item \url{https://hsimonis.github.io/WTCP2023/}
\end{itemize}
\end{itemize}
\end{frame}


}

\mode<all>{
\part{Global Constraints}
}
\begin{frame}
\partpage
\end{frame}
\mode<all>{
\begin{frame}
\frametitle{Objectives}
\begin{itemize}
\item Overview of Core Constraint Programming
\item Three Main Concepts
\begin{itemize}
\item Constraint Propagation
\item Global Constraints
\item Customizing Search
\end{itemize}
%\item Get Some Experience with MiniZinc
\item Topics will be treated in more detail in later parts of the school
\item Based on Examples, not Formal Description
\end{itemize}
\end{frame}

\begin{frame}
\frametitle{Outline}
\begin{itemize}
\item Why Constraint Programming?
\item Constraint Propagation
\item Global Constraints
\item Customizing Search
%\item Choices in Modelling
%\item Decomposition
%\item More Global Constraints
%\item Methodology
%\item What is missing?
\end{itemize}
\end{frame}

% \begin{frame}
  % \frametitle{Using MiniZinc IDE}
  % \begin{itemize}
  % \item Developed in the Australian NICTA project
  % \item Maintained by Monash University
  % \item Modelling tool with multiple back-end solvers
  % \item Available from \url{https://www.minizinc.org/}
  % \end{itemize}
% \end{frame}


% \begin{frame}
% \frametitle{Examples in ECLiPSe}
% \begin{itemize}
% \item Open sourced constraint programming language
% \item Development goes back to 1985
% \item ECRC, ICL, IC-Parc, PTL, Cisco
% \item \url{https://eclipseclp.org/}
% \item Specialities
% \begin{itemize}
% \item Develop new solvers for specific domains
% \item Integration with MIP
% \end{itemize}
% \item Not included in bundled MiniZinc IDE
% \item Specialized visualization tools used here
  % \begin{itemize}
  % \item CP-Viz, Simonis et al. 2010
    % \end{itemize}
% \end{itemize}
% \end{frame}

\begin{frame}
\frametitle{Tutorial Based on ECLiPSe ELearning Course}
\begin{itemize}
\item Self-study course in constraint programming
\item Supported by Cisco Systems and Silicon Valley Community Foundation
\item Multi-media format, video lectures, slides, handout etc
\item {\small \url{https://eclipseclp.org/ELearning/index.html}}
\end{itemize}
\end{frame}

\begin{frame}
\frametitle{Also Part of CRT-AI Constraint Week}
\begin{itemize}
\item Annual one week course on CP and Optimization in Ireland
\item Part of national training program for PhD students in AI
\item \url{https://www.crt-ai.ie/}
\end{itemize}
\end{frame}



\begin{frame}
\frametitle{Constraint Programming - in a nutshell}
\begin{itemize}
\item Declarative description of problems with
\begin{itemize}
\item {\em Variables} which range over (finite) sets of values
\item {\em Constraints} over subsets of variables which restrict possible value combinations
\item A {\em solution} is a value assignment which satisfies all constraints
\end{itemize}

\item Constraint propagation/reasoning
\begin{itemize}
\item Removing inconsistent values for variables
\item Detect failure if constraint can not be satisfied
\item Interaction of constraints via shared variables
\item Incomplete
\end{itemize}

\item Search
\begin{itemize}
\item User controlled assignment of values to variables
\item Each step triggers constraint propagation 
\end{itemize}
\item Different domains require/allow different methods
\end{itemize}
\end{frame}

\begin{frame}
  \frametitle{Constraint Programming is Different}
  \begin{itemize}
  \item Declarative Programming
    \begin{itemize}
    \item Concentrate on what you want
      \item Not how to get there
      \item Program != Algorithm
      \item Program = Model
    \end{itemize}
    \item Applied to Combinatorial Problems
      \begin{itemize}
        \item No complete polynomial algorithms known (exist?)
        \item CP less ad-hoc than heuristics
        \item Models can evolve
  \end{itemize}
  \end{itemize}
  \end{frame}
    
\begin{frame}
  \frametitle{A Subtractive Process}
  \begin{textblock}{4}(8,-3)
    \includegraphics[width=4cm]{../introduction/images/stages}
  \end{textblock}
  \vfill
  \begin{quote}
    ``Oh, bosh, as Mr. Ruskin says. Sculpture, per se, is the simplest thing in the world. All you have to do is to take a big chunk of marble and a hammer and chisel, make up your mind what you are about to create and chip off all the marble you don't want.''-Paris Gaulois.
  \end{quote}
  
  {\tiny Source: \url{https://quoteinvestigator.com/2014/06/22/chip-away/}}
\end{frame}

\begin{frame}
\frametitle{Basic Process}
\begin{center}
\begin{tikzpicture}
\node[shape=rectangle,fill=pantone127-4] (problem) at (2,5) {Problem};
\node (human) at (2,4) {Human};
\node[shape=rectangle,fill=pantone127-4] (model) at (2,3) {Model};
\node (solver) at (2,2) {Constraint Solver/Search};
\node[shape=rectangle,fill=pantone157-8] (solution) at (2,1) {Solution};
\draw[-] (problem) -- (human);
\draw[->] (human) -- (model);
\draw[-] (model) -- (solver);
\draw[->] (solver) -- (solution);
\end{tikzpicture}
\end{center}
\end{frame}

\begin{frame}
\frametitle{More Realistic}
\begin{center}
\begin{tikzpicture}[xscale=4.0]
\node[shape=rectangle,fill=pantone127-4] (problem) at (2,5) {Problem};
\node (human) at (2,4) {Human};
\node[shape=rectangle,fill=pantone127-4] (model) at (2,3) {Model};
\node (solver) at (2,2) {Constraint Solver/Search};
\node[shape=rectangle,fill=pantone24-8] (hangs) at (1,1) {Hangs};
\node[shape=rectangle,fill=pantone157-8] (solution) at (2,1) {Solution};
\node[shape=rectangle,fill=pantone24-8] (wrong) at (3,1) {Wrong Solution};
\draw[-] (problem) -- (human);
\draw[->] (human) -- (model);
\draw[-] (model) -- (solver);
\draw[->] (solver) -- (hangs);
\draw[->] (solver) -- (solution);
\draw[->] (solver) -- (wrong);
\draw[->] (hangs) -- (1,4) -- (human);
\draw[->] (wrong) -- (3,4) -- (human);
\end{tikzpicture}
\end{center}
\end{frame}

\begin{frame}
\frametitle{Dual Role of Model}
\begin{itemize}
\item Allows Human to Express Problem
\begin{itemize}
\item Close to Problem Domain
\item Constraints as Abstractions
\end{itemize}
\item Allows Solver to Execute
\begin{itemize}
\item Variables as Communication Mechanism
\item Constraints as Algorithms
\end{itemize}
\end{itemize}
\end{frame}

\begin{frame}
\frametitle{Modelling Frameworks}
\begin{itemize}
\item \href{https://www.minizinc.org/}{MiniZinc} (NICTA, Monash University, Australia)
\item \href{https://github.com/eomahony/Numberjack}{NumberJack} (Insight, Ireland) 
\item \href{https://www-users.york.ac.uk/peter.nightingale/savilerow//}{EssencePrime/SavilleRow} (UK) 
\item \href{https://cpmpy.readthedocs.io/en/latest/}{CPMpy} (KU Leuven)
\item Allow use of multiple back-end solvers
\item Compile model into variants for each solver
\item A priori solver independent model(CP, MIP, SAT)
\end{itemize}
\end{frame} 

\begin{frame}
\frametitle{Framework Process}
\begin{center}
\begin{tikzpicture}[xscale=2]
\node[shape=rectangle,fill=pantone127-4] (problem) at (2.5,5) {Problem};
\node (human) at (2.5,4) {Human};
\node[shape=rectangle,fill=pantone127-4] (model) at (2.5,3) {Model};
\node (compile) at (2.5,2) {Compile/Reformulate};
\node (solver1) at (1,1) {CP};
\node (solver2) at (2,1) {MIP};
\node (solver3) at (3,1) {SAT};
\node (solver4) at (4,1) {Other};
\node[shape=rectangle,fill=pantone157-8] (solution1) at (1,0) {Solution};
\node[shape=rectangle,fill=pantone157-8] (solution2) at (2,0) {Solution};
\node[shape=rectangle,fill=pantone157-8] (solution3) at (3,0) {Solution};
\node[shape=rectangle,fill=pantone157-8] (solution4) at (4,0) {Solution};
\draw[-] (problem) -- (human);
\draw[->] (human) -- (model);
\draw[-] (model) -- (compile);
\draw[-] (compile) -- (solver1);
\draw[-] (compile) -- (solver2);
\draw[-] (compile) -- (solver3);
\draw[-] (compile) -- (solver4);
\draw[->] (solver1) -- (solution1);
\draw[->] (solver2) -- (solution2);
\draw[->] (solver3) -- (solution3);
\draw[->] (solver4) -- (solution4);
\end{tikzpicture}
\end{center}
\end{frame}

% \begin{frame}
  % \frametitle{Do It Now!}
  % \begin{itemize}
  % \item Download and install Minizinc
  % \item \url{https://www.minizinc.org/}
  % \end{itemize}
% \end{frame}


\begin{frame}
\frametitle{Why use Puzzles as Examples?}
\begin{itemize}
\item Easy to understand the problem
\item Solvable by hand without specialized knowledge
\item Possible to compare automated to manual solving process
\end{itemize}

\begin{quote}
The puzzle, though inanimate, is presented as a solvable problem without lasting negative consequences, a very low-risk low-reward situation. By being a puzzle, the object is attempting to convince the user that it must be completed.
\end{quote}
{\tiny Source: \href{https://www.everydayrhetoric.com/2020/03/the-rhetoric-of-puzzles/}{Every Day Rhetoric}}
\end{frame}


\mode<all>{
\part{Basic Constraint Propagation}
}
\begin{frame}
\partpage
\end{frame}
\mode<all>{
\input{../sendmore/body}
}

\mode<all>{
\part{Global Constraints}
}
\begin{frame}
\partpage
\end{frame}
\mode<all>{
\input{../sudoku/body}
}

\clearpage
\mode<all>{
\part{Customizing Search}
}
\begin{frame}
\partpage
\end{frame}
\mode<all>{
\input{../nqueen/body}
}


%% \clearpage
%% \mode<all>{
%%   \part{Problem Decomposition}
%% }
%% \begin{frame}
%%   \partpage
%% \end{frame}
%% \mode<all>{
%%   \input{../party/body}
%%   }

%% \clearpage
%% \mode<all>{
%%   \part{More Global Constraints}
%% }
%% \begin{frame}
%%   \partpage
%% \end{frame}
%% \mode<all>{
%%   \input{../car/body}
%%   }

%% \clearpage
%% \mode<all>{
%%   \part{Methodology}
%% }
%% \begin{frame}
%%   \partpage
%% \end{frame}
%% \mode<all>{
%%   \input{../methodology/body}
%%   }

\clearpage
\mode<all>{
\part{What is missing?}
}
\begin{frame}
\partpage
\end{frame}

\begin{frame}
\frametitle{Many Specialized Topics}
\begin{itemize}
\item How to design efficient core engine
\item Hybrids with LP/MIP tools
\item Hybrids with SAT
\item Symmetry breaking
\item Use of MDD/BDD to encode sets of solutions
\item High level modelling tools
\item Debugging/visualization
\end{itemize}
\end{frame}

\begin{frame}
\frametitle{Reformulation}
\begin{itemize}
\item Just because the user has modelled it this way, it doesn't mean we have to solve it that way
\begin{itemize}
\item Replace some constraint(s) by other, equivalent constraints
\item Because we don't have that constraint in our system
\item For performance
\end{itemize}
\end{itemize}
\end{frame}

\begin{frame}
\frametitle{Learning}
\begin{itemize}
\item While solving the problem we can learn how to strengthen the model/search 
\begin{itemize}
\item Understand which constraints/method contribute to propagation and change schedule
\item Learn no-good constraints by explaining failure
\item Adapt search strategy based on search experience
\end{itemize}
\end{itemize}
\end{frame}

%% \begin{frame}
%% \frametitle{Refined Process}
%% \begin{center}
%% \begin{tikzpicture}
%% \node[shape=rectangle,fill=pantone127-4] (problem) at (3,7) {Problem};
%% \node (human) at (3,6) {Human};
%% \node[shape=rectangle,fill=pantone127-4] (model) at (3,5) {User Model};
%% \node (reformulation) at (3,4) {Reformulation};
%% \node[shape=rectangle,fill=pantone127-4] (implementation) at (3,3) {Implementation Model};
%% \node (solver) at (3,2) {Constraint Solver/Search};
%% \node[shape=rectangle,fill=pantone157-8] (solution) at (3,1) {Solution};
%% \draw[-] (problem) -- (human);
%% \draw[->] (human) -- (model);
%% \draw[-] (model) -- (reformulation);
%% \draw[->] (reformulation) -- (implementation);
%% \draw[-] (implementation) -- (solver);
%% \draw[->] (solver) -- (0,2) -- (0,3) -- (implementation);
%% \draw[->] (solver) -- (solution);
%% \end{tikzpicture}
%% \end{center}
%% \end{frame}

%\begin{frame}
%\frametitle{What is CP actually used for?}
%\url{http://hsimonis.wordpress.com}
%\includegraphics[width=10cm]{../introduction/images/blog}
%\end{frame}

%% \begin{frame}
%% \frametitle{Exercises}
%% \begin{itemize}
%% \item Install Constraint Solver of your Choice
%% \item MiniZinc (MiningZinc) will be used in later sessions
%% \item Run Sendmoremoney, Sudoku, Queens examples
%% \item Can you choose which propagator to use for alldifferent?
%% \item Can you use different search strategies?
%% \end{itemize}
%% \end{frame}

\begin{frame}
\frametitle{More Learning Resources}
\begin{itemize}
\item Survey of Methods, Resources, and Formats for Teaching Constraint Programming
\begin{itemize}
\item by Tejas Santanam, Helmut Simonis
\item \url{https://doi.org/10.48550/arXiv.2403.12717}
\item Based on survey of community for WTCP 2023
\item \url{https://hsimonis.github.io/WTCP2023/}
\end{itemize}
\end{itemize}
\end{frame}


}

\clearpage
\mode<all>{
\part{Customizing Search}
}
\begin{frame}
\partpage
\end{frame}
\mode<all>{
\begin{frame}
\frametitle{Objectives}
\begin{itemize}
\item Overview of Core Constraint Programming
\item Three Main Concepts
\begin{itemize}
\item Constraint Propagation
\item Global Constraints
\item Customizing Search
\end{itemize}
%\item Get Some Experience with MiniZinc
\item Topics will be treated in more detail in later parts of the school
\item Based on Examples, not Formal Description
\end{itemize}
\end{frame}

\begin{frame}
\frametitle{Outline}
\begin{itemize}
\item Why Constraint Programming?
\item Constraint Propagation
\item Global Constraints
\item Customizing Search
%\item Choices in Modelling
%\item Decomposition
%\item More Global Constraints
%\item Methodology
%\item What is missing?
\end{itemize}
\end{frame}

% \begin{frame}
  % \frametitle{Using MiniZinc IDE}
  % \begin{itemize}
  % \item Developed in the Australian NICTA project
  % \item Maintained by Monash University
  % \item Modelling tool with multiple back-end solvers
  % \item Available from \url{https://www.minizinc.org/}
  % \end{itemize}
% \end{frame}


% \begin{frame}
% \frametitle{Examples in ECLiPSe}
% \begin{itemize}
% \item Open sourced constraint programming language
% \item Development goes back to 1985
% \item ECRC, ICL, IC-Parc, PTL, Cisco
% \item \url{https://eclipseclp.org/}
% \item Specialities
% \begin{itemize}
% \item Develop new solvers for specific domains
% \item Integration with MIP
% \end{itemize}
% \item Not included in bundled MiniZinc IDE
% \item Specialized visualization tools used here
  % \begin{itemize}
  % \item CP-Viz, Simonis et al. 2010
    % \end{itemize}
% \end{itemize}
% \end{frame}

\begin{frame}
\frametitle{Tutorial Based on ECLiPSe ELearning Course}
\begin{itemize}
\item Self-study course in constraint programming
\item Supported by Cisco Systems and Silicon Valley Community Foundation
\item Multi-media format, video lectures, slides, handout etc
\item {\small \url{https://eclipseclp.org/ELearning/index.html}}
\end{itemize}
\end{frame}

\begin{frame}
\frametitle{Also Part of CRT-AI Constraint Week}
\begin{itemize}
\item Annual one week course on CP and Optimization in Ireland
\item Part of national training program for PhD students in AI
\item \url{https://www.crt-ai.ie/}
\end{itemize}
\end{frame}



\begin{frame}
\frametitle{Constraint Programming - in a nutshell}
\begin{itemize}
\item Declarative description of problems with
\begin{itemize}
\item {\em Variables} which range over (finite) sets of values
\item {\em Constraints} over subsets of variables which restrict possible value combinations
\item A {\em solution} is a value assignment which satisfies all constraints
\end{itemize}

\item Constraint propagation/reasoning
\begin{itemize}
\item Removing inconsistent values for variables
\item Detect failure if constraint can not be satisfied
\item Interaction of constraints via shared variables
\item Incomplete
\end{itemize}

\item Search
\begin{itemize}
\item User controlled assignment of values to variables
\item Each step triggers constraint propagation 
\end{itemize}
\item Different domains require/allow different methods
\end{itemize}
\end{frame}

\begin{frame}
  \frametitle{Constraint Programming is Different}
  \begin{itemize}
  \item Declarative Programming
    \begin{itemize}
    \item Concentrate on what you want
      \item Not how to get there
      \item Program != Algorithm
      \item Program = Model
    \end{itemize}
    \item Applied to Combinatorial Problems
      \begin{itemize}
        \item No complete polynomial algorithms known (exist?)
        \item CP less ad-hoc than heuristics
        \item Models can evolve
  \end{itemize}
  \end{itemize}
  \end{frame}
    
\begin{frame}
  \frametitle{A Subtractive Process}
  \begin{textblock}{4}(8,-3)
    \includegraphics[width=4cm]{../introduction/images/stages}
  \end{textblock}
  \vfill
  \begin{quote}
    ``Oh, bosh, as Mr. Ruskin says. Sculpture, per se, is the simplest thing in the world. All you have to do is to take a big chunk of marble and a hammer and chisel, make up your mind what you are about to create and chip off all the marble you don't want.''-Paris Gaulois.
  \end{quote}
  
  {\tiny Source: \url{https://quoteinvestigator.com/2014/06/22/chip-away/}}
\end{frame}

\begin{frame}
\frametitle{Basic Process}
\begin{center}
\begin{tikzpicture}
\node[shape=rectangle,fill=pantone127-4] (problem) at (2,5) {Problem};
\node (human) at (2,4) {Human};
\node[shape=rectangle,fill=pantone127-4] (model) at (2,3) {Model};
\node (solver) at (2,2) {Constraint Solver/Search};
\node[shape=rectangle,fill=pantone157-8] (solution) at (2,1) {Solution};
\draw[-] (problem) -- (human);
\draw[->] (human) -- (model);
\draw[-] (model) -- (solver);
\draw[->] (solver) -- (solution);
\end{tikzpicture}
\end{center}
\end{frame}

\begin{frame}
\frametitle{More Realistic}
\begin{center}
\begin{tikzpicture}[xscale=4.0]
\node[shape=rectangle,fill=pantone127-4] (problem) at (2,5) {Problem};
\node (human) at (2,4) {Human};
\node[shape=rectangle,fill=pantone127-4] (model) at (2,3) {Model};
\node (solver) at (2,2) {Constraint Solver/Search};
\node[shape=rectangle,fill=pantone24-8] (hangs) at (1,1) {Hangs};
\node[shape=rectangle,fill=pantone157-8] (solution) at (2,1) {Solution};
\node[shape=rectangle,fill=pantone24-8] (wrong) at (3,1) {Wrong Solution};
\draw[-] (problem) -- (human);
\draw[->] (human) -- (model);
\draw[-] (model) -- (solver);
\draw[->] (solver) -- (hangs);
\draw[->] (solver) -- (solution);
\draw[->] (solver) -- (wrong);
\draw[->] (hangs) -- (1,4) -- (human);
\draw[->] (wrong) -- (3,4) -- (human);
\end{tikzpicture}
\end{center}
\end{frame}

\begin{frame}
\frametitle{Dual Role of Model}
\begin{itemize}
\item Allows Human to Express Problem
\begin{itemize}
\item Close to Problem Domain
\item Constraints as Abstractions
\end{itemize}
\item Allows Solver to Execute
\begin{itemize}
\item Variables as Communication Mechanism
\item Constraints as Algorithms
\end{itemize}
\end{itemize}
\end{frame}

\begin{frame}
\frametitle{Modelling Frameworks}
\begin{itemize}
\item \href{https://www.minizinc.org/}{MiniZinc} (NICTA, Monash University, Australia)
\item \href{https://github.com/eomahony/Numberjack}{NumberJack} (Insight, Ireland) 
\item \href{https://www-users.york.ac.uk/peter.nightingale/savilerow//}{EssencePrime/SavilleRow} (UK) 
\item \href{https://cpmpy.readthedocs.io/en/latest/}{CPMpy} (KU Leuven)
\item Allow use of multiple back-end solvers
\item Compile model into variants for each solver
\item A priori solver independent model(CP, MIP, SAT)
\end{itemize}
\end{frame} 

\begin{frame}
\frametitle{Framework Process}
\begin{center}
\begin{tikzpicture}[xscale=2]
\node[shape=rectangle,fill=pantone127-4] (problem) at (2.5,5) {Problem};
\node (human) at (2.5,4) {Human};
\node[shape=rectangle,fill=pantone127-4] (model) at (2.5,3) {Model};
\node (compile) at (2.5,2) {Compile/Reformulate};
\node (solver1) at (1,1) {CP};
\node (solver2) at (2,1) {MIP};
\node (solver3) at (3,1) {SAT};
\node (solver4) at (4,1) {Other};
\node[shape=rectangle,fill=pantone157-8] (solution1) at (1,0) {Solution};
\node[shape=rectangle,fill=pantone157-8] (solution2) at (2,0) {Solution};
\node[shape=rectangle,fill=pantone157-8] (solution3) at (3,0) {Solution};
\node[shape=rectangle,fill=pantone157-8] (solution4) at (4,0) {Solution};
\draw[-] (problem) -- (human);
\draw[->] (human) -- (model);
\draw[-] (model) -- (compile);
\draw[-] (compile) -- (solver1);
\draw[-] (compile) -- (solver2);
\draw[-] (compile) -- (solver3);
\draw[-] (compile) -- (solver4);
\draw[->] (solver1) -- (solution1);
\draw[->] (solver2) -- (solution2);
\draw[->] (solver3) -- (solution3);
\draw[->] (solver4) -- (solution4);
\end{tikzpicture}
\end{center}
\end{frame}

% \begin{frame}
  % \frametitle{Do It Now!}
  % \begin{itemize}
  % \item Download and install Minizinc
  % \item \url{https://www.minizinc.org/}
  % \end{itemize}
% \end{frame}


\begin{frame}
\frametitle{Why use Puzzles as Examples?}
\begin{itemize}
\item Easy to understand the problem
\item Solvable by hand without specialized knowledge
\item Possible to compare automated to manual solving process
\end{itemize}

\begin{quote}
The puzzle, though inanimate, is presented as a solvable problem without lasting negative consequences, a very low-risk low-reward situation. By being a puzzle, the object is attempting to convince the user that it must be completed.
\end{quote}
{\tiny Source: \href{https://www.everydayrhetoric.com/2020/03/the-rhetoric-of-puzzles/}{Every Day Rhetoric}}
\end{frame}


\mode<all>{
\part{Basic Constraint Propagation}
}
\begin{frame}
\partpage
\end{frame}
\mode<all>{
\input{../sendmore/body}
}

\mode<all>{
\part{Global Constraints}
}
\begin{frame}
\partpage
\end{frame}
\mode<all>{
\input{../sudoku/body}
}

\clearpage
\mode<all>{
\part{Customizing Search}
}
\begin{frame}
\partpage
\end{frame}
\mode<all>{
\input{../nqueen/body}
}


%% \clearpage
%% \mode<all>{
%%   \part{Problem Decomposition}
%% }
%% \begin{frame}
%%   \partpage
%% \end{frame}
%% \mode<all>{
%%   \input{../party/body}
%%   }

%% \clearpage
%% \mode<all>{
%%   \part{More Global Constraints}
%% }
%% \begin{frame}
%%   \partpage
%% \end{frame}
%% \mode<all>{
%%   \input{../car/body}
%%   }

%% \clearpage
%% \mode<all>{
%%   \part{Methodology}
%% }
%% \begin{frame}
%%   \partpage
%% \end{frame}
%% \mode<all>{
%%   \input{../methodology/body}
%%   }

\clearpage
\mode<all>{
\part{What is missing?}
}
\begin{frame}
\partpage
\end{frame}

\begin{frame}
\frametitle{Many Specialized Topics}
\begin{itemize}
\item How to design efficient core engine
\item Hybrids with LP/MIP tools
\item Hybrids with SAT
\item Symmetry breaking
\item Use of MDD/BDD to encode sets of solutions
\item High level modelling tools
\item Debugging/visualization
\end{itemize}
\end{frame}

\begin{frame}
\frametitle{Reformulation}
\begin{itemize}
\item Just because the user has modelled it this way, it doesn't mean we have to solve it that way
\begin{itemize}
\item Replace some constraint(s) by other, equivalent constraints
\item Because we don't have that constraint in our system
\item For performance
\end{itemize}
\end{itemize}
\end{frame}

\begin{frame}
\frametitle{Learning}
\begin{itemize}
\item While solving the problem we can learn how to strengthen the model/search 
\begin{itemize}
\item Understand which constraints/method contribute to propagation and change schedule
\item Learn no-good constraints by explaining failure
\item Adapt search strategy based on search experience
\end{itemize}
\end{itemize}
\end{frame}

%% \begin{frame}
%% \frametitle{Refined Process}
%% \begin{center}
%% \begin{tikzpicture}
%% \node[shape=rectangle,fill=pantone127-4] (problem) at (3,7) {Problem};
%% \node (human) at (3,6) {Human};
%% \node[shape=rectangle,fill=pantone127-4] (model) at (3,5) {User Model};
%% \node (reformulation) at (3,4) {Reformulation};
%% \node[shape=rectangle,fill=pantone127-4] (implementation) at (3,3) {Implementation Model};
%% \node (solver) at (3,2) {Constraint Solver/Search};
%% \node[shape=rectangle,fill=pantone157-8] (solution) at (3,1) {Solution};
%% \draw[-] (problem) -- (human);
%% \draw[->] (human) -- (model);
%% \draw[-] (model) -- (reformulation);
%% \draw[->] (reformulation) -- (implementation);
%% \draw[-] (implementation) -- (solver);
%% \draw[->] (solver) -- (0,2) -- (0,3) -- (implementation);
%% \draw[->] (solver) -- (solution);
%% \end{tikzpicture}
%% \end{center}
%% \end{frame}

%\begin{frame}
%\frametitle{What is CP actually used for?}
%\url{http://hsimonis.wordpress.com}
%\includegraphics[width=10cm]{../introduction/images/blog}
%\end{frame}

%% \begin{frame}
%% \frametitle{Exercises}
%% \begin{itemize}
%% \item Install Constraint Solver of your Choice
%% \item MiniZinc (MiningZinc) will be used in later sessions
%% \item Run Sendmoremoney, Sudoku, Queens examples
%% \item Can you choose which propagator to use for alldifferent?
%% \item Can you use different search strategies?
%% \end{itemize}
%% \end{frame}

\begin{frame}
\frametitle{More Learning Resources}
\begin{itemize}
\item Survey of Methods, Resources, and Formats for Teaching Constraint Programming
\begin{itemize}
\item by Tejas Santanam, Helmut Simonis
\item \url{https://doi.org/10.48550/arXiv.2403.12717}
\item Based on survey of community for WTCP 2023
\item \url{https://hsimonis.github.io/WTCP2023/}
\end{itemize}
\end{itemize}
\end{frame}


}


%% \clearpage
%% \mode<all>{
%%   \part{Problem Decomposition}
%% }
%% \begin{frame}
%%   \partpage
%% \end{frame}
%% \mode<all>{
%%   \begin{frame}
\frametitle{Objectives}
\begin{itemize}
\item Overview of Core Constraint Programming
\item Three Main Concepts
\begin{itemize}
\item Constraint Propagation
\item Global Constraints
\item Customizing Search
\end{itemize}
%\item Get Some Experience with MiniZinc
\item Topics will be treated in more detail in later parts of the school
\item Based on Examples, not Formal Description
\end{itemize}
\end{frame}

\begin{frame}
\frametitle{Outline}
\begin{itemize}
\item Why Constraint Programming?
\item Constraint Propagation
\item Global Constraints
\item Customizing Search
%\item Choices in Modelling
%\item Decomposition
%\item More Global Constraints
%\item Methodology
%\item What is missing?
\end{itemize}
\end{frame}

% \begin{frame}
  % \frametitle{Using MiniZinc IDE}
  % \begin{itemize}
  % \item Developed in the Australian NICTA project
  % \item Maintained by Monash University
  % \item Modelling tool with multiple back-end solvers
  % \item Available from \url{https://www.minizinc.org/}
  % \end{itemize}
% \end{frame}


% \begin{frame}
% \frametitle{Examples in ECLiPSe}
% \begin{itemize}
% \item Open sourced constraint programming language
% \item Development goes back to 1985
% \item ECRC, ICL, IC-Parc, PTL, Cisco
% \item \url{https://eclipseclp.org/}
% \item Specialities
% \begin{itemize}
% \item Develop new solvers for specific domains
% \item Integration with MIP
% \end{itemize}
% \item Not included in bundled MiniZinc IDE
% \item Specialized visualization tools used here
  % \begin{itemize}
  % \item CP-Viz, Simonis et al. 2010
    % \end{itemize}
% \end{itemize}
% \end{frame}

\begin{frame}
\frametitle{Tutorial Based on ECLiPSe ELearning Course}
\begin{itemize}
\item Self-study course in constraint programming
\item Supported by Cisco Systems and Silicon Valley Community Foundation
\item Multi-media format, video lectures, slides, handout etc
\item {\small \url{https://eclipseclp.org/ELearning/index.html}}
\end{itemize}
\end{frame}

\begin{frame}
\frametitle{Also Part of CRT-AI Constraint Week}
\begin{itemize}
\item Annual one week course on CP and Optimization in Ireland
\item Part of national training program for PhD students in AI
\item \url{https://www.crt-ai.ie/}
\end{itemize}
\end{frame}



\begin{frame}
\frametitle{Constraint Programming - in a nutshell}
\begin{itemize}
\item Declarative description of problems with
\begin{itemize}
\item {\em Variables} which range over (finite) sets of values
\item {\em Constraints} over subsets of variables which restrict possible value combinations
\item A {\em solution} is a value assignment which satisfies all constraints
\end{itemize}

\item Constraint propagation/reasoning
\begin{itemize}
\item Removing inconsistent values for variables
\item Detect failure if constraint can not be satisfied
\item Interaction of constraints via shared variables
\item Incomplete
\end{itemize}

\item Search
\begin{itemize}
\item User controlled assignment of values to variables
\item Each step triggers constraint propagation 
\end{itemize}
\item Different domains require/allow different methods
\end{itemize}
\end{frame}

\begin{frame}
  \frametitle{Constraint Programming is Different}
  \begin{itemize}
  \item Declarative Programming
    \begin{itemize}
    \item Concentrate on what you want
      \item Not how to get there
      \item Program != Algorithm
      \item Program = Model
    \end{itemize}
    \item Applied to Combinatorial Problems
      \begin{itemize}
        \item No complete polynomial algorithms known (exist?)
        \item CP less ad-hoc than heuristics
        \item Models can evolve
  \end{itemize}
  \end{itemize}
  \end{frame}
    
\begin{frame}
  \frametitle{A Subtractive Process}
  \begin{textblock}{4}(8,-3)
    \includegraphics[width=4cm]{../introduction/images/stages}
  \end{textblock}
  \vfill
  \begin{quote}
    ``Oh, bosh, as Mr. Ruskin says. Sculpture, per se, is the simplest thing in the world. All you have to do is to take a big chunk of marble and a hammer and chisel, make up your mind what you are about to create and chip off all the marble you don't want.''-Paris Gaulois.
  \end{quote}
  
  {\tiny Source: \url{https://quoteinvestigator.com/2014/06/22/chip-away/}}
\end{frame}

\begin{frame}
\frametitle{Basic Process}
\begin{center}
\begin{tikzpicture}
\node[shape=rectangle,fill=pantone127-4] (problem) at (2,5) {Problem};
\node (human) at (2,4) {Human};
\node[shape=rectangle,fill=pantone127-4] (model) at (2,3) {Model};
\node (solver) at (2,2) {Constraint Solver/Search};
\node[shape=rectangle,fill=pantone157-8] (solution) at (2,1) {Solution};
\draw[-] (problem) -- (human);
\draw[->] (human) -- (model);
\draw[-] (model) -- (solver);
\draw[->] (solver) -- (solution);
\end{tikzpicture}
\end{center}
\end{frame}

\begin{frame}
\frametitle{More Realistic}
\begin{center}
\begin{tikzpicture}[xscale=4.0]
\node[shape=rectangle,fill=pantone127-4] (problem) at (2,5) {Problem};
\node (human) at (2,4) {Human};
\node[shape=rectangle,fill=pantone127-4] (model) at (2,3) {Model};
\node (solver) at (2,2) {Constraint Solver/Search};
\node[shape=rectangle,fill=pantone24-8] (hangs) at (1,1) {Hangs};
\node[shape=rectangle,fill=pantone157-8] (solution) at (2,1) {Solution};
\node[shape=rectangle,fill=pantone24-8] (wrong) at (3,1) {Wrong Solution};
\draw[-] (problem) -- (human);
\draw[->] (human) -- (model);
\draw[-] (model) -- (solver);
\draw[->] (solver) -- (hangs);
\draw[->] (solver) -- (solution);
\draw[->] (solver) -- (wrong);
\draw[->] (hangs) -- (1,4) -- (human);
\draw[->] (wrong) -- (3,4) -- (human);
\end{tikzpicture}
\end{center}
\end{frame}

\begin{frame}
\frametitle{Dual Role of Model}
\begin{itemize}
\item Allows Human to Express Problem
\begin{itemize}
\item Close to Problem Domain
\item Constraints as Abstractions
\end{itemize}
\item Allows Solver to Execute
\begin{itemize}
\item Variables as Communication Mechanism
\item Constraints as Algorithms
\end{itemize}
\end{itemize}
\end{frame}

\begin{frame}
\frametitle{Modelling Frameworks}
\begin{itemize}
\item \href{https://www.minizinc.org/}{MiniZinc} (NICTA, Monash University, Australia)
\item \href{https://github.com/eomahony/Numberjack}{NumberJack} (Insight, Ireland) 
\item \href{https://www-users.york.ac.uk/peter.nightingale/savilerow//}{EssencePrime/SavilleRow} (UK) 
\item \href{https://cpmpy.readthedocs.io/en/latest/}{CPMpy} (KU Leuven)
\item Allow use of multiple back-end solvers
\item Compile model into variants for each solver
\item A priori solver independent model(CP, MIP, SAT)
\end{itemize}
\end{frame} 

\begin{frame}
\frametitle{Framework Process}
\begin{center}
\begin{tikzpicture}[xscale=2]
\node[shape=rectangle,fill=pantone127-4] (problem) at (2.5,5) {Problem};
\node (human) at (2.5,4) {Human};
\node[shape=rectangle,fill=pantone127-4] (model) at (2.5,3) {Model};
\node (compile) at (2.5,2) {Compile/Reformulate};
\node (solver1) at (1,1) {CP};
\node (solver2) at (2,1) {MIP};
\node (solver3) at (3,1) {SAT};
\node (solver4) at (4,1) {Other};
\node[shape=rectangle,fill=pantone157-8] (solution1) at (1,0) {Solution};
\node[shape=rectangle,fill=pantone157-8] (solution2) at (2,0) {Solution};
\node[shape=rectangle,fill=pantone157-8] (solution3) at (3,0) {Solution};
\node[shape=rectangle,fill=pantone157-8] (solution4) at (4,0) {Solution};
\draw[-] (problem) -- (human);
\draw[->] (human) -- (model);
\draw[-] (model) -- (compile);
\draw[-] (compile) -- (solver1);
\draw[-] (compile) -- (solver2);
\draw[-] (compile) -- (solver3);
\draw[-] (compile) -- (solver4);
\draw[->] (solver1) -- (solution1);
\draw[->] (solver2) -- (solution2);
\draw[->] (solver3) -- (solution3);
\draw[->] (solver4) -- (solution4);
\end{tikzpicture}
\end{center}
\end{frame}

% \begin{frame}
  % \frametitle{Do It Now!}
  % \begin{itemize}
  % \item Download and install Minizinc
  % \item \url{https://www.minizinc.org/}
  % \end{itemize}
% \end{frame}


\begin{frame}
\frametitle{Why use Puzzles as Examples?}
\begin{itemize}
\item Easy to understand the problem
\item Solvable by hand without specialized knowledge
\item Possible to compare automated to manual solving process
\end{itemize}

\begin{quote}
The puzzle, though inanimate, is presented as a solvable problem without lasting negative consequences, a very low-risk low-reward situation. By being a puzzle, the object is attempting to convince the user that it must be completed.
\end{quote}
{\tiny Source: \href{https://www.everydayrhetoric.com/2020/03/the-rhetoric-of-puzzles/}{Every Day Rhetoric}}
\end{frame}


\mode<all>{
\part{Basic Constraint Propagation}
}
\begin{frame}
\partpage
\end{frame}
\mode<all>{
\input{../sendmore/body}
}

\mode<all>{
\part{Global Constraints}
}
\begin{frame}
\partpage
\end{frame}
\mode<all>{
\input{../sudoku/body}
}

\clearpage
\mode<all>{
\part{Customizing Search}
}
\begin{frame}
\partpage
\end{frame}
\mode<all>{
\input{../nqueen/body}
}


%% \clearpage
%% \mode<all>{
%%   \part{Problem Decomposition}
%% }
%% \begin{frame}
%%   \partpage
%% \end{frame}
%% \mode<all>{
%%   \input{../party/body}
%%   }

%% \clearpage
%% \mode<all>{
%%   \part{More Global Constraints}
%% }
%% \begin{frame}
%%   \partpage
%% \end{frame}
%% \mode<all>{
%%   \input{../car/body}
%%   }

%% \clearpage
%% \mode<all>{
%%   \part{Methodology}
%% }
%% \begin{frame}
%%   \partpage
%% \end{frame}
%% \mode<all>{
%%   \input{../methodology/body}
%%   }

\clearpage
\mode<all>{
\part{What is missing?}
}
\begin{frame}
\partpage
\end{frame}

\begin{frame}
\frametitle{Many Specialized Topics}
\begin{itemize}
\item How to design efficient core engine
\item Hybrids with LP/MIP tools
\item Hybrids with SAT
\item Symmetry breaking
\item Use of MDD/BDD to encode sets of solutions
\item High level modelling tools
\item Debugging/visualization
\end{itemize}
\end{frame}

\begin{frame}
\frametitle{Reformulation}
\begin{itemize}
\item Just because the user has modelled it this way, it doesn't mean we have to solve it that way
\begin{itemize}
\item Replace some constraint(s) by other, equivalent constraints
\item Because we don't have that constraint in our system
\item For performance
\end{itemize}
\end{itemize}
\end{frame}

\begin{frame}
\frametitle{Learning}
\begin{itemize}
\item While solving the problem we can learn how to strengthen the model/search 
\begin{itemize}
\item Understand which constraints/method contribute to propagation and change schedule
\item Learn no-good constraints by explaining failure
\item Adapt search strategy based on search experience
\end{itemize}
\end{itemize}
\end{frame}

%% \begin{frame}
%% \frametitle{Refined Process}
%% \begin{center}
%% \begin{tikzpicture}
%% \node[shape=rectangle,fill=pantone127-4] (problem) at (3,7) {Problem};
%% \node (human) at (3,6) {Human};
%% \node[shape=rectangle,fill=pantone127-4] (model) at (3,5) {User Model};
%% \node (reformulation) at (3,4) {Reformulation};
%% \node[shape=rectangle,fill=pantone127-4] (implementation) at (3,3) {Implementation Model};
%% \node (solver) at (3,2) {Constraint Solver/Search};
%% \node[shape=rectangle,fill=pantone157-8] (solution) at (3,1) {Solution};
%% \draw[-] (problem) -- (human);
%% \draw[->] (human) -- (model);
%% \draw[-] (model) -- (reformulation);
%% \draw[->] (reformulation) -- (implementation);
%% \draw[-] (implementation) -- (solver);
%% \draw[->] (solver) -- (0,2) -- (0,3) -- (implementation);
%% \draw[->] (solver) -- (solution);
%% \end{tikzpicture}
%% \end{center}
%% \end{frame}

%\begin{frame}
%\frametitle{What is CP actually used for?}
%\url{http://hsimonis.wordpress.com}
%\includegraphics[width=10cm]{../introduction/images/blog}
%\end{frame}

%% \begin{frame}
%% \frametitle{Exercises}
%% \begin{itemize}
%% \item Install Constraint Solver of your Choice
%% \item MiniZinc (MiningZinc) will be used in later sessions
%% \item Run Sendmoremoney, Sudoku, Queens examples
%% \item Can you choose which propagator to use for alldifferent?
%% \item Can you use different search strategies?
%% \end{itemize}
%% \end{frame}

\begin{frame}
\frametitle{More Learning Resources}
\begin{itemize}
\item Survey of Methods, Resources, and Formats for Teaching Constraint Programming
\begin{itemize}
\item by Tejas Santanam, Helmut Simonis
\item \url{https://doi.org/10.48550/arXiv.2403.12717}
\item Based on survey of community for WTCP 2023
\item \url{https://hsimonis.github.io/WTCP2023/}
\end{itemize}
\end{itemize}
\end{frame}


%%   }

%% \clearpage
%% \mode<all>{
%%   \part{More Global Constraints}
%% }
%% \begin{frame}
%%   \partpage
%% \end{frame}
%% \mode<all>{
%%   \begin{frame}
\frametitle{Objectives}
\begin{itemize}
\item Overview of Core Constraint Programming
\item Three Main Concepts
\begin{itemize}
\item Constraint Propagation
\item Global Constraints
\item Customizing Search
\end{itemize}
%\item Get Some Experience with MiniZinc
\item Topics will be treated in more detail in later parts of the school
\item Based on Examples, not Formal Description
\end{itemize}
\end{frame}

\begin{frame}
\frametitle{Outline}
\begin{itemize}
\item Why Constraint Programming?
\item Constraint Propagation
\item Global Constraints
\item Customizing Search
%\item Choices in Modelling
%\item Decomposition
%\item More Global Constraints
%\item Methodology
%\item What is missing?
\end{itemize}
\end{frame}

% \begin{frame}
  % \frametitle{Using MiniZinc IDE}
  % \begin{itemize}
  % \item Developed in the Australian NICTA project
  % \item Maintained by Monash University
  % \item Modelling tool with multiple back-end solvers
  % \item Available from \url{https://www.minizinc.org/}
  % \end{itemize}
% \end{frame}


% \begin{frame}
% \frametitle{Examples in ECLiPSe}
% \begin{itemize}
% \item Open sourced constraint programming language
% \item Development goes back to 1985
% \item ECRC, ICL, IC-Parc, PTL, Cisco
% \item \url{https://eclipseclp.org/}
% \item Specialities
% \begin{itemize}
% \item Develop new solvers for specific domains
% \item Integration with MIP
% \end{itemize}
% \item Not included in bundled MiniZinc IDE
% \item Specialized visualization tools used here
  % \begin{itemize}
  % \item CP-Viz, Simonis et al. 2010
    % \end{itemize}
% \end{itemize}
% \end{frame}

\begin{frame}
\frametitle{Tutorial Based on ECLiPSe ELearning Course}
\begin{itemize}
\item Self-study course in constraint programming
\item Supported by Cisco Systems and Silicon Valley Community Foundation
\item Multi-media format, video lectures, slides, handout etc
\item {\small \url{https://eclipseclp.org/ELearning/index.html}}
\end{itemize}
\end{frame}

\begin{frame}
\frametitle{Also Part of CRT-AI Constraint Week}
\begin{itemize}
\item Annual one week course on CP and Optimization in Ireland
\item Part of national training program for PhD students in AI
\item \url{https://www.crt-ai.ie/}
\end{itemize}
\end{frame}



\begin{frame}
\frametitle{Constraint Programming - in a nutshell}
\begin{itemize}
\item Declarative description of problems with
\begin{itemize}
\item {\em Variables} which range over (finite) sets of values
\item {\em Constraints} over subsets of variables which restrict possible value combinations
\item A {\em solution} is a value assignment which satisfies all constraints
\end{itemize}

\item Constraint propagation/reasoning
\begin{itemize}
\item Removing inconsistent values for variables
\item Detect failure if constraint can not be satisfied
\item Interaction of constraints via shared variables
\item Incomplete
\end{itemize}

\item Search
\begin{itemize}
\item User controlled assignment of values to variables
\item Each step triggers constraint propagation 
\end{itemize}
\item Different domains require/allow different methods
\end{itemize}
\end{frame}

\begin{frame}
  \frametitle{Constraint Programming is Different}
  \begin{itemize}
  \item Declarative Programming
    \begin{itemize}
    \item Concentrate on what you want
      \item Not how to get there
      \item Program != Algorithm
      \item Program = Model
    \end{itemize}
    \item Applied to Combinatorial Problems
      \begin{itemize}
        \item No complete polynomial algorithms known (exist?)
        \item CP less ad-hoc than heuristics
        \item Models can evolve
  \end{itemize}
  \end{itemize}
  \end{frame}
    
\begin{frame}
  \frametitle{A Subtractive Process}
  \begin{textblock}{4}(8,-3)
    \includegraphics[width=4cm]{../introduction/images/stages}
  \end{textblock}
  \vfill
  \begin{quote}
    ``Oh, bosh, as Mr. Ruskin says. Sculpture, per se, is the simplest thing in the world. All you have to do is to take a big chunk of marble and a hammer and chisel, make up your mind what you are about to create and chip off all the marble you don't want.''-Paris Gaulois.
  \end{quote}
  
  {\tiny Source: \url{https://quoteinvestigator.com/2014/06/22/chip-away/}}
\end{frame}

\begin{frame}
\frametitle{Basic Process}
\begin{center}
\begin{tikzpicture}
\node[shape=rectangle,fill=pantone127-4] (problem) at (2,5) {Problem};
\node (human) at (2,4) {Human};
\node[shape=rectangle,fill=pantone127-4] (model) at (2,3) {Model};
\node (solver) at (2,2) {Constraint Solver/Search};
\node[shape=rectangle,fill=pantone157-8] (solution) at (2,1) {Solution};
\draw[-] (problem) -- (human);
\draw[->] (human) -- (model);
\draw[-] (model) -- (solver);
\draw[->] (solver) -- (solution);
\end{tikzpicture}
\end{center}
\end{frame}

\begin{frame}
\frametitle{More Realistic}
\begin{center}
\begin{tikzpicture}[xscale=4.0]
\node[shape=rectangle,fill=pantone127-4] (problem) at (2,5) {Problem};
\node (human) at (2,4) {Human};
\node[shape=rectangle,fill=pantone127-4] (model) at (2,3) {Model};
\node (solver) at (2,2) {Constraint Solver/Search};
\node[shape=rectangle,fill=pantone24-8] (hangs) at (1,1) {Hangs};
\node[shape=rectangle,fill=pantone157-8] (solution) at (2,1) {Solution};
\node[shape=rectangle,fill=pantone24-8] (wrong) at (3,1) {Wrong Solution};
\draw[-] (problem) -- (human);
\draw[->] (human) -- (model);
\draw[-] (model) -- (solver);
\draw[->] (solver) -- (hangs);
\draw[->] (solver) -- (solution);
\draw[->] (solver) -- (wrong);
\draw[->] (hangs) -- (1,4) -- (human);
\draw[->] (wrong) -- (3,4) -- (human);
\end{tikzpicture}
\end{center}
\end{frame}

\begin{frame}
\frametitle{Dual Role of Model}
\begin{itemize}
\item Allows Human to Express Problem
\begin{itemize}
\item Close to Problem Domain
\item Constraints as Abstractions
\end{itemize}
\item Allows Solver to Execute
\begin{itemize}
\item Variables as Communication Mechanism
\item Constraints as Algorithms
\end{itemize}
\end{itemize}
\end{frame}

\begin{frame}
\frametitle{Modelling Frameworks}
\begin{itemize}
\item \href{https://www.minizinc.org/}{MiniZinc} (NICTA, Monash University, Australia)
\item \href{https://github.com/eomahony/Numberjack}{NumberJack} (Insight, Ireland) 
\item \href{https://www-users.york.ac.uk/peter.nightingale/savilerow//}{EssencePrime/SavilleRow} (UK) 
\item \href{https://cpmpy.readthedocs.io/en/latest/}{CPMpy} (KU Leuven)
\item Allow use of multiple back-end solvers
\item Compile model into variants for each solver
\item A priori solver independent model(CP, MIP, SAT)
\end{itemize}
\end{frame} 

\begin{frame}
\frametitle{Framework Process}
\begin{center}
\begin{tikzpicture}[xscale=2]
\node[shape=rectangle,fill=pantone127-4] (problem) at (2.5,5) {Problem};
\node (human) at (2.5,4) {Human};
\node[shape=rectangle,fill=pantone127-4] (model) at (2.5,3) {Model};
\node (compile) at (2.5,2) {Compile/Reformulate};
\node (solver1) at (1,1) {CP};
\node (solver2) at (2,1) {MIP};
\node (solver3) at (3,1) {SAT};
\node (solver4) at (4,1) {Other};
\node[shape=rectangle,fill=pantone157-8] (solution1) at (1,0) {Solution};
\node[shape=rectangle,fill=pantone157-8] (solution2) at (2,0) {Solution};
\node[shape=rectangle,fill=pantone157-8] (solution3) at (3,0) {Solution};
\node[shape=rectangle,fill=pantone157-8] (solution4) at (4,0) {Solution};
\draw[-] (problem) -- (human);
\draw[->] (human) -- (model);
\draw[-] (model) -- (compile);
\draw[-] (compile) -- (solver1);
\draw[-] (compile) -- (solver2);
\draw[-] (compile) -- (solver3);
\draw[-] (compile) -- (solver4);
\draw[->] (solver1) -- (solution1);
\draw[->] (solver2) -- (solution2);
\draw[->] (solver3) -- (solution3);
\draw[->] (solver4) -- (solution4);
\end{tikzpicture}
\end{center}
\end{frame}

% \begin{frame}
  % \frametitle{Do It Now!}
  % \begin{itemize}
  % \item Download and install Minizinc
  % \item \url{https://www.minizinc.org/}
  % \end{itemize}
% \end{frame}


\begin{frame}
\frametitle{Why use Puzzles as Examples?}
\begin{itemize}
\item Easy to understand the problem
\item Solvable by hand without specialized knowledge
\item Possible to compare automated to manual solving process
\end{itemize}

\begin{quote}
The puzzle, though inanimate, is presented as a solvable problem without lasting negative consequences, a very low-risk low-reward situation. By being a puzzle, the object is attempting to convince the user that it must be completed.
\end{quote}
{\tiny Source: \href{https://www.everydayrhetoric.com/2020/03/the-rhetoric-of-puzzles/}{Every Day Rhetoric}}
\end{frame}


\mode<all>{
\part{Basic Constraint Propagation}
}
\begin{frame}
\partpage
\end{frame}
\mode<all>{
\input{../sendmore/body}
}

\mode<all>{
\part{Global Constraints}
}
\begin{frame}
\partpage
\end{frame}
\mode<all>{
\input{../sudoku/body}
}

\clearpage
\mode<all>{
\part{Customizing Search}
}
\begin{frame}
\partpage
\end{frame}
\mode<all>{
\input{../nqueen/body}
}


%% \clearpage
%% \mode<all>{
%%   \part{Problem Decomposition}
%% }
%% \begin{frame}
%%   \partpage
%% \end{frame}
%% \mode<all>{
%%   \input{../party/body}
%%   }

%% \clearpage
%% \mode<all>{
%%   \part{More Global Constraints}
%% }
%% \begin{frame}
%%   \partpage
%% \end{frame}
%% \mode<all>{
%%   \input{../car/body}
%%   }

%% \clearpage
%% \mode<all>{
%%   \part{Methodology}
%% }
%% \begin{frame}
%%   \partpage
%% \end{frame}
%% \mode<all>{
%%   \input{../methodology/body}
%%   }

\clearpage
\mode<all>{
\part{What is missing?}
}
\begin{frame}
\partpage
\end{frame}

\begin{frame}
\frametitle{Many Specialized Topics}
\begin{itemize}
\item How to design efficient core engine
\item Hybrids with LP/MIP tools
\item Hybrids with SAT
\item Symmetry breaking
\item Use of MDD/BDD to encode sets of solutions
\item High level modelling tools
\item Debugging/visualization
\end{itemize}
\end{frame}

\begin{frame}
\frametitle{Reformulation}
\begin{itemize}
\item Just because the user has modelled it this way, it doesn't mean we have to solve it that way
\begin{itemize}
\item Replace some constraint(s) by other, equivalent constraints
\item Because we don't have that constraint in our system
\item For performance
\end{itemize}
\end{itemize}
\end{frame}

\begin{frame}
\frametitle{Learning}
\begin{itemize}
\item While solving the problem we can learn how to strengthen the model/search 
\begin{itemize}
\item Understand which constraints/method contribute to propagation and change schedule
\item Learn no-good constraints by explaining failure
\item Adapt search strategy based on search experience
\end{itemize}
\end{itemize}
\end{frame}

%% \begin{frame}
%% \frametitle{Refined Process}
%% \begin{center}
%% \begin{tikzpicture}
%% \node[shape=rectangle,fill=pantone127-4] (problem) at (3,7) {Problem};
%% \node (human) at (3,6) {Human};
%% \node[shape=rectangle,fill=pantone127-4] (model) at (3,5) {User Model};
%% \node (reformulation) at (3,4) {Reformulation};
%% \node[shape=rectangle,fill=pantone127-4] (implementation) at (3,3) {Implementation Model};
%% \node (solver) at (3,2) {Constraint Solver/Search};
%% \node[shape=rectangle,fill=pantone157-8] (solution) at (3,1) {Solution};
%% \draw[-] (problem) -- (human);
%% \draw[->] (human) -- (model);
%% \draw[-] (model) -- (reformulation);
%% \draw[->] (reformulation) -- (implementation);
%% \draw[-] (implementation) -- (solver);
%% \draw[->] (solver) -- (0,2) -- (0,3) -- (implementation);
%% \draw[->] (solver) -- (solution);
%% \end{tikzpicture}
%% \end{center}
%% \end{frame}

%\begin{frame}
%\frametitle{What is CP actually used for?}
%\url{http://hsimonis.wordpress.com}
%\includegraphics[width=10cm]{../introduction/images/blog}
%\end{frame}

%% \begin{frame}
%% \frametitle{Exercises}
%% \begin{itemize}
%% \item Install Constraint Solver of your Choice
%% \item MiniZinc (MiningZinc) will be used in later sessions
%% \item Run Sendmoremoney, Sudoku, Queens examples
%% \item Can you choose which propagator to use for alldifferent?
%% \item Can you use different search strategies?
%% \end{itemize}
%% \end{frame}

\begin{frame}
\frametitle{More Learning Resources}
\begin{itemize}
\item Survey of Methods, Resources, and Formats for Teaching Constraint Programming
\begin{itemize}
\item by Tejas Santanam, Helmut Simonis
\item \url{https://doi.org/10.48550/arXiv.2403.12717}
\item Based on survey of community for WTCP 2023
\item \url{https://hsimonis.github.io/WTCP2023/}
\end{itemize}
\end{itemize}
\end{frame}


%%   }

%% \clearpage
%% \mode<all>{
%%   \part{Methodology}
%% }
%% \begin{frame}
%%   \partpage
%% \end{frame}
%% \mode<all>{
%%   \begin{frame}
\frametitle{Objectives}
\begin{itemize}
\item Overview of Core Constraint Programming
\item Three Main Concepts
\begin{itemize}
\item Constraint Propagation
\item Global Constraints
\item Customizing Search
\end{itemize}
%\item Get Some Experience with MiniZinc
\item Topics will be treated in more detail in later parts of the school
\item Based on Examples, not Formal Description
\end{itemize}
\end{frame}

\begin{frame}
\frametitle{Outline}
\begin{itemize}
\item Why Constraint Programming?
\item Constraint Propagation
\item Global Constraints
\item Customizing Search
%\item Choices in Modelling
%\item Decomposition
%\item More Global Constraints
%\item Methodology
%\item What is missing?
\end{itemize}
\end{frame}

% \begin{frame}
  % \frametitle{Using MiniZinc IDE}
  % \begin{itemize}
  % \item Developed in the Australian NICTA project
  % \item Maintained by Monash University
  % \item Modelling tool with multiple back-end solvers
  % \item Available from \url{https://www.minizinc.org/}
  % \end{itemize}
% \end{frame}


% \begin{frame}
% \frametitle{Examples in ECLiPSe}
% \begin{itemize}
% \item Open sourced constraint programming language
% \item Development goes back to 1985
% \item ECRC, ICL, IC-Parc, PTL, Cisco
% \item \url{https://eclipseclp.org/}
% \item Specialities
% \begin{itemize}
% \item Develop new solvers for specific domains
% \item Integration with MIP
% \end{itemize}
% \item Not included in bundled MiniZinc IDE
% \item Specialized visualization tools used here
  % \begin{itemize}
  % \item CP-Viz, Simonis et al. 2010
    % \end{itemize}
% \end{itemize}
% \end{frame}

\begin{frame}
\frametitle{Tutorial Based on ECLiPSe ELearning Course}
\begin{itemize}
\item Self-study course in constraint programming
\item Supported by Cisco Systems and Silicon Valley Community Foundation
\item Multi-media format, video lectures, slides, handout etc
\item {\small \url{https://eclipseclp.org/ELearning/index.html}}
\end{itemize}
\end{frame}

\begin{frame}
\frametitle{Also Part of CRT-AI Constraint Week}
\begin{itemize}
\item Annual one week course on CP and Optimization in Ireland
\item Part of national training program for PhD students in AI
\item \url{https://www.crt-ai.ie/}
\end{itemize}
\end{frame}



\begin{frame}
\frametitle{Constraint Programming - in a nutshell}
\begin{itemize}
\item Declarative description of problems with
\begin{itemize}
\item {\em Variables} which range over (finite) sets of values
\item {\em Constraints} over subsets of variables which restrict possible value combinations
\item A {\em solution} is a value assignment which satisfies all constraints
\end{itemize}

\item Constraint propagation/reasoning
\begin{itemize}
\item Removing inconsistent values for variables
\item Detect failure if constraint can not be satisfied
\item Interaction of constraints via shared variables
\item Incomplete
\end{itemize}

\item Search
\begin{itemize}
\item User controlled assignment of values to variables
\item Each step triggers constraint propagation 
\end{itemize}
\item Different domains require/allow different methods
\end{itemize}
\end{frame}

\begin{frame}
  \frametitle{Constraint Programming is Different}
  \begin{itemize}
  \item Declarative Programming
    \begin{itemize}
    \item Concentrate on what you want
      \item Not how to get there
      \item Program != Algorithm
      \item Program = Model
    \end{itemize}
    \item Applied to Combinatorial Problems
      \begin{itemize}
        \item No complete polynomial algorithms known (exist?)
        \item CP less ad-hoc than heuristics
        \item Models can evolve
  \end{itemize}
  \end{itemize}
  \end{frame}
    
\begin{frame}
  \frametitle{A Subtractive Process}
  \begin{textblock}{4}(8,-3)
    \includegraphics[width=4cm]{../introduction/images/stages}
  \end{textblock}
  \vfill
  \begin{quote}
    ``Oh, bosh, as Mr. Ruskin says. Sculpture, per se, is the simplest thing in the world. All you have to do is to take a big chunk of marble and a hammer and chisel, make up your mind what you are about to create and chip off all the marble you don't want.''-Paris Gaulois.
  \end{quote}
  
  {\tiny Source: \url{https://quoteinvestigator.com/2014/06/22/chip-away/}}
\end{frame}

\begin{frame}
\frametitle{Basic Process}
\begin{center}
\begin{tikzpicture}
\node[shape=rectangle,fill=pantone127-4] (problem) at (2,5) {Problem};
\node (human) at (2,4) {Human};
\node[shape=rectangle,fill=pantone127-4] (model) at (2,3) {Model};
\node (solver) at (2,2) {Constraint Solver/Search};
\node[shape=rectangle,fill=pantone157-8] (solution) at (2,1) {Solution};
\draw[-] (problem) -- (human);
\draw[->] (human) -- (model);
\draw[-] (model) -- (solver);
\draw[->] (solver) -- (solution);
\end{tikzpicture}
\end{center}
\end{frame}

\begin{frame}
\frametitle{More Realistic}
\begin{center}
\begin{tikzpicture}[xscale=4.0]
\node[shape=rectangle,fill=pantone127-4] (problem) at (2,5) {Problem};
\node (human) at (2,4) {Human};
\node[shape=rectangle,fill=pantone127-4] (model) at (2,3) {Model};
\node (solver) at (2,2) {Constraint Solver/Search};
\node[shape=rectangle,fill=pantone24-8] (hangs) at (1,1) {Hangs};
\node[shape=rectangle,fill=pantone157-8] (solution) at (2,1) {Solution};
\node[shape=rectangle,fill=pantone24-8] (wrong) at (3,1) {Wrong Solution};
\draw[-] (problem) -- (human);
\draw[->] (human) -- (model);
\draw[-] (model) -- (solver);
\draw[->] (solver) -- (hangs);
\draw[->] (solver) -- (solution);
\draw[->] (solver) -- (wrong);
\draw[->] (hangs) -- (1,4) -- (human);
\draw[->] (wrong) -- (3,4) -- (human);
\end{tikzpicture}
\end{center}
\end{frame}

\begin{frame}
\frametitle{Dual Role of Model}
\begin{itemize}
\item Allows Human to Express Problem
\begin{itemize}
\item Close to Problem Domain
\item Constraints as Abstractions
\end{itemize}
\item Allows Solver to Execute
\begin{itemize}
\item Variables as Communication Mechanism
\item Constraints as Algorithms
\end{itemize}
\end{itemize}
\end{frame}

\begin{frame}
\frametitle{Modelling Frameworks}
\begin{itemize}
\item \href{https://www.minizinc.org/}{MiniZinc} (NICTA, Monash University, Australia)
\item \href{https://github.com/eomahony/Numberjack}{NumberJack} (Insight, Ireland) 
\item \href{https://www-users.york.ac.uk/peter.nightingale/savilerow//}{EssencePrime/SavilleRow} (UK) 
\item \href{https://cpmpy.readthedocs.io/en/latest/}{CPMpy} (KU Leuven)
\item Allow use of multiple back-end solvers
\item Compile model into variants for each solver
\item A priori solver independent model(CP, MIP, SAT)
\end{itemize}
\end{frame} 

\begin{frame}
\frametitle{Framework Process}
\begin{center}
\begin{tikzpicture}[xscale=2]
\node[shape=rectangle,fill=pantone127-4] (problem) at (2.5,5) {Problem};
\node (human) at (2.5,4) {Human};
\node[shape=rectangle,fill=pantone127-4] (model) at (2.5,3) {Model};
\node (compile) at (2.5,2) {Compile/Reformulate};
\node (solver1) at (1,1) {CP};
\node (solver2) at (2,1) {MIP};
\node (solver3) at (3,1) {SAT};
\node (solver4) at (4,1) {Other};
\node[shape=rectangle,fill=pantone157-8] (solution1) at (1,0) {Solution};
\node[shape=rectangle,fill=pantone157-8] (solution2) at (2,0) {Solution};
\node[shape=rectangle,fill=pantone157-8] (solution3) at (3,0) {Solution};
\node[shape=rectangle,fill=pantone157-8] (solution4) at (4,0) {Solution};
\draw[-] (problem) -- (human);
\draw[->] (human) -- (model);
\draw[-] (model) -- (compile);
\draw[-] (compile) -- (solver1);
\draw[-] (compile) -- (solver2);
\draw[-] (compile) -- (solver3);
\draw[-] (compile) -- (solver4);
\draw[->] (solver1) -- (solution1);
\draw[->] (solver2) -- (solution2);
\draw[->] (solver3) -- (solution3);
\draw[->] (solver4) -- (solution4);
\end{tikzpicture}
\end{center}
\end{frame}

% \begin{frame}
  % \frametitle{Do It Now!}
  % \begin{itemize}
  % \item Download and install Minizinc
  % \item \url{https://www.minizinc.org/}
  % \end{itemize}
% \end{frame}


\begin{frame}
\frametitle{Why use Puzzles as Examples?}
\begin{itemize}
\item Easy to understand the problem
\item Solvable by hand without specialized knowledge
\item Possible to compare automated to manual solving process
\end{itemize}

\begin{quote}
The puzzle, though inanimate, is presented as a solvable problem without lasting negative consequences, a very low-risk low-reward situation. By being a puzzle, the object is attempting to convince the user that it must be completed.
\end{quote}
{\tiny Source: \href{https://www.everydayrhetoric.com/2020/03/the-rhetoric-of-puzzles/}{Every Day Rhetoric}}
\end{frame}


\mode<all>{
\part{Basic Constraint Propagation}
}
\begin{frame}
\partpage
\end{frame}
\mode<all>{
\input{../sendmore/body}
}

\mode<all>{
\part{Global Constraints}
}
\begin{frame}
\partpage
\end{frame}
\mode<all>{
\input{../sudoku/body}
}

\clearpage
\mode<all>{
\part{Customizing Search}
}
\begin{frame}
\partpage
\end{frame}
\mode<all>{
\input{../nqueen/body}
}


%% \clearpage
%% \mode<all>{
%%   \part{Problem Decomposition}
%% }
%% \begin{frame}
%%   \partpage
%% \end{frame}
%% \mode<all>{
%%   \input{../party/body}
%%   }

%% \clearpage
%% \mode<all>{
%%   \part{More Global Constraints}
%% }
%% \begin{frame}
%%   \partpage
%% \end{frame}
%% \mode<all>{
%%   \input{../car/body}
%%   }

%% \clearpage
%% \mode<all>{
%%   \part{Methodology}
%% }
%% \begin{frame}
%%   \partpage
%% \end{frame}
%% \mode<all>{
%%   \input{../methodology/body}
%%   }

\clearpage
\mode<all>{
\part{What is missing?}
}
\begin{frame}
\partpage
\end{frame}

\begin{frame}
\frametitle{Many Specialized Topics}
\begin{itemize}
\item How to design efficient core engine
\item Hybrids with LP/MIP tools
\item Hybrids with SAT
\item Symmetry breaking
\item Use of MDD/BDD to encode sets of solutions
\item High level modelling tools
\item Debugging/visualization
\end{itemize}
\end{frame}

\begin{frame}
\frametitle{Reformulation}
\begin{itemize}
\item Just because the user has modelled it this way, it doesn't mean we have to solve it that way
\begin{itemize}
\item Replace some constraint(s) by other, equivalent constraints
\item Because we don't have that constraint in our system
\item For performance
\end{itemize}
\end{itemize}
\end{frame}

\begin{frame}
\frametitle{Learning}
\begin{itemize}
\item While solving the problem we can learn how to strengthen the model/search 
\begin{itemize}
\item Understand which constraints/method contribute to propagation and change schedule
\item Learn no-good constraints by explaining failure
\item Adapt search strategy based on search experience
\end{itemize}
\end{itemize}
\end{frame}

%% \begin{frame}
%% \frametitle{Refined Process}
%% \begin{center}
%% \begin{tikzpicture}
%% \node[shape=rectangle,fill=pantone127-4] (problem) at (3,7) {Problem};
%% \node (human) at (3,6) {Human};
%% \node[shape=rectangle,fill=pantone127-4] (model) at (3,5) {User Model};
%% \node (reformulation) at (3,4) {Reformulation};
%% \node[shape=rectangle,fill=pantone127-4] (implementation) at (3,3) {Implementation Model};
%% \node (solver) at (3,2) {Constraint Solver/Search};
%% \node[shape=rectangle,fill=pantone157-8] (solution) at (3,1) {Solution};
%% \draw[-] (problem) -- (human);
%% \draw[->] (human) -- (model);
%% \draw[-] (model) -- (reformulation);
%% \draw[->] (reformulation) -- (implementation);
%% \draw[-] (implementation) -- (solver);
%% \draw[->] (solver) -- (0,2) -- (0,3) -- (implementation);
%% \draw[->] (solver) -- (solution);
%% \end{tikzpicture}
%% \end{center}
%% \end{frame}

%\begin{frame}
%\frametitle{What is CP actually used for?}
%\url{http://hsimonis.wordpress.com}
%\includegraphics[width=10cm]{../introduction/images/blog}
%\end{frame}

%% \begin{frame}
%% \frametitle{Exercises}
%% \begin{itemize}
%% \item Install Constraint Solver of your Choice
%% \item MiniZinc (MiningZinc) will be used in later sessions
%% \item Run Sendmoremoney, Sudoku, Queens examples
%% \item Can you choose which propagator to use for alldifferent?
%% \item Can you use different search strategies?
%% \end{itemize}
%% \end{frame}

\begin{frame}
\frametitle{More Learning Resources}
\begin{itemize}
\item Survey of Methods, Resources, and Formats for Teaching Constraint Programming
\begin{itemize}
\item by Tejas Santanam, Helmut Simonis
\item \url{https://doi.org/10.48550/arXiv.2403.12717}
\item Based on survey of community for WTCP 2023
\item \url{https://hsimonis.github.io/WTCP2023/}
\end{itemize}
\end{itemize}
\end{frame}


%%   }

\clearpage
\mode<all>{
\part{What is missing?}
}
\begin{frame}
\partpage
\end{frame}

\begin{frame}
\frametitle{Many Specialized Topics}
\begin{itemize}
\item How to design efficient core engine
\item Hybrids with LP/MIP tools
\item Hybrids with SAT
\item Symmetry breaking
\item Use of MDD/BDD to encode sets of solutions
\item High level modelling tools
\item Debugging/visualization
\end{itemize}
\end{frame}

\begin{frame}
\frametitle{Reformulation}
\begin{itemize}
\item Just because the user has modelled it this way, it doesn't mean we have to solve it that way
\begin{itemize}
\item Replace some constraint(s) by other, equivalent constraints
\item Because we don't have that constraint in our system
\item For performance
\end{itemize}
\end{itemize}
\end{frame}

\begin{frame}
\frametitle{Learning}
\begin{itemize}
\item While solving the problem we can learn how to strengthen the model/search 
\begin{itemize}
\item Understand which constraints/method contribute to propagation and change schedule
\item Learn no-good constraints by explaining failure
\item Adapt search strategy based on search experience
\end{itemize}
\end{itemize}
\end{frame}

%% \begin{frame}
%% \frametitle{Refined Process}
%% \begin{center}
%% \begin{tikzpicture}
%% \node[shape=rectangle,fill=pantone127-4] (problem) at (3,7) {Problem};
%% \node (human) at (3,6) {Human};
%% \node[shape=rectangle,fill=pantone127-4] (model) at (3,5) {User Model};
%% \node (reformulation) at (3,4) {Reformulation};
%% \node[shape=rectangle,fill=pantone127-4] (implementation) at (3,3) {Implementation Model};
%% \node (solver) at (3,2) {Constraint Solver/Search};
%% \node[shape=rectangle,fill=pantone157-8] (solution) at (3,1) {Solution};
%% \draw[-] (problem) -- (human);
%% \draw[->] (human) -- (model);
%% \draw[-] (model) -- (reformulation);
%% \draw[->] (reformulation) -- (implementation);
%% \draw[-] (implementation) -- (solver);
%% \draw[->] (solver) -- (0,2) -- (0,3) -- (implementation);
%% \draw[->] (solver) -- (solution);
%% \end{tikzpicture}
%% \end{center}
%% \end{frame}

%\begin{frame}
%\frametitle{What is CP actually used for?}
%\url{http://hsimonis.wordpress.com}
%\includegraphics[width=10cm]{../introduction/images/blog}
%\end{frame}

%% \begin{frame}
%% \frametitle{Exercises}
%% \begin{itemize}
%% \item Install Constraint Solver of your Choice
%% \item MiniZinc (MiningZinc) will be used in later sessions
%% \item Run Sendmoremoney, Sudoku, Queens examples
%% \item Can you choose which propagator to use for alldifferent?
%% \item Can you use different search strategies?
%% \end{itemize}
%% \end{frame}

\begin{frame}
\frametitle{More Learning Resources}
\begin{itemize}
\item Survey of Methods, Resources, and Formats for Teaching Constraint Programming
\begin{itemize}
\item by Tejas Santanam, Helmut Simonis
\item \url{https://doi.org/10.48550/arXiv.2403.12717}
\item Based on survey of community for WTCP 2023
\item \url{https://hsimonis.github.io/WTCP2023/}
\end{itemize}
\end{itemize}
\end{frame}


%%   }

%% \clearpage
%% \mode<all>{
%%   \part{More Global Constraints}
%% }
%% \begin{frame}
%%   \partpage
%% \end{frame}
%% \mode<all>{
%%   \begin{frame}
\frametitle{Objectives}
\begin{itemize}
\item Overview of Core Constraint Programming
\item Three Main Concepts
\begin{itemize}
\item Constraint Propagation
\item Global Constraints
\item Customizing Search
\end{itemize}
%\item Get Some Experience with MiniZinc
\item Topics will be treated in more detail in later parts of the school
\item Based on Examples, not Formal Description
\end{itemize}
\end{frame}

\begin{frame}
\frametitle{Outline}
\begin{itemize}
\item Why Constraint Programming?
\item Constraint Propagation
\item Global Constraints
\item Customizing Search
%\item Choices in Modelling
%\item Decomposition
%\item More Global Constraints
%\item Methodology
%\item What is missing?
\end{itemize}
\end{frame}

% \begin{frame}
  % \frametitle{Using MiniZinc IDE}
  % \begin{itemize}
  % \item Developed in the Australian NICTA project
  % \item Maintained by Monash University
  % \item Modelling tool with multiple back-end solvers
  % \item Available from \url{https://www.minizinc.org/}
  % \end{itemize}
% \end{frame}


% \begin{frame}
% \frametitle{Examples in ECLiPSe}
% \begin{itemize}
% \item Open sourced constraint programming language
% \item Development goes back to 1985
% \item ECRC, ICL, IC-Parc, PTL, Cisco
% \item \url{https://eclipseclp.org/}
% \item Specialities
% \begin{itemize}
% \item Develop new solvers for specific domains
% \item Integration with MIP
% \end{itemize}
% \item Not included in bundled MiniZinc IDE
% \item Specialized visualization tools used here
  % \begin{itemize}
  % \item CP-Viz, Simonis et al. 2010
    % \end{itemize}
% \end{itemize}
% \end{frame}

\begin{frame}
\frametitle{Tutorial Based on ECLiPSe ELearning Course}
\begin{itemize}
\item Self-study course in constraint programming
\item Supported by Cisco Systems and Silicon Valley Community Foundation
\item Multi-media format, video lectures, slides, handout etc
\item {\small \url{https://eclipseclp.org/ELearning/index.html}}
\end{itemize}
\end{frame}

\begin{frame}
\frametitle{Also Part of CRT-AI Constraint Week}
\begin{itemize}
\item Annual one week course on CP and Optimization in Ireland
\item Part of national training program for PhD students in AI
\item \url{https://www.crt-ai.ie/}
\end{itemize}
\end{frame}



\begin{frame}
\frametitle{Constraint Programming - in a nutshell}
\begin{itemize}
\item Declarative description of problems with
\begin{itemize}
\item {\em Variables} which range over (finite) sets of values
\item {\em Constraints} over subsets of variables which restrict possible value combinations
\item A {\em solution} is a value assignment which satisfies all constraints
\end{itemize}

\item Constraint propagation/reasoning
\begin{itemize}
\item Removing inconsistent values for variables
\item Detect failure if constraint can not be satisfied
\item Interaction of constraints via shared variables
\item Incomplete
\end{itemize}

\item Search
\begin{itemize}
\item User controlled assignment of values to variables
\item Each step triggers constraint propagation 
\end{itemize}
\item Different domains require/allow different methods
\end{itemize}
\end{frame}

\begin{frame}
  \frametitle{Constraint Programming is Different}
  \begin{itemize}
  \item Declarative Programming
    \begin{itemize}
    \item Concentrate on what you want
      \item Not how to get there
      \item Program != Algorithm
      \item Program = Model
    \end{itemize}
    \item Applied to Combinatorial Problems
      \begin{itemize}
        \item No complete polynomial algorithms known (exist?)
        \item CP less ad-hoc than heuristics
        \item Models can evolve
  \end{itemize}
  \end{itemize}
  \end{frame}
    
\begin{frame}
  \frametitle{A Subtractive Process}
  \begin{textblock}{4}(8,-3)
    \includegraphics[width=4cm]{../introduction/images/stages}
  \end{textblock}
  \vfill
  \begin{quote}
    ``Oh, bosh, as Mr. Ruskin says. Sculpture, per se, is the simplest thing in the world. All you have to do is to take a big chunk of marble and a hammer and chisel, make up your mind what you are about to create and chip off all the marble you don't want.''-Paris Gaulois.
  \end{quote}
  
  {\tiny Source: \url{https://quoteinvestigator.com/2014/06/22/chip-away/}}
\end{frame}

\begin{frame}
\frametitle{Basic Process}
\begin{center}
\begin{tikzpicture}
\node[shape=rectangle,fill=pantone127-4] (problem) at (2,5) {Problem};
\node (human) at (2,4) {Human};
\node[shape=rectangle,fill=pantone127-4] (model) at (2,3) {Model};
\node (solver) at (2,2) {Constraint Solver/Search};
\node[shape=rectangle,fill=pantone157-8] (solution) at (2,1) {Solution};
\draw[-] (problem) -- (human);
\draw[->] (human) -- (model);
\draw[-] (model) -- (solver);
\draw[->] (solver) -- (solution);
\end{tikzpicture}
\end{center}
\end{frame}

\begin{frame}
\frametitle{More Realistic}
\begin{center}
\begin{tikzpicture}[xscale=4.0]
\node[shape=rectangle,fill=pantone127-4] (problem) at (2,5) {Problem};
\node (human) at (2,4) {Human};
\node[shape=rectangle,fill=pantone127-4] (model) at (2,3) {Model};
\node (solver) at (2,2) {Constraint Solver/Search};
\node[shape=rectangle,fill=pantone24-8] (hangs) at (1,1) {Hangs};
\node[shape=rectangle,fill=pantone157-8] (solution) at (2,1) {Solution};
\node[shape=rectangle,fill=pantone24-8] (wrong) at (3,1) {Wrong Solution};
\draw[-] (problem) -- (human);
\draw[->] (human) -- (model);
\draw[-] (model) -- (solver);
\draw[->] (solver) -- (hangs);
\draw[->] (solver) -- (solution);
\draw[->] (solver) -- (wrong);
\draw[->] (hangs) -- (1,4) -- (human);
\draw[->] (wrong) -- (3,4) -- (human);
\end{tikzpicture}
\end{center}
\end{frame}

\begin{frame}
\frametitle{Dual Role of Model}
\begin{itemize}
\item Allows Human to Express Problem
\begin{itemize}
\item Close to Problem Domain
\item Constraints as Abstractions
\end{itemize}
\item Allows Solver to Execute
\begin{itemize}
\item Variables as Communication Mechanism
\item Constraints as Algorithms
\end{itemize}
\end{itemize}
\end{frame}

\begin{frame}
\frametitle{Modelling Frameworks}
\begin{itemize}
\item \href{https://www.minizinc.org/}{MiniZinc} (NICTA, Monash University, Australia)
\item \href{https://github.com/eomahony/Numberjack}{NumberJack} (Insight, Ireland) 
\item \href{https://www-users.york.ac.uk/peter.nightingale/savilerow//}{EssencePrime/SavilleRow} (UK) 
\item \href{https://cpmpy.readthedocs.io/en/latest/}{CPMpy} (KU Leuven)
\item Allow use of multiple back-end solvers
\item Compile model into variants for each solver
\item A priori solver independent model(CP, MIP, SAT)
\end{itemize}
\end{frame} 

\begin{frame}
\frametitle{Framework Process}
\begin{center}
\begin{tikzpicture}[xscale=2]
\node[shape=rectangle,fill=pantone127-4] (problem) at (2.5,5) {Problem};
\node (human) at (2.5,4) {Human};
\node[shape=rectangle,fill=pantone127-4] (model) at (2.5,3) {Model};
\node (compile) at (2.5,2) {Compile/Reformulate};
\node (solver1) at (1,1) {CP};
\node (solver2) at (2,1) {MIP};
\node (solver3) at (3,1) {SAT};
\node (solver4) at (4,1) {Other};
\node[shape=rectangle,fill=pantone157-8] (solution1) at (1,0) {Solution};
\node[shape=rectangle,fill=pantone157-8] (solution2) at (2,0) {Solution};
\node[shape=rectangle,fill=pantone157-8] (solution3) at (3,0) {Solution};
\node[shape=rectangle,fill=pantone157-8] (solution4) at (4,0) {Solution};
\draw[-] (problem) -- (human);
\draw[->] (human) -- (model);
\draw[-] (model) -- (compile);
\draw[-] (compile) -- (solver1);
\draw[-] (compile) -- (solver2);
\draw[-] (compile) -- (solver3);
\draw[-] (compile) -- (solver4);
\draw[->] (solver1) -- (solution1);
\draw[->] (solver2) -- (solution2);
\draw[->] (solver3) -- (solution3);
\draw[->] (solver4) -- (solution4);
\end{tikzpicture}
\end{center}
\end{frame}

% \begin{frame}
  % \frametitle{Do It Now!}
  % \begin{itemize}
  % \item Download and install Minizinc
  % \item \url{https://www.minizinc.org/}
  % \end{itemize}
% \end{frame}


\begin{frame}
\frametitle{Why use Puzzles as Examples?}
\begin{itemize}
\item Easy to understand the problem
\item Solvable by hand without specialized knowledge
\item Possible to compare automated to manual solving process
\end{itemize}

\begin{quote}
The puzzle, though inanimate, is presented as a solvable problem without lasting negative consequences, a very low-risk low-reward situation. By being a puzzle, the object is attempting to convince the user that it must be completed.
\end{quote}
{\tiny Source: \href{https://www.everydayrhetoric.com/2020/03/the-rhetoric-of-puzzles/}{Every Day Rhetoric}}
\end{frame}


\mode<all>{
\part{Basic Constraint Propagation}
}
\begin{frame}
\partpage
\end{frame}
\mode<all>{
\begin{frame}
\frametitle{Objectives}
\begin{itemize}
\item Overview of Core Constraint Programming
\item Three Main Concepts
\begin{itemize}
\item Constraint Propagation
\item Global Constraints
\item Customizing Search
\end{itemize}
%\item Get Some Experience with MiniZinc
\item Topics will be treated in more detail in later parts of the school
\item Based on Examples, not Formal Description
\end{itemize}
\end{frame}

\begin{frame}
\frametitle{Outline}
\begin{itemize}
\item Why Constraint Programming?
\item Constraint Propagation
\item Global Constraints
\item Customizing Search
%\item Choices in Modelling
%\item Decomposition
%\item More Global Constraints
%\item Methodology
%\item What is missing?
\end{itemize}
\end{frame}

% \begin{frame}
  % \frametitle{Using MiniZinc IDE}
  % \begin{itemize}
  % \item Developed in the Australian NICTA project
  % \item Maintained by Monash University
  % \item Modelling tool with multiple back-end solvers
  % \item Available from \url{https://www.minizinc.org/}
  % \end{itemize}
% \end{frame}


% \begin{frame}
% \frametitle{Examples in ECLiPSe}
% \begin{itemize}
% \item Open sourced constraint programming language
% \item Development goes back to 1985
% \item ECRC, ICL, IC-Parc, PTL, Cisco
% \item \url{https://eclipseclp.org/}
% \item Specialities
% \begin{itemize}
% \item Develop new solvers for specific domains
% \item Integration with MIP
% \end{itemize}
% \item Not included in bundled MiniZinc IDE
% \item Specialized visualization tools used here
  % \begin{itemize}
  % \item CP-Viz, Simonis et al. 2010
    % \end{itemize}
% \end{itemize}
% \end{frame}

\begin{frame}
\frametitle{Tutorial Based on ECLiPSe ELearning Course}
\begin{itemize}
\item Self-study course in constraint programming
\item Supported by Cisco Systems and Silicon Valley Community Foundation
\item Multi-media format, video lectures, slides, handout etc
\item {\small \url{https://eclipseclp.org/ELearning/index.html}}
\end{itemize}
\end{frame}

\begin{frame}
\frametitle{Also Part of CRT-AI Constraint Week}
\begin{itemize}
\item Annual one week course on CP and Optimization in Ireland
\item Part of national training program for PhD students in AI
\item \url{https://www.crt-ai.ie/}
\end{itemize}
\end{frame}



\begin{frame}
\frametitle{Constraint Programming - in a nutshell}
\begin{itemize}
\item Declarative description of problems with
\begin{itemize}
\item {\em Variables} which range over (finite) sets of values
\item {\em Constraints} over subsets of variables which restrict possible value combinations
\item A {\em solution} is a value assignment which satisfies all constraints
\end{itemize}

\item Constraint propagation/reasoning
\begin{itemize}
\item Removing inconsistent values for variables
\item Detect failure if constraint can not be satisfied
\item Interaction of constraints via shared variables
\item Incomplete
\end{itemize}

\item Search
\begin{itemize}
\item User controlled assignment of values to variables
\item Each step triggers constraint propagation 
\end{itemize}
\item Different domains require/allow different methods
\end{itemize}
\end{frame}

\begin{frame}
  \frametitle{Constraint Programming is Different}
  \begin{itemize}
  \item Declarative Programming
    \begin{itemize}
    \item Concentrate on what you want
      \item Not how to get there
      \item Program != Algorithm
      \item Program = Model
    \end{itemize}
    \item Applied to Combinatorial Problems
      \begin{itemize}
        \item No complete polynomial algorithms known (exist?)
        \item CP less ad-hoc than heuristics
        \item Models can evolve
  \end{itemize}
  \end{itemize}
  \end{frame}
    
\begin{frame}
  \frametitle{A Subtractive Process}
  \begin{textblock}{4}(8,-3)
    \includegraphics[width=4cm]{../introduction/images/stages}
  \end{textblock}
  \vfill
  \begin{quote}
    ``Oh, bosh, as Mr. Ruskin says. Sculpture, per se, is the simplest thing in the world. All you have to do is to take a big chunk of marble and a hammer and chisel, make up your mind what you are about to create and chip off all the marble you don't want.''-Paris Gaulois.
  \end{quote}
  
  {\tiny Source: \url{https://quoteinvestigator.com/2014/06/22/chip-away/}}
\end{frame}

\begin{frame}
\frametitle{Basic Process}
\begin{center}
\begin{tikzpicture}
\node[shape=rectangle,fill=pantone127-4] (problem) at (2,5) {Problem};
\node (human) at (2,4) {Human};
\node[shape=rectangle,fill=pantone127-4] (model) at (2,3) {Model};
\node (solver) at (2,2) {Constraint Solver/Search};
\node[shape=rectangle,fill=pantone157-8] (solution) at (2,1) {Solution};
\draw[-] (problem) -- (human);
\draw[->] (human) -- (model);
\draw[-] (model) -- (solver);
\draw[->] (solver) -- (solution);
\end{tikzpicture}
\end{center}
\end{frame}

\begin{frame}
\frametitle{More Realistic}
\begin{center}
\begin{tikzpicture}[xscale=4.0]
\node[shape=rectangle,fill=pantone127-4] (problem) at (2,5) {Problem};
\node (human) at (2,4) {Human};
\node[shape=rectangle,fill=pantone127-4] (model) at (2,3) {Model};
\node (solver) at (2,2) {Constraint Solver/Search};
\node[shape=rectangle,fill=pantone24-8] (hangs) at (1,1) {Hangs};
\node[shape=rectangle,fill=pantone157-8] (solution) at (2,1) {Solution};
\node[shape=rectangle,fill=pantone24-8] (wrong) at (3,1) {Wrong Solution};
\draw[-] (problem) -- (human);
\draw[->] (human) -- (model);
\draw[-] (model) -- (solver);
\draw[->] (solver) -- (hangs);
\draw[->] (solver) -- (solution);
\draw[->] (solver) -- (wrong);
\draw[->] (hangs) -- (1,4) -- (human);
\draw[->] (wrong) -- (3,4) -- (human);
\end{tikzpicture}
\end{center}
\end{frame}

\begin{frame}
\frametitle{Dual Role of Model}
\begin{itemize}
\item Allows Human to Express Problem
\begin{itemize}
\item Close to Problem Domain
\item Constraints as Abstractions
\end{itemize}
\item Allows Solver to Execute
\begin{itemize}
\item Variables as Communication Mechanism
\item Constraints as Algorithms
\end{itemize}
\end{itemize}
\end{frame}

\begin{frame}
\frametitle{Modelling Frameworks}
\begin{itemize}
\item \href{https://www.minizinc.org/}{MiniZinc} (NICTA, Monash University, Australia)
\item \href{https://github.com/eomahony/Numberjack}{NumberJack} (Insight, Ireland) 
\item \href{https://www-users.york.ac.uk/peter.nightingale/savilerow//}{EssencePrime/SavilleRow} (UK) 
\item \href{https://cpmpy.readthedocs.io/en/latest/}{CPMpy} (KU Leuven)
\item Allow use of multiple back-end solvers
\item Compile model into variants for each solver
\item A priori solver independent model(CP, MIP, SAT)
\end{itemize}
\end{frame} 

\begin{frame}
\frametitle{Framework Process}
\begin{center}
\begin{tikzpicture}[xscale=2]
\node[shape=rectangle,fill=pantone127-4] (problem) at (2.5,5) {Problem};
\node (human) at (2.5,4) {Human};
\node[shape=rectangle,fill=pantone127-4] (model) at (2.5,3) {Model};
\node (compile) at (2.5,2) {Compile/Reformulate};
\node (solver1) at (1,1) {CP};
\node (solver2) at (2,1) {MIP};
\node (solver3) at (3,1) {SAT};
\node (solver4) at (4,1) {Other};
\node[shape=rectangle,fill=pantone157-8] (solution1) at (1,0) {Solution};
\node[shape=rectangle,fill=pantone157-8] (solution2) at (2,0) {Solution};
\node[shape=rectangle,fill=pantone157-8] (solution3) at (3,0) {Solution};
\node[shape=rectangle,fill=pantone157-8] (solution4) at (4,0) {Solution};
\draw[-] (problem) -- (human);
\draw[->] (human) -- (model);
\draw[-] (model) -- (compile);
\draw[-] (compile) -- (solver1);
\draw[-] (compile) -- (solver2);
\draw[-] (compile) -- (solver3);
\draw[-] (compile) -- (solver4);
\draw[->] (solver1) -- (solution1);
\draw[->] (solver2) -- (solution2);
\draw[->] (solver3) -- (solution3);
\draw[->] (solver4) -- (solution4);
\end{tikzpicture}
\end{center}
\end{frame}

% \begin{frame}
  % \frametitle{Do It Now!}
  % \begin{itemize}
  % \item Download and install Minizinc
  % \item \url{https://www.minizinc.org/}
  % \end{itemize}
% \end{frame}


\begin{frame}
\frametitle{Why use Puzzles as Examples?}
\begin{itemize}
\item Easy to understand the problem
\item Solvable by hand without specialized knowledge
\item Possible to compare automated to manual solving process
\end{itemize}

\begin{quote}
The puzzle, though inanimate, is presented as a solvable problem without lasting negative consequences, a very low-risk low-reward situation. By being a puzzle, the object is attempting to convince the user that it must be completed.
\end{quote}
{\tiny Source: \href{https://www.everydayrhetoric.com/2020/03/the-rhetoric-of-puzzles/}{Every Day Rhetoric}}
\end{frame}


\mode<all>{
\part{Basic Constraint Propagation}
}
\begin{frame}
\partpage
\end{frame}
\mode<all>{
\input{../sendmore/body}
}

\mode<all>{
\part{Global Constraints}
}
\begin{frame}
\partpage
\end{frame}
\mode<all>{
\input{../sudoku/body}
}

\clearpage
\mode<all>{
\part{Customizing Search}
}
\begin{frame}
\partpage
\end{frame}
\mode<all>{
\input{../nqueen/body}
}


%% \clearpage
%% \mode<all>{
%%   \part{Problem Decomposition}
%% }
%% \begin{frame}
%%   \partpage
%% \end{frame}
%% \mode<all>{
%%   \input{../party/body}
%%   }

%% \clearpage
%% \mode<all>{
%%   \part{More Global Constraints}
%% }
%% \begin{frame}
%%   \partpage
%% \end{frame}
%% \mode<all>{
%%   \input{../car/body}
%%   }

%% \clearpage
%% \mode<all>{
%%   \part{Methodology}
%% }
%% \begin{frame}
%%   \partpage
%% \end{frame}
%% \mode<all>{
%%   \input{../methodology/body}
%%   }

\clearpage
\mode<all>{
\part{What is missing?}
}
\begin{frame}
\partpage
\end{frame}

\begin{frame}
\frametitle{Many Specialized Topics}
\begin{itemize}
\item How to design efficient core engine
\item Hybrids with LP/MIP tools
\item Hybrids with SAT
\item Symmetry breaking
\item Use of MDD/BDD to encode sets of solutions
\item High level modelling tools
\item Debugging/visualization
\end{itemize}
\end{frame}

\begin{frame}
\frametitle{Reformulation}
\begin{itemize}
\item Just because the user has modelled it this way, it doesn't mean we have to solve it that way
\begin{itemize}
\item Replace some constraint(s) by other, equivalent constraints
\item Because we don't have that constraint in our system
\item For performance
\end{itemize}
\end{itemize}
\end{frame}

\begin{frame}
\frametitle{Learning}
\begin{itemize}
\item While solving the problem we can learn how to strengthen the model/search 
\begin{itemize}
\item Understand which constraints/method contribute to propagation and change schedule
\item Learn no-good constraints by explaining failure
\item Adapt search strategy based on search experience
\end{itemize}
\end{itemize}
\end{frame}

%% \begin{frame}
%% \frametitle{Refined Process}
%% \begin{center}
%% \begin{tikzpicture}
%% \node[shape=rectangle,fill=pantone127-4] (problem) at (3,7) {Problem};
%% \node (human) at (3,6) {Human};
%% \node[shape=rectangle,fill=pantone127-4] (model) at (3,5) {User Model};
%% \node (reformulation) at (3,4) {Reformulation};
%% \node[shape=rectangle,fill=pantone127-4] (implementation) at (3,3) {Implementation Model};
%% \node (solver) at (3,2) {Constraint Solver/Search};
%% \node[shape=rectangle,fill=pantone157-8] (solution) at (3,1) {Solution};
%% \draw[-] (problem) -- (human);
%% \draw[->] (human) -- (model);
%% \draw[-] (model) -- (reformulation);
%% \draw[->] (reformulation) -- (implementation);
%% \draw[-] (implementation) -- (solver);
%% \draw[->] (solver) -- (0,2) -- (0,3) -- (implementation);
%% \draw[->] (solver) -- (solution);
%% \end{tikzpicture}
%% \end{center}
%% \end{frame}

%\begin{frame}
%\frametitle{What is CP actually used for?}
%\url{http://hsimonis.wordpress.com}
%\includegraphics[width=10cm]{../introduction/images/blog}
%\end{frame}

%% \begin{frame}
%% \frametitle{Exercises}
%% \begin{itemize}
%% \item Install Constraint Solver of your Choice
%% \item MiniZinc (MiningZinc) will be used in later sessions
%% \item Run Sendmoremoney, Sudoku, Queens examples
%% \item Can you choose which propagator to use for alldifferent?
%% \item Can you use different search strategies?
%% \end{itemize}
%% \end{frame}

\begin{frame}
\frametitle{More Learning Resources}
\begin{itemize}
\item Survey of Methods, Resources, and Formats for Teaching Constraint Programming
\begin{itemize}
\item by Tejas Santanam, Helmut Simonis
\item \url{https://doi.org/10.48550/arXiv.2403.12717}
\item Based on survey of community for WTCP 2023
\item \url{https://hsimonis.github.io/WTCP2023/}
\end{itemize}
\end{itemize}
\end{frame}


}

\mode<all>{
\part{Global Constraints}
}
\begin{frame}
\partpage
\end{frame}
\mode<all>{
\begin{frame}
\frametitle{Objectives}
\begin{itemize}
\item Overview of Core Constraint Programming
\item Three Main Concepts
\begin{itemize}
\item Constraint Propagation
\item Global Constraints
\item Customizing Search
\end{itemize}
%\item Get Some Experience with MiniZinc
\item Topics will be treated in more detail in later parts of the school
\item Based on Examples, not Formal Description
\end{itemize}
\end{frame}

\begin{frame}
\frametitle{Outline}
\begin{itemize}
\item Why Constraint Programming?
\item Constraint Propagation
\item Global Constraints
\item Customizing Search
%\item Choices in Modelling
%\item Decomposition
%\item More Global Constraints
%\item Methodology
%\item What is missing?
\end{itemize}
\end{frame}

% \begin{frame}
  % \frametitle{Using MiniZinc IDE}
  % \begin{itemize}
  % \item Developed in the Australian NICTA project
  % \item Maintained by Monash University
  % \item Modelling tool with multiple back-end solvers
  % \item Available from \url{https://www.minizinc.org/}
  % \end{itemize}
% \end{frame}


% \begin{frame}
% \frametitle{Examples in ECLiPSe}
% \begin{itemize}
% \item Open sourced constraint programming language
% \item Development goes back to 1985
% \item ECRC, ICL, IC-Parc, PTL, Cisco
% \item \url{https://eclipseclp.org/}
% \item Specialities
% \begin{itemize}
% \item Develop new solvers for specific domains
% \item Integration with MIP
% \end{itemize}
% \item Not included in bundled MiniZinc IDE
% \item Specialized visualization tools used here
  % \begin{itemize}
  % \item CP-Viz, Simonis et al. 2010
    % \end{itemize}
% \end{itemize}
% \end{frame}

\begin{frame}
\frametitle{Tutorial Based on ECLiPSe ELearning Course}
\begin{itemize}
\item Self-study course in constraint programming
\item Supported by Cisco Systems and Silicon Valley Community Foundation
\item Multi-media format, video lectures, slides, handout etc
\item {\small \url{https://eclipseclp.org/ELearning/index.html}}
\end{itemize}
\end{frame}

\begin{frame}
\frametitle{Also Part of CRT-AI Constraint Week}
\begin{itemize}
\item Annual one week course on CP and Optimization in Ireland
\item Part of national training program for PhD students in AI
\item \url{https://www.crt-ai.ie/}
\end{itemize}
\end{frame}



\begin{frame}
\frametitle{Constraint Programming - in a nutshell}
\begin{itemize}
\item Declarative description of problems with
\begin{itemize}
\item {\em Variables} which range over (finite) sets of values
\item {\em Constraints} over subsets of variables which restrict possible value combinations
\item A {\em solution} is a value assignment which satisfies all constraints
\end{itemize}

\item Constraint propagation/reasoning
\begin{itemize}
\item Removing inconsistent values for variables
\item Detect failure if constraint can not be satisfied
\item Interaction of constraints via shared variables
\item Incomplete
\end{itemize}

\item Search
\begin{itemize}
\item User controlled assignment of values to variables
\item Each step triggers constraint propagation 
\end{itemize}
\item Different domains require/allow different methods
\end{itemize}
\end{frame}

\begin{frame}
  \frametitle{Constraint Programming is Different}
  \begin{itemize}
  \item Declarative Programming
    \begin{itemize}
    \item Concentrate on what you want
      \item Not how to get there
      \item Program != Algorithm
      \item Program = Model
    \end{itemize}
    \item Applied to Combinatorial Problems
      \begin{itemize}
        \item No complete polynomial algorithms known (exist?)
        \item CP less ad-hoc than heuristics
        \item Models can evolve
  \end{itemize}
  \end{itemize}
  \end{frame}
    
\begin{frame}
  \frametitle{A Subtractive Process}
  \begin{textblock}{4}(8,-3)
    \includegraphics[width=4cm]{../introduction/images/stages}
  \end{textblock}
  \vfill
  \begin{quote}
    ``Oh, bosh, as Mr. Ruskin says. Sculpture, per se, is the simplest thing in the world. All you have to do is to take a big chunk of marble and a hammer and chisel, make up your mind what you are about to create and chip off all the marble you don't want.''-Paris Gaulois.
  \end{quote}
  
  {\tiny Source: \url{https://quoteinvestigator.com/2014/06/22/chip-away/}}
\end{frame}

\begin{frame}
\frametitle{Basic Process}
\begin{center}
\begin{tikzpicture}
\node[shape=rectangle,fill=pantone127-4] (problem) at (2,5) {Problem};
\node (human) at (2,4) {Human};
\node[shape=rectangle,fill=pantone127-4] (model) at (2,3) {Model};
\node (solver) at (2,2) {Constraint Solver/Search};
\node[shape=rectangle,fill=pantone157-8] (solution) at (2,1) {Solution};
\draw[-] (problem) -- (human);
\draw[->] (human) -- (model);
\draw[-] (model) -- (solver);
\draw[->] (solver) -- (solution);
\end{tikzpicture}
\end{center}
\end{frame}

\begin{frame}
\frametitle{More Realistic}
\begin{center}
\begin{tikzpicture}[xscale=4.0]
\node[shape=rectangle,fill=pantone127-4] (problem) at (2,5) {Problem};
\node (human) at (2,4) {Human};
\node[shape=rectangle,fill=pantone127-4] (model) at (2,3) {Model};
\node (solver) at (2,2) {Constraint Solver/Search};
\node[shape=rectangle,fill=pantone24-8] (hangs) at (1,1) {Hangs};
\node[shape=rectangle,fill=pantone157-8] (solution) at (2,1) {Solution};
\node[shape=rectangle,fill=pantone24-8] (wrong) at (3,1) {Wrong Solution};
\draw[-] (problem) -- (human);
\draw[->] (human) -- (model);
\draw[-] (model) -- (solver);
\draw[->] (solver) -- (hangs);
\draw[->] (solver) -- (solution);
\draw[->] (solver) -- (wrong);
\draw[->] (hangs) -- (1,4) -- (human);
\draw[->] (wrong) -- (3,4) -- (human);
\end{tikzpicture}
\end{center}
\end{frame}

\begin{frame}
\frametitle{Dual Role of Model}
\begin{itemize}
\item Allows Human to Express Problem
\begin{itemize}
\item Close to Problem Domain
\item Constraints as Abstractions
\end{itemize}
\item Allows Solver to Execute
\begin{itemize}
\item Variables as Communication Mechanism
\item Constraints as Algorithms
\end{itemize}
\end{itemize}
\end{frame}

\begin{frame}
\frametitle{Modelling Frameworks}
\begin{itemize}
\item \href{https://www.minizinc.org/}{MiniZinc} (NICTA, Monash University, Australia)
\item \href{https://github.com/eomahony/Numberjack}{NumberJack} (Insight, Ireland) 
\item \href{https://www-users.york.ac.uk/peter.nightingale/savilerow//}{EssencePrime/SavilleRow} (UK) 
\item \href{https://cpmpy.readthedocs.io/en/latest/}{CPMpy} (KU Leuven)
\item Allow use of multiple back-end solvers
\item Compile model into variants for each solver
\item A priori solver independent model(CP, MIP, SAT)
\end{itemize}
\end{frame} 

\begin{frame}
\frametitle{Framework Process}
\begin{center}
\begin{tikzpicture}[xscale=2]
\node[shape=rectangle,fill=pantone127-4] (problem) at (2.5,5) {Problem};
\node (human) at (2.5,4) {Human};
\node[shape=rectangle,fill=pantone127-4] (model) at (2.5,3) {Model};
\node (compile) at (2.5,2) {Compile/Reformulate};
\node (solver1) at (1,1) {CP};
\node (solver2) at (2,1) {MIP};
\node (solver3) at (3,1) {SAT};
\node (solver4) at (4,1) {Other};
\node[shape=rectangle,fill=pantone157-8] (solution1) at (1,0) {Solution};
\node[shape=rectangle,fill=pantone157-8] (solution2) at (2,0) {Solution};
\node[shape=rectangle,fill=pantone157-8] (solution3) at (3,0) {Solution};
\node[shape=rectangle,fill=pantone157-8] (solution4) at (4,0) {Solution};
\draw[-] (problem) -- (human);
\draw[->] (human) -- (model);
\draw[-] (model) -- (compile);
\draw[-] (compile) -- (solver1);
\draw[-] (compile) -- (solver2);
\draw[-] (compile) -- (solver3);
\draw[-] (compile) -- (solver4);
\draw[->] (solver1) -- (solution1);
\draw[->] (solver2) -- (solution2);
\draw[->] (solver3) -- (solution3);
\draw[->] (solver4) -- (solution4);
\end{tikzpicture}
\end{center}
\end{frame}

% \begin{frame}
  % \frametitle{Do It Now!}
  % \begin{itemize}
  % \item Download and install Minizinc
  % \item \url{https://www.minizinc.org/}
  % \end{itemize}
% \end{frame}


\begin{frame}
\frametitle{Why use Puzzles as Examples?}
\begin{itemize}
\item Easy to understand the problem
\item Solvable by hand without specialized knowledge
\item Possible to compare automated to manual solving process
\end{itemize}

\begin{quote}
The puzzle, though inanimate, is presented as a solvable problem without lasting negative consequences, a very low-risk low-reward situation. By being a puzzle, the object is attempting to convince the user that it must be completed.
\end{quote}
{\tiny Source: \href{https://www.everydayrhetoric.com/2020/03/the-rhetoric-of-puzzles/}{Every Day Rhetoric}}
\end{frame}


\mode<all>{
\part{Basic Constraint Propagation}
}
\begin{frame}
\partpage
\end{frame}
\mode<all>{
\input{../sendmore/body}
}

\mode<all>{
\part{Global Constraints}
}
\begin{frame}
\partpage
\end{frame}
\mode<all>{
\input{../sudoku/body}
}

\clearpage
\mode<all>{
\part{Customizing Search}
}
\begin{frame}
\partpage
\end{frame}
\mode<all>{
\input{../nqueen/body}
}


%% \clearpage
%% \mode<all>{
%%   \part{Problem Decomposition}
%% }
%% \begin{frame}
%%   \partpage
%% \end{frame}
%% \mode<all>{
%%   \input{../party/body}
%%   }

%% \clearpage
%% \mode<all>{
%%   \part{More Global Constraints}
%% }
%% \begin{frame}
%%   \partpage
%% \end{frame}
%% \mode<all>{
%%   \input{../car/body}
%%   }

%% \clearpage
%% \mode<all>{
%%   \part{Methodology}
%% }
%% \begin{frame}
%%   \partpage
%% \end{frame}
%% \mode<all>{
%%   \input{../methodology/body}
%%   }

\clearpage
\mode<all>{
\part{What is missing?}
}
\begin{frame}
\partpage
\end{frame}

\begin{frame}
\frametitle{Many Specialized Topics}
\begin{itemize}
\item How to design efficient core engine
\item Hybrids with LP/MIP tools
\item Hybrids with SAT
\item Symmetry breaking
\item Use of MDD/BDD to encode sets of solutions
\item High level modelling tools
\item Debugging/visualization
\end{itemize}
\end{frame}

\begin{frame}
\frametitle{Reformulation}
\begin{itemize}
\item Just because the user has modelled it this way, it doesn't mean we have to solve it that way
\begin{itemize}
\item Replace some constraint(s) by other, equivalent constraints
\item Because we don't have that constraint in our system
\item For performance
\end{itemize}
\end{itemize}
\end{frame}

\begin{frame}
\frametitle{Learning}
\begin{itemize}
\item While solving the problem we can learn how to strengthen the model/search 
\begin{itemize}
\item Understand which constraints/method contribute to propagation and change schedule
\item Learn no-good constraints by explaining failure
\item Adapt search strategy based on search experience
\end{itemize}
\end{itemize}
\end{frame}

%% \begin{frame}
%% \frametitle{Refined Process}
%% \begin{center}
%% \begin{tikzpicture}
%% \node[shape=rectangle,fill=pantone127-4] (problem) at (3,7) {Problem};
%% \node (human) at (3,6) {Human};
%% \node[shape=rectangle,fill=pantone127-4] (model) at (3,5) {User Model};
%% \node (reformulation) at (3,4) {Reformulation};
%% \node[shape=rectangle,fill=pantone127-4] (implementation) at (3,3) {Implementation Model};
%% \node (solver) at (3,2) {Constraint Solver/Search};
%% \node[shape=rectangle,fill=pantone157-8] (solution) at (3,1) {Solution};
%% \draw[-] (problem) -- (human);
%% \draw[->] (human) -- (model);
%% \draw[-] (model) -- (reformulation);
%% \draw[->] (reformulation) -- (implementation);
%% \draw[-] (implementation) -- (solver);
%% \draw[->] (solver) -- (0,2) -- (0,3) -- (implementation);
%% \draw[->] (solver) -- (solution);
%% \end{tikzpicture}
%% \end{center}
%% \end{frame}

%\begin{frame}
%\frametitle{What is CP actually used for?}
%\url{http://hsimonis.wordpress.com}
%\includegraphics[width=10cm]{../introduction/images/blog}
%\end{frame}

%% \begin{frame}
%% \frametitle{Exercises}
%% \begin{itemize}
%% \item Install Constraint Solver of your Choice
%% \item MiniZinc (MiningZinc) will be used in later sessions
%% \item Run Sendmoremoney, Sudoku, Queens examples
%% \item Can you choose which propagator to use for alldifferent?
%% \item Can you use different search strategies?
%% \end{itemize}
%% \end{frame}

\begin{frame}
\frametitle{More Learning Resources}
\begin{itemize}
\item Survey of Methods, Resources, and Formats for Teaching Constraint Programming
\begin{itemize}
\item by Tejas Santanam, Helmut Simonis
\item \url{https://doi.org/10.48550/arXiv.2403.12717}
\item Based on survey of community for WTCP 2023
\item \url{https://hsimonis.github.io/WTCP2023/}
\end{itemize}
\end{itemize}
\end{frame}


}

\clearpage
\mode<all>{
\part{Customizing Search}
}
\begin{frame}
\partpage
\end{frame}
\mode<all>{
\begin{frame}
\frametitle{Objectives}
\begin{itemize}
\item Overview of Core Constraint Programming
\item Three Main Concepts
\begin{itemize}
\item Constraint Propagation
\item Global Constraints
\item Customizing Search
\end{itemize}
%\item Get Some Experience with MiniZinc
\item Topics will be treated in more detail in later parts of the school
\item Based on Examples, not Formal Description
\end{itemize}
\end{frame}

\begin{frame}
\frametitle{Outline}
\begin{itemize}
\item Why Constraint Programming?
\item Constraint Propagation
\item Global Constraints
\item Customizing Search
%\item Choices in Modelling
%\item Decomposition
%\item More Global Constraints
%\item Methodology
%\item What is missing?
\end{itemize}
\end{frame}

% \begin{frame}
  % \frametitle{Using MiniZinc IDE}
  % \begin{itemize}
  % \item Developed in the Australian NICTA project
  % \item Maintained by Monash University
  % \item Modelling tool with multiple back-end solvers
  % \item Available from \url{https://www.minizinc.org/}
  % \end{itemize}
% \end{frame}


% \begin{frame}
% \frametitle{Examples in ECLiPSe}
% \begin{itemize}
% \item Open sourced constraint programming language
% \item Development goes back to 1985
% \item ECRC, ICL, IC-Parc, PTL, Cisco
% \item \url{https://eclipseclp.org/}
% \item Specialities
% \begin{itemize}
% \item Develop new solvers for specific domains
% \item Integration with MIP
% \end{itemize}
% \item Not included in bundled MiniZinc IDE
% \item Specialized visualization tools used here
  % \begin{itemize}
  % \item CP-Viz, Simonis et al. 2010
    % \end{itemize}
% \end{itemize}
% \end{frame}

\begin{frame}
\frametitle{Tutorial Based on ECLiPSe ELearning Course}
\begin{itemize}
\item Self-study course in constraint programming
\item Supported by Cisco Systems and Silicon Valley Community Foundation
\item Multi-media format, video lectures, slides, handout etc
\item {\small \url{https://eclipseclp.org/ELearning/index.html}}
\end{itemize}
\end{frame}

\begin{frame}
\frametitle{Also Part of CRT-AI Constraint Week}
\begin{itemize}
\item Annual one week course on CP and Optimization in Ireland
\item Part of national training program for PhD students in AI
\item \url{https://www.crt-ai.ie/}
\end{itemize}
\end{frame}



\begin{frame}
\frametitle{Constraint Programming - in a nutshell}
\begin{itemize}
\item Declarative description of problems with
\begin{itemize}
\item {\em Variables} which range over (finite) sets of values
\item {\em Constraints} over subsets of variables which restrict possible value combinations
\item A {\em solution} is a value assignment which satisfies all constraints
\end{itemize}

\item Constraint propagation/reasoning
\begin{itemize}
\item Removing inconsistent values for variables
\item Detect failure if constraint can not be satisfied
\item Interaction of constraints via shared variables
\item Incomplete
\end{itemize}

\item Search
\begin{itemize}
\item User controlled assignment of values to variables
\item Each step triggers constraint propagation 
\end{itemize}
\item Different domains require/allow different methods
\end{itemize}
\end{frame}

\begin{frame}
  \frametitle{Constraint Programming is Different}
  \begin{itemize}
  \item Declarative Programming
    \begin{itemize}
    \item Concentrate on what you want
      \item Not how to get there
      \item Program != Algorithm
      \item Program = Model
    \end{itemize}
    \item Applied to Combinatorial Problems
      \begin{itemize}
        \item No complete polynomial algorithms known (exist?)
        \item CP less ad-hoc than heuristics
        \item Models can evolve
  \end{itemize}
  \end{itemize}
  \end{frame}
    
\begin{frame}
  \frametitle{A Subtractive Process}
  \begin{textblock}{4}(8,-3)
    \includegraphics[width=4cm]{../introduction/images/stages}
  \end{textblock}
  \vfill
  \begin{quote}
    ``Oh, bosh, as Mr. Ruskin says. Sculpture, per se, is the simplest thing in the world. All you have to do is to take a big chunk of marble and a hammer and chisel, make up your mind what you are about to create and chip off all the marble you don't want.''-Paris Gaulois.
  \end{quote}
  
  {\tiny Source: \url{https://quoteinvestigator.com/2014/06/22/chip-away/}}
\end{frame}

\begin{frame}
\frametitle{Basic Process}
\begin{center}
\begin{tikzpicture}
\node[shape=rectangle,fill=pantone127-4] (problem) at (2,5) {Problem};
\node (human) at (2,4) {Human};
\node[shape=rectangle,fill=pantone127-4] (model) at (2,3) {Model};
\node (solver) at (2,2) {Constraint Solver/Search};
\node[shape=rectangle,fill=pantone157-8] (solution) at (2,1) {Solution};
\draw[-] (problem) -- (human);
\draw[->] (human) -- (model);
\draw[-] (model) -- (solver);
\draw[->] (solver) -- (solution);
\end{tikzpicture}
\end{center}
\end{frame}

\begin{frame}
\frametitle{More Realistic}
\begin{center}
\begin{tikzpicture}[xscale=4.0]
\node[shape=rectangle,fill=pantone127-4] (problem) at (2,5) {Problem};
\node (human) at (2,4) {Human};
\node[shape=rectangle,fill=pantone127-4] (model) at (2,3) {Model};
\node (solver) at (2,2) {Constraint Solver/Search};
\node[shape=rectangle,fill=pantone24-8] (hangs) at (1,1) {Hangs};
\node[shape=rectangle,fill=pantone157-8] (solution) at (2,1) {Solution};
\node[shape=rectangle,fill=pantone24-8] (wrong) at (3,1) {Wrong Solution};
\draw[-] (problem) -- (human);
\draw[->] (human) -- (model);
\draw[-] (model) -- (solver);
\draw[->] (solver) -- (hangs);
\draw[->] (solver) -- (solution);
\draw[->] (solver) -- (wrong);
\draw[->] (hangs) -- (1,4) -- (human);
\draw[->] (wrong) -- (3,4) -- (human);
\end{tikzpicture}
\end{center}
\end{frame}

\begin{frame}
\frametitle{Dual Role of Model}
\begin{itemize}
\item Allows Human to Express Problem
\begin{itemize}
\item Close to Problem Domain
\item Constraints as Abstractions
\end{itemize}
\item Allows Solver to Execute
\begin{itemize}
\item Variables as Communication Mechanism
\item Constraints as Algorithms
\end{itemize}
\end{itemize}
\end{frame}

\begin{frame}
\frametitle{Modelling Frameworks}
\begin{itemize}
\item \href{https://www.minizinc.org/}{MiniZinc} (NICTA, Monash University, Australia)
\item \href{https://github.com/eomahony/Numberjack}{NumberJack} (Insight, Ireland) 
\item \href{https://www-users.york.ac.uk/peter.nightingale/savilerow//}{EssencePrime/SavilleRow} (UK) 
\item \href{https://cpmpy.readthedocs.io/en/latest/}{CPMpy} (KU Leuven)
\item Allow use of multiple back-end solvers
\item Compile model into variants for each solver
\item A priori solver independent model(CP, MIP, SAT)
\end{itemize}
\end{frame} 

\begin{frame}
\frametitle{Framework Process}
\begin{center}
\begin{tikzpicture}[xscale=2]
\node[shape=rectangle,fill=pantone127-4] (problem) at (2.5,5) {Problem};
\node (human) at (2.5,4) {Human};
\node[shape=rectangle,fill=pantone127-4] (model) at (2.5,3) {Model};
\node (compile) at (2.5,2) {Compile/Reformulate};
\node (solver1) at (1,1) {CP};
\node (solver2) at (2,1) {MIP};
\node (solver3) at (3,1) {SAT};
\node (solver4) at (4,1) {Other};
\node[shape=rectangle,fill=pantone157-8] (solution1) at (1,0) {Solution};
\node[shape=rectangle,fill=pantone157-8] (solution2) at (2,0) {Solution};
\node[shape=rectangle,fill=pantone157-8] (solution3) at (3,0) {Solution};
\node[shape=rectangle,fill=pantone157-8] (solution4) at (4,0) {Solution};
\draw[-] (problem) -- (human);
\draw[->] (human) -- (model);
\draw[-] (model) -- (compile);
\draw[-] (compile) -- (solver1);
\draw[-] (compile) -- (solver2);
\draw[-] (compile) -- (solver3);
\draw[-] (compile) -- (solver4);
\draw[->] (solver1) -- (solution1);
\draw[->] (solver2) -- (solution2);
\draw[->] (solver3) -- (solution3);
\draw[->] (solver4) -- (solution4);
\end{tikzpicture}
\end{center}
\end{frame}

% \begin{frame}
  % \frametitle{Do It Now!}
  % \begin{itemize}
  % \item Download and install Minizinc
  % \item \url{https://www.minizinc.org/}
  % \end{itemize}
% \end{frame}


\begin{frame}
\frametitle{Why use Puzzles as Examples?}
\begin{itemize}
\item Easy to understand the problem
\item Solvable by hand without specialized knowledge
\item Possible to compare automated to manual solving process
\end{itemize}

\begin{quote}
The puzzle, though inanimate, is presented as a solvable problem without lasting negative consequences, a very low-risk low-reward situation. By being a puzzle, the object is attempting to convince the user that it must be completed.
\end{quote}
{\tiny Source: \href{https://www.everydayrhetoric.com/2020/03/the-rhetoric-of-puzzles/}{Every Day Rhetoric}}
\end{frame}


\mode<all>{
\part{Basic Constraint Propagation}
}
\begin{frame}
\partpage
\end{frame}
\mode<all>{
\input{../sendmore/body}
}

\mode<all>{
\part{Global Constraints}
}
\begin{frame}
\partpage
\end{frame}
\mode<all>{
\input{../sudoku/body}
}

\clearpage
\mode<all>{
\part{Customizing Search}
}
\begin{frame}
\partpage
\end{frame}
\mode<all>{
\input{../nqueen/body}
}


%% \clearpage
%% \mode<all>{
%%   \part{Problem Decomposition}
%% }
%% \begin{frame}
%%   \partpage
%% \end{frame}
%% \mode<all>{
%%   \input{../party/body}
%%   }

%% \clearpage
%% \mode<all>{
%%   \part{More Global Constraints}
%% }
%% \begin{frame}
%%   \partpage
%% \end{frame}
%% \mode<all>{
%%   \input{../car/body}
%%   }

%% \clearpage
%% \mode<all>{
%%   \part{Methodology}
%% }
%% \begin{frame}
%%   \partpage
%% \end{frame}
%% \mode<all>{
%%   \input{../methodology/body}
%%   }

\clearpage
\mode<all>{
\part{What is missing?}
}
\begin{frame}
\partpage
\end{frame}

\begin{frame}
\frametitle{Many Specialized Topics}
\begin{itemize}
\item How to design efficient core engine
\item Hybrids with LP/MIP tools
\item Hybrids with SAT
\item Symmetry breaking
\item Use of MDD/BDD to encode sets of solutions
\item High level modelling tools
\item Debugging/visualization
\end{itemize}
\end{frame}

\begin{frame}
\frametitle{Reformulation}
\begin{itemize}
\item Just because the user has modelled it this way, it doesn't mean we have to solve it that way
\begin{itemize}
\item Replace some constraint(s) by other, equivalent constraints
\item Because we don't have that constraint in our system
\item For performance
\end{itemize}
\end{itemize}
\end{frame}

\begin{frame}
\frametitle{Learning}
\begin{itemize}
\item While solving the problem we can learn how to strengthen the model/search 
\begin{itemize}
\item Understand which constraints/method contribute to propagation and change schedule
\item Learn no-good constraints by explaining failure
\item Adapt search strategy based on search experience
\end{itemize}
\end{itemize}
\end{frame}

%% \begin{frame}
%% \frametitle{Refined Process}
%% \begin{center}
%% \begin{tikzpicture}
%% \node[shape=rectangle,fill=pantone127-4] (problem) at (3,7) {Problem};
%% \node (human) at (3,6) {Human};
%% \node[shape=rectangle,fill=pantone127-4] (model) at (3,5) {User Model};
%% \node (reformulation) at (3,4) {Reformulation};
%% \node[shape=rectangle,fill=pantone127-4] (implementation) at (3,3) {Implementation Model};
%% \node (solver) at (3,2) {Constraint Solver/Search};
%% \node[shape=rectangle,fill=pantone157-8] (solution) at (3,1) {Solution};
%% \draw[-] (problem) -- (human);
%% \draw[->] (human) -- (model);
%% \draw[-] (model) -- (reformulation);
%% \draw[->] (reformulation) -- (implementation);
%% \draw[-] (implementation) -- (solver);
%% \draw[->] (solver) -- (0,2) -- (0,3) -- (implementation);
%% \draw[->] (solver) -- (solution);
%% \end{tikzpicture}
%% \end{center}
%% \end{frame}

%\begin{frame}
%\frametitle{What is CP actually used for?}
%\url{http://hsimonis.wordpress.com}
%\includegraphics[width=10cm]{../introduction/images/blog}
%\end{frame}

%% \begin{frame}
%% \frametitle{Exercises}
%% \begin{itemize}
%% \item Install Constraint Solver of your Choice
%% \item MiniZinc (MiningZinc) will be used in later sessions
%% \item Run Sendmoremoney, Sudoku, Queens examples
%% \item Can you choose which propagator to use for alldifferent?
%% \item Can you use different search strategies?
%% \end{itemize}
%% \end{frame}

\begin{frame}
\frametitle{More Learning Resources}
\begin{itemize}
\item Survey of Methods, Resources, and Formats for Teaching Constraint Programming
\begin{itemize}
\item by Tejas Santanam, Helmut Simonis
\item \url{https://doi.org/10.48550/arXiv.2403.12717}
\item Based on survey of community for WTCP 2023
\item \url{https://hsimonis.github.io/WTCP2023/}
\end{itemize}
\end{itemize}
\end{frame}


}


%% \clearpage
%% \mode<all>{
%%   \part{Problem Decomposition}
%% }
%% \begin{frame}
%%   \partpage
%% \end{frame}
%% \mode<all>{
%%   \begin{frame}
\frametitle{Objectives}
\begin{itemize}
\item Overview of Core Constraint Programming
\item Three Main Concepts
\begin{itemize}
\item Constraint Propagation
\item Global Constraints
\item Customizing Search
\end{itemize}
%\item Get Some Experience with MiniZinc
\item Topics will be treated in more detail in later parts of the school
\item Based on Examples, not Formal Description
\end{itemize}
\end{frame}

\begin{frame}
\frametitle{Outline}
\begin{itemize}
\item Why Constraint Programming?
\item Constraint Propagation
\item Global Constraints
\item Customizing Search
%\item Choices in Modelling
%\item Decomposition
%\item More Global Constraints
%\item Methodology
%\item What is missing?
\end{itemize}
\end{frame}

% \begin{frame}
  % \frametitle{Using MiniZinc IDE}
  % \begin{itemize}
  % \item Developed in the Australian NICTA project
  % \item Maintained by Monash University
  % \item Modelling tool with multiple back-end solvers
  % \item Available from \url{https://www.minizinc.org/}
  % \end{itemize}
% \end{frame}


% \begin{frame}
% \frametitle{Examples in ECLiPSe}
% \begin{itemize}
% \item Open sourced constraint programming language
% \item Development goes back to 1985
% \item ECRC, ICL, IC-Parc, PTL, Cisco
% \item \url{https://eclipseclp.org/}
% \item Specialities
% \begin{itemize}
% \item Develop new solvers for specific domains
% \item Integration with MIP
% \end{itemize}
% \item Not included in bundled MiniZinc IDE
% \item Specialized visualization tools used here
  % \begin{itemize}
  % \item CP-Viz, Simonis et al. 2010
    % \end{itemize}
% \end{itemize}
% \end{frame}

\begin{frame}
\frametitle{Tutorial Based on ECLiPSe ELearning Course}
\begin{itemize}
\item Self-study course in constraint programming
\item Supported by Cisco Systems and Silicon Valley Community Foundation
\item Multi-media format, video lectures, slides, handout etc
\item {\small \url{https://eclipseclp.org/ELearning/index.html}}
\end{itemize}
\end{frame}

\begin{frame}
\frametitle{Also Part of CRT-AI Constraint Week}
\begin{itemize}
\item Annual one week course on CP and Optimization in Ireland
\item Part of national training program for PhD students in AI
\item \url{https://www.crt-ai.ie/}
\end{itemize}
\end{frame}



\begin{frame}
\frametitle{Constraint Programming - in a nutshell}
\begin{itemize}
\item Declarative description of problems with
\begin{itemize}
\item {\em Variables} which range over (finite) sets of values
\item {\em Constraints} over subsets of variables which restrict possible value combinations
\item A {\em solution} is a value assignment which satisfies all constraints
\end{itemize}

\item Constraint propagation/reasoning
\begin{itemize}
\item Removing inconsistent values for variables
\item Detect failure if constraint can not be satisfied
\item Interaction of constraints via shared variables
\item Incomplete
\end{itemize}

\item Search
\begin{itemize}
\item User controlled assignment of values to variables
\item Each step triggers constraint propagation 
\end{itemize}
\item Different domains require/allow different methods
\end{itemize}
\end{frame}

\begin{frame}
  \frametitle{Constraint Programming is Different}
  \begin{itemize}
  \item Declarative Programming
    \begin{itemize}
    \item Concentrate on what you want
      \item Not how to get there
      \item Program != Algorithm
      \item Program = Model
    \end{itemize}
    \item Applied to Combinatorial Problems
      \begin{itemize}
        \item No complete polynomial algorithms known (exist?)
        \item CP less ad-hoc than heuristics
        \item Models can evolve
  \end{itemize}
  \end{itemize}
  \end{frame}
    
\begin{frame}
  \frametitle{A Subtractive Process}
  \begin{textblock}{4}(8,-3)
    \includegraphics[width=4cm]{../introduction/images/stages}
  \end{textblock}
  \vfill
  \begin{quote}
    ``Oh, bosh, as Mr. Ruskin says. Sculpture, per se, is the simplest thing in the world. All you have to do is to take a big chunk of marble and a hammer and chisel, make up your mind what you are about to create and chip off all the marble you don't want.''-Paris Gaulois.
  \end{quote}
  
  {\tiny Source: \url{https://quoteinvestigator.com/2014/06/22/chip-away/}}
\end{frame}

\begin{frame}
\frametitle{Basic Process}
\begin{center}
\begin{tikzpicture}
\node[shape=rectangle,fill=pantone127-4] (problem) at (2,5) {Problem};
\node (human) at (2,4) {Human};
\node[shape=rectangle,fill=pantone127-4] (model) at (2,3) {Model};
\node (solver) at (2,2) {Constraint Solver/Search};
\node[shape=rectangle,fill=pantone157-8] (solution) at (2,1) {Solution};
\draw[-] (problem) -- (human);
\draw[->] (human) -- (model);
\draw[-] (model) -- (solver);
\draw[->] (solver) -- (solution);
\end{tikzpicture}
\end{center}
\end{frame}

\begin{frame}
\frametitle{More Realistic}
\begin{center}
\begin{tikzpicture}[xscale=4.0]
\node[shape=rectangle,fill=pantone127-4] (problem) at (2,5) {Problem};
\node (human) at (2,4) {Human};
\node[shape=rectangle,fill=pantone127-4] (model) at (2,3) {Model};
\node (solver) at (2,2) {Constraint Solver/Search};
\node[shape=rectangle,fill=pantone24-8] (hangs) at (1,1) {Hangs};
\node[shape=rectangle,fill=pantone157-8] (solution) at (2,1) {Solution};
\node[shape=rectangle,fill=pantone24-8] (wrong) at (3,1) {Wrong Solution};
\draw[-] (problem) -- (human);
\draw[->] (human) -- (model);
\draw[-] (model) -- (solver);
\draw[->] (solver) -- (hangs);
\draw[->] (solver) -- (solution);
\draw[->] (solver) -- (wrong);
\draw[->] (hangs) -- (1,4) -- (human);
\draw[->] (wrong) -- (3,4) -- (human);
\end{tikzpicture}
\end{center}
\end{frame}

\begin{frame}
\frametitle{Dual Role of Model}
\begin{itemize}
\item Allows Human to Express Problem
\begin{itemize}
\item Close to Problem Domain
\item Constraints as Abstractions
\end{itemize}
\item Allows Solver to Execute
\begin{itemize}
\item Variables as Communication Mechanism
\item Constraints as Algorithms
\end{itemize}
\end{itemize}
\end{frame}

\begin{frame}
\frametitle{Modelling Frameworks}
\begin{itemize}
\item \href{https://www.minizinc.org/}{MiniZinc} (NICTA, Monash University, Australia)
\item \href{https://github.com/eomahony/Numberjack}{NumberJack} (Insight, Ireland) 
\item \href{https://www-users.york.ac.uk/peter.nightingale/savilerow//}{EssencePrime/SavilleRow} (UK) 
\item \href{https://cpmpy.readthedocs.io/en/latest/}{CPMpy} (KU Leuven)
\item Allow use of multiple back-end solvers
\item Compile model into variants for each solver
\item A priori solver independent model(CP, MIP, SAT)
\end{itemize}
\end{frame} 

\begin{frame}
\frametitle{Framework Process}
\begin{center}
\begin{tikzpicture}[xscale=2]
\node[shape=rectangle,fill=pantone127-4] (problem) at (2.5,5) {Problem};
\node (human) at (2.5,4) {Human};
\node[shape=rectangle,fill=pantone127-4] (model) at (2.5,3) {Model};
\node (compile) at (2.5,2) {Compile/Reformulate};
\node (solver1) at (1,1) {CP};
\node (solver2) at (2,1) {MIP};
\node (solver3) at (3,1) {SAT};
\node (solver4) at (4,1) {Other};
\node[shape=rectangle,fill=pantone157-8] (solution1) at (1,0) {Solution};
\node[shape=rectangle,fill=pantone157-8] (solution2) at (2,0) {Solution};
\node[shape=rectangle,fill=pantone157-8] (solution3) at (3,0) {Solution};
\node[shape=rectangle,fill=pantone157-8] (solution4) at (4,0) {Solution};
\draw[-] (problem) -- (human);
\draw[->] (human) -- (model);
\draw[-] (model) -- (compile);
\draw[-] (compile) -- (solver1);
\draw[-] (compile) -- (solver2);
\draw[-] (compile) -- (solver3);
\draw[-] (compile) -- (solver4);
\draw[->] (solver1) -- (solution1);
\draw[->] (solver2) -- (solution2);
\draw[->] (solver3) -- (solution3);
\draw[->] (solver4) -- (solution4);
\end{tikzpicture}
\end{center}
\end{frame}

% \begin{frame}
  % \frametitle{Do It Now!}
  % \begin{itemize}
  % \item Download and install Minizinc
  % \item \url{https://www.minizinc.org/}
  % \end{itemize}
% \end{frame}


\begin{frame}
\frametitle{Why use Puzzles as Examples?}
\begin{itemize}
\item Easy to understand the problem
\item Solvable by hand without specialized knowledge
\item Possible to compare automated to manual solving process
\end{itemize}

\begin{quote}
The puzzle, though inanimate, is presented as a solvable problem without lasting negative consequences, a very low-risk low-reward situation. By being a puzzle, the object is attempting to convince the user that it must be completed.
\end{quote}
{\tiny Source: \href{https://www.everydayrhetoric.com/2020/03/the-rhetoric-of-puzzles/}{Every Day Rhetoric}}
\end{frame}


\mode<all>{
\part{Basic Constraint Propagation}
}
\begin{frame}
\partpage
\end{frame}
\mode<all>{
\input{../sendmore/body}
}

\mode<all>{
\part{Global Constraints}
}
\begin{frame}
\partpage
\end{frame}
\mode<all>{
\input{../sudoku/body}
}

\clearpage
\mode<all>{
\part{Customizing Search}
}
\begin{frame}
\partpage
\end{frame}
\mode<all>{
\input{../nqueen/body}
}


%% \clearpage
%% \mode<all>{
%%   \part{Problem Decomposition}
%% }
%% \begin{frame}
%%   \partpage
%% \end{frame}
%% \mode<all>{
%%   \input{../party/body}
%%   }

%% \clearpage
%% \mode<all>{
%%   \part{More Global Constraints}
%% }
%% \begin{frame}
%%   \partpage
%% \end{frame}
%% \mode<all>{
%%   \input{../car/body}
%%   }

%% \clearpage
%% \mode<all>{
%%   \part{Methodology}
%% }
%% \begin{frame}
%%   \partpage
%% \end{frame}
%% \mode<all>{
%%   \input{../methodology/body}
%%   }

\clearpage
\mode<all>{
\part{What is missing?}
}
\begin{frame}
\partpage
\end{frame}

\begin{frame}
\frametitle{Many Specialized Topics}
\begin{itemize}
\item How to design efficient core engine
\item Hybrids with LP/MIP tools
\item Hybrids with SAT
\item Symmetry breaking
\item Use of MDD/BDD to encode sets of solutions
\item High level modelling tools
\item Debugging/visualization
\end{itemize}
\end{frame}

\begin{frame}
\frametitle{Reformulation}
\begin{itemize}
\item Just because the user has modelled it this way, it doesn't mean we have to solve it that way
\begin{itemize}
\item Replace some constraint(s) by other, equivalent constraints
\item Because we don't have that constraint in our system
\item For performance
\end{itemize}
\end{itemize}
\end{frame}

\begin{frame}
\frametitle{Learning}
\begin{itemize}
\item While solving the problem we can learn how to strengthen the model/search 
\begin{itemize}
\item Understand which constraints/method contribute to propagation and change schedule
\item Learn no-good constraints by explaining failure
\item Adapt search strategy based on search experience
\end{itemize}
\end{itemize}
\end{frame}

%% \begin{frame}
%% \frametitle{Refined Process}
%% \begin{center}
%% \begin{tikzpicture}
%% \node[shape=rectangle,fill=pantone127-4] (problem) at (3,7) {Problem};
%% \node (human) at (3,6) {Human};
%% \node[shape=rectangle,fill=pantone127-4] (model) at (3,5) {User Model};
%% \node (reformulation) at (3,4) {Reformulation};
%% \node[shape=rectangle,fill=pantone127-4] (implementation) at (3,3) {Implementation Model};
%% \node (solver) at (3,2) {Constraint Solver/Search};
%% \node[shape=rectangle,fill=pantone157-8] (solution) at (3,1) {Solution};
%% \draw[-] (problem) -- (human);
%% \draw[->] (human) -- (model);
%% \draw[-] (model) -- (reformulation);
%% \draw[->] (reformulation) -- (implementation);
%% \draw[-] (implementation) -- (solver);
%% \draw[->] (solver) -- (0,2) -- (0,3) -- (implementation);
%% \draw[->] (solver) -- (solution);
%% \end{tikzpicture}
%% \end{center}
%% \end{frame}

%\begin{frame}
%\frametitle{What is CP actually used for?}
%\url{http://hsimonis.wordpress.com}
%\includegraphics[width=10cm]{../introduction/images/blog}
%\end{frame}

%% \begin{frame}
%% \frametitle{Exercises}
%% \begin{itemize}
%% \item Install Constraint Solver of your Choice
%% \item MiniZinc (MiningZinc) will be used in later sessions
%% \item Run Sendmoremoney, Sudoku, Queens examples
%% \item Can you choose which propagator to use for alldifferent?
%% \item Can you use different search strategies?
%% \end{itemize}
%% \end{frame}

\begin{frame}
\frametitle{More Learning Resources}
\begin{itemize}
\item Survey of Methods, Resources, and Formats for Teaching Constraint Programming
\begin{itemize}
\item by Tejas Santanam, Helmut Simonis
\item \url{https://doi.org/10.48550/arXiv.2403.12717}
\item Based on survey of community for WTCP 2023
\item \url{https://hsimonis.github.io/WTCP2023/}
\end{itemize}
\end{itemize}
\end{frame}


%%   }

%% \clearpage
%% \mode<all>{
%%   \part{More Global Constraints}
%% }
%% \begin{frame}
%%   \partpage
%% \end{frame}
%% \mode<all>{
%%   \begin{frame}
\frametitle{Objectives}
\begin{itemize}
\item Overview of Core Constraint Programming
\item Three Main Concepts
\begin{itemize}
\item Constraint Propagation
\item Global Constraints
\item Customizing Search
\end{itemize}
%\item Get Some Experience with MiniZinc
\item Topics will be treated in more detail in later parts of the school
\item Based on Examples, not Formal Description
\end{itemize}
\end{frame}

\begin{frame}
\frametitle{Outline}
\begin{itemize}
\item Why Constraint Programming?
\item Constraint Propagation
\item Global Constraints
\item Customizing Search
%\item Choices in Modelling
%\item Decomposition
%\item More Global Constraints
%\item Methodology
%\item What is missing?
\end{itemize}
\end{frame}

% \begin{frame}
  % \frametitle{Using MiniZinc IDE}
  % \begin{itemize}
  % \item Developed in the Australian NICTA project
  % \item Maintained by Monash University
  % \item Modelling tool with multiple back-end solvers
  % \item Available from \url{https://www.minizinc.org/}
  % \end{itemize}
% \end{frame}


% \begin{frame}
% \frametitle{Examples in ECLiPSe}
% \begin{itemize}
% \item Open sourced constraint programming language
% \item Development goes back to 1985
% \item ECRC, ICL, IC-Parc, PTL, Cisco
% \item \url{https://eclipseclp.org/}
% \item Specialities
% \begin{itemize}
% \item Develop new solvers for specific domains
% \item Integration with MIP
% \end{itemize}
% \item Not included in bundled MiniZinc IDE
% \item Specialized visualization tools used here
  % \begin{itemize}
  % \item CP-Viz, Simonis et al. 2010
    % \end{itemize}
% \end{itemize}
% \end{frame}

\begin{frame}
\frametitle{Tutorial Based on ECLiPSe ELearning Course}
\begin{itemize}
\item Self-study course in constraint programming
\item Supported by Cisco Systems and Silicon Valley Community Foundation
\item Multi-media format, video lectures, slides, handout etc
\item {\small \url{https://eclipseclp.org/ELearning/index.html}}
\end{itemize}
\end{frame}

\begin{frame}
\frametitle{Also Part of CRT-AI Constraint Week}
\begin{itemize}
\item Annual one week course on CP and Optimization in Ireland
\item Part of national training program for PhD students in AI
\item \url{https://www.crt-ai.ie/}
\end{itemize}
\end{frame}



\begin{frame}
\frametitle{Constraint Programming - in a nutshell}
\begin{itemize}
\item Declarative description of problems with
\begin{itemize}
\item {\em Variables} which range over (finite) sets of values
\item {\em Constraints} over subsets of variables which restrict possible value combinations
\item A {\em solution} is a value assignment which satisfies all constraints
\end{itemize}

\item Constraint propagation/reasoning
\begin{itemize}
\item Removing inconsistent values for variables
\item Detect failure if constraint can not be satisfied
\item Interaction of constraints via shared variables
\item Incomplete
\end{itemize}

\item Search
\begin{itemize}
\item User controlled assignment of values to variables
\item Each step triggers constraint propagation 
\end{itemize}
\item Different domains require/allow different methods
\end{itemize}
\end{frame}

\begin{frame}
  \frametitle{Constraint Programming is Different}
  \begin{itemize}
  \item Declarative Programming
    \begin{itemize}
    \item Concentrate on what you want
      \item Not how to get there
      \item Program != Algorithm
      \item Program = Model
    \end{itemize}
    \item Applied to Combinatorial Problems
      \begin{itemize}
        \item No complete polynomial algorithms known (exist?)
        \item CP less ad-hoc than heuristics
        \item Models can evolve
  \end{itemize}
  \end{itemize}
  \end{frame}
    
\begin{frame}
  \frametitle{A Subtractive Process}
  \begin{textblock}{4}(8,-3)
    \includegraphics[width=4cm]{../introduction/images/stages}
  \end{textblock}
  \vfill
  \begin{quote}
    ``Oh, bosh, as Mr. Ruskin says. Sculpture, per se, is the simplest thing in the world. All you have to do is to take a big chunk of marble and a hammer and chisel, make up your mind what you are about to create and chip off all the marble you don't want.''-Paris Gaulois.
  \end{quote}
  
  {\tiny Source: \url{https://quoteinvestigator.com/2014/06/22/chip-away/}}
\end{frame}

\begin{frame}
\frametitle{Basic Process}
\begin{center}
\begin{tikzpicture}
\node[shape=rectangle,fill=pantone127-4] (problem) at (2,5) {Problem};
\node (human) at (2,4) {Human};
\node[shape=rectangle,fill=pantone127-4] (model) at (2,3) {Model};
\node (solver) at (2,2) {Constraint Solver/Search};
\node[shape=rectangle,fill=pantone157-8] (solution) at (2,1) {Solution};
\draw[-] (problem) -- (human);
\draw[->] (human) -- (model);
\draw[-] (model) -- (solver);
\draw[->] (solver) -- (solution);
\end{tikzpicture}
\end{center}
\end{frame}

\begin{frame}
\frametitle{More Realistic}
\begin{center}
\begin{tikzpicture}[xscale=4.0]
\node[shape=rectangle,fill=pantone127-4] (problem) at (2,5) {Problem};
\node (human) at (2,4) {Human};
\node[shape=rectangle,fill=pantone127-4] (model) at (2,3) {Model};
\node (solver) at (2,2) {Constraint Solver/Search};
\node[shape=rectangle,fill=pantone24-8] (hangs) at (1,1) {Hangs};
\node[shape=rectangle,fill=pantone157-8] (solution) at (2,1) {Solution};
\node[shape=rectangle,fill=pantone24-8] (wrong) at (3,1) {Wrong Solution};
\draw[-] (problem) -- (human);
\draw[->] (human) -- (model);
\draw[-] (model) -- (solver);
\draw[->] (solver) -- (hangs);
\draw[->] (solver) -- (solution);
\draw[->] (solver) -- (wrong);
\draw[->] (hangs) -- (1,4) -- (human);
\draw[->] (wrong) -- (3,4) -- (human);
\end{tikzpicture}
\end{center}
\end{frame}

\begin{frame}
\frametitle{Dual Role of Model}
\begin{itemize}
\item Allows Human to Express Problem
\begin{itemize}
\item Close to Problem Domain
\item Constraints as Abstractions
\end{itemize}
\item Allows Solver to Execute
\begin{itemize}
\item Variables as Communication Mechanism
\item Constraints as Algorithms
\end{itemize}
\end{itemize}
\end{frame}

\begin{frame}
\frametitle{Modelling Frameworks}
\begin{itemize}
\item \href{https://www.minizinc.org/}{MiniZinc} (NICTA, Monash University, Australia)
\item \href{https://github.com/eomahony/Numberjack}{NumberJack} (Insight, Ireland) 
\item \href{https://www-users.york.ac.uk/peter.nightingale/savilerow//}{EssencePrime/SavilleRow} (UK) 
\item \href{https://cpmpy.readthedocs.io/en/latest/}{CPMpy} (KU Leuven)
\item Allow use of multiple back-end solvers
\item Compile model into variants for each solver
\item A priori solver independent model(CP, MIP, SAT)
\end{itemize}
\end{frame} 

\begin{frame}
\frametitle{Framework Process}
\begin{center}
\begin{tikzpicture}[xscale=2]
\node[shape=rectangle,fill=pantone127-4] (problem) at (2.5,5) {Problem};
\node (human) at (2.5,4) {Human};
\node[shape=rectangle,fill=pantone127-4] (model) at (2.5,3) {Model};
\node (compile) at (2.5,2) {Compile/Reformulate};
\node (solver1) at (1,1) {CP};
\node (solver2) at (2,1) {MIP};
\node (solver3) at (3,1) {SAT};
\node (solver4) at (4,1) {Other};
\node[shape=rectangle,fill=pantone157-8] (solution1) at (1,0) {Solution};
\node[shape=rectangle,fill=pantone157-8] (solution2) at (2,0) {Solution};
\node[shape=rectangle,fill=pantone157-8] (solution3) at (3,0) {Solution};
\node[shape=rectangle,fill=pantone157-8] (solution4) at (4,0) {Solution};
\draw[-] (problem) -- (human);
\draw[->] (human) -- (model);
\draw[-] (model) -- (compile);
\draw[-] (compile) -- (solver1);
\draw[-] (compile) -- (solver2);
\draw[-] (compile) -- (solver3);
\draw[-] (compile) -- (solver4);
\draw[->] (solver1) -- (solution1);
\draw[->] (solver2) -- (solution2);
\draw[->] (solver3) -- (solution3);
\draw[->] (solver4) -- (solution4);
\end{tikzpicture}
\end{center}
\end{frame}

% \begin{frame}
  % \frametitle{Do It Now!}
  % \begin{itemize}
  % \item Download and install Minizinc
  % \item \url{https://www.minizinc.org/}
  % \end{itemize}
% \end{frame}


\begin{frame}
\frametitle{Why use Puzzles as Examples?}
\begin{itemize}
\item Easy to understand the problem
\item Solvable by hand without specialized knowledge
\item Possible to compare automated to manual solving process
\end{itemize}

\begin{quote}
The puzzle, though inanimate, is presented as a solvable problem without lasting negative consequences, a very low-risk low-reward situation. By being a puzzle, the object is attempting to convince the user that it must be completed.
\end{quote}
{\tiny Source: \href{https://www.everydayrhetoric.com/2020/03/the-rhetoric-of-puzzles/}{Every Day Rhetoric}}
\end{frame}


\mode<all>{
\part{Basic Constraint Propagation}
}
\begin{frame}
\partpage
\end{frame}
\mode<all>{
\input{../sendmore/body}
}

\mode<all>{
\part{Global Constraints}
}
\begin{frame}
\partpage
\end{frame}
\mode<all>{
\input{../sudoku/body}
}

\clearpage
\mode<all>{
\part{Customizing Search}
}
\begin{frame}
\partpage
\end{frame}
\mode<all>{
\input{../nqueen/body}
}


%% \clearpage
%% \mode<all>{
%%   \part{Problem Decomposition}
%% }
%% \begin{frame}
%%   \partpage
%% \end{frame}
%% \mode<all>{
%%   \input{../party/body}
%%   }

%% \clearpage
%% \mode<all>{
%%   \part{More Global Constraints}
%% }
%% \begin{frame}
%%   \partpage
%% \end{frame}
%% \mode<all>{
%%   \input{../car/body}
%%   }

%% \clearpage
%% \mode<all>{
%%   \part{Methodology}
%% }
%% \begin{frame}
%%   \partpage
%% \end{frame}
%% \mode<all>{
%%   \input{../methodology/body}
%%   }

\clearpage
\mode<all>{
\part{What is missing?}
}
\begin{frame}
\partpage
\end{frame}

\begin{frame}
\frametitle{Many Specialized Topics}
\begin{itemize}
\item How to design efficient core engine
\item Hybrids with LP/MIP tools
\item Hybrids with SAT
\item Symmetry breaking
\item Use of MDD/BDD to encode sets of solutions
\item High level modelling tools
\item Debugging/visualization
\end{itemize}
\end{frame}

\begin{frame}
\frametitle{Reformulation}
\begin{itemize}
\item Just because the user has modelled it this way, it doesn't mean we have to solve it that way
\begin{itemize}
\item Replace some constraint(s) by other, equivalent constraints
\item Because we don't have that constraint in our system
\item For performance
\end{itemize}
\end{itemize}
\end{frame}

\begin{frame}
\frametitle{Learning}
\begin{itemize}
\item While solving the problem we can learn how to strengthen the model/search 
\begin{itemize}
\item Understand which constraints/method contribute to propagation and change schedule
\item Learn no-good constraints by explaining failure
\item Adapt search strategy based on search experience
\end{itemize}
\end{itemize}
\end{frame}

%% \begin{frame}
%% \frametitle{Refined Process}
%% \begin{center}
%% \begin{tikzpicture}
%% \node[shape=rectangle,fill=pantone127-4] (problem) at (3,7) {Problem};
%% \node (human) at (3,6) {Human};
%% \node[shape=rectangle,fill=pantone127-4] (model) at (3,5) {User Model};
%% \node (reformulation) at (3,4) {Reformulation};
%% \node[shape=rectangle,fill=pantone127-4] (implementation) at (3,3) {Implementation Model};
%% \node (solver) at (3,2) {Constraint Solver/Search};
%% \node[shape=rectangle,fill=pantone157-8] (solution) at (3,1) {Solution};
%% \draw[-] (problem) -- (human);
%% \draw[->] (human) -- (model);
%% \draw[-] (model) -- (reformulation);
%% \draw[->] (reformulation) -- (implementation);
%% \draw[-] (implementation) -- (solver);
%% \draw[->] (solver) -- (0,2) -- (0,3) -- (implementation);
%% \draw[->] (solver) -- (solution);
%% \end{tikzpicture}
%% \end{center}
%% \end{frame}

%\begin{frame}
%\frametitle{What is CP actually used for?}
%\url{http://hsimonis.wordpress.com}
%\includegraphics[width=10cm]{../introduction/images/blog}
%\end{frame}

%% \begin{frame}
%% \frametitle{Exercises}
%% \begin{itemize}
%% \item Install Constraint Solver of your Choice
%% \item MiniZinc (MiningZinc) will be used in later sessions
%% \item Run Sendmoremoney, Sudoku, Queens examples
%% \item Can you choose which propagator to use for alldifferent?
%% \item Can you use different search strategies?
%% \end{itemize}
%% \end{frame}

\begin{frame}
\frametitle{More Learning Resources}
\begin{itemize}
\item Survey of Methods, Resources, and Formats for Teaching Constraint Programming
\begin{itemize}
\item by Tejas Santanam, Helmut Simonis
\item \url{https://doi.org/10.48550/arXiv.2403.12717}
\item Based on survey of community for WTCP 2023
\item \url{https://hsimonis.github.io/WTCP2023/}
\end{itemize}
\end{itemize}
\end{frame}


%%   }

%% \clearpage
%% \mode<all>{
%%   \part{Methodology}
%% }
%% \begin{frame}
%%   \partpage
%% \end{frame}
%% \mode<all>{
%%   \begin{frame}
\frametitle{Objectives}
\begin{itemize}
\item Overview of Core Constraint Programming
\item Three Main Concepts
\begin{itemize}
\item Constraint Propagation
\item Global Constraints
\item Customizing Search
\end{itemize}
%\item Get Some Experience with MiniZinc
\item Topics will be treated in more detail in later parts of the school
\item Based on Examples, not Formal Description
\end{itemize}
\end{frame}

\begin{frame}
\frametitle{Outline}
\begin{itemize}
\item Why Constraint Programming?
\item Constraint Propagation
\item Global Constraints
\item Customizing Search
%\item Choices in Modelling
%\item Decomposition
%\item More Global Constraints
%\item Methodology
%\item What is missing?
\end{itemize}
\end{frame}

% \begin{frame}
  % \frametitle{Using MiniZinc IDE}
  % \begin{itemize}
  % \item Developed in the Australian NICTA project
  % \item Maintained by Monash University
  % \item Modelling tool with multiple back-end solvers
  % \item Available from \url{https://www.minizinc.org/}
  % \end{itemize}
% \end{frame}


% \begin{frame}
% \frametitle{Examples in ECLiPSe}
% \begin{itemize}
% \item Open sourced constraint programming language
% \item Development goes back to 1985
% \item ECRC, ICL, IC-Parc, PTL, Cisco
% \item \url{https://eclipseclp.org/}
% \item Specialities
% \begin{itemize}
% \item Develop new solvers for specific domains
% \item Integration with MIP
% \end{itemize}
% \item Not included in bundled MiniZinc IDE
% \item Specialized visualization tools used here
  % \begin{itemize}
  % \item CP-Viz, Simonis et al. 2010
    % \end{itemize}
% \end{itemize}
% \end{frame}

\begin{frame}
\frametitle{Tutorial Based on ECLiPSe ELearning Course}
\begin{itemize}
\item Self-study course in constraint programming
\item Supported by Cisco Systems and Silicon Valley Community Foundation
\item Multi-media format, video lectures, slides, handout etc
\item {\small \url{https://eclipseclp.org/ELearning/index.html}}
\end{itemize}
\end{frame}

\begin{frame}
\frametitle{Also Part of CRT-AI Constraint Week}
\begin{itemize}
\item Annual one week course on CP and Optimization in Ireland
\item Part of national training program for PhD students in AI
\item \url{https://www.crt-ai.ie/}
\end{itemize}
\end{frame}



\begin{frame}
\frametitle{Constraint Programming - in a nutshell}
\begin{itemize}
\item Declarative description of problems with
\begin{itemize}
\item {\em Variables} which range over (finite) sets of values
\item {\em Constraints} over subsets of variables which restrict possible value combinations
\item A {\em solution} is a value assignment which satisfies all constraints
\end{itemize}

\item Constraint propagation/reasoning
\begin{itemize}
\item Removing inconsistent values for variables
\item Detect failure if constraint can not be satisfied
\item Interaction of constraints via shared variables
\item Incomplete
\end{itemize}

\item Search
\begin{itemize}
\item User controlled assignment of values to variables
\item Each step triggers constraint propagation 
\end{itemize}
\item Different domains require/allow different methods
\end{itemize}
\end{frame}

\begin{frame}
  \frametitle{Constraint Programming is Different}
  \begin{itemize}
  \item Declarative Programming
    \begin{itemize}
    \item Concentrate on what you want
      \item Not how to get there
      \item Program != Algorithm
      \item Program = Model
    \end{itemize}
    \item Applied to Combinatorial Problems
      \begin{itemize}
        \item No complete polynomial algorithms known (exist?)
        \item CP less ad-hoc than heuristics
        \item Models can evolve
  \end{itemize}
  \end{itemize}
  \end{frame}
    
\begin{frame}
  \frametitle{A Subtractive Process}
  \begin{textblock}{4}(8,-3)
    \includegraphics[width=4cm]{../introduction/images/stages}
  \end{textblock}
  \vfill
  \begin{quote}
    ``Oh, bosh, as Mr. Ruskin says. Sculpture, per se, is the simplest thing in the world. All you have to do is to take a big chunk of marble and a hammer and chisel, make up your mind what you are about to create and chip off all the marble you don't want.''-Paris Gaulois.
  \end{quote}
  
  {\tiny Source: \url{https://quoteinvestigator.com/2014/06/22/chip-away/}}
\end{frame}

\begin{frame}
\frametitle{Basic Process}
\begin{center}
\begin{tikzpicture}
\node[shape=rectangle,fill=pantone127-4] (problem) at (2,5) {Problem};
\node (human) at (2,4) {Human};
\node[shape=rectangle,fill=pantone127-4] (model) at (2,3) {Model};
\node (solver) at (2,2) {Constraint Solver/Search};
\node[shape=rectangle,fill=pantone157-8] (solution) at (2,1) {Solution};
\draw[-] (problem) -- (human);
\draw[->] (human) -- (model);
\draw[-] (model) -- (solver);
\draw[->] (solver) -- (solution);
\end{tikzpicture}
\end{center}
\end{frame}

\begin{frame}
\frametitle{More Realistic}
\begin{center}
\begin{tikzpicture}[xscale=4.0]
\node[shape=rectangle,fill=pantone127-4] (problem) at (2,5) {Problem};
\node (human) at (2,4) {Human};
\node[shape=rectangle,fill=pantone127-4] (model) at (2,3) {Model};
\node (solver) at (2,2) {Constraint Solver/Search};
\node[shape=rectangle,fill=pantone24-8] (hangs) at (1,1) {Hangs};
\node[shape=rectangle,fill=pantone157-8] (solution) at (2,1) {Solution};
\node[shape=rectangle,fill=pantone24-8] (wrong) at (3,1) {Wrong Solution};
\draw[-] (problem) -- (human);
\draw[->] (human) -- (model);
\draw[-] (model) -- (solver);
\draw[->] (solver) -- (hangs);
\draw[->] (solver) -- (solution);
\draw[->] (solver) -- (wrong);
\draw[->] (hangs) -- (1,4) -- (human);
\draw[->] (wrong) -- (3,4) -- (human);
\end{tikzpicture}
\end{center}
\end{frame}

\begin{frame}
\frametitle{Dual Role of Model}
\begin{itemize}
\item Allows Human to Express Problem
\begin{itemize}
\item Close to Problem Domain
\item Constraints as Abstractions
\end{itemize}
\item Allows Solver to Execute
\begin{itemize}
\item Variables as Communication Mechanism
\item Constraints as Algorithms
\end{itemize}
\end{itemize}
\end{frame}

\begin{frame}
\frametitle{Modelling Frameworks}
\begin{itemize}
\item \href{https://www.minizinc.org/}{MiniZinc} (NICTA, Monash University, Australia)
\item \href{https://github.com/eomahony/Numberjack}{NumberJack} (Insight, Ireland) 
\item \href{https://www-users.york.ac.uk/peter.nightingale/savilerow//}{EssencePrime/SavilleRow} (UK) 
\item \href{https://cpmpy.readthedocs.io/en/latest/}{CPMpy} (KU Leuven)
\item Allow use of multiple back-end solvers
\item Compile model into variants for each solver
\item A priori solver independent model(CP, MIP, SAT)
\end{itemize}
\end{frame} 

\begin{frame}
\frametitle{Framework Process}
\begin{center}
\begin{tikzpicture}[xscale=2]
\node[shape=rectangle,fill=pantone127-4] (problem) at (2.5,5) {Problem};
\node (human) at (2.5,4) {Human};
\node[shape=rectangle,fill=pantone127-4] (model) at (2.5,3) {Model};
\node (compile) at (2.5,2) {Compile/Reformulate};
\node (solver1) at (1,1) {CP};
\node (solver2) at (2,1) {MIP};
\node (solver3) at (3,1) {SAT};
\node (solver4) at (4,1) {Other};
\node[shape=rectangle,fill=pantone157-8] (solution1) at (1,0) {Solution};
\node[shape=rectangle,fill=pantone157-8] (solution2) at (2,0) {Solution};
\node[shape=rectangle,fill=pantone157-8] (solution3) at (3,0) {Solution};
\node[shape=rectangle,fill=pantone157-8] (solution4) at (4,0) {Solution};
\draw[-] (problem) -- (human);
\draw[->] (human) -- (model);
\draw[-] (model) -- (compile);
\draw[-] (compile) -- (solver1);
\draw[-] (compile) -- (solver2);
\draw[-] (compile) -- (solver3);
\draw[-] (compile) -- (solver4);
\draw[->] (solver1) -- (solution1);
\draw[->] (solver2) -- (solution2);
\draw[->] (solver3) -- (solution3);
\draw[->] (solver4) -- (solution4);
\end{tikzpicture}
\end{center}
\end{frame}

% \begin{frame}
  % \frametitle{Do It Now!}
  % \begin{itemize}
  % \item Download and install Minizinc
  % \item \url{https://www.minizinc.org/}
  % \end{itemize}
% \end{frame}


\begin{frame}
\frametitle{Why use Puzzles as Examples?}
\begin{itemize}
\item Easy to understand the problem
\item Solvable by hand without specialized knowledge
\item Possible to compare automated to manual solving process
\end{itemize}

\begin{quote}
The puzzle, though inanimate, is presented as a solvable problem without lasting negative consequences, a very low-risk low-reward situation. By being a puzzle, the object is attempting to convince the user that it must be completed.
\end{quote}
{\tiny Source: \href{https://www.everydayrhetoric.com/2020/03/the-rhetoric-of-puzzles/}{Every Day Rhetoric}}
\end{frame}


\mode<all>{
\part{Basic Constraint Propagation}
}
\begin{frame}
\partpage
\end{frame}
\mode<all>{
\input{../sendmore/body}
}

\mode<all>{
\part{Global Constraints}
}
\begin{frame}
\partpage
\end{frame}
\mode<all>{
\input{../sudoku/body}
}

\clearpage
\mode<all>{
\part{Customizing Search}
}
\begin{frame}
\partpage
\end{frame}
\mode<all>{
\input{../nqueen/body}
}


%% \clearpage
%% \mode<all>{
%%   \part{Problem Decomposition}
%% }
%% \begin{frame}
%%   \partpage
%% \end{frame}
%% \mode<all>{
%%   \input{../party/body}
%%   }

%% \clearpage
%% \mode<all>{
%%   \part{More Global Constraints}
%% }
%% \begin{frame}
%%   \partpage
%% \end{frame}
%% \mode<all>{
%%   \input{../car/body}
%%   }

%% \clearpage
%% \mode<all>{
%%   \part{Methodology}
%% }
%% \begin{frame}
%%   \partpage
%% \end{frame}
%% \mode<all>{
%%   \input{../methodology/body}
%%   }

\clearpage
\mode<all>{
\part{What is missing?}
}
\begin{frame}
\partpage
\end{frame}

\begin{frame}
\frametitle{Many Specialized Topics}
\begin{itemize}
\item How to design efficient core engine
\item Hybrids with LP/MIP tools
\item Hybrids with SAT
\item Symmetry breaking
\item Use of MDD/BDD to encode sets of solutions
\item High level modelling tools
\item Debugging/visualization
\end{itemize}
\end{frame}

\begin{frame}
\frametitle{Reformulation}
\begin{itemize}
\item Just because the user has modelled it this way, it doesn't mean we have to solve it that way
\begin{itemize}
\item Replace some constraint(s) by other, equivalent constraints
\item Because we don't have that constraint in our system
\item For performance
\end{itemize}
\end{itemize}
\end{frame}

\begin{frame}
\frametitle{Learning}
\begin{itemize}
\item While solving the problem we can learn how to strengthen the model/search 
\begin{itemize}
\item Understand which constraints/method contribute to propagation and change schedule
\item Learn no-good constraints by explaining failure
\item Adapt search strategy based on search experience
\end{itemize}
\end{itemize}
\end{frame}

%% \begin{frame}
%% \frametitle{Refined Process}
%% \begin{center}
%% \begin{tikzpicture}
%% \node[shape=rectangle,fill=pantone127-4] (problem) at (3,7) {Problem};
%% \node (human) at (3,6) {Human};
%% \node[shape=rectangle,fill=pantone127-4] (model) at (3,5) {User Model};
%% \node (reformulation) at (3,4) {Reformulation};
%% \node[shape=rectangle,fill=pantone127-4] (implementation) at (3,3) {Implementation Model};
%% \node (solver) at (3,2) {Constraint Solver/Search};
%% \node[shape=rectangle,fill=pantone157-8] (solution) at (3,1) {Solution};
%% \draw[-] (problem) -- (human);
%% \draw[->] (human) -- (model);
%% \draw[-] (model) -- (reformulation);
%% \draw[->] (reformulation) -- (implementation);
%% \draw[-] (implementation) -- (solver);
%% \draw[->] (solver) -- (0,2) -- (0,3) -- (implementation);
%% \draw[->] (solver) -- (solution);
%% \end{tikzpicture}
%% \end{center}
%% \end{frame}

%\begin{frame}
%\frametitle{What is CP actually used for?}
%\url{http://hsimonis.wordpress.com}
%\includegraphics[width=10cm]{../introduction/images/blog}
%\end{frame}

%% \begin{frame}
%% \frametitle{Exercises}
%% \begin{itemize}
%% \item Install Constraint Solver of your Choice
%% \item MiniZinc (MiningZinc) will be used in later sessions
%% \item Run Sendmoremoney, Sudoku, Queens examples
%% \item Can you choose which propagator to use for alldifferent?
%% \item Can you use different search strategies?
%% \end{itemize}
%% \end{frame}

\begin{frame}
\frametitle{More Learning Resources}
\begin{itemize}
\item Survey of Methods, Resources, and Formats for Teaching Constraint Programming
\begin{itemize}
\item by Tejas Santanam, Helmut Simonis
\item \url{https://doi.org/10.48550/arXiv.2403.12717}
\item Based on survey of community for WTCP 2023
\item \url{https://hsimonis.github.io/WTCP2023/}
\end{itemize}
\end{itemize}
\end{frame}


%%   }

\clearpage
\mode<all>{
\part{What is missing?}
}
\begin{frame}
\partpage
\end{frame}

\begin{frame}
\frametitle{Many Specialized Topics}
\begin{itemize}
\item How to design efficient core engine
\item Hybrids with LP/MIP tools
\item Hybrids with SAT
\item Symmetry breaking
\item Use of MDD/BDD to encode sets of solutions
\item High level modelling tools
\item Debugging/visualization
\end{itemize}
\end{frame}

\begin{frame}
\frametitle{Reformulation}
\begin{itemize}
\item Just because the user has modelled it this way, it doesn't mean we have to solve it that way
\begin{itemize}
\item Replace some constraint(s) by other, equivalent constraints
\item Because we don't have that constraint in our system
\item For performance
\end{itemize}
\end{itemize}
\end{frame}

\begin{frame}
\frametitle{Learning}
\begin{itemize}
\item While solving the problem we can learn how to strengthen the model/search 
\begin{itemize}
\item Understand which constraints/method contribute to propagation and change schedule
\item Learn no-good constraints by explaining failure
\item Adapt search strategy based on search experience
\end{itemize}
\end{itemize}
\end{frame}

%% \begin{frame}
%% \frametitle{Refined Process}
%% \begin{center}
%% \begin{tikzpicture}
%% \node[shape=rectangle,fill=pantone127-4] (problem) at (3,7) {Problem};
%% \node (human) at (3,6) {Human};
%% \node[shape=rectangle,fill=pantone127-4] (model) at (3,5) {User Model};
%% \node (reformulation) at (3,4) {Reformulation};
%% \node[shape=rectangle,fill=pantone127-4] (implementation) at (3,3) {Implementation Model};
%% \node (solver) at (3,2) {Constraint Solver/Search};
%% \node[shape=rectangle,fill=pantone157-8] (solution) at (3,1) {Solution};
%% \draw[-] (problem) -- (human);
%% \draw[->] (human) -- (model);
%% \draw[-] (model) -- (reformulation);
%% \draw[->] (reformulation) -- (implementation);
%% \draw[-] (implementation) -- (solver);
%% \draw[->] (solver) -- (0,2) -- (0,3) -- (implementation);
%% \draw[->] (solver) -- (solution);
%% \end{tikzpicture}
%% \end{center}
%% \end{frame}

%\begin{frame}
%\frametitle{What is CP actually used for?}
%\url{http://hsimonis.wordpress.com}
%\includegraphics[width=10cm]{../introduction/images/blog}
%\end{frame}

%% \begin{frame}
%% \frametitle{Exercises}
%% \begin{itemize}
%% \item Install Constraint Solver of your Choice
%% \item MiniZinc (MiningZinc) will be used in later sessions
%% \item Run Sendmoremoney, Sudoku, Queens examples
%% \item Can you choose which propagator to use for alldifferent?
%% \item Can you use different search strategies?
%% \end{itemize}
%% \end{frame}

\begin{frame}
\frametitle{More Learning Resources}
\begin{itemize}
\item Survey of Methods, Resources, and Formats for Teaching Constraint Programming
\begin{itemize}
\item by Tejas Santanam, Helmut Simonis
\item \url{https://doi.org/10.48550/arXiv.2403.12717}
\item Based on survey of community for WTCP 2023
\item \url{https://hsimonis.github.io/WTCP2023/}
\end{itemize}
\end{itemize}
\end{frame}


%%   }

%% \clearpage
%% \mode<all>{
%%   \part{Methodology}
%% }
%% \begin{frame}
%%   \partpage
%% \end{frame}
%% \mode<all>{
%%   \begin{frame}
\frametitle{Objectives}
\begin{itemize}
\item Overview of Core Constraint Programming
\item Three Main Concepts
\begin{itemize}
\item Constraint Propagation
\item Global Constraints
\item Customizing Search
\end{itemize}
%\item Get Some Experience with MiniZinc
\item Topics will be treated in more detail in later parts of the school
\item Based on Examples, not Formal Description
\end{itemize}
\end{frame}

\begin{frame}
\frametitle{Outline}
\begin{itemize}
\item Why Constraint Programming?
\item Constraint Propagation
\item Global Constraints
\item Customizing Search
%\item Choices in Modelling
%\item Decomposition
%\item More Global Constraints
%\item Methodology
%\item What is missing?
\end{itemize}
\end{frame}

% \begin{frame}
  % \frametitle{Using MiniZinc IDE}
  % \begin{itemize}
  % \item Developed in the Australian NICTA project
  % \item Maintained by Monash University
  % \item Modelling tool with multiple back-end solvers
  % \item Available from \url{https://www.minizinc.org/}
  % \end{itemize}
% \end{frame}


% \begin{frame}
% \frametitle{Examples in ECLiPSe}
% \begin{itemize}
% \item Open sourced constraint programming language
% \item Development goes back to 1985
% \item ECRC, ICL, IC-Parc, PTL, Cisco
% \item \url{https://eclipseclp.org/}
% \item Specialities
% \begin{itemize}
% \item Develop new solvers for specific domains
% \item Integration with MIP
% \end{itemize}
% \item Not included in bundled MiniZinc IDE
% \item Specialized visualization tools used here
  % \begin{itemize}
  % \item CP-Viz, Simonis et al. 2010
    % \end{itemize}
% \end{itemize}
% \end{frame}

\begin{frame}
\frametitle{Tutorial Based on ECLiPSe ELearning Course}
\begin{itemize}
\item Self-study course in constraint programming
\item Supported by Cisco Systems and Silicon Valley Community Foundation
\item Multi-media format, video lectures, slides, handout etc
\item {\small \url{https://eclipseclp.org/ELearning/index.html}}
\end{itemize}
\end{frame}

\begin{frame}
\frametitle{Also Part of CRT-AI Constraint Week}
\begin{itemize}
\item Annual one week course on CP and Optimization in Ireland
\item Part of national training program for PhD students in AI
\item \url{https://www.crt-ai.ie/}
\end{itemize}
\end{frame}



\begin{frame}
\frametitle{Constraint Programming - in a nutshell}
\begin{itemize}
\item Declarative description of problems with
\begin{itemize}
\item {\em Variables} which range over (finite) sets of values
\item {\em Constraints} over subsets of variables which restrict possible value combinations
\item A {\em solution} is a value assignment which satisfies all constraints
\end{itemize}

\item Constraint propagation/reasoning
\begin{itemize}
\item Removing inconsistent values for variables
\item Detect failure if constraint can not be satisfied
\item Interaction of constraints via shared variables
\item Incomplete
\end{itemize}

\item Search
\begin{itemize}
\item User controlled assignment of values to variables
\item Each step triggers constraint propagation 
\end{itemize}
\item Different domains require/allow different methods
\end{itemize}
\end{frame}

\begin{frame}
  \frametitle{Constraint Programming is Different}
  \begin{itemize}
  \item Declarative Programming
    \begin{itemize}
    \item Concentrate on what you want
      \item Not how to get there
      \item Program != Algorithm
      \item Program = Model
    \end{itemize}
    \item Applied to Combinatorial Problems
      \begin{itemize}
        \item No complete polynomial algorithms known (exist?)
        \item CP less ad-hoc than heuristics
        \item Models can evolve
  \end{itemize}
  \end{itemize}
  \end{frame}
    
\begin{frame}
  \frametitle{A Subtractive Process}
  \begin{textblock}{4}(8,-3)
    \includegraphics[width=4cm]{../introduction/images/stages}
  \end{textblock}
  \vfill
  \begin{quote}
    ``Oh, bosh, as Mr. Ruskin says. Sculpture, per se, is the simplest thing in the world. All you have to do is to take a big chunk of marble and a hammer and chisel, make up your mind what you are about to create and chip off all the marble you don't want.''-Paris Gaulois.
  \end{quote}
  
  {\tiny Source: \url{https://quoteinvestigator.com/2014/06/22/chip-away/}}
\end{frame}

\begin{frame}
\frametitle{Basic Process}
\begin{center}
\begin{tikzpicture}
\node[shape=rectangle,fill=pantone127-4] (problem) at (2,5) {Problem};
\node (human) at (2,4) {Human};
\node[shape=rectangle,fill=pantone127-4] (model) at (2,3) {Model};
\node (solver) at (2,2) {Constraint Solver/Search};
\node[shape=rectangle,fill=pantone157-8] (solution) at (2,1) {Solution};
\draw[-] (problem) -- (human);
\draw[->] (human) -- (model);
\draw[-] (model) -- (solver);
\draw[->] (solver) -- (solution);
\end{tikzpicture}
\end{center}
\end{frame}

\begin{frame}
\frametitle{More Realistic}
\begin{center}
\begin{tikzpicture}[xscale=4.0]
\node[shape=rectangle,fill=pantone127-4] (problem) at (2,5) {Problem};
\node (human) at (2,4) {Human};
\node[shape=rectangle,fill=pantone127-4] (model) at (2,3) {Model};
\node (solver) at (2,2) {Constraint Solver/Search};
\node[shape=rectangle,fill=pantone24-8] (hangs) at (1,1) {Hangs};
\node[shape=rectangle,fill=pantone157-8] (solution) at (2,1) {Solution};
\node[shape=rectangle,fill=pantone24-8] (wrong) at (3,1) {Wrong Solution};
\draw[-] (problem) -- (human);
\draw[->] (human) -- (model);
\draw[-] (model) -- (solver);
\draw[->] (solver) -- (hangs);
\draw[->] (solver) -- (solution);
\draw[->] (solver) -- (wrong);
\draw[->] (hangs) -- (1,4) -- (human);
\draw[->] (wrong) -- (3,4) -- (human);
\end{tikzpicture}
\end{center}
\end{frame}

\begin{frame}
\frametitle{Dual Role of Model}
\begin{itemize}
\item Allows Human to Express Problem
\begin{itemize}
\item Close to Problem Domain
\item Constraints as Abstractions
\end{itemize}
\item Allows Solver to Execute
\begin{itemize}
\item Variables as Communication Mechanism
\item Constraints as Algorithms
\end{itemize}
\end{itemize}
\end{frame}

\begin{frame}
\frametitle{Modelling Frameworks}
\begin{itemize}
\item \href{https://www.minizinc.org/}{MiniZinc} (NICTA, Monash University, Australia)
\item \href{https://github.com/eomahony/Numberjack}{NumberJack} (Insight, Ireland) 
\item \href{https://www-users.york.ac.uk/peter.nightingale/savilerow//}{EssencePrime/SavilleRow} (UK) 
\item \href{https://cpmpy.readthedocs.io/en/latest/}{CPMpy} (KU Leuven)
\item Allow use of multiple back-end solvers
\item Compile model into variants for each solver
\item A priori solver independent model(CP, MIP, SAT)
\end{itemize}
\end{frame} 

\begin{frame}
\frametitle{Framework Process}
\begin{center}
\begin{tikzpicture}[xscale=2]
\node[shape=rectangle,fill=pantone127-4] (problem) at (2.5,5) {Problem};
\node (human) at (2.5,4) {Human};
\node[shape=rectangle,fill=pantone127-4] (model) at (2.5,3) {Model};
\node (compile) at (2.5,2) {Compile/Reformulate};
\node (solver1) at (1,1) {CP};
\node (solver2) at (2,1) {MIP};
\node (solver3) at (3,1) {SAT};
\node (solver4) at (4,1) {Other};
\node[shape=rectangle,fill=pantone157-8] (solution1) at (1,0) {Solution};
\node[shape=rectangle,fill=pantone157-8] (solution2) at (2,0) {Solution};
\node[shape=rectangle,fill=pantone157-8] (solution3) at (3,0) {Solution};
\node[shape=rectangle,fill=pantone157-8] (solution4) at (4,0) {Solution};
\draw[-] (problem) -- (human);
\draw[->] (human) -- (model);
\draw[-] (model) -- (compile);
\draw[-] (compile) -- (solver1);
\draw[-] (compile) -- (solver2);
\draw[-] (compile) -- (solver3);
\draw[-] (compile) -- (solver4);
\draw[->] (solver1) -- (solution1);
\draw[->] (solver2) -- (solution2);
\draw[->] (solver3) -- (solution3);
\draw[->] (solver4) -- (solution4);
\end{tikzpicture}
\end{center}
\end{frame}

% \begin{frame}
  % \frametitle{Do It Now!}
  % \begin{itemize}
  % \item Download and install Minizinc
  % \item \url{https://www.minizinc.org/}
  % \end{itemize}
% \end{frame}


\begin{frame}
\frametitle{Why use Puzzles as Examples?}
\begin{itemize}
\item Easy to understand the problem
\item Solvable by hand without specialized knowledge
\item Possible to compare automated to manual solving process
\end{itemize}

\begin{quote}
The puzzle, though inanimate, is presented as a solvable problem without lasting negative consequences, a very low-risk low-reward situation. By being a puzzle, the object is attempting to convince the user that it must be completed.
\end{quote}
{\tiny Source: \href{https://www.everydayrhetoric.com/2020/03/the-rhetoric-of-puzzles/}{Every Day Rhetoric}}
\end{frame}


\mode<all>{
\part{Basic Constraint Propagation}
}
\begin{frame}
\partpage
\end{frame}
\mode<all>{
\begin{frame}
\frametitle{Objectives}
\begin{itemize}
\item Overview of Core Constraint Programming
\item Three Main Concepts
\begin{itemize}
\item Constraint Propagation
\item Global Constraints
\item Customizing Search
\end{itemize}
%\item Get Some Experience with MiniZinc
\item Topics will be treated in more detail in later parts of the school
\item Based on Examples, not Formal Description
\end{itemize}
\end{frame}

\begin{frame}
\frametitle{Outline}
\begin{itemize}
\item Why Constraint Programming?
\item Constraint Propagation
\item Global Constraints
\item Customizing Search
%\item Choices in Modelling
%\item Decomposition
%\item More Global Constraints
%\item Methodology
%\item What is missing?
\end{itemize}
\end{frame}

% \begin{frame}
  % \frametitle{Using MiniZinc IDE}
  % \begin{itemize}
  % \item Developed in the Australian NICTA project
  % \item Maintained by Monash University
  % \item Modelling tool with multiple back-end solvers
  % \item Available from \url{https://www.minizinc.org/}
  % \end{itemize}
% \end{frame}


% \begin{frame}
% \frametitle{Examples in ECLiPSe}
% \begin{itemize}
% \item Open sourced constraint programming language
% \item Development goes back to 1985
% \item ECRC, ICL, IC-Parc, PTL, Cisco
% \item \url{https://eclipseclp.org/}
% \item Specialities
% \begin{itemize}
% \item Develop new solvers for specific domains
% \item Integration with MIP
% \end{itemize}
% \item Not included in bundled MiniZinc IDE
% \item Specialized visualization tools used here
  % \begin{itemize}
  % \item CP-Viz, Simonis et al. 2010
    % \end{itemize}
% \end{itemize}
% \end{frame}

\begin{frame}
\frametitle{Tutorial Based on ECLiPSe ELearning Course}
\begin{itemize}
\item Self-study course in constraint programming
\item Supported by Cisco Systems and Silicon Valley Community Foundation
\item Multi-media format, video lectures, slides, handout etc
\item {\small \url{https://eclipseclp.org/ELearning/index.html}}
\end{itemize}
\end{frame}

\begin{frame}
\frametitle{Also Part of CRT-AI Constraint Week}
\begin{itemize}
\item Annual one week course on CP and Optimization in Ireland
\item Part of national training program for PhD students in AI
\item \url{https://www.crt-ai.ie/}
\end{itemize}
\end{frame}



\begin{frame}
\frametitle{Constraint Programming - in a nutshell}
\begin{itemize}
\item Declarative description of problems with
\begin{itemize}
\item {\em Variables} which range over (finite) sets of values
\item {\em Constraints} over subsets of variables which restrict possible value combinations
\item A {\em solution} is a value assignment which satisfies all constraints
\end{itemize}

\item Constraint propagation/reasoning
\begin{itemize}
\item Removing inconsistent values for variables
\item Detect failure if constraint can not be satisfied
\item Interaction of constraints via shared variables
\item Incomplete
\end{itemize}

\item Search
\begin{itemize}
\item User controlled assignment of values to variables
\item Each step triggers constraint propagation 
\end{itemize}
\item Different domains require/allow different methods
\end{itemize}
\end{frame}

\begin{frame}
  \frametitle{Constraint Programming is Different}
  \begin{itemize}
  \item Declarative Programming
    \begin{itemize}
    \item Concentrate on what you want
      \item Not how to get there
      \item Program != Algorithm
      \item Program = Model
    \end{itemize}
    \item Applied to Combinatorial Problems
      \begin{itemize}
        \item No complete polynomial algorithms known (exist?)
        \item CP less ad-hoc than heuristics
        \item Models can evolve
  \end{itemize}
  \end{itemize}
  \end{frame}
    
\begin{frame}
  \frametitle{A Subtractive Process}
  \begin{textblock}{4}(8,-3)
    \includegraphics[width=4cm]{../introduction/images/stages}
  \end{textblock}
  \vfill
  \begin{quote}
    ``Oh, bosh, as Mr. Ruskin says. Sculpture, per se, is the simplest thing in the world. All you have to do is to take a big chunk of marble and a hammer and chisel, make up your mind what you are about to create and chip off all the marble you don't want.''-Paris Gaulois.
  \end{quote}
  
  {\tiny Source: \url{https://quoteinvestigator.com/2014/06/22/chip-away/}}
\end{frame}

\begin{frame}
\frametitle{Basic Process}
\begin{center}
\begin{tikzpicture}
\node[shape=rectangle,fill=pantone127-4] (problem) at (2,5) {Problem};
\node (human) at (2,4) {Human};
\node[shape=rectangle,fill=pantone127-4] (model) at (2,3) {Model};
\node (solver) at (2,2) {Constraint Solver/Search};
\node[shape=rectangle,fill=pantone157-8] (solution) at (2,1) {Solution};
\draw[-] (problem) -- (human);
\draw[->] (human) -- (model);
\draw[-] (model) -- (solver);
\draw[->] (solver) -- (solution);
\end{tikzpicture}
\end{center}
\end{frame}

\begin{frame}
\frametitle{More Realistic}
\begin{center}
\begin{tikzpicture}[xscale=4.0]
\node[shape=rectangle,fill=pantone127-4] (problem) at (2,5) {Problem};
\node (human) at (2,4) {Human};
\node[shape=rectangle,fill=pantone127-4] (model) at (2,3) {Model};
\node (solver) at (2,2) {Constraint Solver/Search};
\node[shape=rectangle,fill=pantone24-8] (hangs) at (1,1) {Hangs};
\node[shape=rectangle,fill=pantone157-8] (solution) at (2,1) {Solution};
\node[shape=rectangle,fill=pantone24-8] (wrong) at (3,1) {Wrong Solution};
\draw[-] (problem) -- (human);
\draw[->] (human) -- (model);
\draw[-] (model) -- (solver);
\draw[->] (solver) -- (hangs);
\draw[->] (solver) -- (solution);
\draw[->] (solver) -- (wrong);
\draw[->] (hangs) -- (1,4) -- (human);
\draw[->] (wrong) -- (3,4) -- (human);
\end{tikzpicture}
\end{center}
\end{frame}

\begin{frame}
\frametitle{Dual Role of Model}
\begin{itemize}
\item Allows Human to Express Problem
\begin{itemize}
\item Close to Problem Domain
\item Constraints as Abstractions
\end{itemize}
\item Allows Solver to Execute
\begin{itemize}
\item Variables as Communication Mechanism
\item Constraints as Algorithms
\end{itemize}
\end{itemize}
\end{frame}

\begin{frame}
\frametitle{Modelling Frameworks}
\begin{itemize}
\item \href{https://www.minizinc.org/}{MiniZinc} (NICTA, Monash University, Australia)
\item \href{https://github.com/eomahony/Numberjack}{NumberJack} (Insight, Ireland) 
\item \href{https://www-users.york.ac.uk/peter.nightingale/savilerow//}{EssencePrime/SavilleRow} (UK) 
\item \href{https://cpmpy.readthedocs.io/en/latest/}{CPMpy} (KU Leuven)
\item Allow use of multiple back-end solvers
\item Compile model into variants for each solver
\item A priori solver independent model(CP, MIP, SAT)
\end{itemize}
\end{frame} 

\begin{frame}
\frametitle{Framework Process}
\begin{center}
\begin{tikzpicture}[xscale=2]
\node[shape=rectangle,fill=pantone127-4] (problem) at (2.5,5) {Problem};
\node (human) at (2.5,4) {Human};
\node[shape=rectangle,fill=pantone127-4] (model) at (2.5,3) {Model};
\node (compile) at (2.5,2) {Compile/Reformulate};
\node (solver1) at (1,1) {CP};
\node (solver2) at (2,1) {MIP};
\node (solver3) at (3,1) {SAT};
\node (solver4) at (4,1) {Other};
\node[shape=rectangle,fill=pantone157-8] (solution1) at (1,0) {Solution};
\node[shape=rectangle,fill=pantone157-8] (solution2) at (2,0) {Solution};
\node[shape=rectangle,fill=pantone157-8] (solution3) at (3,0) {Solution};
\node[shape=rectangle,fill=pantone157-8] (solution4) at (4,0) {Solution};
\draw[-] (problem) -- (human);
\draw[->] (human) -- (model);
\draw[-] (model) -- (compile);
\draw[-] (compile) -- (solver1);
\draw[-] (compile) -- (solver2);
\draw[-] (compile) -- (solver3);
\draw[-] (compile) -- (solver4);
\draw[->] (solver1) -- (solution1);
\draw[->] (solver2) -- (solution2);
\draw[->] (solver3) -- (solution3);
\draw[->] (solver4) -- (solution4);
\end{tikzpicture}
\end{center}
\end{frame}

% \begin{frame}
  % \frametitle{Do It Now!}
  % \begin{itemize}
  % \item Download and install Minizinc
  % \item \url{https://www.minizinc.org/}
  % \end{itemize}
% \end{frame}


\begin{frame}
\frametitle{Why use Puzzles as Examples?}
\begin{itemize}
\item Easy to understand the problem
\item Solvable by hand without specialized knowledge
\item Possible to compare automated to manual solving process
\end{itemize}

\begin{quote}
The puzzle, though inanimate, is presented as a solvable problem without lasting negative consequences, a very low-risk low-reward situation. By being a puzzle, the object is attempting to convince the user that it must be completed.
\end{quote}
{\tiny Source: \href{https://www.everydayrhetoric.com/2020/03/the-rhetoric-of-puzzles/}{Every Day Rhetoric}}
\end{frame}


\mode<all>{
\part{Basic Constraint Propagation}
}
\begin{frame}
\partpage
\end{frame}
\mode<all>{
\input{../sendmore/body}
}

\mode<all>{
\part{Global Constraints}
}
\begin{frame}
\partpage
\end{frame}
\mode<all>{
\input{../sudoku/body}
}

\clearpage
\mode<all>{
\part{Customizing Search}
}
\begin{frame}
\partpage
\end{frame}
\mode<all>{
\input{../nqueen/body}
}


%% \clearpage
%% \mode<all>{
%%   \part{Problem Decomposition}
%% }
%% \begin{frame}
%%   \partpage
%% \end{frame}
%% \mode<all>{
%%   \input{../party/body}
%%   }

%% \clearpage
%% \mode<all>{
%%   \part{More Global Constraints}
%% }
%% \begin{frame}
%%   \partpage
%% \end{frame}
%% \mode<all>{
%%   \input{../car/body}
%%   }

%% \clearpage
%% \mode<all>{
%%   \part{Methodology}
%% }
%% \begin{frame}
%%   \partpage
%% \end{frame}
%% \mode<all>{
%%   \input{../methodology/body}
%%   }

\clearpage
\mode<all>{
\part{What is missing?}
}
\begin{frame}
\partpage
\end{frame}

\begin{frame}
\frametitle{Many Specialized Topics}
\begin{itemize}
\item How to design efficient core engine
\item Hybrids with LP/MIP tools
\item Hybrids with SAT
\item Symmetry breaking
\item Use of MDD/BDD to encode sets of solutions
\item High level modelling tools
\item Debugging/visualization
\end{itemize}
\end{frame}

\begin{frame}
\frametitle{Reformulation}
\begin{itemize}
\item Just because the user has modelled it this way, it doesn't mean we have to solve it that way
\begin{itemize}
\item Replace some constraint(s) by other, equivalent constraints
\item Because we don't have that constraint in our system
\item For performance
\end{itemize}
\end{itemize}
\end{frame}

\begin{frame}
\frametitle{Learning}
\begin{itemize}
\item While solving the problem we can learn how to strengthen the model/search 
\begin{itemize}
\item Understand which constraints/method contribute to propagation and change schedule
\item Learn no-good constraints by explaining failure
\item Adapt search strategy based on search experience
\end{itemize}
\end{itemize}
\end{frame}

%% \begin{frame}
%% \frametitle{Refined Process}
%% \begin{center}
%% \begin{tikzpicture}
%% \node[shape=rectangle,fill=pantone127-4] (problem) at (3,7) {Problem};
%% \node (human) at (3,6) {Human};
%% \node[shape=rectangle,fill=pantone127-4] (model) at (3,5) {User Model};
%% \node (reformulation) at (3,4) {Reformulation};
%% \node[shape=rectangle,fill=pantone127-4] (implementation) at (3,3) {Implementation Model};
%% \node (solver) at (3,2) {Constraint Solver/Search};
%% \node[shape=rectangle,fill=pantone157-8] (solution) at (3,1) {Solution};
%% \draw[-] (problem) -- (human);
%% \draw[->] (human) -- (model);
%% \draw[-] (model) -- (reformulation);
%% \draw[->] (reformulation) -- (implementation);
%% \draw[-] (implementation) -- (solver);
%% \draw[->] (solver) -- (0,2) -- (0,3) -- (implementation);
%% \draw[->] (solver) -- (solution);
%% \end{tikzpicture}
%% \end{center}
%% \end{frame}

%\begin{frame}
%\frametitle{What is CP actually used for?}
%\url{http://hsimonis.wordpress.com}
%\includegraphics[width=10cm]{../introduction/images/blog}
%\end{frame}

%% \begin{frame}
%% \frametitle{Exercises}
%% \begin{itemize}
%% \item Install Constraint Solver of your Choice
%% \item MiniZinc (MiningZinc) will be used in later sessions
%% \item Run Sendmoremoney, Sudoku, Queens examples
%% \item Can you choose which propagator to use for alldifferent?
%% \item Can you use different search strategies?
%% \end{itemize}
%% \end{frame}

\begin{frame}
\frametitle{More Learning Resources}
\begin{itemize}
\item Survey of Methods, Resources, and Formats for Teaching Constraint Programming
\begin{itemize}
\item by Tejas Santanam, Helmut Simonis
\item \url{https://doi.org/10.48550/arXiv.2403.12717}
\item Based on survey of community for WTCP 2023
\item \url{https://hsimonis.github.io/WTCP2023/}
\end{itemize}
\end{itemize}
\end{frame}


}

\mode<all>{
\part{Global Constraints}
}
\begin{frame}
\partpage
\end{frame}
\mode<all>{
\begin{frame}
\frametitle{Objectives}
\begin{itemize}
\item Overview of Core Constraint Programming
\item Three Main Concepts
\begin{itemize}
\item Constraint Propagation
\item Global Constraints
\item Customizing Search
\end{itemize}
%\item Get Some Experience with MiniZinc
\item Topics will be treated in more detail in later parts of the school
\item Based on Examples, not Formal Description
\end{itemize}
\end{frame}

\begin{frame}
\frametitle{Outline}
\begin{itemize}
\item Why Constraint Programming?
\item Constraint Propagation
\item Global Constraints
\item Customizing Search
%\item Choices in Modelling
%\item Decomposition
%\item More Global Constraints
%\item Methodology
%\item What is missing?
\end{itemize}
\end{frame}

% \begin{frame}
  % \frametitle{Using MiniZinc IDE}
  % \begin{itemize}
  % \item Developed in the Australian NICTA project
  % \item Maintained by Monash University
  % \item Modelling tool with multiple back-end solvers
  % \item Available from \url{https://www.minizinc.org/}
  % \end{itemize}
% \end{frame}


% \begin{frame}
% \frametitle{Examples in ECLiPSe}
% \begin{itemize}
% \item Open sourced constraint programming language
% \item Development goes back to 1985
% \item ECRC, ICL, IC-Parc, PTL, Cisco
% \item \url{https://eclipseclp.org/}
% \item Specialities
% \begin{itemize}
% \item Develop new solvers for specific domains
% \item Integration with MIP
% \end{itemize}
% \item Not included in bundled MiniZinc IDE
% \item Specialized visualization tools used here
  % \begin{itemize}
  % \item CP-Viz, Simonis et al. 2010
    % \end{itemize}
% \end{itemize}
% \end{frame}

\begin{frame}
\frametitle{Tutorial Based on ECLiPSe ELearning Course}
\begin{itemize}
\item Self-study course in constraint programming
\item Supported by Cisco Systems and Silicon Valley Community Foundation
\item Multi-media format, video lectures, slides, handout etc
\item {\small \url{https://eclipseclp.org/ELearning/index.html}}
\end{itemize}
\end{frame}

\begin{frame}
\frametitle{Also Part of CRT-AI Constraint Week}
\begin{itemize}
\item Annual one week course on CP and Optimization in Ireland
\item Part of national training program for PhD students in AI
\item \url{https://www.crt-ai.ie/}
\end{itemize}
\end{frame}



\begin{frame}
\frametitle{Constraint Programming - in a nutshell}
\begin{itemize}
\item Declarative description of problems with
\begin{itemize}
\item {\em Variables} which range over (finite) sets of values
\item {\em Constraints} over subsets of variables which restrict possible value combinations
\item A {\em solution} is a value assignment which satisfies all constraints
\end{itemize}

\item Constraint propagation/reasoning
\begin{itemize}
\item Removing inconsistent values for variables
\item Detect failure if constraint can not be satisfied
\item Interaction of constraints via shared variables
\item Incomplete
\end{itemize}

\item Search
\begin{itemize}
\item User controlled assignment of values to variables
\item Each step triggers constraint propagation 
\end{itemize}
\item Different domains require/allow different methods
\end{itemize}
\end{frame}

\begin{frame}
  \frametitle{Constraint Programming is Different}
  \begin{itemize}
  \item Declarative Programming
    \begin{itemize}
    \item Concentrate on what you want
      \item Not how to get there
      \item Program != Algorithm
      \item Program = Model
    \end{itemize}
    \item Applied to Combinatorial Problems
      \begin{itemize}
        \item No complete polynomial algorithms known (exist?)
        \item CP less ad-hoc than heuristics
        \item Models can evolve
  \end{itemize}
  \end{itemize}
  \end{frame}
    
\begin{frame}
  \frametitle{A Subtractive Process}
  \begin{textblock}{4}(8,-3)
    \includegraphics[width=4cm]{../introduction/images/stages}
  \end{textblock}
  \vfill
  \begin{quote}
    ``Oh, bosh, as Mr. Ruskin says. Sculpture, per se, is the simplest thing in the world. All you have to do is to take a big chunk of marble and a hammer and chisel, make up your mind what you are about to create and chip off all the marble you don't want.''-Paris Gaulois.
  \end{quote}
  
  {\tiny Source: \url{https://quoteinvestigator.com/2014/06/22/chip-away/}}
\end{frame}

\begin{frame}
\frametitle{Basic Process}
\begin{center}
\begin{tikzpicture}
\node[shape=rectangle,fill=pantone127-4] (problem) at (2,5) {Problem};
\node (human) at (2,4) {Human};
\node[shape=rectangle,fill=pantone127-4] (model) at (2,3) {Model};
\node (solver) at (2,2) {Constraint Solver/Search};
\node[shape=rectangle,fill=pantone157-8] (solution) at (2,1) {Solution};
\draw[-] (problem) -- (human);
\draw[->] (human) -- (model);
\draw[-] (model) -- (solver);
\draw[->] (solver) -- (solution);
\end{tikzpicture}
\end{center}
\end{frame}

\begin{frame}
\frametitle{More Realistic}
\begin{center}
\begin{tikzpicture}[xscale=4.0]
\node[shape=rectangle,fill=pantone127-4] (problem) at (2,5) {Problem};
\node (human) at (2,4) {Human};
\node[shape=rectangle,fill=pantone127-4] (model) at (2,3) {Model};
\node (solver) at (2,2) {Constraint Solver/Search};
\node[shape=rectangle,fill=pantone24-8] (hangs) at (1,1) {Hangs};
\node[shape=rectangle,fill=pantone157-8] (solution) at (2,1) {Solution};
\node[shape=rectangle,fill=pantone24-8] (wrong) at (3,1) {Wrong Solution};
\draw[-] (problem) -- (human);
\draw[->] (human) -- (model);
\draw[-] (model) -- (solver);
\draw[->] (solver) -- (hangs);
\draw[->] (solver) -- (solution);
\draw[->] (solver) -- (wrong);
\draw[->] (hangs) -- (1,4) -- (human);
\draw[->] (wrong) -- (3,4) -- (human);
\end{tikzpicture}
\end{center}
\end{frame}

\begin{frame}
\frametitle{Dual Role of Model}
\begin{itemize}
\item Allows Human to Express Problem
\begin{itemize}
\item Close to Problem Domain
\item Constraints as Abstractions
\end{itemize}
\item Allows Solver to Execute
\begin{itemize}
\item Variables as Communication Mechanism
\item Constraints as Algorithms
\end{itemize}
\end{itemize}
\end{frame}

\begin{frame}
\frametitle{Modelling Frameworks}
\begin{itemize}
\item \href{https://www.minizinc.org/}{MiniZinc} (NICTA, Monash University, Australia)
\item \href{https://github.com/eomahony/Numberjack}{NumberJack} (Insight, Ireland) 
\item \href{https://www-users.york.ac.uk/peter.nightingale/savilerow//}{EssencePrime/SavilleRow} (UK) 
\item \href{https://cpmpy.readthedocs.io/en/latest/}{CPMpy} (KU Leuven)
\item Allow use of multiple back-end solvers
\item Compile model into variants for each solver
\item A priori solver independent model(CP, MIP, SAT)
\end{itemize}
\end{frame} 

\begin{frame}
\frametitle{Framework Process}
\begin{center}
\begin{tikzpicture}[xscale=2]
\node[shape=rectangle,fill=pantone127-4] (problem) at (2.5,5) {Problem};
\node (human) at (2.5,4) {Human};
\node[shape=rectangle,fill=pantone127-4] (model) at (2.5,3) {Model};
\node (compile) at (2.5,2) {Compile/Reformulate};
\node (solver1) at (1,1) {CP};
\node (solver2) at (2,1) {MIP};
\node (solver3) at (3,1) {SAT};
\node (solver4) at (4,1) {Other};
\node[shape=rectangle,fill=pantone157-8] (solution1) at (1,0) {Solution};
\node[shape=rectangle,fill=pantone157-8] (solution2) at (2,0) {Solution};
\node[shape=rectangle,fill=pantone157-8] (solution3) at (3,0) {Solution};
\node[shape=rectangle,fill=pantone157-8] (solution4) at (4,0) {Solution};
\draw[-] (problem) -- (human);
\draw[->] (human) -- (model);
\draw[-] (model) -- (compile);
\draw[-] (compile) -- (solver1);
\draw[-] (compile) -- (solver2);
\draw[-] (compile) -- (solver3);
\draw[-] (compile) -- (solver4);
\draw[->] (solver1) -- (solution1);
\draw[->] (solver2) -- (solution2);
\draw[->] (solver3) -- (solution3);
\draw[->] (solver4) -- (solution4);
\end{tikzpicture}
\end{center}
\end{frame}

% \begin{frame}
  % \frametitle{Do It Now!}
  % \begin{itemize}
  % \item Download and install Minizinc
  % \item \url{https://www.minizinc.org/}
  % \end{itemize}
% \end{frame}


\begin{frame}
\frametitle{Why use Puzzles as Examples?}
\begin{itemize}
\item Easy to understand the problem
\item Solvable by hand without specialized knowledge
\item Possible to compare automated to manual solving process
\end{itemize}

\begin{quote}
The puzzle, though inanimate, is presented as a solvable problem without lasting negative consequences, a very low-risk low-reward situation. By being a puzzle, the object is attempting to convince the user that it must be completed.
\end{quote}
{\tiny Source: \href{https://www.everydayrhetoric.com/2020/03/the-rhetoric-of-puzzles/}{Every Day Rhetoric}}
\end{frame}


\mode<all>{
\part{Basic Constraint Propagation}
}
\begin{frame}
\partpage
\end{frame}
\mode<all>{
\input{../sendmore/body}
}

\mode<all>{
\part{Global Constraints}
}
\begin{frame}
\partpage
\end{frame}
\mode<all>{
\input{../sudoku/body}
}

\clearpage
\mode<all>{
\part{Customizing Search}
}
\begin{frame}
\partpage
\end{frame}
\mode<all>{
\input{../nqueen/body}
}


%% \clearpage
%% \mode<all>{
%%   \part{Problem Decomposition}
%% }
%% \begin{frame}
%%   \partpage
%% \end{frame}
%% \mode<all>{
%%   \input{../party/body}
%%   }

%% \clearpage
%% \mode<all>{
%%   \part{More Global Constraints}
%% }
%% \begin{frame}
%%   \partpage
%% \end{frame}
%% \mode<all>{
%%   \input{../car/body}
%%   }

%% \clearpage
%% \mode<all>{
%%   \part{Methodology}
%% }
%% \begin{frame}
%%   \partpage
%% \end{frame}
%% \mode<all>{
%%   \input{../methodology/body}
%%   }

\clearpage
\mode<all>{
\part{What is missing?}
}
\begin{frame}
\partpage
\end{frame}

\begin{frame}
\frametitle{Many Specialized Topics}
\begin{itemize}
\item How to design efficient core engine
\item Hybrids with LP/MIP tools
\item Hybrids with SAT
\item Symmetry breaking
\item Use of MDD/BDD to encode sets of solutions
\item High level modelling tools
\item Debugging/visualization
\end{itemize}
\end{frame}

\begin{frame}
\frametitle{Reformulation}
\begin{itemize}
\item Just because the user has modelled it this way, it doesn't mean we have to solve it that way
\begin{itemize}
\item Replace some constraint(s) by other, equivalent constraints
\item Because we don't have that constraint in our system
\item For performance
\end{itemize}
\end{itemize}
\end{frame}

\begin{frame}
\frametitle{Learning}
\begin{itemize}
\item While solving the problem we can learn how to strengthen the model/search 
\begin{itemize}
\item Understand which constraints/method contribute to propagation and change schedule
\item Learn no-good constraints by explaining failure
\item Adapt search strategy based on search experience
\end{itemize}
\end{itemize}
\end{frame}

%% \begin{frame}
%% \frametitle{Refined Process}
%% \begin{center}
%% \begin{tikzpicture}
%% \node[shape=rectangle,fill=pantone127-4] (problem) at (3,7) {Problem};
%% \node (human) at (3,6) {Human};
%% \node[shape=rectangle,fill=pantone127-4] (model) at (3,5) {User Model};
%% \node (reformulation) at (3,4) {Reformulation};
%% \node[shape=rectangle,fill=pantone127-4] (implementation) at (3,3) {Implementation Model};
%% \node (solver) at (3,2) {Constraint Solver/Search};
%% \node[shape=rectangle,fill=pantone157-8] (solution) at (3,1) {Solution};
%% \draw[-] (problem) -- (human);
%% \draw[->] (human) -- (model);
%% \draw[-] (model) -- (reformulation);
%% \draw[->] (reformulation) -- (implementation);
%% \draw[-] (implementation) -- (solver);
%% \draw[->] (solver) -- (0,2) -- (0,3) -- (implementation);
%% \draw[->] (solver) -- (solution);
%% \end{tikzpicture}
%% \end{center}
%% \end{frame}

%\begin{frame}
%\frametitle{What is CP actually used for?}
%\url{http://hsimonis.wordpress.com}
%\includegraphics[width=10cm]{../introduction/images/blog}
%\end{frame}

%% \begin{frame}
%% \frametitle{Exercises}
%% \begin{itemize}
%% \item Install Constraint Solver of your Choice
%% \item MiniZinc (MiningZinc) will be used in later sessions
%% \item Run Sendmoremoney, Sudoku, Queens examples
%% \item Can you choose which propagator to use for alldifferent?
%% \item Can you use different search strategies?
%% \end{itemize}
%% \end{frame}

\begin{frame}
\frametitle{More Learning Resources}
\begin{itemize}
\item Survey of Methods, Resources, and Formats for Teaching Constraint Programming
\begin{itemize}
\item by Tejas Santanam, Helmut Simonis
\item \url{https://doi.org/10.48550/arXiv.2403.12717}
\item Based on survey of community for WTCP 2023
\item \url{https://hsimonis.github.io/WTCP2023/}
\end{itemize}
\end{itemize}
\end{frame}


}

\clearpage
\mode<all>{
\part{Customizing Search}
}
\begin{frame}
\partpage
\end{frame}
\mode<all>{
\begin{frame}
\frametitle{Objectives}
\begin{itemize}
\item Overview of Core Constraint Programming
\item Three Main Concepts
\begin{itemize}
\item Constraint Propagation
\item Global Constraints
\item Customizing Search
\end{itemize}
%\item Get Some Experience with MiniZinc
\item Topics will be treated in more detail in later parts of the school
\item Based on Examples, not Formal Description
\end{itemize}
\end{frame}

\begin{frame}
\frametitle{Outline}
\begin{itemize}
\item Why Constraint Programming?
\item Constraint Propagation
\item Global Constraints
\item Customizing Search
%\item Choices in Modelling
%\item Decomposition
%\item More Global Constraints
%\item Methodology
%\item What is missing?
\end{itemize}
\end{frame}

% \begin{frame}
  % \frametitle{Using MiniZinc IDE}
  % \begin{itemize}
  % \item Developed in the Australian NICTA project
  % \item Maintained by Monash University
  % \item Modelling tool with multiple back-end solvers
  % \item Available from \url{https://www.minizinc.org/}
  % \end{itemize}
% \end{frame}


% \begin{frame}
% \frametitle{Examples in ECLiPSe}
% \begin{itemize}
% \item Open sourced constraint programming language
% \item Development goes back to 1985
% \item ECRC, ICL, IC-Parc, PTL, Cisco
% \item \url{https://eclipseclp.org/}
% \item Specialities
% \begin{itemize}
% \item Develop new solvers for specific domains
% \item Integration with MIP
% \end{itemize}
% \item Not included in bundled MiniZinc IDE
% \item Specialized visualization tools used here
  % \begin{itemize}
  % \item CP-Viz, Simonis et al. 2010
    % \end{itemize}
% \end{itemize}
% \end{frame}

\begin{frame}
\frametitle{Tutorial Based on ECLiPSe ELearning Course}
\begin{itemize}
\item Self-study course in constraint programming
\item Supported by Cisco Systems and Silicon Valley Community Foundation
\item Multi-media format, video lectures, slides, handout etc
\item {\small \url{https://eclipseclp.org/ELearning/index.html}}
\end{itemize}
\end{frame}

\begin{frame}
\frametitle{Also Part of CRT-AI Constraint Week}
\begin{itemize}
\item Annual one week course on CP and Optimization in Ireland
\item Part of national training program for PhD students in AI
\item \url{https://www.crt-ai.ie/}
\end{itemize}
\end{frame}



\begin{frame}
\frametitle{Constraint Programming - in a nutshell}
\begin{itemize}
\item Declarative description of problems with
\begin{itemize}
\item {\em Variables} which range over (finite) sets of values
\item {\em Constraints} over subsets of variables which restrict possible value combinations
\item A {\em solution} is a value assignment which satisfies all constraints
\end{itemize}

\item Constraint propagation/reasoning
\begin{itemize}
\item Removing inconsistent values for variables
\item Detect failure if constraint can not be satisfied
\item Interaction of constraints via shared variables
\item Incomplete
\end{itemize}

\item Search
\begin{itemize}
\item User controlled assignment of values to variables
\item Each step triggers constraint propagation 
\end{itemize}
\item Different domains require/allow different methods
\end{itemize}
\end{frame}

\begin{frame}
  \frametitle{Constraint Programming is Different}
  \begin{itemize}
  \item Declarative Programming
    \begin{itemize}
    \item Concentrate on what you want
      \item Not how to get there
      \item Program != Algorithm
      \item Program = Model
    \end{itemize}
    \item Applied to Combinatorial Problems
      \begin{itemize}
        \item No complete polynomial algorithms known (exist?)
        \item CP less ad-hoc than heuristics
        \item Models can evolve
  \end{itemize}
  \end{itemize}
  \end{frame}
    
\begin{frame}
  \frametitle{A Subtractive Process}
  \begin{textblock}{4}(8,-3)
    \includegraphics[width=4cm]{../introduction/images/stages}
  \end{textblock}
  \vfill
  \begin{quote}
    ``Oh, bosh, as Mr. Ruskin says. Sculpture, per se, is the simplest thing in the world. All you have to do is to take a big chunk of marble and a hammer and chisel, make up your mind what you are about to create and chip off all the marble you don't want.''-Paris Gaulois.
  \end{quote}
  
  {\tiny Source: \url{https://quoteinvestigator.com/2014/06/22/chip-away/}}
\end{frame}

\begin{frame}
\frametitle{Basic Process}
\begin{center}
\begin{tikzpicture}
\node[shape=rectangle,fill=pantone127-4] (problem) at (2,5) {Problem};
\node (human) at (2,4) {Human};
\node[shape=rectangle,fill=pantone127-4] (model) at (2,3) {Model};
\node (solver) at (2,2) {Constraint Solver/Search};
\node[shape=rectangle,fill=pantone157-8] (solution) at (2,1) {Solution};
\draw[-] (problem) -- (human);
\draw[->] (human) -- (model);
\draw[-] (model) -- (solver);
\draw[->] (solver) -- (solution);
\end{tikzpicture}
\end{center}
\end{frame}

\begin{frame}
\frametitle{More Realistic}
\begin{center}
\begin{tikzpicture}[xscale=4.0]
\node[shape=rectangle,fill=pantone127-4] (problem) at (2,5) {Problem};
\node (human) at (2,4) {Human};
\node[shape=rectangle,fill=pantone127-4] (model) at (2,3) {Model};
\node (solver) at (2,2) {Constraint Solver/Search};
\node[shape=rectangle,fill=pantone24-8] (hangs) at (1,1) {Hangs};
\node[shape=rectangle,fill=pantone157-8] (solution) at (2,1) {Solution};
\node[shape=rectangle,fill=pantone24-8] (wrong) at (3,1) {Wrong Solution};
\draw[-] (problem) -- (human);
\draw[->] (human) -- (model);
\draw[-] (model) -- (solver);
\draw[->] (solver) -- (hangs);
\draw[->] (solver) -- (solution);
\draw[->] (solver) -- (wrong);
\draw[->] (hangs) -- (1,4) -- (human);
\draw[->] (wrong) -- (3,4) -- (human);
\end{tikzpicture}
\end{center}
\end{frame}

\begin{frame}
\frametitle{Dual Role of Model}
\begin{itemize}
\item Allows Human to Express Problem
\begin{itemize}
\item Close to Problem Domain
\item Constraints as Abstractions
\end{itemize}
\item Allows Solver to Execute
\begin{itemize}
\item Variables as Communication Mechanism
\item Constraints as Algorithms
\end{itemize}
\end{itemize}
\end{frame}

\begin{frame}
\frametitle{Modelling Frameworks}
\begin{itemize}
\item \href{https://www.minizinc.org/}{MiniZinc} (NICTA, Monash University, Australia)
\item \href{https://github.com/eomahony/Numberjack}{NumberJack} (Insight, Ireland) 
\item \href{https://www-users.york.ac.uk/peter.nightingale/savilerow//}{EssencePrime/SavilleRow} (UK) 
\item \href{https://cpmpy.readthedocs.io/en/latest/}{CPMpy} (KU Leuven)
\item Allow use of multiple back-end solvers
\item Compile model into variants for each solver
\item A priori solver independent model(CP, MIP, SAT)
\end{itemize}
\end{frame} 

\begin{frame}
\frametitle{Framework Process}
\begin{center}
\begin{tikzpicture}[xscale=2]
\node[shape=rectangle,fill=pantone127-4] (problem) at (2.5,5) {Problem};
\node (human) at (2.5,4) {Human};
\node[shape=rectangle,fill=pantone127-4] (model) at (2.5,3) {Model};
\node (compile) at (2.5,2) {Compile/Reformulate};
\node (solver1) at (1,1) {CP};
\node (solver2) at (2,1) {MIP};
\node (solver3) at (3,1) {SAT};
\node (solver4) at (4,1) {Other};
\node[shape=rectangle,fill=pantone157-8] (solution1) at (1,0) {Solution};
\node[shape=rectangle,fill=pantone157-8] (solution2) at (2,0) {Solution};
\node[shape=rectangle,fill=pantone157-8] (solution3) at (3,0) {Solution};
\node[shape=rectangle,fill=pantone157-8] (solution4) at (4,0) {Solution};
\draw[-] (problem) -- (human);
\draw[->] (human) -- (model);
\draw[-] (model) -- (compile);
\draw[-] (compile) -- (solver1);
\draw[-] (compile) -- (solver2);
\draw[-] (compile) -- (solver3);
\draw[-] (compile) -- (solver4);
\draw[->] (solver1) -- (solution1);
\draw[->] (solver2) -- (solution2);
\draw[->] (solver3) -- (solution3);
\draw[->] (solver4) -- (solution4);
\end{tikzpicture}
\end{center}
\end{frame}

% \begin{frame}
  % \frametitle{Do It Now!}
  % \begin{itemize}
  % \item Download and install Minizinc
  % \item \url{https://www.minizinc.org/}
  % \end{itemize}
% \end{frame}


\begin{frame}
\frametitle{Why use Puzzles as Examples?}
\begin{itemize}
\item Easy to understand the problem
\item Solvable by hand without specialized knowledge
\item Possible to compare automated to manual solving process
\end{itemize}

\begin{quote}
The puzzle, though inanimate, is presented as a solvable problem without lasting negative consequences, a very low-risk low-reward situation. By being a puzzle, the object is attempting to convince the user that it must be completed.
\end{quote}
{\tiny Source: \href{https://www.everydayrhetoric.com/2020/03/the-rhetoric-of-puzzles/}{Every Day Rhetoric}}
\end{frame}


\mode<all>{
\part{Basic Constraint Propagation}
}
\begin{frame}
\partpage
\end{frame}
\mode<all>{
\input{../sendmore/body}
}

\mode<all>{
\part{Global Constraints}
}
\begin{frame}
\partpage
\end{frame}
\mode<all>{
\input{../sudoku/body}
}

\clearpage
\mode<all>{
\part{Customizing Search}
}
\begin{frame}
\partpage
\end{frame}
\mode<all>{
\input{../nqueen/body}
}


%% \clearpage
%% \mode<all>{
%%   \part{Problem Decomposition}
%% }
%% \begin{frame}
%%   \partpage
%% \end{frame}
%% \mode<all>{
%%   \input{../party/body}
%%   }

%% \clearpage
%% \mode<all>{
%%   \part{More Global Constraints}
%% }
%% \begin{frame}
%%   \partpage
%% \end{frame}
%% \mode<all>{
%%   \input{../car/body}
%%   }

%% \clearpage
%% \mode<all>{
%%   \part{Methodology}
%% }
%% \begin{frame}
%%   \partpage
%% \end{frame}
%% \mode<all>{
%%   \input{../methodology/body}
%%   }

\clearpage
\mode<all>{
\part{What is missing?}
}
\begin{frame}
\partpage
\end{frame}

\begin{frame}
\frametitle{Many Specialized Topics}
\begin{itemize}
\item How to design efficient core engine
\item Hybrids with LP/MIP tools
\item Hybrids with SAT
\item Symmetry breaking
\item Use of MDD/BDD to encode sets of solutions
\item High level modelling tools
\item Debugging/visualization
\end{itemize}
\end{frame}

\begin{frame}
\frametitle{Reformulation}
\begin{itemize}
\item Just because the user has modelled it this way, it doesn't mean we have to solve it that way
\begin{itemize}
\item Replace some constraint(s) by other, equivalent constraints
\item Because we don't have that constraint in our system
\item For performance
\end{itemize}
\end{itemize}
\end{frame}

\begin{frame}
\frametitle{Learning}
\begin{itemize}
\item While solving the problem we can learn how to strengthen the model/search 
\begin{itemize}
\item Understand which constraints/method contribute to propagation and change schedule
\item Learn no-good constraints by explaining failure
\item Adapt search strategy based on search experience
\end{itemize}
\end{itemize}
\end{frame}

%% \begin{frame}
%% \frametitle{Refined Process}
%% \begin{center}
%% \begin{tikzpicture}
%% \node[shape=rectangle,fill=pantone127-4] (problem) at (3,7) {Problem};
%% \node (human) at (3,6) {Human};
%% \node[shape=rectangle,fill=pantone127-4] (model) at (3,5) {User Model};
%% \node (reformulation) at (3,4) {Reformulation};
%% \node[shape=rectangle,fill=pantone127-4] (implementation) at (3,3) {Implementation Model};
%% \node (solver) at (3,2) {Constraint Solver/Search};
%% \node[shape=rectangle,fill=pantone157-8] (solution) at (3,1) {Solution};
%% \draw[-] (problem) -- (human);
%% \draw[->] (human) -- (model);
%% \draw[-] (model) -- (reformulation);
%% \draw[->] (reformulation) -- (implementation);
%% \draw[-] (implementation) -- (solver);
%% \draw[->] (solver) -- (0,2) -- (0,3) -- (implementation);
%% \draw[->] (solver) -- (solution);
%% \end{tikzpicture}
%% \end{center}
%% \end{frame}

%\begin{frame}
%\frametitle{What is CP actually used for?}
%\url{http://hsimonis.wordpress.com}
%\includegraphics[width=10cm]{../introduction/images/blog}
%\end{frame}

%% \begin{frame}
%% \frametitle{Exercises}
%% \begin{itemize}
%% \item Install Constraint Solver of your Choice
%% \item MiniZinc (MiningZinc) will be used in later sessions
%% \item Run Sendmoremoney, Sudoku, Queens examples
%% \item Can you choose which propagator to use for alldifferent?
%% \item Can you use different search strategies?
%% \end{itemize}
%% \end{frame}

\begin{frame}
\frametitle{More Learning Resources}
\begin{itemize}
\item Survey of Methods, Resources, and Formats for Teaching Constraint Programming
\begin{itemize}
\item by Tejas Santanam, Helmut Simonis
\item \url{https://doi.org/10.48550/arXiv.2403.12717}
\item Based on survey of community for WTCP 2023
\item \url{https://hsimonis.github.io/WTCP2023/}
\end{itemize}
\end{itemize}
\end{frame}


}


%% \clearpage
%% \mode<all>{
%%   \part{Problem Decomposition}
%% }
%% \begin{frame}
%%   \partpage
%% \end{frame}
%% \mode<all>{
%%   \begin{frame}
\frametitle{Objectives}
\begin{itemize}
\item Overview of Core Constraint Programming
\item Three Main Concepts
\begin{itemize}
\item Constraint Propagation
\item Global Constraints
\item Customizing Search
\end{itemize}
%\item Get Some Experience with MiniZinc
\item Topics will be treated in more detail in later parts of the school
\item Based on Examples, not Formal Description
\end{itemize}
\end{frame}

\begin{frame}
\frametitle{Outline}
\begin{itemize}
\item Why Constraint Programming?
\item Constraint Propagation
\item Global Constraints
\item Customizing Search
%\item Choices in Modelling
%\item Decomposition
%\item More Global Constraints
%\item Methodology
%\item What is missing?
\end{itemize}
\end{frame}

% \begin{frame}
  % \frametitle{Using MiniZinc IDE}
  % \begin{itemize}
  % \item Developed in the Australian NICTA project
  % \item Maintained by Monash University
  % \item Modelling tool with multiple back-end solvers
  % \item Available from \url{https://www.minizinc.org/}
  % \end{itemize}
% \end{frame}


% \begin{frame}
% \frametitle{Examples in ECLiPSe}
% \begin{itemize}
% \item Open sourced constraint programming language
% \item Development goes back to 1985
% \item ECRC, ICL, IC-Parc, PTL, Cisco
% \item \url{https://eclipseclp.org/}
% \item Specialities
% \begin{itemize}
% \item Develop new solvers for specific domains
% \item Integration with MIP
% \end{itemize}
% \item Not included in bundled MiniZinc IDE
% \item Specialized visualization tools used here
  % \begin{itemize}
  % \item CP-Viz, Simonis et al. 2010
    % \end{itemize}
% \end{itemize}
% \end{frame}

\begin{frame}
\frametitle{Tutorial Based on ECLiPSe ELearning Course}
\begin{itemize}
\item Self-study course in constraint programming
\item Supported by Cisco Systems and Silicon Valley Community Foundation
\item Multi-media format, video lectures, slides, handout etc
\item {\small \url{https://eclipseclp.org/ELearning/index.html}}
\end{itemize}
\end{frame}

\begin{frame}
\frametitle{Also Part of CRT-AI Constraint Week}
\begin{itemize}
\item Annual one week course on CP and Optimization in Ireland
\item Part of national training program for PhD students in AI
\item \url{https://www.crt-ai.ie/}
\end{itemize}
\end{frame}



\begin{frame}
\frametitle{Constraint Programming - in a nutshell}
\begin{itemize}
\item Declarative description of problems with
\begin{itemize}
\item {\em Variables} which range over (finite) sets of values
\item {\em Constraints} over subsets of variables which restrict possible value combinations
\item A {\em solution} is a value assignment which satisfies all constraints
\end{itemize}

\item Constraint propagation/reasoning
\begin{itemize}
\item Removing inconsistent values for variables
\item Detect failure if constraint can not be satisfied
\item Interaction of constraints via shared variables
\item Incomplete
\end{itemize}

\item Search
\begin{itemize}
\item User controlled assignment of values to variables
\item Each step triggers constraint propagation 
\end{itemize}
\item Different domains require/allow different methods
\end{itemize}
\end{frame}

\begin{frame}
  \frametitle{Constraint Programming is Different}
  \begin{itemize}
  \item Declarative Programming
    \begin{itemize}
    \item Concentrate on what you want
      \item Not how to get there
      \item Program != Algorithm
      \item Program = Model
    \end{itemize}
    \item Applied to Combinatorial Problems
      \begin{itemize}
        \item No complete polynomial algorithms known (exist?)
        \item CP less ad-hoc than heuristics
        \item Models can evolve
  \end{itemize}
  \end{itemize}
  \end{frame}
    
\begin{frame}
  \frametitle{A Subtractive Process}
  \begin{textblock}{4}(8,-3)
    \includegraphics[width=4cm]{../introduction/images/stages}
  \end{textblock}
  \vfill
  \begin{quote}
    ``Oh, bosh, as Mr. Ruskin says. Sculpture, per se, is the simplest thing in the world. All you have to do is to take a big chunk of marble and a hammer and chisel, make up your mind what you are about to create and chip off all the marble you don't want.''-Paris Gaulois.
  \end{quote}
  
  {\tiny Source: \url{https://quoteinvestigator.com/2014/06/22/chip-away/}}
\end{frame}

\begin{frame}
\frametitle{Basic Process}
\begin{center}
\begin{tikzpicture}
\node[shape=rectangle,fill=pantone127-4] (problem) at (2,5) {Problem};
\node (human) at (2,4) {Human};
\node[shape=rectangle,fill=pantone127-4] (model) at (2,3) {Model};
\node (solver) at (2,2) {Constraint Solver/Search};
\node[shape=rectangle,fill=pantone157-8] (solution) at (2,1) {Solution};
\draw[-] (problem) -- (human);
\draw[->] (human) -- (model);
\draw[-] (model) -- (solver);
\draw[->] (solver) -- (solution);
\end{tikzpicture}
\end{center}
\end{frame}

\begin{frame}
\frametitle{More Realistic}
\begin{center}
\begin{tikzpicture}[xscale=4.0]
\node[shape=rectangle,fill=pantone127-4] (problem) at (2,5) {Problem};
\node (human) at (2,4) {Human};
\node[shape=rectangle,fill=pantone127-4] (model) at (2,3) {Model};
\node (solver) at (2,2) {Constraint Solver/Search};
\node[shape=rectangle,fill=pantone24-8] (hangs) at (1,1) {Hangs};
\node[shape=rectangle,fill=pantone157-8] (solution) at (2,1) {Solution};
\node[shape=rectangle,fill=pantone24-8] (wrong) at (3,1) {Wrong Solution};
\draw[-] (problem) -- (human);
\draw[->] (human) -- (model);
\draw[-] (model) -- (solver);
\draw[->] (solver) -- (hangs);
\draw[->] (solver) -- (solution);
\draw[->] (solver) -- (wrong);
\draw[->] (hangs) -- (1,4) -- (human);
\draw[->] (wrong) -- (3,4) -- (human);
\end{tikzpicture}
\end{center}
\end{frame}

\begin{frame}
\frametitle{Dual Role of Model}
\begin{itemize}
\item Allows Human to Express Problem
\begin{itemize}
\item Close to Problem Domain
\item Constraints as Abstractions
\end{itemize}
\item Allows Solver to Execute
\begin{itemize}
\item Variables as Communication Mechanism
\item Constraints as Algorithms
\end{itemize}
\end{itemize}
\end{frame}

\begin{frame}
\frametitle{Modelling Frameworks}
\begin{itemize}
\item \href{https://www.minizinc.org/}{MiniZinc} (NICTA, Monash University, Australia)
\item \href{https://github.com/eomahony/Numberjack}{NumberJack} (Insight, Ireland) 
\item \href{https://www-users.york.ac.uk/peter.nightingale/savilerow//}{EssencePrime/SavilleRow} (UK) 
\item \href{https://cpmpy.readthedocs.io/en/latest/}{CPMpy} (KU Leuven)
\item Allow use of multiple back-end solvers
\item Compile model into variants for each solver
\item A priori solver independent model(CP, MIP, SAT)
\end{itemize}
\end{frame} 

\begin{frame}
\frametitle{Framework Process}
\begin{center}
\begin{tikzpicture}[xscale=2]
\node[shape=rectangle,fill=pantone127-4] (problem) at (2.5,5) {Problem};
\node (human) at (2.5,4) {Human};
\node[shape=rectangle,fill=pantone127-4] (model) at (2.5,3) {Model};
\node (compile) at (2.5,2) {Compile/Reformulate};
\node (solver1) at (1,1) {CP};
\node (solver2) at (2,1) {MIP};
\node (solver3) at (3,1) {SAT};
\node (solver4) at (4,1) {Other};
\node[shape=rectangle,fill=pantone157-8] (solution1) at (1,0) {Solution};
\node[shape=rectangle,fill=pantone157-8] (solution2) at (2,0) {Solution};
\node[shape=rectangle,fill=pantone157-8] (solution3) at (3,0) {Solution};
\node[shape=rectangle,fill=pantone157-8] (solution4) at (4,0) {Solution};
\draw[-] (problem) -- (human);
\draw[->] (human) -- (model);
\draw[-] (model) -- (compile);
\draw[-] (compile) -- (solver1);
\draw[-] (compile) -- (solver2);
\draw[-] (compile) -- (solver3);
\draw[-] (compile) -- (solver4);
\draw[->] (solver1) -- (solution1);
\draw[->] (solver2) -- (solution2);
\draw[->] (solver3) -- (solution3);
\draw[->] (solver4) -- (solution4);
\end{tikzpicture}
\end{center}
\end{frame}

% \begin{frame}
  % \frametitle{Do It Now!}
  % \begin{itemize}
  % \item Download and install Minizinc
  % \item \url{https://www.minizinc.org/}
  % \end{itemize}
% \end{frame}


\begin{frame}
\frametitle{Why use Puzzles as Examples?}
\begin{itemize}
\item Easy to understand the problem
\item Solvable by hand without specialized knowledge
\item Possible to compare automated to manual solving process
\end{itemize}

\begin{quote}
The puzzle, though inanimate, is presented as a solvable problem without lasting negative consequences, a very low-risk low-reward situation. By being a puzzle, the object is attempting to convince the user that it must be completed.
\end{quote}
{\tiny Source: \href{https://www.everydayrhetoric.com/2020/03/the-rhetoric-of-puzzles/}{Every Day Rhetoric}}
\end{frame}


\mode<all>{
\part{Basic Constraint Propagation}
}
\begin{frame}
\partpage
\end{frame}
\mode<all>{
\input{../sendmore/body}
}

\mode<all>{
\part{Global Constraints}
}
\begin{frame}
\partpage
\end{frame}
\mode<all>{
\input{../sudoku/body}
}

\clearpage
\mode<all>{
\part{Customizing Search}
}
\begin{frame}
\partpage
\end{frame}
\mode<all>{
\input{../nqueen/body}
}


%% \clearpage
%% \mode<all>{
%%   \part{Problem Decomposition}
%% }
%% \begin{frame}
%%   \partpage
%% \end{frame}
%% \mode<all>{
%%   \input{../party/body}
%%   }

%% \clearpage
%% \mode<all>{
%%   \part{More Global Constraints}
%% }
%% \begin{frame}
%%   \partpage
%% \end{frame}
%% \mode<all>{
%%   \input{../car/body}
%%   }

%% \clearpage
%% \mode<all>{
%%   \part{Methodology}
%% }
%% \begin{frame}
%%   \partpage
%% \end{frame}
%% \mode<all>{
%%   \input{../methodology/body}
%%   }

\clearpage
\mode<all>{
\part{What is missing?}
}
\begin{frame}
\partpage
\end{frame}

\begin{frame}
\frametitle{Many Specialized Topics}
\begin{itemize}
\item How to design efficient core engine
\item Hybrids with LP/MIP tools
\item Hybrids with SAT
\item Symmetry breaking
\item Use of MDD/BDD to encode sets of solutions
\item High level modelling tools
\item Debugging/visualization
\end{itemize}
\end{frame}

\begin{frame}
\frametitle{Reformulation}
\begin{itemize}
\item Just because the user has modelled it this way, it doesn't mean we have to solve it that way
\begin{itemize}
\item Replace some constraint(s) by other, equivalent constraints
\item Because we don't have that constraint in our system
\item For performance
\end{itemize}
\end{itemize}
\end{frame}

\begin{frame}
\frametitle{Learning}
\begin{itemize}
\item While solving the problem we can learn how to strengthen the model/search 
\begin{itemize}
\item Understand which constraints/method contribute to propagation and change schedule
\item Learn no-good constraints by explaining failure
\item Adapt search strategy based on search experience
\end{itemize}
\end{itemize}
\end{frame}

%% \begin{frame}
%% \frametitle{Refined Process}
%% \begin{center}
%% \begin{tikzpicture}
%% \node[shape=rectangle,fill=pantone127-4] (problem) at (3,7) {Problem};
%% \node (human) at (3,6) {Human};
%% \node[shape=rectangle,fill=pantone127-4] (model) at (3,5) {User Model};
%% \node (reformulation) at (3,4) {Reformulation};
%% \node[shape=rectangle,fill=pantone127-4] (implementation) at (3,3) {Implementation Model};
%% \node (solver) at (3,2) {Constraint Solver/Search};
%% \node[shape=rectangle,fill=pantone157-8] (solution) at (3,1) {Solution};
%% \draw[-] (problem) -- (human);
%% \draw[->] (human) -- (model);
%% \draw[-] (model) -- (reformulation);
%% \draw[->] (reformulation) -- (implementation);
%% \draw[-] (implementation) -- (solver);
%% \draw[->] (solver) -- (0,2) -- (0,3) -- (implementation);
%% \draw[->] (solver) -- (solution);
%% \end{tikzpicture}
%% \end{center}
%% \end{frame}

%\begin{frame}
%\frametitle{What is CP actually used for?}
%\url{http://hsimonis.wordpress.com}
%\includegraphics[width=10cm]{../introduction/images/blog}
%\end{frame}

%% \begin{frame}
%% \frametitle{Exercises}
%% \begin{itemize}
%% \item Install Constraint Solver of your Choice
%% \item MiniZinc (MiningZinc) will be used in later sessions
%% \item Run Sendmoremoney, Sudoku, Queens examples
%% \item Can you choose which propagator to use for alldifferent?
%% \item Can you use different search strategies?
%% \end{itemize}
%% \end{frame}

\begin{frame}
\frametitle{More Learning Resources}
\begin{itemize}
\item Survey of Methods, Resources, and Formats for Teaching Constraint Programming
\begin{itemize}
\item by Tejas Santanam, Helmut Simonis
\item \url{https://doi.org/10.48550/arXiv.2403.12717}
\item Based on survey of community for WTCP 2023
\item \url{https://hsimonis.github.io/WTCP2023/}
\end{itemize}
\end{itemize}
\end{frame}


%%   }

%% \clearpage
%% \mode<all>{
%%   \part{More Global Constraints}
%% }
%% \begin{frame}
%%   \partpage
%% \end{frame}
%% \mode<all>{
%%   \begin{frame}
\frametitle{Objectives}
\begin{itemize}
\item Overview of Core Constraint Programming
\item Three Main Concepts
\begin{itemize}
\item Constraint Propagation
\item Global Constraints
\item Customizing Search
\end{itemize}
%\item Get Some Experience with MiniZinc
\item Topics will be treated in more detail in later parts of the school
\item Based on Examples, not Formal Description
\end{itemize}
\end{frame}

\begin{frame}
\frametitle{Outline}
\begin{itemize}
\item Why Constraint Programming?
\item Constraint Propagation
\item Global Constraints
\item Customizing Search
%\item Choices in Modelling
%\item Decomposition
%\item More Global Constraints
%\item Methodology
%\item What is missing?
\end{itemize}
\end{frame}

% \begin{frame}
  % \frametitle{Using MiniZinc IDE}
  % \begin{itemize}
  % \item Developed in the Australian NICTA project
  % \item Maintained by Monash University
  % \item Modelling tool with multiple back-end solvers
  % \item Available from \url{https://www.minizinc.org/}
  % \end{itemize}
% \end{frame}


% \begin{frame}
% \frametitle{Examples in ECLiPSe}
% \begin{itemize}
% \item Open sourced constraint programming language
% \item Development goes back to 1985
% \item ECRC, ICL, IC-Parc, PTL, Cisco
% \item \url{https://eclipseclp.org/}
% \item Specialities
% \begin{itemize}
% \item Develop new solvers for specific domains
% \item Integration with MIP
% \end{itemize}
% \item Not included in bundled MiniZinc IDE
% \item Specialized visualization tools used here
  % \begin{itemize}
  % \item CP-Viz, Simonis et al. 2010
    % \end{itemize}
% \end{itemize}
% \end{frame}

\begin{frame}
\frametitle{Tutorial Based on ECLiPSe ELearning Course}
\begin{itemize}
\item Self-study course in constraint programming
\item Supported by Cisco Systems and Silicon Valley Community Foundation
\item Multi-media format, video lectures, slides, handout etc
\item {\small \url{https://eclipseclp.org/ELearning/index.html}}
\end{itemize}
\end{frame}

\begin{frame}
\frametitle{Also Part of CRT-AI Constraint Week}
\begin{itemize}
\item Annual one week course on CP and Optimization in Ireland
\item Part of national training program for PhD students in AI
\item \url{https://www.crt-ai.ie/}
\end{itemize}
\end{frame}



\begin{frame}
\frametitle{Constraint Programming - in a nutshell}
\begin{itemize}
\item Declarative description of problems with
\begin{itemize}
\item {\em Variables} which range over (finite) sets of values
\item {\em Constraints} over subsets of variables which restrict possible value combinations
\item A {\em solution} is a value assignment which satisfies all constraints
\end{itemize}

\item Constraint propagation/reasoning
\begin{itemize}
\item Removing inconsistent values for variables
\item Detect failure if constraint can not be satisfied
\item Interaction of constraints via shared variables
\item Incomplete
\end{itemize}

\item Search
\begin{itemize}
\item User controlled assignment of values to variables
\item Each step triggers constraint propagation 
\end{itemize}
\item Different domains require/allow different methods
\end{itemize}
\end{frame}

\begin{frame}
  \frametitle{Constraint Programming is Different}
  \begin{itemize}
  \item Declarative Programming
    \begin{itemize}
    \item Concentrate on what you want
      \item Not how to get there
      \item Program != Algorithm
      \item Program = Model
    \end{itemize}
    \item Applied to Combinatorial Problems
      \begin{itemize}
        \item No complete polynomial algorithms known (exist?)
        \item CP less ad-hoc than heuristics
        \item Models can evolve
  \end{itemize}
  \end{itemize}
  \end{frame}
    
\begin{frame}
  \frametitle{A Subtractive Process}
  \begin{textblock}{4}(8,-3)
    \includegraphics[width=4cm]{../introduction/images/stages}
  \end{textblock}
  \vfill
  \begin{quote}
    ``Oh, bosh, as Mr. Ruskin says. Sculpture, per se, is the simplest thing in the world. All you have to do is to take a big chunk of marble and a hammer and chisel, make up your mind what you are about to create and chip off all the marble you don't want.''-Paris Gaulois.
  \end{quote}
  
  {\tiny Source: \url{https://quoteinvestigator.com/2014/06/22/chip-away/}}
\end{frame}

\begin{frame}
\frametitle{Basic Process}
\begin{center}
\begin{tikzpicture}
\node[shape=rectangle,fill=pantone127-4] (problem) at (2,5) {Problem};
\node (human) at (2,4) {Human};
\node[shape=rectangle,fill=pantone127-4] (model) at (2,3) {Model};
\node (solver) at (2,2) {Constraint Solver/Search};
\node[shape=rectangle,fill=pantone157-8] (solution) at (2,1) {Solution};
\draw[-] (problem) -- (human);
\draw[->] (human) -- (model);
\draw[-] (model) -- (solver);
\draw[->] (solver) -- (solution);
\end{tikzpicture}
\end{center}
\end{frame}

\begin{frame}
\frametitle{More Realistic}
\begin{center}
\begin{tikzpicture}[xscale=4.0]
\node[shape=rectangle,fill=pantone127-4] (problem) at (2,5) {Problem};
\node (human) at (2,4) {Human};
\node[shape=rectangle,fill=pantone127-4] (model) at (2,3) {Model};
\node (solver) at (2,2) {Constraint Solver/Search};
\node[shape=rectangle,fill=pantone24-8] (hangs) at (1,1) {Hangs};
\node[shape=rectangle,fill=pantone157-8] (solution) at (2,1) {Solution};
\node[shape=rectangle,fill=pantone24-8] (wrong) at (3,1) {Wrong Solution};
\draw[-] (problem) -- (human);
\draw[->] (human) -- (model);
\draw[-] (model) -- (solver);
\draw[->] (solver) -- (hangs);
\draw[->] (solver) -- (solution);
\draw[->] (solver) -- (wrong);
\draw[->] (hangs) -- (1,4) -- (human);
\draw[->] (wrong) -- (3,4) -- (human);
\end{tikzpicture}
\end{center}
\end{frame}

\begin{frame}
\frametitle{Dual Role of Model}
\begin{itemize}
\item Allows Human to Express Problem
\begin{itemize}
\item Close to Problem Domain
\item Constraints as Abstractions
\end{itemize}
\item Allows Solver to Execute
\begin{itemize}
\item Variables as Communication Mechanism
\item Constraints as Algorithms
\end{itemize}
\end{itemize}
\end{frame}

\begin{frame}
\frametitle{Modelling Frameworks}
\begin{itemize}
\item \href{https://www.minizinc.org/}{MiniZinc} (NICTA, Monash University, Australia)
\item \href{https://github.com/eomahony/Numberjack}{NumberJack} (Insight, Ireland) 
\item \href{https://www-users.york.ac.uk/peter.nightingale/savilerow//}{EssencePrime/SavilleRow} (UK) 
\item \href{https://cpmpy.readthedocs.io/en/latest/}{CPMpy} (KU Leuven)
\item Allow use of multiple back-end solvers
\item Compile model into variants for each solver
\item A priori solver independent model(CP, MIP, SAT)
\end{itemize}
\end{frame} 

\begin{frame}
\frametitle{Framework Process}
\begin{center}
\begin{tikzpicture}[xscale=2]
\node[shape=rectangle,fill=pantone127-4] (problem) at (2.5,5) {Problem};
\node (human) at (2.5,4) {Human};
\node[shape=rectangle,fill=pantone127-4] (model) at (2.5,3) {Model};
\node (compile) at (2.5,2) {Compile/Reformulate};
\node (solver1) at (1,1) {CP};
\node (solver2) at (2,1) {MIP};
\node (solver3) at (3,1) {SAT};
\node (solver4) at (4,1) {Other};
\node[shape=rectangle,fill=pantone157-8] (solution1) at (1,0) {Solution};
\node[shape=rectangle,fill=pantone157-8] (solution2) at (2,0) {Solution};
\node[shape=rectangle,fill=pantone157-8] (solution3) at (3,0) {Solution};
\node[shape=rectangle,fill=pantone157-8] (solution4) at (4,0) {Solution};
\draw[-] (problem) -- (human);
\draw[->] (human) -- (model);
\draw[-] (model) -- (compile);
\draw[-] (compile) -- (solver1);
\draw[-] (compile) -- (solver2);
\draw[-] (compile) -- (solver3);
\draw[-] (compile) -- (solver4);
\draw[->] (solver1) -- (solution1);
\draw[->] (solver2) -- (solution2);
\draw[->] (solver3) -- (solution3);
\draw[->] (solver4) -- (solution4);
\end{tikzpicture}
\end{center}
\end{frame}

% \begin{frame}
  % \frametitle{Do It Now!}
  % \begin{itemize}
  % \item Download and install Minizinc
  % \item \url{https://www.minizinc.org/}
  % \end{itemize}
% \end{frame}


\begin{frame}
\frametitle{Why use Puzzles as Examples?}
\begin{itemize}
\item Easy to understand the problem
\item Solvable by hand without specialized knowledge
\item Possible to compare automated to manual solving process
\end{itemize}

\begin{quote}
The puzzle, though inanimate, is presented as a solvable problem without lasting negative consequences, a very low-risk low-reward situation. By being a puzzle, the object is attempting to convince the user that it must be completed.
\end{quote}
{\tiny Source: \href{https://www.everydayrhetoric.com/2020/03/the-rhetoric-of-puzzles/}{Every Day Rhetoric}}
\end{frame}


\mode<all>{
\part{Basic Constraint Propagation}
}
\begin{frame}
\partpage
\end{frame}
\mode<all>{
\input{../sendmore/body}
}

\mode<all>{
\part{Global Constraints}
}
\begin{frame}
\partpage
\end{frame}
\mode<all>{
\input{../sudoku/body}
}

\clearpage
\mode<all>{
\part{Customizing Search}
}
\begin{frame}
\partpage
\end{frame}
\mode<all>{
\input{../nqueen/body}
}


%% \clearpage
%% \mode<all>{
%%   \part{Problem Decomposition}
%% }
%% \begin{frame}
%%   \partpage
%% \end{frame}
%% \mode<all>{
%%   \input{../party/body}
%%   }

%% \clearpage
%% \mode<all>{
%%   \part{More Global Constraints}
%% }
%% \begin{frame}
%%   \partpage
%% \end{frame}
%% \mode<all>{
%%   \input{../car/body}
%%   }

%% \clearpage
%% \mode<all>{
%%   \part{Methodology}
%% }
%% \begin{frame}
%%   \partpage
%% \end{frame}
%% \mode<all>{
%%   \input{../methodology/body}
%%   }

\clearpage
\mode<all>{
\part{What is missing?}
}
\begin{frame}
\partpage
\end{frame}

\begin{frame}
\frametitle{Many Specialized Topics}
\begin{itemize}
\item How to design efficient core engine
\item Hybrids with LP/MIP tools
\item Hybrids with SAT
\item Symmetry breaking
\item Use of MDD/BDD to encode sets of solutions
\item High level modelling tools
\item Debugging/visualization
\end{itemize}
\end{frame}

\begin{frame}
\frametitle{Reformulation}
\begin{itemize}
\item Just because the user has modelled it this way, it doesn't mean we have to solve it that way
\begin{itemize}
\item Replace some constraint(s) by other, equivalent constraints
\item Because we don't have that constraint in our system
\item For performance
\end{itemize}
\end{itemize}
\end{frame}

\begin{frame}
\frametitle{Learning}
\begin{itemize}
\item While solving the problem we can learn how to strengthen the model/search 
\begin{itemize}
\item Understand which constraints/method contribute to propagation and change schedule
\item Learn no-good constraints by explaining failure
\item Adapt search strategy based on search experience
\end{itemize}
\end{itemize}
\end{frame}

%% \begin{frame}
%% \frametitle{Refined Process}
%% \begin{center}
%% \begin{tikzpicture}
%% \node[shape=rectangle,fill=pantone127-4] (problem) at (3,7) {Problem};
%% \node (human) at (3,6) {Human};
%% \node[shape=rectangle,fill=pantone127-4] (model) at (3,5) {User Model};
%% \node (reformulation) at (3,4) {Reformulation};
%% \node[shape=rectangle,fill=pantone127-4] (implementation) at (3,3) {Implementation Model};
%% \node (solver) at (3,2) {Constraint Solver/Search};
%% \node[shape=rectangle,fill=pantone157-8] (solution) at (3,1) {Solution};
%% \draw[-] (problem) -- (human);
%% \draw[->] (human) -- (model);
%% \draw[-] (model) -- (reformulation);
%% \draw[->] (reformulation) -- (implementation);
%% \draw[-] (implementation) -- (solver);
%% \draw[->] (solver) -- (0,2) -- (0,3) -- (implementation);
%% \draw[->] (solver) -- (solution);
%% \end{tikzpicture}
%% \end{center}
%% \end{frame}

%\begin{frame}
%\frametitle{What is CP actually used for?}
%\url{http://hsimonis.wordpress.com}
%\includegraphics[width=10cm]{../introduction/images/blog}
%\end{frame}

%% \begin{frame}
%% \frametitle{Exercises}
%% \begin{itemize}
%% \item Install Constraint Solver of your Choice
%% \item MiniZinc (MiningZinc) will be used in later sessions
%% \item Run Sendmoremoney, Sudoku, Queens examples
%% \item Can you choose which propagator to use for alldifferent?
%% \item Can you use different search strategies?
%% \end{itemize}
%% \end{frame}

\begin{frame}
\frametitle{More Learning Resources}
\begin{itemize}
\item Survey of Methods, Resources, and Formats for Teaching Constraint Programming
\begin{itemize}
\item by Tejas Santanam, Helmut Simonis
\item \url{https://doi.org/10.48550/arXiv.2403.12717}
\item Based on survey of community for WTCP 2023
\item \url{https://hsimonis.github.io/WTCP2023/}
\end{itemize}
\end{itemize}
\end{frame}


%%   }

%% \clearpage
%% \mode<all>{
%%   \part{Methodology}
%% }
%% \begin{frame}
%%   \partpage
%% \end{frame}
%% \mode<all>{
%%   \begin{frame}
\frametitle{Objectives}
\begin{itemize}
\item Overview of Core Constraint Programming
\item Three Main Concepts
\begin{itemize}
\item Constraint Propagation
\item Global Constraints
\item Customizing Search
\end{itemize}
%\item Get Some Experience with MiniZinc
\item Topics will be treated in more detail in later parts of the school
\item Based on Examples, not Formal Description
\end{itemize}
\end{frame}

\begin{frame}
\frametitle{Outline}
\begin{itemize}
\item Why Constraint Programming?
\item Constraint Propagation
\item Global Constraints
\item Customizing Search
%\item Choices in Modelling
%\item Decomposition
%\item More Global Constraints
%\item Methodology
%\item What is missing?
\end{itemize}
\end{frame}

% \begin{frame}
  % \frametitle{Using MiniZinc IDE}
  % \begin{itemize}
  % \item Developed in the Australian NICTA project
  % \item Maintained by Monash University
  % \item Modelling tool with multiple back-end solvers
  % \item Available from \url{https://www.minizinc.org/}
  % \end{itemize}
% \end{frame}


% \begin{frame}
% \frametitle{Examples in ECLiPSe}
% \begin{itemize}
% \item Open sourced constraint programming language
% \item Development goes back to 1985
% \item ECRC, ICL, IC-Parc, PTL, Cisco
% \item \url{https://eclipseclp.org/}
% \item Specialities
% \begin{itemize}
% \item Develop new solvers for specific domains
% \item Integration with MIP
% \end{itemize}
% \item Not included in bundled MiniZinc IDE
% \item Specialized visualization tools used here
  % \begin{itemize}
  % \item CP-Viz, Simonis et al. 2010
    % \end{itemize}
% \end{itemize}
% \end{frame}

\begin{frame}
\frametitle{Tutorial Based on ECLiPSe ELearning Course}
\begin{itemize}
\item Self-study course in constraint programming
\item Supported by Cisco Systems and Silicon Valley Community Foundation
\item Multi-media format, video lectures, slides, handout etc
\item {\small \url{https://eclipseclp.org/ELearning/index.html}}
\end{itemize}
\end{frame}

\begin{frame}
\frametitle{Also Part of CRT-AI Constraint Week}
\begin{itemize}
\item Annual one week course on CP and Optimization in Ireland
\item Part of national training program for PhD students in AI
\item \url{https://www.crt-ai.ie/}
\end{itemize}
\end{frame}



\begin{frame}
\frametitle{Constraint Programming - in a nutshell}
\begin{itemize}
\item Declarative description of problems with
\begin{itemize}
\item {\em Variables} which range over (finite) sets of values
\item {\em Constraints} over subsets of variables which restrict possible value combinations
\item A {\em solution} is a value assignment which satisfies all constraints
\end{itemize}

\item Constraint propagation/reasoning
\begin{itemize}
\item Removing inconsistent values for variables
\item Detect failure if constraint can not be satisfied
\item Interaction of constraints via shared variables
\item Incomplete
\end{itemize}

\item Search
\begin{itemize}
\item User controlled assignment of values to variables
\item Each step triggers constraint propagation 
\end{itemize}
\item Different domains require/allow different methods
\end{itemize}
\end{frame}

\begin{frame}
  \frametitle{Constraint Programming is Different}
  \begin{itemize}
  \item Declarative Programming
    \begin{itemize}
    \item Concentrate on what you want
      \item Not how to get there
      \item Program != Algorithm
      \item Program = Model
    \end{itemize}
    \item Applied to Combinatorial Problems
      \begin{itemize}
        \item No complete polynomial algorithms known (exist?)
        \item CP less ad-hoc than heuristics
        \item Models can evolve
  \end{itemize}
  \end{itemize}
  \end{frame}
    
\begin{frame}
  \frametitle{A Subtractive Process}
  \begin{textblock}{4}(8,-3)
    \includegraphics[width=4cm]{../introduction/images/stages}
  \end{textblock}
  \vfill
  \begin{quote}
    ``Oh, bosh, as Mr. Ruskin says. Sculpture, per se, is the simplest thing in the world. All you have to do is to take a big chunk of marble and a hammer and chisel, make up your mind what you are about to create and chip off all the marble you don't want.''-Paris Gaulois.
  \end{quote}
  
  {\tiny Source: \url{https://quoteinvestigator.com/2014/06/22/chip-away/}}
\end{frame}

\begin{frame}
\frametitle{Basic Process}
\begin{center}
\begin{tikzpicture}
\node[shape=rectangle,fill=pantone127-4] (problem) at (2,5) {Problem};
\node (human) at (2,4) {Human};
\node[shape=rectangle,fill=pantone127-4] (model) at (2,3) {Model};
\node (solver) at (2,2) {Constraint Solver/Search};
\node[shape=rectangle,fill=pantone157-8] (solution) at (2,1) {Solution};
\draw[-] (problem) -- (human);
\draw[->] (human) -- (model);
\draw[-] (model) -- (solver);
\draw[->] (solver) -- (solution);
\end{tikzpicture}
\end{center}
\end{frame}

\begin{frame}
\frametitle{More Realistic}
\begin{center}
\begin{tikzpicture}[xscale=4.0]
\node[shape=rectangle,fill=pantone127-4] (problem) at (2,5) {Problem};
\node (human) at (2,4) {Human};
\node[shape=rectangle,fill=pantone127-4] (model) at (2,3) {Model};
\node (solver) at (2,2) {Constraint Solver/Search};
\node[shape=rectangle,fill=pantone24-8] (hangs) at (1,1) {Hangs};
\node[shape=rectangle,fill=pantone157-8] (solution) at (2,1) {Solution};
\node[shape=rectangle,fill=pantone24-8] (wrong) at (3,1) {Wrong Solution};
\draw[-] (problem) -- (human);
\draw[->] (human) -- (model);
\draw[-] (model) -- (solver);
\draw[->] (solver) -- (hangs);
\draw[->] (solver) -- (solution);
\draw[->] (solver) -- (wrong);
\draw[->] (hangs) -- (1,4) -- (human);
\draw[->] (wrong) -- (3,4) -- (human);
\end{tikzpicture}
\end{center}
\end{frame}

\begin{frame}
\frametitle{Dual Role of Model}
\begin{itemize}
\item Allows Human to Express Problem
\begin{itemize}
\item Close to Problem Domain
\item Constraints as Abstractions
\end{itemize}
\item Allows Solver to Execute
\begin{itemize}
\item Variables as Communication Mechanism
\item Constraints as Algorithms
\end{itemize}
\end{itemize}
\end{frame}

\begin{frame}
\frametitle{Modelling Frameworks}
\begin{itemize}
\item \href{https://www.minizinc.org/}{MiniZinc} (NICTA, Monash University, Australia)
\item \href{https://github.com/eomahony/Numberjack}{NumberJack} (Insight, Ireland) 
\item \href{https://www-users.york.ac.uk/peter.nightingale/savilerow//}{EssencePrime/SavilleRow} (UK) 
\item \href{https://cpmpy.readthedocs.io/en/latest/}{CPMpy} (KU Leuven)
\item Allow use of multiple back-end solvers
\item Compile model into variants for each solver
\item A priori solver independent model(CP, MIP, SAT)
\end{itemize}
\end{frame} 

\begin{frame}
\frametitle{Framework Process}
\begin{center}
\begin{tikzpicture}[xscale=2]
\node[shape=rectangle,fill=pantone127-4] (problem) at (2.5,5) {Problem};
\node (human) at (2.5,4) {Human};
\node[shape=rectangle,fill=pantone127-4] (model) at (2.5,3) {Model};
\node (compile) at (2.5,2) {Compile/Reformulate};
\node (solver1) at (1,1) {CP};
\node (solver2) at (2,1) {MIP};
\node (solver3) at (3,1) {SAT};
\node (solver4) at (4,1) {Other};
\node[shape=rectangle,fill=pantone157-8] (solution1) at (1,0) {Solution};
\node[shape=rectangle,fill=pantone157-8] (solution2) at (2,0) {Solution};
\node[shape=rectangle,fill=pantone157-8] (solution3) at (3,0) {Solution};
\node[shape=rectangle,fill=pantone157-8] (solution4) at (4,0) {Solution};
\draw[-] (problem) -- (human);
\draw[->] (human) -- (model);
\draw[-] (model) -- (compile);
\draw[-] (compile) -- (solver1);
\draw[-] (compile) -- (solver2);
\draw[-] (compile) -- (solver3);
\draw[-] (compile) -- (solver4);
\draw[->] (solver1) -- (solution1);
\draw[->] (solver2) -- (solution2);
\draw[->] (solver3) -- (solution3);
\draw[->] (solver4) -- (solution4);
\end{tikzpicture}
\end{center}
\end{frame}

% \begin{frame}
  % \frametitle{Do It Now!}
  % \begin{itemize}
  % \item Download and install Minizinc
  % \item \url{https://www.minizinc.org/}
  % \end{itemize}
% \end{frame}


\begin{frame}
\frametitle{Why use Puzzles as Examples?}
\begin{itemize}
\item Easy to understand the problem
\item Solvable by hand without specialized knowledge
\item Possible to compare automated to manual solving process
\end{itemize}

\begin{quote}
The puzzle, though inanimate, is presented as a solvable problem without lasting negative consequences, a very low-risk low-reward situation. By being a puzzle, the object is attempting to convince the user that it must be completed.
\end{quote}
{\tiny Source: \href{https://www.everydayrhetoric.com/2020/03/the-rhetoric-of-puzzles/}{Every Day Rhetoric}}
\end{frame}


\mode<all>{
\part{Basic Constraint Propagation}
}
\begin{frame}
\partpage
\end{frame}
\mode<all>{
\input{../sendmore/body}
}

\mode<all>{
\part{Global Constraints}
}
\begin{frame}
\partpage
\end{frame}
\mode<all>{
\input{../sudoku/body}
}

\clearpage
\mode<all>{
\part{Customizing Search}
}
\begin{frame}
\partpage
\end{frame}
\mode<all>{
\input{../nqueen/body}
}


%% \clearpage
%% \mode<all>{
%%   \part{Problem Decomposition}
%% }
%% \begin{frame}
%%   \partpage
%% \end{frame}
%% \mode<all>{
%%   \input{../party/body}
%%   }

%% \clearpage
%% \mode<all>{
%%   \part{More Global Constraints}
%% }
%% \begin{frame}
%%   \partpage
%% \end{frame}
%% \mode<all>{
%%   \input{../car/body}
%%   }

%% \clearpage
%% \mode<all>{
%%   \part{Methodology}
%% }
%% \begin{frame}
%%   \partpage
%% \end{frame}
%% \mode<all>{
%%   \input{../methodology/body}
%%   }

\clearpage
\mode<all>{
\part{What is missing?}
}
\begin{frame}
\partpage
\end{frame}

\begin{frame}
\frametitle{Many Specialized Topics}
\begin{itemize}
\item How to design efficient core engine
\item Hybrids with LP/MIP tools
\item Hybrids with SAT
\item Symmetry breaking
\item Use of MDD/BDD to encode sets of solutions
\item High level modelling tools
\item Debugging/visualization
\end{itemize}
\end{frame}

\begin{frame}
\frametitle{Reformulation}
\begin{itemize}
\item Just because the user has modelled it this way, it doesn't mean we have to solve it that way
\begin{itemize}
\item Replace some constraint(s) by other, equivalent constraints
\item Because we don't have that constraint in our system
\item For performance
\end{itemize}
\end{itemize}
\end{frame}

\begin{frame}
\frametitle{Learning}
\begin{itemize}
\item While solving the problem we can learn how to strengthen the model/search 
\begin{itemize}
\item Understand which constraints/method contribute to propagation and change schedule
\item Learn no-good constraints by explaining failure
\item Adapt search strategy based on search experience
\end{itemize}
\end{itemize}
\end{frame}

%% \begin{frame}
%% \frametitle{Refined Process}
%% \begin{center}
%% \begin{tikzpicture}
%% \node[shape=rectangle,fill=pantone127-4] (problem) at (3,7) {Problem};
%% \node (human) at (3,6) {Human};
%% \node[shape=rectangle,fill=pantone127-4] (model) at (3,5) {User Model};
%% \node (reformulation) at (3,4) {Reformulation};
%% \node[shape=rectangle,fill=pantone127-4] (implementation) at (3,3) {Implementation Model};
%% \node (solver) at (3,2) {Constraint Solver/Search};
%% \node[shape=rectangle,fill=pantone157-8] (solution) at (3,1) {Solution};
%% \draw[-] (problem) -- (human);
%% \draw[->] (human) -- (model);
%% \draw[-] (model) -- (reformulation);
%% \draw[->] (reformulation) -- (implementation);
%% \draw[-] (implementation) -- (solver);
%% \draw[->] (solver) -- (0,2) -- (0,3) -- (implementation);
%% \draw[->] (solver) -- (solution);
%% \end{tikzpicture}
%% \end{center}
%% \end{frame}

%\begin{frame}
%\frametitle{What is CP actually used for?}
%\url{http://hsimonis.wordpress.com}
%\includegraphics[width=10cm]{../introduction/images/blog}
%\end{frame}

%% \begin{frame}
%% \frametitle{Exercises}
%% \begin{itemize}
%% \item Install Constraint Solver of your Choice
%% \item MiniZinc (MiningZinc) will be used in later sessions
%% \item Run Sendmoremoney, Sudoku, Queens examples
%% \item Can you choose which propagator to use for alldifferent?
%% \item Can you use different search strategies?
%% \end{itemize}
%% \end{frame}

\begin{frame}
\frametitle{More Learning Resources}
\begin{itemize}
\item Survey of Methods, Resources, and Formats for Teaching Constraint Programming
\begin{itemize}
\item by Tejas Santanam, Helmut Simonis
\item \url{https://doi.org/10.48550/arXiv.2403.12717}
\item Based on survey of community for WTCP 2023
\item \url{https://hsimonis.github.io/WTCP2023/}
\end{itemize}
\end{itemize}
\end{frame}


%%   }

\clearpage
\mode<all>{
\part{What is missing?}
}
\begin{frame}
\partpage
\end{frame}

\begin{frame}
\frametitle{Many Specialized Topics}
\begin{itemize}
\item How to design efficient core engine
\item Hybrids with LP/MIP tools
\item Hybrids with SAT
\item Symmetry breaking
\item Use of MDD/BDD to encode sets of solutions
\item High level modelling tools
\item Debugging/visualization
\end{itemize}
\end{frame}

\begin{frame}
\frametitle{Reformulation}
\begin{itemize}
\item Just because the user has modelled it this way, it doesn't mean we have to solve it that way
\begin{itemize}
\item Replace some constraint(s) by other, equivalent constraints
\item Because we don't have that constraint in our system
\item For performance
\end{itemize}
\end{itemize}
\end{frame}

\begin{frame}
\frametitle{Learning}
\begin{itemize}
\item While solving the problem we can learn how to strengthen the model/search 
\begin{itemize}
\item Understand which constraints/method contribute to propagation and change schedule
\item Learn no-good constraints by explaining failure
\item Adapt search strategy based on search experience
\end{itemize}
\end{itemize}
\end{frame}

%% \begin{frame}
%% \frametitle{Refined Process}
%% \begin{center}
%% \begin{tikzpicture}
%% \node[shape=rectangle,fill=pantone127-4] (problem) at (3,7) {Problem};
%% \node (human) at (3,6) {Human};
%% \node[shape=rectangle,fill=pantone127-4] (model) at (3,5) {User Model};
%% \node (reformulation) at (3,4) {Reformulation};
%% \node[shape=rectangle,fill=pantone127-4] (implementation) at (3,3) {Implementation Model};
%% \node (solver) at (3,2) {Constraint Solver/Search};
%% \node[shape=rectangle,fill=pantone157-8] (solution) at (3,1) {Solution};
%% \draw[-] (problem) -- (human);
%% \draw[->] (human) -- (model);
%% \draw[-] (model) -- (reformulation);
%% \draw[->] (reformulation) -- (implementation);
%% \draw[-] (implementation) -- (solver);
%% \draw[->] (solver) -- (0,2) -- (0,3) -- (implementation);
%% \draw[->] (solver) -- (solution);
%% \end{tikzpicture}
%% \end{center}
%% \end{frame}

%\begin{frame}
%\frametitle{What is CP actually used for?}
%\url{http://hsimonis.wordpress.com}
%\includegraphics[width=10cm]{../introduction/images/blog}
%\end{frame}

%% \begin{frame}
%% \frametitle{Exercises}
%% \begin{itemize}
%% \item Install Constraint Solver of your Choice
%% \item MiniZinc (MiningZinc) will be used in later sessions
%% \item Run Sendmoremoney, Sudoku, Queens examples
%% \item Can you choose which propagator to use for alldifferent?
%% \item Can you use different search strategies?
%% \end{itemize}
%% \end{frame}

\begin{frame}
\frametitle{More Learning Resources}
\begin{itemize}
\item Survey of Methods, Resources, and Formats for Teaching Constraint Programming
\begin{itemize}
\item by Tejas Santanam, Helmut Simonis
\item \url{https://doi.org/10.48550/arXiv.2403.12717}
\item Based on survey of community for WTCP 2023
\item \url{https://hsimonis.github.io/WTCP2023/}
\end{itemize}
\end{itemize}
\end{frame}


%%   }

\clearpage
\mode<all>{
\part{What is missing?}
}
\begin{frame}
\partpage
\end{frame}

\begin{frame}
\frametitle{Many Specialized Topics}
\begin{itemize}
\item How to design efficient core engine
\item Hybrids with LP/MIP tools
\item Hybrids with SAT
\item Symmetry breaking
\item Use of MDD/BDD to encode sets of solutions
\item High level modelling tools
\item Debugging/visualization
\end{itemize}
\end{frame}

\begin{frame}
\frametitle{Reformulation}
\begin{itemize}
\item Just because the user has modelled it this way, it doesn't mean we have to solve it that way
\begin{itemize}
\item Replace some constraint(s) by other, equivalent constraints
\item Because we don't have that constraint in our system
\item For performance
\end{itemize}
\end{itemize}
\end{frame}

\begin{frame}
\frametitle{Learning}
\begin{itemize}
\item While solving the problem we can learn how to strengthen the model/search 
\begin{itemize}
\item Understand which constraints/method contribute to propagation and change schedule
\item Learn no-good constraints by explaining failure
\item Adapt search strategy based on search experience
\end{itemize}
\end{itemize}
\end{frame}

%% \begin{frame}
%% \frametitle{Refined Process}
%% \begin{center}
%% \begin{tikzpicture}
%% \node[shape=rectangle,fill=pantone127-4] (problem) at (3,7) {Problem};
%% \node (human) at (3,6) {Human};
%% \node[shape=rectangle,fill=pantone127-4] (model) at (3,5) {User Model};
%% \node (reformulation) at (3,4) {Reformulation};
%% \node[shape=rectangle,fill=pantone127-4] (implementation) at (3,3) {Implementation Model};
%% \node (solver) at (3,2) {Constraint Solver/Search};
%% \node[shape=rectangle,fill=pantone157-8] (solution) at (3,1) {Solution};
%% \draw[-] (problem) -- (human);
%% \draw[->] (human) -- (model);
%% \draw[-] (model) -- (reformulation);
%% \draw[->] (reformulation) -- (implementation);
%% \draw[-] (implementation) -- (solver);
%% \draw[->] (solver) -- (0,2) -- (0,3) -- (implementation);
%% \draw[->] (solver) -- (solution);
%% \end{tikzpicture}
%% \end{center}
%% \end{frame}

%\begin{frame}
%\frametitle{What is CP actually used for?}
%\url{http://hsimonis.wordpress.com}
%\includegraphics[width=10cm]{../introduction/images/blog}
%\end{frame}

%% \begin{frame}
%% \frametitle{Exercises}
%% \begin{itemize}
%% \item Install Constraint Solver of your Choice
%% \item MiniZinc (MiningZinc) will be used in later sessions
%% \item Run Sendmoremoney, Sudoku, Queens examples
%% \item Can you choose which propagator to use for alldifferent?
%% \item Can you use different search strategies?
%% \end{itemize}
%% \end{frame}

\begin{frame}
\frametitle{More Learning Resources}
\begin{itemize}
\item Survey of Methods, Resources, and Formats for Teaching Constraint Programming
\begin{itemize}
\item by Tejas Santanam, Helmut Simonis
\item \url{https://doi.org/10.48550/arXiv.2403.12717}
\item Based on survey of community for WTCP 2023
\item \url{https://hsimonis.github.io/WTCP2023/}
\end{itemize}
\end{itemize}
\end{frame}


}

\clearpage
\mode<all>{
\part{Customizing Search}
}
\begin{frame}
\partpage
\end{frame}
\mode<all>{
\begin{frame}
\frametitle{Objectives}
\begin{itemize}
\item Overview of Core Constraint Programming
\item Three Main Concepts
\begin{itemize}
\item Constraint Propagation
\item Global Constraints
\item Customizing Search
\end{itemize}
%\item Get Some Experience with MiniZinc
\item Topics will be treated in more detail in later parts of the school
\item Based on Examples, not Formal Description
\end{itemize}
\end{frame}

\begin{frame}
\frametitle{Outline}
\begin{itemize}
\item Why Constraint Programming?
\item Constraint Propagation
\item Global Constraints
\item Customizing Search
%\item Choices in Modelling
%\item Decomposition
%\item More Global Constraints
%\item Methodology
%\item What is missing?
\end{itemize}
\end{frame}

% \begin{frame}
  % \frametitle{Using MiniZinc IDE}
  % \begin{itemize}
  % \item Developed in the Australian NICTA project
  % \item Maintained by Monash University
  % \item Modelling tool with multiple back-end solvers
  % \item Available from \url{https://www.minizinc.org/}
  % \end{itemize}
% \end{frame}


% \begin{frame}
% \frametitle{Examples in ECLiPSe}
% \begin{itemize}
% \item Open sourced constraint programming language
% \item Development goes back to 1985
% \item ECRC, ICL, IC-Parc, PTL, Cisco
% \item \url{https://eclipseclp.org/}
% \item Specialities
% \begin{itemize}
% \item Develop new solvers for specific domains
% \item Integration with MIP
% \end{itemize}
% \item Not included in bundled MiniZinc IDE
% \item Specialized visualization tools used here
  % \begin{itemize}
  % \item CP-Viz, Simonis et al. 2010
    % \end{itemize}
% \end{itemize}
% \end{frame}

\begin{frame}
\frametitle{Tutorial Based on ECLiPSe ELearning Course}
\begin{itemize}
\item Self-study course in constraint programming
\item Supported by Cisco Systems and Silicon Valley Community Foundation
\item Multi-media format, video lectures, slides, handout etc
\item {\small \url{https://eclipseclp.org/ELearning/index.html}}
\end{itemize}
\end{frame}

\begin{frame}
\frametitle{Also Part of CRT-AI Constraint Week}
\begin{itemize}
\item Annual one week course on CP and Optimization in Ireland
\item Part of national training program for PhD students in AI
\item \url{https://www.crt-ai.ie/}
\end{itemize}
\end{frame}



\begin{frame}
\frametitle{Constraint Programming - in a nutshell}
\begin{itemize}
\item Declarative description of problems with
\begin{itemize}
\item {\em Variables} which range over (finite) sets of values
\item {\em Constraints} over subsets of variables which restrict possible value combinations
\item A {\em solution} is a value assignment which satisfies all constraints
\end{itemize}

\item Constraint propagation/reasoning
\begin{itemize}
\item Removing inconsistent values for variables
\item Detect failure if constraint can not be satisfied
\item Interaction of constraints via shared variables
\item Incomplete
\end{itemize}

\item Search
\begin{itemize}
\item User controlled assignment of values to variables
\item Each step triggers constraint propagation 
\end{itemize}
\item Different domains require/allow different methods
\end{itemize}
\end{frame}

\begin{frame}
  \frametitle{Constraint Programming is Different}
  \begin{itemize}
  \item Declarative Programming
    \begin{itemize}
    \item Concentrate on what you want
      \item Not how to get there
      \item Program != Algorithm
      \item Program = Model
    \end{itemize}
    \item Applied to Combinatorial Problems
      \begin{itemize}
        \item No complete polynomial algorithms known (exist?)
        \item CP less ad-hoc than heuristics
        \item Models can evolve
  \end{itemize}
  \end{itemize}
  \end{frame}
    
\begin{frame}
  \frametitle{A Subtractive Process}
  \begin{textblock}{4}(8,-3)
    \includegraphics[width=4cm]{../introduction/images/stages}
  \end{textblock}
  \vfill
  \begin{quote}
    ``Oh, bosh, as Mr. Ruskin says. Sculpture, per se, is the simplest thing in the world. All you have to do is to take a big chunk of marble and a hammer and chisel, make up your mind what you are about to create and chip off all the marble you don't want.''-Paris Gaulois.
  \end{quote}
  
  {\tiny Source: \url{https://quoteinvestigator.com/2014/06/22/chip-away/}}
\end{frame}

\begin{frame}
\frametitle{Basic Process}
\begin{center}
\begin{tikzpicture}
\node[shape=rectangle,fill=pantone127-4] (problem) at (2,5) {Problem};
\node (human) at (2,4) {Human};
\node[shape=rectangle,fill=pantone127-4] (model) at (2,3) {Model};
\node (solver) at (2,2) {Constraint Solver/Search};
\node[shape=rectangle,fill=pantone157-8] (solution) at (2,1) {Solution};
\draw[-] (problem) -- (human);
\draw[->] (human) -- (model);
\draw[-] (model) -- (solver);
\draw[->] (solver) -- (solution);
\end{tikzpicture}
\end{center}
\end{frame}

\begin{frame}
\frametitle{More Realistic}
\begin{center}
\begin{tikzpicture}[xscale=4.0]
\node[shape=rectangle,fill=pantone127-4] (problem) at (2,5) {Problem};
\node (human) at (2,4) {Human};
\node[shape=rectangle,fill=pantone127-4] (model) at (2,3) {Model};
\node (solver) at (2,2) {Constraint Solver/Search};
\node[shape=rectangle,fill=pantone24-8] (hangs) at (1,1) {Hangs};
\node[shape=rectangle,fill=pantone157-8] (solution) at (2,1) {Solution};
\node[shape=rectangle,fill=pantone24-8] (wrong) at (3,1) {Wrong Solution};
\draw[-] (problem) -- (human);
\draw[->] (human) -- (model);
\draw[-] (model) -- (solver);
\draw[->] (solver) -- (hangs);
\draw[->] (solver) -- (solution);
\draw[->] (solver) -- (wrong);
\draw[->] (hangs) -- (1,4) -- (human);
\draw[->] (wrong) -- (3,4) -- (human);
\end{tikzpicture}
\end{center}
\end{frame}

\begin{frame}
\frametitle{Dual Role of Model}
\begin{itemize}
\item Allows Human to Express Problem
\begin{itemize}
\item Close to Problem Domain
\item Constraints as Abstractions
\end{itemize}
\item Allows Solver to Execute
\begin{itemize}
\item Variables as Communication Mechanism
\item Constraints as Algorithms
\end{itemize}
\end{itemize}
\end{frame}

\begin{frame}
\frametitle{Modelling Frameworks}
\begin{itemize}
\item \href{https://www.minizinc.org/}{MiniZinc} (NICTA, Monash University, Australia)
\item \href{https://github.com/eomahony/Numberjack}{NumberJack} (Insight, Ireland) 
\item \href{https://www-users.york.ac.uk/peter.nightingale/savilerow//}{EssencePrime/SavilleRow} (UK) 
\item \href{https://cpmpy.readthedocs.io/en/latest/}{CPMpy} (KU Leuven)
\item Allow use of multiple back-end solvers
\item Compile model into variants for each solver
\item A priori solver independent model(CP, MIP, SAT)
\end{itemize}
\end{frame} 

\begin{frame}
\frametitle{Framework Process}
\begin{center}
\begin{tikzpicture}[xscale=2]
\node[shape=rectangle,fill=pantone127-4] (problem) at (2.5,5) {Problem};
\node (human) at (2.5,4) {Human};
\node[shape=rectangle,fill=pantone127-4] (model) at (2.5,3) {Model};
\node (compile) at (2.5,2) {Compile/Reformulate};
\node (solver1) at (1,1) {CP};
\node (solver2) at (2,1) {MIP};
\node (solver3) at (3,1) {SAT};
\node (solver4) at (4,1) {Other};
\node[shape=rectangle,fill=pantone157-8] (solution1) at (1,0) {Solution};
\node[shape=rectangle,fill=pantone157-8] (solution2) at (2,0) {Solution};
\node[shape=rectangle,fill=pantone157-8] (solution3) at (3,0) {Solution};
\node[shape=rectangle,fill=pantone157-8] (solution4) at (4,0) {Solution};
\draw[-] (problem) -- (human);
\draw[->] (human) -- (model);
\draw[-] (model) -- (compile);
\draw[-] (compile) -- (solver1);
\draw[-] (compile) -- (solver2);
\draw[-] (compile) -- (solver3);
\draw[-] (compile) -- (solver4);
\draw[->] (solver1) -- (solution1);
\draw[->] (solver2) -- (solution2);
\draw[->] (solver3) -- (solution3);
\draw[->] (solver4) -- (solution4);
\end{tikzpicture}
\end{center}
\end{frame}

% \begin{frame}
  % \frametitle{Do It Now!}
  % \begin{itemize}
  % \item Download and install Minizinc
  % \item \url{https://www.minizinc.org/}
  % \end{itemize}
% \end{frame}


\begin{frame}
\frametitle{Why use Puzzles as Examples?}
\begin{itemize}
\item Easy to understand the problem
\item Solvable by hand without specialized knowledge
\item Possible to compare automated to manual solving process
\end{itemize}

\begin{quote}
The puzzle, though inanimate, is presented as a solvable problem without lasting negative consequences, a very low-risk low-reward situation. By being a puzzle, the object is attempting to convince the user that it must be completed.
\end{quote}
{\tiny Source: \href{https://www.everydayrhetoric.com/2020/03/the-rhetoric-of-puzzles/}{Every Day Rhetoric}}
\end{frame}


\mode<all>{
\part{Basic Constraint Propagation}
}
\begin{frame}
\partpage
\end{frame}
\mode<all>{
\begin{frame}
\frametitle{Objectives}
\begin{itemize}
\item Overview of Core Constraint Programming
\item Three Main Concepts
\begin{itemize}
\item Constraint Propagation
\item Global Constraints
\item Customizing Search
\end{itemize}
%\item Get Some Experience with MiniZinc
\item Topics will be treated in more detail in later parts of the school
\item Based on Examples, not Formal Description
\end{itemize}
\end{frame}

\begin{frame}
\frametitle{Outline}
\begin{itemize}
\item Why Constraint Programming?
\item Constraint Propagation
\item Global Constraints
\item Customizing Search
%\item Choices in Modelling
%\item Decomposition
%\item More Global Constraints
%\item Methodology
%\item What is missing?
\end{itemize}
\end{frame}

% \begin{frame}
  % \frametitle{Using MiniZinc IDE}
  % \begin{itemize}
  % \item Developed in the Australian NICTA project
  % \item Maintained by Monash University
  % \item Modelling tool with multiple back-end solvers
  % \item Available from \url{https://www.minizinc.org/}
  % \end{itemize}
% \end{frame}


% \begin{frame}
% \frametitle{Examples in ECLiPSe}
% \begin{itemize}
% \item Open sourced constraint programming language
% \item Development goes back to 1985
% \item ECRC, ICL, IC-Parc, PTL, Cisco
% \item \url{https://eclipseclp.org/}
% \item Specialities
% \begin{itemize}
% \item Develop new solvers for specific domains
% \item Integration with MIP
% \end{itemize}
% \item Not included in bundled MiniZinc IDE
% \item Specialized visualization tools used here
  % \begin{itemize}
  % \item CP-Viz, Simonis et al. 2010
    % \end{itemize}
% \end{itemize}
% \end{frame}

\begin{frame}
\frametitle{Tutorial Based on ECLiPSe ELearning Course}
\begin{itemize}
\item Self-study course in constraint programming
\item Supported by Cisco Systems and Silicon Valley Community Foundation
\item Multi-media format, video lectures, slides, handout etc
\item {\small \url{https://eclipseclp.org/ELearning/index.html}}
\end{itemize}
\end{frame}

\begin{frame}
\frametitle{Also Part of CRT-AI Constraint Week}
\begin{itemize}
\item Annual one week course on CP and Optimization in Ireland
\item Part of national training program for PhD students in AI
\item \url{https://www.crt-ai.ie/}
\end{itemize}
\end{frame}



\begin{frame}
\frametitle{Constraint Programming - in a nutshell}
\begin{itemize}
\item Declarative description of problems with
\begin{itemize}
\item {\em Variables} which range over (finite) sets of values
\item {\em Constraints} over subsets of variables which restrict possible value combinations
\item A {\em solution} is a value assignment which satisfies all constraints
\end{itemize}

\item Constraint propagation/reasoning
\begin{itemize}
\item Removing inconsistent values for variables
\item Detect failure if constraint can not be satisfied
\item Interaction of constraints via shared variables
\item Incomplete
\end{itemize}

\item Search
\begin{itemize}
\item User controlled assignment of values to variables
\item Each step triggers constraint propagation 
\end{itemize}
\item Different domains require/allow different methods
\end{itemize}
\end{frame}

\begin{frame}
  \frametitle{Constraint Programming is Different}
  \begin{itemize}
  \item Declarative Programming
    \begin{itemize}
    \item Concentrate on what you want
      \item Not how to get there
      \item Program != Algorithm
      \item Program = Model
    \end{itemize}
    \item Applied to Combinatorial Problems
      \begin{itemize}
        \item No complete polynomial algorithms known (exist?)
        \item CP less ad-hoc than heuristics
        \item Models can evolve
  \end{itemize}
  \end{itemize}
  \end{frame}
    
\begin{frame}
  \frametitle{A Subtractive Process}
  \begin{textblock}{4}(8,-3)
    \includegraphics[width=4cm]{../introduction/images/stages}
  \end{textblock}
  \vfill
  \begin{quote}
    ``Oh, bosh, as Mr. Ruskin says. Sculpture, per se, is the simplest thing in the world. All you have to do is to take a big chunk of marble and a hammer and chisel, make up your mind what you are about to create and chip off all the marble you don't want.''-Paris Gaulois.
  \end{quote}
  
  {\tiny Source: \url{https://quoteinvestigator.com/2014/06/22/chip-away/}}
\end{frame}

\begin{frame}
\frametitle{Basic Process}
\begin{center}
\begin{tikzpicture}
\node[shape=rectangle,fill=pantone127-4] (problem) at (2,5) {Problem};
\node (human) at (2,4) {Human};
\node[shape=rectangle,fill=pantone127-4] (model) at (2,3) {Model};
\node (solver) at (2,2) {Constraint Solver/Search};
\node[shape=rectangle,fill=pantone157-8] (solution) at (2,1) {Solution};
\draw[-] (problem) -- (human);
\draw[->] (human) -- (model);
\draw[-] (model) -- (solver);
\draw[->] (solver) -- (solution);
\end{tikzpicture}
\end{center}
\end{frame}

\begin{frame}
\frametitle{More Realistic}
\begin{center}
\begin{tikzpicture}[xscale=4.0]
\node[shape=rectangle,fill=pantone127-4] (problem) at (2,5) {Problem};
\node (human) at (2,4) {Human};
\node[shape=rectangle,fill=pantone127-4] (model) at (2,3) {Model};
\node (solver) at (2,2) {Constraint Solver/Search};
\node[shape=rectangle,fill=pantone24-8] (hangs) at (1,1) {Hangs};
\node[shape=rectangle,fill=pantone157-8] (solution) at (2,1) {Solution};
\node[shape=rectangle,fill=pantone24-8] (wrong) at (3,1) {Wrong Solution};
\draw[-] (problem) -- (human);
\draw[->] (human) -- (model);
\draw[-] (model) -- (solver);
\draw[->] (solver) -- (hangs);
\draw[->] (solver) -- (solution);
\draw[->] (solver) -- (wrong);
\draw[->] (hangs) -- (1,4) -- (human);
\draw[->] (wrong) -- (3,4) -- (human);
\end{tikzpicture}
\end{center}
\end{frame}

\begin{frame}
\frametitle{Dual Role of Model}
\begin{itemize}
\item Allows Human to Express Problem
\begin{itemize}
\item Close to Problem Domain
\item Constraints as Abstractions
\end{itemize}
\item Allows Solver to Execute
\begin{itemize}
\item Variables as Communication Mechanism
\item Constraints as Algorithms
\end{itemize}
\end{itemize}
\end{frame}

\begin{frame}
\frametitle{Modelling Frameworks}
\begin{itemize}
\item \href{https://www.minizinc.org/}{MiniZinc} (NICTA, Monash University, Australia)
\item \href{https://github.com/eomahony/Numberjack}{NumberJack} (Insight, Ireland) 
\item \href{https://www-users.york.ac.uk/peter.nightingale/savilerow//}{EssencePrime/SavilleRow} (UK) 
\item \href{https://cpmpy.readthedocs.io/en/latest/}{CPMpy} (KU Leuven)
\item Allow use of multiple back-end solvers
\item Compile model into variants for each solver
\item A priori solver independent model(CP, MIP, SAT)
\end{itemize}
\end{frame} 

\begin{frame}
\frametitle{Framework Process}
\begin{center}
\begin{tikzpicture}[xscale=2]
\node[shape=rectangle,fill=pantone127-4] (problem) at (2.5,5) {Problem};
\node (human) at (2.5,4) {Human};
\node[shape=rectangle,fill=pantone127-4] (model) at (2.5,3) {Model};
\node (compile) at (2.5,2) {Compile/Reformulate};
\node (solver1) at (1,1) {CP};
\node (solver2) at (2,1) {MIP};
\node (solver3) at (3,1) {SAT};
\node (solver4) at (4,1) {Other};
\node[shape=rectangle,fill=pantone157-8] (solution1) at (1,0) {Solution};
\node[shape=rectangle,fill=pantone157-8] (solution2) at (2,0) {Solution};
\node[shape=rectangle,fill=pantone157-8] (solution3) at (3,0) {Solution};
\node[shape=rectangle,fill=pantone157-8] (solution4) at (4,0) {Solution};
\draw[-] (problem) -- (human);
\draw[->] (human) -- (model);
\draw[-] (model) -- (compile);
\draw[-] (compile) -- (solver1);
\draw[-] (compile) -- (solver2);
\draw[-] (compile) -- (solver3);
\draw[-] (compile) -- (solver4);
\draw[->] (solver1) -- (solution1);
\draw[->] (solver2) -- (solution2);
\draw[->] (solver3) -- (solution3);
\draw[->] (solver4) -- (solution4);
\end{tikzpicture}
\end{center}
\end{frame}

% \begin{frame}
  % \frametitle{Do It Now!}
  % \begin{itemize}
  % \item Download and install Minizinc
  % \item \url{https://www.minizinc.org/}
  % \end{itemize}
% \end{frame}


\begin{frame}
\frametitle{Why use Puzzles as Examples?}
\begin{itemize}
\item Easy to understand the problem
\item Solvable by hand without specialized knowledge
\item Possible to compare automated to manual solving process
\end{itemize}

\begin{quote}
The puzzle, though inanimate, is presented as a solvable problem without lasting negative consequences, a very low-risk low-reward situation. By being a puzzle, the object is attempting to convince the user that it must be completed.
\end{quote}
{\tiny Source: \href{https://www.everydayrhetoric.com/2020/03/the-rhetoric-of-puzzles/}{Every Day Rhetoric}}
\end{frame}


\mode<all>{
\part{Basic Constraint Propagation}
}
\begin{frame}
\partpage
\end{frame}
\mode<all>{
\begin{frame}
\frametitle{Objectives}
\begin{itemize}
\item Overview of Core Constraint Programming
\item Three Main Concepts
\begin{itemize}
\item Constraint Propagation
\item Global Constraints
\item Customizing Search
\end{itemize}
%\item Get Some Experience with MiniZinc
\item Topics will be treated in more detail in later parts of the school
\item Based on Examples, not Formal Description
\end{itemize}
\end{frame}

\begin{frame}
\frametitle{Outline}
\begin{itemize}
\item Why Constraint Programming?
\item Constraint Propagation
\item Global Constraints
\item Customizing Search
%\item Choices in Modelling
%\item Decomposition
%\item More Global Constraints
%\item Methodology
%\item What is missing?
\end{itemize}
\end{frame}

% \begin{frame}
  % \frametitle{Using MiniZinc IDE}
  % \begin{itemize}
  % \item Developed in the Australian NICTA project
  % \item Maintained by Monash University
  % \item Modelling tool with multiple back-end solvers
  % \item Available from \url{https://www.minizinc.org/}
  % \end{itemize}
% \end{frame}


% \begin{frame}
% \frametitle{Examples in ECLiPSe}
% \begin{itemize}
% \item Open sourced constraint programming language
% \item Development goes back to 1985
% \item ECRC, ICL, IC-Parc, PTL, Cisco
% \item \url{https://eclipseclp.org/}
% \item Specialities
% \begin{itemize}
% \item Develop new solvers for specific domains
% \item Integration with MIP
% \end{itemize}
% \item Not included in bundled MiniZinc IDE
% \item Specialized visualization tools used here
  % \begin{itemize}
  % \item CP-Viz, Simonis et al. 2010
    % \end{itemize}
% \end{itemize}
% \end{frame}

\begin{frame}
\frametitle{Tutorial Based on ECLiPSe ELearning Course}
\begin{itemize}
\item Self-study course in constraint programming
\item Supported by Cisco Systems and Silicon Valley Community Foundation
\item Multi-media format, video lectures, slides, handout etc
\item {\small \url{https://eclipseclp.org/ELearning/index.html}}
\end{itemize}
\end{frame}

\begin{frame}
\frametitle{Also Part of CRT-AI Constraint Week}
\begin{itemize}
\item Annual one week course on CP and Optimization in Ireland
\item Part of national training program for PhD students in AI
\item \url{https://www.crt-ai.ie/}
\end{itemize}
\end{frame}



\begin{frame}
\frametitle{Constraint Programming - in a nutshell}
\begin{itemize}
\item Declarative description of problems with
\begin{itemize}
\item {\em Variables} which range over (finite) sets of values
\item {\em Constraints} over subsets of variables which restrict possible value combinations
\item A {\em solution} is a value assignment which satisfies all constraints
\end{itemize}

\item Constraint propagation/reasoning
\begin{itemize}
\item Removing inconsistent values for variables
\item Detect failure if constraint can not be satisfied
\item Interaction of constraints via shared variables
\item Incomplete
\end{itemize}

\item Search
\begin{itemize}
\item User controlled assignment of values to variables
\item Each step triggers constraint propagation 
\end{itemize}
\item Different domains require/allow different methods
\end{itemize}
\end{frame}

\begin{frame}
  \frametitle{Constraint Programming is Different}
  \begin{itemize}
  \item Declarative Programming
    \begin{itemize}
    \item Concentrate on what you want
      \item Not how to get there
      \item Program != Algorithm
      \item Program = Model
    \end{itemize}
    \item Applied to Combinatorial Problems
      \begin{itemize}
        \item No complete polynomial algorithms known (exist?)
        \item CP less ad-hoc than heuristics
        \item Models can evolve
  \end{itemize}
  \end{itemize}
  \end{frame}
    
\begin{frame}
  \frametitle{A Subtractive Process}
  \begin{textblock}{4}(8,-3)
    \includegraphics[width=4cm]{../introduction/images/stages}
  \end{textblock}
  \vfill
  \begin{quote}
    ``Oh, bosh, as Mr. Ruskin says. Sculpture, per se, is the simplest thing in the world. All you have to do is to take a big chunk of marble and a hammer and chisel, make up your mind what you are about to create and chip off all the marble you don't want.''-Paris Gaulois.
  \end{quote}
  
  {\tiny Source: \url{https://quoteinvestigator.com/2014/06/22/chip-away/}}
\end{frame}

\begin{frame}
\frametitle{Basic Process}
\begin{center}
\begin{tikzpicture}
\node[shape=rectangle,fill=pantone127-4] (problem) at (2,5) {Problem};
\node (human) at (2,4) {Human};
\node[shape=rectangle,fill=pantone127-4] (model) at (2,3) {Model};
\node (solver) at (2,2) {Constraint Solver/Search};
\node[shape=rectangle,fill=pantone157-8] (solution) at (2,1) {Solution};
\draw[-] (problem) -- (human);
\draw[->] (human) -- (model);
\draw[-] (model) -- (solver);
\draw[->] (solver) -- (solution);
\end{tikzpicture}
\end{center}
\end{frame}

\begin{frame}
\frametitle{More Realistic}
\begin{center}
\begin{tikzpicture}[xscale=4.0]
\node[shape=rectangle,fill=pantone127-4] (problem) at (2,5) {Problem};
\node (human) at (2,4) {Human};
\node[shape=rectangle,fill=pantone127-4] (model) at (2,3) {Model};
\node (solver) at (2,2) {Constraint Solver/Search};
\node[shape=rectangle,fill=pantone24-8] (hangs) at (1,1) {Hangs};
\node[shape=rectangle,fill=pantone157-8] (solution) at (2,1) {Solution};
\node[shape=rectangle,fill=pantone24-8] (wrong) at (3,1) {Wrong Solution};
\draw[-] (problem) -- (human);
\draw[->] (human) -- (model);
\draw[-] (model) -- (solver);
\draw[->] (solver) -- (hangs);
\draw[->] (solver) -- (solution);
\draw[->] (solver) -- (wrong);
\draw[->] (hangs) -- (1,4) -- (human);
\draw[->] (wrong) -- (3,4) -- (human);
\end{tikzpicture}
\end{center}
\end{frame}

\begin{frame}
\frametitle{Dual Role of Model}
\begin{itemize}
\item Allows Human to Express Problem
\begin{itemize}
\item Close to Problem Domain
\item Constraints as Abstractions
\end{itemize}
\item Allows Solver to Execute
\begin{itemize}
\item Variables as Communication Mechanism
\item Constraints as Algorithms
\end{itemize}
\end{itemize}
\end{frame}

\begin{frame}
\frametitle{Modelling Frameworks}
\begin{itemize}
\item \href{https://www.minizinc.org/}{MiniZinc} (NICTA, Monash University, Australia)
\item \href{https://github.com/eomahony/Numberjack}{NumberJack} (Insight, Ireland) 
\item \href{https://www-users.york.ac.uk/peter.nightingale/savilerow//}{EssencePrime/SavilleRow} (UK) 
\item \href{https://cpmpy.readthedocs.io/en/latest/}{CPMpy} (KU Leuven)
\item Allow use of multiple back-end solvers
\item Compile model into variants for each solver
\item A priori solver independent model(CP, MIP, SAT)
\end{itemize}
\end{frame} 

\begin{frame}
\frametitle{Framework Process}
\begin{center}
\begin{tikzpicture}[xscale=2]
\node[shape=rectangle,fill=pantone127-4] (problem) at (2.5,5) {Problem};
\node (human) at (2.5,4) {Human};
\node[shape=rectangle,fill=pantone127-4] (model) at (2.5,3) {Model};
\node (compile) at (2.5,2) {Compile/Reformulate};
\node (solver1) at (1,1) {CP};
\node (solver2) at (2,1) {MIP};
\node (solver3) at (3,1) {SAT};
\node (solver4) at (4,1) {Other};
\node[shape=rectangle,fill=pantone157-8] (solution1) at (1,0) {Solution};
\node[shape=rectangle,fill=pantone157-8] (solution2) at (2,0) {Solution};
\node[shape=rectangle,fill=pantone157-8] (solution3) at (3,0) {Solution};
\node[shape=rectangle,fill=pantone157-8] (solution4) at (4,0) {Solution};
\draw[-] (problem) -- (human);
\draw[->] (human) -- (model);
\draw[-] (model) -- (compile);
\draw[-] (compile) -- (solver1);
\draw[-] (compile) -- (solver2);
\draw[-] (compile) -- (solver3);
\draw[-] (compile) -- (solver4);
\draw[->] (solver1) -- (solution1);
\draw[->] (solver2) -- (solution2);
\draw[->] (solver3) -- (solution3);
\draw[->] (solver4) -- (solution4);
\end{tikzpicture}
\end{center}
\end{frame}

% \begin{frame}
  % \frametitle{Do It Now!}
  % \begin{itemize}
  % \item Download and install Minizinc
  % \item \url{https://www.minizinc.org/}
  % \end{itemize}
% \end{frame}


\begin{frame}
\frametitle{Why use Puzzles as Examples?}
\begin{itemize}
\item Easy to understand the problem
\item Solvable by hand without specialized knowledge
\item Possible to compare automated to manual solving process
\end{itemize}

\begin{quote}
The puzzle, though inanimate, is presented as a solvable problem without lasting negative consequences, a very low-risk low-reward situation. By being a puzzle, the object is attempting to convince the user that it must be completed.
\end{quote}
{\tiny Source: \href{https://www.everydayrhetoric.com/2020/03/the-rhetoric-of-puzzles/}{Every Day Rhetoric}}
\end{frame}


\mode<all>{
\part{Basic Constraint Propagation}
}
\begin{frame}
\partpage
\end{frame}
\mode<all>{
\input{../sendmore/body}
}

\mode<all>{
\part{Global Constraints}
}
\begin{frame}
\partpage
\end{frame}
\mode<all>{
\input{../sudoku/body}
}

\clearpage
\mode<all>{
\part{Customizing Search}
}
\begin{frame}
\partpage
\end{frame}
\mode<all>{
\input{../nqueen/body}
}


%% \clearpage
%% \mode<all>{
%%   \part{Problem Decomposition}
%% }
%% \begin{frame}
%%   \partpage
%% \end{frame}
%% \mode<all>{
%%   \input{../party/body}
%%   }

%% \clearpage
%% \mode<all>{
%%   \part{More Global Constraints}
%% }
%% \begin{frame}
%%   \partpage
%% \end{frame}
%% \mode<all>{
%%   \input{../car/body}
%%   }

%% \clearpage
%% \mode<all>{
%%   \part{Methodology}
%% }
%% \begin{frame}
%%   \partpage
%% \end{frame}
%% \mode<all>{
%%   \input{../methodology/body}
%%   }

\clearpage
\mode<all>{
\part{What is missing?}
}
\begin{frame}
\partpage
\end{frame}

\begin{frame}
\frametitle{Many Specialized Topics}
\begin{itemize}
\item How to design efficient core engine
\item Hybrids with LP/MIP tools
\item Hybrids with SAT
\item Symmetry breaking
\item Use of MDD/BDD to encode sets of solutions
\item High level modelling tools
\item Debugging/visualization
\end{itemize}
\end{frame}

\begin{frame}
\frametitle{Reformulation}
\begin{itemize}
\item Just because the user has modelled it this way, it doesn't mean we have to solve it that way
\begin{itemize}
\item Replace some constraint(s) by other, equivalent constraints
\item Because we don't have that constraint in our system
\item For performance
\end{itemize}
\end{itemize}
\end{frame}

\begin{frame}
\frametitle{Learning}
\begin{itemize}
\item While solving the problem we can learn how to strengthen the model/search 
\begin{itemize}
\item Understand which constraints/method contribute to propagation and change schedule
\item Learn no-good constraints by explaining failure
\item Adapt search strategy based on search experience
\end{itemize}
\end{itemize}
\end{frame}

%% \begin{frame}
%% \frametitle{Refined Process}
%% \begin{center}
%% \begin{tikzpicture}
%% \node[shape=rectangle,fill=pantone127-4] (problem) at (3,7) {Problem};
%% \node (human) at (3,6) {Human};
%% \node[shape=rectangle,fill=pantone127-4] (model) at (3,5) {User Model};
%% \node (reformulation) at (3,4) {Reformulation};
%% \node[shape=rectangle,fill=pantone127-4] (implementation) at (3,3) {Implementation Model};
%% \node (solver) at (3,2) {Constraint Solver/Search};
%% \node[shape=rectangle,fill=pantone157-8] (solution) at (3,1) {Solution};
%% \draw[-] (problem) -- (human);
%% \draw[->] (human) -- (model);
%% \draw[-] (model) -- (reformulation);
%% \draw[->] (reformulation) -- (implementation);
%% \draw[-] (implementation) -- (solver);
%% \draw[->] (solver) -- (0,2) -- (0,3) -- (implementation);
%% \draw[->] (solver) -- (solution);
%% \end{tikzpicture}
%% \end{center}
%% \end{frame}

%\begin{frame}
%\frametitle{What is CP actually used for?}
%\url{http://hsimonis.wordpress.com}
%\includegraphics[width=10cm]{../introduction/images/blog}
%\end{frame}

%% \begin{frame}
%% \frametitle{Exercises}
%% \begin{itemize}
%% \item Install Constraint Solver of your Choice
%% \item MiniZinc (MiningZinc) will be used in later sessions
%% \item Run Sendmoremoney, Sudoku, Queens examples
%% \item Can you choose which propagator to use for alldifferent?
%% \item Can you use different search strategies?
%% \end{itemize}
%% \end{frame}

\begin{frame}
\frametitle{More Learning Resources}
\begin{itemize}
\item Survey of Methods, Resources, and Formats for Teaching Constraint Programming
\begin{itemize}
\item by Tejas Santanam, Helmut Simonis
\item \url{https://doi.org/10.48550/arXiv.2403.12717}
\item Based on survey of community for WTCP 2023
\item \url{https://hsimonis.github.io/WTCP2023/}
\end{itemize}
\end{itemize}
\end{frame}


}

\mode<all>{
\part{Global Constraints}
}
\begin{frame}
\partpage
\end{frame}
\mode<all>{
\begin{frame}
\frametitle{Objectives}
\begin{itemize}
\item Overview of Core Constraint Programming
\item Three Main Concepts
\begin{itemize}
\item Constraint Propagation
\item Global Constraints
\item Customizing Search
\end{itemize}
%\item Get Some Experience with MiniZinc
\item Topics will be treated in more detail in later parts of the school
\item Based on Examples, not Formal Description
\end{itemize}
\end{frame}

\begin{frame}
\frametitle{Outline}
\begin{itemize}
\item Why Constraint Programming?
\item Constraint Propagation
\item Global Constraints
\item Customizing Search
%\item Choices in Modelling
%\item Decomposition
%\item More Global Constraints
%\item Methodology
%\item What is missing?
\end{itemize}
\end{frame}

% \begin{frame}
  % \frametitle{Using MiniZinc IDE}
  % \begin{itemize}
  % \item Developed in the Australian NICTA project
  % \item Maintained by Monash University
  % \item Modelling tool with multiple back-end solvers
  % \item Available from \url{https://www.minizinc.org/}
  % \end{itemize}
% \end{frame}


% \begin{frame}
% \frametitle{Examples in ECLiPSe}
% \begin{itemize}
% \item Open sourced constraint programming language
% \item Development goes back to 1985
% \item ECRC, ICL, IC-Parc, PTL, Cisco
% \item \url{https://eclipseclp.org/}
% \item Specialities
% \begin{itemize}
% \item Develop new solvers for specific domains
% \item Integration with MIP
% \end{itemize}
% \item Not included in bundled MiniZinc IDE
% \item Specialized visualization tools used here
  % \begin{itemize}
  % \item CP-Viz, Simonis et al. 2010
    % \end{itemize}
% \end{itemize}
% \end{frame}

\begin{frame}
\frametitle{Tutorial Based on ECLiPSe ELearning Course}
\begin{itemize}
\item Self-study course in constraint programming
\item Supported by Cisco Systems and Silicon Valley Community Foundation
\item Multi-media format, video lectures, slides, handout etc
\item {\small \url{https://eclipseclp.org/ELearning/index.html}}
\end{itemize}
\end{frame}

\begin{frame}
\frametitle{Also Part of CRT-AI Constraint Week}
\begin{itemize}
\item Annual one week course on CP and Optimization in Ireland
\item Part of national training program for PhD students in AI
\item \url{https://www.crt-ai.ie/}
\end{itemize}
\end{frame}



\begin{frame}
\frametitle{Constraint Programming - in a nutshell}
\begin{itemize}
\item Declarative description of problems with
\begin{itemize}
\item {\em Variables} which range over (finite) sets of values
\item {\em Constraints} over subsets of variables which restrict possible value combinations
\item A {\em solution} is a value assignment which satisfies all constraints
\end{itemize}

\item Constraint propagation/reasoning
\begin{itemize}
\item Removing inconsistent values for variables
\item Detect failure if constraint can not be satisfied
\item Interaction of constraints via shared variables
\item Incomplete
\end{itemize}

\item Search
\begin{itemize}
\item User controlled assignment of values to variables
\item Each step triggers constraint propagation 
\end{itemize}
\item Different domains require/allow different methods
\end{itemize}
\end{frame}

\begin{frame}
  \frametitle{Constraint Programming is Different}
  \begin{itemize}
  \item Declarative Programming
    \begin{itemize}
    \item Concentrate on what you want
      \item Not how to get there
      \item Program != Algorithm
      \item Program = Model
    \end{itemize}
    \item Applied to Combinatorial Problems
      \begin{itemize}
        \item No complete polynomial algorithms known (exist?)
        \item CP less ad-hoc than heuristics
        \item Models can evolve
  \end{itemize}
  \end{itemize}
  \end{frame}
    
\begin{frame}
  \frametitle{A Subtractive Process}
  \begin{textblock}{4}(8,-3)
    \includegraphics[width=4cm]{../introduction/images/stages}
  \end{textblock}
  \vfill
  \begin{quote}
    ``Oh, bosh, as Mr. Ruskin says. Sculpture, per se, is the simplest thing in the world. All you have to do is to take a big chunk of marble and a hammer and chisel, make up your mind what you are about to create and chip off all the marble you don't want.''-Paris Gaulois.
  \end{quote}
  
  {\tiny Source: \url{https://quoteinvestigator.com/2014/06/22/chip-away/}}
\end{frame}

\begin{frame}
\frametitle{Basic Process}
\begin{center}
\begin{tikzpicture}
\node[shape=rectangle,fill=pantone127-4] (problem) at (2,5) {Problem};
\node (human) at (2,4) {Human};
\node[shape=rectangle,fill=pantone127-4] (model) at (2,3) {Model};
\node (solver) at (2,2) {Constraint Solver/Search};
\node[shape=rectangle,fill=pantone157-8] (solution) at (2,1) {Solution};
\draw[-] (problem) -- (human);
\draw[->] (human) -- (model);
\draw[-] (model) -- (solver);
\draw[->] (solver) -- (solution);
\end{tikzpicture}
\end{center}
\end{frame}

\begin{frame}
\frametitle{More Realistic}
\begin{center}
\begin{tikzpicture}[xscale=4.0]
\node[shape=rectangle,fill=pantone127-4] (problem) at (2,5) {Problem};
\node (human) at (2,4) {Human};
\node[shape=rectangle,fill=pantone127-4] (model) at (2,3) {Model};
\node (solver) at (2,2) {Constraint Solver/Search};
\node[shape=rectangle,fill=pantone24-8] (hangs) at (1,1) {Hangs};
\node[shape=rectangle,fill=pantone157-8] (solution) at (2,1) {Solution};
\node[shape=rectangle,fill=pantone24-8] (wrong) at (3,1) {Wrong Solution};
\draw[-] (problem) -- (human);
\draw[->] (human) -- (model);
\draw[-] (model) -- (solver);
\draw[->] (solver) -- (hangs);
\draw[->] (solver) -- (solution);
\draw[->] (solver) -- (wrong);
\draw[->] (hangs) -- (1,4) -- (human);
\draw[->] (wrong) -- (3,4) -- (human);
\end{tikzpicture}
\end{center}
\end{frame}

\begin{frame}
\frametitle{Dual Role of Model}
\begin{itemize}
\item Allows Human to Express Problem
\begin{itemize}
\item Close to Problem Domain
\item Constraints as Abstractions
\end{itemize}
\item Allows Solver to Execute
\begin{itemize}
\item Variables as Communication Mechanism
\item Constraints as Algorithms
\end{itemize}
\end{itemize}
\end{frame}

\begin{frame}
\frametitle{Modelling Frameworks}
\begin{itemize}
\item \href{https://www.minizinc.org/}{MiniZinc} (NICTA, Monash University, Australia)
\item \href{https://github.com/eomahony/Numberjack}{NumberJack} (Insight, Ireland) 
\item \href{https://www-users.york.ac.uk/peter.nightingale/savilerow//}{EssencePrime/SavilleRow} (UK) 
\item \href{https://cpmpy.readthedocs.io/en/latest/}{CPMpy} (KU Leuven)
\item Allow use of multiple back-end solvers
\item Compile model into variants for each solver
\item A priori solver independent model(CP, MIP, SAT)
\end{itemize}
\end{frame} 

\begin{frame}
\frametitle{Framework Process}
\begin{center}
\begin{tikzpicture}[xscale=2]
\node[shape=rectangle,fill=pantone127-4] (problem) at (2.5,5) {Problem};
\node (human) at (2.5,4) {Human};
\node[shape=rectangle,fill=pantone127-4] (model) at (2.5,3) {Model};
\node (compile) at (2.5,2) {Compile/Reformulate};
\node (solver1) at (1,1) {CP};
\node (solver2) at (2,1) {MIP};
\node (solver3) at (3,1) {SAT};
\node (solver4) at (4,1) {Other};
\node[shape=rectangle,fill=pantone157-8] (solution1) at (1,0) {Solution};
\node[shape=rectangle,fill=pantone157-8] (solution2) at (2,0) {Solution};
\node[shape=rectangle,fill=pantone157-8] (solution3) at (3,0) {Solution};
\node[shape=rectangle,fill=pantone157-8] (solution4) at (4,0) {Solution};
\draw[-] (problem) -- (human);
\draw[->] (human) -- (model);
\draw[-] (model) -- (compile);
\draw[-] (compile) -- (solver1);
\draw[-] (compile) -- (solver2);
\draw[-] (compile) -- (solver3);
\draw[-] (compile) -- (solver4);
\draw[->] (solver1) -- (solution1);
\draw[->] (solver2) -- (solution2);
\draw[->] (solver3) -- (solution3);
\draw[->] (solver4) -- (solution4);
\end{tikzpicture}
\end{center}
\end{frame}

% \begin{frame}
  % \frametitle{Do It Now!}
  % \begin{itemize}
  % \item Download and install Minizinc
  % \item \url{https://www.minizinc.org/}
  % \end{itemize}
% \end{frame}


\begin{frame}
\frametitle{Why use Puzzles as Examples?}
\begin{itemize}
\item Easy to understand the problem
\item Solvable by hand without specialized knowledge
\item Possible to compare automated to manual solving process
\end{itemize}

\begin{quote}
The puzzle, though inanimate, is presented as a solvable problem without lasting negative consequences, a very low-risk low-reward situation. By being a puzzle, the object is attempting to convince the user that it must be completed.
\end{quote}
{\tiny Source: \href{https://www.everydayrhetoric.com/2020/03/the-rhetoric-of-puzzles/}{Every Day Rhetoric}}
\end{frame}


\mode<all>{
\part{Basic Constraint Propagation}
}
\begin{frame}
\partpage
\end{frame}
\mode<all>{
\input{../sendmore/body}
}

\mode<all>{
\part{Global Constraints}
}
\begin{frame}
\partpage
\end{frame}
\mode<all>{
\input{../sudoku/body}
}

\clearpage
\mode<all>{
\part{Customizing Search}
}
\begin{frame}
\partpage
\end{frame}
\mode<all>{
\input{../nqueen/body}
}


%% \clearpage
%% \mode<all>{
%%   \part{Problem Decomposition}
%% }
%% \begin{frame}
%%   \partpage
%% \end{frame}
%% \mode<all>{
%%   \input{../party/body}
%%   }

%% \clearpage
%% \mode<all>{
%%   \part{More Global Constraints}
%% }
%% \begin{frame}
%%   \partpage
%% \end{frame}
%% \mode<all>{
%%   \input{../car/body}
%%   }

%% \clearpage
%% \mode<all>{
%%   \part{Methodology}
%% }
%% \begin{frame}
%%   \partpage
%% \end{frame}
%% \mode<all>{
%%   \input{../methodology/body}
%%   }

\clearpage
\mode<all>{
\part{What is missing?}
}
\begin{frame}
\partpage
\end{frame}

\begin{frame}
\frametitle{Many Specialized Topics}
\begin{itemize}
\item How to design efficient core engine
\item Hybrids with LP/MIP tools
\item Hybrids with SAT
\item Symmetry breaking
\item Use of MDD/BDD to encode sets of solutions
\item High level modelling tools
\item Debugging/visualization
\end{itemize}
\end{frame}

\begin{frame}
\frametitle{Reformulation}
\begin{itemize}
\item Just because the user has modelled it this way, it doesn't mean we have to solve it that way
\begin{itemize}
\item Replace some constraint(s) by other, equivalent constraints
\item Because we don't have that constraint in our system
\item For performance
\end{itemize}
\end{itemize}
\end{frame}

\begin{frame}
\frametitle{Learning}
\begin{itemize}
\item While solving the problem we can learn how to strengthen the model/search 
\begin{itemize}
\item Understand which constraints/method contribute to propagation and change schedule
\item Learn no-good constraints by explaining failure
\item Adapt search strategy based on search experience
\end{itemize}
\end{itemize}
\end{frame}

%% \begin{frame}
%% \frametitle{Refined Process}
%% \begin{center}
%% \begin{tikzpicture}
%% \node[shape=rectangle,fill=pantone127-4] (problem) at (3,7) {Problem};
%% \node (human) at (3,6) {Human};
%% \node[shape=rectangle,fill=pantone127-4] (model) at (3,5) {User Model};
%% \node (reformulation) at (3,4) {Reformulation};
%% \node[shape=rectangle,fill=pantone127-4] (implementation) at (3,3) {Implementation Model};
%% \node (solver) at (3,2) {Constraint Solver/Search};
%% \node[shape=rectangle,fill=pantone157-8] (solution) at (3,1) {Solution};
%% \draw[-] (problem) -- (human);
%% \draw[->] (human) -- (model);
%% \draw[-] (model) -- (reformulation);
%% \draw[->] (reformulation) -- (implementation);
%% \draw[-] (implementation) -- (solver);
%% \draw[->] (solver) -- (0,2) -- (0,3) -- (implementation);
%% \draw[->] (solver) -- (solution);
%% \end{tikzpicture}
%% \end{center}
%% \end{frame}

%\begin{frame}
%\frametitle{What is CP actually used for?}
%\url{http://hsimonis.wordpress.com}
%\includegraphics[width=10cm]{../introduction/images/blog}
%\end{frame}

%% \begin{frame}
%% \frametitle{Exercises}
%% \begin{itemize}
%% \item Install Constraint Solver of your Choice
%% \item MiniZinc (MiningZinc) will be used in later sessions
%% \item Run Sendmoremoney, Sudoku, Queens examples
%% \item Can you choose which propagator to use for alldifferent?
%% \item Can you use different search strategies?
%% \end{itemize}
%% \end{frame}

\begin{frame}
\frametitle{More Learning Resources}
\begin{itemize}
\item Survey of Methods, Resources, and Formats for Teaching Constraint Programming
\begin{itemize}
\item by Tejas Santanam, Helmut Simonis
\item \url{https://doi.org/10.48550/arXiv.2403.12717}
\item Based on survey of community for WTCP 2023
\item \url{https://hsimonis.github.io/WTCP2023/}
\end{itemize}
\end{itemize}
\end{frame}


}

\clearpage
\mode<all>{
\part{Customizing Search}
}
\begin{frame}
\partpage
\end{frame}
\mode<all>{
\begin{frame}
\frametitle{Objectives}
\begin{itemize}
\item Overview of Core Constraint Programming
\item Three Main Concepts
\begin{itemize}
\item Constraint Propagation
\item Global Constraints
\item Customizing Search
\end{itemize}
%\item Get Some Experience with MiniZinc
\item Topics will be treated in more detail in later parts of the school
\item Based on Examples, not Formal Description
\end{itemize}
\end{frame}

\begin{frame}
\frametitle{Outline}
\begin{itemize}
\item Why Constraint Programming?
\item Constraint Propagation
\item Global Constraints
\item Customizing Search
%\item Choices in Modelling
%\item Decomposition
%\item More Global Constraints
%\item Methodology
%\item What is missing?
\end{itemize}
\end{frame}

% \begin{frame}
  % \frametitle{Using MiniZinc IDE}
  % \begin{itemize}
  % \item Developed in the Australian NICTA project
  % \item Maintained by Monash University
  % \item Modelling tool with multiple back-end solvers
  % \item Available from \url{https://www.minizinc.org/}
  % \end{itemize}
% \end{frame}


% \begin{frame}
% \frametitle{Examples in ECLiPSe}
% \begin{itemize}
% \item Open sourced constraint programming language
% \item Development goes back to 1985
% \item ECRC, ICL, IC-Parc, PTL, Cisco
% \item \url{https://eclipseclp.org/}
% \item Specialities
% \begin{itemize}
% \item Develop new solvers for specific domains
% \item Integration with MIP
% \end{itemize}
% \item Not included in bundled MiniZinc IDE
% \item Specialized visualization tools used here
  % \begin{itemize}
  % \item CP-Viz, Simonis et al. 2010
    % \end{itemize}
% \end{itemize}
% \end{frame}

\begin{frame}
\frametitle{Tutorial Based on ECLiPSe ELearning Course}
\begin{itemize}
\item Self-study course in constraint programming
\item Supported by Cisco Systems and Silicon Valley Community Foundation
\item Multi-media format, video lectures, slides, handout etc
\item {\small \url{https://eclipseclp.org/ELearning/index.html}}
\end{itemize}
\end{frame}

\begin{frame}
\frametitle{Also Part of CRT-AI Constraint Week}
\begin{itemize}
\item Annual one week course on CP and Optimization in Ireland
\item Part of national training program for PhD students in AI
\item \url{https://www.crt-ai.ie/}
\end{itemize}
\end{frame}



\begin{frame}
\frametitle{Constraint Programming - in a nutshell}
\begin{itemize}
\item Declarative description of problems with
\begin{itemize}
\item {\em Variables} which range over (finite) sets of values
\item {\em Constraints} over subsets of variables which restrict possible value combinations
\item A {\em solution} is a value assignment which satisfies all constraints
\end{itemize}

\item Constraint propagation/reasoning
\begin{itemize}
\item Removing inconsistent values for variables
\item Detect failure if constraint can not be satisfied
\item Interaction of constraints via shared variables
\item Incomplete
\end{itemize}

\item Search
\begin{itemize}
\item User controlled assignment of values to variables
\item Each step triggers constraint propagation 
\end{itemize}
\item Different domains require/allow different methods
\end{itemize}
\end{frame}

\begin{frame}
  \frametitle{Constraint Programming is Different}
  \begin{itemize}
  \item Declarative Programming
    \begin{itemize}
    \item Concentrate on what you want
      \item Not how to get there
      \item Program != Algorithm
      \item Program = Model
    \end{itemize}
    \item Applied to Combinatorial Problems
      \begin{itemize}
        \item No complete polynomial algorithms known (exist?)
        \item CP less ad-hoc than heuristics
        \item Models can evolve
  \end{itemize}
  \end{itemize}
  \end{frame}
    
\begin{frame}
  \frametitle{A Subtractive Process}
  \begin{textblock}{4}(8,-3)
    \includegraphics[width=4cm]{../introduction/images/stages}
  \end{textblock}
  \vfill
  \begin{quote}
    ``Oh, bosh, as Mr. Ruskin says. Sculpture, per se, is the simplest thing in the world. All you have to do is to take a big chunk of marble and a hammer and chisel, make up your mind what you are about to create and chip off all the marble you don't want.''-Paris Gaulois.
  \end{quote}
  
  {\tiny Source: \url{https://quoteinvestigator.com/2014/06/22/chip-away/}}
\end{frame}

\begin{frame}
\frametitle{Basic Process}
\begin{center}
\begin{tikzpicture}
\node[shape=rectangle,fill=pantone127-4] (problem) at (2,5) {Problem};
\node (human) at (2,4) {Human};
\node[shape=rectangle,fill=pantone127-4] (model) at (2,3) {Model};
\node (solver) at (2,2) {Constraint Solver/Search};
\node[shape=rectangle,fill=pantone157-8] (solution) at (2,1) {Solution};
\draw[-] (problem) -- (human);
\draw[->] (human) -- (model);
\draw[-] (model) -- (solver);
\draw[->] (solver) -- (solution);
\end{tikzpicture}
\end{center}
\end{frame}

\begin{frame}
\frametitle{More Realistic}
\begin{center}
\begin{tikzpicture}[xscale=4.0]
\node[shape=rectangle,fill=pantone127-4] (problem) at (2,5) {Problem};
\node (human) at (2,4) {Human};
\node[shape=rectangle,fill=pantone127-4] (model) at (2,3) {Model};
\node (solver) at (2,2) {Constraint Solver/Search};
\node[shape=rectangle,fill=pantone24-8] (hangs) at (1,1) {Hangs};
\node[shape=rectangle,fill=pantone157-8] (solution) at (2,1) {Solution};
\node[shape=rectangle,fill=pantone24-8] (wrong) at (3,1) {Wrong Solution};
\draw[-] (problem) -- (human);
\draw[->] (human) -- (model);
\draw[-] (model) -- (solver);
\draw[->] (solver) -- (hangs);
\draw[->] (solver) -- (solution);
\draw[->] (solver) -- (wrong);
\draw[->] (hangs) -- (1,4) -- (human);
\draw[->] (wrong) -- (3,4) -- (human);
\end{tikzpicture}
\end{center}
\end{frame}

\begin{frame}
\frametitle{Dual Role of Model}
\begin{itemize}
\item Allows Human to Express Problem
\begin{itemize}
\item Close to Problem Domain
\item Constraints as Abstractions
\end{itemize}
\item Allows Solver to Execute
\begin{itemize}
\item Variables as Communication Mechanism
\item Constraints as Algorithms
\end{itemize}
\end{itemize}
\end{frame}

\begin{frame}
\frametitle{Modelling Frameworks}
\begin{itemize}
\item \href{https://www.minizinc.org/}{MiniZinc} (NICTA, Monash University, Australia)
\item \href{https://github.com/eomahony/Numberjack}{NumberJack} (Insight, Ireland) 
\item \href{https://www-users.york.ac.uk/peter.nightingale/savilerow//}{EssencePrime/SavilleRow} (UK) 
\item \href{https://cpmpy.readthedocs.io/en/latest/}{CPMpy} (KU Leuven)
\item Allow use of multiple back-end solvers
\item Compile model into variants for each solver
\item A priori solver independent model(CP, MIP, SAT)
\end{itemize}
\end{frame} 

\begin{frame}
\frametitle{Framework Process}
\begin{center}
\begin{tikzpicture}[xscale=2]
\node[shape=rectangle,fill=pantone127-4] (problem) at (2.5,5) {Problem};
\node (human) at (2.5,4) {Human};
\node[shape=rectangle,fill=pantone127-4] (model) at (2.5,3) {Model};
\node (compile) at (2.5,2) {Compile/Reformulate};
\node (solver1) at (1,1) {CP};
\node (solver2) at (2,1) {MIP};
\node (solver3) at (3,1) {SAT};
\node (solver4) at (4,1) {Other};
\node[shape=rectangle,fill=pantone157-8] (solution1) at (1,0) {Solution};
\node[shape=rectangle,fill=pantone157-8] (solution2) at (2,0) {Solution};
\node[shape=rectangle,fill=pantone157-8] (solution3) at (3,0) {Solution};
\node[shape=rectangle,fill=pantone157-8] (solution4) at (4,0) {Solution};
\draw[-] (problem) -- (human);
\draw[->] (human) -- (model);
\draw[-] (model) -- (compile);
\draw[-] (compile) -- (solver1);
\draw[-] (compile) -- (solver2);
\draw[-] (compile) -- (solver3);
\draw[-] (compile) -- (solver4);
\draw[->] (solver1) -- (solution1);
\draw[->] (solver2) -- (solution2);
\draw[->] (solver3) -- (solution3);
\draw[->] (solver4) -- (solution4);
\end{tikzpicture}
\end{center}
\end{frame}

% \begin{frame}
  % \frametitle{Do It Now!}
  % \begin{itemize}
  % \item Download and install Minizinc
  % \item \url{https://www.minizinc.org/}
  % \end{itemize}
% \end{frame}


\begin{frame}
\frametitle{Why use Puzzles as Examples?}
\begin{itemize}
\item Easy to understand the problem
\item Solvable by hand without specialized knowledge
\item Possible to compare automated to manual solving process
\end{itemize}

\begin{quote}
The puzzle, though inanimate, is presented as a solvable problem without lasting negative consequences, a very low-risk low-reward situation. By being a puzzle, the object is attempting to convince the user that it must be completed.
\end{quote}
{\tiny Source: \href{https://www.everydayrhetoric.com/2020/03/the-rhetoric-of-puzzles/}{Every Day Rhetoric}}
\end{frame}


\mode<all>{
\part{Basic Constraint Propagation}
}
\begin{frame}
\partpage
\end{frame}
\mode<all>{
\input{../sendmore/body}
}

\mode<all>{
\part{Global Constraints}
}
\begin{frame}
\partpage
\end{frame}
\mode<all>{
\input{../sudoku/body}
}

\clearpage
\mode<all>{
\part{Customizing Search}
}
\begin{frame}
\partpage
\end{frame}
\mode<all>{
\input{../nqueen/body}
}


%% \clearpage
%% \mode<all>{
%%   \part{Problem Decomposition}
%% }
%% \begin{frame}
%%   \partpage
%% \end{frame}
%% \mode<all>{
%%   \input{../party/body}
%%   }

%% \clearpage
%% \mode<all>{
%%   \part{More Global Constraints}
%% }
%% \begin{frame}
%%   \partpage
%% \end{frame}
%% \mode<all>{
%%   \input{../car/body}
%%   }

%% \clearpage
%% \mode<all>{
%%   \part{Methodology}
%% }
%% \begin{frame}
%%   \partpage
%% \end{frame}
%% \mode<all>{
%%   \input{../methodology/body}
%%   }

\clearpage
\mode<all>{
\part{What is missing?}
}
\begin{frame}
\partpage
\end{frame}

\begin{frame}
\frametitle{Many Specialized Topics}
\begin{itemize}
\item How to design efficient core engine
\item Hybrids with LP/MIP tools
\item Hybrids with SAT
\item Symmetry breaking
\item Use of MDD/BDD to encode sets of solutions
\item High level modelling tools
\item Debugging/visualization
\end{itemize}
\end{frame}

\begin{frame}
\frametitle{Reformulation}
\begin{itemize}
\item Just because the user has modelled it this way, it doesn't mean we have to solve it that way
\begin{itemize}
\item Replace some constraint(s) by other, equivalent constraints
\item Because we don't have that constraint in our system
\item For performance
\end{itemize}
\end{itemize}
\end{frame}

\begin{frame}
\frametitle{Learning}
\begin{itemize}
\item While solving the problem we can learn how to strengthen the model/search 
\begin{itemize}
\item Understand which constraints/method contribute to propagation and change schedule
\item Learn no-good constraints by explaining failure
\item Adapt search strategy based on search experience
\end{itemize}
\end{itemize}
\end{frame}

%% \begin{frame}
%% \frametitle{Refined Process}
%% \begin{center}
%% \begin{tikzpicture}
%% \node[shape=rectangle,fill=pantone127-4] (problem) at (3,7) {Problem};
%% \node (human) at (3,6) {Human};
%% \node[shape=rectangle,fill=pantone127-4] (model) at (3,5) {User Model};
%% \node (reformulation) at (3,4) {Reformulation};
%% \node[shape=rectangle,fill=pantone127-4] (implementation) at (3,3) {Implementation Model};
%% \node (solver) at (3,2) {Constraint Solver/Search};
%% \node[shape=rectangle,fill=pantone157-8] (solution) at (3,1) {Solution};
%% \draw[-] (problem) -- (human);
%% \draw[->] (human) -- (model);
%% \draw[-] (model) -- (reformulation);
%% \draw[->] (reformulation) -- (implementation);
%% \draw[-] (implementation) -- (solver);
%% \draw[->] (solver) -- (0,2) -- (0,3) -- (implementation);
%% \draw[->] (solver) -- (solution);
%% \end{tikzpicture}
%% \end{center}
%% \end{frame}

%\begin{frame}
%\frametitle{What is CP actually used for?}
%\url{http://hsimonis.wordpress.com}
%\includegraphics[width=10cm]{../introduction/images/blog}
%\end{frame}

%% \begin{frame}
%% \frametitle{Exercises}
%% \begin{itemize}
%% \item Install Constraint Solver of your Choice
%% \item MiniZinc (MiningZinc) will be used in later sessions
%% \item Run Sendmoremoney, Sudoku, Queens examples
%% \item Can you choose which propagator to use for alldifferent?
%% \item Can you use different search strategies?
%% \end{itemize}
%% \end{frame}

\begin{frame}
\frametitle{More Learning Resources}
\begin{itemize}
\item Survey of Methods, Resources, and Formats for Teaching Constraint Programming
\begin{itemize}
\item by Tejas Santanam, Helmut Simonis
\item \url{https://doi.org/10.48550/arXiv.2403.12717}
\item Based on survey of community for WTCP 2023
\item \url{https://hsimonis.github.io/WTCP2023/}
\end{itemize}
\end{itemize}
\end{frame}


}


%% \clearpage
%% \mode<all>{
%%   \part{Problem Decomposition}
%% }
%% \begin{frame}
%%   \partpage
%% \end{frame}
%% \mode<all>{
%%   \begin{frame}
\frametitle{Objectives}
\begin{itemize}
\item Overview of Core Constraint Programming
\item Three Main Concepts
\begin{itemize}
\item Constraint Propagation
\item Global Constraints
\item Customizing Search
\end{itemize}
%\item Get Some Experience with MiniZinc
\item Topics will be treated in more detail in later parts of the school
\item Based on Examples, not Formal Description
\end{itemize}
\end{frame}

\begin{frame}
\frametitle{Outline}
\begin{itemize}
\item Why Constraint Programming?
\item Constraint Propagation
\item Global Constraints
\item Customizing Search
%\item Choices in Modelling
%\item Decomposition
%\item More Global Constraints
%\item Methodology
%\item What is missing?
\end{itemize}
\end{frame}

% \begin{frame}
  % \frametitle{Using MiniZinc IDE}
  % \begin{itemize}
  % \item Developed in the Australian NICTA project
  % \item Maintained by Monash University
  % \item Modelling tool with multiple back-end solvers
  % \item Available from \url{https://www.minizinc.org/}
  % \end{itemize}
% \end{frame}


% \begin{frame}
% \frametitle{Examples in ECLiPSe}
% \begin{itemize}
% \item Open sourced constraint programming language
% \item Development goes back to 1985
% \item ECRC, ICL, IC-Parc, PTL, Cisco
% \item \url{https://eclipseclp.org/}
% \item Specialities
% \begin{itemize}
% \item Develop new solvers for specific domains
% \item Integration with MIP
% \end{itemize}
% \item Not included in bundled MiniZinc IDE
% \item Specialized visualization tools used here
  % \begin{itemize}
  % \item CP-Viz, Simonis et al. 2010
    % \end{itemize}
% \end{itemize}
% \end{frame}

\begin{frame}
\frametitle{Tutorial Based on ECLiPSe ELearning Course}
\begin{itemize}
\item Self-study course in constraint programming
\item Supported by Cisco Systems and Silicon Valley Community Foundation
\item Multi-media format, video lectures, slides, handout etc
\item {\small \url{https://eclipseclp.org/ELearning/index.html}}
\end{itemize}
\end{frame}

\begin{frame}
\frametitle{Also Part of CRT-AI Constraint Week}
\begin{itemize}
\item Annual one week course on CP and Optimization in Ireland
\item Part of national training program for PhD students in AI
\item \url{https://www.crt-ai.ie/}
\end{itemize}
\end{frame}



\begin{frame}
\frametitle{Constraint Programming - in a nutshell}
\begin{itemize}
\item Declarative description of problems with
\begin{itemize}
\item {\em Variables} which range over (finite) sets of values
\item {\em Constraints} over subsets of variables which restrict possible value combinations
\item A {\em solution} is a value assignment which satisfies all constraints
\end{itemize}

\item Constraint propagation/reasoning
\begin{itemize}
\item Removing inconsistent values for variables
\item Detect failure if constraint can not be satisfied
\item Interaction of constraints via shared variables
\item Incomplete
\end{itemize}

\item Search
\begin{itemize}
\item User controlled assignment of values to variables
\item Each step triggers constraint propagation 
\end{itemize}
\item Different domains require/allow different methods
\end{itemize}
\end{frame}

\begin{frame}
  \frametitle{Constraint Programming is Different}
  \begin{itemize}
  \item Declarative Programming
    \begin{itemize}
    \item Concentrate on what you want
      \item Not how to get there
      \item Program != Algorithm
      \item Program = Model
    \end{itemize}
    \item Applied to Combinatorial Problems
      \begin{itemize}
        \item No complete polynomial algorithms known (exist?)
        \item CP less ad-hoc than heuristics
        \item Models can evolve
  \end{itemize}
  \end{itemize}
  \end{frame}
    
\begin{frame}
  \frametitle{A Subtractive Process}
  \begin{textblock}{4}(8,-3)
    \includegraphics[width=4cm]{../introduction/images/stages}
  \end{textblock}
  \vfill
  \begin{quote}
    ``Oh, bosh, as Mr. Ruskin says. Sculpture, per se, is the simplest thing in the world. All you have to do is to take a big chunk of marble and a hammer and chisel, make up your mind what you are about to create and chip off all the marble you don't want.''-Paris Gaulois.
  \end{quote}
  
  {\tiny Source: \url{https://quoteinvestigator.com/2014/06/22/chip-away/}}
\end{frame}

\begin{frame}
\frametitle{Basic Process}
\begin{center}
\begin{tikzpicture}
\node[shape=rectangle,fill=pantone127-4] (problem) at (2,5) {Problem};
\node (human) at (2,4) {Human};
\node[shape=rectangle,fill=pantone127-4] (model) at (2,3) {Model};
\node (solver) at (2,2) {Constraint Solver/Search};
\node[shape=rectangle,fill=pantone157-8] (solution) at (2,1) {Solution};
\draw[-] (problem) -- (human);
\draw[->] (human) -- (model);
\draw[-] (model) -- (solver);
\draw[->] (solver) -- (solution);
\end{tikzpicture}
\end{center}
\end{frame}

\begin{frame}
\frametitle{More Realistic}
\begin{center}
\begin{tikzpicture}[xscale=4.0]
\node[shape=rectangle,fill=pantone127-4] (problem) at (2,5) {Problem};
\node (human) at (2,4) {Human};
\node[shape=rectangle,fill=pantone127-4] (model) at (2,3) {Model};
\node (solver) at (2,2) {Constraint Solver/Search};
\node[shape=rectangle,fill=pantone24-8] (hangs) at (1,1) {Hangs};
\node[shape=rectangle,fill=pantone157-8] (solution) at (2,1) {Solution};
\node[shape=rectangle,fill=pantone24-8] (wrong) at (3,1) {Wrong Solution};
\draw[-] (problem) -- (human);
\draw[->] (human) -- (model);
\draw[-] (model) -- (solver);
\draw[->] (solver) -- (hangs);
\draw[->] (solver) -- (solution);
\draw[->] (solver) -- (wrong);
\draw[->] (hangs) -- (1,4) -- (human);
\draw[->] (wrong) -- (3,4) -- (human);
\end{tikzpicture}
\end{center}
\end{frame}

\begin{frame}
\frametitle{Dual Role of Model}
\begin{itemize}
\item Allows Human to Express Problem
\begin{itemize}
\item Close to Problem Domain
\item Constraints as Abstractions
\end{itemize}
\item Allows Solver to Execute
\begin{itemize}
\item Variables as Communication Mechanism
\item Constraints as Algorithms
\end{itemize}
\end{itemize}
\end{frame}

\begin{frame}
\frametitle{Modelling Frameworks}
\begin{itemize}
\item \href{https://www.minizinc.org/}{MiniZinc} (NICTA, Monash University, Australia)
\item \href{https://github.com/eomahony/Numberjack}{NumberJack} (Insight, Ireland) 
\item \href{https://www-users.york.ac.uk/peter.nightingale/savilerow//}{EssencePrime/SavilleRow} (UK) 
\item \href{https://cpmpy.readthedocs.io/en/latest/}{CPMpy} (KU Leuven)
\item Allow use of multiple back-end solvers
\item Compile model into variants for each solver
\item A priori solver independent model(CP, MIP, SAT)
\end{itemize}
\end{frame} 

\begin{frame}
\frametitle{Framework Process}
\begin{center}
\begin{tikzpicture}[xscale=2]
\node[shape=rectangle,fill=pantone127-4] (problem) at (2.5,5) {Problem};
\node (human) at (2.5,4) {Human};
\node[shape=rectangle,fill=pantone127-4] (model) at (2.5,3) {Model};
\node (compile) at (2.5,2) {Compile/Reformulate};
\node (solver1) at (1,1) {CP};
\node (solver2) at (2,1) {MIP};
\node (solver3) at (3,1) {SAT};
\node (solver4) at (4,1) {Other};
\node[shape=rectangle,fill=pantone157-8] (solution1) at (1,0) {Solution};
\node[shape=rectangle,fill=pantone157-8] (solution2) at (2,0) {Solution};
\node[shape=rectangle,fill=pantone157-8] (solution3) at (3,0) {Solution};
\node[shape=rectangle,fill=pantone157-8] (solution4) at (4,0) {Solution};
\draw[-] (problem) -- (human);
\draw[->] (human) -- (model);
\draw[-] (model) -- (compile);
\draw[-] (compile) -- (solver1);
\draw[-] (compile) -- (solver2);
\draw[-] (compile) -- (solver3);
\draw[-] (compile) -- (solver4);
\draw[->] (solver1) -- (solution1);
\draw[->] (solver2) -- (solution2);
\draw[->] (solver3) -- (solution3);
\draw[->] (solver4) -- (solution4);
\end{tikzpicture}
\end{center}
\end{frame}

% \begin{frame}
  % \frametitle{Do It Now!}
  % \begin{itemize}
  % \item Download and install Minizinc
  % \item \url{https://www.minizinc.org/}
  % \end{itemize}
% \end{frame}


\begin{frame}
\frametitle{Why use Puzzles as Examples?}
\begin{itemize}
\item Easy to understand the problem
\item Solvable by hand without specialized knowledge
\item Possible to compare automated to manual solving process
\end{itemize}

\begin{quote}
The puzzle, though inanimate, is presented as a solvable problem without lasting negative consequences, a very low-risk low-reward situation. By being a puzzle, the object is attempting to convince the user that it must be completed.
\end{quote}
{\tiny Source: \href{https://www.everydayrhetoric.com/2020/03/the-rhetoric-of-puzzles/}{Every Day Rhetoric}}
\end{frame}


\mode<all>{
\part{Basic Constraint Propagation}
}
\begin{frame}
\partpage
\end{frame}
\mode<all>{
\input{../sendmore/body}
}

\mode<all>{
\part{Global Constraints}
}
\begin{frame}
\partpage
\end{frame}
\mode<all>{
\input{../sudoku/body}
}

\clearpage
\mode<all>{
\part{Customizing Search}
}
\begin{frame}
\partpage
\end{frame}
\mode<all>{
\input{../nqueen/body}
}


%% \clearpage
%% \mode<all>{
%%   \part{Problem Decomposition}
%% }
%% \begin{frame}
%%   \partpage
%% \end{frame}
%% \mode<all>{
%%   \input{../party/body}
%%   }

%% \clearpage
%% \mode<all>{
%%   \part{More Global Constraints}
%% }
%% \begin{frame}
%%   \partpage
%% \end{frame}
%% \mode<all>{
%%   \input{../car/body}
%%   }

%% \clearpage
%% \mode<all>{
%%   \part{Methodology}
%% }
%% \begin{frame}
%%   \partpage
%% \end{frame}
%% \mode<all>{
%%   \input{../methodology/body}
%%   }

\clearpage
\mode<all>{
\part{What is missing?}
}
\begin{frame}
\partpage
\end{frame}

\begin{frame}
\frametitle{Many Specialized Topics}
\begin{itemize}
\item How to design efficient core engine
\item Hybrids with LP/MIP tools
\item Hybrids with SAT
\item Symmetry breaking
\item Use of MDD/BDD to encode sets of solutions
\item High level modelling tools
\item Debugging/visualization
\end{itemize}
\end{frame}

\begin{frame}
\frametitle{Reformulation}
\begin{itemize}
\item Just because the user has modelled it this way, it doesn't mean we have to solve it that way
\begin{itemize}
\item Replace some constraint(s) by other, equivalent constraints
\item Because we don't have that constraint in our system
\item For performance
\end{itemize}
\end{itemize}
\end{frame}

\begin{frame}
\frametitle{Learning}
\begin{itemize}
\item While solving the problem we can learn how to strengthen the model/search 
\begin{itemize}
\item Understand which constraints/method contribute to propagation and change schedule
\item Learn no-good constraints by explaining failure
\item Adapt search strategy based on search experience
\end{itemize}
\end{itemize}
\end{frame}

%% \begin{frame}
%% \frametitle{Refined Process}
%% \begin{center}
%% \begin{tikzpicture}
%% \node[shape=rectangle,fill=pantone127-4] (problem) at (3,7) {Problem};
%% \node (human) at (3,6) {Human};
%% \node[shape=rectangle,fill=pantone127-4] (model) at (3,5) {User Model};
%% \node (reformulation) at (3,4) {Reformulation};
%% \node[shape=rectangle,fill=pantone127-4] (implementation) at (3,3) {Implementation Model};
%% \node (solver) at (3,2) {Constraint Solver/Search};
%% \node[shape=rectangle,fill=pantone157-8] (solution) at (3,1) {Solution};
%% \draw[-] (problem) -- (human);
%% \draw[->] (human) -- (model);
%% \draw[-] (model) -- (reformulation);
%% \draw[->] (reformulation) -- (implementation);
%% \draw[-] (implementation) -- (solver);
%% \draw[->] (solver) -- (0,2) -- (0,3) -- (implementation);
%% \draw[->] (solver) -- (solution);
%% \end{tikzpicture}
%% \end{center}
%% \end{frame}

%\begin{frame}
%\frametitle{What is CP actually used for?}
%\url{http://hsimonis.wordpress.com}
%\includegraphics[width=10cm]{../introduction/images/blog}
%\end{frame}

%% \begin{frame}
%% \frametitle{Exercises}
%% \begin{itemize}
%% \item Install Constraint Solver of your Choice
%% \item MiniZinc (MiningZinc) will be used in later sessions
%% \item Run Sendmoremoney, Sudoku, Queens examples
%% \item Can you choose which propagator to use for alldifferent?
%% \item Can you use different search strategies?
%% \end{itemize}
%% \end{frame}

\begin{frame}
\frametitle{More Learning Resources}
\begin{itemize}
\item Survey of Methods, Resources, and Formats for Teaching Constraint Programming
\begin{itemize}
\item by Tejas Santanam, Helmut Simonis
\item \url{https://doi.org/10.48550/arXiv.2403.12717}
\item Based on survey of community for WTCP 2023
\item \url{https://hsimonis.github.io/WTCP2023/}
\end{itemize}
\end{itemize}
\end{frame}


%%   }

%% \clearpage
%% \mode<all>{
%%   \part{More Global Constraints}
%% }
%% \begin{frame}
%%   \partpage
%% \end{frame}
%% \mode<all>{
%%   \begin{frame}
\frametitle{Objectives}
\begin{itemize}
\item Overview of Core Constraint Programming
\item Three Main Concepts
\begin{itemize}
\item Constraint Propagation
\item Global Constraints
\item Customizing Search
\end{itemize}
%\item Get Some Experience with MiniZinc
\item Topics will be treated in more detail in later parts of the school
\item Based on Examples, not Formal Description
\end{itemize}
\end{frame}

\begin{frame}
\frametitle{Outline}
\begin{itemize}
\item Why Constraint Programming?
\item Constraint Propagation
\item Global Constraints
\item Customizing Search
%\item Choices in Modelling
%\item Decomposition
%\item More Global Constraints
%\item Methodology
%\item What is missing?
\end{itemize}
\end{frame}

% \begin{frame}
  % \frametitle{Using MiniZinc IDE}
  % \begin{itemize}
  % \item Developed in the Australian NICTA project
  % \item Maintained by Monash University
  % \item Modelling tool with multiple back-end solvers
  % \item Available from \url{https://www.minizinc.org/}
  % \end{itemize}
% \end{frame}


% \begin{frame}
% \frametitle{Examples in ECLiPSe}
% \begin{itemize}
% \item Open sourced constraint programming language
% \item Development goes back to 1985
% \item ECRC, ICL, IC-Parc, PTL, Cisco
% \item \url{https://eclipseclp.org/}
% \item Specialities
% \begin{itemize}
% \item Develop new solvers for specific domains
% \item Integration with MIP
% \end{itemize}
% \item Not included in bundled MiniZinc IDE
% \item Specialized visualization tools used here
  % \begin{itemize}
  % \item CP-Viz, Simonis et al. 2010
    % \end{itemize}
% \end{itemize}
% \end{frame}

\begin{frame}
\frametitle{Tutorial Based on ECLiPSe ELearning Course}
\begin{itemize}
\item Self-study course in constraint programming
\item Supported by Cisco Systems and Silicon Valley Community Foundation
\item Multi-media format, video lectures, slides, handout etc
\item {\small \url{https://eclipseclp.org/ELearning/index.html}}
\end{itemize}
\end{frame}

\begin{frame}
\frametitle{Also Part of CRT-AI Constraint Week}
\begin{itemize}
\item Annual one week course on CP and Optimization in Ireland
\item Part of national training program for PhD students in AI
\item \url{https://www.crt-ai.ie/}
\end{itemize}
\end{frame}



\begin{frame}
\frametitle{Constraint Programming - in a nutshell}
\begin{itemize}
\item Declarative description of problems with
\begin{itemize}
\item {\em Variables} which range over (finite) sets of values
\item {\em Constraints} over subsets of variables which restrict possible value combinations
\item A {\em solution} is a value assignment which satisfies all constraints
\end{itemize}

\item Constraint propagation/reasoning
\begin{itemize}
\item Removing inconsistent values for variables
\item Detect failure if constraint can not be satisfied
\item Interaction of constraints via shared variables
\item Incomplete
\end{itemize}

\item Search
\begin{itemize}
\item User controlled assignment of values to variables
\item Each step triggers constraint propagation 
\end{itemize}
\item Different domains require/allow different methods
\end{itemize}
\end{frame}

\begin{frame}
  \frametitle{Constraint Programming is Different}
  \begin{itemize}
  \item Declarative Programming
    \begin{itemize}
    \item Concentrate on what you want
      \item Not how to get there
      \item Program != Algorithm
      \item Program = Model
    \end{itemize}
    \item Applied to Combinatorial Problems
      \begin{itemize}
        \item No complete polynomial algorithms known (exist?)
        \item CP less ad-hoc than heuristics
        \item Models can evolve
  \end{itemize}
  \end{itemize}
  \end{frame}
    
\begin{frame}
  \frametitle{A Subtractive Process}
  \begin{textblock}{4}(8,-3)
    \includegraphics[width=4cm]{../introduction/images/stages}
  \end{textblock}
  \vfill
  \begin{quote}
    ``Oh, bosh, as Mr. Ruskin says. Sculpture, per se, is the simplest thing in the world. All you have to do is to take a big chunk of marble and a hammer and chisel, make up your mind what you are about to create and chip off all the marble you don't want.''-Paris Gaulois.
  \end{quote}
  
  {\tiny Source: \url{https://quoteinvestigator.com/2014/06/22/chip-away/}}
\end{frame}

\begin{frame}
\frametitle{Basic Process}
\begin{center}
\begin{tikzpicture}
\node[shape=rectangle,fill=pantone127-4] (problem) at (2,5) {Problem};
\node (human) at (2,4) {Human};
\node[shape=rectangle,fill=pantone127-4] (model) at (2,3) {Model};
\node (solver) at (2,2) {Constraint Solver/Search};
\node[shape=rectangle,fill=pantone157-8] (solution) at (2,1) {Solution};
\draw[-] (problem) -- (human);
\draw[->] (human) -- (model);
\draw[-] (model) -- (solver);
\draw[->] (solver) -- (solution);
\end{tikzpicture}
\end{center}
\end{frame}

\begin{frame}
\frametitle{More Realistic}
\begin{center}
\begin{tikzpicture}[xscale=4.0]
\node[shape=rectangle,fill=pantone127-4] (problem) at (2,5) {Problem};
\node (human) at (2,4) {Human};
\node[shape=rectangle,fill=pantone127-4] (model) at (2,3) {Model};
\node (solver) at (2,2) {Constraint Solver/Search};
\node[shape=rectangle,fill=pantone24-8] (hangs) at (1,1) {Hangs};
\node[shape=rectangle,fill=pantone157-8] (solution) at (2,1) {Solution};
\node[shape=rectangle,fill=pantone24-8] (wrong) at (3,1) {Wrong Solution};
\draw[-] (problem) -- (human);
\draw[->] (human) -- (model);
\draw[-] (model) -- (solver);
\draw[->] (solver) -- (hangs);
\draw[->] (solver) -- (solution);
\draw[->] (solver) -- (wrong);
\draw[->] (hangs) -- (1,4) -- (human);
\draw[->] (wrong) -- (3,4) -- (human);
\end{tikzpicture}
\end{center}
\end{frame}

\begin{frame}
\frametitle{Dual Role of Model}
\begin{itemize}
\item Allows Human to Express Problem
\begin{itemize}
\item Close to Problem Domain
\item Constraints as Abstractions
\end{itemize}
\item Allows Solver to Execute
\begin{itemize}
\item Variables as Communication Mechanism
\item Constraints as Algorithms
\end{itemize}
\end{itemize}
\end{frame}

\begin{frame}
\frametitle{Modelling Frameworks}
\begin{itemize}
\item \href{https://www.minizinc.org/}{MiniZinc} (NICTA, Monash University, Australia)
\item \href{https://github.com/eomahony/Numberjack}{NumberJack} (Insight, Ireland) 
\item \href{https://www-users.york.ac.uk/peter.nightingale/savilerow//}{EssencePrime/SavilleRow} (UK) 
\item \href{https://cpmpy.readthedocs.io/en/latest/}{CPMpy} (KU Leuven)
\item Allow use of multiple back-end solvers
\item Compile model into variants for each solver
\item A priori solver independent model(CP, MIP, SAT)
\end{itemize}
\end{frame} 

\begin{frame}
\frametitle{Framework Process}
\begin{center}
\begin{tikzpicture}[xscale=2]
\node[shape=rectangle,fill=pantone127-4] (problem) at (2.5,5) {Problem};
\node (human) at (2.5,4) {Human};
\node[shape=rectangle,fill=pantone127-4] (model) at (2.5,3) {Model};
\node (compile) at (2.5,2) {Compile/Reformulate};
\node (solver1) at (1,1) {CP};
\node (solver2) at (2,1) {MIP};
\node (solver3) at (3,1) {SAT};
\node (solver4) at (4,1) {Other};
\node[shape=rectangle,fill=pantone157-8] (solution1) at (1,0) {Solution};
\node[shape=rectangle,fill=pantone157-8] (solution2) at (2,0) {Solution};
\node[shape=rectangle,fill=pantone157-8] (solution3) at (3,0) {Solution};
\node[shape=rectangle,fill=pantone157-8] (solution4) at (4,0) {Solution};
\draw[-] (problem) -- (human);
\draw[->] (human) -- (model);
\draw[-] (model) -- (compile);
\draw[-] (compile) -- (solver1);
\draw[-] (compile) -- (solver2);
\draw[-] (compile) -- (solver3);
\draw[-] (compile) -- (solver4);
\draw[->] (solver1) -- (solution1);
\draw[->] (solver2) -- (solution2);
\draw[->] (solver3) -- (solution3);
\draw[->] (solver4) -- (solution4);
\end{tikzpicture}
\end{center}
\end{frame}

% \begin{frame}
  % \frametitle{Do It Now!}
  % \begin{itemize}
  % \item Download and install Minizinc
  % \item \url{https://www.minizinc.org/}
  % \end{itemize}
% \end{frame}


\begin{frame}
\frametitle{Why use Puzzles as Examples?}
\begin{itemize}
\item Easy to understand the problem
\item Solvable by hand without specialized knowledge
\item Possible to compare automated to manual solving process
\end{itemize}

\begin{quote}
The puzzle, though inanimate, is presented as a solvable problem without lasting negative consequences, a very low-risk low-reward situation. By being a puzzle, the object is attempting to convince the user that it must be completed.
\end{quote}
{\tiny Source: \href{https://www.everydayrhetoric.com/2020/03/the-rhetoric-of-puzzles/}{Every Day Rhetoric}}
\end{frame}


\mode<all>{
\part{Basic Constraint Propagation}
}
\begin{frame}
\partpage
\end{frame}
\mode<all>{
\input{../sendmore/body}
}

\mode<all>{
\part{Global Constraints}
}
\begin{frame}
\partpage
\end{frame}
\mode<all>{
\input{../sudoku/body}
}

\clearpage
\mode<all>{
\part{Customizing Search}
}
\begin{frame}
\partpage
\end{frame}
\mode<all>{
\input{../nqueen/body}
}


%% \clearpage
%% \mode<all>{
%%   \part{Problem Decomposition}
%% }
%% \begin{frame}
%%   \partpage
%% \end{frame}
%% \mode<all>{
%%   \input{../party/body}
%%   }

%% \clearpage
%% \mode<all>{
%%   \part{More Global Constraints}
%% }
%% \begin{frame}
%%   \partpage
%% \end{frame}
%% \mode<all>{
%%   \input{../car/body}
%%   }

%% \clearpage
%% \mode<all>{
%%   \part{Methodology}
%% }
%% \begin{frame}
%%   \partpage
%% \end{frame}
%% \mode<all>{
%%   \input{../methodology/body}
%%   }

\clearpage
\mode<all>{
\part{What is missing?}
}
\begin{frame}
\partpage
\end{frame}

\begin{frame}
\frametitle{Many Specialized Topics}
\begin{itemize}
\item How to design efficient core engine
\item Hybrids with LP/MIP tools
\item Hybrids with SAT
\item Symmetry breaking
\item Use of MDD/BDD to encode sets of solutions
\item High level modelling tools
\item Debugging/visualization
\end{itemize}
\end{frame}

\begin{frame}
\frametitle{Reformulation}
\begin{itemize}
\item Just because the user has modelled it this way, it doesn't mean we have to solve it that way
\begin{itemize}
\item Replace some constraint(s) by other, equivalent constraints
\item Because we don't have that constraint in our system
\item For performance
\end{itemize}
\end{itemize}
\end{frame}

\begin{frame}
\frametitle{Learning}
\begin{itemize}
\item While solving the problem we can learn how to strengthen the model/search 
\begin{itemize}
\item Understand which constraints/method contribute to propagation and change schedule
\item Learn no-good constraints by explaining failure
\item Adapt search strategy based on search experience
\end{itemize}
\end{itemize}
\end{frame}

%% \begin{frame}
%% \frametitle{Refined Process}
%% \begin{center}
%% \begin{tikzpicture}
%% \node[shape=rectangle,fill=pantone127-4] (problem) at (3,7) {Problem};
%% \node (human) at (3,6) {Human};
%% \node[shape=rectangle,fill=pantone127-4] (model) at (3,5) {User Model};
%% \node (reformulation) at (3,4) {Reformulation};
%% \node[shape=rectangle,fill=pantone127-4] (implementation) at (3,3) {Implementation Model};
%% \node (solver) at (3,2) {Constraint Solver/Search};
%% \node[shape=rectangle,fill=pantone157-8] (solution) at (3,1) {Solution};
%% \draw[-] (problem) -- (human);
%% \draw[->] (human) -- (model);
%% \draw[-] (model) -- (reformulation);
%% \draw[->] (reformulation) -- (implementation);
%% \draw[-] (implementation) -- (solver);
%% \draw[->] (solver) -- (0,2) -- (0,3) -- (implementation);
%% \draw[->] (solver) -- (solution);
%% \end{tikzpicture}
%% \end{center}
%% \end{frame}

%\begin{frame}
%\frametitle{What is CP actually used for?}
%\url{http://hsimonis.wordpress.com}
%\includegraphics[width=10cm]{../introduction/images/blog}
%\end{frame}

%% \begin{frame}
%% \frametitle{Exercises}
%% \begin{itemize}
%% \item Install Constraint Solver of your Choice
%% \item MiniZinc (MiningZinc) will be used in later sessions
%% \item Run Sendmoremoney, Sudoku, Queens examples
%% \item Can you choose which propagator to use for alldifferent?
%% \item Can you use different search strategies?
%% \end{itemize}
%% \end{frame}

\begin{frame}
\frametitle{More Learning Resources}
\begin{itemize}
\item Survey of Methods, Resources, and Formats for Teaching Constraint Programming
\begin{itemize}
\item by Tejas Santanam, Helmut Simonis
\item \url{https://doi.org/10.48550/arXiv.2403.12717}
\item Based on survey of community for WTCP 2023
\item \url{https://hsimonis.github.io/WTCP2023/}
\end{itemize}
\end{itemize}
\end{frame}


%%   }

%% \clearpage
%% \mode<all>{
%%   \part{Methodology}
%% }
%% \begin{frame}
%%   \partpage
%% \end{frame}
%% \mode<all>{
%%   \begin{frame}
\frametitle{Objectives}
\begin{itemize}
\item Overview of Core Constraint Programming
\item Three Main Concepts
\begin{itemize}
\item Constraint Propagation
\item Global Constraints
\item Customizing Search
\end{itemize}
%\item Get Some Experience with MiniZinc
\item Topics will be treated in more detail in later parts of the school
\item Based on Examples, not Formal Description
\end{itemize}
\end{frame}

\begin{frame}
\frametitle{Outline}
\begin{itemize}
\item Why Constraint Programming?
\item Constraint Propagation
\item Global Constraints
\item Customizing Search
%\item Choices in Modelling
%\item Decomposition
%\item More Global Constraints
%\item Methodology
%\item What is missing?
\end{itemize}
\end{frame}

% \begin{frame}
  % \frametitle{Using MiniZinc IDE}
  % \begin{itemize}
  % \item Developed in the Australian NICTA project
  % \item Maintained by Monash University
  % \item Modelling tool with multiple back-end solvers
  % \item Available from \url{https://www.minizinc.org/}
  % \end{itemize}
% \end{frame}


% \begin{frame}
% \frametitle{Examples in ECLiPSe}
% \begin{itemize}
% \item Open sourced constraint programming language
% \item Development goes back to 1985
% \item ECRC, ICL, IC-Parc, PTL, Cisco
% \item \url{https://eclipseclp.org/}
% \item Specialities
% \begin{itemize}
% \item Develop new solvers for specific domains
% \item Integration with MIP
% \end{itemize}
% \item Not included in bundled MiniZinc IDE
% \item Specialized visualization tools used here
  % \begin{itemize}
  % \item CP-Viz, Simonis et al. 2010
    % \end{itemize}
% \end{itemize}
% \end{frame}

\begin{frame}
\frametitle{Tutorial Based on ECLiPSe ELearning Course}
\begin{itemize}
\item Self-study course in constraint programming
\item Supported by Cisco Systems and Silicon Valley Community Foundation
\item Multi-media format, video lectures, slides, handout etc
\item {\small \url{https://eclipseclp.org/ELearning/index.html}}
\end{itemize}
\end{frame}

\begin{frame}
\frametitle{Also Part of CRT-AI Constraint Week}
\begin{itemize}
\item Annual one week course on CP and Optimization in Ireland
\item Part of national training program for PhD students in AI
\item \url{https://www.crt-ai.ie/}
\end{itemize}
\end{frame}



\begin{frame}
\frametitle{Constraint Programming - in a nutshell}
\begin{itemize}
\item Declarative description of problems with
\begin{itemize}
\item {\em Variables} which range over (finite) sets of values
\item {\em Constraints} over subsets of variables which restrict possible value combinations
\item A {\em solution} is a value assignment which satisfies all constraints
\end{itemize}

\item Constraint propagation/reasoning
\begin{itemize}
\item Removing inconsistent values for variables
\item Detect failure if constraint can not be satisfied
\item Interaction of constraints via shared variables
\item Incomplete
\end{itemize}

\item Search
\begin{itemize}
\item User controlled assignment of values to variables
\item Each step triggers constraint propagation 
\end{itemize}
\item Different domains require/allow different methods
\end{itemize}
\end{frame}

\begin{frame}
  \frametitle{Constraint Programming is Different}
  \begin{itemize}
  \item Declarative Programming
    \begin{itemize}
    \item Concentrate on what you want
      \item Not how to get there
      \item Program != Algorithm
      \item Program = Model
    \end{itemize}
    \item Applied to Combinatorial Problems
      \begin{itemize}
        \item No complete polynomial algorithms known (exist?)
        \item CP less ad-hoc than heuristics
        \item Models can evolve
  \end{itemize}
  \end{itemize}
  \end{frame}
    
\begin{frame}
  \frametitle{A Subtractive Process}
  \begin{textblock}{4}(8,-3)
    \includegraphics[width=4cm]{../introduction/images/stages}
  \end{textblock}
  \vfill
  \begin{quote}
    ``Oh, bosh, as Mr. Ruskin says. Sculpture, per se, is the simplest thing in the world. All you have to do is to take a big chunk of marble and a hammer and chisel, make up your mind what you are about to create and chip off all the marble you don't want.''-Paris Gaulois.
  \end{quote}
  
  {\tiny Source: \url{https://quoteinvestigator.com/2014/06/22/chip-away/}}
\end{frame}

\begin{frame}
\frametitle{Basic Process}
\begin{center}
\begin{tikzpicture}
\node[shape=rectangle,fill=pantone127-4] (problem) at (2,5) {Problem};
\node (human) at (2,4) {Human};
\node[shape=rectangle,fill=pantone127-4] (model) at (2,3) {Model};
\node (solver) at (2,2) {Constraint Solver/Search};
\node[shape=rectangle,fill=pantone157-8] (solution) at (2,1) {Solution};
\draw[-] (problem) -- (human);
\draw[->] (human) -- (model);
\draw[-] (model) -- (solver);
\draw[->] (solver) -- (solution);
\end{tikzpicture}
\end{center}
\end{frame}

\begin{frame}
\frametitle{More Realistic}
\begin{center}
\begin{tikzpicture}[xscale=4.0]
\node[shape=rectangle,fill=pantone127-4] (problem) at (2,5) {Problem};
\node (human) at (2,4) {Human};
\node[shape=rectangle,fill=pantone127-4] (model) at (2,3) {Model};
\node (solver) at (2,2) {Constraint Solver/Search};
\node[shape=rectangle,fill=pantone24-8] (hangs) at (1,1) {Hangs};
\node[shape=rectangle,fill=pantone157-8] (solution) at (2,1) {Solution};
\node[shape=rectangle,fill=pantone24-8] (wrong) at (3,1) {Wrong Solution};
\draw[-] (problem) -- (human);
\draw[->] (human) -- (model);
\draw[-] (model) -- (solver);
\draw[->] (solver) -- (hangs);
\draw[->] (solver) -- (solution);
\draw[->] (solver) -- (wrong);
\draw[->] (hangs) -- (1,4) -- (human);
\draw[->] (wrong) -- (3,4) -- (human);
\end{tikzpicture}
\end{center}
\end{frame}

\begin{frame}
\frametitle{Dual Role of Model}
\begin{itemize}
\item Allows Human to Express Problem
\begin{itemize}
\item Close to Problem Domain
\item Constraints as Abstractions
\end{itemize}
\item Allows Solver to Execute
\begin{itemize}
\item Variables as Communication Mechanism
\item Constraints as Algorithms
\end{itemize}
\end{itemize}
\end{frame}

\begin{frame}
\frametitle{Modelling Frameworks}
\begin{itemize}
\item \href{https://www.minizinc.org/}{MiniZinc} (NICTA, Monash University, Australia)
\item \href{https://github.com/eomahony/Numberjack}{NumberJack} (Insight, Ireland) 
\item \href{https://www-users.york.ac.uk/peter.nightingale/savilerow//}{EssencePrime/SavilleRow} (UK) 
\item \href{https://cpmpy.readthedocs.io/en/latest/}{CPMpy} (KU Leuven)
\item Allow use of multiple back-end solvers
\item Compile model into variants for each solver
\item A priori solver independent model(CP, MIP, SAT)
\end{itemize}
\end{frame} 

\begin{frame}
\frametitle{Framework Process}
\begin{center}
\begin{tikzpicture}[xscale=2]
\node[shape=rectangle,fill=pantone127-4] (problem) at (2.5,5) {Problem};
\node (human) at (2.5,4) {Human};
\node[shape=rectangle,fill=pantone127-4] (model) at (2.5,3) {Model};
\node (compile) at (2.5,2) {Compile/Reformulate};
\node (solver1) at (1,1) {CP};
\node (solver2) at (2,1) {MIP};
\node (solver3) at (3,1) {SAT};
\node (solver4) at (4,1) {Other};
\node[shape=rectangle,fill=pantone157-8] (solution1) at (1,0) {Solution};
\node[shape=rectangle,fill=pantone157-8] (solution2) at (2,0) {Solution};
\node[shape=rectangle,fill=pantone157-8] (solution3) at (3,0) {Solution};
\node[shape=rectangle,fill=pantone157-8] (solution4) at (4,0) {Solution};
\draw[-] (problem) -- (human);
\draw[->] (human) -- (model);
\draw[-] (model) -- (compile);
\draw[-] (compile) -- (solver1);
\draw[-] (compile) -- (solver2);
\draw[-] (compile) -- (solver3);
\draw[-] (compile) -- (solver4);
\draw[->] (solver1) -- (solution1);
\draw[->] (solver2) -- (solution2);
\draw[->] (solver3) -- (solution3);
\draw[->] (solver4) -- (solution4);
\end{tikzpicture}
\end{center}
\end{frame}

% \begin{frame}
  % \frametitle{Do It Now!}
  % \begin{itemize}
  % \item Download and install Minizinc
  % \item \url{https://www.minizinc.org/}
  % \end{itemize}
% \end{frame}


\begin{frame}
\frametitle{Why use Puzzles as Examples?}
\begin{itemize}
\item Easy to understand the problem
\item Solvable by hand without specialized knowledge
\item Possible to compare automated to manual solving process
\end{itemize}

\begin{quote}
The puzzle, though inanimate, is presented as a solvable problem without lasting negative consequences, a very low-risk low-reward situation. By being a puzzle, the object is attempting to convince the user that it must be completed.
\end{quote}
{\tiny Source: \href{https://www.everydayrhetoric.com/2020/03/the-rhetoric-of-puzzles/}{Every Day Rhetoric}}
\end{frame}


\mode<all>{
\part{Basic Constraint Propagation}
}
\begin{frame}
\partpage
\end{frame}
\mode<all>{
\input{../sendmore/body}
}

\mode<all>{
\part{Global Constraints}
}
\begin{frame}
\partpage
\end{frame}
\mode<all>{
\input{../sudoku/body}
}

\clearpage
\mode<all>{
\part{Customizing Search}
}
\begin{frame}
\partpage
\end{frame}
\mode<all>{
\input{../nqueen/body}
}


%% \clearpage
%% \mode<all>{
%%   \part{Problem Decomposition}
%% }
%% \begin{frame}
%%   \partpage
%% \end{frame}
%% \mode<all>{
%%   \input{../party/body}
%%   }

%% \clearpage
%% \mode<all>{
%%   \part{More Global Constraints}
%% }
%% \begin{frame}
%%   \partpage
%% \end{frame}
%% \mode<all>{
%%   \input{../car/body}
%%   }

%% \clearpage
%% \mode<all>{
%%   \part{Methodology}
%% }
%% \begin{frame}
%%   \partpage
%% \end{frame}
%% \mode<all>{
%%   \input{../methodology/body}
%%   }

\clearpage
\mode<all>{
\part{What is missing?}
}
\begin{frame}
\partpage
\end{frame}

\begin{frame}
\frametitle{Many Specialized Topics}
\begin{itemize}
\item How to design efficient core engine
\item Hybrids with LP/MIP tools
\item Hybrids with SAT
\item Symmetry breaking
\item Use of MDD/BDD to encode sets of solutions
\item High level modelling tools
\item Debugging/visualization
\end{itemize}
\end{frame}

\begin{frame}
\frametitle{Reformulation}
\begin{itemize}
\item Just because the user has modelled it this way, it doesn't mean we have to solve it that way
\begin{itemize}
\item Replace some constraint(s) by other, equivalent constraints
\item Because we don't have that constraint in our system
\item For performance
\end{itemize}
\end{itemize}
\end{frame}

\begin{frame}
\frametitle{Learning}
\begin{itemize}
\item While solving the problem we can learn how to strengthen the model/search 
\begin{itemize}
\item Understand which constraints/method contribute to propagation and change schedule
\item Learn no-good constraints by explaining failure
\item Adapt search strategy based on search experience
\end{itemize}
\end{itemize}
\end{frame}

%% \begin{frame}
%% \frametitle{Refined Process}
%% \begin{center}
%% \begin{tikzpicture}
%% \node[shape=rectangle,fill=pantone127-4] (problem) at (3,7) {Problem};
%% \node (human) at (3,6) {Human};
%% \node[shape=rectangle,fill=pantone127-4] (model) at (3,5) {User Model};
%% \node (reformulation) at (3,4) {Reformulation};
%% \node[shape=rectangle,fill=pantone127-4] (implementation) at (3,3) {Implementation Model};
%% \node (solver) at (3,2) {Constraint Solver/Search};
%% \node[shape=rectangle,fill=pantone157-8] (solution) at (3,1) {Solution};
%% \draw[-] (problem) -- (human);
%% \draw[->] (human) -- (model);
%% \draw[-] (model) -- (reformulation);
%% \draw[->] (reformulation) -- (implementation);
%% \draw[-] (implementation) -- (solver);
%% \draw[->] (solver) -- (0,2) -- (0,3) -- (implementation);
%% \draw[->] (solver) -- (solution);
%% \end{tikzpicture}
%% \end{center}
%% \end{frame}

%\begin{frame}
%\frametitle{What is CP actually used for?}
%\url{http://hsimonis.wordpress.com}
%\includegraphics[width=10cm]{../introduction/images/blog}
%\end{frame}

%% \begin{frame}
%% \frametitle{Exercises}
%% \begin{itemize}
%% \item Install Constraint Solver of your Choice
%% \item MiniZinc (MiningZinc) will be used in later sessions
%% \item Run Sendmoremoney, Sudoku, Queens examples
%% \item Can you choose which propagator to use for alldifferent?
%% \item Can you use different search strategies?
%% \end{itemize}
%% \end{frame}

\begin{frame}
\frametitle{More Learning Resources}
\begin{itemize}
\item Survey of Methods, Resources, and Formats for Teaching Constraint Programming
\begin{itemize}
\item by Tejas Santanam, Helmut Simonis
\item \url{https://doi.org/10.48550/arXiv.2403.12717}
\item Based on survey of community for WTCP 2023
\item \url{https://hsimonis.github.io/WTCP2023/}
\end{itemize}
\end{itemize}
\end{frame}


%%   }

\clearpage
\mode<all>{
\part{What is missing?}
}
\begin{frame}
\partpage
\end{frame}

\begin{frame}
\frametitle{Many Specialized Topics}
\begin{itemize}
\item How to design efficient core engine
\item Hybrids with LP/MIP tools
\item Hybrids with SAT
\item Symmetry breaking
\item Use of MDD/BDD to encode sets of solutions
\item High level modelling tools
\item Debugging/visualization
\end{itemize}
\end{frame}

\begin{frame}
\frametitle{Reformulation}
\begin{itemize}
\item Just because the user has modelled it this way, it doesn't mean we have to solve it that way
\begin{itemize}
\item Replace some constraint(s) by other, equivalent constraints
\item Because we don't have that constraint in our system
\item For performance
\end{itemize}
\end{itemize}
\end{frame}

\begin{frame}
\frametitle{Learning}
\begin{itemize}
\item While solving the problem we can learn how to strengthen the model/search 
\begin{itemize}
\item Understand which constraints/method contribute to propagation and change schedule
\item Learn no-good constraints by explaining failure
\item Adapt search strategy based on search experience
\end{itemize}
\end{itemize}
\end{frame}

%% \begin{frame}
%% \frametitle{Refined Process}
%% \begin{center}
%% \begin{tikzpicture}
%% \node[shape=rectangle,fill=pantone127-4] (problem) at (3,7) {Problem};
%% \node (human) at (3,6) {Human};
%% \node[shape=rectangle,fill=pantone127-4] (model) at (3,5) {User Model};
%% \node (reformulation) at (3,4) {Reformulation};
%% \node[shape=rectangle,fill=pantone127-4] (implementation) at (3,3) {Implementation Model};
%% \node (solver) at (3,2) {Constraint Solver/Search};
%% \node[shape=rectangle,fill=pantone157-8] (solution) at (3,1) {Solution};
%% \draw[-] (problem) -- (human);
%% \draw[->] (human) -- (model);
%% \draw[-] (model) -- (reformulation);
%% \draw[->] (reformulation) -- (implementation);
%% \draw[-] (implementation) -- (solver);
%% \draw[->] (solver) -- (0,2) -- (0,3) -- (implementation);
%% \draw[->] (solver) -- (solution);
%% \end{tikzpicture}
%% \end{center}
%% \end{frame}

%\begin{frame}
%\frametitle{What is CP actually used for?}
%\url{http://hsimonis.wordpress.com}
%\includegraphics[width=10cm]{../introduction/images/blog}
%\end{frame}

%% \begin{frame}
%% \frametitle{Exercises}
%% \begin{itemize}
%% \item Install Constraint Solver of your Choice
%% \item MiniZinc (MiningZinc) will be used in later sessions
%% \item Run Sendmoremoney, Sudoku, Queens examples
%% \item Can you choose which propagator to use for alldifferent?
%% \item Can you use different search strategies?
%% \end{itemize}
%% \end{frame}

\begin{frame}
\frametitle{More Learning Resources}
\begin{itemize}
\item Survey of Methods, Resources, and Formats for Teaching Constraint Programming
\begin{itemize}
\item by Tejas Santanam, Helmut Simonis
\item \url{https://doi.org/10.48550/arXiv.2403.12717}
\item Based on survey of community for WTCP 2023
\item \url{https://hsimonis.github.io/WTCP2023/}
\end{itemize}
\end{itemize}
\end{frame}


}

\mode<all>{
\part{Global Constraints}
}
\begin{frame}
\partpage
\end{frame}
\mode<all>{
\begin{frame}
\frametitle{Objectives}
\begin{itemize}
\item Overview of Core Constraint Programming
\item Three Main Concepts
\begin{itemize}
\item Constraint Propagation
\item Global Constraints
\item Customizing Search
\end{itemize}
%\item Get Some Experience with MiniZinc
\item Topics will be treated in more detail in later parts of the school
\item Based on Examples, not Formal Description
\end{itemize}
\end{frame}

\begin{frame}
\frametitle{Outline}
\begin{itemize}
\item Why Constraint Programming?
\item Constraint Propagation
\item Global Constraints
\item Customizing Search
%\item Choices in Modelling
%\item Decomposition
%\item More Global Constraints
%\item Methodology
%\item What is missing?
\end{itemize}
\end{frame}

% \begin{frame}
  % \frametitle{Using MiniZinc IDE}
  % \begin{itemize}
  % \item Developed in the Australian NICTA project
  % \item Maintained by Monash University
  % \item Modelling tool with multiple back-end solvers
  % \item Available from \url{https://www.minizinc.org/}
  % \end{itemize}
% \end{frame}


% \begin{frame}
% \frametitle{Examples in ECLiPSe}
% \begin{itemize}
% \item Open sourced constraint programming language
% \item Development goes back to 1985
% \item ECRC, ICL, IC-Parc, PTL, Cisco
% \item \url{https://eclipseclp.org/}
% \item Specialities
% \begin{itemize}
% \item Develop new solvers for specific domains
% \item Integration with MIP
% \end{itemize}
% \item Not included in bundled MiniZinc IDE
% \item Specialized visualization tools used here
  % \begin{itemize}
  % \item CP-Viz, Simonis et al. 2010
    % \end{itemize}
% \end{itemize}
% \end{frame}

\begin{frame}
\frametitle{Tutorial Based on ECLiPSe ELearning Course}
\begin{itemize}
\item Self-study course in constraint programming
\item Supported by Cisco Systems and Silicon Valley Community Foundation
\item Multi-media format, video lectures, slides, handout etc
\item {\small \url{https://eclipseclp.org/ELearning/index.html}}
\end{itemize}
\end{frame}

\begin{frame}
\frametitle{Also Part of CRT-AI Constraint Week}
\begin{itemize}
\item Annual one week course on CP and Optimization in Ireland
\item Part of national training program for PhD students in AI
\item \url{https://www.crt-ai.ie/}
\end{itemize}
\end{frame}



\begin{frame}
\frametitle{Constraint Programming - in a nutshell}
\begin{itemize}
\item Declarative description of problems with
\begin{itemize}
\item {\em Variables} which range over (finite) sets of values
\item {\em Constraints} over subsets of variables which restrict possible value combinations
\item A {\em solution} is a value assignment which satisfies all constraints
\end{itemize}

\item Constraint propagation/reasoning
\begin{itemize}
\item Removing inconsistent values for variables
\item Detect failure if constraint can not be satisfied
\item Interaction of constraints via shared variables
\item Incomplete
\end{itemize}

\item Search
\begin{itemize}
\item User controlled assignment of values to variables
\item Each step triggers constraint propagation 
\end{itemize}
\item Different domains require/allow different methods
\end{itemize}
\end{frame}

\begin{frame}
  \frametitle{Constraint Programming is Different}
  \begin{itemize}
  \item Declarative Programming
    \begin{itemize}
    \item Concentrate on what you want
      \item Not how to get there
      \item Program != Algorithm
      \item Program = Model
    \end{itemize}
    \item Applied to Combinatorial Problems
      \begin{itemize}
        \item No complete polynomial algorithms known (exist?)
        \item CP less ad-hoc than heuristics
        \item Models can evolve
  \end{itemize}
  \end{itemize}
  \end{frame}
    
\begin{frame}
  \frametitle{A Subtractive Process}
  \begin{textblock}{4}(8,-3)
    \includegraphics[width=4cm]{../introduction/images/stages}
  \end{textblock}
  \vfill
  \begin{quote}
    ``Oh, bosh, as Mr. Ruskin says. Sculpture, per se, is the simplest thing in the world. All you have to do is to take a big chunk of marble and a hammer and chisel, make up your mind what you are about to create and chip off all the marble you don't want.''-Paris Gaulois.
  \end{quote}
  
  {\tiny Source: \url{https://quoteinvestigator.com/2014/06/22/chip-away/}}
\end{frame}

\begin{frame}
\frametitle{Basic Process}
\begin{center}
\begin{tikzpicture}
\node[shape=rectangle,fill=pantone127-4] (problem) at (2,5) {Problem};
\node (human) at (2,4) {Human};
\node[shape=rectangle,fill=pantone127-4] (model) at (2,3) {Model};
\node (solver) at (2,2) {Constraint Solver/Search};
\node[shape=rectangle,fill=pantone157-8] (solution) at (2,1) {Solution};
\draw[-] (problem) -- (human);
\draw[->] (human) -- (model);
\draw[-] (model) -- (solver);
\draw[->] (solver) -- (solution);
\end{tikzpicture}
\end{center}
\end{frame}

\begin{frame}
\frametitle{More Realistic}
\begin{center}
\begin{tikzpicture}[xscale=4.0]
\node[shape=rectangle,fill=pantone127-4] (problem) at (2,5) {Problem};
\node (human) at (2,4) {Human};
\node[shape=rectangle,fill=pantone127-4] (model) at (2,3) {Model};
\node (solver) at (2,2) {Constraint Solver/Search};
\node[shape=rectangle,fill=pantone24-8] (hangs) at (1,1) {Hangs};
\node[shape=rectangle,fill=pantone157-8] (solution) at (2,1) {Solution};
\node[shape=rectangle,fill=pantone24-8] (wrong) at (3,1) {Wrong Solution};
\draw[-] (problem) -- (human);
\draw[->] (human) -- (model);
\draw[-] (model) -- (solver);
\draw[->] (solver) -- (hangs);
\draw[->] (solver) -- (solution);
\draw[->] (solver) -- (wrong);
\draw[->] (hangs) -- (1,4) -- (human);
\draw[->] (wrong) -- (3,4) -- (human);
\end{tikzpicture}
\end{center}
\end{frame}

\begin{frame}
\frametitle{Dual Role of Model}
\begin{itemize}
\item Allows Human to Express Problem
\begin{itemize}
\item Close to Problem Domain
\item Constraints as Abstractions
\end{itemize}
\item Allows Solver to Execute
\begin{itemize}
\item Variables as Communication Mechanism
\item Constraints as Algorithms
\end{itemize}
\end{itemize}
\end{frame}

\begin{frame}
\frametitle{Modelling Frameworks}
\begin{itemize}
\item \href{https://www.minizinc.org/}{MiniZinc} (NICTA, Monash University, Australia)
\item \href{https://github.com/eomahony/Numberjack}{NumberJack} (Insight, Ireland) 
\item \href{https://www-users.york.ac.uk/peter.nightingale/savilerow//}{EssencePrime/SavilleRow} (UK) 
\item \href{https://cpmpy.readthedocs.io/en/latest/}{CPMpy} (KU Leuven)
\item Allow use of multiple back-end solvers
\item Compile model into variants for each solver
\item A priori solver independent model(CP, MIP, SAT)
\end{itemize}
\end{frame} 

\begin{frame}
\frametitle{Framework Process}
\begin{center}
\begin{tikzpicture}[xscale=2]
\node[shape=rectangle,fill=pantone127-4] (problem) at (2.5,5) {Problem};
\node (human) at (2.5,4) {Human};
\node[shape=rectangle,fill=pantone127-4] (model) at (2.5,3) {Model};
\node (compile) at (2.5,2) {Compile/Reformulate};
\node (solver1) at (1,1) {CP};
\node (solver2) at (2,1) {MIP};
\node (solver3) at (3,1) {SAT};
\node (solver4) at (4,1) {Other};
\node[shape=rectangle,fill=pantone157-8] (solution1) at (1,0) {Solution};
\node[shape=rectangle,fill=pantone157-8] (solution2) at (2,0) {Solution};
\node[shape=rectangle,fill=pantone157-8] (solution3) at (3,0) {Solution};
\node[shape=rectangle,fill=pantone157-8] (solution4) at (4,0) {Solution};
\draw[-] (problem) -- (human);
\draw[->] (human) -- (model);
\draw[-] (model) -- (compile);
\draw[-] (compile) -- (solver1);
\draw[-] (compile) -- (solver2);
\draw[-] (compile) -- (solver3);
\draw[-] (compile) -- (solver4);
\draw[->] (solver1) -- (solution1);
\draw[->] (solver2) -- (solution2);
\draw[->] (solver3) -- (solution3);
\draw[->] (solver4) -- (solution4);
\end{tikzpicture}
\end{center}
\end{frame}

% \begin{frame}
  % \frametitle{Do It Now!}
  % \begin{itemize}
  % \item Download and install Minizinc
  % \item \url{https://www.minizinc.org/}
  % \end{itemize}
% \end{frame}


\begin{frame}
\frametitle{Why use Puzzles as Examples?}
\begin{itemize}
\item Easy to understand the problem
\item Solvable by hand without specialized knowledge
\item Possible to compare automated to manual solving process
\end{itemize}

\begin{quote}
The puzzle, though inanimate, is presented as a solvable problem without lasting negative consequences, a very low-risk low-reward situation. By being a puzzle, the object is attempting to convince the user that it must be completed.
\end{quote}
{\tiny Source: \href{https://www.everydayrhetoric.com/2020/03/the-rhetoric-of-puzzles/}{Every Day Rhetoric}}
\end{frame}


\mode<all>{
\part{Basic Constraint Propagation}
}
\begin{frame}
\partpage
\end{frame}
\mode<all>{
\begin{frame}
\frametitle{Objectives}
\begin{itemize}
\item Overview of Core Constraint Programming
\item Three Main Concepts
\begin{itemize}
\item Constraint Propagation
\item Global Constraints
\item Customizing Search
\end{itemize}
%\item Get Some Experience with MiniZinc
\item Topics will be treated in more detail in later parts of the school
\item Based on Examples, not Formal Description
\end{itemize}
\end{frame}

\begin{frame}
\frametitle{Outline}
\begin{itemize}
\item Why Constraint Programming?
\item Constraint Propagation
\item Global Constraints
\item Customizing Search
%\item Choices in Modelling
%\item Decomposition
%\item More Global Constraints
%\item Methodology
%\item What is missing?
\end{itemize}
\end{frame}

% \begin{frame}
  % \frametitle{Using MiniZinc IDE}
  % \begin{itemize}
  % \item Developed in the Australian NICTA project
  % \item Maintained by Monash University
  % \item Modelling tool with multiple back-end solvers
  % \item Available from \url{https://www.minizinc.org/}
  % \end{itemize}
% \end{frame}


% \begin{frame}
% \frametitle{Examples in ECLiPSe}
% \begin{itemize}
% \item Open sourced constraint programming language
% \item Development goes back to 1985
% \item ECRC, ICL, IC-Parc, PTL, Cisco
% \item \url{https://eclipseclp.org/}
% \item Specialities
% \begin{itemize}
% \item Develop new solvers for specific domains
% \item Integration with MIP
% \end{itemize}
% \item Not included in bundled MiniZinc IDE
% \item Specialized visualization tools used here
  % \begin{itemize}
  % \item CP-Viz, Simonis et al. 2010
    % \end{itemize}
% \end{itemize}
% \end{frame}

\begin{frame}
\frametitle{Tutorial Based on ECLiPSe ELearning Course}
\begin{itemize}
\item Self-study course in constraint programming
\item Supported by Cisco Systems and Silicon Valley Community Foundation
\item Multi-media format, video lectures, slides, handout etc
\item {\small \url{https://eclipseclp.org/ELearning/index.html}}
\end{itemize}
\end{frame}

\begin{frame}
\frametitle{Also Part of CRT-AI Constraint Week}
\begin{itemize}
\item Annual one week course on CP and Optimization in Ireland
\item Part of national training program for PhD students in AI
\item \url{https://www.crt-ai.ie/}
\end{itemize}
\end{frame}



\begin{frame}
\frametitle{Constraint Programming - in a nutshell}
\begin{itemize}
\item Declarative description of problems with
\begin{itemize}
\item {\em Variables} which range over (finite) sets of values
\item {\em Constraints} over subsets of variables which restrict possible value combinations
\item A {\em solution} is a value assignment which satisfies all constraints
\end{itemize}

\item Constraint propagation/reasoning
\begin{itemize}
\item Removing inconsistent values for variables
\item Detect failure if constraint can not be satisfied
\item Interaction of constraints via shared variables
\item Incomplete
\end{itemize}

\item Search
\begin{itemize}
\item User controlled assignment of values to variables
\item Each step triggers constraint propagation 
\end{itemize}
\item Different domains require/allow different methods
\end{itemize}
\end{frame}

\begin{frame}
  \frametitle{Constraint Programming is Different}
  \begin{itemize}
  \item Declarative Programming
    \begin{itemize}
    \item Concentrate on what you want
      \item Not how to get there
      \item Program != Algorithm
      \item Program = Model
    \end{itemize}
    \item Applied to Combinatorial Problems
      \begin{itemize}
        \item No complete polynomial algorithms known (exist?)
        \item CP less ad-hoc than heuristics
        \item Models can evolve
  \end{itemize}
  \end{itemize}
  \end{frame}
    
\begin{frame}
  \frametitle{A Subtractive Process}
  \begin{textblock}{4}(8,-3)
    \includegraphics[width=4cm]{../introduction/images/stages}
  \end{textblock}
  \vfill
  \begin{quote}
    ``Oh, bosh, as Mr. Ruskin says. Sculpture, per se, is the simplest thing in the world. All you have to do is to take a big chunk of marble and a hammer and chisel, make up your mind what you are about to create and chip off all the marble you don't want.''-Paris Gaulois.
  \end{quote}
  
  {\tiny Source: \url{https://quoteinvestigator.com/2014/06/22/chip-away/}}
\end{frame}

\begin{frame}
\frametitle{Basic Process}
\begin{center}
\begin{tikzpicture}
\node[shape=rectangle,fill=pantone127-4] (problem) at (2,5) {Problem};
\node (human) at (2,4) {Human};
\node[shape=rectangle,fill=pantone127-4] (model) at (2,3) {Model};
\node (solver) at (2,2) {Constraint Solver/Search};
\node[shape=rectangle,fill=pantone157-8] (solution) at (2,1) {Solution};
\draw[-] (problem) -- (human);
\draw[->] (human) -- (model);
\draw[-] (model) -- (solver);
\draw[->] (solver) -- (solution);
\end{tikzpicture}
\end{center}
\end{frame}

\begin{frame}
\frametitle{More Realistic}
\begin{center}
\begin{tikzpicture}[xscale=4.0]
\node[shape=rectangle,fill=pantone127-4] (problem) at (2,5) {Problem};
\node (human) at (2,4) {Human};
\node[shape=rectangle,fill=pantone127-4] (model) at (2,3) {Model};
\node (solver) at (2,2) {Constraint Solver/Search};
\node[shape=rectangle,fill=pantone24-8] (hangs) at (1,1) {Hangs};
\node[shape=rectangle,fill=pantone157-8] (solution) at (2,1) {Solution};
\node[shape=rectangle,fill=pantone24-8] (wrong) at (3,1) {Wrong Solution};
\draw[-] (problem) -- (human);
\draw[->] (human) -- (model);
\draw[-] (model) -- (solver);
\draw[->] (solver) -- (hangs);
\draw[->] (solver) -- (solution);
\draw[->] (solver) -- (wrong);
\draw[->] (hangs) -- (1,4) -- (human);
\draw[->] (wrong) -- (3,4) -- (human);
\end{tikzpicture}
\end{center}
\end{frame}

\begin{frame}
\frametitle{Dual Role of Model}
\begin{itemize}
\item Allows Human to Express Problem
\begin{itemize}
\item Close to Problem Domain
\item Constraints as Abstractions
\end{itemize}
\item Allows Solver to Execute
\begin{itemize}
\item Variables as Communication Mechanism
\item Constraints as Algorithms
\end{itemize}
\end{itemize}
\end{frame}

\begin{frame}
\frametitle{Modelling Frameworks}
\begin{itemize}
\item \href{https://www.minizinc.org/}{MiniZinc} (NICTA, Monash University, Australia)
\item \href{https://github.com/eomahony/Numberjack}{NumberJack} (Insight, Ireland) 
\item \href{https://www-users.york.ac.uk/peter.nightingale/savilerow//}{EssencePrime/SavilleRow} (UK) 
\item \href{https://cpmpy.readthedocs.io/en/latest/}{CPMpy} (KU Leuven)
\item Allow use of multiple back-end solvers
\item Compile model into variants for each solver
\item A priori solver independent model(CP, MIP, SAT)
\end{itemize}
\end{frame} 

\begin{frame}
\frametitle{Framework Process}
\begin{center}
\begin{tikzpicture}[xscale=2]
\node[shape=rectangle,fill=pantone127-4] (problem) at (2.5,5) {Problem};
\node (human) at (2.5,4) {Human};
\node[shape=rectangle,fill=pantone127-4] (model) at (2.5,3) {Model};
\node (compile) at (2.5,2) {Compile/Reformulate};
\node (solver1) at (1,1) {CP};
\node (solver2) at (2,1) {MIP};
\node (solver3) at (3,1) {SAT};
\node (solver4) at (4,1) {Other};
\node[shape=rectangle,fill=pantone157-8] (solution1) at (1,0) {Solution};
\node[shape=rectangle,fill=pantone157-8] (solution2) at (2,0) {Solution};
\node[shape=rectangle,fill=pantone157-8] (solution3) at (3,0) {Solution};
\node[shape=rectangle,fill=pantone157-8] (solution4) at (4,0) {Solution};
\draw[-] (problem) -- (human);
\draw[->] (human) -- (model);
\draw[-] (model) -- (compile);
\draw[-] (compile) -- (solver1);
\draw[-] (compile) -- (solver2);
\draw[-] (compile) -- (solver3);
\draw[-] (compile) -- (solver4);
\draw[->] (solver1) -- (solution1);
\draw[->] (solver2) -- (solution2);
\draw[->] (solver3) -- (solution3);
\draw[->] (solver4) -- (solution4);
\end{tikzpicture}
\end{center}
\end{frame}

% \begin{frame}
  % \frametitle{Do It Now!}
  % \begin{itemize}
  % \item Download and install Minizinc
  % \item \url{https://www.minizinc.org/}
  % \end{itemize}
% \end{frame}


\begin{frame}
\frametitle{Why use Puzzles as Examples?}
\begin{itemize}
\item Easy to understand the problem
\item Solvable by hand without specialized knowledge
\item Possible to compare automated to manual solving process
\end{itemize}

\begin{quote}
The puzzle, though inanimate, is presented as a solvable problem without lasting negative consequences, a very low-risk low-reward situation. By being a puzzle, the object is attempting to convince the user that it must be completed.
\end{quote}
{\tiny Source: \href{https://www.everydayrhetoric.com/2020/03/the-rhetoric-of-puzzles/}{Every Day Rhetoric}}
\end{frame}


\mode<all>{
\part{Basic Constraint Propagation}
}
\begin{frame}
\partpage
\end{frame}
\mode<all>{
\input{../sendmore/body}
}

\mode<all>{
\part{Global Constraints}
}
\begin{frame}
\partpage
\end{frame}
\mode<all>{
\input{../sudoku/body}
}

\clearpage
\mode<all>{
\part{Customizing Search}
}
\begin{frame}
\partpage
\end{frame}
\mode<all>{
\input{../nqueen/body}
}


%% \clearpage
%% \mode<all>{
%%   \part{Problem Decomposition}
%% }
%% \begin{frame}
%%   \partpage
%% \end{frame}
%% \mode<all>{
%%   \input{../party/body}
%%   }

%% \clearpage
%% \mode<all>{
%%   \part{More Global Constraints}
%% }
%% \begin{frame}
%%   \partpage
%% \end{frame}
%% \mode<all>{
%%   \input{../car/body}
%%   }

%% \clearpage
%% \mode<all>{
%%   \part{Methodology}
%% }
%% \begin{frame}
%%   \partpage
%% \end{frame}
%% \mode<all>{
%%   \input{../methodology/body}
%%   }

\clearpage
\mode<all>{
\part{What is missing?}
}
\begin{frame}
\partpage
\end{frame}

\begin{frame}
\frametitle{Many Specialized Topics}
\begin{itemize}
\item How to design efficient core engine
\item Hybrids with LP/MIP tools
\item Hybrids with SAT
\item Symmetry breaking
\item Use of MDD/BDD to encode sets of solutions
\item High level modelling tools
\item Debugging/visualization
\end{itemize}
\end{frame}

\begin{frame}
\frametitle{Reformulation}
\begin{itemize}
\item Just because the user has modelled it this way, it doesn't mean we have to solve it that way
\begin{itemize}
\item Replace some constraint(s) by other, equivalent constraints
\item Because we don't have that constraint in our system
\item For performance
\end{itemize}
\end{itemize}
\end{frame}

\begin{frame}
\frametitle{Learning}
\begin{itemize}
\item While solving the problem we can learn how to strengthen the model/search 
\begin{itemize}
\item Understand which constraints/method contribute to propagation and change schedule
\item Learn no-good constraints by explaining failure
\item Adapt search strategy based on search experience
\end{itemize}
\end{itemize}
\end{frame}

%% \begin{frame}
%% \frametitle{Refined Process}
%% \begin{center}
%% \begin{tikzpicture}
%% \node[shape=rectangle,fill=pantone127-4] (problem) at (3,7) {Problem};
%% \node (human) at (3,6) {Human};
%% \node[shape=rectangle,fill=pantone127-4] (model) at (3,5) {User Model};
%% \node (reformulation) at (3,4) {Reformulation};
%% \node[shape=rectangle,fill=pantone127-4] (implementation) at (3,3) {Implementation Model};
%% \node (solver) at (3,2) {Constraint Solver/Search};
%% \node[shape=rectangle,fill=pantone157-8] (solution) at (3,1) {Solution};
%% \draw[-] (problem) -- (human);
%% \draw[->] (human) -- (model);
%% \draw[-] (model) -- (reformulation);
%% \draw[->] (reformulation) -- (implementation);
%% \draw[-] (implementation) -- (solver);
%% \draw[->] (solver) -- (0,2) -- (0,3) -- (implementation);
%% \draw[->] (solver) -- (solution);
%% \end{tikzpicture}
%% \end{center}
%% \end{frame}

%\begin{frame}
%\frametitle{What is CP actually used for?}
%\url{http://hsimonis.wordpress.com}
%\includegraphics[width=10cm]{../introduction/images/blog}
%\end{frame}

%% \begin{frame}
%% \frametitle{Exercises}
%% \begin{itemize}
%% \item Install Constraint Solver of your Choice
%% \item MiniZinc (MiningZinc) will be used in later sessions
%% \item Run Sendmoremoney, Sudoku, Queens examples
%% \item Can you choose which propagator to use for alldifferent?
%% \item Can you use different search strategies?
%% \end{itemize}
%% \end{frame}

\begin{frame}
\frametitle{More Learning Resources}
\begin{itemize}
\item Survey of Methods, Resources, and Formats for Teaching Constraint Programming
\begin{itemize}
\item by Tejas Santanam, Helmut Simonis
\item \url{https://doi.org/10.48550/arXiv.2403.12717}
\item Based on survey of community for WTCP 2023
\item \url{https://hsimonis.github.io/WTCP2023/}
\end{itemize}
\end{itemize}
\end{frame}


}

\mode<all>{
\part{Global Constraints}
}
\begin{frame}
\partpage
\end{frame}
\mode<all>{
\begin{frame}
\frametitle{Objectives}
\begin{itemize}
\item Overview of Core Constraint Programming
\item Three Main Concepts
\begin{itemize}
\item Constraint Propagation
\item Global Constraints
\item Customizing Search
\end{itemize}
%\item Get Some Experience with MiniZinc
\item Topics will be treated in more detail in later parts of the school
\item Based on Examples, not Formal Description
\end{itemize}
\end{frame}

\begin{frame}
\frametitle{Outline}
\begin{itemize}
\item Why Constraint Programming?
\item Constraint Propagation
\item Global Constraints
\item Customizing Search
%\item Choices in Modelling
%\item Decomposition
%\item More Global Constraints
%\item Methodology
%\item What is missing?
\end{itemize}
\end{frame}

% \begin{frame}
  % \frametitle{Using MiniZinc IDE}
  % \begin{itemize}
  % \item Developed in the Australian NICTA project
  % \item Maintained by Monash University
  % \item Modelling tool with multiple back-end solvers
  % \item Available from \url{https://www.minizinc.org/}
  % \end{itemize}
% \end{frame}


% \begin{frame}
% \frametitle{Examples in ECLiPSe}
% \begin{itemize}
% \item Open sourced constraint programming language
% \item Development goes back to 1985
% \item ECRC, ICL, IC-Parc, PTL, Cisco
% \item \url{https://eclipseclp.org/}
% \item Specialities
% \begin{itemize}
% \item Develop new solvers for specific domains
% \item Integration with MIP
% \end{itemize}
% \item Not included in bundled MiniZinc IDE
% \item Specialized visualization tools used here
  % \begin{itemize}
  % \item CP-Viz, Simonis et al. 2010
    % \end{itemize}
% \end{itemize}
% \end{frame}

\begin{frame}
\frametitle{Tutorial Based on ECLiPSe ELearning Course}
\begin{itemize}
\item Self-study course in constraint programming
\item Supported by Cisco Systems and Silicon Valley Community Foundation
\item Multi-media format, video lectures, slides, handout etc
\item {\small \url{https://eclipseclp.org/ELearning/index.html}}
\end{itemize}
\end{frame}

\begin{frame}
\frametitle{Also Part of CRT-AI Constraint Week}
\begin{itemize}
\item Annual one week course on CP and Optimization in Ireland
\item Part of national training program for PhD students in AI
\item \url{https://www.crt-ai.ie/}
\end{itemize}
\end{frame}



\begin{frame}
\frametitle{Constraint Programming - in a nutshell}
\begin{itemize}
\item Declarative description of problems with
\begin{itemize}
\item {\em Variables} which range over (finite) sets of values
\item {\em Constraints} over subsets of variables which restrict possible value combinations
\item A {\em solution} is a value assignment which satisfies all constraints
\end{itemize}

\item Constraint propagation/reasoning
\begin{itemize}
\item Removing inconsistent values for variables
\item Detect failure if constraint can not be satisfied
\item Interaction of constraints via shared variables
\item Incomplete
\end{itemize}

\item Search
\begin{itemize}
\item User controlled assignment of values to variables
\item Each step triggers constraint propagation 
\end{itemize}
\item Different domains require/allow different methods
\end{itemize}
\end{frame}

\begin{frame}
  \frametitle{Constraint Programming is Different}
  \begin{itemize}
  \item Declarative Programming
    \begin{itemize}
    \item Concentrate on what you want
      \item Not how to get there
      \item Program != Algorithm
      \item Program = Model
    \end{itemize}
    \item Applied to Combinatorial Problems
      \begin{itemize}
        \item No complete polynomial algorithms known (exist?)
        \item CP less ad-hoc than heuristics
        \item Models can evolve
  \end{itemize}
  \end{itemize}
  \end{frame}
    
\begin{frame}
  \frametitle{A Subtractive Process}
  \begin{textblock}{4}(8,-3)
    \includegraphics[width=4cm]{../introduction/images/stages}
  \end{textblock}
  \vfill
  \begin{quote}
    ``Oh, bosh, as Mr. Ruskin says. Sculpture, per se, is the simplest thing in the world. All you have to do is to take a big chunk of marble and a hammer and chisel, make up your mind what you are about to create and chip off all the marble you don't want.''-Paris Gaulois.
  \end{quote}
  
  {\tiny Source: \url{https://quoteinvestigator.com/2014/06/22/chip-away/}}
\end{frame}

\begin{frame}
\frametitle{Basic Process}
\begin{center}
\begin{tikzpicture}
\node[shape=rectangle,fill=pantone127-4] (problem) at (2,5) {Problem};
\node (human) at (2,4) {Human};
\node[shape=rectangle,fill=pantone127-4] (model) at (2,3) {Model};
\node (solver) at (2,2) {Constraint Solver/Search};
\node[shape=rectangle,fill=pantone157-8] (solution) at (2,1) {Solution};
\draw[-] (problem) -- (human);
\draw[->] (human) -- (model);
\draw[-] (model) -- (solver);
\draw[->] (solver) -- (solution);
\end{tikzpicture}
\end{center}
\end{frame}

\begin{frame}
\frametitle{More Realistic}
\begin{center}
\begin{tikzpicture}[xscale=4.0]
\node[shape=rectangle,fill=pantone127-4] (problem) at (2,5) {Problem};
\node (human) at (2,4) {Human};
\node[shape=rectangle,fill=pantone127-4] (model) at (2,3) {Model};
\node (solver) at (2,2) {Constraint Solver/Search};
\node[shape=rectangle,fill=pantone24-8] (hangs) at (1,1) {Hangs};
\node[shape=rectangle,fill=pantone157-8] (solution) at (2,1) {Solution};
\node[shape=rectangle,fill=pantone24-8] (wrong) at (3,1) {Wrong Solution};
\draw[-] (problem) -- (human);
\draw[->] (human) -- (model);
\draw[-] (model) -- (solver);
\draw[->] (solver) -- (hangs);
\draw[->] (solver) -- (solution);
\draw[->] (solver) -- (wrong);
\draw[->] (hangs) -- (1,4) -- (human);
\draw[->] (wrong) -- (3,4) -- (human);
\end{tikzpicture}
\end{center}
\end{frame}

\begin{frame}
\frametitle{Dual Role of Model}
\begin{itemize}
\item Allows Human to Express Problem
\begin{itemize}
\item Close to Problem Domain
\item Constraints as Abstractions
\end{itemize}
\item Allows Solver to Execute
\begin{itemize}
\item Variables as Communication Mechanism
\item Constraints as Algorithms
\end{itemize}
\end{itemize}
\end{frame}

\begin{frame}
\frametitle{Modelling Frameworks}
\begin{itemize}
\item \href{https://www.minizinc.org/}{MiniZinc} (NICTA, Monash University, Australia)
\item \href{https://github.com/eomahony/Numberjack}{NumberJack} (Insight, Ireland) 
\item \href{https://www-users.york.ac.uk/peter.nightingale/savilerow//}{EssencePrime/SavilleRow} (UK) 
\item \href{https://cpmpy.readthedocs.io/en/latest/}{CPMpy} (KU Leuven)
\item Allow use of multiple back-end solvers
\item Compile model into variants for each solver
\item A priori solver independent model(CP, MIP, SAT)
\end{itemize}
\end{frame} 

\begin{frame}
\frametitle{Framework Process}
\begin{center}
\begin{tikzpicture}[xscale=2]
\node[shape=rectangle,fill=pantone127-4] (problem) at (2.5,5) {Problem};
\node (human) at (2.5,4) {Human};
\node[shape=rectangle,fill=pantone127-4] (model) at (2.5,3) {Model};
\node (compile) at (2.5,2) {Compile/Reformulate};
\node (solver1) at (1,1) {CP};
\node (solver2) at (2,1) {MIP};
\node (solver3) at (3,1) {SAT};
\node (solver4) at (4,1) {Other};
\node[shape=rectangle,fill=pantone157-8] (solution1) at (1,0) {Solution};
\node[shape=rectangle,fill=pantone157-8] (solution2) at (2,0) {Solution};
\node[shape=rectangle,fill=pantone157-8] (solution3) at (3,0) {Solution};
\node[shape=rectangle,fill=pantone157-8] (solution4) at (4,0) {Solution};
\draw[-] (problem) -- (human);
\draw[->] (human) -- (model);
\draw[-] (model) -- (compile);
\draw[-] (compile) -- (solver1);
\draw[-] (compile) -- (solver2);
\draw[-] (compile) -- (solver3);
\draw[-] (compile) -- (solver4);
\draw[->] (solver1) -- (solution1);
\draw[->] (solver2) -- (solution2);
\draw[->] (solver3) -- (solution3);
\draw[->] (solver4) -- (solution4);
\end{tikzpicture}
\end{center}
\end{frame}

% \begin{frame}
  % \frametitle{Do It Now!}
  % \begin{itemize}
  % \item Download and install Minizinc
  % \item \url{https://www.minizinc.org/}
  % \end{itemize}
% \end{frame}


\begin{frame}
\frametitle{Why use Puzzles as Examples?}
\begin{itemize}
\item Easy to understand the problem
\item Solvable by hand without specialized knowledge
\item Possible to compare automated to manual solving process
\end{itemize}

\begin{quote}
The puzzle, though inanimate, is presented as a solvable problem without lasting negative consequences, a very low-risk low-reward situation. By being a puzzle, the object is attempting to convince the user that it must be completed.
\end{quote}
{\tiny Source: \href{https://www.everydayrhetoric.com/2020/03/the-rhetoric-of-puzzles/}{Every Day Rhetoric}}
\end{frame}


\mode<all>{
\part{Basic Constraint Propagation}
}
\begin{frame}
\partpage
\end{frame}
\mode<all>{
\input{../sendmore/body}
}

\mode<all>{
\part{Global Constraints}
}
\begin{frame}
\partpage
\end{frame}
\mode<all>{
\input{../sudoku/body}
}

\clearpage
\mode<all>{
\part{Customizing Search}
}
\begin{frame}
\partpage
\end{frame}
\mode<all>{
\input{../nqueen/body}
}


%% \clearpage
%% \mode<all>{
%%   \part{Problem Decomposition}
%% }
%% \begin{frame}
%%   \partpage
%% \end{frame}
%% \mode<all>{
%%   \input{../party/body}
%%   }

%% \clearpage
%% \mode<all>{
%%   \part{More Global Constraints}
%% }
%% \begin{frame}
%%   \partpage
%% \end{frame}
%% \mode<all>{
%%   \input{../car/body}
%%   }

%% \clearpage
%% \mode<all>{
%%   \part{Methodology}
%% }
%% \begin{frame}
%%   \partpage
%% \end{frame}
%% \mode<all>{
%%   \input{../methodology/body}
%%   }

\clearpage
\mode<all>{
\part{What is missing?}
}
\begin{frame}
\partpage
\end{frame}

\begin{frame}
\frametitle{Many Specialized Topics}
\begin{itemize}
\item How to design efficient core engine
\item Hybrids with LP/MIP tools
\item Hybrids with SAT
\item Symmetry breaking
\item Use of MDD/BDD to encode sets of solutions
\item High level modelling tools
\item Debugging/visualization
\end{itemize}
\end{frame}

\begin{frame}
\frametitle{Reformulation}
\begin{itemize}
\item Just because the user has modelled it this way, it doesn't mean we have to solve it that way
\begin{itemize}
\item Replace some constraint(s) by other, equivalent constraints
\item Because we don't have that constraint in our system
\item For performance
\end{itemize}
\end{itemize}
\end{frame}

\begin{frame}
\frametitle{Learning}
\begin{itemize}
\item While solving the problem we can learn how to strengthen the model/search 
\begin{itemize}
\item Understand which constraints/method contribute to propagation and change schedule
\item Learn no-good constraints by explaining failure
\item Adapt search strategy based on search experience
\end{itemize}
\end{itemize}
\end{frame}

%% \begin{frame}
%% \frametitle{Refined Process}
%% \begin{center}
%% \begin{tikzpicture}
%% \node[shape=rectangle,fill=pantone127-4] (problem) at (3,7) {Problem};
%% \node (human) at (3,6) {Human};
%% \node[shape=rectangle,fill=pantone127-4] (model) at (3,5) {User Model};
%% \node (reformulation) at (3,4) {Reformulation};
%% \node[shape=rectangle,fill=pantone127-4] (implementation) at (3,3) {Implementation Model};
%% \node (solver) at (3,2) {Constraint Solver/Search};
%% \node[shape=rectangle,fill=pantone157-8] (solution) at (3,1) {Solution};
%% \draw[-] (problem) -- (human);
%% \draw[->] (human) -- (model);
%% \draw[-] (model) -- (reformulation);
%% \draw[->] (reformulation) -- (implementation);
%% \draw[-] (implementation) -- (solver);
%% \draw[->] (solver) -- (0,2) -- (0,3) -- (implementation);
%% \draw[->] (solver) -- (solution);
%% \end{tikzpicture}
%% \end{center}
%% \end{frame}

%\begin{frame}
%\frametitle{What is CP actually used for?}
%\url{http://hsimonis.wordpress.com}
%\includegraphics[width=10cm]{../introduction/images/blog}
%\end{frame}

%% \begin{frame}
%% \frametitle{Exercises}
%% \begin{itemize}
%% \item Install Constraint Solver of your Choice
%% \item MiniZinc (MiningZinc) will be used in later sessions
%% \item Run Sendmoremoney, Sudoku, Queens examples
%% \item Can you choose which propagator to use for alldifferent?
%% \item Can you use different search strategies?
%% \end{itemize}
%% \end{frame}

\begin{frame}
\frametitle{More Learning Resources}
\begin{itemize}
\item Survey of Methods, Resources, and Formats for Teaching Constraint Programming
\begin{itemize}
\item by Tejas Santanam, Helmut Simonis
\item \url{https://doi.org/10.48550/arXiv.2403.12717}
\item Based on survey of community for WTCP 2023
\item \url{https://hsimonis.github.io/WTCP2023/}
\end{itemize}
\end{itemize}
\end{frame}


}

\clearpage
\mode<all>{
\part{Customizing Search}
}
\begin{frame}
\partpage
\end{frame}
\mode<all>{
\begin{frame}
\frametitle{Objectives}
\begin{itemize}
\item Overview of Core Constraint Programming
\item Three Main Concepts
\begin{itemize}
\item Constraint Propagation
\item Global Constraints
\item Customizing Search
\end{itemize}
%\item Get Some Experience with MiniZinc
\item Topics will be treated in more detail in later parts of the school
\item Based on Examples, not Formal Description
\end{itemize}
\end{frame}

\begin{frame}
\frametitle{Outline}
\begin{itemize}
\item Why Constraint Programming?
\item Constraint Propagation
\item Global Constraints
\item Customizing Search
%\item Choices in Modelling
%\item Decomposition
%\item More Global Constraints
%\item Methodology
%\item What is missing?
\end{itemize}
\end{frame}

% \begin{frame}
  % \frametitle{Using MiniZinc IDE}
  % \begin{itemize}
  % \item Developed in the Australian NICTA project
  % \item Maintained by Monash University
  % \item Modelling tool with multiple back-end solvers
  % \item Available from \url{https://www.minizinc.org/}
  % \end{itemize}
% \end{frame}


% \begin{frame}
% \frametitle{Examples in ECLiPSe}
% \begin{itemize}
% \item Open sourced constraint programming language
% \item Development goes back to 1985
% \item ECRC, ICL, IC-Parc, PTL, Cisco
% \item \url{https://eclipseclp.org/}
% \item Specialities
% \begin{itemize}
% \item Develop new solvers for specific domains
% \item Integration with MIP
% \end{itemize}
% \item Not included in bundled MiniZinc IDE
% \item Specialized visualization tools used here
  % \begin{itemize}
  % \item CP-Viz, Simonis et al. 2010
    % \end{itemize}
% \end{itemize}
% \end{frame}

\begin{frame}
\frametitle{Tutorial Based on ECLiPSe ELearning Course}
\begin{itemize}
\item Self-study course in constraint programming
\item Supported by Cisco Systems and Silicon Valley Community Foundation
\item Multi-media format, video lectures, slides, handout etc
\item {\small \url{https://eclipseclp.org/ELearning/index.html}}
\end{itemize}
\end{frame}

\begin{frame}
\frametitle{Also Part of CRT-AI Constraint Week}
\begin{itemize}
\item Annual one week course on CP and Optimization in Ireland
\item Part of national training program for PhD students in AI
\item \url{https://www.crt-ai.ie/}
\end{itemize}
\end{frame}



\begin{frame}
\frametitle{Constraint Programming - in a nutshell}
\begin{itemize}
\item Declarative description of problems with
\begin{itemize}
\item {\em Variables} which range over (finite) sets of values
\item {\em Constraints} over subsets of variables which restrict possible value combinations
\item A {\em solution} is a value assignment which satisfies all constraints
\end{itemize}

\item Constraint propagation/reasoning
\begin{itemize}
\item Removing inconsistent values for variables
\item Detect failure if constraint can not be satisfied
\item Interaction of constraints via shared variables
\item Incomplete
\end{itemize}

\item Search
\begin{itemize}
\item User controlled assignment of values to variables
\item Each step triggers constraint propagation 
\end{itemize}
\item Different domains require/allow different methods
\end{itemize}
\end{frame}

\begin{frame}
  \frametitle{Constraint Programming is Different}
  \begin{itemize}
  \item Declarative Programming
    \begin{itemize}
    \item Concentrate on what you want
      \item Not how to get there
      \item Program != Algorithm
      \item Program = Model
    \end{itemize}
    \item Applied to Combinatorial Problems
      \begin{itemize}
        \item No complete polynomial algorithms known (exist?)
        \item CP less ad-hoc than heuristics
        \item Models can evolve
  \end{itemize}
  \end{itemize}
  \end{frame}
    
\begin{frame}
  \frametitle{A Subtractive Process}
  \begin{textblock}{4}(8,-3)
    \includegraphics[width=4cm]{../introduction/images/stages}
  \end{textblock}
  \vfill
  \begin{quote}
    ``Oh, bosh, as Mr. Ruskin says. Sculpture, per se, is the simplest thing in the world. All you have to do is to take a big chunk of marble and a hammer and chisel, make up your mind what you are about to create and chip off all the marble you don't want.''-Paris Gaulois.
  \end{quote}
  
  {\tiny Source: \url{https://quoteinvestigator.com/2014/06/22/chip-away/}}
\end{frame}

\begin{frame}
\frametitle{Basic Process}
\begin{center}
\begin{tikzpicture}
\node[shape=rectangle,fill=pantone127-4] (problem) at (2,5) {Problem};
\node (human) at (2,4) {Human};
\node[shape=rectangle,fill=pantone127-4] (model) at (2,3) {Model};
\node (solver) at (2,2) {Constraint Solver/Search};
\node[shape=rectangle,fill=pantone157-8] (solution) at (2,1) {Solution};
\draw[-] (problem) -- (human);
\draw[->] (human) -- (model);
\draw[-] (model) -- (solver);
\draw[->] (solver) -- (solution);
\end{tikzpicture}
\end{center}
\end{frame}

\begin{frame}
\frametitle{More Realistic}
\begin{center}
\begin{tikzpicture}[xscale=4.0]
\node[shape=rectangle,fill=pantone127-4] (problem) at (2,5) {Problem};
\node (human) at (2,4) {Human};
\node[shape=rectangle,fill=pantone127-4] (model) at (2,3) {Model};
\node (solver) at (2,2) {Constraint Solver/Search};
\node[shape=rectangle,fill=pantone24-8] (hangs) at (1,1) {Hangs};
\node[shape=rectangle,fill=pantone157-8] (solution) at (2,1) {Solution};
\node[shape=rectangle,fill=pantone24-8] (wrong) at (3,1) {Wrong Solution};
\draw[-] (problem) -- (human);
\draw[->] (human) -- (model);
\draw[-] (model) -- (solver);
\draw[->] (solver) -- (hangs);
\draw[->] (solver) -- (solution);
\draw[->] (solver) -- (wrong);
\draw[->] (hangs) -- (1,4) -- (human);
\draw[->] (wrong) -- (3,4) -- (human);
\end{tikzpicture}
\end{center}
\end{frame}

\begin{frame}
\frametitle{Dual Role of Model}
\begin{itemize}
\item Allows Human to Express Problem
\begin{itemize}
\item Close to Problem Domain
\item Constraints as Abstractions
\end{itemize}
\item Allows Solver to Execute
\begin{itemize}
\item Variables as Communication Mechanism
\item Constraints as Algorithms
\end{itemize}
\end{itemize}
\end{frame}

\begin{frame}
\frametitle{Modelling Frameworks}
\begin{itemize}
\item \href{https://www.minizinc.org/}{MiniZinc} (NICTA, Monash University, Australia)
\item \href{https://github.com/eomahony/Numberjack}{NumberJack} (Insight, Ireland) 
\item \href{https://www-users.york.ac.uk/peter.nightingale/savilerow//}{EssencePrime/SavilleRow} (UK) 
\item \href{https://cpmpy.readthedocs.io/en/latest/}{CPMpy} (KU Leuven)
\item Allow use of multiple back-end solvers
\item Compile model into variants for each solver
\item A priori solver independent model(CP, MIP, SAT)
\end{itemize}
\end{frame} 

\begin{frame}
\frametitle{Framework Process}
\begin{center}
\begin{tikzpicture}[xscale=2]
\node[shape=rectangle,fill=pantone127-4] (problem) at (2.5,5) {Problem};
\node (human) at (2.5,4) {Human};
\node[shape=rectangle,fill=pantone127-4] (model) at (2.5,3) {Model};
\node (compile) at (2.5,2) {Compile/Reformulate};
\node (solver1) at (1,1) {CP};
\node (solver2) at (2,1) {MIP};
\node (solver3) at (3,1) {SAT};
\node (solver4) at (4,1) {Other};
\node[shape=rectangle,fill=pantone157-8] (solution1) at (1,0) {Solution};
\node[shape=rectangle,fill=pantone157-8] (solution2) at (2,0) {Solution};
\node[shape=rectangle,fill=pantone157-8] (solution3) at (3,0) {Solution};
\node[shape=rectangle,fill=pantone157-8] (solution4) at (4,0) {Solution};
\draw[-] (problem) -- (human);
\draw[->] (human) -- (model);
\draw[-] (model) -- (compile);
\draw[-] (compile) -- (solver1);
\draw[-] (compile) -- (solver2);
\draw[-] (compile) -- (solver3);
\draw[-] (compile) -- (solver4);
\draw[->] (solver1) -- (solution1);
\draw[->] (solver2) -- (solution2);
\draw[->] (solver3) -- (solution3);
\draw[->] (solver4) -- (solution4);
\end{tikzpicture}
\end{center}
\end{frame}

% \begin{frame}
  % \frametitle{Do It Now!}
  % \begin{itemize}
  % \item Download and install Minizinc
  % \item \url{https://www.minizinc.org/}
  % \end{itemize}
% \end{frame}


\begin{frame}
\frametitle{Why use Puzzles as Examples?}
\begin{itemize}
\item Easy to understand the problem
\item Solvable by hand without specialized knowledge
\item Possible to compare automated to manual solving process
\end{itemize}

\begin{quote}
The puzzle, though inanimate, is presented as a solvable problem without lasting negative consequences, a very low-risk low-reward situation. By being a puzzle, the object is attempting to convince the user that it must be completed.
\end{quote}
{\tiny Source: \href{https://www.everydayrhetoric.com/2020/03/the-rhetoric-of-puzzles/}{Every Day Rhetoric}}
\end{frame}


\mode<all>{
\part{Basic Constraint Propagation}
}
\begin{frame}
\partpage
\end{frame}
\mode<all>{
\input{../sendmore/body}
}

\mode<all>{
\part{Global Constraints}
}
\begin{frame}
\partpage
\end{frame}
\mode<all>{
\input{../sudoku/body}
}

\clearpage
\mode<all>{
\part{Customizing Search}
}
\begin{frame}
\partpage
\end{frame}
\mode<all>{
\input{../nqueen/body}
}


%% \clearpage
%% \mode<all>{
%%   \part{Problem Decomposition}
%% }
%% \begin{frame}
%%   \partpage
%% \end{frame}
%% \mode<all>{
%%   \input{../party/body}
%%   }

%% \clearpage
%% \mode<all>{
%%   \part{More Global Constraints}
%% }
%% \begin{frame}
%%   \partpage
%% \end{frame}
%% \mode<all>{
%%   \input{../car/body}
%%   }

%% \clearpage
%% \mode<all>{
%%   \part{Methodology}
%% }
%% \begin{frame}
%%   \partpage
%% \end{frame}
%% \mode<all>{
%%   \input{../methodology/body}
%%   }

\clearpage
\mode<all>{
\part{What is missing?}
}
\begin{frame}
\partpage
\end{frame}

\begin{frame}
\frametitle{Many Specialized Topics}
\begin{itemize}
\item How to design efficient core engine
\item Hybrids with LP/MIP tools
\item Hybrids with SAT
\item Symmetry breaking
\item Use of MDD/BDD to encode sets of solutions
\item High level modelling tools
\item Debugging/visualization
\end{itemize}
\end{frame}

\begin{frame}
\frametitle{Reformulation}
\begin{itemize}
\item Just because the user has modelled it this way, it doesn't mean we have to solve it that way
\begin{itemize}
\item Replace some constraint(s) by other, equivalent constraints
\item Because we don't have that constraint in our system
\item For performance
\end{itemize}
\end{itemize}
\end{frame}

\begin{frame}
\frametitle{Learning}
\begin{itemize}
\item While solving the problem we can learn how to strengthen the model/search 
\begin{itemize}
\item Understand which constraints/method contribute to propagation and change schedule
\item Learn no-good constraints by explaining failure
\item Adapt search strategy based on search experience
\end{itemize}
\end{itemize}
\end{frame}

%% \begin{frame}
%% \frametitle{Refined Process}
%% \begin{center}
%% \begin{tikzpicture}
%% \node[shape=rectangle,fill=pantone127-4] (problem) at (3,7) {Problem};
%% \node (human) at (3,6) {Human};
%% \node[shape=rectangle,fill=pantone127-4] (model) at (3,5) {User Model};
%% \node (reformulation) at (3,4) {Reformulation};
%% \node[shape=rectangle,fill=pantone127-4] (implementation) at (3,3) {Implementation Model};
%% \node (solver) at (3,2) {Constraint Solver/Search};
%% \node[shape=rectangle,fill=pantone157-8] (solution) at (3,1) {Solution};
%% \draw[-] (problem) -- (human);
%% \draw[->] (human) -- (model);
%% \draw[-] (model) -- (reformulation);
%% \draw[->] (reformulation) -- (implementation);
%% \draw[-] (implementation) -- (solver);
%% \draw[->] (solver) -- (0,2) -- (0,3) -- (implementation);
%% \draw[->] (solver) -- (solution);
%% \end{tikzpicture}
%% \end{center}
%% \end{frame}

%\begin{frame}
%\frametitle{What is CP actually used for?}
%\url{http://hsimonis.wordpress.com}
%\includegraphics[width=10cm]{../introduction/images/blog}
%\end{frame}

%% \begin{frame}
%% \frametitle{Exercises}
%% \begin{itemize}
%% \item Install Constraint Solver of your Choice
%% \item MiniZinc (MiningZinc) will be used in later sessions
%% \item Run Sendmoremoney, Sudoku, Queens examples
%% \item Can you choose which propagator to use for alldifferent?
%% \item Can you use different search strategies?
%% \end{itemize}
%% \end{frame}

\begin{frame}
\frametitle{More Learning Resources}
\begin{itemize}
\item Survey of Methods, Resources, and Formats for Teaching Constraint Programming
\begin{itemize}
\item by Tejas Santanam, Helmut Simonis
\item \url{https://doi.org/10.48550/arXiv.2403.12717}
\item Based on survey of community for WTCP 2023
\item \url{https://hsimonis.github.io/WTCP2023/}
\end{itemize}
\end{itemize}
\end{frame}


}


%% \clearpage
%% \mode<all>{
%%   \part{Problem Decomposition}
%% }
%% \begin{frame}
%%   \partpage
%% \end{frame}
%% \mode<all>{
%%   \begin{frame}
\frametitle{Objectives}
\begin{itemize}
\item Overview of Core Constraint Programming
\item Three Main Concepts
\begin{itemize}
\item Constraint Propagation
\item Global Constraints
\item Customizing Search
\end{itemize}
%\item Get Some Experience with MiniZinc
\item Topics will be treated in more detail in later parts of the school
\item Based on Examples, not Formal Description
\end{itemize}
\end{frame}

\begin{frame}
\frametitle{Outline}
\begin{itemize}
\item Why Constraint Programming?
\item Constraint Propagation
\item Global Constraints
\item Customizing Search
%\item Choices in Modelling
%\item Decomposition
%\item More Global Constraints
%\item Methodology
%\item What is missing?
\end{itemize}
\end{frame}

% \begin{frame}
  % \frametitle{Using MiniZinc IDE}
  % \begin{itemize}
  % \item Developed in the Australian NICTA project
  % \item Maintained by Monash University
  % \item Modelling tool with multiple back-end solvers
  % \item Available from \url{https://www.minizinc.org/}
  % \end{itemize}
% \end{frame}


% \begin{frame}
% \frametitle{Examples in ECLiPSe}
% \begin{itemize}
% \item Open sourced constraint programming language
% \item Development goes back to 1985
% \item ECRC, ICL, IC-Parc, PTL, Cisco
% \item \url{https://eclipseclp.org/}
% \item Specialities
% \begin{itemize}
% \item Develop new solvers for specific domains
% \item Integration with MIP
% \end{itemize}
% \item Not included in bundled MiniZinc IDE
% \item Specialized visualization tools used here
  % \begin{itemize}
  % \item CP-Viz, Simonis et al. 2010
    % \end{itemize}
% \end{itemize}
% \end{frame}

\begin{frame}
\frametitle{Tutorial Based on ECLiPSe ELearning Course}
\begin{itemize}
\item Self-study course in constraint programming
\item Supported by Cisco Systems and Silicon Valley Community Foundation
\item Multi-media format, video lectures, slides, handout etc
\item {\small \url{https://eclipseclp.org/ELearning/index.html}}
\end{itemize}
\end{frame}

\begin{frame}
\frametitle{Also Part of CRT-AI Constraint Week}
\begin{itemize}
\item Annual one week course on CP and Optimization in Ireland
\item Part of national training program for PhD students in AI
\item \url{https://www.crt-ai.ie/}
\end{itemize}
\end{frame}



\begin{frame}
\frametitle{Constraint Programming - in a nutshell}
\begin{itemize}
\item Declarative description of problems with
\begin{itemize}
\item {\em Variables} which range over (finite) sets of values
\item {\em Constraints} over subsets of variables which restrict possible value combinations
\item A {\em solution} is a value assignment which satisfies all constraints
\end{itemize}

\item Constraint propagation/reasoning
\begin{itemize}
\item Removing inconsistent values for variables
\item Detect failure if constraint can not be satisfied
\item Interaction of constraints via shared variables
\item Incomplete
\end{itemize}

\item Search
\begin{itemize}
\item User controlled assignment of values to variables
\item Each step triggers constraint propagation 
\end{itemize}
\item Different domains require/allow different methods
\end{itemize}
\end{frame}

\begin{frame}
  \frametitle{Constraint Programming is Different}
  \begin{itemize}
  \item Declarative Programming
    \begin{itemize}
    \item Concentrate on what you want
      \item Not how to get there
      \item Program != Algorithm
      \item Program = Model
    \end{itemize}
    \item Applied to Combinatorial Problems
      \begin{itemize}
        \item No complete polynomial algorithms known (exist?)
        \item CP less ad-hoc than heuristics
        \item Models can evolve
  \end{itemize}
  \end{itemize}
  \end{frame}
    
\begin{frame}
  \frametitle{A Subtractive Process}
  \begin{textblock}{4}(8,-3)
    \includegraphics[width=4cm]{../introduction/images/stages}
  \end{textblock}
  \vfill
  \begin{quote}
    ``Oh, bosh, as Mr. Ruskin says. Sculpture, per se, is the simplest thing in the world. All you have to do is to take a big chunk of marble and a hammer and chisel, make up your mind what you are about to create and chip off all the marble you don't want.''-Paris Gaulois.
  \end{quote}
  
  {\tiny Source: \url{https://quoteinvestigator.com/2014/06/22/chip-away/}}
\end{frame}

\begin{frame}
\frametitle{Basic Process}
\begin{center}
\begin{tikzpicture}
\node[shape=rectangle,fill=pantone127-4] (problem) at (2,5) {Problem};
\node (human) at (2,4) {Human};
\node[shape=rectangle,fill=pantone127-4] (model) at (2,3) {Model};
\node (solver) at (2,2) {Constraint Solver/Search};
\node[shape=rectangle,fill=pantone157-8] (solution) at (2,1) {Solution};
\draw[-] (problem) -- (human);
\draw[->] (human) -- (model);
\draw[-] (model) -- (solver);
\draw[->] (solver) -- (solution);
\end{tikzpicture}
\end{center}
\end{frame}

\begin{frame}
\frametitle{More Realistic}
\begin{center}
\begin{tikzpicture}[xscale=4.0]
\node[shape=rectangle,fill=pantone127-4] (problem) at (2,5) {Problem};
\node (human) at (2,4) {Human};
\node[shape=rectangle,fill=pantone127-4] (model) at (2,3) {Model};
\node (solver) at (2,2) {Constraint Solver/Search};
\node[shape=rectangle,fill=pantone24-8] (hangs) at (1,1) {Hangs};
\node[shape=rectangle,fill=pantone157-8] (solution) at (2,1) {Solution};
\node[shape=rectangle,fill=pantone24-8] (wrong) at (3,1) {Wrong Solution};
\draw[-] (problem) -- (human);
\draw[->] (human) -- (model);
\draw[-] (model) -- (solver);
\draw[->] (solver) -- (hangs);
\draw[->] (solver) -- (solution);
\draw[->] (solver) -- (wrong);
\draw[->] (hangs) -- (1,4) -- (human);
\draw[->] (wrong) -- (3,4) -- (human);
\end{tikzpicture}
\end{center}
\end{frame}

\begin{frame}
\frametitle{Dual Role of Model}
\begin{itemize}
\item Allows Human to Express Problem
\begin{itemize}
\item Close to Problem Domain
\item Constraints as Abstractions
\end{itemize}
\item Allows Solver to Execute
\begin{itemize}
\item Variables as Communication Mechanism
\item Constraints as Algorithms
\end{itemize}
\end{itemize}
\end{frame}

\begin{frame}
\frametitle{Modelling Frameworks}
\begin{itemize}
\item \href{https://www.minizinc.org/}{MiniZinc} (NICTA, Monash University, Australia)
\item \href{https://github.com/eomahony/Numberjack}{NumberJack} (Insight, Ireland) 
\item \href{https://www-users.york.ac.uk/peter.nightingale/savilerow//}{EssencePrime/SavilleRow} (UK) 
\item \href{https://cpmpy.readthedocs.io/en/latest/}{CPMpy} (KU Leuven)
\item Allow use of multiple back-end solvers
\item Compile model into variants for each solver
\item A priori solver independent model(CP, MIP, SAT)
\end{itemize}
\end{frame} 

\begin{frame}
\frametitle{Framework Process}
\begin{center}
\begin{tikzpicture}[xscale=2]
\node[shape=rectangle,fill=pantone127-4] (problem) at (2.5,5) {Problem};
\node (human) at (2.5,4) {Human};
\node[shape=rectangle,fill=pantone127-4] (model) at (2.5,3) {Model};
\node (compile) at (2.5,2) {Compile/Reformulate};
\node (solver1) at (1,1) {CP};
\node (solver2) at (2,1) {MIP};
\node (solver3) at (3,1) {SAT};
\node (solver4) at (4,1) {Other};
\node[shape=rectangle,fill=pantone157-8] (solution1) at (1,0) {Solution};
\node[shape=rectangle,fill=pantone157-8] (solution2) at (2,0) {Solution};
\node[shape=rectangle,fill=pantone157-8] (solution3) at (3,0) {Solution};
\node[shape=rectangle,fill=pantone157-8] (solution4) at (4,0) {Solution};
\draw[-] (problem) -- (human);
\draw[->] (human) -- (model);
\draw[-] (model) -- (compile);
\draw[-] (compile) -- (solver1);
\draw[-] (compile) -- (solver2);
\draw[-] (compile) -- (solver3);
\draw[-] (compile) -- (solver4);
\draw[->] (solver1) -- (solution1);
\draw[->] (solver2) -- (solution2);
\draw[->] (solver3) -- (solution3);
\draw[->] (solver4) -- (solution4);
\end{tikzpicture}
\end{center}
\end{frame}

% \begin{frame}
  % \frametitle{Do It Now!}
  % \begin{itemize}
  % \item Download and install Minizinc
  % \item \url{https://www.minizinc.org/}
  % \end{itemize}
% \end{frame}


\begin{frame}
\frametitle{Why use Puzzles as Examples?}
\begin{itemize}
\item Easy to understand the problem
\item Solvable by hand without specialized knowledge
\item Possible to compare automated to manual solving process
\end{itemize}

\begin{quote}
The puzzle, though inanimate, is presented as a solvable problem without lasting negative consequences, a very low-risk low-reward situation. By being a puzzle, the object is attempting to convince the user that it must be completed.
\end{quote}
{\tiny Source: \href{https://www.everydayrhetoric.com/2020/03/the-rhetoric-of-puzzles/}{Every Day Rhetoric}}
\end{frame}


\mode<all>{
\part{Basic Constraint Propagation}
}
\begin{frame}
\partpage
\end{frame}
\mode<all>{
\input{../sendmore/body}
}

\mode<all>{
\part{Global Constraints}
}
\begin{frame}
\partpage
\end{frame}
\mode<all>{
\input{../sudoku/body}
}

\clearpage
\mode<all>{
\part{Customizing Search}
}
\begin{frame}
\partpage
\end{frame}
\mode<all>{
\input{../nqueen/body}
}


%% \clearpage
%% \mode<all>{
%%   \part{Problem Decomposition}
%% }
%% \begin{frame}
%%   \partpage
%% \end{frame}
%% \mode<all>{
%%   \input{../party/body}
%%   }

%% \clearpage
%% \mode<all>{
%%   \part{More Global Constraints}
%% }
%% \begin{frame}
%%   \partpage
%% \end{frame}
%% \mode<all>{
%%   \input{../car/body}
%%   }

%% \clearpage
%% \mode<all>{
%%   \part{Methodology}
%% }
%% \begin{frame}
%%   \partpage
%% \end{frame}
%% \mode<all>{
%%   \input{../methodology/body}
%%   }

\clearpage
\mode<all>{
\part{What is missing?}
}
\begin{frame}
\partpage
\end{frame}

\begin{frame}
\frametitle{Many Specialized Topics}
\begin{itemize}
\item How to design efficient core engine
\item Hybrids with LP/MIP tools
\item Hybrids with SAT
\item Symmetry breaking
\item Use of MDD/BDD to encode sets of solutions
\item High level modelling tools
\item Debugging/visualization
\end{itemize}
\end{frame}

\begin{frame}
\frametitle{Reformulation}
\begin{itemize}
\item Just because the user has modelled it this way, it doesn't mean we have to solve it that way
\begin{itemize}
\item Replace some constraint(s) by other, equivalent constraints
\item Because we don't have that constraint in our system
\item For performance
\end{itemize}
\end{itemize}
\end{frame}

\begin{frame}
\frametitle{Learning}
\begin{itemize}
\item While solving the problem we can learn how to strengthen the model/search 
\begin{itemize}
\item Understand which constraints/method contribute to propagation and change schedule
\item Learn no-good constraints by explaining failure
\item Adapt search strategy based on search experience
\end{itemize}
\end{itemize}
\end{frame}

%% \begin{frame}
%% \frametitle{Refined Process}
%% \begin{center}
%% \begin{tikzpicture}
%% \node[shape=rectangle,fill=pantone127-4] (problem) at (3,7) {Problem};
%% \node (human) at (3,6) {Human};
%% \node[shape=rectangle,fill=pantone127-4] (model) at (3,5) {User Model};
%% \node (reformulation) at (3,4) {Reformulation};
%% \node[shape=rectangle,fill=pantone127-4] (implementation) at (3,3) {Implementation Model};
%% \node (solver) at (3,2) {Constraint Solver/Search};
%% \node[shape=rectangle,fill=pantone157-8] (solution) at (3,1) {Solution};
%% \draw[-] (problem) -- (human);
%% \draw[->] (human) -- (model);
%% \draw[-] (model) -- (reformulation);
%% \draw[->] (reformulation) -- (implementation);
%% \draw[-] (implementation) -- (solver);
%% \draw[->] (solver) -- (0,2) -- (0,3) -- (implementation);
%% \draw[->] (solver) -- (solution);
%% \end{tikzpicture}
%% \end{center}
%% \end{frame}

%\begin{frame}
%\frametitle{What is CP actually used for?}
%\url{http://hsimonis.wordpress.com}
%\includegraphics[width=10cm]{../introduction/images/blog}
%\end{frame}

%% \begin{frame}
%% \frametitle{Exercises}
%% \begin{itemize}
%% \item Install Constraint Solver of your Choice
%% \item MiniZinc (MiningZinc) will be used in later sessions
%% \item Run Sendmoremoney, Sudoku, Queens examples
%% \item Can you choose which propagator to use for alldifferent?
%% \item Can you use different search strategies?
%% \end{itemize}
%% \end{frame}

\begin{frame}
\frametitle{More Learning Resources}
\begin{itemize}
\item Survey of Methods, Resources, and Formats for Teaching Constraint Programming
\begin{itemize}
\item by Tejas Santanam, Helmut Simonis
\item \url{https://doi.org/10.48550/arXiv.2403.12717}
\item Based on survey of community for WTCP 2023
\item \url{https://hsimonis.github.io/WTCP2023/}
\end{itemize}
\end{itemize}
\end{frame}


%%   }

%% \clearpage
%% \mode<all>{
%%   \part{More Global Constraints}
%% }
%% \begin{frame}
%%   \partpage
%% \end{frame}
%% \mode<all>{
%%   \begin{frame}
\frametitle{Objectives}
\begin{itemize}
\item Overview of Core Constraint Programming
\item Three Main Concepts
\begin{itemize}
\item Constraint Propagation
\item Global Constraints
\item Customizing Search
\end{itemize}
%\item Get Some Experience with MiniZinc
\item Topics will be treated in more detail in later parts of the school
\item Based on Examples, not Formal Description
\end{itemize}
\end{frame}

\begin{frame}
\frametitle{Outline}
\begin{itemize}
\item Why Constraint Programming?
\item Constraint Propagation
\item Global Constraints
\item Customizing Search
%\item Choices in Modelling
%\item Decomposition
%\item More Global Constraints
%\item Methodology
%\item What is missing?
\end{itemize}
\end{frame}

% \begin{frame}
  % \frametitle{Using MiniZinc IDE}
  % \begin{itemize}
  % \item Developed in the Australian NICTA project
  % \item Maintained by Monash University
  % \item Modelling tool with multiple back-end solvers
  % \item Available from \url{https://www.minizinc.org/}
  % \end{itemize}
% \end{frame}


% \begin{frame}
% \frametitle{Examples in ECLiPSe}
% \begin{itemize}
% \item Open sourced constraint programming language
% \item Development goes back to 1985
% \item ECRC, ICL, IC-Parc, PTL, Cisco
% \item \url{https://eclipseclp.org/}
% \item Specialities
% \begin{itemize}
% \item Develop new solvers for specific domains
% \item Integration with MIP
% \end{itemize}
% \item Not included in bundled MiniZinc IDE
% \item Specialized visualization tools used here
  % \begin{itemize}
  % \item CP-Viz, Simonis et al. 2010
    % \end{itemize}
% \end{itemize}
% \end{frame}

\begin{frame}
\frametitle{Tutorial Based on ECLiPSe ELearning Course}
\begin{itemize}
\item Self-study course in constraint programming
\item Supported by Cisco Systems and Silicon Valley Community Foundation
\item Multi-media format, video lectures, slides, handout etc
\item {\small \url{https://eclipseclp.org/ELearning/index.html}}
\end{itemize}
\end{frame}

\begin{frame}
\frametitle{Also Part of CRT-AI Constraint Week}
\begin{itemize}
\item Annual one week course on CP and Optimization in Ireland
\item Part of national training program for PhD students in AI
\item \url{https://www.crt-ai.ie/}
\end{itemize}
\end{frame}



\begin{frame}
\frametitle{Constraint Programming - in a nutshell}
\begin{itemize}
\item Declarative description of problems with
\begin{itemize}
\item {\em Variables} which range over (finite) sets of values
\item {\em Constraints} over subsets of variables which restrict possible value combinations
\item A {\em solution} is a value assignment which satisfies all constraints
\end{itemize}

\item Constraint propagation/reasoning
\begin{itemize}
\item Removing inconsistent values for variables
\item Detect failure if constraint can not be satisfied
\item Interaction of constraints via shared variables
\item Incomplete
\end{itemize}

\item Search
\begin{itemize}
\item User controlled assignment of values to variables
\item Each step triggers constraint propagation 
\end{itemize}
\item Different domains require/allow different methods
\end{itemize}
\end{frame}

\begin{frame}
  \frametitle{Constraint Programming is Different}
  \begin{itemize}
  \item Declarative Programming
    \begin{itemize}
    \item Concentrate on what you want
      \item Not how to get there
      \item Program != Algorithm
      \item Program = Model
    \end{itemize}
    \item Applied to Combinatorial Problems
      \begin{itemize}
        \item No complete polynomial algorithms known (exist?)
        \item CP less ad-hoc than heuristics
        \item Models can evolve
  \end{itemize}
  \end{itemize}
  \end{frame}
    
\begin{frame}
  \frametitle{A Subtractive Process}
  \begin{textblock}{4}(8,-3)
    \includegraphics[width=4cm]{../introduction/images/stages}
  \end{textblock}
  \vfill
  \begin{quote}
    ``Oh, bosh, as Mr. Ruskin says. Sculpture, per se, is the simplest thing in the world. All you have to do is to take a big chunk of marble and a hammer and chisel, make up your mind what you are about to create and chip off all the marble you don't want.''-Paris Gaulois.
  \end{quote}
  
  {\tiny Source: \url{https://quoteinvestigator.com/2014/06/22/chip-away/}}
\end{frame}

\begin{frame}
\frametitle{Basic Process}
\begin{center}
\begin{tikzpicture}
\node[shape=rectangle,fill=pantone127-4] (problem) at (2,5) {Problem};
\node (human) at (2,4) {Human};
\node[shape=rectangle,fill=pantone127-4] (model) at (2,3) {Model};
\node (solver) at (2,2) {Constraint Solver/Search};
\node[shape=rectangle,fill=pantone157-8] (solution) at (2,1) {Solution};
\draw[-] (problem) -- (human);
\draw[->] (human) -- (model);
\draw[-] (model) -- (solver);
\draw[->] (solver) -- (solution);
\end{tikzpicture}
\end{center}
\end{frame}

\begin{frame}
\frametitle{More Realistic}
\begin{center}
\begin{tikzpicture}[xscale=4.0]
\node[shape=rectangle,fill=pantone127-4] (problem) at (2,5) {Problem};
\node (human) at (2,4) {Human};
\node[shape=rectangle,fill=pantone127-4] (model) at (2,3) {Model};
\node (solver) at (2,2) {Constraint Solver/Search};
\node[shape=rectangle,fill=pantone24-8] (hangs) at (1,1) {Hangs};
\node[shape=rectangle,fill=pantone157-8] (solution) at (2,1) {Solution};
\node[shape=rectangle,fill=pantone24-8] (wrong) at (3,1) {Wrong Solution};
\draw[-] (problem) -- (human);
\draw[->] (human) -- (model);
\draw[-] (model) -- (solver);
\draw[->] (solver) -- (hangs);
\draw[->] (solver) -- (solution);
\draw[->] (solver) -- (wrong);
\draw[->] (hangs) -- (1,4) -- (human);
\draw[->] (wrong) -- (3,4) -- (human);
\end{tikzpicture}
\end{center}
\end{frame}

\begin{frame}
\frametitle{Dual Role of Model}
\begin{itemize}
\item Allows Human to Express Problem
\begin{itemize}
\item Close to Problem Domain
\item Constraints as Abstractions
\end{itemize}
\item Allows Solver to Execute
\begin{itemize}
\item Variables as Communication Mechanism
\item Constraints as Algorithms
\end{itemize}
\end{itemize}
\end{frame}

\begin{frame}
\frametitle{Modelling Frameworks}
\begin{itemize}
\item \href{https://www.minizinc.org/}{MiniZinc} (NICTA, Monash University, Australia)
\item \href{https://github.com/eomahony/Numberjack}{NumberJack} (Insight, Ireland) 
\item \href{https://www-users.york.ac.uk/peter.nightingale/savilerow//}{EssencePrime/SavilleRow} (UK) 
\item \href{https://cpmpy.readthedocs.io/en/latest/}{CPMpy} (KU Leuven)
\item Allow use of multiple back-end solvers
\item Compile model into variants for each solver
\item A priori solver independent model(CP, MIP, SAT)
\end{itemize}
\end{frame} 

\begin{frame}
\frametitle{Framework Process}
\begin{center}
\begin{tikzpicture}[xscale=2]
\node[shape=rectangle,fill=pantone127-4] (problem) at (2.5,5) {Problem};
\node (human) at (2.5,4) {Human};
\node[shape=rectangle,fill=pantone127-4] (model) at (2.5,3) {Model};
\node (compile) at (2.5,2) {Compile/Reformulate};
\node (solver1) at (1,1) {CP};
\node (solver2) at (2,1) {MIP};
\node (solver3) at (3,1) {SAT};
\node (solver4) at (4,1) {Other};
\node[shape=rectangle,fill=pantone157-8] (solution1) at (1,0) {Solution};
\node[shape=rectangle,fill=pantone157-8] (solution2) at (2,0) {Solution};
\node[shape=rectangle,fill=pantone157-8] (solution3) at (3,0) {Solution};
\node[shape=rectangle,fill=pantone157-8] (solution4) at (4,0) {Solution};
\draw[-] (problem) -- (human);
\draw[->] (human) -- (model);
\draw[-] (model) -- (compile);
\draw[-] (compile) -- (solver1);
\draw[-] (compile) -- (solver2);
\draw[-] (compile) -- (solver3);
\draw[-] (compile) -- (solver4);
\draw[->] (solver1) -- (solution1);
\draw[->] (solver2) -- (solution2);
\draw[->] (solver3) -- (solution3);
\draw[->] (solver4) -- (solution4);
\end{tikzpicture}
\end{center}
\end{frame}

% \begin{frame}
  % \frametitle{Do It Now!}
  % \begin{itemize}
  % \item Download and install Minizinc
  % \item \url{https://www.minizinc.org/}
  % \end{itemize}
% \end{frame}


\begin{frame}
\frametitle{Why use Puzzles as Examples?}
\begin{itemize}
\item Easy to understand the problem
\item Solvable by hand without specialized knowledge
\item Possible to compare automated to manual solving process
\end{itemize}

\begin{quote}
The puzzle, though inanimate, is presented as a solvable problem without lasting negative consequences, a very low-risk low-reward situation. By being a puzzle, the object is attempting to convince the user that it must be completed.
\end{quote}
{\tiny Source: \href{https://www.everydayrhetoric.com/2020/03/the-rhetoric-of-puzzles/}{Every Day Rhetoric}}
\end{frame}


\mode<all>{
\part{Basic Constraint Propagation}
}
\begin{frame}
\partpage
\end{frame}
\mode<all>{
\input{../sendmore/body}
}

\mode<all>{
\part{Global Constraints}
}
\begin{frame}
\partpage
\end{frame}
\mode<all>{
\input{../sudoku/body}
}

\clearpage
\mode<all>{
\part{Customizing Search}
}
\begin{frame}
\partpage
\end{frame}
\mode<all>{
\input{../nqueen/body}
}


%% \clearpage
%% \mode<all>{
%%   \part{Problem Decomposition}
%% }
%% \begin{frame}
%%   \partpage
%% \end{frame}
%% \mode<all>{
%%   \input{../party/body}
%%   }

%% \clearpage
%% \mode<all>{
%%   \part{More Global Constraints}
%% }
%% \begin{frame}
%%   \partpage
%% \end{frame}
%% \mode<all>{
%%   \input{../car/body}
%%   }

%% \clearpage
%% \mode<all>{
%%   \part{Methodology}
%% }
%% \begin{frame}
%%   \partpage
%% \end{frame}
%% \mode<all>{
%%   \input{../methodology/body}
%%   }

\clearpage
\mode<all>{
\part{What is missing?}
}
\begin{frame}
\partpage
\end{frame}

\begin{frame}
\frametitle{Many Specialized Topics}
\begin{itemize}
\item How to design efficient core engine
\item Hybrids with LP/MIP tools
\item Hybrids with SAT
\item Symmetry breaking
\item Use of MDD/BDD to encode sets of solutions
\item High level modelling tools
\item Debugging/visualization
\end{itemize}
\end{frame}

\begin{frame}
\frametitle{Reformulation}
\begin{itemize}
\item Just because the user has modelled it this way, it doesn't mean we have to solve it that way
\begin{itemize}
\item Replace some constraint(s) by other, equivalent constraints
\item Because we don't have that constraint in our system
\item For performance
\end{itemize}
\end{itemize}
\end{frame}

\begin{frame}
\frametitle{Learning}
\begin{itemize}
\item While solving the problem we can learn how to strengthen the model/search 
\begin{itemize}
\item Understand which constraints/method contribute to propagation and change schedule
\item Learn no-good constraints by explaining failure
\item Adapt search strategy based on search experience
\end{itemize}
\end{itemize}
\end{frame}

%% \begin{frame}
%% \frametitle{Refined Process}
%% \begin{center}
%% \begin{tikzpicture}
%% \node[shape=rectangle,fill=pantone127-4] (problem) at (3,7) {Problem};
%% \node (human) at (3,6) {Human};
%% \node[shape=rectangle,fill=pantone127-4] (model) at (3,5) {User Model};
%% \node (reformulation) at (3,4) {Reformulation};
%% \node[shape=rectangle,fill=pantone127-4] (implementation) at (3,3) {Implementation Model};
%% \node (solver) at (3,2) {Constraint Solver/Search};
%% \node[shape=rectangle,fill=pantone157-8] (solution) at (3,1) {Solution};
%% \draw[-] (problem) -- (human);
%% \draw[->] (human) -- (model);
%% \draw[-] (model) -- (reformulation);
%% \draw[->] (reformulation) -- (implementation);
%% \draw[-] (implementation) -- (solver);
%% \draw[->] (solver) -- (0,2) -- (0,3) -- (implementation);
%% \draw[->] (solver) -- (solution);
%% \end{tikzpicture}
%% \end{center}
%% \end{frame}

%\begin{frame}
%\frametitle{What is CP actually used for?}
%\url{http://hsimonis.wordpress.com}
%\includegraphics[width=10cm]{../introduction/images/blog}
%\end{frame}

%% \begin{frame}
%% \frametitle{Exercises}
%% \begin{itemize}
%% \item Install Constraint Solver of your Choice
%% \item MiniZinc (MiningZinc) will be used in later sessions
%% \item Run Sendmoremoney, Sudoku, Queens examples
%% \item Can you choose which propagator to use for alldifferent?
%% \item Can you use different search strategies?
%% \end{itemize}
%% \end{frame}

\begin{frame}
\frametitle{More Learning Resources}
\begin{itemize}
\item Survey of Methods, Resources, and Formats for Teaching Constraint Programming
\begin{itemize}
\item by Tejas Santanam, Helmut Simonis
\item \url{https://doi.org/10.48550/arXiv.2403.12717}
\item Based on survey of community for WTCP 2023
\item \url{https://hsimonis.github.io/WTCP2023/}
\end{itemize}
\end{itemize}
\end{frame}


%%   }

%% \clearpage
%% \mode<all>{
%%   \part{Methodology}
%% }
%% \begin{frame}
%%   \partpage
%% \end{frame}
%% \mode<all>{
%%   \begin{frame}
\frametitle{Objectives}
\begin{itemize}
\item Overview of Core Constraint Programming
\item Three Main Concepts
\begin{itemize}
\item Constraint Propagation
\item Global Constraints
\item Customizing Search
\end{itemize}
%\item Get Some Experience with MiniZinc
\item Topics will be treated in more detail in later parts of the school
\item Based on Examples, not Formal Description
\end{itemize}
\end{frame}

\begin{frame}
\frametitle{Outline}
\begin{itemize}
\item Why Constraint Programming?
\item Constraint Propagation
\item Global Constraints
\item Customizing Search
%\item Choices in Modelling
%\item Decomposition
%\item More Global Constraints
%\item Methodology
%\item What is missing?
\end{itemize}
\end{frame}

% \begin{frame}
  % \frametitle{Using MiniZinc IDE}
  % \begin{itemize}
  % \item Developed in the Australian NICTA project
  % \item Maintained by Monash University
  % \item Modelling tool with multiple back-end solvers
  % \item Available from \url{https://www.minizinc.org/}
  % \end{itemize}
% \end{frame}


% \begin{frame}
% \frametitle{Examples in ECLiPSe}
% \begin{itemize}
% \item Open sourced constraint programming language
% \item Development goes back to 1985
% \item ECRC, ICL, IC-Parc, PTL, Cisco
% \item \url{https://eclipseclp.org/}
% \item Specialities
% \begin{itemize}
% \item Develop new solvers for specific domains
% \item Integration with MIP
% \end{itemize}
% \item Not included in bundled MiniZinc IDE
% \item Specialized visualization tools used here
  % \begin{itemize}
  % \item CP-Viz, Simonis et al. 2010
    % \end{itemize}
% \end{itemize}
% \end{frame}

\begin{frame}
\frametitle{Tutorial Based on ECLiPSe ELearning Course}
\begin{itemize}
\item Self-study course in constraint programming
\item Supported by Cisco Systems and Silicon Valley Community Foundation
\item Multi-media format, video lectures, slides, handout etc
\item {\small \url{https://eclipseclp.org/ELearning/index.html}}
\end{itemize}
\end{frame}

\begin{frame}
\frametitle{Also Part of CRT-AI Constraint Week}
\begin{itemize}
\item Annual one week course on CP and Optimization in Ireland
\item Part of national training program for PhD students in AI
\item \url{https://www.crt-ai.ie/}
\end{itemize}
\end{frame}



\begin{frame}
\frametitle{Constraint Programming - in a nutshell}
\begin{itemize}
\item Declarative description of problems with
\begin{itemize}
\item {\em Variables} which range over (finite) sets of values
\item {\em Constraints} over subsets of variables which restrict possible value combinations
\item A {\em solution} is a value assignment which satisfies all constraints
\end{itemize}

\item Constraint propagation/reasoning
\begin{itemize}
\item Removing inconsistent values for variables
\item Detect failure if constraint can not be satisfied
\item Interaction of constraints via shared variables
\item Incomplete
\end{itemize}

\item Search
\begin{itemize}
\item User controlled assignment of values to variables
\item Each step triggers constraint propagation 
\end{itemize}
\item Different domains require/allow different methods
\end{itemize}
\end{frame}

\begin{frame}
  \frametitle{Constraint Programming is Different}
  \begin{itemize}
  \item Declarative Programming
    \begin{itemize}
    \item Concentrate on what you want
      \item Not how to get there
      \item Program != Algorithm
      \item Program = Model
    \end{itemize}
    \item Applied to Combinatorial Problems
      \begin{itemize}
        \item No complete polynomial algorithms known (exist?)
        \item CP less ad-hoc than heuristics
        \item Models can evolve
  \end{itemize}
  \end{itemize}
  \end{frame}
    
\begin{frame}
  \frametitle{A Subtractive Process}
  \begin{textblock}{4}(8,-3)
    \includegraphics[width=4cm]{../introduction/images/stages}
  \end{textblock}
  \vfill
  \begin{quote}
    ``Oh, bosh, as Mr. Ruskin says. Sculpture, per se, is the simplest thing in the world. All you have to do is to take a big chunk of marble and a hammer and chisel, make up your mind what you are about to create and chip off all the marble you don't want.''-Paris Gaulois.
  \end{quote}
  
  {\tiny Source: \url{https://quoteinvestigator.com/2014/06/22/chip-away/}}
\end{frame}

\begin{frame}
\frametitle{Basic Process}
\begin{center}
\begin{tikzpicture}
\node[shape=rectangle,fill=pantone127-4] (problem) at (2,5) {Problem};
\node (human) at (2,4) {Human};
\node[shape=rectangle,fill=pantone127-4] (model) at (2,3) {Model};
\node (solver) at (2,2) {Constraint Solver/Search};
\node[shape=rectangle,fill=pantone157-8] (solution) at (2,1) {Solution};
\draw[-] (problem) -- (human);
\draw[->] (human) -- (model);
\draw[-] (model) -- (solver);
\draw[->] (solver) -- (solution);
\end{tikzpicture}
\end{center}
\end{frame}

\begin{frame}
\frametitle{More Realistic}
\begin{center}
\begin{tikzpicture}[xscale=4.0]
\node[shape=rectangle,fill=pantone127-4] (problem) at (2,5) {Problem};
\node (human) at (2,4) {Human};
\node[shape=rectangle,fill=pantone127-4] (model) at (2,3) {Model};
\node (solver) at (2,2) {Constraint Solver/Search};
\node[shape=rectangle,fill=pantone24-8] (hangs) at (1,1) {Hangs};
\node[shape=rectangle,fill=pantone157-8] (solution) at (2,1) {Solution};
\node[shape=rectangle,fill=pantone24-8] (wrong) at (3,1) {Wrong Solution};
\draw[-] (problem) -- (human);
\draw[->] (human) -- (model);
\draw[-] (model) -- (solver);
\draw[->] (solver) -- (hangs);
\draw[->] (solver) -- (solution);
\draw[->] (solver) -- (wrong);
\draw[->] (hangs) -- (1,4) -- (human);
\draw[->] (wrong) -- (3,4) -- (human);
\end{tikzpicture}
\end{center}
\end{frame}

\begin{frame}
\frametitle{Dual Role of Model}
\begin{itemize}
\item Allows Human to Express Problem
\begin{itemize}
\item Close to Problem Domain
\item Constraints as Abstractions
\end{itemize}
\item Allows Solver to Execute
\begin{itemize}
\item Variables as Communication Mechanism
\item Constraints as Algorithms
\end{itemize}
\end{itemize}
\end{frame}

\begin{frame}
\frametitle{Modelling Frameworks}
\begin{itemize}
\item \href{https://www.minizinc.org/}{MiniZinc} (NICTA, Monash University, Australia)
\item \href{https://github.com/eomahony/Numberjack}{NumberJack} (Insight, Ireland) 
\item \href{https://www-users.york.ac.uk/peter.nightingale/savilerow//}{EssencePrime/SavilleRow} (UK) 
\item \href{https://cpmpy.readthedocs.io/en/latest/}{CPMpy} (KU Leuven)
\item Allow use of multiple back-end solvers
\item Compile model into variants for each solver
\item A priori solver independent model(CP, MIP, SAT)
\end{itemize}
\end{frame} 

\begin{frame}
\frametitle{Framework Process}
\begin{center}
\begin{tikzpicture}[xscale=2]
\node[shape=rectangle,fill=pantone127-4] (problem) at (2.5,5) {Problem};
\node (human) at (2.5,4) {Human};
\node[shape=rectangle,fill=pantone127-4] (model) at (2.5,3) {Model};
\node (compile) at (2.5,2) {Compile/Reformulate};
\node (solver1) at (1,1) {CP};
\node (solver2) at (2,1) {MIP};
\node (solver3) at (3,1) {SAT};
\node (solver4) at (4,1) {Other};
\node[shape=rectangle,fill=pantone157-8] (solution1) at (1,0) {Solution};
\node[shape=rectangle,fill=pantone157-8] (solution2) at (2,0) {Solution};
\node[shape=rectangle,fill=pantone157-8] (solution3) at (3,0) {Solution};
\node[shape=rectangle,fill=pantone157-8] (solution4) at (4,0) {Solution};
\draw[-] (problem) -- (human);
\draw[->] (human) -- (model);
\draw[-] (model) -- (compile);
\draw[-] (compile) -- (solver1);
\draw[-] (compile) -- (solver2);
\draw[-] (compile) -- (solver3);
\draw[-] (compile) -- (solver4);
\draw[->] (solver1) -- (solution1);
\draw[->] (solver2) -- (solution2);
\draw[->] (solver3) -- (solution3);
\draw[->] (solver4) -- (solution4);
\end{tikzpicture}
\end{center}
\end{frame}

% \begin{frame}
  % \frametitle{Do It Now!}
  % \begin{itemize}
  % \item Download and install Minizinc
  % \item \url{https://www.minizinc.org/}
  % \end{itemize}
% \end{frame}


\begin{frame}
\frametitle{Why use Puzzles as Examples?}
\begin{itemize}
\item Easy to understand the problem
\item Solvable by hand without specialized knowledge
\item Possible to compare automated to manual solving process
\end{itemize}

\begin{quote}
The puzzle, though inanimate, is presented as a solvable problem without lasting negative consequences, a very low-risk low-reward situation. By being a puzzle, the object is attempting to convince the user that it must be completed.
\end{quote}
{\tiny Source: \href{https://www.everydayrhetoric.com/2020/03/the-rhetoric-of-puzzles/}{Every Day Rhetoric}}
\end{frame}


\mode<all>{
\part{Basic Constraint Propagation}
}
\begin{frame}
\partpage
\end{frame}
\mode<all>{
\input{../sendmore/body}
}

\mode<all>{
\part{Global Constraints}
}
\begin{frame}
\partpage
\end{frame}
\mode<all>{
\input{../sudoku/body}
}

\clearpage
\mode<all>{
\part{Customizing Search}
}
\begin{frame}
\partpage
\end{frame}
\mode<all>{
\input{../nqueen/body}
}


%% \clearpage
%% \mode<all>{
%%   \part{Problem Decomposition}
%% }
%% \begin{frame}
%%   \partpage
%% \end{frame}
%% \mode<all>{
%%   \input{../party/body}
%%   }

%% \clearpage
%% \mode<all>{
%%   \part{More Global Constraints}
%% }
%% \begin{frame}
%%   \partpage
%% \end{frame}
%% \mode<all>{
%%   \input{../car/body}
%%   }

%% \clearpage
%% \mode<all>{
%%   \part{Methodology}
%% }
%% \begin{frame}
%%   \partpage
%% \end{frame}
%% \mode<all>{
%%   \input{../methodology/body}
%%   }

\clearpage
\mode<all>{
\part{What is missing?}
}
\begin{frame}
\partpage
\end{frame}

\begin{frame}
\frametitle{Many Specialized Topics}
\begin{itemize}
\item How to design efficient core engine
\item Hybrids with LP/MIP tools
\item Hybrids with SAT
\item Symmetry breaking
\item Use of MDD/BDD to encode sets of solutions
\item High level modelling tools
\item Debugging/visualization
\end{itemize}
\end{frame}

\begin{frame}
\frametitle{Reformulation}
\begin{itemize}
\item Just because the user has modelled it this way, it doesn't mean we have to solve it that way
\begin{itemize}
\item Replace some constraint(s) by other, equivalent constraints
\item Because we don't have that constraint in our system
\item For performance
\end{itemize}
\end{itemize}
\end{frame}

\begin{frame}
\frametitle{Learning}
\begin{itemize}
\item While solving the problem we can learn how to strengthen the model/search 
\begin{itemize}
\item Understand which constraints/method contribute to propagation and change schedule
\item Learn no-good constraints by explaining failure
\item Adapt search strategy based on search experience
\end{itemize}
\end{itemize}
\end{frame}

%% \begin{frame}
%% \frametitle{Refined Process}
%% \begin{center}
%% \begin{tikzpicture}
%% \node[shape=rectangle,fill=pantone127-4] (problem) at (3,7) {Problem};
%% \node (human) at (3,6) {Human};
%% \node[shape=rectangle,fill=pantone127-4] (model) at (3,5) {User Model};
%% \node (reformulation) at (3,4) {Reformulation};
%% \node[shape=rectangle,fill=pantone127-4] (implementation) at (3,3) {Implementation Model};
%% \node (solver) at (3,2) {Constraint Solver/Search};
%% \node[shape=rectangle,fill=pantone157-8] (solution) at (3,1) {Solution};
%% \draw[-] (problem) -- (human);
%% \draw[->] (human) -- (model);
%% \draw[-] (model) -- (reformulation);
%% \draw[->] (reformulation) -- (implementation);
%% \draw[-] (implementation) -- (solver);
%% \draw[->] (solver) -- (0,2) -- (0,3) -- (implementation);
%% \draw[->] (solver) -- (solution);
%% \end{tikzpicture}
%% \end{center}
%% \end{frame}

%\begin{frame}
%\frametitle{What is CP actually used for?}
%\url{http://hsimonis.wordpress.com}
%\includegraphics[width=10cm]{../introduction/images/blog}
%\end{frame}

%% \begin{frame}
%% \frametitle{Exercises}
%% \begin{itemize}
%% \item Install Constraint Solver of your Choice
%% \item MiniZinc (MiningZinc) will be used in later sessions
%% \item Run Sendmoremoney, Sudoku, Queens examples
%% \item Can you choose which propagator to use for alldifferent?
%% \item Can you use different search strategies?
%% \end{itemize}
%% \end{frame}

\begin{frame}
\frametitle{More Learning Resources}
\begin{itemize}
\item Survey of Methods, Resources, and Formats for Teaching Constraint Programming
\begin{itemize}
\item by Tejas Santanam, Helmut Simonis
\item \url{https://doi.org/10.48550/arXiv.2403.12717}
\item Based on survey of community for WTCP 2023
\item \url{https://hsimonis.github.io/WTCP2023/}
\end{itemize}
\end{itemize}
\end{frame}


%%   }

\clearpage
\mode<all>{
\part{What is missing?}
}
\begin{frame}
\partpage
\end{frame}

\begin{frame}
\frametitle{Many Specialized Topics}
\begin{itemize}
\item How to design efficient core engine
\item Hybrids with LP/MIP tools
\item Hybrids with SAT
\item Symmetry breaking
\item Use of MDD/BDD to encode sets of solutions
\item High level modelling tools
\item Debugging/visualization
\end{itemize}
\end{frame}

\begin{frame}
\frametitle{Reformulation}
\begin{itemize}
\item Just because the user has modelled it this way, it doesn't mean we have to solve it that way
\begin{itemize}
\item Replace some constraint(s) by other, equivalent constraints
\item Because we don't have that constraint in our system
\item For performance
\end{itemize}
\end{itemize}
\end{frame}

\begin{frame}
\frametitle{Learning}
\begin{itemize}
\item While solving the problem we can learn how to strengthen the model/search 
\begin{itemize}
\item Understand which constraints/method contribute to propagation and change schedule
\item Learn no-good constraints by explaining failure
\item Adapt search strategy based on search experience
\end{itemize}
\end{itemize}
\end{frame}

%% \begin{frame}
%% \frametitle{Refined Process}
%% \begin{center}
%% \begin{tikzpicture}
%% \node[shape=rectangle,fill=pantone127-4] (problem) at (3,7) {Problem};
%% \node (human) at (3,6) {Human};
%% \node[shape=rectangle,fill=pantone127-4] (model) at (3,5) {User Model};
%% \node (reformulation) at (3,4) {Reformulation};
%% \node[shape=rectangle,fill=pantone127-4] (implementation) at (3,3) {Implementation Model};
%% \node (solver) at (3,2) {Constraint Solver/Search};
%% \node[shape=rectangle,fill=pantone157-8] (solution) at (3,1) {Solution};
%% \draw[-] (problem) -- (human);
%% \draw[->] (human) -- (model);
%% \draw[-] (model) -- (reformulation);
%% \draw[->] (reformulation) -- (implementation);
%% \draw[-] (implementation) -- (solver);
%% \draw[->] (solver) -- (0,2) -- (0,3) -- (implementation);
%% \draw[->] (solver) -- (solution);
%% \end{tikzpicture}
%% \end{center}
%% \end{frame}

%\begin{frame}
%\frametitle{What is CP actually used for?}
%\url{http://hsimonis.wordpress.com}
%\includegraphics[width=10cm]{../introduction/images/blog}
%\end{frame}

%% \begin{frame}
%% \frametitle{Exercises}
%% \begin{itemize}
%% \item Install Constraint Solver of your Choice
%% \item MiniZinc (MiningZinc) will be used in later sessions
%% \item Run Sendmoremoney, Sudoku, Queens examples
%% \item Can you choose which propagator to use for alldifferent?
%% \item Can you use different search strategies?
%% \end{itemize}
%% \end{frame}

\begin{frame}
\frametitle{More Learning Resources}
\begin{itemize}
\item Survey of Methods, Resources, and Formats for Teaching Constraint Programming
\begin{itemize}
\item by Tejas Santanam, Helmut Simonis
\item \url{https://doi.org/10.48550/arXiv.2403.12717}
\item Based on survey of community for WTCP 2023
\item \url{https://hsimonis.github.io/WTCP2023/}
\end{itemize}
\end{itemize}
\end{frame}


}

\clearpage
\mode<all>{
\part{Customizing Search}
}
\begin{frame}
\partpage
\end{frame}
\mode<all>{
\begin{frame}
\frametitle{Objectives}
\begin{itemize}
\item Overview of Core Constraint Programming
\item Three Main Concepts
\begin{itemize}
\item Constraint Propagation
\item Global Constraints
\item Customizing Search
\end{itemize}
%\item Get Some Experience with MiniZinc
\item Topics will be treated in more detail in later parts of the school
\item Based on Examples, not Formal Description
\end{itemize}
\end{frame}

\begin{frame}
\frametitle{Outline}
\begin{itemize}
\item Why Constraint Programming?
\item Constraint Propagation
\item Global Constraints
\item Customizing Search
%\item Choices in Modelling
%\item Decomposition
%\item More Global Constraints
%\item Methodology
%\item What is missing?
\end{itemize}
\end{frame}

% \begin{frame}
  % \frametitle{Using MiniZinc IDE}
  % \begin{itemize}
  % \item Developed in the Australian NICTA project
  % \item Maintained by Monash University
  % \item Modelling tool with multiple back-end solvers
  % \item Available from \url{https://www.minizinc.org/}
  % \end{itemize}
% \end{frame}


% \begin{frame}
% \frametitle{Examples in ECLiPSe}
% \begin{itemize}
% \item Open sourced constraint programming language
% \item Development goes back to 1985
% \item ECRC, ICL, IC-Parc, PTL, Cisco
% \item \url{https://eclipseclp.org/}
% \item Specialities
% \begin{itemize}
% \item Develop new solvers for specific domains
% \item Integration with MIP
% \end{itemize}
% \item Not included in bundled MiniZinc IDE
% \item Specialized visualization tools used here
  % \begin{itemize}
  % \item CP-Viz, Simonis et al. 2010
    % \end{itemize}
% \end{itemize}
% \end{frame}

\begin{frame}
\frametitle{Tutorial Based on ECLiPSe ELearning Course}
\begin{itemize}
\item Self-study course in constraint programming
\item Supported by Cisco Systems and Silicon Valley Community Foundation
\item Multi-media format, video lectures, slides, handout etc
\item {\small \url{https://eclipseclp.org/ELearning/index.html}}
\end{itemize}
\end{frame}

\begin{frame}
\frametitle{Also Part of CRT-AI Constraint Week}
\begin{itemize}
\item Annual one week course on CP and Optimization in Ireland
\item Part of national training program for PhD students in AI
\item \url{https://www.crt-ai.ie/}
\end{itemize}
\end{frame}



\begin{frame}
\frametitle{Constraint Programming - in a nutshell}
\begin{itemize}
\item Declarative description of problems with
\begin{itemize}
\item {\em Variables} which range over (finite) sets of values
\item {\em Constraints} over subsets of variables which restrict possible value combinations
\item A {\em solution} is a value assignment which satisfies all constraints
\end{itemize}

\item Constraint propagation/reasoning
\begin{itemize}
\item Removing inconsistent values for variables
\item Detect failure if constraint can not be satisfied
\item Interaction of constraints via shared variables
\item Incomplete
\end{itemize}

\item Search
\begin{itemize}
\item User controlled assignment of values to variables
\item Each step triggers constraint propagation 
\end{itemize}
\item Different domains require/allow different methods
\end{itemize}
\end{frame}

\begin{frame}
  \frametitle{Constraint Programming is Different}
  \begin{itemize}
  \item Declarative Programming
    \begin{itemize}
    \item Concentrate on what you want
      \item Not how to get there
      \item Program != Algorithm
      \item Program = Model
    \end{itemize}
    \item Applied to Combinatorial Problems
      \begin{itemize}
        \item No complete polynomial algorithms known (exist?)
        \item CP less ad-hoc than heuristics
        \item Models can evolve
  \end{itemize}
  \end{itemize}
  \end{frame}
    
\begin{frame}
  \frametitle{A Subtractive Process}
  \begin{textblock}{4}(8,-3)
    \includegraphics[width=4cm]{../introduction/images/stages}
  \end{textblock}
  \vfill
  \begin{quote}
    ``Oh, bosh, as Mr. Ruskin says. Sculpture, per se, is the simplest thing in the world. All you have to do is to take a big chunk of marble and a hammer and chisel, make up your mind what you are about to create and chip off all the marble you don't want.''-Paris Gaulois.
  \end{quote}
  
  {\tiny Source: \url{https://quoteinvestigator.com/2014/06/22/chip-away/}}
\end{frame}

\begin{frame}
\frametitle{Basic Process}
\begin{center}
\begin{tikzpicture}
\node[shape=rectangle,fill=pantone127-4] (problem) at (2,5) {Problem};
\node (human) at (2,4) {Human};
\node[shape=rectangle,fill=pantone127-4] (model) at (2,3) {Model};
\node (solver) at (2,2) {Constraint Solver/Search};
\node[shape=rectangle,fill=pantone157-8] (solution) at (2,1) {Solution};
\draw[-] (problem) -- (human);
\draw[->] (human) -- (model);
\draw[-] (model) -- (solver);
\draw[->] (solver) -- (solution);
\end{tikzpicture}
\end{center}
\end{frame}

\begin{frame}
\frametitle{More Realistic}
\begin{center}
\begin{tikzpicture}[xscale=4.0]
\node[shape=rectangle,fill=pantone127-4] (problem) at (2,5) {Problem};
\node (human) at (2,4) {Human};
\node[shape=rectangle,fill=pantone127-4] (model) at (2,3) {Model};
\node (solver) at (2,2) {Constraint Solver/Search};
\node[shape=rectangle,fill=pantone24-8] (hangs) at (1,1) {Hangs};
\node[shape=rectangle,fill=pantone157-8] (solution) at (2,1) {Solution};
\node[shape=rectangle,fill=pantone24-8] (wrong) at (3,1) {Wrong Solution};
\draw[-] (problem) -- (human);
\draw[->] (human) -- (model);
\draw[-] (model) -- (solver);
\draw[->] (solver) -- (hangs);
\draw[->] (solver) -- (solution);
\draw[->] (solver) -- (wrong);
\draw[->] (hangs) -- (1,4) -- (human);
\draw[->] (wrong) -- (3,4) -- (human);
\end{tikzpicture}
\end{center}
\end{frame}

\begin{frame}
\frametitle{Dual Role of Model}
\begin{itemize}
\item Allows Human to Express Problem
\begin{itemize}
\item Close to Problem Domain
\item Constraints as Abstractions
\end{itemize}
\item Allows Solver to Execute
\begin{itemize}
\item Variables as Communication Mechanism
\item Constraints as Algorithms
\end{itemize}
\end{itemize}
\end{frame}

\begin{frame}
\frametitle{Modelling Frameworks}
\begin{itemize}
\item \href{https://www.minizinc.org/}{MiniZinc} (NICTA, Monash University, Australia)
\item \href{https://github.com/eomahony/Numberjack}{NumberJack} (Insight, Ireland) 
\item \href{https://www-users.york.ac.uk/peter.nightingale/savilerow//}{EssencePrime/SavilleRow} (UK) 
\item \href{https://cpmpy.readthedocs.io/en/latest/}{CPMpy} (KU Leuven)
\item Allow use of multiple back-end solvers
\item Compile model into variants for each solver
\item A priori solver independent model(CP, MIP, SAT)
\end{itemize}
\end{frame} 

\begin{frame}
\frametitle{Framework Process}
\begin{center}
\begin{tikzpicture}[xscale=2]
\node[shape=rectangle,fill=pantone127-4] (problem) at (2.5,5) {Problem};
\node (human) at (2.5,4) {Human};
\node[shape=rectangle,fill=pantone127-4] (model) at (2.5,3) {Model};
\node (compile) at (2.5,2) {Compile/Reformulate};
\node (solver1) at (1,1) {CP};
\node (solver2) at (2,1) {MIP};
\node (solver3) at (3,1) {SAT};
\node (solver4) at (4,1) {Other};
\node[shape=rectangle,fill=pantone157-8] (solution1) at (1,0) {Solution};
\node[shape=rectangle,fill=pantone157-8] (solution2) at (2,0) {Solution};
\node[shape=rectangle,fill=pantone157-8] (solution3) at (3,0) {Solution};
\node[shape=rectangle,fill=pantone157-8] (solution4) at (4,0) {Solution};
\draw[-] (problem) -- (human);
\draw[->] (human) -- (model);
\draw[-] (model) -- (compile);
\draw[-] (compile) -- (solver1);
\draw[-] (compile) -- (solver2);
\draw[-] (compile) -- (solver3);
\draw[-] (compile) -- (solver4);
\draw[->] (solver1) -- (solution1);
\draw[->] (solver2) -- (solution2);
\draw[->] (solver3) -- (solution3);
\draw[->] (solver4) -- (solution4);
\end{tikzpicture}
\end{center}
\end{frame}

% \begin{frame}
  % \frametitle{Do It Now!}
  % \begin{itemize}
  % \item Download and install Minizinc
  % \item \url{https://www.minizinc.org/}
  % \end{itemize}
% \end{frame}


\begin{frame}
\frametitle{Why use Puzzles as Examples?}
\begin{itemize}
\item Easy to understand the problem
\item Solvable by hand without specialized knowledge
\item Possible to compare automated to manual solving process
\end{itemize}

\begin{quote}
The puzzle, though inanimate, is presented as a solvable problem without lasting negative consequences, a very low-risk low-reward situation. By being a puzzle, the object is attempting to convince the user that it must be completed.
\end{quote}
{\tiny Source: \href{https://www.everydayrhetoric.com/2020/03/the-rhetoric-of-puzzles/}{Every Day Rhetoric}}
\end{frame}


\mode<all>{
\part{Basic Constraint Propagation}
}
\begin{frame}
\partpage
\end{frame}
\mode<all>{
\begin{frame}
\frametitle{Objectives}
\begin{itemize}
\item Overview of Core Constraint Programming
\item Three Main Concepts
\begin{itemize}
\item Constraint Propagation
\item Global Constraints
\item Customizing Search
\end{itemize}
%\item Get Some Experience with MiniZinc
\item Topics will be treated in more detail in later parts of the school
\item Based on Examples, not Formal Description
\end{itemize}
\end{frame}

\begin{frame}
\frametitle{Outline}
\begin{itemize}
\item Why Constraint Programming?
\item Constraint Propagation
\item Global Constraints
\item Customizing Search
%\item Choices in Modelling
%\item Decomposition
%\item More Global Constraints
%\item Methodology
%\item What is missing?
\end{itemize}
\end{frame}

% \begin{frame}
  % \frametitle{Using MiniZinc IDE}
  % \begin{itemize}
  % \item Developed in the Australian NICTA project
  % \item Maintained by Monash University
  % \item Modelling tool with multiple back-end solvers
  % \item Available from \url{https://www.minizinc.org/}
  % \end{itemize}
% \end{frame}


% \begin{frame}
% \frametitle{Examples in ECLiPSe}
% \begin{itemize}
% \item Open sourced constraint programming language
% \item Development goes back to 1985
% \item ECRC, ICL, IC-Parc, PTL, Cisco
% \item \url{https://eclipseclp.org/}
% \item Specialities
% \begin{itemize}
% \item Develop new solvers for specific domains
% \item Integration with MIP
% \end{itemize}
% \item Not included in bundled MiniZinc IDE
% \item Specialized visualization tools used here
  % \begin{itemize}
  % \item CP-Viz, Simonis et al. 2010
    % \end{itemize}
% \end{itemize}
% \end{frame}

\begin{frame}
\frametitle{Tutorial Based on ECLiPSe ELearning Course}
\begin{itemize}
\item Self-study course in constraint programming
\item Supported by Cisco Systems and Silicon Valley Community Foundation
\item Multi-media format, video lectures, slides, handout etc
\item {\small \url{https://eclipseclp.org/ELearning/index.html}}
\end{itemize}
\end{frame}

\begin{frame}
\frametitle{Also Part of CRT-AI Constraint Week}
\begin{itemize}
\item Annual one week course on CP and Optimization in Ireland
\item Part of national training program for PhD students in AI
\item \url{https://www.crt-ai.ie/}
\end{itemize}
\end{frame}



\begin{frame}
\frametitle{Constraint Programming - in a nutshell}
\begin{itemize}
\item Declarative description of problems with
\begin{itemize}
\item {\em Variables} which range over (finite) sets of values
\item {\em Constraints} over subsets of variables which restrict possible value combinations
\item A {\em solution} is a value assignment which satisfies all constraints
\end{itemize}

\item Constraint propagation/reasoning
\begin{itemize}
\item Removing inconsistent values for variables
\item Detect failure if constraint can not be satisfied
\item Interaction of constraints via shared variables
\item Incomplete
\end{itemize}

\item Search
\begin{itemize}
\item User controlled assignment of values to variables
\item Each step triggers constraint propagation 
\end{itemize}
\item Different domains require/allow different methods
\end{itemize}
\end{frame}

\begin{frame}
  \frametitle{Constraint Programming is Different}
  \begin{itemize}
  \item Declarative Programming
    \begin{itemize}
    \item Concentrate on what you want
      \item Not how to get there
      \item Program != Algorithm
      \item Program = Model
    \end{itemize}
    \item Applied to Combinatorial Problems
      \begin{itemize}
        \item No complete polynomial algorithms known (exist?)
        \item CP less ad-hoc than heuristics
        \item Models can evolve
  \end{itemize}
  \end{itemize}
  \end{frame}
    
\begin{frame}
  \frametitle{A Subtractive Process}
  \begin{textblock}{4}(8,-3)
    \includegraphics[width=4cm]{../introduction/images/stages}
  \end{textblock}
  \vfill
  \begin{quote}
    ``Oh, bosh, as Mr. Ruskin says. Sculpture, per se, is the simplest thing in the world. All you have to do is to take a big chunk of marble and a hammer and chisel, make up your mind what you are about to create and chip off all the marble you don't want.''-Paris Gaulois.
  \end{quote}
  
  {\tiny Source: \url{https://quoteinvestigator.com/2014/06/22/chip-away/}}
\end{frame}

\begin{frame}
\frametitle{Basic Process}
\begin{center}
\begin{tikzpicture}
\node[shape=rectangle,fill=pantone127-4] (problem) at (2,5) {Problem};
\node (human) at (2,4) {Human};
\node[shape=rectangle,fill=pantone127-4] (model) at (2,3) {Model};
\node (solver) at (2,2) {Constraint Solver/Search};
\node[shape=rectangle,fill=pantone157-8] (solution) at (2,1) {Solution};
\draw[-] (problem) -- (human);
\draw[->] (human) -- (model);
\draw[-] (model) -- (solver);
\draw[->] (solver) -- (solution);
\end{tikzpicture}
\end{center}
\end{frame}

\begin{frame}
\frametitle{More Realistic}
\begin{center}
\begin{tikzpicture}[xscale=4.0]
\node[shape=rectangle,fill=pantone127-4] (problem) at (2,5) {Problem};
\node (human) at (2,4) {Human};
\node[shape=rectangle,fill=pantone127-4] (model) at (2,3) {Model};
\node (solver) at (2,2) {Constraint Solver/Search};
\node[shape=rectangle,fill=pantone24-8] (hangs) at (1,1) {Hangs};
\node[shape=rectangle,fill=pantone157-8] (solution) at (2,1) {Solution};
\node[shape=rectangle,fill=pantone24-8] (wrong) at (3,1) {Wrong Solution};
\draw[-] (problem) -- (human);
\draw[->] (human) -- (model);
\draw[-] (model) -- (solver);
\draw[->] (solver) -- (hangs);
\draw[->] (solver) -- (solution);
\draw[->] (solver) -- (wrong);
\draw[->] (hangs) -- (1,4) -- (human);
\draw[->] (wrong) -- (3,4) -- (human);
\end{tikzpicture}
\end{center}
\end{frame}

\begin{frame}
\frametitle{Dual Role of Model}
\begin{itemize}
\item Allows Human to Express Problem
\begin{itemize}
\item Close to Problem Domain
\item Constraints as Abstractions
\end{itemize}
\item Allows Solver to Execute
\begin{itemize}
\item Variables as Communication Mechanism
\item Constraints as Algorithms
\end{itemize}
\end{itemize}
\end{frame}

\begin{frame}
\frametitle{Modelling Frameworks}
\begin{itemize}
\item \href{https://www.minizinc.org/}{MiniZinc} (NICTA, Monash University, Australia)
\item \href{https://github.com/eomahony/Numberjack}{NumberJack} (Insight, Ireland) 
\item \href{https://www-users.york.ac.uk/peter.nightingale/savilerow//}{EssencePrime/SavilleRow} (UK) 
\item \href{https://cpmpy.readthedocs.io/en/latest/}{CPMpy} (KU Leuven)
\item Allow use of multiple back-end solvers
\item Compile model into variants for each solver
\item A priori solver independent model(CP, MIP, SAT)
\end{itemize}
\end{frame} 

\begin{frame}
\frametitle{Framework Process}
\begin{center}
\begin{tikzpicture}[xscale=2]
\node[shape=rectangle,fill=pantone127-4] (problem) at (2.5,5) {Problem};
\node (human) at (2.5,4) {Human};
\node[shape=rectangle,fill=pantone127-4] (model) at (2.5,3) {Model};
\node (compile) at (2.5,2) {Compile/Reformulate};
\node (solver1) at (1,1) {CP};
\node (solver2) at (2,1) {MIP};
\node (solver3) at (3,1) {SAT};
\node (solver4) at (4,1) {Other};
\node[shape=rectangle,fill=pantone157-8] (solution1) at (1,0) {Solution};
\node[shape=rectangle,fill=pantone157-8] (solution2) at (2,0) {Solution};
\node[shape=rectangle,fill=pantone157-8] (solution3) at (3,0) {Solution};
\node[shape=rectangle,fill=pantone157-8] (solution4) at (4,0) {Solution};
\draw[-] (problem) -- (human);
\draw[->] (human) -- (model);
\draw[-] (model) -- (compile);
\draw[-] (compile) -- (solver1);
\draw[-] (compile) -- (solver2);
\draw[-] (compile) -- (solver3);
\draw[-] (compile) -- (solver4);
\draw[->] (solver1) -- (solution1);
\draw[->] (solver2) -- (solution2);
\draw[->] (solver3) -- (solution3);
\draw[->] (solver4) -- (solution4);
\end{tikzpicture}
\end{center}
\end{frame}

% \begin{frame}
  % \frametitle{Do It Now!}
  % \begin{itemize}
  % \item Download and install Minizinc
  % \item \url{https://www.minizinc.org/}
  % \end{itemize}
% \end{frame}


\begin{frame}
\frametitle{Why use Puzzles as Examples?}
\begin{itemize}
\item Easy to understand the problem
\item Solvable by hand without specialized knowledge
\item Possible to compare automated to manual solving process
\end{itemize}

\begin{quote}
The puzzle, though inanimate, is presented as a solvable problem without lasting negative consequences, a very low-risk low-reward situation. By being a puzzle, the object is attempting to convince the user that it must be completed.
\end{quote}
{\tiny Source: \href{https://www.everydayrhetoric.com/2020/03/the-rhetoric-of-puzzles/}{Every Day Rhetoric}}
\end{frame}


\mode<all>{
\part{Basic Constraint Propagation}
}
\begin{frame}
\partpage
\end{frame}
\mode<all>{
\input{../sendmore/body}
}

\mode<all>{
\part{Global Constraints}
}
\begin{frame}
\partpage
\end{frame}
\mode<all>{
\input{../sudoku/body}
}

\clearpage
\mode<all>{
\part{Customizing Search}
}
\begin{frame}
\partpage
\end{frame}
\mode<all>{
\input{../nqueen/body}
}


%% \clearpage
%% \mode<all>{
%%   \part{Problem Decomposition}
%% }
%% \begin{frame}
%%   \partpage
%% \end{frame}
%% \mode<all>{
%%   \input{../party/body}
%%   }

%% \clearpage
%% \mode<all>{
%%   \part{More Global Constraints}
%% }
%% \begin{frame}
%%   \partpage
%% \end{frame}
%% \mode<all>{
%%   \input{../car/body}
%%   }

%% \clearpage
%% \mode<all>{
%%   \part{Methodology}
%% }
%% \begin{frame}
%%   \partpage
%% \end{frame}
%% \mode<all>{
%%   \input{../methodology/body}
%%   }

\clearpage
\mode<all>{
\part{What is missing?}
}
\begin{frame}
\partpage
\end{frame}

\begin{frame}
\frametitle{Many Specialized Topics}
\begin{itemize}
\item How to design efficient core engine
\item Hybrids with LP/MIP tools
\item Hybrids with SAT
\item Symmetry breaking
\item Use of MDD/BDD to encode sets of solutions
\item High level modelling tools
\item Debugging/visualization
\end{itemize}
\end{frame}

\begin{frame}
\frametitle{Reformulation}
\begin{itemize}
\item Just because the user has modelled it this way, it doesn't mean we have to solve it that way
\begin{itemize}
\item Replace some constraint(s) by other, equivalent constraints
\item Because we don't have that constraint in our system
\item For performance
\end{itemize}
\end{itemize}
\end{frame}

\begin{frame}
\frametitle{Learning}
\begin{itemize}
\item While solving the problem we can learn how to strengthen the model/search 
\begin{itemize}
\item Understand which constraints/method contribute to propagation and change schedule
\item Learn no-good constraints by explaining failure
\item Adapt search strategy based on search experience
\end{itemize}
\end{itemize}
\end{frame}

%% \begin{frame}
%% \frametitle{Refined Process}
%% \begin{center}
%% \begin{tikzpicture}
%% \node[shape=rectangle,fill=pantone127-4] (problem) at (3,7) {Problem};
%% \node (human) at (3,6) {Human};
%% \node[shape=rectangle,fill=pantone127-4] (model) at (3,5) {User Model};
%% \node (reformulation) at (3,4) {Reformulation};
%% \node[shape=rectangle,fill=pantone127-4] (implementation) at (3,3) {Implementation Model};
%% \node (solver) at (3,2) {Constraint Solver/Search};
%% \node[shape=rectangle,fill=pantone157-8] (solution) at (3,1) {Solution};
%% \draw[-] (problem) -- (human);
%% \draw[->] (human) -- (model);
%% \draw[-] (model) -- (reformulation);
%% \draw[->] (reformulation) -- (implementation);
%% \draw[-] (implementation) -- (solver);
%% \draw[->] (solver) -- (0,2) -- (0,3) -- (implementation);
%% \draw[->] (solver) -- (solution);
%% \end{tikzpicture}
%% \end{center}
%% \end{frame}

%\begin{frame}
%\frametitle{What is CP actually used for?}
%\url{http://hsimonis.wordpress.com}
%\includegraphics[width=10cm]{../introduction/images/blog}
%\end{frame}

%% \begin{frame}
%% \frametitle{Exercises}
%% \begin{itemize}
%% \item Install Constraint Solver of your Choice
%% \item MiniZinc (MiningZinc) will be used in later sessions
%% \item Run Sendmoremoney, Sudoku, Queens examples
%% \item Can you choose which propagator to use for alldifferent?
%% \item Can you use different search strategies?
%% \end{itemize}
%% \end{frame}

\begin{frame}
\frametitle{More Learning Resources}
\begin{itemize}
\item Survey of Methods, Resources, and Formats for Teaching Constraint Programming
\begin{itemize}
\item by Tejas Santanam, Helmut Simonis
\item \url{https://doi.org/10.48550/arXiv.2403.12717}
\item Based on survey of community for WTCP 2023
\item \url{https://hsimonis.github.io/WTCP2023/}
\end{itemize}
\end{itemize}
\end{frame}


}

\mode<all>{
\part{Global Constraints}
}
\begin{frame}
\partpage
\end{frame}
\mode<all>{
\begin{frame}
\frametitle{Objectives}
\begin{itemize}
\item Overview of Core Constraint Programming
\item Three Main Concepts
\begin{itemize}
\item Constraint Propagation
\item Global Constraints
\item Customizing Search
\end{itemize}
%\item Get Some Experience with MiniZinc
\item Topics will be treated in more detail in later parts of the school
\item Based on Examples, not Formal Description
\end{itemize}
\end{frame}

\begin{frame}
\frametitle{Outline}
\begin{itemize}
\item Why Constraint Programming?
\item Constraint Propagation
\item Global Constraints
\item Customizing Search
%\item Choices in Modelling
%\item Decomposition
%\item More Global Constraints
%\item Methodology
%\item What is missing?
\end{itemize}
\end{frame}

% \begin{frame}
  % \frametitle{Using MiniZinc IDE}
  % \begin{itemize}
  % \item Developed in the Australian NICTA project
  % \item Maintained by Monash University
  % \item Modelling tool with multiple back-end solvers
  % \item Available from \url{https://www.minizinc.org/}
  % \end{itemize}
% \end{frame}


% \begin{frame}
% \frametitle{Examples in ECLiPSe}
% \begin{itemize}
% \item Open sourced constraint programming language
% \item Development goes back to 1985
% \item ECRC, ICL, IC-Parc, PTL, Cisco
% \item \url{https://eclipseclp.org/}
% \item Specialities
% \begin{itemize}
% \item Develop new solvers for specific domains
% \item Integration with MIP
% \end{itemize}
% \item Not included in bundled MiniZinc IDE
% \item Specialized visualization tools used here
  % \begin{itemize}
  % \item CP-Viz, Simonis et al. 2010
    % \end{itemize}
% \end{itemize}
% \end{frame}

\begin{frame}
\frametitle{Tutorial Based on ECLiPSe ELearning Course}
\begin{itemize}
\item Self-study course in constraint programming
\item Supported by Cisco Systems and Silicon Valley Community Foundation
\item Multi-media format, video lectures, slides, handout etc
\item {\small \url{https://eclipseclp.org/ELearning/index.html}}
\end{itemize}
\end{frame}

\begin{frame}
\frametitle{Also Part of CRT-AI Constraint Week}
\begin{itemize}
\item Annual one week course on CP and Optimization in Ireland
\item Part of national training program for PhD students in AI
\item \url{https://www.crt-ai.ie/}
\end{itemize}
\end{frame}



\begin{frame}
\frametitle{Constraint Programming - in a nutshell}
\begin{itemize}
\item Declarative description of problems with
\begin{itemize}
\item {\em Variables} which range over (finite) sets of values
\item {\em Constraints} over subsets of variables which restrict possible value combinations
\item A {\em solution} is a value assignment which satisfies all constraints
\end{itemize}

\item Constraint propagation/reasoning
\begin{itemize}
\item Removing inconsistent values for variables
\item Detect failure if constraint can not be satisfied
\item Interaction of constraints via shared variables
\item Incomplete
\end{itemize}

\item Search
\begin{itemize}
\item User controlled assignment of values to variables
\item Each step triggers constraint propagation 
\end{itemize}
\item Different domains require/allow different methods
\end{itemize}
\end{frame}

\begin{frame}
  \frametitle{Constraint Programming is Different}
  \begin{itemize}
  \item Declarative Programming
    \begin{itemize}
    \item Concentrate on what you want
      \item Not how to get there
      \item Program != Algorithm
      \item Program = Model
    \end{itemize}
    \item Applied to Combinatorial Problems
      \begin{itemize}
        \item No complete polynomial algorithms known (exist?)
        \item CP less ad-hoc than heuristics
        \item Models can evolve
  \end{itemize}
  \end{itemize}
  \end{frame}
    
\begin{frame}
  \frametitle{A Subtractive Process}
  \begin{textblock}{4}(8,-3)
    \includegraphics[width=4cm]{../introduction/images/stages}
  \end{textblock}
  \vfill
  \begin{quote}
    ``Oh, bosh, as Mr. Ruskin says. Sculpture, per se, is the simplest thing in the world. All you have to do is to take a big chunk of marble and a hammer and chisel, make up your mind what you are about to create and chip off all the marble you don't want.''-Paris Gaulois.
  \end{quote}
  
  {\tiny Source: \url{https://quoteinvestigator.com/2014/06/22/chip-away/}}
\end{frame}

\begin{frame}
\frametitle{Basic Process}
\begin{center}
\begin{tikzpicture}
\node[shape=rectangle,fill=pantone127-4] (problem) at (2,5) {Problem};
\node (human) at (2,4) {Human};
\node[shape=rectangle,fill=pantone127-4] (model) at (2,3) {Model};
\node (solver) at (2,2) {Constraint Solver/Search};
\node[shape=rectangle,fill=pantone157-8] (solution) at (2,1) {Solution};
\draw[-] (problem) -- (human);
\draw[->] (human) -- (model);
\draw[-] (model) -- (solver);
\draw[->] (solver) -- (solution);
\end{tikzpicture}
\end{center}
\end{frame}

\begin{frame}
\frametitle{More Realistic}
\begin{center}
\begin{tikzpicture}[xscale=4.0]
\node[shape=rectangle,fill=pantone127-4] (problem) at (2,5) {Problem};
\node (human) at (2,4) {Human};
\node[shape=rectangle,fill=pantone127-4] (model) at (2,3) {Model};
\node (solver) at (2,2) {Constraint Solver/Search};
\node[shape=rectangle,fill=pantone24-8] (hangs) at (1,1) {Hangs};
\node[shape=rectangle,fill=pantone157-8] (solution) at (2,1) {Solution};
\node[shape=rectangle,fill=pantone24-8] (wrong) at (3,1) {Wrong Solution};
\draw[-] (problem) -- (human);
\draw[->] (human) -- (model);
\draw[-] (model) -- (solver);
\draw[->] (solver) -- (hangs);
\draw[->] (solver) -- (solution);
\draw[->] (solver) -- (wrong);
\draw[->] (hangs) -- (1,4) -- (human);
\draw[->] (wrong) -- (3,4) -- (human);
\end{tikzpicture}
\end{center}
\end{frame}

\begin{frame}
\frametitle{Dual Role of Model}
\begin{itemize}
\item Allows Human to Express Problem
\begin{itemize}
\item Close to Problem Domain
\item Constraints as Abstractions
\end{itemize}
\item Allows Solver to Execute
\begin{itemize}
\item Variables as Communication Mechanism
\item Constraints as Algorithms
\end{itemize}
\end{itemize}
\end{frame}

\begin{frame}
\frametitle{Modelling Frameworks}
\begin{itemize}
\item \href{https://www.minizinc.org/}{MiniZinc} (NICTA, Monash University, Australia)
\item \href{https://github.com/eomahony/Numberjack}{NumberJack} (Insight, Ireland) 
\item \href{https://www-users.york.ac.uk/peter.nightingale/savilerow//}{EssencePrime/SavilleRow} (UK) 
\item \href{https://cpmpy.readthedocs.io/en/latest/}{CPMpy} (KU Leuven)
\item Allow use of multiple back-end solvers
\item Compile model into variants for each solver
\item A priori solver independent model(CP, MIP, SAT)
\end{itemize}
\end{frame} 

\begin{frame}
\frametitle{Framework Process}
\begin{center}
\begin{tikzpicture}[xscale=2]
\node[shape=rectangle,fill=pantone127-4] (problem) at (2.5,5) {Problem};
\node (human) at (2.5,4) {Human};
\node[shape=rectangle,fill=pantone127-4] (model) at (2.5,3) {Model};
\node (compile) at (2.5,2) {Compile/Reformulate};
\node (solver1) at (1,1) {CP};
\node (solver2) at (2,1) {MIP};
\node (solver3) at (3,1) {SAT};
\node (solver4) at (4,1) {Other};
\node[shape=rectangle,fill=pantone157-8] (solution1) at (1,0) {Solution};
\node[shape=rectangle,fill=pantone157-8] (solution2) at (2,0) {Solution};
\node[shape=rectangle,fill=pantone157-8] (solution3) at (3,0) {Solution};
\node[shape=rectangle,fill=pantone157-8] (solution4) at (4,0) {Solution};
\draw[-] (problem) -- (human);
\draw[->] (human) -- (model);
\draw[-] (model) -- (compile);
\draw[-] (compile) -- (solver1);
\draw[-] (compile) -- (solver2);
\draw[-] (compile) -- (solver3);
\draw[-] (compile) -- (solver4);
\draw[->] (solver1) -- (solution1);
\draw[->] (solver2) -- (solution2);
\draw[->] (solver3) -- (solution3);
\draw[->] (solver4) -- (solution4);
\end{tikzpicture}
\end{center}
\end{frame}

% \begin{frame}
  % \frametitle{Do It Now!}
  % \begin{itemize}
  % \item Download and install Minizinc
  % \item \url{https://www.minizinc.org/}
  % \end{itemize}
% \end{frame}


\begin{frame}
\frametitle{Why use Puzzles as Examples?}
\begin{itemize}
\item Easy to understand the problem
\item Solvable by hand without specialized knowledge
\item Possible to compare automated to manual solving process
\end{itemize}

\begin{quote}
The puzzle, though inanimate, is presented as a solvable problem without lasting negative consequences, a very low-risk low-reward situation. By being a puzzle, the object is attempting to convince the user that it must be completed.
\end{quote}
{\tiny Source: \href{https://www.everydayrhetoric.com/2020/03/the-rhetoric-of-puzzles/}{Every Day Rhetoric}}
\end{frame}


\mode<all>{
\part{Basic Constraint Propagation}
}
\begin{frame}
\partpage
\end{frame}
\mode<all>{
\input{../sendmore/body}
}

\mode<all>{
\part{Global Constraints}
}
\begin{frame}
\partpage
\end{frame}
\mode<all>{
\input{../sudoku/body}
}

\clearpage
\mode<all>{
\part{Customizing Search}
}
\begin{frame}
\partpage
\end{frame}
\mode<all>{
\input{../nqueen/body}
}


%% \clearpage
%% \mode<all>{
%%   \part{Problem Decomposition}
%% }
%% \begin{frame}
%%   \partpage
%% \end{frame}
%% \mode<all>{
%%   \input{../party/body}
%%   }

%% \clearpage
%% \mode<all>{
%%   \part{More Global Constraints}
%% }
%% \begin{frame}
%%   \partpage
%% \end{frame}
%% \mode<all>{
%%   \input{../car/body}
%%   }

%% \clearpage
%% \mode<all>{
%%   \part{Methodology}
%% }
%% \begin{frame}
%%   \partpage
%% \end{frame}
%% \mode<all>{
%%   \input{../methodology/body}
%%   }

\clearpage
\mode<all>{
\part{What is missing?}
}
\begin{frame}
\partpage
\end{frame}

\begin{frame}
\frametitle{Many Specialized Topics}
\begin{itemize}
\item How to design efficient core engine
\item Hybrids with LP/MIP tools
\item Hybrids with SAT
\item Symmetry breaking
\item Use of MDD/BDD to encode sets of solutions
\item High level modelling tools
\item Debugging/visualization
\end{itemize}
\end{frame}

\begin{frame}
\frametitle{Reformulation}
\begin{itemize}
\item Just because the user has modelled it this way, it doesn't mean we have to solve it that way
\begin{itemize}
\item Replace some constraint(s) by other, equivalent constraints
\item Because we don't have that constraint in our system
\item For performance
\end{itemize}
\end{itemize}
\end{frame}

\begin{frame}
\frametitle{Learning}
\begin{itemize}
\item While solving the problem we can learn how to strengthen the model/search 
\begin{itemize}
\item Understand which constraints/method contribute to propagation and change schedule
\item Learn no-good constraints by explaining failure
\item Adapt search strategy based on search experience
\end{itemize}
\end{itemize}
\end{frame}

%% \begin{frame}
%% \frametitle{Refined Process}
%% \begin{center}
%% \begin{tikzpicture}
%% \node[shape=rectangle,fill=pantone127-4] (problem) at (3,7) {Problem};
%% \node (human) at (3,6) {Human};
%% \node[shape=rectangle,fill=pantone127-4] (model) at (3,5) {User Model};
%% \node (reformulation) at (3,4) {Reformulation};
%% \node[shape=rectangle,fill=pantone127-4] (implementation) at (3,3) {Implementation Model};
%% \node (solver) at (3,2) {Constraint Solver/Search};
%% \node[shape=rectangle,fill=pantone157-8] (solution) at (3,1) {Solution};
%% \draw[-] (problem) -- (human);
%% \draw[->] (human) -- (model);
%% \draw[-] (model) -- (reformulation);
%% \draw[->] (reformulation) -- (implementation);
%% \draw[-] (implementation) -- (solver);
%% \draw[->] (solver) -- (0,2) -- (0,3) -- (implementation);
%% \draw[->] (solver) -- (solution);
%% \end{tikzpicture}
%% \end{center}
%% \end{frame}

%\begin{frame}
%\frametitle{What is CP actually used for?}
%\url{http://hsimonis.wordpress.com}
%\includegraphics[width=10cm]{../introduction/images/blog}
%\end{frame}

%% \begin{frame}
%% \frametitle{Exercises}
%% \begin{itemize}
%% \item Install Constraint Solver of your Choice
%% \item MiniZinc (MiningZinc) will be used in later sessions
%% \item Run Sendmoremoney, Sudoku, Queens examples
%% \item Can you choose which propagator to use for alldifferent?
%% \item Can you use different search strategies?
%% \end{itemize}
%% \end{frame}

\begin{frame}
\frametitle{More Learning Resources}
\begin{itemize}
\item Survey of Methods, Resources, and Formats for Teaching Constraint Programming
\begin{itemize}
\item by Tejas Santanam, Helmut Simonis
\item \url{https://doi.org/10.48550/arXiv.2403.12717}
\item Based on survey of community for WTCP 2023
\item \url{https://hsimonis.github.io/WTCP2023/}
\end{itemize}
\end{itemize}
\end{frame}


}

\clearpage
\mode<all>{
\part{Customizing Search}
}
\begin{frame}
\partpage
\end{frame}
\mode<all>{
\begin{frame}
\frametitle{Objectives}
\begin{itemize}
\item Overview of Core Constraint Programming
\item Three Main Concepts
\begin{itemize}
\item Constraint Propagation
\item Global Constraints
\item Customizing Search
\end{itemize}
%\item Get Some Experience with MiniZinc
\item Topics will be treated in more detail in later parts of the school
\item Based on Examples, not Formal Description
\end{itemize}
\end{frame}

\begin{frame}
\frametitle{Outline}
\begin{itemize}
\item Why Constraint Programming?
\item Constraint Propagation
\item Global Constraints
\item Customizing Search
%\item Choices in Modelling
%\item Decomposition
%\item More Global Constraints
%\item Methodology
%\item What is missing?
\end{itemize}
\end{frame}

% \begin{frame}
  % \frametitle{Using MiniZinc IDE}
  % \begin{itemize}
  % \item Developed in the Australian NICTA project
  % \item Maintained by Monash University
  % \item Modelling tool with multiple back-end solvers
  % \item Available from \url{https://www.minizinc.org/}
  % \end{itemize}
% \end{frame}


% \begin{frame}
% \frametitle{Examples in ECLiPSe}
% \begin{itemize}
% \item Open sourced constraint programming language
% \item Development goes back to 1985
% \item ECRC, ICL, IC-Parc, PTL, Cisco
% \item \url{https://eclipseclp.org/}
% \item Specialities
% \begin{itemize}
% \item Develop new solvers for specific domains
% \item Integration with MIP
% \end{itemize}
% \item Not included in bundled MiniZinc IDE
% \item Specialized visualization tools used here
  % \begin{itemize}
  % \item CP-Viz, Simonis et al. 2010
    % \end{itemize}
% \end{itemize}
% \end{frame}

\begin{frame}
\frametitle{Tutorial Based on ECLiPSe ELearning Course}
\begin{itemize}
\item Self-study course in constraint programming
\item Supported by Cisco Systems and Silicon Valley Community Foundation
\item Multi-media format, video lectures, slides, handout etc
\item {\small \url{https://eclipseclp.org/ELearning/index.html}}
\end{itemize}
\end{frame}

\begin{frame}
\frametitle{Also Part of CRT-AI Constraint Week}
\begin{itemize}
\item Annual one week course on CP and Optimization in Ireland
\item Part of national training program for PhD students in AI
\item \url{https://www.crt-ai.ie/}
\end{itemize}
\end{frame}



\begin{frame}
\frametitle{Constraint Programming - in a nutshell}
\begin{itemize}
\item Declarative description of problems with
\begin{itemize}
\item {\em Variables} which range over (finite) sets of values
\item {\em Constraints} over subsets of variables which restrict possible value combinations
\item A {\em solution} is a value assignment which satisfies all constraints
\end{itemize}

\item Constraint propagation/reasoning
\begin{itemize}
\item Removing inconsistent values for variables
\item Detect failure if constraint can not be satisfied
\item Interaction of constraints via shared variables
\item Incomplete
\end{itemize}

\item Search
\begin{itemize}
\item User controlled assignment of values to variables
\item Each step triggers constraint propagation 
\end{itemize}
\item Different domains require/allow different methods
\end{itemize}
\end{frame}

\begin{frame}
  \frametitle{Constraint Programming is Different}
  \begin{itemize}
  \item Declarative Programming
    \begin{itemize}
    \item Concentrate on what you want
      \item Not how to get there
      \item Program != Algorithm
      \item Program = Model
    \end{itemize}
    \item Applied to Combinatorial Problems
      \begin{itemize}
        \item No complete polynomial algorithms known (exist?)
        \item CP less ad-hoc than heuristics
        \item Models can evolve
  \end{itemize}
  \end{itemize}
  \end{frame}
    
\begin{frame}
  \frametitle{A Subtractive Process}
  \begin{textblock}{4}(8,-3)
    \includegraphics[width=4cm]{../introduction/images/stages}
  \end{textblock}
  \vfill
  \begin{quote}
    ``Oh, bosh, as Mr. Ruskin says. Sculpture, per se, is the simplest thing in the world. All you have to do is to take a big chunk of marble and a hammer and chisel, make up your mind what you are about to create and chip off all the marble you don't want.''-Paris Gaulois.
  \end{quote}
  
  {\tiny Source: \url{https://quoteinvestigator.com/2014/06/22/chip-away/}}
\end{frame}

\begin{frame}
\frametitle{Basic Process}
\begin{center}
\begin{tikzpicture}
\node[shape=rectangle,fill=pantone127-4] (problem) at (2,5) {Problem};
\node (human) at (2,4) {Human};
\node[shape=rectangle,fill=pantone127-4] (model) at (2,3) {Model};
\node (solver) at (2,2) {Constraint Solver/Search};
\node[shape=rectangle,fill=pantone157-8] (solution) at (2,1) {Solution};
\draw[-] (problem) -- (human);
\draw[->] (human) -- (model);
\draw[-] (model) -- (solver);
\draw[->] (solver) -- (solution);
\end{tikzpicture}
\end{center}
\end{frame}

\begin{frame}
\frametitle{More Realistic}
\begin{center}
\begin{tikzpicture}[xscale=4.0]
\node[shape=rectangle,fill=pantone127-4] (problem) at (2,5) {Problem};
\node (human) at (2,4) {Human};
\node[shape=rectangle,fill=pantone127-4] (model) at (2,3) {Model};
\node (solver) at (2,2) {Constraint Solver/Search};
\node[shape=rectangle,fill=pantone24-8] (hangs) at (1,1) {Hangs};
\node[shape=rectangle,fill=pantone157-8] (solution) at (2,1) {Solution};
\node[shape=rectangle,fill=pantone24-8] (wrong) at (3,1) {Wrong Solution};
\draw[-] (problem) -- (human);
\draw[->] (human) -- (model);
\draw[-] (model) -- (solver);
\draw[->] (solver) -- (hangs);
\draw[->] (solver) -- (solution);
\draw[->] (solver) -- (wrong);
\draw[->] (hangs) -- (1,4) -- (human);
\draw[->] (wrong) -- (3,4) -- (human);
\end{tikzpicture}
\end{center}
\end{frame}

\begin{frame}
\frametitle{Dual Role of Model}
\begin{itemize}
\item Allows Human to Express Problem
\begin{itemize}
\item Close to Problem Domain
\item Constraints as Abstractions
\end{itemize}
\item Allows Solver to Execute
\begin{itemize}
\item Variables as Communication Mechanism
\item Constraints as Algorithms
\end{itemize}
\end{itemize}
\end{frame}

\begin{frame}
\frametitle{Modelling Frameworks}
\begin{itemize}
\item \href{https://www.minizinc.org/}{MiniZinc} (NICTA, Monash University, Australia)
\item \href{https://github.com/eomahony/Numberjack}{NumberJack} (Insight, Ireland) 
\item \href{https://www-users.york.ac.uk/peter.nightingale/savilerow//}{EssencePrime/SavilleRow} (UK) 
\item \href{https://cpmpy.readthedocs.io/en/latest/}{CPMpy} (KU Leuven)
\item Allow use of multiple back-end solvers
\item Compile model into variants for each solver
\item A priori solver independent model(CP, MIP, SAT)
\end{itemize}
\end{frame} 

\begin{frame}
\frametitle{Framework Process}
\begin{center}
\begin{tikzpicture}[xscale=2]
\node[shape=rectangle,fill=pantone127-4] (problem) at (2.5,5) {Problem};
\node (human) at (2.5,4) {Human};
\node[shape=rectangle,fill=pantone127-4] (model) at (2.5,3) {Model};
\node (compile) at (2.5,2) {Compile/Reformulate};
\node (solver1) at (1,1) {CP};
\node (solver2) at (2,1) {MIP};
\node (solver3) at (3,1) {SAT};
\node (solver4) at (4,1) {Other};
\node[shape=rectangle,fill=pantone157-8] (solution1) at (1,0) {Solution};
\node[shape=rectangle,fill=pantone157-8] (solution2) at (2,0) {Solution};
\node[shape=rectangle,fill=pantone157-8] (solution3) at (3,0) {Solution};
\node[shape=rectangle,fill=pantone157-8] (solution4) at (4,0) {Solution};
\draw[-] (problem) -- (human);
\draw[->] (human) -- (model);
\draw[-] (model) -- (compile);
\draw[-] (compile) -- (solver1);
\draw[-] (compile) -- (solver2);
\draw[-] (compile) -- (solver3);
\draw[-] (compile) -- (solver4);
\draw[->] (solver1) -- (solution1);
\draw[->] (solver2) -- (solution2);
\draw[->] (solver3) -- (solution3);
\draw[->] (solver4) -- (solution4);
\end{tikzpicture}
\end{center}
\end{frame}

% \begin{frame}
  % \frametitle{Do It Now!}
  % \begin{itemize}
  % \item Download and install Minizinc
  % \item \url{https://www.minizinc.org/}
  % \end{itemize}
% \end{frame}


\begin{frame}
\frametitle{Why use Puzzles as Examples?}
\begin{itemize}
\item Easy to understand the problem
\item Solvable by hand without specialized knowledge
\item Possible to compare automated to manual solving process
\end{itemize}

\begin{quote}
The puzzle, though inanimate, is presented as a solvable problem without lasting negative consequences, a very low-risk low-reward situation. By being a puzzle, the object is attempting to convince the user that it must be completed.
\end{quote}
{\tiny Source: \href{https://www.everydayrhetoric.com/2020/03/the-rhetoric-of-puzzles/}{Every Day Rhetoric}}
\end{frame}


\mode<all>{
\part{Basic Constraint Propagation}
}
\begin{frame}
\partpage
\end{frame}
\mode<all>{
\input{../sendmore/body}
}

\mode<all>{
\part{Global Constraints}
}
\begin{frame}
\partpage
\end{frame}
\mode<all>{
\input{../sudoku/body}
}

\clearpage
\mode<all>{
\part{Customizing Search}
}
\begin{frame}
\partpage
\end{frame}
\mode<all>{
\input{../nqueen/body}
}


%% \clearpage
%% \mode<all>{
%%   \part{Problem Decomposition}
%% }
%% \begin{frame}
%%   \partpage
%% \end{frame}
%% \mode<all>{
%%   \input{../party/body}
%%   }

%% \clearpage
%% \mode<all>{
%%   \part{More Global Constraints}
%% }
%% \begin{frame}
%%   \partpage
%% \end{frame}
%% \mode<all>{
%%   \input{../car/body}
%%   }

%% \clearpage
%% \mode<all>{
%%   \part{Methodology}
%% }
%% \begin{frame}
%%   \partpage
%% \end{frame}
%% \mode<all>{
%%   \input{../methodology/body}
%%   }

\clearpage
\mode<all>{
\part{What is missing?}
}
\begin{frame}
\partpage
\end{frame}

\begin{frame}
\frametitle{Many Specialized Topics}
\begin{itemize}
\item How to design efficient core engine
\item Hybrids with LP/MIP tools
\item Hybrids with SAT
\item Symmetry breaking
\item Use of MDD/BDD to encode sets of solutions
\item High level modelling tools
\item Debugging/visualization
\end{itemize}
\end{frame}

\begin{frame}
\frametitle{Reformulation}
\begin{itemize}
\item Just because the user has modelled it this way, it doesn't mean we have to solve it that way
\begin{itemize}
\item Replace some constraint(s) by other, equivalent constraints
\item Because we don't have that constraint in our system
\item For performance
\end{itemize}
\end{itemize}
\end{frame}

\begin{frame}
\frametitle{Learning}
\begin{itemize}
\item While solving the problem we can learn how to strengthen the model/search 
\begin{itemize}
\item Understand which constraints/method contribute to propagation and change schedule
\item Learn no-good constraints by explaining failure
\item Adapt search strategy based on search experience
\end{itemize}
\end{itemize}
\end{frame}

%% \begin{frame}
%% \frametitle{Refined Process}
%% \begin{center}
%% \begin{tikzpicture}
%% \node[shape=rectangle,fill=pantone127-4] (problem) at (3,7) {Problem};
%% \node (human) at (3,6) {Human};
%% \node[shape=rectangle,fill=pantone127-4] (model) at (3,5) {User Model};
%% \node (reformulation) at (3,4) {Reformulation};
%% \node[shape=rectangle,fill=pantone127-4] (implementation) at (3,3) {Implementation Model};
%% \node (solver) at (3,2) {Constraint Solver/Search};
%% \node[shape=rectangle,fill=pantone157-8] (solution) at (3,1) {Solution};
%% \draw[-] (problem) -- (human);
%% \draw[->] (human) -- (model);
%% \draw[-] (model) -- (reformulation);
%% \draw[->] (reformulation) -- (implementation);
%% \draw[-] (implementation) -- (solver);
%% \draw[->] (solver) -- (0,2) -- (0,3) -- (implementation);
%% \draw[->] (solver) -- (solution);
%% \end{tikzpicture}
%% \end{center}
%% \end{frame}

%\begin{frame}
%\frametitle{What is CP actually used for?}
%\url{http://hsimonis.wordpress.com}
%\includegraphics[width=10cm]{../introduction/images/blog}
%\end{frame}

%% \begin{frame}
%% \frametitle{Exercises}
%% \begin{itemize}
%% \item Install Constraint Solver of your Choice
%% \item MiniZinc (MiningZinc) will be used in later sessions
%% \item Run Sendmoremoney, Sudoku, Queens examples
%% \item Can you choose which propagator to use for alldifferent?
%% \item Can you use different search strategies?
%% \end{itemize}
%% \end{frame}

\begin{frame}
\frametitle{More Learning Resources}
\begin{itemize}
\item Survey of Methods, Resources, and Formats for Teaching Constraint Programming
\begin{itemize}
\item by Tejas Santanam, Helmut Simonis
\item \url{https://doi.org/10.48550/arXiv.2403.12717}
\item Based on survey of community for WTCP 2023
\item \url{https://hsimonis.github.io/WTCP2023/}
\end{itemize}
\end{itemize}
\end{frame}


}


%% \clearpage
%% \mode<all>{
%%   \part{Problem Decomposition}
%% }
%% \begin{frame}
%%   \partpage
%% \end{frame}
%% \mode<all>{
%%   \begin{frame}
\frametitle{Objectives}
\begin{itemize}
\item Overview of Core Constraint Programming
\item Three Main Concepts
\begin{itemize}
\item Constraint Propagation
\item Global Constraints
\item Customizing Search
\end{itemize}
%\item Get Some Experience with MiniZinc
\item Topics will be treated in more detail in later parts of the school
\item Based on Examples, not Formal Description
\end{itemize}
\end{frame}

\begin{frame}
\frametitle{Outline}
\begin{itemize}
\item Why Constraint Programming?
\item Constraint Propagation
\item Global Constraints
\item Customizing Search
%\item Choices in Modelling
%\item Decomposition
%\item More Global Constraints
%\item Methodology
%\item What is missing?
\end{itemize}
\end{frame}

% \begin{frame}
  % \frametitle{Using MiniZinc IDE}
  % \begin{itemize}
  % \item Developed in the Australian NICTA project
  % \item Maintained by Monash University
  % \item Modelling tool with multiple back-end solvers
  % \item Available from \url{https://www.minizinc.org/}
  % \end{itemize}
% \end{frame}


% \begin{frame}
% \frametitle{Examples in ECLiPSe}
% \begin{itemize}
% \item Open sourced constraint programming language
% \item Development goes back to 1985
% \item ECRC, ICL, IC-Parc, PTL, Cisco
% \item \url{https://eclipseclp.org/}
% \item Specialities
% \begin{itemize}
% \item Develop new solvers for specific domains
% \item Integration with MIP
% \end{itemize}
% \item Not included in bundled MiniZinc IDE
% \item Specialized visualization tools used here
  % \begin{itemize}
  % \item CP-Viz, Simonis et al. 2010
    % \end{itemize}
% \end{itemize}
% \end{frame}

\begin{frame}
\frametitle{Tutorial Based on ECLiPSe ELearning Course}
\begin{itemize}
\item Self-study course in constraint programming
\item Supported by Cisco Systems and Silicon Valley Community Foundation
\item Multi-media format, video lectures, slides, handout etc
\item {\small \url{https://eclipseclp.org/ELearning/index.html}}
\end{itemize}
\end{frame}

\begin{frame}
\frametitle{Also Part of CRT-AI Constraint Week}
\begin{itemize}
\item Annual one week course on CP and Optimization in Ireland
\item Part of national training program for PhD students in AI
\item \url{https://www.crt-ai.ie/}
\end{itemize}
\end{frame}



\begin{frame}
\frametitle{Constraint Programming - in a nutshell}
\begin{itemize}
\item Declarative description of problems with
\begin{itemize}
\item {\em Variables} which range over (finite) sets of values
\item {\em Constraints} over subsets of variables which restrict possible value combinations
\item A {\em solution} is a value assignment which satisfies all constraints
\end{itemize}

\item Constraint propagation/reasoning
\begin{itemize}
\item Removing inconsistent values for variables
\item Detect failure if constraint can not be satisfied
\item Interaction of constraints via shared variables
\item Incomplete
\end{itemize}

\item Search
\begin{itemize}
\item User controlled assignment of values to variables
\item Each step triggers constraint propagation 
\end{itemize}
\item Different domains require/allow different methods
\end{itemize}
\end{frame}

\begin{frame}
  \frametitle{Constraint Programming is Different}
  \begin{itemize}
  \item Declarative Programming
    \begin{itemize}
    \item Concentrate on what you want
      \item Not how to get there
      \item Program != Algorithm
      \item Program = Model
    \end{itemize}
    \item Applied to Combinatorial Problems
      \begin{itemize}
        \item No complete polynomial algorithms known (exist?)
        \item CP less ad-hoc than heuristics
        \item Models can evolve
  \end{itemize}
  \end{itemize}
  \end{frame}
    
\begin{frame}
  \frametitle{A Subtractive Process}
  \begin{textblock}{4}(8,-3)
    \includegraphics[width=4cm]{../introduction/images/stages}
  \end{textblock}
  \vfill
  \begin{quote}
    ``Oh, bosh, as Mr. Ruskin says. Sculpture, per se, is the simplest thing in the world. All you have to do is to take a big chunk of marble and a hammer and chisel, make up your mind what you are about to create and chip off all the marble you don't want.''-Paris Gaulois.
  \end{quote}
  
  {\tiny Source: \url{https://quoteinvestigator.com/2014/06/22/chip-away/}}
\end{frame}

\begin{frame}
\frametitle{Basic Process}
\begin{center}
\begin{tikzpicture}
\node[shape=rectangle,fill=pantone127-4] (problem) at (2,5) {Problem};
\node (human) at (2,4) {Human};
\node[shape=rectangle,fill=pantone127-4] (model) at (2,3) {Model};
\node (solver) at (2,2) {Constraint Solver/Search};
\node[shape=rectangle,fill=pantone157-8] (solution) at (2,1) {Solution};
\draw[-] (problem) -- (human);
\draw[->] (human) -- (model);
\draw[-] (model) -- (solver);
\draw[->] (solver) -- (solution);
\end{tikzpicture}
\end{center}
\end{frame}

\begin{frame}
\frametitle{More Realistic}
\begin{center}
\begin{tikzpicture}[xscale=4.0]
\node[shape=rectangle,fill=pantone127-4] (problem) at (2,5) {Problem};
\node (human) at (2,4) {Human};
\node[shape=rectangle,fill=pantone127-4] (model) at (2,3) {Model};
\node (solver) at (2,2) {Constraint Solver/Search};
\node[shape=rectangle,fill=pantone24-8] (hangs) at (1,1) {Hangs};
\node[shape=rectangle,fill=pantone157-8] (solution) at (2,1) {Solution};
\node[shape=rectangle,fill=pantone24-8] (wrong) at (3,1) {Wrong Solution};
\draw[-] (problem) -- (human);
\draw[->] (human) -- (model);
\draw[-] (model) -- (solver);
\draw[->] (solver) -- (hangs);
\draw[->] (solver) -- (solution);
\draw[->] (solver) -- (wrong);
\draw[->] (hangs) -- (1,4) -- (human);
\draw[->] (wrong) -- (3,4) -- (human);
\end{tikzpicture}
\end{center}
\end{frame}

\begin{frame}
\frametitle{Dual Role of Model}
\begin{itemize}
\item Allows Human to Express Problem
\begin{itemize}
\item Close to Problem Domain
\item Constraints as Abstractions
\end{itemize}
\item Allows Solver to Execute
\begin{itemize}
\item Variables as Communication Mechanism
\item Constraints as Algorithms
\end{itemize}
\end{itemize}
\end{frame}

\begin{frame}
\frametitle{Modelling Frameworks}
\begin{itemize}
\item \href{https://www.minizinc.org/}{MiniZinc} (NICTA, Monash University, Australia)
\item \href{https://github.com/eomahony/Numberjack}{NumberJack} (Insight, Ireland) 
\item \href{https://www-users.york.ac.uk/peter.nightingale/savilerow//}{EssencePrime/SavilleRow} (UK) 
\item \href{https://cpmpy.readthedocs.io/en/latest/}{CPMpy} (KU Leuven)
\item Allow use of multiple back-end solvers
\item Compile model into variants for each solver
\item A priori solver independent model(CP, MIP, SAT)
\end{itemize}
\end{frame} 

\begin{frame}
\frametitle{Framework Process}
\begin{center}
\begin{tikzpicture}[xscale=2]
\node[shape=rectangle,fill=pantone127-4] (problem) at (2.5,5) {Problem};
\node (human) at (2.5,4) {Human};
\node[shape=rectangle,fill=pantone127-4] (model) at (2.5,3) {Model};
\node (compile) at (2.5,2) {Compile/Reformulate};
\node (solver1) at (1,1) {CP};
\node (solver2) at (2,1) {MIP};
\node (solver3) at (3,1) {SAT};
\node (solver4) at (4,1) {Other};
\node[shape=rectangle,fill=pantone157-8] (solution1) at (1,0) {Solution};
\node[shape=rectangle,fill=pantone157-8] (solution2) at (2,0) {Solution};
\node[shape=rectangle,fill=pantone157-8] (solution3) at (3,0) {Solution};
\node[shape=rectangle,fill=pantone157-8] (solution4) at (4,0) {Solution};
\draw[-] (problem) -- (human);
\draw[->] (human) -- (model);
\draw[-] (model) -- (compile);
\draw[-] (compile) -- (solver1);
\draw[-] (compile) -- (solver2);
\draw[-] (compile) -- (solver3);
\draw[-] (compile) -- (solver4);
\draw[->] (solver1) -- (solution1);
\draw[->] (solver2) -- (solution2);
\draw[->] (solver3) -- (solution3);
\draw[->] (solver4) -- (solution4);
\end{tikzpicture}
\end{center}
\end{frame}

% \begin{frame}
  % \frametitle{Do It Now!}
  % \begin{itemize}
  % \item Download and install Minizinc
  % \item \url{https://www.minizinc.org/}
  % \end{itemize}
% \end{frame}


\begin{frame}
\frametitle{Why use Puzzles as Examples?}
\begin{itemize}
\item Easy to understand the problem
\item Solvable by hand without specialized knowledge
\item Possible to compare automated to manual solving process
\end{itemize}

\begin{quote}
The puzzle, though inanimate, is presented as a solvable problem without lasting negative consequences, a very low-risk low-reward situation. By being a puzzle, the object is attempting to convince the user that it must be completed.
\end{quote}
{\tiny Source: \href{https://www.everydayrhetoric.com/2020/03/the-rhetoric-of-puzzles/}{Every Day Rhetoric}}
\end{frame}


\mode<all>{
\part{Basic Constraint Propagation}
}
\begin{frame}
\partpage
\end{frame}
\mode<all>{
\input{../sendmore/body}
}

\mode<all>{
\part{Global Constraints}
}
\begin{frame}
\partpage
\end{frame}
\mode<all>{
\input{../sudoku/body}
}

\clearpage
\mode<all>{
\part{Customizing Search}
}
\begin{frame}
\partpage
\end{frame}
\mode<all>{
\input{../nqueen/body}
}


%% \clearpage
%% \mode<all>{
%%   \part{Problem Decomposition}
%% }
%% \begin{frame}
%%   \partpage
%% \end{frame}
%% \mode<all>{
%%   \input{../party/body}
%%   }

%% \clearpage
%% \mode<all>{
%%   \part{More Global Constraints}
%% }
%% \begin{frame}
%%   \partpage
%% \end{frame}
%% \mode<all>{
%%   \input{../car/body}
%%   }

%% \clearpage
%% \mode<all>{
%%   \part{Methodology}
%% }
%% \begin{frame}
%%   \partpage
%% \end{frame}
%% \mode<all>{
%%   \input{../methodology/body}
%%   }

\clearpage
\mode<all>{
\part{What is missing?}
}
\begin{frame}
\partpage
\end{frame}

\begin{frame}
\frametitle{Many Specialized Topics}
\begin{itemize}
\item How to design efficient core engine
\item Hybrids with LP/MIP tools
\item Hybrids with SAT
\item Symmetry breaking
\item Use of MDD/BDD to encode sets of solutions
\item High level modelling tools
\item Debugging/visualization
\end{itemize}
\end{frame}

\begin{frame}
\frametitle{Reformulation}
\begin{itemize}
\item Just because the user has modelled it this way, it doesn't mean we have to solve it that way
\begin{itemize}
\item Replace some constraint(s) by other, equivalent constraints
\item Because we don't have that constraint in our system
\item For performance
\end{itemize}
\end{itemize}
\end{frame}

\begin{frame}
\frametitle{Learning}
\begin{itemize}
\item While solving the problem we can learn how to strengthen the model/search 
\begin{itemize}
\item Understand which constraints/method contribute to propagation and change schedule
\item Learn no-good constraints by explaining failure
\item Adapt search strategy based on search experience
\end{itemize}
\end{itemize}
\end{frame}

%% \begin{frame}
%% \frametitle{Refined Process}
%% \begin{center}
%% \begin{tikzpicture}
%% \node[shape=rectangle,fill=pantone127-4] (problem) at (3,7) {Problem};
%% \node (human) at (3,6) {Human};
%% \node[shape=rectangle,fill=pantone127-4] (model) at (3,5) {User Model};
%% \node (reformulation) at (3,4) {Reformulation};
%% \node[shape=rectangle,fill=pantone127-4] (implementation) at (3,3) {Implementation Model};
%% \node (solver) at (3,2) {Constraint Solver/Search};
%% \node[shape=rectangle,fill=pantone157-8] (solution) at (3,1) {Solution};
%% \draw[-] (problem) -- (human);
%% \draw[->] (human) -- (model);
%% \draw[-] (model) -- (reformulation);
%% \draw[->] (reformulation) -- (implementation);
%% \draw[-] (implementation) -- (solver);
%% \draw[->] (solver) -- (0,2) -- (0,3) -- (implementation);
%% \draw[->] (solver) -- (solution);
%% \end{tikzpicture}
%% \end{center}
%% \end{frame}

%\begin{frame}
%\frametitle{What is CP actually used for?}
%\url{http://hsimonis.wordpress.com}
%\includegraphics[width=10cm]{../introduction/images/blog}
%\end{frame}

%% \begin{frame}
%% \frametitle{Exercises}
%% \begin{itemize}
%% \item Install Constraint Solver of your Choice
%% \item MiniZinc (MiningZinc) will be used in later sessions
%% \item Run Sendmoremoney, Sudoku, Queens examples
%% \item Can you choose which propagator to use for alldifferent?
%% \item Can you use different search strategies?
%% \end{itemize}
%% \end{frame}

\begin{frame}
\frametitle{More Learning Resources}
\begin{itemize}
\item Survey of Methods, Resources, and Formats for Teaching Constraint Programming
\begin{itemize}
\item by Tejas Santanam, Helmut Simonis
\item \url{https://doi.org/10.48550/arXiv.2403.12717}
\item Based on survey of community for WTCP 2023
\item \url{https://hsimonis.github.io/WTCP2023/}
\end{itemize}
\end{itemize}
\end{frame}


%%   }

%% \clearpage
%% \mode<all>{
%%   \part{More Global Constraints}
%% }
%% \begin{frame}
%%   \partpage
%% \end{frame}
%% \mode<all>{
%%   \begin{frame}
\frametitle{Objectives}
\begin{itemize}
\item Overview of Core Constraint Programming
\item Three Main Concepts
\begin{itemize}
\item Constraint Propagation
\item Global Constraints
\item Customizing Search
\end{itemize}
%\item Get Some Experience with MiniZinc
\item Topics will be treated in more detail in later parts of the school
\item Based on Examples, not Formal Description
\end{itemize}
\end{frame}

\begin{frame}
\frametitle{Outline}
\begin{itemize}
\item Why Constraint Programming?
\item Constraint Propagation
\item Global Constraints
\item Customizing Search
%\item Choices in Modelling
%\item Decomposition
%\item More Global Constraints
%\item Methodology
%\item What is missing?
\end{itemize}
\end{frame}

% \begin{frame}
  % \frametitle{Using MiniZinc IDE}
  % \begin{itemize}
  % \item Developed in the Australian NICTA project
  % \item Maintained by Monash University
  % \item Modelling tool with multiple back-end solvers
  % \item Available from \url{https://www.minizinc.org/}
  % \end{itemize}
% \end{frame}


% \begin{frame}
% \frametitle{Examples in ECLiPSe}
% \begin{itemize}
% \item Open sourced constraint programming language
% \item Development goes back to 1985
% \item ECRC, ICL, IC-Parc, PTL, Cisco
% \item \url{https://eclipseclp.org/}
% \item Specialities
% \begin{itemize}
% \item Develop new solvers for specific domains
% \item Integration with MIP
% \end{itemize}
% \item Not included in bundled MiniZinc IDE
% \item Specialized visualization tools used here
  % \begin{itemize}
  % \item CP-Viz, Simonis et al. 2010
    % \end{itemize}
% \end{itemize}
% \end{frame}

\begin{frame}
\frametitle{Tutorial Based on ECLiPSe ELearning Course}
\begin{itemize}
\item Self-study course in constraint programming
\item Supported by Cisco Systems and Silicon Valley Community Foundation
\item Multi-media format, video lectures, slides, handout etc
\item {\small \url{https://eclipseclp.org/ELearning/index.html}}
\end{itemize}
\end{frame}

\begin{frame}
\frametitle{Also Part of CRT-AI Constraint Week}
\begin{itemize}
\item Annual one week course on CP and Optimization in Ireland
\item Part of national training program for PhD students in AI
\item \url{https://www.crt-ai.ie/}
\end{itemize}
\end{frame}



\begin{frame}
\frametitle{Constraint Programming - in a nutshell}
\begin{itemize}
\item Declarative description of problems with
\begin{itemize}
\item {\em Variables} which range over (finite) sets of values
\item {\em Constraints} over subsets of variables which restrict possible value combinations
\item A {\em solution} is a value assignment which satisfies all constraints
\end{itemize}

\item Constraint propagation/reasoning
\begin{itemize}
\item Removing inconsistent values for variables
\item Detect failure if constraint can not be satisfied
\item Interaction of constraints via shared variables
\item Incomplete
\end{itemize}

\item Search
\begin{itemize}
\item User controlled assignment of values to variables
\item Each step triggers constraint propagation 
\end{itemize}
\item Different domains require/allow different methods
\end{itemize}
\end{frame}

\begin{frame}
  \frametitle{Constraint Programming is Different}
  \begin{itemize}
  \item Declarative Programming
    \begin{itemize}
    \item Concentrate on what you want
      \item Not how to get there
      \item Program != Algorithm
      \item Program = Model
    \end{itemize}
    \item Applied to Combinatorial Problems
      \begin{itemize}
        \item No complete polynomial algorithms known (exist?)
        \item CP less ad-hoc than heuristics
        \item Models can evolve
  \end{itemize}
  \end{itemize}
  \end{frame}
    
\begin{frame}
  \frametitle{A Subtractive Process}
  \begin{textblock}{4}(8,-3)
    \includegraphics[width=4cm]{../introduction/images/stages}
  \end{textblock}
  \vfill
  \begin{quote}
    ``Oh, bosh, as Mr. Ruskin says. Sculpture, per se, is the simplest thing in the world. All you have to do is to take a big chunk of marble and a hammer and chisel, make up your mind what you are about to create and chip off all the marble you don't want.''-Paris Gaulois.
  \end{quote}
  
  {\tiny Source: \url{https://quoteinvestigator.com/2014/06/22/chip-away/}}
\end{frame}

\begin{frame}
\frametitle{Basic Process}
\begin{center}
\begin{tikzpicture}
\node[shape=rectangle,fill=pantone127-4] (problem) at (2,5) {Problem};
\node (human) at (2,4) {Human};
\node[shape=rectangle,fill=pantone127-4] (model) at (2,3) {Model};
\node (solver) at (2,2) {Constraint Solver/Search};
\node[shape=rectangle,fill=pantone157-8] (solution) at (2,1) {Solution};
\draw[-] (problem) -- (human);
\draw[->] (human) -- (model);
\draw[-] (model) -- (solver);
\draw[->] (solver) -- (solution);
\end{tikzpicture}
\end{center}
\end{frame}

\begin{frame}
\frametitle{More Realistic}
\begin{center}
\begin{tikzpicture}[xscale=4.0]
\node[shape=rectangle,fill=pantone127-4] (problem) at (2,5) {Problem};
\node (human) at (2,4) {Human};
\node[shape=rectangle,fill=pantone127-4] (model) at (2,3) {Model};
\node (solver) at (2,2) {Constraint Solver/Search};
\node[shape=rectangle,fill=pantone24-8] (hangs) at (1,1) {Hangs};
\node[shape=rectangle,fill=pantone157-8] (solution) at (2,1) {Solution};
\node[shape=rectangle,fill=pantone24-8] (wrong) at (3,1) {Wrong Solution};
\draw[-] (problem) -- (human);
\draw[->] (human) -- (model);
\draw[-] (model) -- (solver);
\draw[->] (solver) -- (hangs);
\draw[->] (solver) -- (solution);
\draw[->] (solver) -- (wrong);
\draw[->] (hangs) -- (1,4) -- (human);
\draw[->] (wrong) -- (3,4) -- (human);
\end{tikzpicture}
\end{center}
\end{frame}

\begin{frame}
\frametitle{Dual Role of Model}
\begin{itemize}
\item Allows Human to Express Problem
\begin{itemize}
\item Close to Problem Domain
\item Constraints as Abstractions
\end{itemize}
\item Allows Solver to Execute
\begin{itemize}
\item Variables as Communication Mechanism
\item Constraints as Algorithms
\end{itemize}
\end{itemize}
\end{frame}

\begin{frame}
\frametitle{Modelling Frameworks}
\begin{itemize}
\item \href{https://www.minizinc.org/}{MiniZinc} (NICTA, Monash University, Australia)
\item \href{https://github.com/eomahony/Numberjack}{NumberJack} (Insight, Ireland) 
\item \href{https://www-users.york.ac.uk/peter.nightingale/savilerow//}{EssencePrime/SavilleRow} (UK) 
\item \href{https://cpmpy.readthedocs.io/en/latest/}{CPMpy} (KU Leuven)
\item Allow use of multiple back-end solvers
\item Compile model into variants for each solver
\item A priori solver independent model(CP, MIP, SAT)
\end{itemize}
\end{frame} 

\begin{frame}
\frametitle{Framework Process}
\begin{center}
\begin{tikzpicture}[xscale=2]
\node[shape=rectangle,fill=pantone127-4] (problem) at (2.5,5) {Problem};
\node (human) at (2.5,4) {Human};
\node[shape=rectangle,fill=pantone127-4] (model) at (2.5,3) {Model};
\node (compile) at (2.5,2) {Compile/Reformulate};
\node (solver1) at (1,1) {CP};
\node (solver2) at (2,1) {MIP};
\node (solver3) at (3,1) {SAT};
\node (solver4) at (4,1) {Other};
\node[shape=rectangle,fill=pantone157-8] (solution1) at (1,0) {Solution};
\node[shape=rectangle,fill=pantone157-8] (solution2) at (2,0) {Solution};
\node[shape=rectangle,fill=pantone157-8] (solution3) at (3,0) {Solution};
\node[shape=rectangle,fill=pantone157-8] (solution4) at (4,0) {Solution};
\draw[-] (problem) -- (human);
\draw[->] (human) -- (model);
\draw[-] (model) -- (compile);
\draw[-] (compile) -- (solver1);
\draw[-] (compile) -- (solver2);
\draw[-] (compile) -- (solver3);
\draw[-] (compile) -- (solver4);
\draw[->] (solver1) -- (solution1);
\draw[->] (solver2) -- (solution2);
\draw[->] (solver3) -- (solution3);
\draw[->] (solver4) -- (solution4);
\end{tikzpicture}
\end{center}
\end{frame}

% \begin{frame}
  % \frametitle{Do It Now!}
  % \begin{itemize}
  % \item Download and install Minizinc
  % \item \url{https://www.minizinc.org/}
  % \end{itemize}
% \end{frame}


\begin{frame}
\frametitle{Why use Puzzles as Examples?}
\begin{itemize}
\item Easy to understand the problem
\item Solvable by hand without specialized knowledge
\item Possible to compare automated to manual solving process
\end{itemize}

\begin{quote}
The puzzle, though inanimate, is presented as a solvable problem without lasting negative consequences, a very low-risk low-reward situation. By being a puzzle, the object is attempting to convince the user that it must be completed.
\end{quote}
{\tiny Source: \href{https://www.everydayrhetoric.com/2020/03/the-rhetoric-of-puzzles/}{Every Day Rhetoric}}
\end{frame}


\mode<all>{
\part{Basic Constraint Propagation}
}
\begin{frame}
\partpage
\end{frame}
\mode<all>{
\input{../sendmore/body}
}

\mode<all>{
\part{Global Constraints}
}
\begin{frame}
\partpage
\end{frame}
\mode<all>{
\input{../sudoku/body}
}

\clearpage
\mode<all>{
\part{Customizing Search}
}
\begin{frame}
\partpage
\end{frame}
\mode<all>{
\input{../nqueen/body}
}


%% \clearpage
%% \mode<all>{
%%   \part{Problem Decomposition}
%% }
%% \begin{frame}
%%   \partpage
%% \end{frame}
%% \mode<all>{
%%   \input{../party/body}
%%   }

%% \clearpage
%% \mode<all>{
%%   \part{More Global Constraints}
%% }
%% \begin{frame}
%%   \partpage
%% \end{frame}
%% \mode<all>{
%%   \input{../car/body}
%%   }

%% \clearpage
%% \mode<all>{
%%   \part{Methodology}
%% }
%% \begin{frame}
%%   \partpage
%% \end{frame}
%% \mode<all>{
%%   \input{../methodology/body}
%%   }

\clearpage
\mode<all>{
\part{What is missing?}
}
\begin{frame}
\partpage
\end{frame}

\begin{frame}
\frametitle{Many Specialized Topics}
\begin{itemize}
\item How to design efficient core engine
\item Hybrids with LP/MIP tools
\item Hybrids with SAT
\item Symmetry breaking
\item Use of MDD/BDD to encode sets of solutions
\item High level modelling tools
\item Debugging/visualization
\end{itemize}
\end{frame}

\begin{frame}
\frametitle{Reformulation}
\begin{itemize}
\item Just because the user has modelled it this way, it doesn't mean we have to solve it that way
\begin{itemize}
\item Replace some constraint(s) by other, equivalent constraints
\item Because we don't have that constraint in our system
\item For performance
\end{itemize}
\end{itemize}
\end{frame}

\begin{frame}
\frametitle{Learning}
\begin{itemize}
\item While solving the problem we can learn how to strengthen the model/search 
\begin{itemize}
\item Understand which constraints/method contribute to propagation and change schedule
\item Learn no-good constraints by explaining failure
\item Adapt search strategy based on search experience
\end{itemize}
\end{itemize}
\end{frame}

%% \begin{frame}
%% \frametitle{Refined Process}
%% \begin{center}
%% \begin{tikzpicture}
%% \node[shape=rectangle,fill=pantone127-4] (problem) at (3,7) {Problem};
%% \node (human) at (3,6) {Human};
%% \node[shape=rectangle,fill=pantone127-4] (model) at (3,5) {User Model};
%% \node (reformulation) at (3,4) {Reformulation};
%% \node[shape=rectangle,fill=pantone127-4] (implementation) at (3,3) {Implementation Model};
%% \node (solver) at (3,2) {Constraint Solver/Search};
%% \node[shape=rectangle,fill=pantone157-8] (solution) at (3,1) {Solution};
%% \draw[-] (problem) -- (human);
%% \draw[->] (human) -- (model);
%% \draw[-] (model) -- (reformulation);
%% \draw[->] (reformulation) -- (implementation);
%% \draw[-] (implementation) -- (solver);
%% \draw[->] (solver) -- (0,2) -- (0,3) -- (implementation);
%% \draw[->] (solver) -- (solution);
%% \end{tikzpicture}
%% \end{center}
%% \end{frame}

%\begin{frame}
%\frametitle{What is CP actually used for?}
%\url{http://hsimonis.wordpress.com}
%\includegraphics[width=10cm]{../introduction/images/blog}
%\end{frame}

%% \begin{frame}
%% \frametitle{Exercises}
%% \begin{itemize}
%% \item Install Constraint Solver of your Choice
%% \item MiniZinc (MiningZinc) will be used in later sessions
%% \item Run Sendmoremoney, Sudoku, Queens examples
%% \item Can you choose which propagator to use for alldifferent?
%% \item Can you use different search strategies?
%% \end{itemize}
%% \end{frame}

\begin{frame}
\frametitle{More Learning Resources}
\begin{itemize}
\item Survey of Methods, Resources, and Formats for Teaching Constraint Programming
\begin{itemize}
\item by Tejas Santanam, Helmut Simonis
\item \url{https://doi.org/10.48550/arXiv.2403.12717}
\item Based on survey of community for WTCP 2023
\item \url{https://hsimonis.github.io/WTCP2023/}
\end{itemize}
\end{itemize}
\end{frame}


%%   }

%% \clearpage
%% \mode<all>{
%%   \part{Methodology}
%% }
%% \begin{frame}
%%   \partpage
%% \end{frame}
%% \mode<all>{
%%   \begin{frame}
\frametitle{Objectives}
\begin{itemize}
\item Overview of Core Constraint Programming
\item Three Main Concepts
\begin{itemize}
\item Constraint Propagation
\item Global Constraints
\item Customizing Search
\end{itemize}
%\item Get Some Experience with MiniZinc
\item Topics will be treated in more detail in later parts of the school
\item Based on Examples, not Formal Description
\end{itemize}
\end{frame}

\begin{frame}
\frametitle{Outline}
\begin{itemize}
\item Why Constraint Programming?
\item Constraint Propagation
\item Global Constraints
\item Customizing Search
%\item Choices in Modelling
%\item Decomposition
%\item More Global Constraints
%\item Methodology
%\item What is missing?
\end{itemize}
\end{frame}

% \begin{frame}
  % \frametitle{Using MiniZinc IDE}
  % \begin{itemize}
  % \item Developed in the Australian NICTA project
  % \item Maintained by Monash University
  % \item Modelling tool with multiple back-end solvers
  % \item Available from \url{https://www.minizinc.org/}
  % \end{itemize}
% \end{frame}


% \begin{frame}
% \frametitle{Examples in ECLiPSe}
% \begin{itemize}
% \item Open sourced constraint programming language
% \item Development goes back to 1985
% \item ECRC, ICL, IC-Parc, PTL, Cisco
% \item \url{https://eclipseclp.org/}
% \item Specialities
% \begin{itemize}
% \item Develop new solvers for specific domains
% \item Integration with MIP
% \end{itemize}
% \item Not included in bundled MiniZinc IDE
% \item Specialized visualization tools used here
  % \begin{itemize}
  % \item CP-Viz, Simonis et al. 2010
    % \end{itemize}
% \end{itemize}
% \end{frame}

\begin{frame}
\frametitle{Tutorial Based on ECLiPSe ELearning Course}
\begin{itemize}
\item Self-study course in constraint programming
\item Supported by Cisco Systems and Silicon Valley Community Foundation
\item Multi-media format, video lectures, slides, handout etc
\item {\small \url{https://eclipseclp.org/ELearning/index.html}}
\end{itemize}
\end{frame}

\begin{frame}
\frametitle{Also Part of CRT-AI Constraint Week}
\begin{itemize}
\item Annual one week course on CP and Optimization in Ireland
\item Part of national training program for PhD students in AI
\item \url{https://www.crt-ai.ie/}
\end{itemize}
\end{frame}



\begin{frame}
\frametitle{Constraint Programming - in a nutshell}
\begin{itemize}
\item Declarative description of problems with
\begin{itemize}
\item {\em Variables} which range over (finite) sets of values
\item {\em Constraints} over subsets of variables which restrict possible value combinations
\item A {\em solution} is a value assignment which satisfies all constraints
\end{itemize}

\item Constraint propagation/reasoning
\begin{itemize}
\item Removing inconsistent values for variables
\item Detect failure if constraint can not be satisfied
\item Interaction of constraints via shared variables
\item Incomplete
\end{itemize}

\item Search
\begin{itemize}
\item User controlled assignment of values to variables
\item Each step triggers constraint propagation 
\end{itemize}
\item Different domains require/allow different methods
\end{itemize}
\end{frame}

\begin{frame}
  \frametitle{Constraint Programming is Different}
  \begin{itemize}
  \item Declarative Programming
    \begin{itemize}
    \item Concentrate on what you want
      \item Not how to get there
      \item Program != Algorithm
      \item Program = Model
    \end{itemize}
    \item Applied to Combinatorial Problems
      \begin{itemize}
        \item No complete polynomial algorithms known (exist?)
        \item CP less ad-hoc than heuristics
        \item Models can evolve
  \end{itemize}
  \end{itemize}
  \end{frame}
    
\begin{frame}
  \frametitle{A Subtractive Process}
  \begin{textblock}{4}(8,-3)
    \includegraphics[width=4cm]{../introduction/images/stages}
  \end{textblock}
  \vfill
  \begin{quote}
    ``Oh, bosh, as Mr. Ruskin says. Sculpture, per se, is the simplest thing in the world. All you have to do is to take a big chunk of marble and a hammer and chisel, make up your mind what you are about to create and chip off all the marble you don't want.''-Paris Gaulois.
  \end{quote}
  
  {\tiny Source: \url{https://quoteinvestigator.com/2014/06/22/chip-away/}}
\end{frame}

\begin{frame}
\frametitle{Basic Process}
\begin{center}
\begin{tikzpicture}
\node[shape=rectangle,fill=pantone127-4] (problem) at (2,5) {Problem};
\node (human) at (2,4) {Human};
\node[shape=rectangle,fill=pantone127-4] (model) at (2,3) {Model};
\node (solver) at (2,2) {Constraint Solver/Search};
\node[shape=rectangle,fill=pantone157-8] (solution) at (2,1) {Solution};
\draw[-] (problem) -- (human);
\draw[->] (human) -- (model);
\draw[-] (model) -- (solver);
\draw[->] (solver) -- (solution);
\end{tikzpicture}
\end{center}
\end{frame}

\begin{frame}
\frametitle{More Realistic}
\begin{center}
\begin{tikzpicture}[xscale=4.0]
\node[shape=rectangle,fill=pantone127-4] (problem) at (2,5) {Problem};
\node (human) at (2,4) {Human};
\node[shape=rectangle,fill=pantone127-4] (model) at (2,3) {Model};
\node (solver) at (2,2) {Constraint Solver/Search};
\node[shape=rectangle,fill=pantone24-8] (hangs) at (1,1) {Hangs};
\node[shape=rectangle,fill=pantone157-8] (solution) at (2,1) {Solution};
\node[shape=rectangle,fill=pantone24-8] (wrong) at (3,1) {Wrong Solution};
\draw[-] (problem) -- (human);
\draw[->] (human) -- (model);
\draw[-] (model) -- (solver);
\draw[->] (solver) -- (hangs);
\draw[->] (solver) -- (solution);
\draw[->] (solver) -- (wrong);
\draw[->] (hangs) -- (1,4) -- (human);
\draw[->] (wrong) -- (3,4) -- (human);
\end{tikzpicture}
\end{center}
\end{frame}

\begin{frame}
\frametitle{Dual Role of Model}
\begin{itemize}
\item Allows Human to Express Problem
\begin{itemize}
\item Close to Problem Domain
\item Constraints as Abstractions
\end{itemize}
\item Allows Solver to Execute
\begin{itemize}
\item Variables as Communication Mechanism
\item Constraints as Algorithms
\end{itemize}
\end{itemize}
\end{frame}

\begin{frame}
\frametitle{Modelling Frameworks}
\begin{itemize}
\item \href{https://www.minizinc.org/}{MiniZinc} (NICTA, Monash University, Australia)
\item \href{https://github.com/eomahony/Numberjack}{NumberJack} (Insight, Ireland) 
\item \href{https://www-users.york.ac.uk/peter.nightingale/savilerow//}{EssencePrime/SavilleRow} (UK) 
\item \href{https://cpmpy.readthedocs.io/en/latest/}{CPMpy} (KU Leuven)
\item Allow use of multiple back-end solvers
\item Compile model into variants for each solver
\item A priori solver independent model(CP, MIP, SAT)
\end{itemize}
\end{frame} 

\begin{frame}
\frametitle{Framework Process}
\begin{center}
\begin{tikzpicture}[xscale=2]
\node[shape=rectangle,fill=pantone127-4] (problem) at (2.5,5) {Problem};
\node (human) at (2.5,4) {Human};
\node[shape=rectangle,fill=pantone127-4] (model) at (2.5,3) {Model};
\node (compile) at (2.5,2) {Compile/Reformulate};
\node (solver1) at (1,1) {CP};
\node (solver2) at (2,1) {MIP};
\node (solver3) at (3,1) {SAT};
\node (solver4) at (4,1) {Other};
\node[shape=rectangle,fill=pantone157-8] (solution1) at (1,0) {Solution};
\node[shape=rectangle,fill=pantone157-8] (solution2) at (2,0) {Solution};
\node[shape=rectangle,fill=pantone157-8] (solution3) at (3,0) {Solution};
\node[shape=rectangle,fill=pantone157-8] (solution4) at (4,0) {Solution};
\draw[-] (problem) -- (human);
\draw[->] (human) -- (model);
\draw[-] (model) -- (compile);
\draw[-] (compile) -- (solver1);
\draw[-] (compile) -- (solver2);
\draw[-] (compile) -- (solver3);
\draw[-] (compile) -- (solver4);
\draw[->] (solver1) -- (solution1);
\draw[->] (solver2) -- (solution2);
\draw[->] (solver3) -- (solution3);
\draw[->] (solver4) -- (solution4);
\end{tikzpicture}
\end{center}
\end{frame}

% \begin{frame}
  % \frametitle{Do It Now!}
  % \begin{itemize}
  % \item Download and install Minizinc
  % \item \url{https://www.minizinc.org/}
  % \end{itemize}
% \end{frame}


\begin{frame}
\frametitle{Why use Puzzles as Examples?}
\begin{itemize}
\item Easy to understand the problem
\item Solvable by hand without specialized knowledge
\item Possible to compare automated to manual solving process
\end{itemize}

\begin{quote}
The puzzle, though inanimate, is presented as a solvable problem without lasting negative consequences, a very low-risk low-reward situation. By being a puzzle, the object is attempting to convince the user that it must be completed.
\end{quote}
{\tiny Source: \href{https://www.everydayrhetoric.com/2020/03/the-rhetoric-of-puzzles/}{Every Day Rhetoric}}
\end{frame}


\mode<all>{
\part{Basic Constraint Propagation}
}
\begin{frame}
\partpage
\end{frame}
\mode<all>{
\input{../sendmore/body}
}

\mode<all>{
\part{Global Constraints}
}
\begin{frame}
\partpage
\end{frame}
\mode<all>{
\input{../sudoku/body}
}

\clearpage
\mode<all>{
\part{Customizing Search}
}
\begin{frame}
\partpage
\end{frame}
\mode<all>{
\input{../nqueen/body}
}


%% \clearpage
%% \mode<all>{
%%   \part{Problem Decomposition}
%% }
%% \begin{frame}
%%   \partpage
%% \end{frame}
%% \mode<all>{
%%   \input{../party/body}
%%   }

%% \clearpage
%% \mode<all>{
%%   \part{More Global Constraints}
%% }
%% \begin{frame}
%%   \partpage
%% \end{frame}
%% \mode<all>{
%%   \input{../car/body}
%%   }

%% \clearpage
%% \mode<all>{
%%   \part{Methodology}
%% }
%% \begin{frame}
%%   \partpage
%% \end{frame}
%% \mode<all>{
%%   \input{../methodology/body}
%%   }

\clearpage
\mode<all>{
\part{What is missing?}
}
\begin{frame}
\partpage
\end{frame}

\begin{frame}
\frametitle{Many Specialized Topics}
\begin{itemize}
\item How to design efficient core engine
\item Hybrids with LP/MIP tools
\item Hybrids with SAT
\item Symmetry breaking
\item Use of MDD/BDD to encode sets of solutions
\item High level modelling tools
\item Debugging/visualization
\end{itemize}
\end{frame}

\begin{frame}
\frametitle{Reformulation}
\begin{itemize}
\item Just because the user has modelled it this way, it doesn't mean we have to solve it that way
\begin{itemize}
\item Replace some constraint(s) by other, equivalent constraints
\item Because we don't have that constraint in our system
\item For performance
\end{itemize}
\end{itemize}
\end{frame}

\begin{frame}
\frametitle{Learning}
\begin{itemize}
\item While solving the problem we can learn how to strengthen the model/search 
\begin{itemize}
\item Understand which constraints/method contribute to propagation and change schedule
\item Learn no-good constraints by explaining failure
\item Adapt search strategy based on search experience
\end{itemize}
\end{itemize}
\end{frame}

%% \begin{frame}
%% \frametitle{Refined Process}
%% \begin{center}
%% \begin{tikzpicture}
%% \node[shape=rectangle,fill=pantone127-4] (problem) at (3,7) {Problem};
%% \node (human) at (3,6) {Human};
%% \node[shape=rectangle,fill=pantone127-4] (model) at (3,5) {User Model};
%% \node (reformulation) at (3,4) {Reformulation};
%% \node[shape=rectangle,fill=pantone127-4] (implementation) at (3,3) {Implementation Model};
%% \node (solver) at (3,2) {Constraint Solver/Search};
%% \node[shape=rectangle,fill=pantone157-8] (solution) at (3,1) {Solution};
%% \draw[-] (problem) -- (human);
%% \draw[->] (human) -- (model);
%% \draw[-] (model) -- (reformulation);
%% \draw[->] (reformulation) -- (implementation);
%% \draw[-] (implementation) -- (solver);
%% \draw[->] (solver) -- (0,2) -- (0,3) -- (implementation);
%% \draw[->] (solver) -- (solution);
%% \end{tikzpicture}
%% \end{center}
%% \end{frame}

%\begin{frame}
%\frametitle{What is CP actually used for?}
%\url{http://hsimonis.wordpress.com}
%\includegraphics[width=10cm]{../introduction/images/blog}
%\end{frame}

%% \begin{frame}
%% \frametitle{Exercises}
%% \begin{itemize}
%% \item Install Constraint Solver of your Choice
%% \item MiniZinc (MiningZinc) will be used in later sessions
%% \item Run Sendmoremoney, Sudoku, Queens examples
%% \item Can you choose which propagator to use for alldifferent?
%% \item Can you use different search strategies?
%% \end{itemize}
%% \end{frame}

\begin{frame}
\frametitle{More Learning Resources}
\begin{itemize}
\item Survey of Methods, Resources, and Formats for Teaching Constraint Programming
\begin{itemize}
\item by Tejas Santanam, Helmut Simonis
\item \url{https://doi.org/10.48550/arXiv.2403.12717}
\item Based on survey of community for WTCP 2023
\item \url{https://hsimonis.github.io/WTCP2023/}
\end{itemize}
\end{itemize}
\end{frame}


%%   }

\clearpage
\mode<all>{
\part{What is missing?}
}
\begin{frame}
\partpage
\end{frame}

\begin{frame}
\frametitle{Many Specialized Topics}
\begin{itemize}
\item How to design efficient core engine
\item Hybrids with LP/MIP tools
\item Hybrids with SAT
\item Symmetry breaking
\item Use of MDD/BDD to encode sets of solutions
\item High level modelling tools
\item Debugging/visualization
\end{itemize}
\end{frame}

\begin{frame}
\frametitle{Reformulation}
\begin{itemize}
\item Just because the user has modelled it this way, it doesn't mean we have to solve it that way
\begin{itemize}
\item Replace some constraint(s) by other, equivalent constraints
\item Because we don't have that constraint in our system
\item For performance
\end{itemize}
\end{itemize}
\end{frame}

\begin{frame}
\frametitle{Learning}
\begin{itemize}
\item While solving the problem we can learn how to strengthen the model/search 
\begin{itemize}
\item Understand which constraints/method contribute to propagation and change schedule
\item Learn no-good constraints by explaining failure
\item Adapt search strategy based on search experience
\end{itemize}
\end{itemize}
\end{frame}

%% \begin{frame}
%% \frametitle{Refined Process}
%% \begin{center}
%% \begin{tikzpicture}
%% \node[shape=rectangle,fill=pantone127-4] (problem) at (3,7) {Problem};
%% \node (human) at (3,6) {Human};
%% \node[shape=rectangle,fill=pantone127-4] (model) at (3,5) {User Model};
%% \node (reformulation) at (3,4) {Reformulation};
%% \node[shape=rectangle,fill=pantone127-4] (implementation) at (3,3) {Implementation Model};
%% \node (solver) at (3,2) {Constraint Solver/Search};
%% \node[shape=rectangle,fill=pantone157-8] (solution) at (3,1) {Solution};
%% \draw[-] (problem) -- (human);
%% \draw[->] (human) -- (model);
%% \draw[-] (model) -- (reformulation);
%% \draw[->] (reformulation) -- (implementation);
%% \draw[-] (implementation) -- (solver);
%% \draw[->] (solver) -- (0,2) -- (0,3) -- (implementation);
%% \draw[->] (solver) -- (solution);
%% \end{tikzpicture}
%% \end{center}
%% \end{frame}

%\begin{frame}
%\frametitle{What is CP actually used for?}
%\url{http://hsimonis.wordpress.com}
%\includegraphics[width=10cm]{../introduction/images/blog}
%\end{frame}

%% \begin{frame}
%% \frametitle{Exercises}
%% \begin{itemize}
%% \item Install Constraint Solver of your Choice
%% \item MiniZinc (MiningZinc) will be used in later sessions
%% \item Run Sendmoremoney, Sudoku, Queens examples
%% \item Can you choose which propagator to use for alldifferent?
%% \item Can you use different search strategies?
%% \end{itemize}
%% \end{frame}

\begin{frame}
\frametitle{More Learning Resources}
\begin{itemize}
\item Survey of Methods, Resources, and Formats for Teaching Constraint Programming
\begin{itemize}
\item by Tejas Santanam, Helmut Simonis
\item \url{https://doi.org/10.48550/arXiv.2403.12717}
\item Based on survey of community for WTCP 2023
\item \url{https://hsimonis.github.io/WTCP2023/}
\end{itemize}
\end{itemize}
\end{frame}


}


%% \clearpage
%% \mode<all>{
%%   \part{Problem Decomposition}
%% }
%% \begin{frame}
%%   \partpage
%% \end{frame}
%% \mode<all>{
%%   \begin{frame}
\frametitle{Objectives}
\begin{itemize}
\item Overview of Core Constraint Programming
\item Three Main Concepts
\begin{itemize}
\item Constraint Propagation
\item Global Constraints
\item Customizing Search
\end{itemize}
%\item Get Some Experience with MiniZinc
\item Topics will be treated in more detail in later parts of the school
\item Based on Examples, not Formal Description
\end{itemize}
\end{frame}

\begin{frame}
\frametitle{Outline}
\begin{itemize}
\item Why Constraint Programming?
\item Constraint Propagation
\item Global Constraints
\item Customizing Search
%\item Choices in Modelling
%\item Decomposition
%\item More Global Constraints
%\item Methodology
%\item What is missing?
\end{itemize}
\end{frame}

% \begin{frame}
  % \frametitle{Using MiniZinc IDE}
  % \begin{itemize}
  % \item Developed in the Australian NICTA project
  % \item Maintained by Monash University
  % \item Modelling tool with multiple back-end solvers
  % \item Available from \url{https://www.minizinc.org/}
  % \end{itemize}
% \end{frame}


% \begin{frame}
% \frametitle{Examples in ECLiPSe}
% \begin{itemize}
% \item Open sourced constraint programming language
% \item Development goes back to 1985
% \item ECRC, ICL, IC-Parc, PTL, Cisco
% \item \url{https://eclipseclp.org/}
% \item Specialities
% \begin{itemize}
% \item Develop new solvers for specific domains
% \item Integration with MIP
% \end{itemize}
% \item Not included in bundled MiniZinc IDE
% \item Specialized visualization tools used here
  % \begin{itemize}
  % \item CP-Viz, Simonis et al. 2010
    % \end{itemize}
% \end{itemize}
% \end{frame}

\begin{frame}
\frametitle{Tutorial Based on ECLiPSe ELearning Course}
\begin{itemize}
\item Self-study course in constraint programming
\item Supported by Cisco Systems and Silicon Valley Community Foundation
\item Multi-media format, video lectures, slides, handout etc
\item {\small \url{https://eclipseclp.org/ELearning/index.html}}
\end{itemize}
\end{frame}

\begin{frame}
\frametitle{Also Part of CRT-AI Constraint Week}
\begin{itemize}
\item Annual one week course on CP and Optimization in Ireland
\item Part of national training program for PhD students in AI
\item \url{https://www.crt-ai.ie/}
\end{itemize}
\end{frame}



\begin{frame}
\frametitle{Constraint Programming - in a nutshell}
\begin{itemize}
\item Declarative description of problems with
\begin{itemize}
\item {\em Variables} which range over (finite) sets of values
\item {\em Constraints} over subsets of variables which restrict possible value combinations
\item A {\em solution} is a value assignment which satisfies all constraints
\end{itemize}

\item Constraint propagation/reasoning
\begin{itemize}
\item Removing inconsistent values for variables
\item Detect failure if constraint can not be satisfied
\item Interaction of constraints via shared variables
\item Incomplete
\end{itemize}

\item Search
\begin{itemize}
\item User controlled assignment of values to variables
\item Each step triggers constraint propagation 
\end{itemize}
\item Different domains require/allow different methods
\end{itemize}
\end{frame}

\begin{frame}
  \frametitle{Constraint Programming is Different}
  \begin{itemize}
  \item Declarative Programming
    \begin{itemize}
    \item Concentrate on what you want
      \item Not how to get there
      \item Program != Algorithm
      \item Program = Model
    \end{itemize}
    \item Applied to Combinatorial Problems
      \begin{itemize}
        \item No complete polynomial algorithms known (exist?)
        \item CP less ad-hoc than heuristics
        \item Models can evolve
  \end{itemize}
  \end{itemize}
  \end{frame}
    
\begin{frame}
  \frametitle{A Subtractive Process}
  \begin{textblock}{4}(8,-3)
    \includegraphics[width=4cm]{../introduction/images/stages}
  \end{textblock}
  \vfill
  \begin{quote}
    ``Oh, bosh, as Mr. Ruskin says. Sculpture, per se, is the simplest thing in the world. All you have to do is to take a big chunk of marble and a hammer and chisel, make up your mind what you are about to create and chip off all the marble you don't want.''-Paris Gaulois.
  \end{quote}
  
  {\tiny Source: \url{https://quoteinvestigator.com/2014/06/22/chip-away/}}
\end{frame}

\begin{frame}
\frametitle{Basic Process}
\begin{center}
\begin{tikzpicture}
\node[shape=rectangle,fill=pantone127-4] (problem) at (2,5) {Problem};
\node (human) at (2,4) {Human};
\node[shape=rectangle,fill=pantone127-4] (model) at (2,3) {Model};
\node (solver) at (2,2) {Constraint Solver/Search};
\node[shape=rectangle,fill=pantone157-8] (solution) at (2,1) {Solution};
\draw[-] (problem) -- (human);
\draw[->] (human) -- (model);
\draw[-] (model) -- (solver);
\draw[->] (solver) -- (solution);
\end{tikzpicture}
\end{center}
\end{frame}

\begin{frame}
\frametitle{More Realistic}
\begin{center}
\begin{tikzpicture}[xscale=4.0]
\node[shape=rectangle,fill=pantone127-4] (problem) at (2,5) {Problem};
\node (human) at (2,4) {Human};
\node[shape=rectangle,fill=pantone127-4] (model) at (2,3) {Model};
\node (solver) at (2,2) {Constraint Solver/Search};
\node[shape=rectangle,fill=pantone24-8] (hangs) at (1,1) {Hangs};
\node[shape=rectangle,fill=pantone157-8] (solution) at (2,1) {Solution};
\node[shape=rectangle,fill=pantone24-8] (wrong) at (3,1) {Wrong Solution};
\draw[-] (problem) -- (human);
\draw[->] (human) -- (model);
\draw[-] (model) -- (solver);
\draw[->] (solver) -- (hangs);
\draw[->] (solver) -- (solution);
\draw[->] (solver) -- (wrong);
\draw[->] (hangs) -- (1,4) -- (human);
\draw[->] (wrong) -- (3,4) -- (human);
\end{tikzpicture}
\end{center}
\end{frame}

\begin{frame}
\frametitle{Dual Role of Model}
\begin{itemize}
\item Allows Human to Express Problem
\begin{itemize}
\item Close to Problem Domain
\item Constraints as Abstractions
\end{itemize}
\item Allows Solver to Execute
\begin{itemize}
\item Variables as Communication Mechanism
\item Constraints as Algorithms
\end{itemize}
\end{itemize}
\end{frame}

\begin{frame}
\frametitle{Modelling Frameworks}
\begin{itemize}
\item \href{https://www.minizinc.org/}{MiniZinc} (NICTA, Monash University, Australia)
\item \href{https://github.com/eomahony/Numberjack}{NumberJack} (Insight, Ireland) 
\item \href{https://www-users.york.ac.uk/peter.nightingale/savilerow//}{EssencePrime/SavilleRow} (UK) 
\item \href{https://cpmpy.readthedocs.io/en/latest/}{CPMpy} (KU Leuven)
\item Allow use of multiple back-end solvers
\item Compile model into variants for each solver
\item A priori solver independent model(CP, MIP, SAT)
\end{itemize}
\end{frame} 

\begin{frame}
\frametitle{Framework Process}
\begin{center}
\begin{tikzpicture}[xscale=2]
\node[shape=rectangle,fill=pantone127-4] (problem) at (2.5,5) {Problem};
\node (human) at (2.5,4) {Human};
\node[shape=rectangle,fill=pantone127-4] (model) at (2.5,3) {Model};
\node (compile) at (2.5,2) {Compile/Reformulate};
\node (solver1) at (1,1) {CP};
\node (solver2) at (2,1) {MIP};
\node (solver3) at (3,1) {SAT};
\node (solver4) at (4,1) {Other};
\node[shape=rectangle,fill=pantone157-8] (solution1) at (1,0) {Solution};
\node[shape=rectangle,fill=pantone157-8] (solution2) at (2,0) {Solution};
\node[shape=rectangle,fill=pantone157-8] (solution3) at (3,0) {Solution};
\node[shape=rectangle,fill=pantone157-8] (solution4) at (4,0) {Solution};
\draw[-] (problem) -- (human);
\draw[->] (human) -- (model);
\draw[-] (model) -- (compile);
\draw[-] (compile) -- (solver1);
\draw[-] (compile) -- (solver2);
\draw[-] (compile) -- (solver3);
\draw[-] (compile) -- (solver4);
\draw[->] (solver1) -- (solution1);
\draw[->] (solver2) -- (solution2);
\draw[->] (solver3) -- (solution3);
\draw[->] (solver4) -- (solution4);
\end{tikzpicture}
\end{center}
\end{frame}

% \begin{frame}
  % \frametitle{Do It Now!}
  % \begin{itemize}
  % \item Download and install Minizinc
  % \item \url{https://www.minizinc.org/}
  % \end{itemize}
% \end{frame}


\begin{frame}
\frametitle{Why use Puzzles as Examples?}
\begin{itemize}
\item Easy to understand the problem
\item Solvable by hand without specialized knowledge
\item Possible to compare automated to manual solving process
\end{itemize}

\begin{quote}
The puzzle, though inanimate, is presented as a solvable problem without lasting negative consequences, a very low-risk low-reward situation. By being a puzzle, the object is attempting to convince the user that it must be completed.
\end{quote}
{\tiny Source: \href{https://www.everydayrhetoric.com/2020/03/the-rhetoric-of-puzzles/}{Every Day Rhetoric}}
\end{frame}


\mode<all>{
\part{Basic Constraint Propagation}
}
\begin{frame}
\partpage
\end{frame}
\mode<all>{
\begin{frame}
\frametitle{Objectives}
\begin{itemize}
\item Overview of Core Constraint Programming
\item Three Main Concepts
\begin{itemize}
\item Constraint Propagation
\item Global Constraints
\item Customizing Search
\end{itemize}
%\item Get Some Experience with MiniZinc
\item Topics will be treated in more detail in later parts of the school
\item Based on Examples, not Formal Description
\end{itemize}
\end{frame}

\begin{frame}
\frametitle{Outline}
\begin{itemize}
\item Why Constraint Programming?
\item Constraint Propagation
\item Global Constraints
\item Customizing Search
%\item Choices in Modelling
%\item Decomposition
%\item More Global Constraints
%\item Methodology
%\item What is missing?
\end{itemize}
\end{frame}

% \begin{frame}
  % \frametitle{Using MiniZinc IDE}
  % \begin{itemize}
  % \item Developed in the Australian NICTA project
  % \item Maintained by Monash University
  % \item Modelling tool with multiple back-end solvers
  % \item Available from \url{https://www.minizinc.org/}
  % \end{itemize}
% \end{frame}


% \begin{frame}
% \frametitle{Examples in ECLiPSe}
% \begin{itemize}
% \item Open sourced constraint programming language
% \item Development goes back to 1985
% \item ECRC, ICL, IC-Parc, PTL, Cisco
% \item \url{https://eclipseclp.org/}
% \item Specialities
% \begin{itemize}
% \item Develop new solvers for specific domains
% \item Integration with MIP
% \end{itemize}
% \item Not included in bundled MiniZinc IDE
% \item Specialized visualization tools used here
  % \begin{itemize}
  % \item CP-Viz, Simonis et al. 2010
    % \end{itemize}
% \end{itemize}
% \end{frame}

\begin{frame}
\frametitle{Tutorial Based on ECLiPSe ELearning Course}
\begin{itemize}
\item Self-study course in constraint programming
\item Supported by Cisco Systems and Silicon Valley Community Foundation
\item Multi-media format, video lectures, slides, handout etc
\item {\small \url{https://eclipseclp.org/ELearning/index.html}}
\end{itemize}
\end{frame}

\begin{frame}
\frametitle{Also Part of CRT-AI Constraint Week}
\begin{itemize}
\item Annual one week course on CP and Optimization in Ireland
\item Part of national training program for PhD students in AI
\item \url{https://www.crt-ai.ie/}
\end{itemize}
\end{frame}



\begin{frame}
\frametitle{Constraint Programming - in a nutshell}
\begin{itemize}
\item Declarative description of problems with
\begin{itemize}
\item {\em Variables} which range over (finite) sets of values
\item {\em Constraints} over subsets of variables which restrict possible value combinations
\item A {\em solution} is a value assignment which satisfies all constraints
\end{itemize}

\item Constraint propagation/reasoning
\begin{itemize}
\item Removing inconsistent values for variables
\item Detect failure if constraint can not be satisfied
\item Interaction of constraints via shared variables
\item Incomplete
\end{itemize}

\item Search
\begin{itemize}
\item User controlled assignment of values to variables
\item Each step triggers constraint propagation 
\end{itemize}
\item Different domains require/allow different methods
\end{itemize}
\end{frame}

\begin{frame}
  \frametitle{Constraint Programming is Different}
  \begin{itemize}
  \item Declarative Programming
    \begin{itemize}
    \item Concentrate on what you want
      \item Not how to get there
      \item Program != Algorithm
      \item Program = Model
    \end{itemize}
    \item Applied to Combinatorial Problems
      \begin{itemize}
        \item No complete polynomial algorithms known (exist?)
        \item CP less ad-hoc than heuristics
        \item Models can evolve
  \end{itemize}
  \end{itemize}
  \end{frame}
    
\begin{frame}
  \frametitle{A Subtractive Process}
  \begin{textblock}{4}(8,-3)
    \includegraphics[width=4cm]{../introduction/images/stages}
  \end{textblock}
  \vfill
  \begin{quote}
    ``Oh, bosh, as Mr. Ruskin says. Sculpture, per se, is the simplest thing in the world. All you have to do is to take a big chunk of marble and a hammer and chisel, make up your mind what you are about to create and chip off all the marble you don't want.''-Paris Gaulois.
  \end{quote}
  
  {\tiny Source: \url{https://quoteinvestigator.com/2014/06/22/chip-away/}}
\end{frame}

\begin{frame}
\frametitle{Basic Process}
\begin{center}
\begin{tikzpicture}
\node[shape=rectangle,fill=pantone127-4] (problem) at (2,5) {Problem};
\node (human) at (2,4) {Human};
\node[shape=rectangle,fill=pantone127-4] (model) at (2,3) {Model};
\node (solver) at (2,2) {Constraint Solver/Search};
\node[shape=rectangle,fill=pantone157-8] (solution) at (2,1) {Solution};
\draw[-] (problem) -- (human);
\draw[->] (human) -- (model);
\draw[-] (model) -- (solver);
\draw[->] (solver) -- (solution);
\end{tikzpicture}
\end{center}
\end{frame}

\begin{frame}
\frametitle{More Realistic}
\begin{center}
\begin{tikzpicture}[xscale=4.0]
\node[shape=rectangle,fill=pantone127-4] (problem) at (2,5) {Problem};
\node (human) at (2,4) {Human};
\node[shape=rectangle,fill=pantone127-4] (model) at (2,3) {Model};
\node (solver) at (2,2) {Constraint Solver/Search};
\node[shape=rectangle,fill=pantone24-8] (hangs) at (1,1) {Hangs};
\node[shape=rectangle,fill=pantone157-8] (solution) at (2,1) {Solution};
\node[shape=rectangle,fill=pantone24-8] (wrong) at (3,1) {Wrong Solution};
\draw[-] (problem) -- (human);
\draw[->] (human) -- (model);
\draw[-] (model) -- (solver);
\draw[->] (solver) -- (hangs);
\draw[->] (solver) -- (solution);
\draw[->] (solver) -- (wrong);
\draw[->] (hangs) -- (1,4) -- (human);
\draw[->] (wrong) -- (3,4) -- (human);
\end{tikzpicture}
\end{center}
\end{frame}

\begin{frame}
\frametitle{Dual Role of Model}
\begin{itemize}
\item Allows Human to Express Problem
\begin{itemize}
\item Close to Problem Domain
\item Constraints as Abstractions
\end{itemize}
\item Allows Solver to Execute
\begin{itemize}
\item Variables as Communication Mechanism
\item Constraints as Algorithms
\end{itemize}
\end{itemize}
\end{frame}

\begin{frame}
\frametitle{Modelling Frameworks}
\begin{itemize}
\item \href{https://www.minizinc.org/}{MiniZinc} (NICTA, Monash University, Australia)
\item \href{https://github.com/eomahony/Numberjack}{NumberJack} (Insight, Ireland) 
\item \href{https://www-users.york.ac.uk/peter.nightingale/savilerow//}{EssencePrime/SavilleRow} (UK) 
\item \href{https://cpmpy.readthedocs.io/en/latest/}{CPMpy} (KU Leuven)
\item Allow use of multiple back-end solvers
\item Compile model into variants for each solver
\item A priori solver independent model(CP, MIP, SAT)
\end{itemize}
\end{frame} 

\begin{frame}
\frametitle{Framework Process}
\begin{center}
\begin{tikzpicture}[xscale=2]
\node[shape=rectangle,fill=pantone127-4] (problem) at (2.5,5) {Problem};
\node (human) at (2.5,4) {Human};
\node[shape=rectangle,fill=pantone127-4] (model) at (2.5,3) {Model};
\node (compile) at (2.5,2) {Compile/Reformulate};
\node (solver1) at (1,1) {CP};
\node (solver2) at (2,1) {MIP};
\node (solver3) at (3,1) {SAT};
\node (solver4) at (4,1) {Other};
\node[shape=rectangle,fill=pantone157-8] (solution1) at (1,0) {Solution};
\node[shape=rectangle,fill=pantone157-8] (solution2) at (2,0) {Solution};
\node[shape=rectangle,fill=pantone157-8] (solution3) at (3,0) {Solution};
\node[shape=rectangle,fill=pantone157-8] (solution4) at (4,0) {Solution};
\draw[-] (problem) -- (human);
\draw[->] (human) -- (model);
\draw[-] (model) -- (compile);
\draw[-] (compile) -- (solver1);
\draw[-] (compile) -- (solver2);
\draw[-] (compile) -- (solver3);
\draw[-] (compile) -- (solver4);
\draw[->] (solver1) -- (solution1);
\draw[->] (solver2) -- (solution2);
\draw[->] (solver3) -- (solution3);
\draw[->] (solver4) -- (solution4);
\end{tikzpicture}
\end{center}
\end{frame}

% \begin{frame}
  % \frametitle{Do It Now!}
  % \begin{itemize}
  % \item Download and install Minizinc
  % \item \url{https://www.minizinc.org/}
  % \end{itemize}
% \end{frame}


\begin{frame}
\frametitle{Why use Puzzles as Examples?}
\begin{itemize}
\item Easy to understand the problem
\item Solvable by hand without specialized knowledge
\item Possible to compare automated to manual solving process
\end{itemize}

\begin{quote}
The puzzle, though inanimate, is presented as a solvable problem without lasting negative consequences, a very low-risk low-reward situation. By being a puzzle, the object is attempting to convince the user that it must be completed.
\end{quote}
{\tiny Source: \href{https://www.everydayrhetoric.com/2020/03/the-rhetoric-of-puzzles/}{Every Day Rhetoric}}
\end{frame}


\mode<all>{
\part{Basic Constraint Propagation}
}
\begin{frame}
\partpage
\end{frame}
\mode<all>{
\input{../sendmore/body}
}

\mode<all>{
\part{Global Constraints}
}
\begin{frame}
\partpage
\end{frame}
\mode<all>{
\input{../sudoku/body}
}

\clearpage
\mode<all>{
\part{Customizing Search}
}
\begin{frame}
\partpage
\end{frame}
\mode<all>{
\input{../nqueen/body}
}


%% \clearpage
%% \mode<all>{
%%   \part{Problem Decomposition}
%% }
%% \begin{frame}
%%   \partpage
%% \end{frame}
%% \mode<all>{
%%   \input{../party/body}
%%   }

%% \clearpage
%% \mode<all>{
%%   \part{More Global Constraints}
%% }
%% \begin{frame}
%%   \partpage
%% \end{frame}
%% \mode<all>{
%%   \input{../car/body}
%%   }

%% \clearpage
%% \mode<all>{
%%   \part{Methodology}
%% }
%% \begin{frame}
%%   \partpage
%% \end{frame}
%% \mode<all>{
%%   \input{../methodology/body}
%%   }

\clearpage
\mode<all>{
\part{What is missing?}
}
\begin{frame}
\partpage
\end{frame}

\begin{frame}
\frametitle{Many Specialized Topics}
\begin{itemize}
\item How to design efficient core engine
\item Hybrids with LP/MIP tools
\item Hybrids with SAT
\item Symmetry breaking
\item Use of MDD/BDD to encode sets of solutions
\item High level modelling tools
\item Debugging/visualization
\end{itemize}
\end{frame}

\begin{frame}
\frametitle{Reformulation}
\begin{itemize}
\item Just because the user has modelled it this way, it doesn't mean we have to solve it that way
\begin{itemize}
\item Replace some constraint(s) by other, equivalent constraints
\item Because we don't have that constraint in our system
\item For performance
\end{itemize}
\end{itemize}
\end{frame}

\begin{frame}
\frametitle{Learning}
\begin{itemize}
\item While solving the problem we can learn how to strengthen the model/search 
\begin{itemize}
\item Understand which constraints/method contribute to propagation and change schedule
\item Learn no-good constraints by explaining failure
\item Adapt search strategy based on search experience
\end{itemize}
\end{itemize}
\end{frame}

%% \begin{frame}
%% \frametitle{Refined Process}
%% \begin{center}
%% \begin{tikzpicture}
%% \node[shape=rectangle,fill=pantone127-4] (problem) at (3,7) {Problem};
%% \node (human) at (3,6) {Human};
%% \node[shape=rectangle,fill=pantone127-4] (model) at (3,5) {User Model};
%% \node (reformulation) at (3,4) {Reformulation};
%% \node[shape=rectangle,fill=pantone127-4] (implementation) at (3,3) {Implementation Model};
%% \node (solver) at (3,2) {Constraint Solver/Search};
%% \node[shape=rectangle,fill=pantone157-8] (solution) at (3,1) {Solution};
%% \draw[-] (problem) -- (human);
%% \draw[->] (human) -- (model);
%% \draw[-] (model) -- (reformulation);
%% \draw[->] (reformulation) -- (implementation);
%% \draw[-] (implementation) -- (solver);
%% \draw[->] (solver) -- (0,2) -- (0,3) -- (implementation);
%% \draw[->] (solver) -- (solution);
%% \end{tikzpicture}
%% \end{center}
%% \end{frame}

%\begin{frame}
%\frametitle{What is CP actually used for?}
%\url{http://hsimonis.wordpress.com}
%\includegraphics[width=10cm]{../introduction/images/blog}
%\end{frame}

%% \begin{frame}
%% \frametitle{Exercises}
%% \begin{itemize}
%% \item Install Constraint Solver of your Choice
%% \item MiniZinc (MiningZinc) will be used in later sessions
%% \item Run Sendmoremoney, Sudoku, Queens examples
%% \item Can you choose which propagator to use for alldifferent?
%% \item Can you use different search strategies?
%% \end{itemize}
%% \end{frame}

\begin{frame}
\frametitle{More Learning Resources}
\begin{itemize}
\item Survey of Methods, Resources, and Formats for Teaching Constraint Programming
\begin{itemize}
\item by Tejas Santanam, Helmut Simonis
\item \url{https://doi.org/10.48550/arXiv.2403.12717}
\item Based on survey of community for WTCP 2023
\item \url{https://hsimonis.github.io/WTCP2023/}
\end{itemize}
\end{itemize}
\end{frame}


}

\mode<all>{
\part{Global Constraints}
}
\begin{frame}
\partpage
\end{frame}
\mode<all>{
\begin{frame}
\frametitle{Objectives}
\begin{itemize}
\item Overview of Core Constraint Programming
\item Three Main Concepts
\begin{itemize}
\item Constraint Propagation
\item Global Constraints
\item Customizing Search
\end{itemize}
%\item Get Some Experience with MiniZinc
\item Topics will be treated in more detail in later parts of the school
\item Based on Examples, not Formal Description
\end{itemize}
\end{frame}

\begin{frame}
\frametitle{Outline}
\begin{itemize}
\item Why Constraint Programming?
\item Constraint Propagation
\item Global Constraints
\item Customizing Search
%\item Choices in Modelling
%\item Decomposition
%\item More Global Constraints
%\item Methodology
%\item What is missing?
\end{itemize}
\end{frame}

% \begin{frame}
  % \frametitle{Using MiniZinc IDE}
  % \begin{itemize}
  % \item Developed in the Australian NICTA project
  % \item Maintained by Monash University
  % \item Modelling tool with multiple back-end solvers
  % \item Available from \url{https://www.minizinc.org/}
  % \end{itemize}
% \end{frame}


% \begin{frame}
% \frametitle{Examples in ECLiPSe}
% \begin{itemize}
% \item Open sourced constraint programming language
% \item Development goes back to 1985
% \item ECRC, ICL, IC-Parc, PTL, Cisco
% \item \url{https://eclipseclp.org/}
% \item Specialities
% \begin{itemize}
% \item Develop new solvers for specific domains
% \item Integration with MIP
% \end{itemize}
% \item Not included in bundled MiniZinc IDE
% \item Specialized visualization tools used here
  % \begin{itemize}
  % \item CP-Viz, Simonis et al. 2010
    % \end{itemize}
% \end{itemize}
% \end{frame}

\begin{frame}
\frametitle{Tutorial Based on ECLiPSe ELearning Course}
\begin{itemize}
\item Self-study course in constraint programming
\item Supported by Cisco Systems and Silicon Valley Community Foundation
\item Multi-media format, video lectures, slides, handout etc
\item {\small \url{https://eclipseclp.org/ELearning/index.html}}
\end{itemize}
\end{frame}

\begin{frame}
\frametitle{Also Part of CRT-AI Constraint Week}
\begin{itemize}
\item Annual one week course on CP and Optimization in Ireland
\item Part of national training program for PhD students in AI
\item \url{https://www.crt-ai.ie/}
\end{itemize}
\end{frame}



\begin{frame}
\frametitle{Constraint Programming - in a nutshell}
\begin{itemize}
\item Declarative description of problems with
\begin{itemize}
\item {\em Variables} which range over (finite) sets of values
\item {\em Constraints} over subsets of variables which restrict possible value combinations
\item A {\em solution} is a value assignment which satisfies all constraints
\end{itemize}

\item Constraint propagation/reasoning
\begin{itemize}
\item Removing inconsistent values for variables
\item Detect failure if constraint can not be satisfied
\item Interaction of constraints via shared variables
\item Incomplete
\end{itemize}

\item Search
\begin{itemize}
\item User controlled assignment of values to variables
\item Each step triggers constraint propagation 
\end{itemize}
\item Different domains require/allow different methods
\end{itemize}
\end{frame}

\begin{frame}
  \frametitle{Constraint Programming is Different}
  \begin{itemize}
  \item Declarative Programming
    \begin{itemize}
    \item Concentrate on what you want
      \item Not how to get there
      \item Program != Algorithm
      \item Program = Model
    \end{itemize}
    \item Applied to Combinatorial Problems
      \begin{itemize}
        \item No complete polynomial algorithms known (exist?)
        \item CP less ad-hoc than heuristics
        \item Models can evolve
  \end{itemize}
  \end{itemize}
  \end{frame}
    
\begin{frame}
  \frametitle{A Subtractive Process}
  \begin{textblock}{4}(8,-3)
    \includegraphics[width=4cm]{../introduction/images/stages}
  \end{textblock}
  \vfill
  \begin{quote}
    ``Oh, bosh, as Mr. Ruskin says. Sculpture, per se, is the simplest thing in the world. All you have to do is to take a big chunk of marble and a hammer and chisel, make up your mind what you are about to create and chip off all the marble you don't want.''-Paris Gaulois.
  \end{quote}
  
  {\tiny Source: \url{https://quoteinvestigator.com/2014/06/22/chip-away/}}
\end{frame}

\begin{frame}
\frametitle{Basic Process}
\begin{center}
\begin{tikzpicture}
\node[shape=rectangle,fill=pantone127-4] (problem) at (2,5) {Problem};
\node (human) at (2,4) {Human};
\node[shape=rectangle,fill=pantone127-4] (model) at (2,3) {Model};
\node (solver) at (2,2) {Constraint Solver/Search};
\node[shape=rectangle,fill=pantone157-8] (solution) at (2,1) {Solution};
\draw[-] (problem) -- (human);
\draw[->] (human) -- (model);
\draw[-] (model) -- (solver);
\draw[->] (solver) -- (solution);
\end{tikzpicture}
\end{center}
\end{frame}

\begin{frame}
\frametitle{More Realistic}
\begin{center}
\begin{tikzpicture}[xscale=4.0]
\node[shape=rectangle,fill=pantone127-4] (problem) at (2,5) {Problem};
\node (human) at (2,4) {Human};
\node[shape=rectangle,fill=pantone127-4] (model) at (2,3) {Model};
\node (solver) at (2,2) {Constraint Solver/Search};
\node[shape=rectangle,fill=pantone24-8] (hangs) at (1,1) {Hangs};
\node[shape=rectangle,fill=pantone157-8] (solution) at (2,1) {Solution};
\node[shape=rectangle,fill=pantone24-8] (wrong) at (3,1) {Wrong Solution};
\draw[-] (problem) -- (human);
\draw[->] (human) -- (model);
\draw[-] (model) -- (solver);
\draw[->] (solver) -- (hangs);
\draw[->] (solver) -- (solution);
\draw[->] (solver) -- (wrong);
\draw[->] (hangs) -- (1,4) -- (human);
\draw[->] (wrong) -- (3,4) -- (human);
\end{tikzpicture}
\end{center}
\end{frame}

\begin{frame}
\frametitle{Dual Role of Model}
\begin{itemize}
\item Allows Human to Express Problem
\begin{itemize}
\item Close to Problem Domain
\item Constraints as Abstractions
\end{itemize}
\item Allows Solver to Execute
\begin{itemize}
\item Variables as Communication Mechanism
\item Constraints as Algorithms
\end{itemize}
\end{itemize}
\end{frame}

\begin{frame}
\frametitle{Modelling Frameworks}
\begin{itemize}
\item \href{https://www.minizinc.org/}{MiniZinc} (NICTA, Monash University, Australia)
\item \href{https://github.com/eomahony/Numberjack}{NumberJack} (Insight, Ireland) 
\item \href{https://www-users.york.ac.uk/peter.nightingale/savilerow//}{EssencePrime/SavilleRow} (UK) 
\item \href{https://cpmpy.readthedocs.io/en/latest/}{CPMpy} (KU Leuven)
\item Allow use of multiple back-end solvers
\item Compile model into variants for each solver
\item A priori solver independent model(CP, MIP, SAT)
\end{itemize}
\end{frame} 

\begin{frame}
\frametitle{Framework Process}
\begin{center}
\begin{tikzpicture}[xscale=2]
\node[shape=rectangle,fill=pantone127-4] (problem) at (2.5,5) {Problem};
\node (human) at (2.5,4) {Human};
\node[shape=rectangle,fill=pantone127-4] (model) at (2.5,3) {Model};
\node (compile) at (2.5,2) {Compile/Reformulate};
\node (solver1) at (1,1) {CP};
\node (solver2) at (2,1) {MIP};
\node (solver3) at (3,1) {SAT};
\node (solver4) at (4,1) {Other};
\node[shape=rectangle,fill=pantone157-8] (solution1) at (1,0) {Solution};
\node[shape=rectangle,fill=pantone157-8] (solution2) at (2,0) {Solution};
\node[shape=rectangle,fill=pantone157-8] (solution3) at (3,0) {Solution};
\node[shape=rectangle,fill=pantone157-8] (solution4) at (4,0) {Solution};
\draw[-] (problem) -- (human);
\draw[->] (human) -- (model);
\draw[-] (model) -- (compile);
\draw[-] (compile) -- (solver1);
\draw[-] (compile) -- (solver2);
\draw[-] (compile) -- (solver3);
\draw[-] (compile) -- (solver4);
\draw[->] (solver1) -- (solution1);
\draw[->] (solver2) -- (solution2);
\draw[->] (solver3) -- (solution3);
\draw[->] (solver4) -- (solution4);
\end{tikzpicture}
\end{center}
\end{frame}

% \begin{frame}
  % \frametitle{Do It Now!}
  % \begin{itemize}
  % \item Download and install Minizinc
  % \item \url{https://www.minizinc.org/}
  % \end{itemize}
% \end{frame}


\begin{frame}
\frametitle{Why use Puzzles as Examples?}
\begin{itemize}
\item Easy to understand the problem
\item Solvable by hand without specialized knowledge
\item Possible to compare automated to manual solving process
\end{itemize}

\begin{quote}
The puzzle, though inanimate, is presented as a solvable problem without lasting negative consequences, a very low-risk low-reward situation. By being a puzzle, the object is attempting to convince the user that it must be completed.
\end{quote}
{\tiny Source: \href{https://www.everydayrhetoric.com/2020/03/the-rhetoric-of-puzzles/}{Every Day Rhetoric}}
\end{frame}


\mode<all>{
\part{Basic Constraint Propagation}
}
\begin{frame}
\partpage
\end{frame}
\mode<all>{
\input{../sendmore/body}
}

\mode<all>{
\part{Global Constraints}
}
\begin{frame}
\partpage
\end{frame}
\mode<all>{
\input{../sudoku/body}
}

\clearpage
\mode<all>{
\part{Customizing Search}
}
\begin{frame}
\partpage
\end{frame}
\mode<all>{
\input{../nqueen/body}
}


%% \clearpage
%% \mode<all>{
%%   \part{Problem Decomposition}
%% }
%% \begin{frame}
%%   \partpage
%% \end{frame}
%% \mode<all>{
%%   \input{../party/body}
%%   }

%% \clearpage
%% \mode<all>{
%%   \part{More Global Constraints}
%% }
%% \begin{frame}
%%   \partpage
%% \end{frame}
%% \mode<all>{
%%   \input{../car/body}
%%   }

%% \clearpage
%% \mode<all>{
%%   \part{Methodology}
%% }
%% \begin{frame}
%%   \partpage
%% \end{frame}
%% \mode<all>{
%%   \input{../methodology/body}
%%   }

\clearpage
\mode<all>{
\part{What is missing?}
}
\begin{frame}
\partpage
\end{frame}

\begin{frame}
\frametitle{Many Specialized Topics}
\begin{itemize}
\item How to design efficient core engine
\item Hybrids with LP/MIP tools
\item Hybrids with SAT
\item Symmetry breaking
\item Use of MDD/BDD to encode sets of solutions
\item High level modelling tools
\item Debugging/visualization
\end{itemize}
\end{frame}

\begin{frame}
\frametitle{Reformulation}
\begin{itemize}
\item Just because the user has modelled it this way, it doesn't mean we have to solve it that way
\begin{itemize}
\item Replace some constraint(s) by other, equivalent constraints
\item Because we don't have that constraint in our system
\item For performance
\end{itemize}
\end{itemize}
\end{frame}

\begin{frame}
\frametitle{Learning}
\begin{itemize}
\item While solving the problem we can learn how to strengthen the model/search 
\begin{itemize}
\item Understand which constraints/method contribute to propagation and change schedule
\item Learn no-good constraints by explaining failure
\item Adapt search strategy based on search experience
\end{itemize}
\end{itemize}
\end{frame}

%% \begin{frame}
%% \frametitle{Refined Process}
%% \begin{center}
%% \begin{tikzpicture}
%% \node[shape=rectangle,fill=pantone127-4] (problem) at (3,7) {Problem};
%% \node (human) at (3,6) {Human};
%% \node[shape=rectangle,fill=pantone127-4] (model) at (3,5) {User Model};
%% \node (reformulation) at (3,4) {Reformulation};
%% \node[shape=rectangle,fill=pantone127-4] (implementation) at (3,3) {Implementation Model};
%% \node (solver) at (3,2) {Constraint Solver/Search};
%% \node[shape=rectangle,fill=pantone157-8] (solution) at (3,1) {Solution};
%% \draw[-] (problem) -- (human);
%% \draw[->] (human) -- (model);
%% \draw[-] (model) -- (reformulation);
%% \draw[->] (reformulation) -- (implementation);
%% \draw[-] (implementation) -- (solver);
%% \draw[->] (solver) -- (0,2) -- (0,3) -- (implementation);
%% \draw[->] (solver) -- (solution);
%% \end{tikzpicture}
%% \end{center}
%% \end{frame}

%\begin{frame}
%\frametitle{What is CP actually used for?}
%\url{http://hsimonis.wordpress.com}
%\includegraphics[width=10cm]{../introduction/images/blog}
%\end{frame}

%% \begin{frame}
%% \frametitle{Exercises}
%% \begin{itemize}
%% \item Install Constraint Solver of your Choice
%% \item MiniZinc (MiningZinc) will be used in later sessions
%% \item Run Sendmoremoney, Sudoku, Queens examples
%% \item Can you choose which propagator to use for alldifferent?
%% \item Can you use different search strategies?
%% \end{itemize}
%% \end{frame}

\begin{frame}
\frametitle{More Learning Resources}
\begin{itemize}
\item Survey of Methods, Resources, and Formats for Teaching Constraint Programming
\begin{itemize}
\item by Tejas Santanam, Helmut Simonis
\item \url{https://doi.org/10.48550/arXiv.2403.12717}
\item Based on survey of community for WTCP 2023
\item \url{https://hsimonis.github.io/WTCP2023/}
\end{itemize}
\end{itemize}
\end{frame}


}

\clearpage
\mode<all>{
\part{Customizing Search}
}
\begin{frame}
\partpage
\end{frame}
\mode<all>{
\begin{frame}
\frametitle{Objectives}
\begin{itemize}
\item Overview of Core Constraint Programming
\item Three Main Concepts
\begin{itemize}
\item Constraint Propagation
\item Global Constraints
\item Customizing Search
\end{itemize}
%\item Get Some Experience with MiniZinc
\item Topics will be treated in more detail in later parts of the school
\item Based on Examples, not Formal Description
\end{itemize}
\end{frame}

\begin{frame}
\frametitle{Outline}
\begin{itemize}
\item Why Constraint Programming?
\item Constraint Propagation
\item Global Constraints
\item Customizing Search
%\item Choices in Modelling
%\item Decomposition
%\item More Global Constraints
%\item Methodology
%\item What is missing?
\end{itemize}
\end{frame}

% \begin{frame}
  % \frametitle{Using MiniZinc IDE}
  % \begin{itemize}
  % \item Developed in the Australian NICTA project
  % \item Maintained by Monash University
  % \item Modelling tool with multiple back-end solvers
  % \item Available from \url{https://www.minizinc.org/}
  % \end{itemize}
% \end{frame}


% \begin{frame}
% \frametitle{Examples in ECLiPSe}
% \begin{itemize}
% \item Open sourced constraint programming language
% \item Development goes back to 1985
% \item ECRC, ICL, IC-Parc, PTL, Cisco
% \item \url{https://eclipseclp.org/}
% \item Specialities
% \begin{itemize}
% \item Develop new solvers for specific domains
% \item Integration with MIP
% \end{itemize}
% \item Not included in bundled MiniZinc IDE
% \item Specialized visualization tools used here
  % \begin{itemize}
  % \item CP-Viz, Simonis et al. 2010
    % \end{itemize}
% \end{itemize}
% \end{frame}

\begin{frame}
\frametitle{Tutorial Based on ECLiPSe ELearning Course}
\begin{itemize}
\item Self-study course in constraint programming
\item Supported by Cisco Systems and Silicon Valley Community Foundation
\item Multi-media format, video lectures, slides, handout etc
\item {\small \url{https://eclipseclp.org/ELearning/index.html}}
\end{itemize}
\end{frame}

\begin{frame}
\frametitle{Also Part of CRT-AI Constraint Week}
\begin{itemize}
\item Annual one week course on CP and Optimization in Ireland
\item Part of national training program for PhD students in AI
\item \url{https://www.crt-ai.ie/}
\end{itemize}
\end{frame}



\begin{frame}
\frametitle{Constraint Programming - in a nutshell}
\begin{itemize}
\item Declarative description of problems with
\begin{itemize}
\item {\em Variables} which range over (finite) sets of values
\item {\em Constraints} over subsets of variables which restrict possible value combinations
\item A {\em solution} is a value assignment which satisfies all constraints
\end{itemize}

\item Constraint propagation/reasoning
\begin{itemize}
\item Removing inconsistent values for variables
\item Detect failure if constraint can not be satisfied
\item Interaction of constraints via shared variables
\item Incomplete
\end{itemize}

\item Search
\begin{itemize}
\item User controlled assignment of values to variables
\item Each step triggers constraint propagation 
\end{itemize}
\item Different domains require/allow different methods
\end{itemize}
\end{frame}

\begin{frame}
  \frametitle{Constraint Programming is Different}
  \begin{itemize}
  \item Declarative Programming
    \begin{itemize}
    \item Concentrate on what you want
      \item Not how to get there
      \item Program != Algorithm
      \item Program = Model
    \end{itemize}
    \item Applied to Combinatorial Problems
      \begin{itemize}
        \item No complete polynomial algorithms known (exist?)
        \item CP less ad-hoc than heuristics
        \item Models can evolve
  \end{itemize}
  \end{itemize}
  \end{frame}
    
\begin{frame}
  \frametitle{A Subtractive Process}
  \begin{textblock}{4}(8,-3)
    \includegraphics[width=4cm]{../introduction/images/stages}
  \end{textblock}
  \vfill
  \begin{quote}
    ``Oh, bosh, as Mr. Ruskin says. Sculpture, per se, is the simplest thing in the world. All you have to do is to take a big chunk of marble and a hammer and chisel, make up your mind what you are about to create and chip off all the marble you don't want.''-Paris Gaulois.
  \end{quote}
  
  {\tiny Source: \url{https://quoteinvestigator.com/2014/06/22/chip-away/}}
\end{frame}

\begin{frame}
\frametitle{Basic Process}
\begin{center}
\begin{tikzpicture}
\node[shape=rectangle,fill=pantone127-4] (problem) at (2,5) {Problem};
\node (human) at (2,4) {Human};
\node[shape=rectangle,fill=pantone127-4] (model) at (2,3) {Model};
\node (solver) at (2,2) {Constraint Solver/Search};
\node[shape=rectangle,fill=pantone157-8] (solution) at (2,1) {Solution};
\draw[-] (problem) -- (human);
\draw[->] (human) -- (model);
\draw[-] (model) -- (solver);
\draw[->] (solver) -- (solution);
\end{tikzpicture}
\end{center}
\end{frame}

\begin{frame}
\frametitle{More Realistic}
\begin{center}
\begin{tikzpicture}[xscale=4.0]
\node[shape=rectangle,fill=pantone127-4] (problem) at (2,5) {Problem};
\node (human) at (2,4) {Human};
\node[shape=rectangle,fill=pantone127-4] (model) at (2,3) {Model};
\node (solver) at (2,2) {Constraint Solver/Search};
\node[shape=rectangle,fill=pantone24-8] (hangs) at (1,1) {Hangs};
\node[shape=rectangle,fill=pantone157-8] (solution) at (2,1) {Solution};
\node[shape=rectangle,fill=pantone24-8] (wrong) at (3,1) {Wrong Solution};
\draw[-] (problem) -- (human);
\draw[->] (human) -- (model);
\draw[-] (model) -- (solver);
\draw[->] (solver) -- (hangs);
\draw[->] (solver) -- (solution);
\draw[->] (solver) -- (wrong);
\draw[->] (hangs) -- (1,4) -- (human);
\draw[->] (wrong) -- (3,4) -- (human);
\end{tikzpicture}
\end{center}
\end{frame}

\begin{frame}
\frametitle{Dual Role of Model}
\begin{itemize}
\item Allows Human to Express Problem
\begin{itemize}
\item Close to Problem Domain
\item Constraints as Abstractions
\end{itemize}
\item Allows Solver to Execute
\begin{itemize}
\item Variables as Communication Mechanism
\item Constraints as Algorithms
\end{itemize}
\end{itemize}
\end{frame}

\begin{frame}
\frametitle{Modelling Frameworks}
\begin{itemize}
\item \href{https://www.minizinc.org/}{MiniZinc} (NICTA, Monash University, Australia)
\item \href{https://github.com/eomahony/Numberjack}{NumberJack} (Insight, Ireland) 
\item \href{https://www-users.york.ac.uk/peter.nightingale/savilerow//}{EssencePrime/SavilleRow} (UK) 
\item \href{https://cpmpy.readthedocs.io/en/latest/}{CPMpy} (KU Leuven)
\item Allow use of multiple back-end solvers
\item Compile model into variants for each solver
\item A priori solver independent model(CP, MIP, SAT)
\end{itemize}
\end{frame} 

\begin{frame}
\frametitle{Framework Process}
\begin{center}
\begin{tikzpicture}[xscale=2]
\node[shape=rectangle,fill=pantone127-4] (problem) at (2.5,5) {Problem};
\node (human) at (2.5,4) {Human};
\node[shape=rectangle,fill=pantone127-4] (model) at (2.5,3) {Model};
\node (compile) at (2.5,2) {Compile/Reformulate};
\node (solver1) at (1,1) {CP};
\node (solver2) at (2,1) {MIP};
\node (solver3) at (3,1) {SAT};
\node (solver4) at (4,1) {Other};
\node[shape=rectangle,fill=pantone157-8] (solution1) at (1,0) {Solution};
\node[shape=rectangle,fill=pantone157-8] (solution2) at (2,0) {Solution};
\node[shape=rectangle,fill=pantone157-8] (solution3) at (3,0) {Solution};
\node[shape=rectangle,fill=pantone157-8] (solution4) at (4,0) {Solution};
\draw[-] (problem) -- (human);
\draw[->] (human) -- (model);
\draw[-] (model) -- (compile);
\draw[-] (compile) -- (solver1);
\draw[-] (compile) -- (solver2);
\draw[-] (compile) -- (solver3);
\draw[-] (compile) -- (solver4);
\draw[->] (solver1) -- (solution1);
\draw[->] (solver2) -- (solution2);
\draw[->] (solver3) -- (solution3);
\draw[->] (solver4) -- (solution4);
\end{tikzpicture}
\end{center}
\end{frame}

% \begin{frame}
  % \frametitle{Do It Now!}
  % \begin{itemize}
  % \item Download and install Minizinc
  % \item \url{https://www.minizinc.org/}
  % \end{itemize}
% \end{frame}


\begin{frame}
\frametitle{Why use Puzzles as Examples?}
\begin{itemize}
\item Easy to understand the problem
\item Solvable by hand without specialized knowledge
\item Possible to compare automated to manual solving process
\end{itemize}

\begin{quote}
The puzzle, though inanimate, is presented as a solvable problem without lasting negative consequences, a very low-risk low-reward situation. By being a puzzle, the object is attempting to convince the user that it must be completed.
\end{quote}
{\tiny Source: \href{https://www.everydayrhetoric.com/2020/03/the-rhetoric-of-puzzles/}{Every Day Rhetoric}}
\end{frame}


\mode<all>{
\part{Basic Constraint Propagation}
}
\begin{frame}
\partpage
\end{frame}
\mode<all>{
\input{../sendmore/body}
}

\mode<all>{
\part{Global Constraints}
}
\begin{frame}
\partpage
\end{frame}
\mode<all>{
\input{../sudoku/body}
}

\clearpage
\mode<all>{
\part{Customizing Search}
}
\begin{frame}
\partpage
\end{frame}
\mode<all>{
\input{../nqueen/body}
}


%% \clearpage
%% \mode<all>{
%%   \part{Problem Decomposition}
%% }
%% \begin{frame}
%%   \partpage
%% \end{frame}
%% \mode<all>{
%%   \input{../party/body}
%%   }

%% \clearpage
%% \mode<all>{
%%   \part{More Global Constraints}
%% }
%% \begin{frame}
%%   \partpage
%% \end{frame}
%% \mode<all>{
%%   \input{../car/body}
%%   }

%% \clearpage
%% \mode<all>{
%%   \part{Methodology}
%% }
%% \begin{frame}
%%   \partpage
%% \end{frame}
%% \mode<all>{
%%   \input{../methodology/body}
%%   }

\clearpage
\mode<all>{
\part{What is missing?}
}
\begin{frame}
\partpage
\end{frame}

\begin{frame}
\frametitle{Many Specialized Topics}
\begin{itemize}
\item How to design efficient core engine
\item Hybrids with LP/MIP tools
\item Hybrids with SAT
\item Symmetry breaking
\item Use of MDD/BDD to encode sets of solutions
\item High level modelling tools
\item Debugging/visualization
\end{itemize}
\end{frame}

\begin{frame}
\frametitle{Reformulation}
\begin{itemize}
\item Just because the user has modelled it this way, it doesn't mean we have to solve it that way
\begin{itemize}
\item Replace some constraint(s) by other, equivalent constraints
\item Because we don't have that constraint in our system
\item For performance
\end{itemize}
\end{itemize}
\end{frame}

\begin{frame}
\frametitle{Learning}
\begin{itemize}
\item While solving the problem we can learn how to strengthen the model/search 
\begin{itemize}
\item Understand which constraints/method contribute to propagation and change schedule
\item Learn no-good constraints by explaining failure
\item Adapt search strategy based on search experience
\end{itemize}
\end{itemize}
\end{frame}

%% \begin{frame}
%% \frametitle{Refined Process}
%% \begin{center}
%% \begin{tikzpicture}
%% \node[shape=rectangle,fill=pantone127-4] (problem) at (3,7) {Problem};
%% \node (human) at (3,6) {Human};
%% \node[shape=rectangle,fill=pantone127-4] (model) at (3,5) {User Model};
%% \node (reformulation) at (3,4) {Reformulation};
%% \node[shape=rectangle,fill=pantone127-4] (implementation) at (3,3) {Implementation Model};
%% \node (solver) at (3,2) {Constraint Solver/Search};
%% \node[shape=rectangle,fill=pantone157-8] (solution) at (3,1) {Solution};
%% \draw[-] (problem) -- (human);
%% \draw[->] (human) -- (model);
%% \draw[-] (model) -- (reformulation);
%% \draw[->] (reformulation) -- (implementation);
%% \draw[-] (implementation) -- (solver);
%% \draw[->] (solver) -- (0,2) -- (0,3) -- (implementation);
%% \draw[->] (solver) -- (solution);
%% \end{tikzpicture}
%% \end{center}
%% \end{frame}

%\begin{frame}
%\frametitle{What is CP actually used for?}
%\url{http://hsimonis.wordpress.com}
%\includegraphics[width=10cm]{../introduction/images/blog}
%\end{frame}

%% \begin{frame}
%% \frametitle{Exercises}
%% \begin{itemize}
%% \item Install Constraint Solver of your Choice
%% \item MiniZinc (MiningZinc) will be used in later sessions
%% \item Run Sendmoremoney, Sudoku, Queens examples
%% \item Can you choose which propagator to use for alldifferent?
%% \item Can you use different search strategies?
%% \end{itemize}
%% \end{frame}

\begin{frame}
\frametitle{More Learning Resources}
\begin{itemize}
\item Survey of Methods, Resources, and Formats for Teaching Constraint Programming
\begin{itemize}
\item by Tejas Santanam, Helmut Simonis
\item \url{https://doi.org/10.48550/arXiv.2403.12717}
\item Based on survey of community for WTCP 2023
\item \url{https://hsimonis.github.io/WTCP2023/}
\end{itemize}
\end{itemize}
\end{frame}


}


%% \clearpage
%% \mode<all>{
%%   \part{Problem Decomposition}
%% }
%% \begin{frame}
%%   \partpage
%% \end{frame}
%% \mode<all>{
%%   \begin{frame}
\frametitle{Objectives}
\begin{itemize}
\item Overview of Core Constraint Programming
\item Three Main Concepts
\begin{itemize}
\item Constraint Propagation
\item Global Constraints
\item Customizing Search
\end{itemize}
%\item Get Some Experience with MiniZinc
\item Topics will be treated in more detail in later parts of the school
\item Based on Examples, not Formal Description
\end{itemize}
\end{frame}

\begin{frame}
\frametitle{Outline}
\begin{itemize}
\item Why Constraint Programming?
\item Constraint Propagation
\item Global Constraints
\item Customizing Search
%\item Choices in Modelling
%\item Decomposition
%\item More Global Constraints
%\item Methodology
%\item What is missing?
\end{itemize}
\end{frame}

% \begin{frame}
  % \frametitle{Using MiniZinc IDE}
  % \begin{itemize}
  % \item Developed in the Australian NICTA project
  % \item Maintained by Monash University
  % \item Modelling tool with multiple back-end solvers
  % \item Available from \url{https://www.minizinc.org/}
  % \end{itemize}
% \end{frame}


% \begin{frame}
% \frametitle{Examples in ECLiPSe}
% \begin{itemize}
% \item Open sourced constraint programming language
% \item Development goes back to 1985
% \item ECRC, ICL, IC-Parc, PTL, Cisco
% \item \url{https://eclipseclp.org/}
% \item Specialities
% \begin{itemize}
% \item Develop new solvers for specific domains
% \item Integration with MIP
% \end{itemize}
% \item Not included in bundled MiniZinc IDE
% \item Specialized visualization tools used here
  % \begin{itemize}
  % \item CP-Viz, Simonis et al. 2010
    % \end{itemize}
% \end{itemize}
% \end{frame}

\begin{frame}
\frametitle{Tutorial Based on ECLiPSe ELearning Course}
\begin{itemize}
\item Self-study course in constraint programming
\item Supported by Cisco Systems and Silicon Valley Community Foundation
\item Multi-media format, video lectures, slides, handout etc
\item {\small \url{https://eclipseclp.org/ELearning/index.html}}
\end{itemize}
\end{frame}

\begin{frame}
\frametitle{Also Part of CRT-AI Constraint Week}
\begin{itemize}
\item Annual one week course on CP and Optimization in Ireland
\item Part of national training program for PhD students in AI
\item \url{https://www.crt-ai.ie/}
\end{itemize}
\end{frame}



\begin{frame}
\frametitle{Constraint Programming - in a nutshell}
\begin{itemize}
\item Declarative description of problems with
\begin{itemize}
\item {\em Variables} which range over (finite) sets of values
\item {\em Constraints} over subsets of variables which restrict possible value combinations
\item A {\em solution} is a value assignment which satisfies all constraints
\end{itemize}

\item Constraint propagation/reasoning
\begin{itemize}
\item Removing inconsistent values for variables
\item Detect failure if constraint can not be satisfied
\item Interaction of constraints via shared variables
\item Incomplete
\end{itemize}

\item Search
\begin{itemize}
\item User controlled assignment of values to variables
\item Each step triggers constraint propagation 
\end{itemize}
\item Different domains require/allow different methods
\end{itemize}
\end{frame}

\begin{frame}
  \frametitle{Constraint Programming is Different}
  \begin{itemize}
  \item Declarative Programming
    \begin{itemize}
    \item Concentrate on what you want
      \item Not how to get there
      \item Program != Algorithm
      \item Program = Model
    \end{itemize}
    \item Applied to Combinatorial Problems
      \begin{itemize}
        \item No complete polynomial algorithms known (exist?)
        \item CP less ad-hoc than heuristics
        \item Models can evolve
  \end{itemize}
  \end{itemize}
  \end{frame}
    
\begin{frame}
  \frametitle{A Subtractive Process}
  \begin{textblock}{4}(8,-3)
    \includegraphics[width=4cm]{../introduction/images/stages}
  \end{textblock}
  \vfill
  \begin{quote}
    ``Oh, bosh, as Mr. Ruskin says. Sculpture, per se, is the simplest thing in the world. All you have to do is to take a big chunk of marble and a hammer and chisel, make up your mind what you are about to create and chip off all the marble you don't want.''-Paris Gaulois.
  \end{quote}
  
  {\tiny Source: \url{https://quoteinvestigator.com/2014/06/22/chip-away/}}
\end{frame}

\begin{frame}
\frametitle{Basic Process}
\begin{center}
\begin{tikzpicture}
\node[shape=rectangle,fill=pantone127-4] (problem) at (2,5) {Problem};
\node (human) at (2,4) {Human};
\node[shape=rectangle,fill=pantone127-4] (model) at (2,3) {Model};
\node (solver) at (2,2) {Constraint Solver/Search};
\node[shape=rectangle,fill=pantone157-8] (solution) at (2,1) {Solution};
\draw[-] (problem) -- (human);
\draw[->] (human) -- (model);
\draw[-] (model) -- (solver);
\draw[->] (solver) -- (solution);
\end{tikzpicture}
\end{center}
\end{frame}

\begin{frame}
\frametitle{More Realistic}
\begin{center}
\begin{tikzpicture}[xscale=4.0]
\node[shape=rectangle,fill=pantone127-4] (problem) at (2,5) {Problem};
\node (human) at (2,4) {Human};
\node[shape=rectangle,fill=pantone127-4] (model) at (2,3) {Model};
\node (solver) at (2,2) {Constraint Solver/Search};
\node[shape=rectangle,fill=pantone24-8] (hangs) at (1,1) {Hangs};
\node[shape=rectangle,fill=pantone157-8] (solution) at (2,1) {Solution};
\node[shape=rectangle,fill=pantone24-8] (wrong) at (3,1) {Wrong Solution};
\draw[-] (problem) -- (human);
\draw[->] (human) -- (model);
\draw[-] (model) -- (solver);
\draw[->] (solver) -- (hangs);
\draw[->] (solver) -- (solution);
\draw[->] (solver) -- (wrong);
\draw[->] (hangs) -- (1,4) -- (human);
\draw[->] (wrong) -- (3,4) -- (human);
\end{tikzpicture}
\end{center}
\end{frame}

\begin{frame}
\frametitle{Dual Role of Model}
\begin{itemize}
\item Allows Human to Express Problem
\begin{itemize}
\item Close to Problem Domain
\item Constraints as Abstractions
\end{itemize}
\item Allows Solver to Execute
\begin{itemize}
\item Variables as Communication Mechanism
\item Constraints as Algorithms
\end{itemize}
\end{itemize}
\end{frame}

\begin{frame}
\frametitle{Modelling Frameworks}
\begin{itemize}
\item \href{https://www.minizinc.org/}{MiniZinc} (NICTA, Monash University, Australia)
\item \href{https://github.com/eomahony/Numberjack}{NumberJack} (Insight, Ireland) 
\item \href{https://www-users.york.ac.uk/peter.nightingale/savilerow//}{EssencePrime/SavilleRow} (UK) 
\item \href{https://cpmpy.readthedocs.io/en/latest/}{CPMpy} (KU Leuven)
\item Allow use of multiple back-end solvers
\item Compile model into variants for each solver
\item A priori solver independent model(CP, MIP, SAT)
\end{itemize}
\end{frame} 

\begin{frame}
\frametitle{Framework Process}
\begin{center}
\begin{tikzpicture}[xscale=2]
\node[shape=rectangle,fill=pantone127-4] (problem) at (2.5,5) {Problem};
\node (human) at (2.5,4) {Human};
\node[shape=rectangle,fill=pantone127-4] (model) at (2.5,3) {Model};
\node (compile) at (2.5,2) {Compile/Reformulate};
\node (solver1) at (1,1) {CP};
\node (solver2) at (2,1) {MIP};
\node (solver3) at (3,1) {SAT};
\node (solver4) at (4,1) {Other};
\node[shape=rectangle,fill=pantone157-8] (solution1) at (1,0) {Solution};
\node[shape=rectangle,fill=pantone157-8] (solution2) at (2,0) {Solution};
\node[shape=rectangle,fill=pantone157-8] (solution3) at (3,0) {Solution};
\node[shape=rectangle,fill=pantone157-8] (solution4) at (4,0) {Solution};
\draw[-] (problem) -- (human);
\draw[->] (human) -- (model);
\draw[-] (model) -- (compile);
\draw[-] (compile) -- (solver1);
\draw[-] (compile) -- (solver2);
\draw[-] (compile) -- (solver3);
\draw[-] (compile) -- (solver4);
\draw[->] (solver1) -- (solution1);
\draw[->] (solver2) -- (solution2);
\draw[->] (solver3) -- (solution3);
\draw[->] (solver4) -- (solution4);
\end{tikzpicture}
\end{center}
\end{frame}

% \begin{frame}
  % \frametitle{Do It Now!}
  % \begin{itemize}
  % \item Download and install Minizinc
  % \item \url{https://www.minizinc.org/}
  % \end{itemize}
% \end{frame}


\begin{frame}
\frametitle{Why use Puzzles as Examples?}
\begin{itemize}
\item Easy to understand the problem
\item Solvable by hand without specialized knowledge
\item Possible to compare automated to manual solving process
\end{itemize}

\begin{quote}
The puzzle, though inanimate, is presented as a solvable problem without lasting negative consequences, a very low-risk low-reward situation. By being a puzzle, the object is attempting to convince the user that it must be completed.
\end{quote}
{\tiny Source: \href{https://www.everydayrhetoric.com/2020/03/the-rhetoric-of-puzzles/}{Every Day Rhetoric}}
\end{frame}


\mode<all>{
\part{Basic Constraint Propagation}
}
\begin{frame}
\partpage
\end{frame}
\mode<all>{
\input{../sendmore/body}
}

\mode<all>{
\part{Global Constraints}
}
\begin{frame}
\partpage
\end{frame}
\mode<all>{
\input{../sudoku/body}
}

\clearpage
\mode<all>{
\part{Customizing Search}
}
\begin{frame}
\partpage
\end{frame}
\mode<all>{
\input{../nqueen/body}
}


%% \clearpage
%% \mode<all>{
%%   \part{Problem Decomposition}
%% }
%% \begin{frame}
%%   \partpage
%% \end{frame}
%% \mode<all>{
%%   \input{../party/body}
%%   }

%% \clearpage
%% \mode<all>{
%%   \part{More Global Constraints}
%% }
%% \begin{frame}
%%   \partpage
%% \end{frame}
%% \mode<all>{
%%   \input{../car/body}
%%   }

%% \clearpage
%% \mode<all>{
%%   \part{Methodology}
%% }
%% \begin{frame}
%%   \partpage
%% \end{frame}
%% \mode<all>{
%%   \input{../methodology/body}
%%   }

\clearpage
\mode<all>{
\part{What is missing?}
}
\begin{frame}
\partpage
\end{frame}

\begin{frame}
\frametitle{Many Specialized Topics}
\begin{itemize}
\item How to design efficient core engine
\item Hybrids with LP/MIP tools
\item Hybrids with SAT
\item Symmetry breaking
\item Use of MDD/BDD to encode sets of solutions
\item High level modelling tools
\item Debugging/visualization
\end{itemize}
\end{frame}

\begin{frame}
\frametitle{Reformulation}
\begin{itemize}
\item Just because the user has modelled it this way, it doesn't mean we have to solve it that way
\begin{itemize}
\item Replace some constraint(s) by other, equivalent constraints
\item Because we don't have that constraint in our system
\item For performance
\end{itemize}
\end{itemize}
\end{frame}

\begin{frame}
\frametitle{Learning}
\begin{itemize}
\item While solving the problem we can learn how to strengthen the model/search 
\begin{itemize}
\item Understand which constraints/method contribute to propagation and change schedule
\item Learn no-good constraints by explaining failure
\item Adapt search strategy based on search experience
\end{itemize}
\end{itemize}
\end{frame}

%% \begin{frame}
%% \frametitle{Refined Process}
%% \begin{center}
%% \begin{tikzpicture}
%% \node[shape=rectangle,fill=pantone127-4] (problem) at (3,7) {Problem};
%% \node (human) at (3,6) {Human};
%% \node[shape=rectangle,fill=pantone127-4] (model) at (3,5) {User Model};
%% \node (reformulation) at (3,4) {Reformulation};
%% \node[shape=rectangle,fill=pantone127-4] (implementation) at (3,3) {Implementation Model};
%% \node (solver) at (3,2) {Constraint Solver/Search};
%% \node[shape=rectangle,fill=pantone157-8] (solution) at (3,1) {Solution};
%% \draw[-] (problem) -- (human);
%% \draw[->] (human) -- (model);
%% \draw[-] (model) -- (reformulation);
%% \draw[->] (reformulation) -- (implementation);
%% \draw[-] (implementation) -- (solver);
%% \draw[->] (solver) -- (0,2) -- (0,3) -- (implementation);
%% \draw[->] (solver) -- (solution);
%% \end{tikzpicture}
%% \end{center}
%% \end{frame}

%\begin{frame}
%\frametitle{What is CP actually used for?}
%\url{http://hsimonis.wordpress.com}
%\includegraphics[width=10cm]{../introduction/images/blog}
%\end{frame}

%% \begin{frame}
%% \frametitle{Exercises}
%% \begin{itemize}
%% \item Install Constraint Solver of your Choice
%% \item MiniZinc (MiningZinc) will be used in later sessions
%% \item Run Sendmoremoney, Sudoku, Queens examples
%% \item Can you choose which propagator to use for alldifferent?
%% \item Can you use different search strategies?
%% \end{itemize}
%% \end{frame}

\begin{frame}
\frametitle{More Learning Resources}
\begin{itemize}
\item Survey of Methods, Resources, and Formats for Teaching Constraint Programming
\begin{itemize}
\item by Tejas Santanam, Helmut Simonis
\item \url{https://doi.org/10.48550/arXiv.2403.12717}
\item Based on survey of community for WTCP 2023
\item \url{https://hsimonis.github.io/WTCP2023/}
\end{itemize}
\end{itemize}
\end{frame}


%%   }

%% \clearpage
%% \mode<all>{
%%   \part{More Global Constraints}
%% }
%% \begin{frame}
%%   \partpage
%% \end{frame}
%% \mode<all>{
%%   \begin{frame}
\frametitle{Objectives}
\begin{itemize}
\item Overview of Core Constraint Programming
\item Three Main Concepts
\begin{itemize}
\item Constraint Propagation
\item Global Constraints
\item Customizing Search
\end{itemize}
%\item Get Some Experience with MiniZinc
\item Topics will be treated in more detail in later parts of the school
\item Based on Examples, not Formal Description
\end{itemize}
\end{frame}

\begin{frame}
\frametitle{Outline}
\begin{itemize}
\item Why Constraint Programming?
\item Constraint Propagation
\item Global Constraints
\item Customizing Search
%\item Choices in Modelling
%\item Decomposition
%\item More Global Constraints
%\item Methodology
%\item What is missing?
\end{itemize}
\end{frame}

% \begin{frame}
  % \frametitle{Using MiniZinc IDE}
  % \begin{itemize}
  % \item Developed in the Australian NICTA project
  % \item Maintained by Monash University
  % \item Modelling tool with multiple back-end solvers
  % \item Available from \url{https://www.minizinc.org/}
  % \end{itemize}
% \end{frame}


% \begin{frame}
% \frametitle{Examples in ECLiPSe}
% \begin{itemize}
% \item Open sourced constraint programming language
% \item Development goes back to 1985
% \item ECRC, ICL, IC-Parc, PTL, Cisco
% \item \url{https://eclipseclp.org/}
% \item Specialities
% \begin{itemize}
% \item Develop new solvers for specific domains
% \item Integration with MIP
% \end{itemize}
% \item Not included in bundled MiniZinc IDE
% \item Specialized visualization tools used here
  % \begin{itemize}
  % \item CP-Viz, Simonis et al. 2010
    % \end{itemize}
% \end{itemize}
% \end{frame}

\begin{frame}
\frametitle{Tutorial Based on ECLiPSe ELearning Course}
\begin{itemize}
\item Self-study course in constraint programming
\item Supported by Cisco Systems and Silicon Valley Community Foundation
\item Multi-media format, video lectures, slides, handout etc
\item {\small \url{https://eclipseclp.org/ELearning/index.html}}
\end{itemize}
\end{frame}

\begin{frame}
\frametitle{Also Part of CRT-AI Constraint Week}
\begin{itemize}
\item Annual one week course on CP and Optimization in Ireland
\item Part of national training program for PhD students in AI
\item \url{https://www.crt-ai.ie/}
\end{itemize}
\end{frame}



\begin{frame}
\frametitle{Constraint Programming - in a nutshell}
\begin{itemize}
\item Declarative description of problems with
\begin{itemize}
\item {\em Variables} which range over (finite) sets of values
\item {\em Constraints} over subsets of variables which restrict possible value combinations
\item A {\em solution} is a value assignment which satisfies all constraints
\end{itemize}

\item Constraint propagation/reasoning
\begin{itemize}
\item Removing inconsistent values for variables
\item Detect failure if constraint can not be satisfied
\item Interaction of constraints via shared variables
\item Incomplete
\end{itemize}

\item Search
\begin{itemize}
\item User controlled assignment of values to variables
\item Each step triggers constraint propagation 
\end{itemize}
\item Different domains require/allow different methods
\end{itemize}
\end{frame}

\begin{frame}
  \frametitle{Constraint Programming is Different}
  \begin{itemize}
  \item Declarative Programming
    \begin{itemize}
    \item Concentrate on what you want
      \item Not how to get there
      \item Program != Algorithm
      \item Program = Model
    \end{itemize}
    \item Applied to Combinatorial Problems
      \begin{itemize}
        \item No complete polynomial algorithms known (exist?)
        \item CP less ad-hoc than heuristics
        \item Models can evolve
  \end{itemize}
  \end{itemize}
  \end{frame}
    
\begin{frame}
  \frametitle{A Subtractive Process}
  \begin{textblock}{4}(8,-3)
    \includegraphics[width=4cm]{../introduction/images/stages}
  \end{textblock}
  \vfill
  \begin{quote}
    ``Oh, bosh, as Mr. Ruskin says. Sculpture, per se, is the simplest thing in the world. All you have to do is to take a big chunk of marble and a hammer and chisel, make up your mind what you are about to create and chip off all the marble you don't want.''-Paris Gaulois.
  \end{quote}
  
  {\tiny Source: \url{https://quoteinvestigator.com/2014/06/22/chip-away/}}
\end{frame}

\begin{frame}
\frametitle{Basic Process}
\begin{center}
\begin{tikzpicture}
\node[shape=rectangle,fill=pantone127-4] (problem) at (2,5) {Problem};
\node (human) at (2,4) {Human};
\node[shape=rectangle,fill=pantone127-4] (model) at (2,3) {Model};
\node (solver) at (2,2) {Constraint Solver/Search};
\node[shape=rectangle,fill=pantone157-8] (solution) at (2,1) {Solution};
\draw[-] (problem) -- (human);
\draw[->] (human) -- (model);
\draw[-] (model) -- (solver);
\draw[->] (solver) -- (solution);
\end{tikzpicture}
\end{center}
\end{frame}

\begin{frame}
\frametitle{More Realistic}
\begin{center}
\begin{tikzpicture}[xscale=4.0]
\node[shape=rectangle,fill=pantone127-4] (problem) at (2,5) {Problem};
\node (human) at (2,4) {Human};
\node[shape=rectangle,fill=pantone127-4] (model) at (2,3) {Model};
\node (solver) at (2,2) {Constraint Solver/Search};
\node[shape=rectangle,fill=pantone24-8] (hangs) at (1,1) {Hangs};
\node[shape=rectangle,fill=pantone157-8] (solution) at (2,1) {Solution};
\node[shape=rectangle,fill=pantone24-8] (wrong) at (3,1) {Wrong Solution};
\draw[-] (problem) -- (human);
\draw[->] (human) -- (model);
\draw[-] (model) -- (solver);
\draw[->] (solver) -- (hangs);
\draw[->] (solver) -- (solution);
\draw[->] (solver) -- (wrong);
\draw[->] (hangs) -- (1,4) -- (human);
\draw[->] (wrong) -- (3,4) -- (human);
\end{tikzpicture}
\end{center}
\end{frame}

\begin{frame}
\frametitle{Dual Role of Model}
\begin{itemize}
\item Allows Human to Express Problem
\begin{itemize}
\item Close to Problem Domain
\item Constraints as Abstractions
\end{itemize}
\item Allows Solver to Execute
\begin{itemize}
\item Variables as Communication Mechanism
\item Constraints as Algorithms
\end{itemize}
\end{itemize}
\end{frame}

\begin{frame}
\frametitle{Modelling Frameworks}
\begin{itemize}
\item \href{https://www.minizinc.org/}{MiniZinc} (NICTA, Monash University, Australia)
\item \href{https://github.com/eomahony/Numberjack}{NumberJack} (Insight, Ireland) 
\item \href{https://www-users.york.ac.uk/peter.nightingale/savilerow//}{EssencePrime/SavilleRow} (UK) 
\item \href{https://cpmpy.readthedocs.io/en/latest/}{CPMpy} (KU Leuven)
\item Allow use of multiple back-end solvers
\item Compile model into variants for each solver
\item A priori solver independent model(CP, MIP, SAT)
\end{itemize}
\end{frame} 

\begin{frame}
\frametitle{Framework Process}
\begin{center}
\begin{tikzpicture}[xscale=2]
\node[shape=rectangle,fill=pantone127-4] (problem) at (2.5,5) {Problem};
\node (human) at (2.5,4) {Human};
\node[shape=rectangle,fill=pantone127-4] (model) at (2.5,3) {Model};
\node (compile) at (2.5,2) {Compile/Reformulate};
\node (solver1) at (1,1) {CP};
\node (solver2) at (2,1) {MIP};
\node (solver3) at (3,1) {SAT};
\node (solver4) at (4,1) {Other};
\node[shape=rectangle,fill=pantone157-8] (solution1) at (1,0) {Solution};
\node[shape=rectangle,fill=pantone157-8] (solution2) at (2,0) {Solution};
\node[shape=rectangle,fill=pantone157-8] (solution3) at (3,0) {Solution};
\node[shape=rectangle,fill=pantone157-8] (solution4) at (4,0) {Solution};
\draw[-] (problem) -- (human);
\draw[->] (human) -- (model);
\draw[-] (model) -- (compile);
\draw[-] (compile) -- (solver1);
\draw[-] (compile) -- (solver2);
\draw[-] (compile) -- (solver3);
\draw[-] (compile) -- (solver4);
\draw[->] (solver1) -- (solution1);
\draw[->] (solver2) -- (solution2);
\draw[->] (solver3) -- (solution3);
\draw[->] (solver4) -- (solution4);
\end{tikzpicture}
\end{center}
\end{frame}

% \begin{frame}
  % \frametitle{Do It Now!}
  % \begin{itemize}
  % \item Download and install Minizinc
  % \item \url{https://www.minizinc.org/}
  % \end{itemize}
% \end{frame}


\begin{frame}
\frametitle{Why use Puzzles as Examples?}
\begin{itemize}
\item Easy to understand the problem
\item Solvable by hand without specialized knowledge
\item Possible to compare automated to manual solving process
\end{itemize}

\begin{quote}
The puzzle, though inanimate, is presented as a solvable problem without lasting negative consequences, a very low-risk low-reward situation. By being a puzzle, the object is attempting to convince the user that it must be completed.
\end{quote}
{\tiny Source: \href{https://www.everydayrhetoric.com/2020/03/the-rhetoric-of-puzzles/}{Every Day Rhetoric}}
\end{frame}


\mode<all>{
\part{Basic Constraint Propagation}
}
\begin{frame}
\partpage
\end{frame}
\mode<all>{
\input{../sendmore/body}
}

\mode<all>{
\part{Global Constraints}
}
\begin{frame}
\partpage
\end{frame}
\mode<all>{
\input{../sudoku/body}
}

\clearpage
\mode<all>{
\part{Customizing Search}
}
\begin{frame}
\partpage
\end{frame}
\mode<all>{
\input{../nqueen/body}
}


%% \clearpage
%% \mode<all>{
%%   \part{Problem Decomposition}
%% }
%% \begin{frame}
%%   \partpage
%% \end{frame}
%% \mode<all>{
%%   \input{../party/body}
%%   }

%% \clearpage
%% \mode<all>{
%%   \part{More Global Constraints}
%% }
%% \begin{frame}
%%   \partpage
%% \end{frame}
%% \mode<all>{
%%   \input{../car/body}
%%   }

%% \clearpage
%% \mode<all>{
%%   \part{Methodology}
%% }
%% \begin{frame}
%%   \partpage
%% \end{frame}
%% \mode<all>{
%%   \input{../methodology/body}
%%   }

\clearpage
\mode<all>{
\part{What is missing?}
}
\begin{frame}
\partpage
\end{frame}

\begin{frame}
\frametitle{Many Specialized Topics}
\begin{itemize}
\item How to design efficient core engine
\item Hybrids with LP/MIP tools
\item Hybrids with SAT
\item Symmetry breaking
\item Use of MDD/BDD to encode sets of solutions
\item High level modelling tools
\item Debugging/visualization
\end{itemize}
\end{frame}

\begin{frame}
\frametitle{Reformulation}
\begin{itemize}
\item Just because the user has modelled it this way, it doesn't mean we have to solve it that way
\begin{itemize}
\item Replace some constraint(s) by other, equivalent constraints
\item Because we don't have that constraint in our system
\item For performance
\end{itemize}
\end{itemize}
\end{frame}

\begin{frame}
\frametitle{Learning}
\begin{itemize}
\item While solving the problem we can learn how to strengthen the model/search 
\begin{itemize}
\item Understand which constraints/method contribute to propagation and change schedule
\item Learn no-good constraints by explaining failure
\item Adapt search strategy based on search experience
\end{itemize}
\end{itemize}
\end{frame}

%% \begin{frame}
%% \frametitle{Refined Process}
%% \begin{center}
%% \begin{tikzpicture}
%% \node[shape=rectangle,fill=pantone127-4] (problem) at (3,7) {Problem};
%% \node (human) at (3,6) {Human};
%% \node[shape=rectangle,fill=pantone127-4] (model) at (3,5) {User Model};
%% \node (reformulation) at (3,4) {Reformulation};
%% \node[shape=rectangle,fill=pantone127-4] (implementation) at (3,3) {Implementation Model};
%% \node (solver) at (3,2) {Constraint Solver/Search};
%% \node[shape=rectangle,fill=pantone157-8] (solution) at (3,1) {Solution};
%% \draw[-] (problem) -- (human);
%% \draw[->] (human) -- (model);
%% \draw[-] (model) -- (reformulation);
%% \draw[->] (reformulation) -- (implementation);
%% \draw[-] (implementation) -- (solver);
%% \draw[->] (solver) -- (0,2) -- (0,3) -- (implementation);
%% \draw[->] (solver) -- (solution);
%% \end{tikzpicture}
%% \end{center}
%% \end{frame}

%\begin{frame}
%\frametitle{What is CP actually used for?}
%\url{http://hsimonis.wordpress.com}
%\includegraphics[width=10cm]{../introduction/images/blog}
%\end{frame}

%% \begin{frame}
%% \frametitle{Exercises}
%% \begin{itemize}
%% \item Install Constraint Solver of your Choice
%% \item MiniZinc (MiningZinc) will be used in later sessions
%% \item Run Sendmoremoney, Sudoku, Queens examples
%% \item Can you choose which propagator to use for alldifferent?
%% \item Can you use different search strategies?
%% \end{itemize}
%% \end{frame}

\begin{frame}
\frametitle{More Learning Resources}
\begin{itemize}
\item Survey of Methods, Resources, and Formats for Teaching Constraint Programming
\begin{itemize}
\item by Tejas Santanam, Helmut Simonis
\item \url{https://doi.org/10.48550/arXiv.2403.12717}
\item Based on survey of community for WTCP 2023
\item \url{https://hsimonis.github.io/WTCP2023/}
\end{itemize}
\end{itemize}
\end{frame}


%%   }

%% \clearpage
%% \mode<all>{
%%   \part{Methodology}
%% }
%% \begin{frame}
%%   \partpage
%% \end{frame}
%% \mode<all>{
%%   \begin{frame}
\frametitle{Objectives}
\begin{itemize}
\item Overview of Core Constraint Programming
\item Three Main Concepts
\begin{itemize}
\item Constraint Propagation
\item Global Constraints
\item Customizing Search
\end{itemize}
%\item Get Some Experience with MiniZinc
\item Topics will be treated in more detail in later parts of the school
\item Based on Examples, not Formal Description
\end{itemize}
\end{frame}

\begin{frame}
\frametitle{Outline}
\begin{itemize}
\item Why Constraint Programming?
\item Constraint Propagation
\item Global Constraints
\item Customizing Search
%\item Choices in Modelling
%\item Decomposition
%\item More Global Constraints
%\item Methodology
%\item What is missing?
\end{itemize}
\end{frame}

% \begin{frame}
  % \frametitle{Using MiniZinc IDE}
  % \begin{itemize}
  % \item Developed in the Australian NICTA project
  % \item Maintained by Monash University
  % \item Modelling tool with multiple back-end solvers
  % \item Available from \url{https://www.minizinc.org/}
  % \end{itemize}
% \end{frame}


% \begin{frame}
% \frametitle{Examples in ECLiPSe}
% \begin{itemize}
% \item Open sourced constraint programming language
% \item Development goes back to 1985
% \item ECRC, ICL, IC-Parc, PTL, Cisco
% \item \url{https://eclipseclp.org/}
% \item Specialities
% \begin{itemize}
% \item Develop new solvers for specific domains
% \item Integration with MIP
% \end{itemize}
% \item Not included in bundled MiniZinc IDE
% \item Specialized visualization tools used here
  % \begin{itemize}
  % \item CP-Viz, Simonis et al. 2010
    % \end{itemize}
% \end{itemize}
% \end{frame}

\begin{frame}
\frametitle{Tutorial Based on ECLiPSe ELearning Course}
\begin{itemize}
\item Self-study course in constraint programming
\item Supported by Cisco Systems and Silicon Valley Community Foundation
\item Multi-media format, video lectures, slides, handout etc
\item {\small \url{https://eclipseclp.org/ELearning/index.html}}
\end{itemize}
\end{frame}

\begin{frame}
\frametitle{Also Part of CRT-AI Constraint Week}
\begin{itemize}
\item Annual one week course on CP and Optimization in Ireland
\item Part of national training program for PhD students in AI
\item \url{https://www.crt-ai.ie/}
\end{itemize}
\end{frame}



\begin{frame}
\frametitle{Constraint Programming - in a nutshell}
\begin{itemize}
\item Declarative description of problems with
\begin{itemize}
\item {\em Variables} which range over (finite) sets of values
\item {\em Constraints} over subsets of variables which restrict possible value combinations
\item A {\em solution} is a value assignment which satisfies all constraints
\end{itemize}

\item Constraint propagation/reasoning
\begin{itemize}
\item Removing inconsistent values for variables
\item Detect failure if constraint can not be satisfied
\item Interaction of constraints via shared variables
\item Incomplete
\end{itemize}

\item Search
\begin{itemize}
\item User controlled assignment of values to variables
\item Each step triggers constraint propagation 
\end{itemize}
\item Different domains require/allow different methods
\end{itemize}
\end{frame}

\begin{frame}
  \frametitle{Constraint Programming is Different}
  \begin{itemize}
  \item Declarative Programming
    \begin{itemize}
    \item Concentrate on what you want
      \item Not how to get there
      \item Program != Algorithm
      \item Program = Model
    \end{itemize}
    \item Applied to Combinatorial Problems
      \begin{itemize}
        \item No complete polynomial algorithms known (exist?)
        \item CP less ad-hoc than heuristics
        \item Models can evolve
  \end{itemize}
  \end{itemize}
  \end{frame}
    
\begin{frame}
  \frametitle{A Subtractive Process}
  \begin{textblock}{4}(8,-3)
    \includegraphics[width=4cm]{../introduction/images/stages}
  \end{textblock}
  \vfill
  \begin{quote}
    ``Oh, bosh, as Mr. Ruskin says. Sculpture, per se, is the simplest thing in the world. All you have to do is to take a big chunk of marble and a hammer and chisel, make up your mind what you are about to create and chip off all the marble you don't want.''-Paris Gaulois.
  \end{quote}
  
  {\tiny Source: \url{https://quoteinvestigator.com/2014/06/22/chip-away/}}
\end{frame}

\begin{frame}
\frametitle{Basic Process}
\begin{center}
\begin{tikzpicture}
\node[shape=rectangle,fill=pantone127-4] (problem) at (2,5) {Problem};
\node (human) at (2,4) {Human};
\node[shape=rectangle,fill=pantone127-4] (model) at (2,3) {Model};
\node (solver) at (2,2) {Constraint Solver/Search};
\node[shape=rectangle,fill=pantone157-8] (solution) at (2,1) {Solution};
\draw[-] (problem) -- (human);
\draw[->] (human) -- (model);
\draw[-] (model) -- (solver);
\draw[->] (solver) -- (solution);
\end{tikzpicture}
\end{center}
\end{frame}

\begin{frame}
\frametitle{More Realistic}
\begin{center}
\begin{tikzpicture}[xscale=4.0]
\node[shape=rectangle,fill=pantone127-4] (problem) at (2,5) {Problem};
\node (human) at (2,4) {Human};
\node[shape=rectangle,fill=pantone127-4] (model) at (2,3) {Model};
\node (solver) at (2,2) {Constraint Solver/Search};
\node[shape=rectangle,fill=pantone24-8] (hangs) at (1,1) {Hangs};
\node[shape=rectangle,fill=pantone157-8] (solution) at (2,1) {Solution};
\node[shape=rectangle,fill=pantone24-8] (wrong) at (3,1) {Wrong Solution};
\draw[-] (problem) -- (human);
\draw[->] (human) -- (model);
\draw[-] (model) -- (solver);
\draw[->] (solver) -- (hangs);
\draw[->] (solver) -- (solution);
\draw[->] (solver) -- (wrong);
\draw[->] (hangs) -- (1,4) -- (human);
\draw[->] (wrong) -- (3,4) -- (human);
\end{tikzpicture}
\end{center}
\end{frame}

\begin{frame}
\frametitle{Dual Role of Model}
\begin{itemize}
\item Allows Human to Express Problem
\begin{itemize}
\item Close to Problem Domain
\item Constraints as Abstractions
\end{itemize}
\item Allows Solver to Execute
\begin{itemize}
\item Variables as Communication Mechanism
\item Constraints as Algorithms
\end{itemize}
\end{itemize}
\end{frame}

\begin{frame}
\frametitle{Modelling Frameworks}
\begin{itemize}
\item \href{https://www.minizinc.org/}{MiniZinc} (NICTA, Monash University, Australia)
\item \href{https://github.com/eomahony/Numberjack}{NumberJack} (Insight, Ireland) 
\item \href{https://www-users.york.ac.uk/peter.nightingale/savilerow//}{EssencePrime/SavilleRow} (UK) 
\item \href{https://cpmpy.readthedocs.io/en/latest/}{CPMpy} (KU Leuven)
\item Allow use of multiple back-end solvers
\item Compile model into variants for each solver
\item A priori solver independent model(CP, MIP, SAT)
\end{itemize}
\end{frame} 

\begin{frame}
\frametitle{Framework Process}
\begin{center}
\begin{tikzpicture}[xscale=2]
\node[shape=rectangle,fill=pantone127-4] (problem) at (2.5,5) {Problem};
\node (human) at (2.5,4) {Human};
\node[shape=rectangle,fill=pantone127-4] (model) at (2.5,3) {Model};
\node (compile) at (2.5,2) {Compile/Reformulate};
\node (solver1) at (1,1) {CP};
\node (solver2) at (2,1) {MIP};
\node (solver3) at (3,1) {SAT};
\node (solver4) at (4,1) {Other};
\node[shape=rectangle,fill=pantone157-8] (solution1) at (1,0) {Solution};
\node[shape=rectangle,fill=pantone157-8] (solution2) at (2,0) {Solution};
\node[shape=rectangle,fill=pantone157-8] (solution3) at (3,0) {Solution};
\node[shape=rectangle,fill=pantone157-8] (solution4) at (4,0) {Solution};
\draw[-] (problem) -- (human);
\draw[->] (human) -- (model);
\draw[-] (model) -- (compile);
\draw[-] (compile) -- (solver1);
\draw[-] (compile) -- (solver2);
\draw[-] (compile) -- (solver3);
\draw[-] (compile) -- (solver4);
\draw[->] (solver1) -- (solution1);
\draw[->] (solver2) -- (solution2);
\draw[->] (solver3) -- (solution3);
\draw[->] (solver4) -- (solution4);
\end{tikzpicture}
\end{center}
\end{frame}

% \begin{frame}
  % \frametitle{Do It Now!}
  % \begin{itemize}
  % \item Download and install Minizinc
  % \item \url{https://www.minizinc.org/}
  % \end{itemize}
% \end{frame}


\begin{frame}
\frametitle{Why use Puzzles as Examples?}
\begin{itemize}
\item Easy to understand the problem
\item Solvable by hand without specialized knowledge
\item Possible to compare automated to manual solving process
\end{itemize}

\begin{quote}
The puzzle, though inanimate, is presented as a solvable problem without lasting negative consequences, a very low-risk low-reward situation. By being a puzzle, the object is attempting to convince the user that it must be completed.
\end{quote}
{\tiny Source: \href{https://www.everydayrhetoric.com/2020/03/the-rhetoric-of-puzzles/}{Every Day Rhetoric}}
\end{frame}


\mode<all>{
\part{Basic Constraint Propagation}
}
\begin{frame}
\partpage
\end{frame}
\mode<all>{
\input{../sendmore/body}
}

\mode<all>{
\part{Global Constraints}
}
\begin{frame}
\partpage
\end{frame}
\mode<all>{
\input{../sudoku/body}
}

\clearpage
\mode<all>{
\part{Customizing Search}
}
\begin{frame}
\partpage
\end{frame}
\mode<all>{
\input{../nqueen/body}
}


%% \clearpage
%% \mode<all>{
%%   \part{Problem Decomposition}
%% }
%% \begin{frame}
%%   \partpage
%% \end{frame}
%% \mode<all>{
%%   \input{../party/body}
%%   }

%% \clearpage
%% \mode<all>{
%%   \part{More Global Constraints}
%% }
%% \begin{frame}
%%   \partpage
%% \end{frame}
%% \mode<all>{
%%   \input{../car/body}
%%   }

%% \clearpage
%% \mode<all>{
%%   \part{Methodology}
%% }
%% \begin{frame}
%%   \partpage
%% \end{frame}
%% \mode<all>{
%%   \input{../methodology/body}
%%   }

\clearpage
\mode<all>{
\part{What is missing?}
}
\begin{frame}
\partpage
\end{frame}

\begin{frame}
\frametitle{Many Specialized Topics}
\begin{itemize}
\item How to design efficient core engine
\item Hybrids with LP/MIP tools
\item Hybrids with SAT
\item Symmetry breaking
\item Use of MDD/BDD to encode sets of solutions
\item High level modelling tools
\item Debugging/visualization
\end{itemize}
\end{frame}

\begin{frame}
\frametitle{Reformulation}
\begin{itemize}
\item Just because the user has modelled it this way, it doesn't mean we have to solve it that way
\begin{itemize}
\item Replace some constraint(s) by other, equivalent constraints
\item Because we don't have that constraint in our system
\item For performance
\end{itemize}
\end{itemize}
\end{frame}

\begin{frame}
\frametitle{Learning}
\begin{itemize}
\item While solving the problem we can learn how to strengthen the model/search 
\begin{itemize}
\item Understand which constraints/method contribute to propagation and change schedule
\item Learn no-good constraints by explaining failure
\item Adapt search strategy based on search experience
\end{itemize}
\end{itemize}
\end{frame}

%% \begin{frame}
%% \frametitle{Refined Process}
%% \begin{center}
%% \begin{tikzpicture}
%% \node[shape=rectangle,fill=pantone127-4] (problem) at (3,7) {Problem};
%% \node (human) at (3,6) {Human};
%% \node[shape=rectangle,fill=pantone127-4] (model) at (3,5) {User Model};
%% \node (reformulation) at (3,4) {Reformulation};
%% \node[shape=rectangle,fill=pantone127-4] (implementation) at (3,3) {Implementation Model};
%% \node (solver) at (3,2) {Constraint Solver/Search};
%% \node[shape=rectangle,fill=pantone157-8] (solution) at (3,1) {Solution};
%% \draw[-] (problem) -- (human);
%% \draw[->] (human) -- (model);
%% \draw[-] (model) -- (reformulation);
%% \draw[->] (reformulation) -- (implementation);
%% \draw[-] (implementation) -- (solver);
%% \draw[->] (solver) -- (0,2) -- (0,3) -- (implementation);
%% \draw[->] (solver) -- (solution);
%% \end{tikzpicture}
%% \end{center}
%% \end{frame}

%\begin{frame}
%\frametitle{What is CP actually used for?}
%\url{http://hsimonis.wordpress.com}
%\includegraphics[width=10cm]{../introduction/images/blog}
%\end{frame}

%% \begin{frame}
%% \frametitle{Exercises}
%% \begin{itemize}
%% \item Install Constraint Solver of your Choice
%% \item MiniZinc (MiningZinc) will be used in later sessions
%% \item Run Sendmoremoney, Sudoku, Queens examples
%% \item Can you choose which propagator to use for alldifferent?
%% \item Can you use different search strategies?
%% \end{itemize}
%% \end{frame}

\begin{frame}
\frametitle{More Learning Resources}
\begin{itemize}
\item Survey of Methods, Resources, and Formats for Teaching Constraint Programming
\begin{itemize}
\item by Tejas Santanam, Helmut Simonis
\item \url{https://doi.org/10.48550/arXiv.2403.12717}
\item Based on survey of community for WTCP 2023
\item \url{https://hsimonis.github.io/WTCP2023/}
\end{itemize}
\end{itemize}
\end{frame}


%%   }

\clearpage
\mode<all>{
\part{What is missing?}
}
\begin{frame}
\partpage
\end{frame}

\begin{frame}
\frametitle{Many Specialized Topics}
\begin{itemize}
\item How to design efficient core engine
\item Hybrids with LP/MIP tools
\item Hybrids with SAT
\item Symmetry breaking
\item Use of MDD/BDD to encode sets of solutions
\item High level modelling tools
\item Debugging/visualization
\end{itemize}
\end{frame}

\begin{frame}
\frametitle{Reformulation}
\begin{itemize}
\item Just because the user has modelled it this way, it doesn't mean we have to solve it that way
\begin{itemize}
\item Replace some constraint(s) by other, equivalent constraints
\item Because we don't have that constraint in our system
\item For performance
\end{itemize}
\end{itemize}
\end{frame}

\begin{frame}
\frametitle{Learning}
\begin{itemize}
\item While solving the problem we can learn how to strengthen the model/search 
\begin{itemize}
\item Understand which constraints/method contribute to propagation and change schedule
\item Learn no-good constraints by explaining failure
\item Adapt search strategy based on search experience
\end{itemize}
\end{itemize}
\end{frame}

%% \begin{frame}
%% \frametitle{Refined Process}
%% \begin{center}
%% \begin{tikzpicture}
%% \node[shape=rectangle,fill=pantone127-4] (problem) at (3,7) {Problem};
%% \node (human) at (3,6) {Human};
%% \node[shape=rectangle,fill=pantone127-4] (model) at (3,5) {User Model};
%% \node (reformulation) at (3,4) {Reformulation};
%% \node[shape=rectangle,fill=pantone127-4] (implementation) at (3,3) {Implementation Model};
%% \node (solver) at (3,2) {Constraint Solver/Search};
%% \node[shape=rectangle,fill=pantone157-8] (solution) at (3,1) {Solution};
%% \draw[-] (problem) -- (human);
%% \draw[->] (human) -- (model);
%% \draw[-] (model) -- (reformulation);
%% \draw[->] (reformulation) -- (implementation);
%% \draw[-] (implementation) -- (solver);
%% \draw[->] (solver) -- (0,2) -- (0,3) -- (implementation);
%% \draw[->] (solver) -- (solution);
%% \end{tikzpicture}
%% \end{center}
%% \end{frame}

%\begin{frame}
%\frametitle{What is CP actually used for?}
%\url{http://hsimonis.wordpress.com}
%\includegraphics[width=10cm]{../introduction/images/blog}
%\end{frame}

%% \begin{frame}
%% \frametitle{Exercises}
%% \begin{itemize}
%% \item Install Constraint Solver of your Choice
%% \item MiniZinc (MiningZinc) will be used in later sessions
%% \item Run Sendmoremoney, Sudoku, Queens examples
%% \item Can you choose which propagator to use for alldifferent?
%% \item Can you use different search strategies?
%% \end{itemize}
%% \end{frame}

\begin{frame}
\frametitle{More Learning Resources}
\begin{itemize}
\item Survey of Methods, Resources, and Formats for Teaching Constraint Programming
\begin{itemize}
\item by Tejas Santanam, Helmut Simonis
\item \url{https://doi.org/10.48550/arXiv.2403.12717}
\item Based on survey of community for WTCP 2023
\item \url{https://hsimonis.github.io/WTCP2023/}
\end{itemize}
\end{itemize}
\end{frame}


%%   }

%% \clearpage
%% \mode<all>{
%%   \part{More Global Constraints}
%% }
%% \begin{frame}
%%   \partpage
%% \end{frame}
%% \mode<all>{
%%   \begin{frame}
\frametitle{Objectives}
\begin{itemize}
\item Overview of Core Constraint Programming
\item Three Main Concepts
\begin{itemize}
\item Constraint Propagation
\item Global Constraints
\item Customizing Search
\end{itemize}
%\item Get Some Experience with MiniZinc
\item Topics will be treated in more detail in later parts of the school
\item Based on Examples, not Formal Description
\end{itemize}
\end{frame}

\begin{frame}
\frametitle{Outline}
\begin{itemize}
\item Why Constraint Programming?
\item Constraint Propagation
\item Global Constraints
\item Customizing Search
%\item Choices in Modelling
%\item Decomposition
%\item More Global Constraints
%\item Methodology
%\item What is missing?
\end{itemize}
\end{frame}

% \begin{frame}
  % \frametitle{Using MiniZinc IDE}
  % \begin{itemize}
  % \item Developed in the Australian NICTA project
  % \item Maintained by Monash University
  % \item Modelling tool with multiple back-end solvers
  % \item Available from \url{https://www.minizinc.org/}
  % \end{itemize}
% \end{frame}


% \begin{frame}
% \frametitle{Examples in ECLiPSe}
% \begin{itemize}
% \item Open sourced constraint programming language
% \item Development goes back to 1985
% \item ECRC, ICL, IC-Parc, PTL, Cisco
% \item \url{https://eclipseclp.org/}
% \item Specialities
% \begin{itemize}
% \item Develop new solvers for specific domains
% \item Integration with MIP
% \end{itemize}
% \item Not included in bundled MiniZinc IDE
% \item Specialized visualization tools used here
  % \begin{itemize}
  % \item CP-Viz, Simonis et al. 2010
    % \end{itemize}
% \end{itemize}
% \end{frame}

\begin{frame}
\frametitle{Tutorial Based on ECLiPSe ELearning Course}
\begin{itemize}
\item Self-study course in constraint programming
\item Supported by Cisco Systems and Silicon Valley Community Foundation
\item Multi-media format, video lectures, slides, handout etc
\item {\small \url{https://eclipseclp.org/ELearning/index.html}}
\end{itemize}
\end{frame}

\begin{frame}
\frametitle{Also Part of CRT-AI Constraint Week}
\begin{itemize}
\item Annual one week course on CP and Optimization in Ireland
\item Part of national training program for PhD students in AI
\item \url{https://www.crt-ai.ie/}
\end{itemize}
\end{frame}



\begin{frame}
\frametitle{Constraint Programming - in a nutshell}
\begin{itemize}
\item Declarative description of problems with
\begin{itemize}
\item {\em Variables} which range over (finite) sets of values
\item {\em Constraints} over subsets of variables which restrict possible value combinations
\item A {\em solution} is a value assignment which satisfies all constraints
\end{itemize}

\item Constraint propagation/reasoning
\begin{itemize}
\item Removing inconsistent values for variables
\item Detect failure if constraint can not be satisfied
\item Interaction of constraints via shared variables
\item Incomplete
\end{itemize}

\item Search
\begin{itemize}
\item User controlled assignment of values to variables
\item Each step triggers constraint propagation 
\end{itemize}
\item Different domains require/allow different methods
\end{itemize}
\end{frame}

\begin{frame}
  \frametitle{Constraint Programming is Different}
  \begin{itemize}
  \item Declarative Programming
    \begin{itemize}
    \item Concentrate on what you want
      \item Not how to get there
      \item Program != Algorithm
      \item Program = Model
    \end{itemize}
    \item Applied to Combinatorial Problems
      \begin{itemize}
        \item No complete polynomial algorithms known (exist?)
        \item CP less ad-hoc than heuristics
        \item Models can evolve
  \end{itemize}
  \end{itemize}
  \end{frame}
    
\begin{frame}
  \frametitle{A Subtractive Process}
  \begin{textblock}{4}(8,-3)
    \includegraphics[width=4cm]{../introduction/images/stages}
  \end{textblock}
  \vfill
  \begin{quote}
    ``Oh, bosh, as Mr. Ruskin says. Sculpture, per se, is the simplest thing in the world. All you have to do is to take a big chunk of marble and a hammer and chisel, make up your mind what you are about to create and chip off all the marble you don't want.''-Paris Gaulois.
  \end{quote}
  
  {\tiny Source: \url{https://quoteinvestigator.com/2014/06/22/chip-away/}}
\end{frame}

\begin{frame}
\frametitle{Basic Process}
\begin{center}
\begin{tikzpicture}
\node[shape=rectangle,fill=pantone127-4] (problem) at (2,5) {Problem};
\node (human) at (2,4) {Human};
\node[shape=rectangle,fill=pantone127-4] (model) at (2,3) {Model};
\node (solver) at (2,2) {Constraint Solver/Search};
\node[shape=rectangle,fill=pantone157-8] (solution) at (2,1) {Solution};
\draw[-] (problem) -- (human);
\draw[->] (human) -- (model);
\draw[-] (model) -- (solver);
\draw[->] (solver) -- (solution);
\end{tikzpicture}
\end{center}
\end{frame}

\begin{frame}
\frametitle{More Realistic}
\begin{center}
\begin{tikzpicture}[xscale=4.0]
\node[shape=rectangle,fill=pantone127-4] (problem) at (2,5) {Problem};
\node (human) at (2,4) {Human};
\node[shape=rectangle,fill=pantone127-4] (model) at (2,3) {Model};
\node (solver) at (2,2) {Constraint Solver/Search};
\node[shape=rectangle,fill=pantone24-8] (hangs) at (1,1) {Hangs};
\node[shape=rectangle,fill=pantone157-8] (solution) at (2,1) {Solution};
\node[shape=rectangle,fill=pantone24-8] (wrong) at (3,1) {Wrong Solution};
\draw[-] (problem) -- (human);
\draw[->] (human) -- (model);
\draw[-] (model) -- (solver);
\draw[->] (solver) -- (hangs);
\draw[->] (solver) -- (solution);
\draw[->] (solver) -- (wrong);
\draw[->] (hangs) -- (1,4) -- (human);
\draw[->] (wrong) -- (3,4) -- (human);
\end{tikzpicture}
\end{center}
\end{frame}

\begin{frame}
\frametitle{Dual Role of Model}
\begin{itemize}
\item Allows Human to Express Problem
\begin{itemize}
\item Close to Problem Domain
\item Constraints as Abstractions
\end{itemize}
\item Allows Solver to Execute
\begin{itemize}
\item Variables as Communication Mechanism
\item Constraints as Algorithms
\end{itemize}
\end{itemize}
\end{frame}

\begin{frame}
\frametitle{Modelling Frameworks}
\begin{itemize}
\item \href{https://www.minizinc.org/}{MiniZinc} (NICTA, Monash University, Australia)
\item \href{https://github.com/eomahony/Numberjack}{NumberJack} (Insight, Ireland) 
\item \href{https://www-users.york.ac.uk/peter.nightingale/savilerow//}{EssencePrime/SavilleRow} (UK) 
\item \href{https://cpmpy.readthedocs.io/en/latest/}{CPMpy} (KU Leuven)
\item Allow use of multiple back-end solvers
\item Compile model into variants for each solver
\item A priori solver independent model(CP, MIP, SAT)
\end{itemize}
\end{frame} 

\begin{frame}
\frametitle{Framework Process}
\begin{center}
\begin{tikzpicture}[xscale=2]
\node[shape=rectangle,fill=pantone127-4] (problem) at (2.5,5) {Problem};
\node (human) at (2.5,4) {Human};
\node[shape=rectangle,fill=pantone127-4] (model) at (2.5,3) {Model};
\node (compile) at (2.5,2) {Compile/Reformulate};
\node (solver1) at (1,1) {CP};
\node (solver2) at (2,1) {MIP};
\node (solver3) at (3,1) {SAT};
\node (solver4) at (4,1) {Other};
\node[shape=rectangle,fill=pantone157-8] (solution1) at (1,0) {Solution};
\node[shape=rectangle,fill=pantone157-8] (solution2) at (2,0) {Solution};
\node[shape=rectangle,fill=pantone157-8] (solution3) at (3,0) {Solution};
\node[shape=rectangle,fill=pantone157-8] (solution4) at (4,0) {Solution};
\draw[-] (problem) -- (human);
\draw[->] (human) -- (model);
\draw[-] (model) -- (compile);
\draw[-] (compile) -- (solver1);
\draw[-] (compile) -- (solver2);
\draw[-] (compile) -- (solver3);
\draw[-] (compile) -- (solver4);
\draw[->] (solver1) -- (solution1);
\draw[->] (solver2) -- (solution2);
\draw[->] (solver3) -- (solution3);
\draw[->] (solver4) -- (solution4);
\end{tikzpicture}
\end{center}
\end{frame}

% \begin{frame}
  % \frametitle{Do It Now!}
  % \begin{itemize}
  % \item Download and install Minizinc
  % \item \url{https://www.minizinc.org/}
  % \end{itemize}
% \end{frame}


\begin{frame}
\frametitle{Why use Puzzles as Examples?}
\begin{itemize}
\item Easy to understand the problem
\item Solvable by hand without specialized knowledge
\item Possible to compare automated to manual solving process
\end{itemize}

\begin{quote}
The puzzle, though inanimate, is presented as a solvable problem without lasting negative consequences, a very low-risk low-reward situation. By being a puzzle, the object is attempting to convince the user that it must be completed.
\end{quote}
{\tiny Source: \href{https://www.everydayrhetoric.com/2020/03/the-rhetoric-of-puzzles/}{Every Day Rhetoric}}
\end{frame}


\mode<all>{
\part{Basic Constraint Propagation}
}
\begin{frame}
\partpage
\end{frame}
\mode<all>{
\begin{frame}
\frametitle{Objectives}
\begin{itemize}
\item Overview of Core Constraint Programming
\item Three Main Concepts
\begin{itemize}
\item Constraint Propagation
\item Global Constraints
\item Customizing Search
\end{itemize}
%\item Get Some Experience with MiniZinc
\item Topics will be treated in more detail in later parts of the school
\item Based on Examples, not Formal Description
\end{itemize}
\end{frame}

\begin{frame}
\frametitle{Outline}
\begin{itemize}
\item Why Constraint Programming?
\item Constraint Propagation
\item Global Constraints
\item Customizing Search
%\item Choices in Modelling
%\item Decomposition
%\item More Global Constraints
%\item Methodology
%\item What is missing?
\end{itemize}
\end{frame}

% \begin{frame}
  % \frametitle{Using MiniZinc IDE}
  % \begin{itemize}
  % \item Developed in the Australian NICTA project
  % \item Maintained by Monash University
  % \item Modelling tool with multiple back-end solvers
  % \item Available from \url{https://www.minizinc.org/}
  % \end{itemize}
% \end{frame}


% \begin{frame}
% \frametitle{Examples in ECLiPSe}
% \begin{itemize}
% \item Open sourced constraint programming language
% \item Development goes back to 1985
% \item ECRC, ICL, IC-Parc, PTL, Cisco
% \item \url{https://eclipseclp.org/}
% \item Specialities
% \begin{itemize}
% \item Develop new solvers for specific domains
% \item Integration with MIP
% \end{itemize}
% \item Not included in bundled MiniZinc IDE
% \item Specialized visualization tools used here
  % \begin{itemize}
  % \item CP-Viz, Simonis et al. 2010
    % \end{itemize}
% \end{itemize}
% \end{frame}

\begin{frame}
\frametitle{Tutorial Based on ECLiPSe ELearning Course}
\begin{itemize}
\item Self-study course in constraint programming
\item Supported by Cisco Systems and Silicon Valley Community Foundation
\item Multi-media format, video lectures, slides, handout etc
\item {\small \url{https://eclipseclp.org/ELearning/index.html}}
\end{itemize}
\end{frame}

\begin{frame}
\frametitle{Also Part of CRT-AI Constraint Week}
\begin{itemize}
\item Annual one week course on CP and Optimization in Ireland
\item Part of national training program for PhD students in AI
\item \url{https://www.crt-ai.ie/}
\end{itemize}
\end{frame}



\begin{frame}
\frametitle{Constraint Programming - in a nutshell}
\begin{itemize}
\item Declarative description of problems with
\begin{itemize}
\item {\em Variables} which range over (finite) sets of values
\item {\em Constraints} over subsets of variables which restrict possible value combinations
\item A {\em solution} is a value assignment which satisfies all constraints
\end{itemize}

\item Constraint propagation/reasoning
\begin{itemize}
\item Removing inconsistent values for variables
\item Detect failure if constraint can not be satisfied
\item Interaction of constraints via shared variables
\item Incomplete
\end{itemize}

\item Search
\begin{itemize}
\item User controlled assignment of values to variables
\item Each step triggers constraint propagation 
\end{itemize}
\item Different domains require/allow different methods
\end{itemize}
\end{frame}

\begin{frame}
  \frametitle{Constraint Programming is Different}
  \begin{itemize}
  \item Declarative Programming
    \begin{itemize}
    \item Concentrate on what you want
      \item Not how to get there
      \item Program != Algorithm
      \item Program = Model
    \end{itemize}
    \item Applied to Combinatorial Problems
      \begin{itemize}
        \item No complete polynomial algorithms known (exist?)
        \item CP less ad-hoc than heuristics
        \item Models can evolve
  \end{itemize}
  \end{itemize}
  \end{frame}
    
\begin{frame}
  \frametitle{A Subtractive Process}
  \begin{textblock}{4}(8,-3)
    \includegraphics[width=4cm]{../introduction/images/stages}
  \end{textblock}
  \vfill
  \begin{quote}
    ``Oh, bosh, as Mr. Ruskin says. Sculpture, per se, is the simplest thing in the world. All you have to do is to take a big chunk of marble and a hammer and chisel, make up your mind what you are about to create and chip off all the marble you don't want.''-Paris Gaulois.
  \end{quote}
  
  {\tiny Source: \url{https://quoteinvestigator.com/2014/06/22/chip-away/}}
\end{frame}

\begin{frame}
\frametitle{Basic Process}
\begin{center}
\begin{tikzpicture}
\node[shape=rectangle,fill=pantone127-4] (problem) at (2,5) {Problem};
\node (human) at (2,4) {Human};
\node[shape=rectangle,fill=pantone127-4] (model) at (2,3) {Model};
\node (solver) at (2,2) {Constraint Solver/Search};
\node[shape=rectangle,fill=pantone157-8] (solution) at (2,1) {Solution};
\draw[-] (problem) -- (human);
\draw[->] (human) -- (model);
\draw[-] (model) -- (solver);
\draw[->] (solver) -- (solution);
\end{tikzpicture}
\end{center}
\end{frame}

\begin{frame}
\frametitle{More Realistic}
\begin{center}
\begin{tikzpicture}[xscale=4.0]
\node[shape=rectangle,fill=pantone127-4] (problem) at (2,5) {Problem};
\node (human) at (2,4) {Human};
\node[shape=rectangle,fill=pantone127-4] (model) at (2,3) {Model};
\node (solver) at (2,2) {Constraint Solver/Search};
\node[shape=rectangle,fill=pantone24-8] (hangs) at (1,1) {Hangs};
\node[shape=rectangle,fill=pantone157-8] (solution) at (2,1) {Solution};
\node[shape=rectangle,fill=pantone24-8] (wrong) at (3,1) {Wrong Solution};
\draw[-] (problem) -- (human);
\draw[->] (human) -- (model);
\draw[-] (model) -- (solver);
\draw[->] (solver) -- (hangs);
\draw[->] (solver) -- (solution);
\draw[->] (solver) -- (wrong);
\draw[->] (hangs) -- (1,4) -- (human);
\draw[->] (wrong) -- (3,4) -- (human);
\end{tikzpicture}
\end{center}
\end{frame}

\begin{frame}
\frametitle{Dual Role of Model}
\begin{itemize}
\item Allows Human to Express Problem
\begin{itemize}
\item Close to Problem Domain
\item Constraints as Abstractions
\end{itemize}
\item Allows Solver to Execute
\begin{itemize}
\item Variables as Communication Mechanism
\item Constraints as Algorithms
\end{itemize}
\end{itemize}
\end{frame}

\begin{frame}
\frametitle{Modelling Frameworks}
\begin{itemize}
\item \href{https://www.minizinc.org/}{MiniZinc} (NICTA, Monash University, Australia)
\item \href{https://github.com/eomahony/Numberjack}{NumberJack} (Insight, Ireland) 
\item \href{https://www-users.york.ac.uk/peter.nightingale/savilerow//}{EssencePrime/SavilleRow} (UK) 
\item \href{https://cpmpy.readthedocs.io/en/latest/}{CPMpy} (KU Leuven)
\item Allow use of multiple back-end solvers
\item Compile model into variants for each solver
\item A priori solver independent model(CP, MIP, SAT)
\end{itemize}
\end{frame} 

\begin{frame}
\frametitle{Framework Process}
\begin{center}
\begin{tikzpicture}[xscale=2]
\node[shape=rectangle,fill=pantone127-4] (problem) at (2.5,5) {Problem};
\node (human) at (2.5,4) {Human};
\node[shape=rectangle,fill=pantone127-4] (model) at (2.5,3) {Model};
\node (compile) at (2.5,2) {Compile/Reformulate};
\node (solver1) at (1,1) {CP};
\node (solver2) at (2,1) {MIP};
\node (solver3) at (3,1) {SAT};
\node (solver4) at (4,1) {Other};
\node[shape=rectangle,fill=pantone157-8] (solution1) at (1,0) {Solution};
\node[shape=rectangle,fill=pantone157-8] (solution2) at (2,0) {Solution};
\node[shape=rectangle,fill=pantone157-8] (solution3) at (3,0) {Solution};
\node[shape=rectangle,fill=pantone157-8] (solution4) at (4,0) {Solution};
\draw[-] (problem) -- (human);
\draw[->] (human) -- (model);
\draw[-] (model) -- (compile);
\draw[-] (compile) -- (solver1);
\draw[-] (compile) -- (solver2);
\draw[-] (compile) -- (solver3);
\draw[-] (compile) -- (solver4);
\draw[->] (solver1) -- (solution1);
\draw[->] (solver2) -- (solution2);
\draw[->] (solver3) -- (solution3);
\draw[->] (solver4) -- (solution4);
\end{tikzpicture}
\end{center}
\end{frame}

% \begin{frame}
  % \frametitle{Do It Now!}
  % \begin{itemize}
  % \item Download and install Minizinc
  % \item \url{https://www.minizinc.org/}
  % \end{itemize}
% \end{frame}


\begin{frame}
\frametitle{Why use Puzzles as Examples?}
\begin{itemize}
\item Easy to understand the problem
\item Solvable by hand without specialized knowledge
\item Possible to compare automated to manual solving process
\end{itemize}

\begin{quote}
The puzzle, though inanimate, is presented as a solvable problem without lasting negative consequences, a very low-risk low-reward situation. By being a puzzle, the object is attempting to convince the user that it must be completed.
\end{quote}
{\tiny Source: \href{https://www.everydayrhetoric.com/2020/03/the-rhetoric-of-puzzles/}{Every Day Rhetoric}}
\end{frame}


\mode<all>{
\part{Basic Constraint Propagation}
}
\begin{frame}
\partpage
\end{frame}
\mode<all>{
\input{../sendmore/body}
}

\mode<all>{
\part{Global Constraints}
}
\begin{frame}
\partpage
\end{frame}
\mode<all>{
\input{../sudoku/body}
}

\clearpage
\mode<all>{
\part{Customizing Search}
}
\begin{frame}
\partpage
\end{frame}
\mode<all>{
\input{../nqueen/body}
}


%% \clearpage
%% \mode<all>{
%%   \part{Problem Decomposition}
%% }
%% \begin{frame}
%%   \partpage
%% \end{frame}
%% \mode<all>{
%%   \input{../party/body}
%%   }

%% \clearpage
%% \mode<all>{
%%   \part{More Global Constraints}
%% }
%% \begin{frame}
%%   \partpage
%% \end{frame}
%% \mode<all>{
%%   \input{../car/body}
%%   }

%% \clearpage
%% \mode<all>{
%%   \part{Methodology}
%% }
%% \begin{frame}
%%   \partpage
%% \end{frame}
%% \mode<all>{
%%   \input{../methodology/body}
%%   }

\clearpage
\mode<all>{
\part{What is missing?}
}
\begin{frame}
\partpage
\end{frame}

\begin{frame}
\frametitle{Many Specialized Topics}
\begin{itemize}
\item How to design efficient core engine
\item Hybrids with LP/MIP tools
\item Hybrids with SAT
\item Symmetry breaking
\item Use of MDD/BDD to encode sets of solutions
\item High level modelling tools
\item Debugging/visualization
\end{itemize}
\end{frame}

\begin{frame}
\frametitle{Reformulation}
\begin{itemize}
\item Just because the user has modelled it this way, it doesn't mean we have to solve it that way
\begin{itemize}
\item Replace some constraint(s) by other, equivalent constraints
\item Because we don't have that constraint in our system
\item For performance
\end{itemize}
\end{itemize}
\end{frame}

\begin{frame}
\frametitle{Learning}
\begin{itemize}
\item While solving the problem we can learn how to strengthen the model/search 
\begin{itemize}
\item Understand which constraints/method contribute to propagation and change schedule
\item Learn no-good constraints by explaining failure
\item Adapt search strategy based on search experience
\end{itemize}
\end{itemize}
\end{frame}

%% \begin{frame}
%% \frametitle{Refined Process}
%% \begin{center}
%% \begin{tikzpicture}
%% \node[shape=rectangle,fill=pantone127-4] (problem) at (3,7) {Problem};
%% \node (human) at (3,6) {Human};
%% \node[shape=rectangle,fill=pantone127-4] (model) at (3,5) {User Model};
%% \node (reformulation) at (3,4) {Reformulation};
%% \node[shape=rectangle,fill=pantone127-4] (implementation) at (3,3) {Implementation Model};
%% \node (solver) at (3,2) {Constraint Solver/Search};
%% \node[shape=rectangle,fill=pantone157-8] (solution) at (3,1) {Solution};
%% \draw[-] (problem) -- (human);
%% \draw[->] (human) -- (model);
%% \draw[-] (model) -- (reformulation);
%% \draw[->] (reformulation) -- (implementation);
%% \draw[-] (implementation) -- (solver);
%% \draw[->] (solver) -- (0,2) -- (0,3) -- (implementation);
%% \draw[->] (solver) -- (solution);
%% \end{tikzpicture}
%% \end{center}
%% \end{frame}

%\begin{frame}
%\frametitle{What is CP actually used for?}
%\url{http://hsimonis.wordpress.com}
%\includegraphics[width=10cm]{../introduction/images/blog}
%\end{frame}

%% \begin{frame}
%% \frametitle{Exercises}
%% \begin{itemize}
%% \item Install Constraint Solver of your Choice
%% \item MiniZinc (MiningZinc) will be used in later sessions
%% \item Run Sendmoremoney, Sudoku, Queens examples
%% \item Can you choose which propagator to use for alldifferent?
%% \item Can you use different search strategies?
%% \end{itemize}
%% \end{frame}

\begin{frame}
\frametitle{More Learning Resources}
\begin{itemize}
\item Survey of Methods, Resources, and Formats for Teaching Constraint Programming
\begin{itemize}
\item by Tejas Santanam, Helmut Simonis
\item \url{https://doi.org/10.48550/arXiv.2403.12717}
\item Based on survey of community for WTCP 2023
\item \url{https://hsimonis.github.io/WTCP2023/}
\end{itemize}
\end{itemize}
\end{frame}


}

\mode<all>{
\part{Global Constraints}
}
\begin{frame}
\partpage
\end{frame}
\mode<all>{
\begin{frame}
\frametitle{Objectives}
\begin{itemize}
\item Overview of Core Constraint Programming
\item Three Main Concepts
\begin{itemize}
\item Constraint Propagation
\item Global Constraints
\item Customizing Search
\end{itemize}
%\item Get Some Experience with MiniZinc
\item Topics will be treated in more detail in later parts of the school
\item Based on Examples, not Formal Description
\end{itemize}
\end{frame}

\begin{frame}
\frametitle{Outline}
\begin{itemize}
\item Why Constraint Programming?
\item Constraint Propagation
\item Global Constraints
\item Customizing Search
%\item Choices in Modelling
%\item Decomposition
%\item More Global Constraints
%\item Methodology
%\item What is missing?
\end{itemize}
\end{frame}

% \begin{frame}
  % \frametitle{Using MiniZinc IDE}
  % \begin{itemize}
  % \item Developed in the Australian NICTA project
  % \item Maintained by Monash University
  % \item Modelling tool with multiple back-end solvers
  % \item Available from \url{https://www.minizinc.org/}
  % \end{itemize}
% \end{frame}


% \begin{frame}
% \frametitle{Examples in ECLiPSe}
% \begin{itemize}
% \item Open sourced constraint programming language
% \item Development goes back to 1985
% \item ECRC, ICL, IC-Parc, PTL, Cisco
% \item \url{https://eclipseclp.org/}
% \item Specialities
% \begin{itemize}
% \item Develop new solvers for specific domains
% \item Integration with MIP
% \end{itemize}
% \item Not included in bundled MiniZinc IDE
% \item Specialized visualization tools used here
  % \begin{itemize}
  % \item CP-Viz, Simonis et al. 2010
    % \end{itemize}
% \end{itemize}
% \end{frame}

\begin{frame}
\frametitle{Tutorial Based on ECLiPSe ELearning Course}
\begin{itemize}
\item Self-study course in constraint programming
\item Supported by Cisco Systems and Silicon Valley Community Foundation
\item Multi-media format, video lectures, slides, handout etc
\item {\small \url{https://eclipseclp.org/ELearning/index.html}}
\end{itemize}
\end{frame}

\begin{frame}
\frametitle{Also Part of CRT-AI Constraint Week}
\begin{itemize}
\item Annual one week course on CP and Optimization in Ireland
\item Part of national training program for PhD students in AI
\item \url{https://www.crt-ai.ie/}
\end{itemize}
\end{frame}



\begin{frame}
\frametitle{Constraint Programming - in a nutshell}
\begin{itemize}
\item Declarative description of problems with
\begin{itemize}
\item {\em Variables} which range over (finite) sets of values
\item {\em Constraints} over subsets of variables which restrict possible value combinations
\item A {\em solution} is a value assignment which satisfies all constraints
\end{itemize}

\item Constraint propagation/reasoning
\begin{itemize}
\item Removing inconsistent values for variables
\item Detect failure if constraint can not be satisfied
\item Interaction of constraints via shared variables
\item Incomplete
\end{itemize}

\item Search
\begin{itemize}
\item User controlled assignment of values to variables
\item Each step triggers constraint propagation 
\end{itemize}
\item Different domains require/allow different methods
\end{itemize}
\end{frame}

\begin{frame}
  \frametitle{Constraint Programming is Different}
  \begin{itemize}
  \item Declarative Programming
    \begin{itemize}
    \item Concentrate on what you want
      \item Not how to get there
      \item Program != Algorithm
      \item Program = Model
    \end{itemize}
    \item Applied to Combinatorial Problems
      \begin{itemize}
        \item No complete polynomial algorithms known (exist?)
        \item CP less ad-hoc than heuristics
        \item Models can evolve
  \end{itemize}
  \end{itemize}
  \end{frame}
    
\begin{frame}
  \frametitle{A Subtractive Process}
  \begin{textblock}{4}(8,-3)
    \includegraphics[width=4cm]{../introduction/images/stages}
  \end{textblock}
  \vfill
  \begin{quote}
    ``Oh, bosh, as Mr. Ruskin says. Sculpture, per se, is the simplest thing in the world. All you have to do is to take a big chunk of marble and a hammer and chisel, make up your mind what you are about to create and chip off all the marble you don't want.''-Paris Gaulois.
  \end{quote}
  
  {\tiny Source: \url{https://quoteinvestigator.com/2014/06/22/chip-away/}}
\end{frame}

\begin{frame}
\frametitle{Basic Process}
\begin{center}
\begin{tikzpicture}
\node[shape=rectangle,fill=pantone127-4] (problem) at (2,5) {Problem};
\node (human) at (2,4) {Human};
\node[shape=rectangle,fill=pantone127-4] (model) at (2,3) {Model};
\node (solver) at (2,2) {Constraint Solver/Search};
\node[shape=rectangle,fill=pantone157-8] (solution) at (2,1) {Solution};
\draw[-] (problem) -- (human);
\draw[->] (human) -- (model);
\draw[-] (model) -- (solver);
\draw[->] (solver) -- (solution);
\end{tikzpicture}
\end{center}
\end{frame}

\begin{frame}
\frametitle{More Realistic}
\begin{center}
\begin{tikzpicture}[xscale=4.0]
\node[shape=rectangle,fill=pantone127-4] (problem) at (2,5) {Problem};
\node (human) at (2,4) {Human};
\node[shape=rectangle,fill=pantone127-4] (model) at (2,3) {Model};
\node (solver) at (2,2) {Constraint Solver/Search};
\node[shape=rectangle,fill=pantone24-8] (hangs) at (1,1) {Hangs};
\node[shape=rectangle,fill=pantone157-8] (solution) at (2,1) {Solution};
\node[shape=rectangle,fill=pantone24-8] (wrong) at (3,1) {Wrong Solution};
\draw[-] (problem) -- (human);
\draw[->] (human) -- (model);
\draw[-] (model) -- (solver);
\draw[->] (solver) -- (hangs);
\draw[->] (solver) -- (solution);
\draw[->] (solver) -- (wrong);
\draw[->] (hangs) -- (1,4) -- (human);
\draw[->] (wrong) -- (3,4) -- (human);
\end{tikzpicture}
\end{center}
\end{frame}

\begin{frame}
\frametitle{Dual Role of Model}
\begin{itemize}
\item Allows Human to Express Problem
\begin{itemize}
\item Close to Problem Domain
\item Constraints as Abstractions
\end{itemize}
\item Allows Solver to Execute
\begin{itemize}
\item Variables as Communication Mechanism
\item Constraints as Algorithms
\end{itemize}
\end{itemize}
\end{frame}

\begin{frame}
\frametitle{Modelling Frameworks}
\begin{itemize}
\item \href{https://www.minizinc.org/}{MiniZinc} (NICTA, Monash University, Australia)
\item \href{https://github.com/eomahony/Numberjack}{NumberJack} (Insight, Ireland) 
\item \href{https://www-users.york.ac.uk/peter.nightingale/savilerow//}{EssencePrime/SavilleRow} (UK) 
\item \href{https://cpmpy.readthedocs.io/en/latest/}{CPMpy} (KU Leuven)
\item Allow use of multiple back-end solvers
\item Compile model into variants for each solver
\item A priori solver independent model(CP, MIP, SAT)
\end{itemize}
\end{frame} 

\begin{frame}
\frametitle{Framework Process}
\begin{center}
\begin{tikzpicture}[xscale=2]
\node[shape=rectangle,fill=pantone127-4] (problem) at (2.5,5) {Problem};
\node (human) at (2.5,4) {Human};
\node[shape=rectangle,fill=pantone127-4] (model) at (2.5,3) {Model};
\node (compile) at (2.5,2) {Compile/Reformulate};
\node (solver1) at (1,1) {CP};
\node (solver2) at (2,1) {MIP};
\node (solver3) at (3,1) {SAT};
\node (solver4) at (4,1) {Other};
\node[shape=rectangle,fill=pantone157-8] (solution1) at (1,0) {Solution};
\node[shape=rectangle,fill=pantone157-8] (solution2) at (2,0) {Solution};
\node[shape=rectangle,fill=pantone157-8] (solution3) at (3,0) {Solution};
\node[shape=rectangle,fill=pantone157-8] (solution4) at (4,0) {Solution};
\draw[-] (problem) -- (human);
\draw[->] (human) -- (model);
\draw[-] (model) -- (compile);
\draw[-] (compile) -- (solver1);
\draw[-] (compile) -- (solver2);
\draw[-] (compile) -- (solver3);
\draw[-] (compile) -- (solver4);
\draw[->] (solver1) -- (solution1);
\draw[->] (solver2) -- (solution2);
\draw[->] (solver3) -- (solution3);
\draw[->] (solver4) -- (solution4);
\end{tikzpicture}
\end{center}
\end{frame}

% \begin{frame}
  % \frametitle{Do It Now!}
  % \begin{itemize}
  % \item Download and install Minizinc
  % \item \url{https://www.minizinc.org/}
  % \end{itemize}
% \end{frame}


\begin{frame}
\frametitle{Why use Puzzles as Examples?}
\begin{itemize}
\item Easy to understand the problem
\item Solvable by hand without specialized knowledge
\item Possible to compare automated to manual solving process
\end{itemize}

\begin{quote}
The puzzle, though inanimate, is presented as a solvable problem without lasting negative consequences, a very low-risk low-reward situation. By being a puzzle, the object is attempting to convince the user that it must be completed.
\end{quote}
{\tiny Source: \href{https://www.everydayrhetoric.com/2020/03/the-rhetoric-of-puzzles/}{Every Day Rhetoric}}
\end{frame}


\mode<all>{
\part{Basic Constraint Propagation}
}
\begin{frame}
\partpage
\end{frame}
\mode<all>{
\input{../sendmore/body}
}

\mode<all>{
\part{Global Constraints}
}
\begin{frame}
\partpage
\end{frame}
\mode<all>{
\input{../sudoku/body}
}

\clearpage
\mode<all>{
\part{Customizing Search}
}
\begin{frame}
\partpage
\end{frame}
\mode<all>{
\input{../nqueen/body}
}


%% \clearpage
%% \mode<all>{
%%   \part{Problem Decomposition}
%% }
%% \begin{frame}
%%   \partpage
%% \end{frame}
%% \mode<all>{
%%   \input{../party/body}
%%   }

%% \clearpage
%% \mode<all>{
%%   \part{More Global Constraints}
%% }
%% \begin{frame}
%%   \partpage
%% \end{frame}
%% \mode<all>{
%%   \input{../car/body}
%%   }

%% \clearpage
%% \mode<all>{
%%   \part{Methodology}
%% }
%% \begin{frame}
%%   \partpage
%% \end{frame}
%% \mode<all>{
%%   \input{../methodology/body}
%%   }

\clearpage
\mode<all>{
\part{What is missing?}
}
\begin{frame}
\partpage
\end{frame}

\begin{frame}
\frametitle{Many Specialized Topics}
\begin{itemize}
\item How to design efficient core engine
\item Hybrids with LP/MIP tools
\item Hybrids with SAT
\item Symmetry breaking
\item Use of MDD/BDD to encode sets of solutions
\item High level modelling tools
\item Debugging/visualization
\end{itemize}
\end{frame}

\begin{frame}
\frametitle{Reformulation}
\begin{itemize}
\item Just because the user has modelled it this way, it doesn't mean we have to solve it that way
\begin{itemize}
\item Replace some constraint(s) by other, equivalent constraints
\item Because we don't have that constraint in our system
\item For performance
\end{itemize}
\end{itemize}
\end{frame}

\begin{frame}
\frametitle{Learning}
\begin{itemize}
\item While solving the problem we can learn how to strengthen the model/search 
\begin{itemize}
\item Understand which constraints/method contribute to propagation and change schedule
\item Learn no-good constraints by explaining failure
\item Adapt search strategy based on search experience
\end{itemize}
\end{itemize}
\end{frame}

%% \begin{frame}
%% \frametitle{Refined Process}
%% \begin{center}
%% \begin{tikzpicture}
%% \node[shape=rectangle,fill=pantone127-4] (problem) at (3,7) {Problem};
%% \node (human) at (3,6) {Human};
%% \node[shape=rectangle,fill=pantone127-4] (model) at (3,5) {User Model};
%% \node (reformulation) at (3,4) {Reformulation};
%% \node[shape=rectangle,fill=pantone127-4] (implementation) at (3,3) {Implementation Model};
%% \node (solver) at (3,2) {Constraint Solver/Search};
%% \node[shape=rectangle,fill=pantone157-8] (solution) at (3,1) {Solution};
%% \draw[-] (problem) -- (human);
%% \draw[->] (human) -- (model);
%% \draw[-] (model) -- (reformulation);
%% \draw[->] (reformulation) -- (implementation);
%% \draw[-] (implementation) -- (solver);
%% \draw[->] (solver) -- (0,2) -- (0,3) -- (implementation);
%% \draw[->] (solver) -- (solution);
%% \end{tikzpicture}
%% \end{center}
%% \end{frame}

%\begin{frame}
%\frametitle{What is CP actually used for?}
%\url{http://hsimonis.wordpress.com}
%\includegraphics[width=10cm]{../introduction/images/blog}
%\end{frame}

%% \begin{frame}
%% \frametitle{Exercises}
%% \begin{itemize}
%% \item Install Constraint Solver of your Choice
%% \item MiniZinc (MiningZinc) will be used in later sessions
%% \item Run Sendmoremoney, Sudoku, Queens examples
%% \item Can you choose which propagator to use for alldifferent?
%% \item Can you use different search strategies?
%% \end{itemize}
%% \end{frame}

\begin{frame}
\frametitle{More Learning Resources}
\begin{itemize}
\item Survey of Methods, Resources, and Formats for Teaching Constraint Programming
\begin{itemize}
\item by Tejas Santanam, Helmut Simonis
\item \url{https://doi.org/10.48550/arXiv.2403.12717}
\item Based on survey of community for WTCP 2023
\item \url{https://hsimonis.github.io/WTCP2023/}
\end{itemize}
\end{itemize}
\end{frame}


}

\clearpage
\mode<all>{
\part{Customizing Search}
}
\begin{frame}
\partpage
\end{frame}
\mode<all>{
\begin{frame}
\frametitle{Objectives}
\begin{itemize}
\item Overview of Core Constraint Programming
\item Three Main Concepts
\begin{itemize}
\item Constraint Propagation
\item Global Constraints
\item Customizing Search
\end{itemize}
%\item Get Some Experience with MiniZinc
\item Topics will be treated in more detail in later parts of the school
\item Based on Examples, not Formal Description
\end{itemize}
\end{frame}

\begin{frame}
\frametitle{Outline}
\begin{itemize}
\item Why Constraint Programming?
\item Constraint Propagation
\item Global Constraints
\item Customizing Search
%\item Choices in Modelling
%\item Decomposition
%\item More Global Constraints
%\item Methodology
%\item What is missing?
\end{itemize}
\end{frame}

% \begin{frame}
  % \frametitle{Using MiniZinc IDE}
  % \begin{itemize}
  % \item Developed in the Australian NICTA project
  % \item Maintained by Monash University
  % \item Modelling tool with multiple back-end solvers
  % \item Available from \url{https://www.minizinc.org/}
  % \end{itemize}
% \end{frame}


% \begin{frame}
% \frametitle{Examples in ECLiPSe}
% \begin{itemize}
% \item Open sourced constraint programming language
% \item Development goes back to 1985
% \item ECRC, ICL, IC-Parc, PTL, Cisco
% \item \url{https://eclipseclp.org/}
% \item Specialities
% \begin{itemize}
% \item Develop new solvers for specific domains
% \item Integration with MIP
% \end{itemize}
% \item Not included in bundled MiniZinc IDE
% \item Specialized visualization tools used here
  % \begin{itemize}
  % \item CP-Viz, Simonis et al. 2010
    % \end{itemize}
% \end{itemize}
% \end{frame}

\begin{frame}
\frametitle{Tutorial Based on ECLiPSe ELearning Course}
\begin{itemize}
\item Self-study course in constraint programming
\item Supported by Cisco Systems and Silicon Valley Community Foundation
\item Multi-media format, video lectures, slides, handout etc
\item {\small \url{https://eclipseclp.org/ELearning/index.html}}
\end{itemize}
\end{frame}

\begin{frame}
\frametitle{Also Part of CRT-AI Constraint Week}
\begin{itemize}
\item Annual one week course on CP and Optimization in Ireland
\item Part of national training program for PhD students in AI
\item \url{https://www.crt-ai.ie/}
\end{itemize}
\end{frame}



\begin{frame}
\frametitle{Constraint Programming - in a nutshell}
\begin{itemize}
\item Declarative description of problems with
\begin{itemize}
\item {\em Variables} which range over (finite) sets of values
\item {\em Constraints} over subsets of variables which restrict possible value combinations
\item A {\em solution} is a value assignment which satisfies all constraints
\end{itemize}

\item Constraint propagation/reasoning
\begin{itemize}
\item Removing inconsistent values for variables
\item Detect failure if constraint can not be satisfied
\item Interaction of constraints via shared variables
\item Incomplete
\end{itemize}

\item Search
\begin{itemize}
\item User controlled assignment of values to variables
\item Each step triggers constraint propagation 
\end{itemize}
\item Different domains require/allow different methods
\end{itemize}
\end{frame}

\begin{frame}
  \frametitle{Constraint Programming is Different}
  \begin{itemize}
  \item Declarative Programming
    \begin{itemize}
    \item Concentrate on what you want
      \item Not how to get there
      \item Program != Algorithm
      \item Program = Model
    \end{itemize}
    \item Applied to Combinatorial Problems
      \begin{itemize}
        \item No complete polynomial algorithms known (exist?)
        \item CP less ad-hoc than heuristics
        \item Models can evolve
  \end{itemize}
  \end{itemize}
  \end{frame}
    
\begin{frame}
  \frametitle{A Subtractive Process}
  \begin{textblock}{4}(8,-3)
    \includegraphics[width=4cm]{../introduction/images/stages}
  \end{textblock}
  \vfill
  \begin{quote}
    ``Oh, bosh, as Mr. Ruskin says. Sculpture, per se, is the simplest thing in the world. All you have to do is to take a big chunk of marble and a hammer and chisel, make up your mind what you are about to create and chip off all the marble you don't want.''-Paris Gaulois.
  \end{quote}
  
  {\tiny Source: \url{https://quoteinvestigator.com/2014/06/22/chip-away/}}
\end{frame}

\begin{frame}
\frametitle{Basic Process}
\begin{center}
\begin{tikzpicture}
\node[shape=rectangle,fill=pantone127-4] (problem) at (2,5) {Problem};
\node (human) at (2,4) {Human};
\node[shape=rectangle,fill=pantone127-4] (model) at (2,3) {Model};
\node (solver) at (2,2) {Constraint Solver/Search};
\node[shape=rectangle,fill=pantone157-8] (solution) at (2,1) {Solution};
\draw[-] (problem) -- (human);
\draw[->] (human) -- (model);
\draw[-] (model) -- (solver);
\draw[->] (solver) -- (solution);
\end{tikzpicture}
\end{center}
\end{frame}

\begin{frame}
\frametitle{More Realistic}
\begin{center}
\begin{tikzpicture}[xscale=4.0]
\node[shape=rectangle,fill=pantone127-4] (problem) at (2,5) {Problem};
\node (human) at (2,4) {Human};
\node[shape=rectangle,fill=pantone127-4] (model) at (2,3) {Model};
\node (solver) at (2,2) {Constraint Solver/Search};
\node[shape=rectangle,fill=pantone24-8] (hangs) at (1,1) {Hangs};
\node[shape=rectangle,fill=pantone157-8] (solution) at (2,1) {Solution};
\node[shape=rectangle,fill=pantone24-8] (wrong) at (3,1) {Wrong Solution};
\draw[-] (problem) -- (human);
\draw[->] (human) -- (model);
\draw[-] (model) -- (solver);
\draw[->] (solver) -- (hangs);
\draw[->] (solver) -- (solution);
\draw[->] (solver) -- (wrong);
\draw[->] (hangs) -- (1,4) -- (human);
\draw[->] (wrong) -- (3,4) -- (human);
\end{tikzpicture}
\end{center}
\end{frame}

\begin{frame}
\frametitle{Dual Role of Model}
\begin{itemize}
\item Allows Human to Express Problem
\begin{itemize}
\item Close to Problem Domain
\item Constraints as Abstractions
\end{itemize}
\item Allows Solver to Execute
\begin{itemize}
\item Variables as Communication Mechanism
\item Constraints as Algorithms
\end{itemize}
\end{itemize}
\end{frame}

\begin{frame}
\frametitle{Modelling Frameworks}
\begin{itemize}
\item \href{https://www.minizinc.org/}{MiniZinc} (NICTA, Monash University, Australia)
\item \href{https://github.com/eomahony/Numberjack}{NumberJack} (Insight, Ireland) 
\item \href{https://www-users.york.ac.uk/peter.nightingale/savilerow//}{EssencePrime/SavilleRow} (UK) 
\item \href{https://cpmpy.readthedocs.io/en/latest/}{CPMpy} (KU Leuven)
\item Allow use of multiple back-end solvers
\item Compile model into variants for each solver
\item A priori solver independent model(CP, MIP, SAT)
\end{itemize}
\end{frame} 

\begin{frame}
\frametitle{Framework Process}
\begin{center}
\begin{tikzpicture}[xscale=2]
\node[shape=rectangle,fill=pantone127-4] (problem) at (2.5,5) {Problem};
\node (human) at (2.5,4) {Human};
\node[shape=rectangle,fill=pantone127-4] (model) at (2.5,3) {Model};
\node (compile) at (2.5,2) {Compile/Reformulate};
\node (solver1) at (1,1) {CP};
\node (solver2) at (2,1) {MIP};
\node (solver3) at (3,1) {SAT};
\node (solver4) at (4,1) {Other};
\node[shape=rectangle,fill=pantone157-8] (solution1) at (1,0) {Solution};
\node[shape=rectangle,fill=pantone157-8] (solution2) at (2,0) {Solution};
\node[shape=rectangle,fill=pantone157-8] (solution3) at (3,0) {Solution};
\node[shape=rectangle,fill=pantone157-8] (solution4) at (4,0) {Solution};
\draw[-] (problem) -- (human);
\draw[->] (human) -- (model);
\draw[-] (model) -- (compile);
\draw[-] (compile) -- (solver1);
\draw[-] (compile) -- (solver2);
\draw[-] (compile) -- (solver3);
\draw[-] (compile) -- (solver4);
\draw[->] (solver1) -- (solution1);
\draw[->] (solver2) -- (solution2);
\draw[->] (solver3) -- (solution3);
\draw[->] (solver4) -- (solution4);
\end{tikzpicture}
\end{center}
\end{frame}

% \begin{frame}
  % \frametitle{Do It Now!}
  % \begin{itemize}
  % \item Download and install Minizinc
  % \item \url{https://www.minizinc.org/}
  % \end{itemize}
% \end{frame}


\begin{frame}
\frametitle{Why use Puzzles as Examples?}
\begin{itemize}
\item Easy to understand the problem
\item Solvable by hand without specialized knowledge
\item Possible to compare automated to manual solving process
\end{itemize}

\begin{quote}
The puzzle, though inanimate, is presented as a solvable problem without lasting negative consequences, a very low-risk low-reward situation. By being a puzzle, the object is attempting to convince the user that it must be completed.
\end{quote}
{\tiny Source: \href{https://www.everydayrhetoric.com/2020/03/the-rhetoric-of-puzzles/}{Every Day Rhetoric}}
\end{frame}


\mode<all>{
\part{Basic Constraint Propagation}
}
\begin{frame}
\partpage
\end{frame}
\mode<all>{
\input{../sendmore/body}
}

\mode<all>{
\part{Global Constraints}
}
\begin{frame}
\partpage
\end{frame}
\mode<all>{
\input{../sudoku/body}
}

\clearpage
\mode<all>{
\part{Customizing Search}
}
\begin{frame}
\partpage
\end{frame}
\mode<all>{
\input{../nqueen/body}
}


%% \clearpage
%% \mode<all>{
%%   \part{Problem Decomposition}
%% }
%% \begin{frame}
%%   \partpage
%% \end{frame}
%% \mode<all>{
%%   \input{../party/body}
%%   }

%% \clearpage
%% \mode<all>{
%%   \part{More Global Constraints}
%% }
%% \begin{frame}
%%   \partpage
%% \end{frame}
%% \mode<all>{
%%   \input{../car/body}
%%   }

%% \clearpage
%% \mode<all>{
%%   \part{Methodology}
%% }
%% \begin{frame}
%%   \partpage
%% \end{frame}
%% \mode<all>{
%%   \input{../methodology/body}
%%   }

\clearpage
\mode<all>{
\part{What is missing?}
}
\begin{frame}
\partpage
\end{frame}

\begin{frame}
\frametitle{Many Specialized Topics}
\begin{itemize}
\item How to design efficient core engine
\item Hybrids with LP/MIP tools
\item Hybrids with SAT
\item Symmetry breaking
\item Use of MDD/BDD to encode sets of solutions
\item High level modelling tools
\item Debugging/visualization
\end{itemize}
\end{frame}

\begin{frame}
\frametitle{Reformulation}
\begin{itemize}
\item Just because the user has modelled it this way, it doesn't mean we have to solve it that way
\begin{itemize}
\item Replace some constraint(s) by other, equivalent constraints
\item Because we don't have that constraint in our system
\item For performance
\end{itemize}
\end{itemize}
\end{frame}

\begin{frame}
\frametitle{Learning}
\begin{itemize}
\item While solving the problem we can learn how to strengthen the model/search 
\begin{itemize}
\item Understand which constraints/method contribute to propagation and change schedule
\item Learn no-good constraints by explaining failure
\item Adapt search strategy based on search experience
\end{itemize}
\end{itemize}
\end{frame}

%% \begin{frame}
%% \frametitle{Refined Process}
%% \begin{center}
%% \begin{tikzpicture}
%% \node[shape=rectangle,fill=pantone127-4] (problem) at (3,7) {Problem};
%% \node (human) at (3,6) {Human};
%% \node[shape=rectangle,fill=pantone127-4] (model) at (3,5) {User Model};
%% \node (reformulation) at (3,4) {Reformulation};
%% \node[shape=rectangle,fill=pantone127-4] (implementation) at (3,3) {Implementation Model};
%% \node (solver) at (3,2) {Constraint Solver/Search};
%% \node[shape=rectangle,fill=pantone157-8] (solution) at (3,1) {Solution};
%% \draw[-] (problem) -- (human);
%% \draw[->] (human) -- (model);
%% \draw[-] (model) -- (reformulation);
%% \draw[->] (reformulation) -- (implementation);
%% \draw[-] (implementation) -- (solver);
%% \draw[->] (solver) -- (0,2) -- (0,3) -- (implementation);
%% \draw[->] (solver) -- (solution);
%% \end{tikzpicture}
%% \end{center}
%% \end{frame}

%\begin{frame}
%\frametitle{What is CP actually used for?}
%\url{http://hsimonis.wordpress.com}
%\includegraphics[width=10cm]{../introduction/images/blog}
%\end{frame}

%% \begin{frame}
%% \frametitle{Exercises}
%% \begin{itemize}
%% \item Install Constraint Solver of your Choice
%% \item MiniZinc (MiningZinc) will be used in later sessions
%% \item Run Sendmoremoney, Sudoku, Queens examples
%% \item Can you choose which propagator to use for alldifferent?
%% \item Can you use different search strategies?
%% \end{itemize}
%% \end{frame}

\begin{frame}
\frametitle{More Learning Resources}
\begin{itemize}
\item Survey of Methods, Resources, and Formats for Teaching Constraint Programming
\begin{itemize}
\item by Tejas Santanam, Helmut Simonis
\item \url{https://doi.org/10.48550/arXiv.2403.12717}
\item Based on survey of community for WTCP 2023
\item \url{https://hsimonis.github.io/WTCP2023/}
\end{itemize}
\end{itemize}
\end{frame}


}


%% \clearpage
%% \mode<all>{
%%   \part{Problem Decomposition}
%% }
%% \begin{frame}
%%   \partpage
%% \end{frame}
%% \mode<all>{
%%   \begin{frame}
\frametitle{Objectives}
\begin{itemize}
\item Overview of Core Constraint Programming
\item Three Main Concepts
\begin{itemize}
\item Constraint Propagation
\item Global Constraints
\item Customizing Search
\end{itemize}
%\item Get Some Experience with MiniZinc
\item Topics will be treated in more detail in later parts of the school
\item Based on Examples, not Formal Description
\end{itemize}
\end{frame}

\begin{frame}
\frametitle{Outline}
\begin{itemize}
\item Why Constraint Programming?
\item Constraint Propagation
\item Global Constraints
\item Customizing Search
%\item Choices in Modelling
%\item Decomposition
%\item More Global Constraints
%\item Methodology
%\item What is missing?
\end{itemize}
\end{frame}

% \begin{frame}
  % \frametitle{Using MiniZinc IDE}
  % \begin{itemize}
  % \item Developed in the Australian NICTA project
  % \item Maintained by Monash University
  % \item Modelling tool with multiple back-end solvers
  % \item Available from \url{https://www.minizinc.org/}
  % \end{itemize}
% \end{frame}


% \begin{frame}
% \frametitle{Examples in ECLiPSe}
% \begin{itemize}
% \item Open sourced constraint programming language
% \item Development goes back to 1985
% \item ECRC, ICL, IC-Parc, PTL, Cisco
% \item \url{https://eclipseclp.org/}
% \item Specialities
% \begin{itemize}
% \item Develop new solvers for specific domains
% \item Integration with MIP
% \end{itemize}
% \item Not included in bundled MiniZinc IDE
% \item Specialized visualization tools used here
  % \begin{itemize}
  % \item CP-Viz, Simonis et al. 2010
    % \end{itemize}
% \end{itemize}
% \end{frame}

\begin{frame}
\frametitle{Tutorial Based on ECLiPSe ELearning Course}
\begin{itemize}
\item Self-study course in constraint programming
\item Supported by Cisco Systems and Silicon Valley Community Foundation
\item Multi-media format, video lectures, slides, handout etc
\item {\small \url{https://eclipseclp.org/ELearning/index.html}}
\end{itemize}
\end{frame}

\begin{frame}
\frametitle{Also Part of CRT-AI Constraint Week}
\begin{itemize}
\item Annual one week course on CP and Optimization in Ireland
\item Part of national training program for PhD students in AI
\item \url{https://www.crt-ai.ie/}
\end{itemize}
\end{frame}



\begin{frame}
\frametitle{Constraint Programming - in a nutshell}
\begin{itemize}
\item Declarative description of problems with
\begin{itemize}
\item {\em Variables} which range over (finite) sets of values
\item {\em Constraints} over subsets of variables which restrict possible value combinations
\item A {\em solution} is a value assignment which satisfies all constraints
\end{itemize}

\item Constraint propagation/reasoning
\begin{itemize}
\item Removing inconsistent values for variables
\item Detect failure if constraint can not be satisfied
\item Interaction of constraints via shared variables
\item Incomplete
\end{itemize}

\item Search
\begin{itemize}
\item User controlled assignment of values to variables
\item Each step triggers constraint propagation 
\end{itemize}
\item Different domains require/allow different methods
\end{itemize}
\end{frame}

\begin{frame}
  \frametitle{Constraint Programming is Different}
  \begin{itemize}
  \item Declarative Programming
    \begin{itemize}
    \item Concentrate on what you want
      \item Not how to get there
      \item Program != Algorithm
      \item Program = Model
    \end{itemize}
    \item Applied to Combinatorial Problems
      \begin{itemize}
        \item No complete polynomial algorithms known (exist?)
        \item CP less ad-hoc than heuristics
        \item Models can evolve
  \end{itemize}
  \end{itemize}
  \end{frame}
    
\begin{frame}
  \frametitle{A Subtractive Process}
  \begin{textblock}{4}(8,-3)
    \includegraphics[width=4cm]{../introduction/images/stages}
  \end{textblock}
  \vfill
  \begin{quote}
    ``Oh, bosh, as Mr. Ruskin says. Sculpture, per se, is the simplest thing in the world. All you have to do is to take a big chunk of marble and a hammer and chisel, make up your mind what you are about to create and chip off all the marble you don't want.''-Paris Gaulois.
  \end{quote}
  
  {\tiny Source: \url{https://quoteinvestigator.com/2014/06/22/chip-away/}}
\end{frame}

\begin{frame}
\frametitle{Basic Process}
\begin{center}
\begin{tikzpicture}
\node[shape=rectangle,fill=pantone127-4] (problem) at (2,5) {Problem};
\node (human) at (2,4) {Human};
\node[shape=rectangle,fill=pantone127-4] (model) at (2,3) {Model};
\node (solver) at (2,2) {Constraint Solver/Search};
\node[shape=rectangle,fill=pantone157-8] (solution) at (2,1) {Solution};
\draw[-] (problem) -- (human);
\draw[->] (human) -- (model);
\draw[-] (model) -- (solver);
\draw[->] (solver) -- (solution);
\end{tikzpicture}
\end{center}
\end{frame}

\begin{frame}
\frametitle{More Realistic}
\begin{center}
\begin{tikzpicture}[xscale=4.0]
\node[shape=rectangle,fill=pantone127-4] (problem) at (2,5) {Problem};
\node (human) at (2,4) {Human};
\node[shape=rectangle,fill=pantone127-4] (model) at (2,3) {Model};
\node (solver) at (2,2) {Constraint Solver/Search};
\node[shape=rectangle,fill=pantone24-8] (hangs) at (1,1) {Hangs};
\node[shape=rectangle,fill=pantone157-8] (solution) at (2,1) {Solution};
\node[shape=rectangle,fill=pantone24-8] (wrong) at (3,1) {Wrong Solution};
\draw[-] (problem) -- (human);
\draw[->] (human) -- (model);
\draw[-] (model) -- (solver);
\draw[->] (solver) -- (hangs);
\draw[->] (solver) -- (solution);
\draw[->] (solver) -- (wrong);
\draw[->] (hangs) -- (1,4) -- (human);
\draw[->] (wrong) -- (3,4) -- (human);
\end{tikzpicture}
\end{center}
\end{frame}

\begin{frame}
\frametitle{Dual Role of Model}
\begin{itemize}
\item Allows Human to Express Problem
\begin{itemize}
\item Close to Problem Domain
\item Constraints as Abstractions
\end{itemize}
\item Allows Solver to Execute
\begin{itemize}
\item Variables as Communication Mechanism
\item Constraints as Algorithms
\end{itemize}
\end{itemize}
\end{frame}

\begin{frame}
\frametitle{Modelling Frameworks}
\begin{itemize}
\item \href{https://www.minizinc.org/}{MiniZinc} (NICTA, Monash University, Australia)
\item \href{https://github.com/eomahony/Numberjack}{NumberJack} (Insight, Ireland) 
\item \href{https://www-users.york.ac.uk/peter.nightingale/savilerow//}{EssencePrime/SavilleRow} (UK) 
\item \href{https://cpmpy.readthedocs.io/en/latest/}{CPMpy} (KU Leuven)
\item Allow use of multiple back-end solvers
\item Compile model into variants for each solver
\item A priori solver independent model(CP, MIP, SAT)
\end{itemize}
\end{frame} 

\begin{frame}
\frametitle{Framework Process}
\begin{center}
\begin{tikzpicture}[xscale=2]
\node[shape=rectangle,fill=pantone127-4] (problem) at (2.5,5) {Problem};
\node (human) at (2.5,4) {Human};
\node[shape=rectangle,fill=pantone127-4] (model) at (2.5,3) {Model};
\node (compile) at (2.5,2) {Compile/Reformulate};
\node (solver1) at (1,1) {CP};
\node (solver2) at (2,1) {MIP};
\node (solver3) at (3,1) {SAT};
\node (solver4) at (4,1) {Other};
\node[shape=rectangle,fill=pantone157-8] (solution1) at (1,0) {Solution};
\node[shape=rectangle,fill=pantone157-8] (solution2) at (2,0) {Solution};
\node[shape=rectangle,fill=pantone157-8] (solution3) at (3,0) {Solution};
\node[shape=rectangle,fill=pantone157-8] (solution4) at (4,0) {Solution};
\draw[-] (problem) -- (human);
\draw[->] (human) -- (model);
\draw[-] (model) -- (compile);
\draw[-] (compile) -- (solver1);
\draw[-] (compile) -- (solver2);
\draw[-] (compile) -- (solver3);
\draw[-] (compile) -- (solver4);
\draw[->] (solver1) -- (solution1);
\draw[->] (solver2) -- (solution2);
\draw[->] (solver3) -- (solution3);
\draw[->] (solver4) -- (solution4);
\end{tikzpicture}
\end{center}
\end{frame}

% \begin{frame}
  % \frametitle{Do It Now!}
  % \begin{itemize}
  % \item Download and install Minizinc
  % \item \url{https://www.minizinc.org/}
  % \end{itemize}
% \end{frame}


\begin{frame}
\frametitle{Why use Puzzles as Examples?}
\begin{itemize}
\item Easy to understand the problem
\item Solvable by hand without specialized knowledge
\item Possible to compare automated to manual solving process
\end{itemize}

\begin{quote}
The puzzle, though inanimate, is presented as a solvable problem without lasting negative consequences, a very low-risk low-reward situation. By being a puzzle, the object is attempting to convince the user that it must be completed.
\end{quote}
{\tiny Source: \href{https://www.everydayrhetoric.com/2020/03/the-rhetoric-of-puzzles/}{Every Day Rhetoric}}
\end{frame}


\mode<all>{
\part{Basic Constraint Propagation}
}
\begin{frame}
\partpage
\end{frame}
\mode<all>{
\input{../sendmore/body}
}

\mode<all>{
\part{Global Constraints}
}
\begin{frame}
\partpage
\end{frame}
\mode<all>{
\input{../sudoku/body}
}

\clearpage
\mode<all>{
\part{Customizing Search}
}
\begin{frame}
\partpage
\end{frame}
\mode<all>{
\input{../nqueen/body}
}


%% \clearpage
%% \mode<all>{
%%   \part{Problem Decomposition}
%% }
%% \begin{frame}
%%   \partpage
%% \end{frame}
%% \mode<all>{
%%   \input{../party/body}
%%   }

%% \clearpage
%% \mode<all>{
%%   \part{More Global Constraints}
%% }
%% \begin{frame}
%%   \partpage
%% \end{frame}
%% \mode<all>{
%%   \input{../car/body}
%%   }

%% \clearpage
%% \mode<all>{
%%   \part{Methodology}
%% }
%% \begin{frame}
%%   \partpage
%% \end{frame}
%% \mode<all>{
%%   \input{../methodology/body}
%%   }

\clearpage
\mode<all>{
\part{What is missing?}
}
\begin{frame}
\partpage
\end{frame}

\begin{frame}
\frametitle{Many Specialized Topics}
\begin{itemize}
\item How to design efficient core engine
\item Hybrids with LP/MIP tools
\item Hybrids with SAT
\item Symmetry breaking
\item Use of MDD/BDD to encode sets of solutions
\item High level modelling tools
\item Debugging/visualization
\end{itemize}
\end{frame}

\begin{frame}
\frametitle{Reformulation}
\begin{itemize}
\item Just because the user has modelled it this way, it doesn't mean we have to solve it that way
\begin{itemize}
\item Replace some constraint(s) by other, equivalent constraints
\item Because we don't have that constraint in our system
\item For performance
\end{itemize}
\end{itemize}
\end{frame}

\begin{frame}
\frametitle{Learning}
\begin{itemize}
\item While solving the problem we can learn how to strengthen the model/search 
\begin{itemize}
\item Understand which constraints/method contribute to propagation and change schedule
\item Learn no-good constraints by explaining failure
\item Adapt search strategy based on search experience
\end{itemize}
\end{itemize}
\end{frame}

%% \begin{frame}
%% \frametitle{Refined Process}
%% \begin{center}
%% \begin{tikzpicture}
%% \node[shape=rectangle,fill=pantone127-4] (problem) at (3,7) {Problem};
%% \node (human) at (3,6) {Human};
%% \node[shape=rectangle,fill=pantone127-4] (model) at (3,5) {User Model};
%% \node (reformulation) at (3,4) {Reformulation};
%% \node[shape=rectangle,fill=pantone127-4] (implementation) at (3,3) {Implementation Model};
%% \node (solver) at (3,2) {Constraint Solver/Search};
%% \node[shape=rectangle,fill=pantone157-8] (solution) at (3,1) {Solution};
%% \draw[-] (problem) -- (human);
%% \draw[->] (human) -- (model);
%% \draw[-] (model) -- (reformulation);
%% \draw[->] (reformulation) -- (implementation);
%% \draw[-] (implementation) -- (solver);
%% \draw[->] (solver) -- (0,2) -- (0,3) -- (implementation);
%% \draw[->] (solver) -- (solution);
%% \end{tikzpicture}
%% \end{center}
%% \end{frame}

%\begin{frame}
%\frametitle{What is CP actually used for?}
%\url{http://hsimonis.wordpress.com}
%\includegraphics[width=10cm]{../introduction/images/blog}
%\end{frame}

%% \begin{frame}
%% \frametitle{Exercises}
%% \begin{itemize}
%% \item Install Constraint Solver of your Choice
%% \item MiniZinc (MiningZinc) will be used in later sessions
%% \item Run Sendmoremoney, Sudoku, Queens examples
%% \item Can you choose which propagator to use for alldifferent?
%% \item Can you use different search strategies?
%% \end{itemize}
%% \end{frame}

\begin{frame}
\frametitle{More Learning Resources}
\begin{itemize}
\item Survey of Methods, Resources, and Formats for Teaching Constraint Programming
\begin{itemize}
\item by Tejas Santanam, Helmut Simonis
\item \url{https://doi.org/10.48550/arXiv.2403.12717}
\item Based on survey of community for WTCP 2023
\item \url{https://hsimonis.github.io/WTCP2023/}
\end{itemize}
\end{itemize}
\end{frame}


%%   }

%% \clearpage
%% \mode<all>{
%%   \part{More Global Constraints}
%% }
%% \begin{frame}
%%   \partpage
%% \end{frame}
%% \mode<all>{
%%   \begin{frame}
\frametitle{Objectives}
\begin{itemize}
\item Overview of Core Constraint Programming
\item Three Main Concepts
\begin{itemize}
\item Constraint Propagation
\item Global Constraints
\item Customizing Search
\end{itemize}
%\item Get Some Experience with MiniZinc
\item Topics will be treated in more detail in later parts of the school
\item Based on Examples, not Formal Description
\end{itemize}
\end{frame}

\begin{frame}
\frametitle{Outline}
\begin{itemize}
\item Why Constraint Programming?
\item Constraint Propagation
\item Global Constraints
\item Customizing Search
%\item Choices in Modelling
%\item Decomposition
%\item More Global Constraints
%\item Methodology
%\item What is missing?
\end{itemize}
\end{frame}

% \begin{frame}
  % \frametitle{Using MiniZinc IDE}
  % \begin{itemize}
  % \item Developed in the Australian NICTA project
  % \item Maintained by Monash University
  % \item Modelling tool with multiple back-end solvers
  % \item Available from \url{https://www.minizinc.org/}
  % \end{itemize}
% \end{frame}


% \begin{frame}
% \frametitle{Examples in ECLiPSe}
% \begin{itemize}
% \item Open sourced constraint programming language
% \item Development goes back to 1985
% \item ECRC, ICL, IC-Parc, PTL, Cisco
% \item \url{https://eclipseclp.org/}
% \item Specialities
% \begin{itemize}
% \item Develop new solvers for specific domains
% \item Integration with MIP
% \end{itemize}
% \item Not included in bundled MiniZinc IDE
% \item Specialized visualization tools used here
  % \begin{itemize}
  % \item CP-Viz, Simonis et al. 2010
    % \end{itemize}
% \end{itemize}
% \end{frame}

\begin{frame}
\frametitle{Tutorial Based on ECLiPSe ELearning Course}
\begin{itemize}
\item Self-study course in constraint programming
\item Supported by Cisco Systems and Silicon Valley Community Foundation
\item Multi-media format, video lectures, slides, handout etc
\item {\small \url{https://eclipseclp.org/ELearning/index.html}}
\end{itemize}
\end{frame}

\begin{frame}
\frametitle{Also Part of CRT-AI Constraint Week}
\begin{itemize}
\item Annual one week course on CP and Optimization in Ireland
\item Part of national training program for PhD students in AI
\item \url{https://www.crt-ai.ie/}
\end{itemize}
\end{frame}



\begin{frame}
\frametitle{Constraint Programming - in a nutshell}
\begin{itemize}
\item Declarative description of problems with
\begin{itemize}
\item {\em Variables} which range over (finite) sets of values
\item {\em Constraints} over subsets of variables which restrict possible value combinations
\item A {\em solution} is a value assignment which satisfies all constraints
\end{itemize}

\item Constraint propagation/reasoning
\begin{itemize}
\item Removing inconsistent values for variables
\item Detect failure if constraint can not be satisfied
\item Interaction of constraints via shared variables
\item Incomplete
\end{itemize}

\item Search
\begin{itemize}
\item User controlled assignment of values to variables
\item Each step triggers constraint propagation 
\end{itemize}
\item Different domains require/allow different methods
\end{itemize}
\end{frame}

\begin{frame}
  \frametitle{Constraint Programming is Different}
  \begin{itemize}
  \item Declarative Programming
    \begin{itemize}
    \item Concentrate on what you want
      \item Not how to get there
      \item Program != Algorithm
      \item Program = Model
    \end{itemize}
    \item Applied to Combinatorial Problems
      \begin{itemize}
        \item No complete polynomial algorithms known (exist?)
        \item CP less ad-hoc than heuristics
        \item Models can evolve
  \end{itemize}
  \end{itemize}
  \end{frame}
    
\begin{frame}
  \frametitle{A Subtractive Process}
  \begin{textblock}{4}(8,-3)
    \includegraphics[width=4cm]{../introduction/images/stages}
  \end{textblock}
  \vfill
  \begin{quote}
    ``Oh, bosh, as Mr. Ruskin says. Sculpture, per se, is the simplest thing in the world. All you have to do is to take a big chunk of marble and a hammer and chisel, make up your mind what you are about to create and chip off all the marble you don't want.''-Paris Gaulois.
  \end{quote}
  
  {\tiny Source: \url{https://quoteinvestigator.com/2014/06/22/chip-away/}}
\end{frame}

\begin{frame}
\frametitle{Basic Process}
\begin{center}
\begin{tikzpicture}
\node[shape=rectangle,fill=pantone127-4] (problem) at (2,5) {Problem};
\node (human) at (2,4) {Human};
\node[shape=rectangle,fill=pantone127-4] (model) at (2,3) {Model};
\node (solver) at (2,2) {Constraint Solver/Search};
\node[shape=rectangle,fill=pantone157-8] (solution) at (2,1) {Solution};
\draw[-] (problem) -- (human);
\draw[->] (human) -- (model);
\draw[-] (model) -- (solver);
\draw[->] (solver) -- (solution);
\end{tikzpicture}
\end{center}
\end{frame}

\begin{frame}
\frametitle{More Realistic}
\begin{center}
\begin{tikzpicture}[xscale=4.0]
\node[shape=rectangle,fill=pantone127-4] (problem) at (2,5) {Problem};
\node (human) at (2,4) {Human};
\node[shape=rectangle,fill=pantone127-4] (model) at (2,3) {Model};
\node (solver) at (2,2) {Constraint Solver/Search};
\node[shape=rectangle,fill=pantone24-8] (hangs) at (1,1) {Hangs};
\node[shape=rectangle,fill=pantone157-8] (solution) at (2,1) {Solution};
\node[shape=rectangle,fill=pantone24-8] (wrong) at (3,1) {Wrong Solution};
\draw[-] (problem) -- (human);
\draw[->] (human) -- (model);
\draw[-] (model) -- (solver);
\draw[->] (solver) -- (hangs);
\draw[->] (solver) -- (solution);
\draw[->] (solver) -- (wrong);
\draw[->] (hangs) -- (1,4) -- (human);
\draw[->] (wrong) -- (3,4) -- (human);
\end{tikzpicture}
\end{center}
\end{frame}

\begin{frame}
\frametitle{Dual Role of Model}
\begin{itemize}
\item Allows Human to Express Problem
\begin{itemize}
\item Close to Problem Domain
\item Constraints as Abstractions
\end{itemize}
\item Allows Solver to Execute
\begin{itemize}
\item Variables as Communication Mechanism
\item Constraints as Algorithms
\end{itemize}
\end{itemize}
\end{frame}

\begin{frame}
\frametitle{Modelling Frameworks}
\begin{itemize}
\item \href{https://www.minizinc.org/}{MiniZinc} (NICTA, Monash University, Australia)
\item \href{https://github.com/eomahony/Numberjack}{NumberJack} (Insight, Ireland) 
\item \href{https://www-users.york.ac.uk/peter.nightingale/savilerow//}{EssencePrime/SavilleRow} (UK) 
\item \href{https://cpmpy.readthedocs.io/en/latest/}{CPMpy} (KU Leuven)
\item Allow use of multiple back-end solvers
\item Compile model into variants for each solver
\item A priori solver independent model(CP, MIP, SAT)
\end{itemize}
\end{frame} 

\begin{frame}
\frametitle{Framework Process}
\begin{center}
\begin{tikzpicture}[xscale=2]
\node[shape=rectangle,fill=pantone127-4] (problem) at (2.5,5) {Problem};
\node (human) at (2.5,4) {Human};
\node[shape=rectangle,fill=pantone127-4] (model) at (2.5,3) {Model};
\node (compile) at (2.5,2) {Compile/Reformulate};
\node (solver1) at (1,1) {CP};
\node (solver2) at (2,1) {MIP};
\node (solver3) at (3,1) {SAT};
\node (solver4) at (4,1) {Other};
\node[shape=rectangle,fill=pantone157-8] (solution1) at (1,0) {Solution};
\node[shape=rectangle,fill=pantone157-8] (solution2) at (2,0) {Solution};
\node[shape=rectangle,fill=pantone157-8] (solution3) at (3,0) {Solution};
\node[shape=rectangle,fill=pantone157-8] (solution4) at (4,0) {Solution};
\draw[-] (problem) -- (human);
\draw[->] (human) -- (model);
\draw[-] (model) -- (compile);
\draw[-] (compile) -- (solver1);
\draw[-] (compile) -- (solver2);
\draw[-] (compile) -- (solver3);
\draw[-] (compile) -- (solver4);
\draw[->] (solver1) -- (solution1);
\draw[->] (solver2) -- (solution2);
\draw[->] (solver3) -- (solution3);
\draw[->] (solver4) -- (solution4);
\end{tikzpicture}
\end{center}
\end{frame}

% \begin{frame}
  % \frametitle{Do It Now!}
  % \begin{itemize}
  % \item Download and install Minizinc
  % \item \url{https://www.minizinc.org/}
  % \end{itemize}
% \end{frame}


\begin{frame}
\frametitle{Why use Puzzles as Examples?}
\begin{itemize}
\item Easy to understand the problem
\item Solvable by hand without specialized knowledge
\item Possible to compare automated to manual solving process
\end{itemize}

\begin{quote}
The puzzle, though inanimate, is presented as a solvable problem without lasting negative consequences, a very low-risk low-reward situation. By being a puzzle, the object is attempting to convince the user that it must be completed.
\end{quote}
{\tiny Source: \href{https://www.everydayrhetoric.com/2020/03/the-rhetoric-of-puzzles/}{Every Day Rhetoric}}
\end{frame}


\mode<all>{
\part{Basic Constraint Propagation}
}
\begin{frame}
\partpage
\end{frame}
\mode<all>{
\input{../sendmore/body}
}

\mode<all>{
\part{Global Constraints}
}
\begin{frame}
\partpage
\end{frame}
\mode<all>{
\input{../sudoku/body}
}

\clearpage
\mode<all>{
\part{Customizing Search}
}
\begin{frame}
\partpage
\end{frame}
\mode<all>{
\input{../nqueen/body}
}


%% \clearpage
%% \mode<all>{
%%   \part{Problem Decomposition}
%% }
%% \begin{frame}
%%   \partpage
%% \end{frame}
%% \mode<all>{
%%   \input{../party/body}
%%   }

%% \clearpage
%% \mode<all>{
%%   \part{More Global Constraints}
%% }
%% \begin{frame}
%%   \partpage
%% \end{frame}
%% \mode<all>{
%%   \input{../car/body}
%%   }

%% \clearpage
%% \mode<all>{
%%   \part{Methodology}
%% }
%% \begin{frame}
%%   \partpage
%% \end{frame}
%% \mode<all>{
%%   \input{../methodology/body}
%%   }

\clearpage
\mode<all>{
\part{What is missing?}
}
\begin{frame}
\partpage
\end{frame}

\begin{frame}
\frametitle{Many Specialized Topics}
\begin{itemize}
\item How to design efficient core engine
\item Hybrids with LP/MIP tools
\item Hybrids with SAT
\item Symmetry breaking
\item Use of MDD/BDD to encode sets of solutions
\item High level modelling tools
\item Debugging/visualization
\end{itemize}
\end{frame}

\begin{frame}
\frametitle{Reformulation}
\begin{itemize}
\item Just because the user has modelled it this way, it doesn't mean we have to solve it that way
\begin{itemize}
\item Replace some constraint(s) by other, equivalent constraints
\item Because we don't have that constraint in our system
\item For performance
\end{itemize}
\end{itemize}
\end{frame}

\begin{frame}
\frametitle{Learning}
\begin{itemize}
\item While solving the problem we can learn how to strengthen the model/search 
\begin{itemize}
\item Understand which constraints/method contribute to propagation and change schedule
\item Learn no-good constraints by explaining failure
\item Adapt search strategy based on search experience
\end{itemize}
\end{itemize}
\end{frame}

%% \begin{frame}
%% \frametitle{Refined Process}
%% \begin{center}
%% \begin{tikzpicture}
%% \node[shape=rectangle,fill=pantone127-4] (problem) at (3,7) {Problem};
%% \node (human) at (3,6) {Human};
%% \node[shape=rectangle,fill=pantone127-4] (model) at (3,5) {User Model};
%% \node (reformulation) at (3,4) {Reformulation};
%% \node[shape=rectangle,fill=pantone127-4] (implementation) at (3,3) {Implementation Model};
%% \node (solver) at (3,2) {Constraint Solver/Search};
%% \node[shape=rectangle,fill=pantone157-8] (solution) at (3,1) {Solution};
%% \draw[-] (problem) -- (human);
%% \draw[->] (human) -- (model);
%% \draw[-] (model) -- (reformulation);
%% \draw[->] (reformulation) -- (implementation);
%% \draw[-] (implementation) -- (solver);
%% \draw[->] (solver) -- (0,2) -- (0,3) -- (implementation);
%% \draw[->] (solver) -- (solution);
%% \end{tikzpicture}
%% \end{center}
%% \end{frame}

%\begin{frame}
%\frametitle{What is CP actually used for?}
%\url{http://hsimonis.wordpress.com}
%\includegraphics[width=10cm]{../introduction/images/blog}
%\end{frame}

%% \begin{frame}
%% \frametitle{Exercises}
%% \begin{itemize}
%% \item Install Constraint Solver of your Choice
%% \item MiniZinc (MiningZinc) will be used in later sessions
%% \item Run Sendmoremoney, Sudoku, Queens examples
%% \item Can you choose which propagator to use for alldifferent?
%% \item Can you use different search strategies?
%% \end{itemize}
%% \end{frame}

\begin{frame}
\frametitle{More Learning Resources}
\begin{itemize}
\item Survey of Methods, Resources, and Formats for Teaching Constraint Programming
\begin{itemize}
\item by Tejas Santanam, Helmut Simonis
\item \url{https://doi.org/10.48550/arXiv.2403.12717}
\item Based on survey of community for WTCP 2023
\item \url{https://hsimonis.github.io/WTCP2023/}
\end{itemize}
\end{itemize}
\end{frame}


%%   }

%% \clearpage
%% \mode<all>{
%%   \part{Methodology}
%% }
%% \begin{frame}
%%   \partpage
%% \end{frame}
%% \mode<all>{
%%   \begin{frame}
\frametitle{Objectives}
\begin{itemize}
\item Overview of Core Constraint Programming
\item Three Main Concepts
\begin{itemize}
\item Constraint Propagation
\item Global Constraints
\item Customizing Search
\end{itemize}
%\item Get Some Experience with MiniZinc
\item Topics will be treated in more detail in later parts of the school
\item Based on Examples, not Formal Description
\end{itemize}
\end{frame}

\begin{frame}
\frametitle{Outline}
\begin{itemize}
\item Why Constraint Programming?
\item Constraint Propagation
\item Global Constraints
\item Customizing Search
%\item Choices in Modelling
%\item Decomposition
%\item More Global Constraints
%\item Methodology
%\item What is missing?
\end{itemize}
\end{frame}

% \begin{frame}
  % \frametitle{Using MiniZinc IDE}
  % \begin{itemize}
  % \item Developed in the Australian NICTA project
  % \item Maintained by Monash University
  % \item Modelling tool with multiple back-end solvers
  % \item Available from \url{https://www.minizinc.org/}
  % \end{itemize}
% \end{frame}


% \begin{frame}
% \frametitle{Examples in ECLiPSe}
% \begin{itemize}
% \item Open sourced constraint programming language
% \item Development goes back to 1985
% \item ECRC, ICL, IC-Parc, PTL, Cisco
% \item \url{https://eclipseclp.org/}
% \item Specialities
% \begin{itemize}
% \item Develop new solvers for specific domains
% \item Integration with MIP
% \end{itemize}
% \item Not included in bundled MiniZinc IDE
% \item Specialized visualization tools used here
  % \begin{itemize}
  % \item CP-Viz, Simonis et al. 2010
    % \end{itemize}
% \end{itemize}
% \end{frame}

\begin{frame}
\frametitle{Tutorial Based on ECLiPSe ELearning Course}
\begin{itemize}
\item Self-study course in constraint programming
\item Supported by Cisco Systems and Silicon Valley Community Foundation
\item Multi-media format, video lectures, slides, handout etc
\item {\small \url{https://eclipseclp.org/ELearning/index.html}}
\end{itemize}
\end{frame}

\begin{frame}
\frametitle{Also Part of CRT-AI Constraint Week}
\begin{itemize}
\item Annual one week course on CP and Optimization in Ireland
\item Part of national training program for PhD students in AI
\item \url{https://www.crt-ai.ie/}
\end{itemize}
\end{frame}



\begin{frame}
\frametitle{Constraint Programming - in a nutshell}
\begin{itemize}
\item Declarative description of problems with
\begin{itemize}
\item {\em Variables} which range over (finite) sets of values
\item {\em Constraints} over subsets of variables which restrict possible value combinations
\item A {\em solution} is a value assignment which satisfies all constraints
\end{itemize}

\item Constraint propagation/reasoning
\begin{itemize}
\item Removing inconsistent values for variables
\item Detect failure if constraint can not be satisfied
\item Interaction of constraints via shared variables
\item Incomplete
\end{itemize}

\item Search
\begin{itemize}
\item User controlled assignment of values to variables
\item Each step triggers constraint propagation 
\end{itemize}
\item Different domains require/allow different methods
\end{itemize}
\end{frame}

\begin{frame}
  \frametitle{Constraint Programming is Different}
  \begin{itemize}
  \item Declarative Programming
    \begin{itemize}
    \item Concentrate on what you want
      \item Not how to get there
      \item Program != Algorithm
      \item Program = Model
    \end{itemize}
    \item Applied to Combinatorial Problems
      \begin{itemize}
        \item No complete polynomial algorithms known (exist?)
        \item CP less ad-hoc than heuristics
        \item Models can evolve
  \end{itemize}
  \end{itemize}
  \end{frame}
    
\begin{frame}
  \frametitle{A Subtractive Process}
  \begin{textblock}{4}(8,-3)
    \includegraphics[width=4cm]{../introduction/images/stages}
  \end{textblock}
  \vfill
  \begin{quote}
    ``Oh, bosh, as Mr. Ruskin says. Sculpture, per se, is the simplest thing in the world. All you have to do is to take a big chunk of marble and a hammer and chisel, make up your mind what you are about to create and chip off all the marble you don't want.''-Paris Gaulois.
  \end{quote}
  
  {\tiny Source: \url{https://quoteinvestigator.com/2014/06/22/chip-away/}}
\end{frame}

\begin{frame}
\frametitle{Basic Process}
\begin{center}
\begin{tikzpicture}
\node[shape=rectangle,fill=pantone127-4] (problem) at (2,5) {Problem};
\node (human) at (2,4) {Human};
\node[shape=rectangle,fill=pantone127-4] (model) at (2,3) {Model};
\node (solver) at (2,2) {Constraint Solver/Search};
\node[shape=rectangle,fill=pantone157-8] (solution) at (2,1) {Solution};
\draw[-] (problem) -- (human);
\draw[->] (human) -- (model);
\draw[-] (model) -- (solver);
\draw[->] (solver) -- (solution);
\end{tikzpicture}
\end{center}
\end{frame}

\begin{frame}
\frametitle{More Realistic}
\begin{center}
\begin{tikzpicture}[xscale=4.0]
\node[shape=rectangle,fill=pantone127-4] (problem) at (2,5) {Problem};
\node (human) at (2,4) {Human};
\node[shape=rectangle,fill=pantone127-4] (model) at (2,3) {Model};
\node (solver) at (2,2) {Constraint Solver/Search};
\node[shape=rectangle,fill=pantone24-8] (hangs) at (1,1) {Hangs};
\node[shape=rectangle,fill=pantone157-8] (solution) at (2,1) {Solution};
\node[shape=rectangle,fill=pantone24-8] (wrong) at (3,1) {Wrong Solution};
\draw[-] (problem) -- (human);
\draw[->] (human) -- (model);
\draw[-] (model) -- (solver);
\draw[->] (solver) -- (hangs);
\draw[->] (solver) -- (solution);
\draw[->] (solver) -- (wrong);
\draw[->] (hangs) -- (1,4) -- (human);
\draw[->] (wrong) -- (3,4) -- (human);
\end{tikzpicture}
\end{center}
\end{frame}

\begin{frame}
\frametitle{Dual Role of Model}
\begin{itemize}
\item Allows Human to Express Problem
\begin{itemize}
\item Close to Problem Domain
\item Constraints as Abstractions
\end{itemize}
\item Allows Solver to Execute
\begin{itemize}
\item Variables as Communication Mechanism
\item Constraints as Algorithms
\end{itemize}
\end{itemize}
\end{frame}

\begin{frame}
\frametitle{Modelling Frameworks}
\begin{itemize}
\item \href{https://www.minizinc.org/}{MiniZinc} (NICTA, Monash University, Australia)
\item \href{https://github.com/eomahony/Numberjack}{NumberJack} (Insight, Ireland) 
\item \href{https://www-users.york.ac.uk/peter.nightingale/savilerow//}{EssencePrime/SavilleRow} (UK) 
\item \href{https://cpmpy.readthedocs.io/en/latest/}{CPMpy} (KU Leuven)
\item Allow use of multiple back-end solvers
\item Compile model into variants for each solver
\item A priori solver independent model(CP, MIP, SAT)
\end{itemize}
\end{frame} 

\begin{frame}
\frametitle{Framework Process}
\begin{center}
\begin{tikzpicture}[xscale=2]
\node[shape=rectangle,fill=pantone127-4] (problem) at (2.5,5) {Problem};
\node (human) at (2.5,4) {Human};
\node[shape=rectangle,fill=pantone127-4] (model) at (2.5,3) {Model};
\node (compile) at (2.5,2) {Compile/Reformulate};
\node (solver1) at (1,1) {CP};
\node (solver2) at (2,1) {MIP};
\node (solver3) at (3,1) {SAT};
\node (solver4) at (4,1) {Other};
\node[shape=rectangle,fill=pantone157-8] (solution1) at (1,0) {Solution};
\node[shape=rectangle,fill=pantone157-8] (solution2) at (2,0) {Solution};
\node[shape=rectangle,fill=pantone157-8] (solution3) at (3,0) {Solution};
\node[shape=rectangle,fill=pantone157-8] (solution4) at (4,0) {Solution};
\draw[-] (problem) -- (human);
\draw[->] (human) -- (model);
\draw[-] (model) -- (compile);
\draw[-] (compile) -- (solver1);
\draw[-] (compile) -- (solver2);
\draw[-] (compile) -- (solver3);
\draw[-] (compile) -- (solver4);
\draw[->] (solver1) -- (solution1);
\draw[->] (solver2) -- (solution2);
\draw[->] (solver3) -- (solution3);
\draw[->] (solver4) -- (solution4);
\end{tikzpicture}
\end{center}
\end{frame}

% \begin{frame}
  % \frametitle{Do It Now!}
  % \begin{itemize}
  % \item Download and install Minizinc
  % \item \url{https://www.minizinc.org/}
  % \end{itemize}
% \end{frame}


\begin{frame}
\frametitle{Why use Puzzles as Examples?}
\begin{itemize}
\item Easy to understand the problem
\item Solvable by hand without specialized knowledge
\item Possible to compare automated to manual solving process
\end{itemize}

\begin{quote}
The puzzle, though inanimate, is presented as a solvable problem without lasting negative consequences, a very low-risk low-reward situation. By being a puzzle, the object is attempting to convince the user that it must be completed.
\end{quote}
{\tiny Source: \href{https://www.everydayrhetoric.com/2020/03/the-rhetoric-of-puzzles/}{Every Day Rhetoric}}
\end{frame}


\mode<all>{
\part{Basic Constraint Propagation}
}
\begin{frame}
\partpage
\end{frame}
\mode<all>{
\input{../sendmore/body}
}

\mode<all>{
\part{Global Constraints}
}
\begin{frame}
\partpage
\end{frame}
\mode<all>{
\input{../sudoku/body}
}

\clearpage
\mode<all>{
\part{Customizing Search}
}
\begin{frame}
\partpage
\end{frame}
\mode<all>{
\input{../nqueen/body}
}


%% \clearpage
%% \mode<all>{
%%   \part{Problem Decomposition}
%% }
%% \begin{frame}
%%   \partpage
%% \end{frame}
%% \mode<all>{
%%   \input{../party/body}
%%   }

%% \clearpage
%% \mode<all>{
%%   \part{More Global Constraints}
%% }
%% \begin{frame}
%%   \partpage
%% \end{frame}
%% \mode<all>{
%%   \input{../car/body}
%%   }

%% \clearpage
%% \mode<all>{
%%   \part{Methodology}
%% }
%% \begin{frame}
%%   \partpage
%% \end{frame}
%% \mode<all>{
%%   \input{../methodology/body}
%%   }

\clearpage
\mode<all>{
\part{What is missing?}
}
\begin{frame}
\partpage
\end{frame}

\begin{frame}
\frametitle{Many Specialized Topics}
\begin{itemize}
\item How to design efficient core engine
\item Hybrids with LP/MIP tools
\item Hybrids with SAT
\item Symmetry breaking
\item Use of MDD/BDD to encode sets of solutions
\item High level modelling tools
\item Debugging/visualization
\end{itemize}
\end{frame}

\begin{frame}
\frametitle{Reformulation}
\begin{itemize}
\item Just because the user has modelled it this way, it doesn't mean we have to solve it that way
\begin{itemize}
\item Replace some constraint(s) by other, equivalent constraints
\item Because we don't have that constraint in our system
\item For performance
\end{itemize}
\end{itemize}
\end{frame}

\begin{frame}
\frametitle{Learning}
\begin{itemize}
\item While solving the problem we can learn how to strengthen the model/search 
\begin{itemize}
\item Understand which constraints/method contribute to propagation and change schedule
\item Learn no-good constraints by explaining failure
\item Adapt search strategy based on search experience
\end{itemize}
\end{itemize}
\end{frame}

%% \begin{frame}
%% \frametitle{Refined Process}
%% \begin{center}
%% \begin{tikzpicture}
%% \node[shape=rectangle,fill=pantone127-4] (problem) at (3,7) {Problem};
%% \node (human) at (3,6) {Human};
%% \node[shape=rectangle,fill=pantone127-4] (model) at (3,5) {User Model};
%% \node (reformulation) at (3,4) {Reformulation};
%% \node[shape=rectangle,fill=pantone127-4] (implementation) at (3,3) {Implementation Model};
%% \node (solver) at (3,2) {Constraint Solver/Search};
%% \node[shape=rectangle,fill=pantone157-8] (solution) at (3,1) {Solution};
%% \draw[-] (problem) -- (human);
%% \draw[->] (human) -- (model);
%% \draw[-] (model) -- (reformulation);
%% \draw[->] (reformulation) -- (implementation);
%% \draw[-] (implementation) -- (solver);
%% \draw[->] (solver) -- (0,2) -- (0,3) -- (implementation);
%% \draw[->] (solver) -- (solution);
%% \end{tikzpicture}
%% \end{center}
%% \end{frame}

%\begin{frame}
%\frametitle{What is CP actually used for?}
%\url{http://hsimonis.wordpress.com}
%\includegraphics[width=10cm]{../introduction/images/blog}
%\end{frame}

%% \begin{frame}
%% \frametitle{Exercises}
%% \begin{itemize}
%% \item Install Constraint Solver of your Choice
%% \item MiniZinc (MiningZinc) will be used in later sessions
%% \item Run Sendmoremoney, Sudoku, Queens examples
%% \item Can you choose which propagator to use for alldifferent?
%% \item Can you use different search strategies?
%% \end{itemize}
%% \end{frame}

\begin{frame}
\frametitle{More Learning Resources}
\begin{itemize}
\item Survey of Methods, Resources, and Formats for Teaching Constraint Programming
\begin{itemize}
\item by Tejas Santanam, Helmut Simonis
\item \url{https://doi.org/10.48550/arXiv.2403.12717}
\item Based on survey of community for WTCP 2023
\item \url{https://hsimonis.github.io/WTCP2023/}
\end{itemize}
\end{itemize}
\end{frame}


%%   }

\clearpage
\mode<all>{
\part{What is missing?}
}
\begin{frame}
\partpage
\end{frame}

\begin{frame}
\frametitle{Many Specialized Topics}
\begin{itemize}
\item How to design efficient core engine
\item Hybrids with LP/MIP tools
\item Hybrids with SAT
\item Symmetry breaking
\item Use of MDD/BDD to encode sets of solutions
\item High level modelling tools
\item Debugging/visualization
\end{itemize}
\end{frame}

\begin{frame}
\frametitle{Reformulation}
\begin{itemize}
\item Just because the user has modelled it this way, it doesn't mean we have to solve it that way
\begin{itemize}
\item Replace some constraint(s) by other, equivalent constraints
\item Because we don't have that constraint in our system
\item For performance
\end{itemize}
\end{itemize}
\end{frame}

\begin{frame}
\frametitle{Learning}
\begin{itemize}
\item While solving the problem we can learn how to strengthen the model/search 
\begin{itemize}
\item Understand which constraints/method contribute to propagation and change schedule
\item Learn no-good constraints by explaining failure
\item Adapt search strategy based on search experience
\end{itemize}
\end{itemize}
\end{frame}

%% \begin{frame}
%% \frametitle{Refined Process}
%% \begin{center}
%% \begin{tikzpicture}
%% \node[shape=rectangle,fill=pantone127-4] (problem) at (3,7) {Problem};
%% \node (human) at (3,6) {Human};
%% \node[shape=rectangle,fill=pantone127-4] (model) at (3,5) {User Model};
%% \node (reformulation) at (3,4) {Reformulation};
%% \node[shape=rectangle,fill=pantone127-4] (implementation) at (3,3) {Implementation Model};
%% \node (solver) at (3,2) {Constraint Solver/Search};
%% \node[shape=rectangle,fill=pantone157-8] (solution) at (3,1) {Solution};
%% \draw[-] (problem) -- (human);
%% \draw[->] (human) -- (model);
%% \draw[-] (model) -- (reformulation);
%% \draw[->] (reformulation) -- (implementation);
%% \draw[-] (implementation) -- (solver);
%% \draw[->] (solver) -- (0,2) -- (0,3) -- (implementation);
%% \draw[->] (solver) -- (solution);
%% \end{tikzpicture}
%% \end{center}
%% \end{frame}

%\begin{frame}
%\frametitle{What is CP actually used for?}
%\url{http://hsimonis.wordpress.com}
%\includegraphics[width=10cm]{../introduction/images/blog}
%\end{frame}

%% \begin{frame}
%% \frametitle{Exercises}
%% \begin{itemize}
%% \item Install Constraint Solver of your Choice
%% \item MiniZinc (MiningZinc) will be used in later sessions
%% \item Run Sendmoremoney, Sudoku, Queens examples
%% \item Can you choose which propagator to use for alldifferent?
%% \item Can you use different search strategies?
%% \end{itemize}
%% \end{frame}

\begin{frame}
\frametitle{More Learning Resources}
\begin{itemize}
\item Survey of Methods, Resources, and Formats for Teaching Constraint Programming
\begin{itemize}
\item by Tejas Santanam, Helmut Simonis
\item \url{https://doi.org/10.48550/arXiv.2403.12717}
\item Based on survey of community for WTCP 2023
\item \url{https://hsimonis.github.io/WTCP2023/}
\end{itemize}
\end{itemize}
\end{frame}


%%   }

%% \clearpage
%% \mode<all>{
%%   \part{Methodology}
%% }
%% \begin{frame}
%%   \partpage
%% \end{frame}
%% \mode<all>{
%%   \begin{frame}
\frametitle{Objectives}
\begin{itemize}
\item Overview of Core Constraint Programming
\item Three Main Concepts
\begin{itemize}
\item Constraint Propagation
\item Global Constraints
\item Customizing Search
\end{itemize}
%\item Get Some Experience with MiniZinc
\item Topics will be treated in more detail in later parts of the school
\item Based on Examples, not Formal Description
\end{itemize}
\end{frame}

\begin{frame}
\frametitle{Outline}
\begin{itemize}
\item Why Constraint Programming?
\item Constraint Propagation
\item Global Constraints
\item Customizing Search
%\item Choices in Modelling
%\item Decomposition
%\item More Global Constraints
%\item Methodology
%\item What is missing?
\end{itemize}
\end{frame}

% \begin{frame}
  % \frametitle{Using MiniZinc IDE}
  % \begin{itemize}
  % \item Developed in the Australian NICTA project
  % \item Maintained by Monash University
  % \item Modelling tool with multiple back-end solvers
  % \item Available from \url{https://www.minizinc.org/}
  % \end{itemize}
% \end{frame}


% \begin{frame}
% \frametitle{Examples in ECLiPSe}
% \begin{itemize}
% \item Open sourced constraint programming language
% \item Development goes back to 1985
% \item ECRC, ICL, IC-Parc, PTL, Cisco
% \item \url{https://eclipseclp.org/}
% \item Specialities
% \begin{itemize}
% \item Develop new solvers for specific domains
% \item Integration with MIP
% \end{itemize}
% \item Not included in bundled MiniZinc IDE
% \item Specialized visualization tools used here
  % \begin{itemize}
  % \item CP-Viz, Simonis et al. 2010
    % \end{itemize}
% \end{itemize}
% \end{frame}

\begin{frame}
\frametitle{Tutorial Based on ECLiPSe ELearning Course}
\begin{itemize}
\item Self-study course in constraint programming
\item Supported by Cisco Systems and Silicon Valley Community Foundation
\item Multi-media format, video lectures, slides, handout etc
\item {\small \url{https://eclipseclp.org/ELearning/index.html}}
\end{itemize}
\end{frame}

\begin{frame}
\frametitle{Also Part of CRT-AI Constraint Week}
\begin{itemize}
\item Annual one week course on CP and Optimization in Ireland
\item Part of national training program for PhD students in AI
\item \url{https://www.crt-ai.ie/}
\end{itemize}
\end{frame}



\begin{frame}
\frametitle{Constraint Programming - in a nutshell}
\begin{itemize}
\item Declarative description of problems with
\begin{itemize}
\item {\em Variables} which range over (finite) sets of values
\item {\em Constraints} over subsets of variables which restrict possible value combinations
\item A {\em solution} is a value assignment which satisfies all constraints
\end{itemize}

\item Constraint propagation/reasoning
\begin{itemize}
\item Removing inconsistent values for variables
\item Detect failure if constraint can not be satisfied
\item Interaction of constraints via shared variables
\item Incomplete
\end{itemize}

\item Search
\begin{itemize}
\item User controlled assignment of values to variables
\item Each step triggers constraint propagation 
\end{itemize}
\item Different domains require/allow different methods
\end{itemize}
\end{frame}

\begin{frame}
  \frametitle{Constraint Programming is Different}
  \begin{itemize}
  \item Declarative Programming
    \begin{itemize}
    \item Concentrate on what you want
      \item Not how to get there
      \item Program != Algorithm
      \item Program = Model
    \end{itemize}
    \item Applied to Combinatorial Problems
      \begin{itemize}
        \item No complete polynomial algorithms known (exist?)
        \item CP less ad-hoc than heuristics
        \item Models can evolve
  \end{itemize}
  \end{itemize}
  \end{frame}
    
\begin{frame}
  \frametitle{A Subtractive Process}
  \begin{textblock}{4}(8,-3)
    \includegraphics[width=4cm]{../introduction/images/stages}
  \end{textblock}
  \vfill
  \begin{quote}
    ``Oh, bosh, as Mr. Ruskin says. Sculpture, per se, is the simplest thing in the world. All you have to do is to take a big chunk of marble and a hammer and chisel, make up your mind what you are about to create and chip off all the marble you don't want.''-Paris Gaulois.
  \end{quote}
  
  {\tiny Source: \url{https://quoteinvestigator.com/2014/06/22/chip-away/}}
\end{frame}

\begin{frame}
\frametitle{Basic Process}
\begin{center}
\begin{tikzpicture}
\node[shape=rectangle,fill=pantone127-4] (problem) at (2,5) {Problem};
\node (human) at (2,4) {Human};
\node[shape=rectangle,fill=pantone127-4] (model) at (2,3) {Model};
\node (solver) at (2,2) {Constraint Solver/Search};
\node[shape=rectangle,fill=pantone157-8] (solution) at (2,1) {Solution};
\draw[-] (problem) -- (human);
\draw[->] (human) -- (model);
\draw[-] (model) -- (solver);
\draw[->] (solver) -- (solution);
\end{tikzpicture}
\end{center}
\end{frame}

\begin{frame}
\frametitle{More Realistic}
\begin{center}
\begin{tikzpicture}[xscale=4.0]
\node[shape=rectangle,fill=pantone127-4] (problem) at (2,5) {Problem};
\node (human) at (2,4) {Human};
\node[shape=rectangle,fill=pantone127-4] (model) at (2,3) {Model};
\node (solver) at (2,2) {Constraint Solver/Search};
\node[shape=rectangle,fill=pantone24-8] (hangs) at (1,1) {Hangs};
\node[shape=rectangle,fill=pantone157-8] (solution) at (2,1) {Solution};
\node[shape=rectangle,fill=pantone24-8] (wrong) at (3,1) {Wrong Solution};
\draw[-] (problem) -- (human);
\draw[->] (human) -- (model);
\draw[-] (model) -- (solver);
\draw[->] (solver) -- (hangs);
\draw[->] (solver) -- (solution);
\draw[->] (solver) -- (wrong);
\draw[->] (hangs) -- (1,4) -- (human);
\draw[->] (wrong) -- (3,4) -- (human);
\end{tikzpicture}
\end{center}
\end{frame}

\begin{frame}
\frametitle{Dual Role of Model}
\begin{itemize}
\item Allows Human to Express Problem
\begin{itemize}
\item Close to Problem Domain
\item Constraints as Abstractions
\end{itemize}
\item Allows Solver to Execute
\begin{itemize}
\item Variables as Communication Mechanism
\item Constraints as Algorithms
\end{itemize}
\end{itemize}
\end{frame}

\begin{frame}
\frametitle{Modelling Frameworks}
\begin{itemize}
\item \href{https://www.minizinc.org/}{MiniZinc} (NICTA, Monash University, Australia)
\item \href{https://github.com/eomahony/Numberjack}{NumberJack} (Insight, Ireland) 
\item \href{https://www-users.york.ac.uk/peter.nightingale/savilerow//}{EssencePrime/SavilleRow} (UK) 
\item \href{https://cpmpy.readthedocs.io/en/latest/}{CPMpy} (KU Leuven)
\item Allow use of multiple back-end solvers
\item Compile model into variants for each solver
\item A priori solver independent model(CP, MIP, SAT)
\end{itemize}
\end{frame} 

\begin{frame}
\frametitle{Framework Process}
\begin{center}
\begin{tikzpicture}[xscale=2]
\node[shape=rectangle,fill=pantone127-4] (problem) at (2.5,5) {Problem};
\node (human) at (2.5,4) {Human};
\node[shape=rectangle,fill=pantone127-4] (model) at (2.5,3) {Model};
\node (compile) at (2.5,2) {Compile/Reformulate};
\node (solver1) at (1,1) {CP};
\node (solver2) at (2,1) {MIP};
\node (solver3) at (3,1) {SAT};
\node (solver4) at (4,1) {Other};
\node[shape=rectangle,fill=pantone157-8] (solution1) at (1,0) {Solution};
\node[shape=rectangle,fill=pantone157-8] (solution2) at (2,0) {Solution};
\node[shape=rectangle,fill=pantone157-8] (solution3) at (3,0) {Solution};
\node[shape=rectangle,fill=pantone157-8] (solution4) at (4,0) {Solution};
\draw[-] (problem) -- (human);
\draw[->] (human) -- (model);
\draw[-] (model) -- (compile);
\draw[-] (compile) -- (solver1);
\draw[-] (compile) -- (solver2);
\draw[-] (compile) -- (solver3);
\draw[-] (compile) -- (solver4);
\draw[->] (solver1) -- (solution1);
\draw[->] (solver2) -- (solution2);
\draw[->] (solver3) -- (solution3);
\draw[->] (solver4) -- (solution4);
\end{tikzpicture}
\end{center}
\end{frame}

% \begin{frame}
  % \frametitle{Do It Now!}
  % \begin{itemize}
  % \item Download and install Minizinc
  % \item \url{https://www.minizinc.org/}
  % \end{itemize}
% \end{frame}


\begin{frame}
\frametitle{Why use Puzzles as Examples?}
\begin{itemize}
\item Easy to understand the problem
\item Solvable by hand without specialized knowledge
\item Possible to compare automated to manual solving process
\end{itemize}

\begin{quote}
The puzzle, though inanimate, is presented as a solvable problem without lasting negative consequences, a very low-risk low-reward situation. By being a puzzle, the object is attempting to convince the user that it must be completed.
\end{quote}
{\tiny Source: \href{https://www.everydayrhetoric.com/2020/03/the-rhetoric-of-puzzles/}{Every Day Rhetoric}}
\end{frame}


\mode<all>{
\part{Basic Constraint Propagation}
}
\begin{frame}
\partpage
\end{frame}
\mode<all>{
\begin{frame}
\frametitle{Objectives}
\begin{itemize}
\item Overview of Core Constraint Programming
\item Three Main Concepts
\begin{itemize}
\item Constraint Propagation
\item Global Constraints
\item Customizing Search
\end{itemize}
%\item Get Some Experience with MiniZinc
\item Topics will be treated in more detail in later parts of the school
\item Based on Examples, not Formal Description
\end{itemize}
\end{frame}

\begin{frame}
\frametitle{Outline}
\begin{itemize}
\item Why Constraint Programming?
\item Constraint Propagation
\item Global Constraints
\item Customizing Search
%\item Choices in Modelling
%\item Decomposition
%\item More Global Constraints
%\item Methodology
%\item What is missing?
\end{itemize}
\end{frame}

% \begin{frame}
  % \frametitle{Using MiniZinc IDE}
  % \begin{itemize}
  % \item Developed in the Australian NICTA project
  % \item Maintained by Monash University
  % \item Modelling tool with multiple back-end solvers
  % \item Available from \url{https://www.minizinc.org/}
  % \end{itemize}
% \end{frame}


% \begin{frame}
% \frametitle{Examples in ECLiPSe}
% \begin{itemize}
% \item Open sourced constraint programming language
% \item Development goes back to 1985
% \item ECRC, ICL, IC-Parc, PTL, Cisco
% \item \url{https://eclipseclp.org/}
% \item Specialities
% \begin{itemize}
% \item Develop new solvers for specific domains
% \item Integration with MIP
% \end{itemize}
% \item Not included in bundled MiniZinc IDE
% \item Specialized visualization tools used here
  % \begin{itemize}
  % \item CP-Viz, Simonis et al. 2010
    % \end{itemize}
% \end{itemize}
% \end{frame}

\begin{frame}
\frametitle{Tutorial Based on ECLiPSe ELearning Course}
\begin{itemize}
\item Self-study course in constraint programming
\item Supported by Cisco Systems and Silicon Valley Community Foundation
\item Multi-media format, video lectures, slides, handout etc
\item {\small \url{https://eclipseclp.org/ELearning/index.html}}
\end{itemize}
\end{frame}

\begin{frame}
\frametitle{Also Part of CRT-AI Constraint Week}
\begin{itemize}
\item Annual one week course on CP and Optimization in Ireland
\item Part of national training program for PhD students in AI
\item \url{https://www.crt-ai.ie/}
\end{itemize}
\end{frame}



\begin{frame}
\frametitle{Constraint Programming - in a nutshell}
\begin{itemize}
\item Declarative description of problems with
\begin{itemize}
\item {\em Variables} which range over (finite) sets of values
\item {\em Constraints} over subsets of variables which restrict possible value combinations
\item A {\em solution} is a value assignment which satisfies all constraints
\end{itemize}

\item Constraint propagation/reasoning
\begin{itemize}
\item Removing inconsistent values for variables
\item Detect failure if constraint can not be satisfied
\item Interaction of constraints via shared variables
\item Incomplete
\end{itemize}

\item Search
\begin{itemize}
\item User controlled assignment of values to variables
\item Each step triggers constraint propagation 
\end{itemize}
\item Different domains require/allow different methods
\end{itemize}
\end{frame}

\begin{frame}
  \frametitle{Constraint Programming is Different}
  \begin{itemize}
  \item Declarative Programming
    \begin{itemize}
    \item Concentrate on what you want
      \item Not how to get there
      \item Program != Algorithm
      \item Program = Model
    \end{itemize}
    \item Applied to Combinatorial Problems
      \begin{itemize}
        \item No complete polynomial algorithms known (exist?)
        \item CP less ad-hoc than heuristics
        \item Models can evolve
  \end{itemize}
  \end{itemize}
  \end{frame}
    
\begin{frame}
  \frametitle{A Subtractive Process}
  \begin{textblock}{4}(8,-3)
    \includegraphics[width=4cm]{../introduction/images/stages}
  \end{textblock}
  \vfill
  \begin{quote}
    ``Oh, bosh, as Mr. Ruskin says. Sculpture, per se, is the simplest thing in the world. All you have to do is to take a big chunk of marble and a hammer and chisel, make up your mind what you are about to create and chip off all the marble you don't want.''-Paris Gaulois.
  \end{quote}
  
  {\tiny Source: \url{https://quoteinvestigator.com/2014/06/22/chip-away/}}
\end{frame}

\begin{frame}
\frametitle{Basic Process}
\begin{center}
\begin{tikzpicture}
\node[shape=rectangle,fill=pantone127-4] (problem) at (2,5) {Problem};
\node (human) at (2,4) {Human};
\node[shape=rectangle,fill=pantone127-4] (model) at (2,3) {Model};
\node (solver) at (2,2) {Constraint Solver/Search};
\node[shape=rectangle,fill=pantone157-8] (solution) at (2,1) {Solution};
\draw[-] (problem) -- (human);
\draw[->] (human) -- (model);
\draw[-] (model) -- (solver);
\draw[->] (solver) -- (solution);
\end{tikzpicture}
\end{center}
\end{frame}

\begin{frame}
\frametitle{More Realistic}
\begin{center}
\begin{tikzpicture}[xscale=4.0]
\node[shape=rectangle,fill=pantone127-4] (problem) at (2,5) {Problem};
\node (human) at (2,4) {Human};
\node[shape=rectangle,fill=pantone127-4] (model) at (2,3) {Model};
\node (solver) at (2,2) {Constraint Solver/Search};
\node[shape=rectangle,fill=pantone24-8] (hangs) at (1,1) {Hangs};
\node[shape=rectangle,fill=pantone157-8] (solution) at (2,1) {Solution};
\node[shape=rectangle,fill=pantone24-8] (wrong) at (3,1) {Wrong Solution};
\draw[-] (problem) -- (human);
\draw[->] (human) -- (model);
\draw[-] (model) -- (solver);
\draw[->] (solver) -- (hangs);
\draw[->] (solver) -- (solution);
\draw[->] (solver) -- (wrong);
\draw[->] (hangs) -- (1,4) -- (human);
\draw[->] (wrong) -- (3,4) -- (human);
\end{tikzpicture}
\end{center}
\end{frame}

\begin{frame}
\frametitle{Dual Role of Model}
\begin{itemize}
\item Allows Human to Express Problem
\begin{itemize}
\item Close to Problem Domain
\item Constraints as Abstractions
\end{itemize}
\item Allows Solver to Execute
\begin{itemize}
\item Variables as Communication Mechanism
\item Constraints as Algorithms
\end{itemize}
\end{itemize}
\end{frame}

\begin{frame}
\frametitle{Modelling Frameworks}
\begin{itemize}
\item \href{https://www.minizinc.org/}{MiniZinc} (NICTA, Monash University, Australia)
\item \href{https://github.com/eomahony/Numberjack}{NumberJack} (Insight, Ireland) 
\item \href{https://www-users.york.ac.uk/peter.nightingale/savilerow//}{EssencePrime/SavilleRow} (UK) 
\item \href{https://cpmpy.readthedocs.io/en/latest/}{CPMpy} (KU Leuven)
\item Allow use of multiple back-end solvers
\item Compile model into variants for each solver
\item A priori solver independent model(CP, MIP, SAT)
\end{itemize}
\end{frame} 

\begin{frame}
\frametitle{Framework Process}
\begin{center}
\begin{tikzpicture}[xscale=2]
\node[shape=rectangle,fill=pantone127-4] (problem) at (2.5,5) {Problem};
\node (human) at (2.5,4) {Human};
\node[shape=rectangle,fill=pantone127-4] (model) at (2.5,3) {Model};
\node (compile) at (2.5,2) {Compile/Reformulate};
\node (solver1) at (1,1) {CP};
\node (solver2) at (2,1) {MIP};
\node (solver3) at (3,1) {SAT};
\node (solver4) at (4,1) {Other};
\node[shape=rectangle,fill=pantone157-8] (solution1) at (1,0) {Solution};
\node[shape=rectangle,fill=pantone157-8] (solution2) at (2,0) {Solution};
\node[shape=rectangle,fill=pantone157-8] (solution3) at (3,0) {Solution};
\node[shape=rectangle,fill=pantone157-8] (solution4) at (4,0) {Solution};
\draw[-] (problem) -- (human);
\draw[->] (human) -- (model);
\draw[-] (model) -- (compile);
\draw[-] (compile) -- (solver1);
\draw[-] (compile) -- (solver2);
\draw[-] (compile) -- (solver3);
\draw[-] (compile) -- (solver4);
\draw[->] (solver1) -- (solution1);
\draw[->] (solver2) -- (solution2);
\draw[->] (solver3) -- (solution3);
\draw[->] (solver4) -- (solution4);
\end{tikzpicture}
\end{center}
\end{frame}

% \begin{frame}
  % \frametitle{Do It Now!}
  % \begin{itemize}
  % \item Download and install Minizinc
  % \item \url{https://www.minizinc.org/}
  % \end{itemize}
% \end{frame}


\begin{frame}
\frametitle{Why use Puzzles as Examples?}
\begin{itemize}
\item Easy to understand the problem
\item Solvable by hand without specialized knowledge
\item Possible to compare automated to manual solving process
\end{itemize}

\begin{quote}
The puzzle, though inanimate, is presented as a solvable problem without lasting negative consequences, a very low-risk low-reward situation. By being a puzzle, the object is attempting to convince the user that it must be completed.
\end{quote}
{\tiny Source: \href{https://www.everydayrhetoric.com/2020/03/the-rhetoric-of-puzzles/}{Every Day Rhetoric}}
\end{frame}


\mode<all>{
\part{Basic Constraint Propagation}
}
\begin{frame}
\partpage
\end{frame}
\mode<all>{
\input{../sendmore/body}
}

\mode<all>{
\part{Global Constraints}
}
\begin{frame}
\partpage
\end{frame}
\mode<all>{
\input{../sudoku/body}
}

\clearpage
\mode<all>{
\part{Customizing Search}
}
\begin{frame}
\partpage
\end{frame}
\mode<all>{
\input{../nqueen/body}
}


%% \clearpage
%% \mode<all>{
%%   \part{Problem Decomposition}
%% }
%% \begin{frame}
%%   \partpage
%% \end{frame}
%% \mode<all>{
%%   \input{../party/body}
%%   }

%% \clearpage
%% \mode<all>{
%%   \part{More Global Constraints}
%% }
%% \begin{frame}
%%   \partpage
%% \end{frame}
%% \mode<all>{
%%   \input{../car/body}
%%   }

%% \clearpage
%% \mode<all>{
%%   \part{Methodology}
%% }
%% \begin{frame}
%%   \partpage
%% \end{frame}
%% \mode<all>{
%%   \input{../methodology/body}
%%   }

\clearpage
\mode<all>{
\part{What is missing?}
}
\begin{frame}
\partpage
\end{frame}

\begin{frame}
\frametitle{Many Specialized Topics}
\begin{itemize}
\item How to design efficient core engine
\item Hybrids with LP/MIP tools
\item Hybrids with SAT
\item Symmetry breaking
\item Use of MDD/BDD to encode sets of solutions
\item High level modelling tools
\item Debugging/visualization
\end{itemize}
\end{frame}

\begin{frame}
\frametitle{Reformulation}
\begin{itemize}
\item Just because the user has modelled it this way, it doesn't mean we have to solve it that way
\begin{itemize}
\item Replace some constraint(s) by other, equivalent constraints
\item Because we don't have that constraint in our system
\item For performance
\end{itemize}
\end{itemize}
\end{frame}

\begin{frame}
\frametitle{Learning}
\begin{itemize}
\item While solving the problem we can learn how to strengthen the model/search 
\begin{itemize}
\item Understand which constraints/method contribute to propagation and change schedule
\item Learn no-good constraints by explaining failure
\item Adapt search strategy based on search experience
\end{itemize}
\end{itemize}
\end{frame}

%% \begin{frame}
%% \frametitle{Refined Process}
%% \begin{center}
%% \begin{tikzpicture}
%% \node[shape=rectangle,fill=pantone127-4] (problem) at (3,7) {Problem};
%% \node (human) at (3,6) {Human};
%% \node[shape=rectangle,fill=pantone127-4] (model) at (3,5) {User Model};
%% \node (reformulation) at (3,4) {Reformulation};
%% \node[shape=rectangle,fill=pantone127-4] (implementation) at (3,3) {Implementation Model};
%% \node (solver) at (3,2) {Constraint Solver/Search};
%% \node[shape=rectangle,fill=pantone157-8] (solution) at (3,1) {Solution};
%% \draw[-] (problem) -- (human);
%% \draw[->] (human) -- (model);
%% \draw[-] (model) -- (reformulation);
%% \draw[->] (reformulation) -- (implementation);
%% \draw[-] (implementation) -- (solver);
%% \draw[->] (solver) -- (0,2) -- (0,3) -- (implementation);
%% \draw[->] (solver) -- (solution);
%% \end{tikzpicture}
%% \end{center}
%% \end{frame}

%\begin{frame}
%\frametitle{What is CP actually used for?}
%\url{http://hsimonis.wordpress.com}
%\includegraphics[width=10cm]{../introduction/images/blog}
%\end{frame}

%% \begin{frame}
%% \frametitle{Exercises}
%% \begin{itemize}
%% \item Install Constraint Solver of your Choice
%% \item MiniZinc (MiningZinc) will be used in later sessions
%% \item Run Sendmoremoney, Sudoku, Queens examples
%% \item Can you choose which propagator to use for alldifferent?
%% \item Can you use different search strategies?
%% \end{itemize}
%% \end{frame}

\begin{frame}
\frametitle{More Learning Resources}
\begin{itemize}
\item Survey of Methods, Resources, and Formats for Teaching Constraint Programming
\begin{itemize}
\item by Tejas Santanam, Helmut Simonis
\item \url{https://doi.org/10.48550/arXiv.2403.12717}
\item Based on survey of community for WTCP 2023
\item \url{https://hsimonis.github.io/WTCP2023/}
\end{itemize}
\end{itemize}
\end{frame}


}

\mode<all>{
\part{Global Constraints}
}
\begin{frame}
\partpage
\end{frame}
\mode<all>{
\begin{frame}
\frametitle{Objectives}
\begin{itemize}
\item Overview of Core Constraint Programming
\item Three Main Concepts
\begin{itemize}
\item Constraint Propagation
\item Global Constraints
\item Customizing Search
\end{itemize}
%\item Get Some Experience with MiniZinc
\item Topics will be treated in more detail in later parts of the school
\item Based on Examples, not Formal Description
\end{itemize}
\end{frame}

\begin{frame}
\frametitle{Outline}
\begin{itemize}
\item Why Constraint Programming?
\item Constraint Propagation
\item Global Constraints
\item Customizing Search
%\item Choices in Modelling
%\item Decomposition
%\item More Global Constraints
%\item Methodology
%\item What is missing?
\end{itemize}
\end{frame}

% \begin{frame}
  % \frametitle{Using MiniZinc IDE}
  % \begin{itemize}
  % \item Developed in the Australian NICTA project
  % \item Maintained by Monash University
  % \item Modelling tool with multiple back-end solvers
  % \item Available from \url{https://www.minizinc.org/}
  % \end{itemize}
% \end{frame}


% \begin{frame}
% \frametitle{Examples in ECLiPSe}
% \begin{itemize}
% \item Open sourced constraint programming language
% \item Development goes back to 1985
% \item ECRC, ICL, IC-Parc, PTL, Cisco
% \item \url{https://eclipseclp.org/}
% \item Specialities
% \begin{itemize}
% \item Develop new solvers for specific domains
% \item Integration with MIP
% \end{itemize}
% \item Not included in bundled MiniZinc IDE
% \item Specialized visualization tools used here
  % \begin{itemize}
  % \item CP-Viz, Simonis et al. 2010
    % \end{itemize}
% \end{itemize}
% \end{frame}

\begin{frame}
\frametitle{Tutorial Based on ECLiPSe ELearning Course}
\begin{itemize}
\item Self-study course in constraint programming
\item Supported by Cisco Systems and Silicon Valley Community Foundation
\item Multi-media format, video lectures, slides, handout etc
\item {\small \url{https://eclipseclp.org/ELearning/index.html}}
\end{itemize}
\end{frame}

\begin{frame}
\frametitle{Also Part of CRT-AI Constraint Week}
\begin{itemize}
\item Annual one week course on CP and Optimization in Ireland
\item Part of national training program for PhD students in AI
\item \url{https://www.crt-ai.ie/}
\end{itemize}
\end{frame}



\begin{frame}
\frametitle{Constraint Programming - in a nutshell}
\begin{itemize}
\item Declarative description of problems with
\begin{itemize}
\item {\em Variables} which range over (finite) sets of values
\item {\em Constraints} over subsets of variables which restrict possible value combinations
\item A {\em solution} is a value assignment which satisfies all constraints
\end{itemize}

\item Constraint propagation/reasoning
\begin{itemize}
\item Removing inconsistent values for variables
\item Detect failure if constraint can not be satisfied
\item Interaction of constraints via shared variables
\item Incomplete
\end{itemize}

\item Search
\begin{itemize}
\item User controlled assignment of values to variables
\item Each step triggers constraint propagation 
\end{itemize}
\item Different domains require/allow different methods
\end{itemize}
\end{frame}

\begin{frame}
  \frametitle{Constraint Programming is Different}
  \begin{itemize}
  \item Declarative Programming
    \begin{itemize}
    \item Concentrate on what you want
      \item Not how to get there
      \item Program != Algorithm
      \item Program = Model
    \end{itemize}
    \item Applied to Combinatorial Problems
      \begin{itemize}
        \item No complete polynomial algorithms known (exist?)
        \item CP less ad-hoc than heuristics
        \item Models can evolve
  \end{itemize}
  \end{itemize}
  \end{frame}
    
\begin{frame}
  \frametitle{A Subtractive Process}
  \begin{textblock}{4}(8,-3)
    \includegraphics[width=4cm]{../introduction/images/stages}
  \end{textblock}
  \vfill
  \begin{quote}
    ``Oh, bosh, as Mr. Ruskin says. Sculpture, per se, is the simplest thing in the world. All you have to do is to take a big chunk of marble and a hammer and chisel, make up your mind what you are about to create and chip off all the marble you don't want.''-Paris Gaulois.
  \end{quote}
  
  {\tiny Source: \url{https://quoteinvestigator.com/2014/06/22/chip-away/}}
\end{frame}

\begin{frame}
\frametitle{Basic Process}
\begin{center}
\begin{tikzpicture}
\node[shape=rectangle,fill=pantone127-4] (problem) at (2,5) {Problem};
\node (human) at (2,4) {Human};
\node[shape=rectangle,fill=pantone127-4] (model) at (2,3) {Model};
\node (solver) at (2,2) {Constraint Solver/Search};
\node[shape=rectangle,fill=pantone157-8] (solution) at (2,1) {Solution};
\draw[-] (problem) -- (human);
\draw[->] (human) -- (model);
\draw[-] (model) -- (solver);
\draw[->] (solver) -- (solution);
\end{tikzpicture}
\end{center}
\end{frame}

\begin{frame}
\frametitle{More Realistic}
\begin{center}
\begin{tikzpicture}[xscale=4.0]
\node[shape=rectangle,fill=pantone127-4] (problem) at (2,5) {Problem};
\node (human) at (2,4) {Human};
\node[shape=rectangle,fill=pantone127-4] (model) at (2,3) {Model};
\node (solver) at (2,2) {Constraint Solver/Search};
\node[shape=rectangle,fill=pantone24-8] (hangs) at (1,1) {Hangs};
\node[shape=rectangle,fill=pantone157-8] (solution) at (2,1) {Solution};
\node[shape=rectangle,fill=pantone24-8] (wrong) at (3,1) {Wrong Solution};
\draw[-] (problem) -- (human);
\draw[->] (human) -- (model);
\draw[-] (model) -- (solver);
\draw[->] (solver) -- (hangs);
\draw[->] (solver) -- (solution);
\draw[->] (solver) -- (wrong);
\draw[->] (hangs) -- (1,4) -- (human);
\draw[->] (wrong) -- (3,4) -- (human);
\end{tikzpicture}
\end{center}
\end{frame}

\begin{frame}
\frametitle{Dual Role of Model}
\begin{itemize}
\item Allows Human to Express Problem
\begin{itemize}
\item Close to Problem Domain
\item Constraints as Abstractions
\end{itemize}
\item Allows Solver to Execute
\begin{itemize}
\item Variables as Communication Mechanism
\item Constraints as Algorithms
\end{itemize}
\end{itemize}
\end{frame}

\begin{frame}
\frametitle{Modelling Frameworks}
\begin{itemize}
\item \href{https://www.minizinc.org/}{MiniZinc} (NICTA, Monash University, Australia)
\item \href{https://github.com/eomahony/Numberjack}{NumberJack} (Insight, Ireland) 
\item \href{https://www-users.york.ac.uk/peter.nightingale/savilerow//}{EssencePrime/SavilleRow} (UK) 
\item \href{https://cpmpy.readthedocs.io/en/latest/}{CPMpy} (KU Leuven)
\item Allow use of multiple back-end solvers
\item Compile model into variants for each solver
\item A priori solver independent model(CP, MIP, SAT)
\end{itemize}
\end{frame} 

\begin{frame}
\frametitle{Framework Process}
\begin{center}
\begin{tikzpicture}[xscale=2]
\node[shape=rectangle,fill=pantone127-4] (problem) at (2.5,5) {Problem};
\node (human) at (2.5,4) {Human};
\node[shape=rectangle,fill=pantone127-4] (model) at (2.5,3) {Model};
\node (compile) at (2.5,2) {Compile/Reformulate};
\node (solver1) at (1,1) {CP};
\node (solver2) at (2,1) {MIP};
\node (solver3) at (3,1) {SAT};
\node (solver4) at (4,1) {Other};
\node[shape=rectangle,fill=pantone157-8] (solution1) at (1,0) {Solution};
\node[shape=rectangle,fill=pantone157-8] (solution2) at (2,0) {Solution};
\node[shape=rectangle,fill=pantone157-8] (solution3) at (3,0) {Solution};
\node[shape=rectangle,fill=pantone157-8] (solution4) at (4,0) {Solution};
\draw[-] (problem) -- (human);
\draw[->] (human) -- (model);
\draw[-] (model) -- (compile);
\draw[-] (compile) -- (solver1);
\draw[-] (compile) -- (solver2);
\draw[-] (compile) -- (solver3);
\draw[-] (compile) -- (solver4);
\draw[->] (solver1) -- (solution1);
\draw[->] (solver2) -- (solution2);
\draw[->] (solver3) -- (solution3);
\draw[->] (solver4) -- (solution4);
\end{tikzpicture}
\end{center}
\end{frame}

% \begin{frame}
  % \frametitle{Do It Now!}
  % \begin{itemize}
  % \item Download and install Minizinc
  % \item \url{https://www.minizinc.org/}
  % \end{itemize}
% \end{frame}


\begin{frame}
\frametitle{Why use Puzzles as Examples?}
\begin{itemize}
\item Easy to understand the problem
\item Solvable by hand without specialized knowledge
\item Possible to compare automated to manual solving process
\end{itemize}

\begin{quote}
The puzzle, though inanimate, is presented as a solvable problem without lasting negative consequences, a very low-risk low-reward situation. By being a puzzle, the object is attempting to convince the user that it must be completed.
\end{quote}
{\tiny Source: \href{https://www.everydayrhetoric.com/2020/03/the-rhetoric-of-puzzles/}{Every Day Rhetoric}}
\end{frame}


\mode<all>{
\part{Basic Constraint Propagation}
}
\begin{frame}
\partpage
\end{frame}
\mode<all>{
\input{../sendmore/body}
}

\mode<all>{
\part{Global Constraints}
}
\begin{frame}
\partpage
\end{frame}
\mode<all>{
\input{../sudoku/body}
}

\clearpage
\mode<all>{
\part{Customizing Search}
}
\begin{frame}
\partpage
\end{frame}
\mode<all>{
\input{../nqueen/body}
}


%% \clearpage
%% \mode<all>{
%%   \part{Problem Decomposition}
%% }
%% \begin{frame}
%%   \partpage
%% \end{frame}
%% \mode<all>{
%%   \input{../party/body}
%%   }

%% \clearpage
%% \mode<all>{
%%   \part{More Global Constraints}
%% }
%% \begin{frame}
%%   \partpage
%% \end{frame}
%% \mode<all>{
%%   \input{../car/body}
%%   }

%% \clearpage
%% \mode<all>{
%%   \part{Methodology}
%% }
%% \begin{frame}
%%   \partpage
%% \end{frame}
%% \mode<all>{
%%   \input{../methodology/body}
%%   }

\clearpage
\mode<all>{
\part{What is missing?}
}
\begin{frame}
\partpage
\end{frame}

\begin{frame}
\frametitle{Many Specialized Topics}
\begin{itemize}
\item How to design efficient core engine
\item Hybrids with LP/MIP tools
\item Hybrids with SAT
\item Symmetry breaking
\item Use of MDD/BDD to encode sets of solutions
\item High level modelling tools
\item Debugging/visualization
\end{itemize}
\end{frame}

\begin{frame}
\frametitle{Reformulation}
\begin{itemize}
\item Just because the user has modelled it this way, it doesn't mean we have to solve it that way
\begin{itemize}
\item Replace some constraint(s) by other, equivalent constraints
\item Because we don't have that constraint in our system
\item For performance
\end{itemize}
\end{itemize}
\end{frame}

\begin{frame}
\frametitle{Learning}
\begin{itemize}
\item While solving the problem we can learn how to strengthen the model/search 
\begin{itemize}
\item Understand which constraints/method contribute to propagation and change schedule
\item Learn no-good constraints by explaining failure
\item Adapt search strategy based on search experience
\end{itemize}
\end{itemize}
\end{frame}

%% \begin{frame}
%% \frametitle{Refined Process}
%% \begin{center}
%% \begin{tikzpicture}
%% \node[shape=rectangle,fill=pantone127-4] (problem) at (3,7) {Problem};
%% \node (human) at (3,6) {Human};
%% \node[shape=rectangle,fill=pantone127-4] (model) at (3,5) {User Model};
%% \node (reformulation) at (3,4) {Reformulation};
%% \node[shape=rectangle,fill=pantone127-4] (implementation) at (3,3) {Implementation Model};
%% \node (solver) at (3,2) {Constraint Solver/Search};
%% \node[shape=rectangle,fill=pantone157-8] (solution) at (3,1) {Solution};
%% \draw[-] (problem) -- (human);
%% \draw[->] (human) -- (model);
%% \draw[-] (model) -- (reformulation);
%% \draw[->] (reformulation) -- (implementation);
%% \draw[-] (implementation) -- (solver);
%% \draw[->] (solver) -- (0,2) -- (0,3) -- (implementation);
%% \draw[->] (solver) -- (solution);
%% \end{tikzpicture}
%% \end{center}
%% \end{frame}

%\begin{frame}
%\frametitle{What is CP actually used for?}
%\url{http://hsimonis.wordpress.com}
%\includegraphics[width=10cm]{../introduction/images/blog}
%\end{frame}

%% \begin{frame}
%% \frametitle{Exercises}
%% \begin{itemize}
%% \item Install Constraint Solver of your Choice
%% \item MiniZinc (MiningZinc) will be used in later sessions
%% \item Run Sendmoremoney, Sudoku, Queens examples
%% \item Can you choose which propagator to use for alldifferent?
%% \item Can you use different search strategies?
%% \end{itemize}
%% \end{frame}

\begin{frame}
\frametitle{More Learning Resources}
\begin{itemize}
\item Survey of Methods, Resources, and Formats for Teaching Constraint Programming
\begin{itemize}
\item by Tejas Santanam, Helmut Simonis
\item \url{https://doi.org/10.48550/arXiv.2403.12717}
\item Based on survey of community for WTCP 2023
\item \url{https://hsimonis.github.io/WTCP2023/}
\end{itemize}
\end{itemize}
\end{frame}


}

\clearpage
\mode<all>{
\part{Customizing Search}
}
\begin{frame}
\partpage
\end{frame}
\mode<all>{
\begin{frame}
\frametitle{Objectives}
\begin{itemize}
\item Overview of Core Constraint Programming
\item Three Main Concepts
\begin{itemize}
\item Constraint Propagation
\item Global Constraints
\item Customizing Search
\end{itemize}
%\item Get Some Experience with MiniZinc
\item Topics will be treated in more detail in later parts of the school
\item Based on Examples, not Formal Description
\end{itemize}
\end{frame}

\begin{frame}
\frametitle{Outline}
\begin{itemize}
\item Why Constraint Programming?
\item Constraint Propagation
\item Global Constraints
\item Customizing Search
%\item Choices in Modelling
%\item Decomposition
%\item More Global Constraints
%\item Methodology
%\item What is missing?
\end{itemize}
\end{frame}

% \begin{frame}
  % \frametitle{Using MiniZinc IDE}
  % \begin{itemize}
  % \item Developed in the Australian NICTA project
  % \item Maintained by Monash University
  % \item Modelling tool with multiple back-end solvers
  % \item Available from \url{https://www.minizinc.org/}
  % \end{itemize}
% \end{frame}


% \begin{frame}
% \frametitle{Examples in ECLiPSe}
% \begin{itemize}
% \item Open sourced constraint programming language
% \item Development goes back to 1985
% \item ECRC, ICL, IC-Parc, PTL, Cisco
% \item \url{https://eclipseclp.org/}
% \item Specialities
% \begin{itemize}
% \item Develop new solvers for specific domains
% \item Integration with MIP
% \end{itemize}
% \item Not included in bundled MiniZinc IDE
% \item Specialized visualization tools used here
  % \begin{itemize}
  % \item CP-Viz, Simonis et al. 2010
    % \end{itemize}
% \end{itemize}
% \end{frame}

\begin{frame}
\frametitle{Tutorial Based on ECLiPSe ELearning Course}
\begin{itemize}
\item Self-study course in constraint programming
\item Supported by Cisco Systems and Silicon Valley Community Foundation
\item Multi-media format, video lectures, slides, handout etc
\item {\small \url{https://eclipseclp.org/ELearning/index.html}}
\end{itemize}
\end{frame}

\begin{frame}
\frametitle{Also Part of CRT-AI Constraint Week}
\begin{itemize}
\item Annual one week course on CP and Optimization in Ireland
\item Part of national training program for PhD students in AI
\item \url{https://www.crt-ai.ie/}
\end{itemize}
\end{frame}



\begin{frame}
\frametitle{Constraint Programming - in a nutshell}
\begin{itemize}
\item Declarative description of problems with
\begin{itemize}
\item {\em Variables} which range over (finite) sets of values
\item {\em Constraints} over subsets of variables which restrict possible value combinations
\item A {\em solution} is a value assignment which satisfies all constraints
\end{itemize}

\item Constraint propagation/reasoning
\begin{itemize}
\item Removing inconsistent values for variables
\item Detect failure if constraint can not be satisfied
\item Interaction of constraints via shared variables
\item Incomplete
\end{itemize}

\item Search
\begin{itemize}
\item User controlled assignment of values to variables
\item Each step triggers constraint propagation 
\end{itemize}
\item Different domains require/allow different methods
\end{itemize}
\end{frame}

\begin{frame}
  \frametitle{Constraint Programming is Different}
  \begin{itemize}
  \item Declarative Programming
    \begin{itemize}
    \item Concentrate on what you want
      \item Not how to get there
      \item Program != Algorithm
      \item Program = Model
    \end{itemize}
    \item Applied to Combinatorial Problems
      \begin{itemize}
        \item No complete polynomial algorithms known (exist?)
        \item CP less ad-hoc than heuristics
        \item Models can evolve
  \end{itemize}
  \end{itemize}
  \end{frame}
    
\begin{frame}
  \frametitle{A Subtractive Process}
  \begin{textblock}{4}(8,-3)
    \includegraphics[width=4cm]{../introduction/images/stages}
  \end{textblock}
  \vfill
  \begin{quote}
    ``Oh, bosh, as Mr. Ruskin says. Sculpture, per se, is the simplest thing in the world. All you have to do is to take a big chunk of marble and a hammer and chisel, make up your mind what you are about to create and chip off all the marble you don't want.''-Paris Gaulois.
  \end{quote}
  
  {\tiny Source: \url{https://quoteinvestigator.com/2014/06/22/chip-away/}}
\end{frame}

\begin{frame}
\frametitle{Basic Process}
\begin{center}
\begin{tikzpicture}
\node[shape=rectangle,fill=pantone127-4] (problem) at (2,5) {Problem};
\node (human) at (2,4) {Human};
\node[shape=rectangle,fill=pantone127-4] (model) at (2,3) {Model};
\node (solver) at (2,2) {Constraint Solver/Search};
\node[shape=rectangle,fill=pantone157-8] (solution) at (2,1) {Solution};
\draw[-] (problem) -- (human);
\draw[->] (human) -- (model);
\draw[-] (model) -- (solver);
\draw[->] (solver) -- (solution);
\end{tikzpicture}
\end{center}
\end{frame}

\begin{frame}
\frametitle{More Realistic}
\begin{center}
\begin{tikzpicture}[xscale=4.0]
\node[shape=rectangle,fill=pantone127-4] (problem) at (2,5) {Problem};
\node (human) at (2,4) {Human};
\node[shape=rectangle,fill=pantone127-4] (model) at (2,3) {Model};
\node (solver) at (2,2) {Constraint Solver/Search};
\node[shape=rectangle,fill=pantone24-8] (hangs) at (1,1) {Hangs};
\node[shape=rectangle,fill=pantone157-8] (solution) at (2,1) {Solution};
\node[shape=rectangle,fill=pantone24-8] (wrong) at (3,1) {Wrong Solution};
\draw[-] (problem) -- (human);
\draw[->] (human) -- (model);
\draw[-] (model) -- (solver);
\draw[->] (solver) -- (hangs);
\draw[->] (solver) -- (solution);
\draw[->] (solver) -- (wrong);
\draw[->] (hangs) -- (1,4) -- (human);
\draw[->] (wrong) -- (3,4) -- (human);
\end{tikzpicture}
\end{center}
\end{frame}

\begin{frame}
\frametitle{Dual Role of Model}
\begin{itemize}
\item Allows Human to Express Problem
\begin{itemize}
\item Close to Problem Domain
\item Constraints as Abstractions
\end{itemize}
\item Allows Solver to Execute
\begin{itemize}
\item Variables as Communication Mechanism
\item Constraints as Algorithms
\end{itemize}
\end{itemize}
\end{frame}

\begin{frame}
\frametitle{Modelling Frameworks}
\begin{itemize}
\item \href{https://www.minizinc.org/}{MiniZinc} (NICTA, Monash University, Australia)
\item \href{https://github.com/eomahony/Numberjack}{NumberJack} (Insight, Ireland) 
\item \href{https://www-users.york.ac.uk/peter.nightingale/savilerow//}{EssencePrime/SavilleRow} (UK) 
\item \href{https://cpmpy.readthedocs.io/en/latest/}{CPMpy} (KU Leuven)
\item Allow use of multiple back-end solvers
\item Compile model into variants for each solver
\item A priori solver independent model(CP, MIP, SAT)
\end{itemize}
\end{frame} 

\begin{frame}
\frametitle{Framework Process}
\begin{center}
\begin{tikzpicture}[xscale=2]
\node[shape=rectangle,fill=pantone127-4] (problem) at (2.5,5) {Problem};
\node (human) at (2.5,4) {Human};
\node[shape=rectangle,fill=pantone127-4] (model) at (2.5,3) {Model};
\node (compile) at (2.5,2) {Compile/Reformulate};
\node (solver1) at (1,1) {CP};
\node (solver2) at (2,1) {MIP};
\node (solver3) at (3,1) {SAT};
\node (solver4) at (4,1) {Other};
\node[shape=rectangle,fill=pantone157-8] (solution1) at (1,0) {Solution};
\node[shape=rectangle,fill=pantone157-8] (solution2) at (2,0) {Solution};
\node[shape=rectangle,fill=pantone157-8] (solution3) at (3,0) {Solution};
\node[shape=rectangle,fill=pantone157-8] (solution4) at (4,0) {Solution};
\draw[-] (problem) -- (human);
\draw[->] (human) -- (model);
\draw[-] (model) -- (compile);
\draw[-] (compile) -- (solver1);
\draw[-] (compile) -- (solver2);
\draw[-] (compile) -- (solver3);
\draw[-] (compile) -- (solver4);
\draw[->] (solver1) -- (solution1);
\draw[->] (solver2) -- (solution2);
\draw[->] (solver3) -- (solution3);
\draw[->] (solver4) -- (solution4);
\end{tikzpicture}
\end{center}
\end{frame}

% \begin{frame}
  % \frametitle{Do It Now!}
  % \begin{itemize}
  % \item Download and install Minizinc
  % \item \url{https://www.minizinc.org/}
  % \end{itemize}
% \end{frame}


\begin{frame}
\frametitle{Why use Puzzles as Examples?}
\begin{itemize}
\item Easy to understand the problem
\item Solvable by hand without specialized knowledge
\item Possible to compare automated to manual solving process
\end{itemize}

\begin{quote}
The puzzle, though inanimate, is presented as a solvable problem without lasting negative consequences, a very low-risk low-reward situation. By being a puzzle, the object is attempting to convince the user that it must be completed.
\end{quote}
{\tiny Source: \href{https://www.everydayrhetoric.com/2020/03/the-rhetoric-of-puzzles/}{Every Day Rhetoric}}
\end{frame}


\mode<all>{
\part{Basic Constraint Propagation}
}
\begin{frame}
\partpage
\end{frame}
\mode<all>{
\input{../sendmore/body}
}

\mode<all>{
\part{Global Constraints}
}
\begin{frame}
\partpage
\end{frame}
\mode<all>{
\input{../sudoku/body}
}

\clearpage
\mode<all>{
\part{Customizing Search}
}
\begin{frame}
\partpage
\end{frame}
\mode<all>{
\input{../nqueen/body}
}


%% \clearpage
%% \mode<all>{
%%   \part{Problem Decomposition}
%% }
%% \begin{frame}
%%   \partpage
%% \end{frame}
%% \mode<all>{
%%   \input{../party/body}
%%   }

%% \clearpage
%% \mode<all>{
%%   \part{More Global Constraints}
%% }
%% \begin{frame}
%%   \partpage
%% \end{frame}
%% \mode<all>{
%%   \input{../car/body}
%%   }

%% \clearpage
%% \mode<all>{
%%   \part{Methodology}
%% }
%% \begin{frame}
%%   \partpage
%% \end{frame}
%% \mode<all>{
%%   \input{../methodology/body}
%%   }

\clearpage
\mode<all>{
\part{What is missing?}
}
\begin{frame}
\partpage
\end{frame}

\begin{frame}
\frametitle{Many Specialized Topics}
\begin{itemize}
\item How to design efficient core engine
\item Hybrids with LP/MIP tools
\item Hybrids with SAT
\item Symmetry breaking
\item Use of MDD/BDD to encode sets of solutions
\item High level modelling tools
\item Debugging/visualization
\end{itemize}
\end{frame}

\begin{frame}
\frametitle{Reformulation}
\begin{itemize}
\item Just because the user has modelled it this way, it doesn't mean we have to solve it that way
\begin{itemize}
\item Replace some constraint(s) by other, equivalent constraints
\item Because we don't have that constraint in our system
\item For performance
\end{itemize}
\end{itemize}
\end{frame}

\begin{frame}
\frametitle{Learning}
\begin{itemize}
\item While solving the problem we can learn how to strengthen the model/search 
\begin{itemize}
\item Understand which constraints/method contribute to propagation and change schedule
\item Learn no-good constraints by explaining failure
\item Adapt search strategy based on search experience
\end{itemize}
\end{itemize}
\end{frame}

%% \begin{frame}
%% \frametitle{Refined Process}
%% \begin{center}
%% \begin{tikzpicture}
%% \node[shape=rectangle,fill=pantone127-4] (problem) at (3,7) {Problem};
%% \node (human) at (3,6) {Human};
%% \node[shape=rectangle,fill=pantone127-4] (model) at (3,5) {User Model};
%% \node (reformulation) at (3,4) {Reformulation};
%% \node[shape=rectangle,fill=pantone127-4] (implementation) at (3,3) {Implementation Model};
%% \node (solver) at (3,2) {Constraint Solver/Search};
%% \node[shape=rectangle,fill=pantone157-8] (solution) at (3,1) {Solution};
%% \draw[-] (problem) -- (human);
%% \draw[->] (human) -- (model);
%% \draw[-] (model) -- (reformulation);
%% \draw[->] (reformulation) -- (implementation);
%% \draw[-] (implementation) -- (solver);
%% \draw[->] (solver) -- (0,2) -- (0,3) -- (implementation);
%% \draw[->] (solver) -- (solution);
%% \end{tikzpicture}
%% \end{center}
%% \end{frame}

%\begin{frame}
%\frametitle{What is CP actually used for?}
%\url{http://hsimonis.wordpress.com}
%\includegraphics[width=10cm]{../introduction/images/blog}
%\end{frame}

%% \begin{frame}
%% \frametitle{Exercises}
%% \begin{itemize}
%% \item Install Constraint Solver of your Choice
%% \item MiniZinc (MiningZinc) will be used in later sessions
%% \item Run Sendmoremoney, Sudoku, Queens examples
%% \item Can you choose which propagator to use for alldifferent?
%% \item Can you use different search strategies?
%% \end{itemize}
%% \end{frame}

\begin{frame}
\frametitle{More Learning Resources}
\begin{itemize}
\item Survey of Methods, Resources, and Formats for Teaching Constraint Programming
\begin{itemize}
\item by Tejas Santanam, Helmut Simonis
\item \url{https://doi.org/10.48550/arXiv.2403.12717}
\item Based on survey of community for WTCP 2023
\item \url{https://hsimonis.github.io/WTCP2023/}
\end{itemize}
\end{itemize}
\end{frame}


}


%% \clearpage
%% \mode<all>{
%%   \part{Problem Decomposition}
%% }
%% \begin{frame}
%%   \partpage
%% \end{frame}
%% \mode<all>{
%%   \begin{frame}
\frametitle{Objectives}
\begin{itemize}
\item Overview of Core Constraint Programming
\item Three Main Concepts
\begin{itemize}
\item Constraint Propagation
\item Global Constraints
\item Customizing Search
\end{itemize}
%\item Get Some Experience with MiniZinc
\item Topics will be treated in more detail in later parts of the school
\item Based on Examples, not Formal Description
\end{itemize}
\end{frame}

\begin{frame}
\frametitle{Outline}
\begin{itemize}
\item Why Constraint Programming?
\item Constraint Propagation
\item Global Constraints
\item Customizing Search
%\item Choices in Modelling
%\item Decomposition
%\item More Global Constraints
%\item Methodology
%\item What is missing?
\end{itemize}
\end{frame}

% \begin{frame}
  % \frametitle{Using MiniZinc IDE}
  % \begin{itemize}
  % \item Developed in the Australian NICTA project
  % \item Maintained by Monash University
  % \item Modelling tool with multiple back-end solvers
  % \item Available from \url{https://www.minizinc.org/}
  % \end{itemize}
% \end{frame}


% \begin{frame}
% \frametitle{Examples in ECLiPSe}
% \begin{itemize}
% \item Open sourced constraint programming language
% \item Development goes back to 1985
% \item ECRC, ICL, IC-Parc, PTL, Cisco
% \item \url{https://eclipseclp.org/}
% \item Specialities
% \begin{itemize}
% \item Develop new solvers for specific domains
% \item Integration with MIP
% \end{itemize}
% \item Not included in bundled MiniZinc IDE
% \item Specialized visualization tools used here
  % \begin{itemize}
  % \item CP-Viz, Simonis et al. 2010
    % \end{itemize}
% \end{itemize}
% \end{frame}

\begin{frame}
\frametitle{Tutorial Based on ECLiPSe ELearning Course}
\begin{itemize}
\item Self-study course in constraint programming
\item Supported by Cisco Systems and Silicon Valley Community Foundation
\item Multi-media format, video lectures, slides, handout etc
\item {\small \url{https://eclipseclp.org/ELearning/index.html}}
\end{itemize}
\end{frame}

\begin{frame}
\frametitle{Also Part of CRT-AI Constraint Week}
\begin{itemize}
\item Annual one week course on CP and Optimization in Ireland
\item Part of national training program for PhD students in AI
\item \url{https://www.crt-ai.ie/}
\end{itemize}
\end{frame}



\begin{frame}
\frametitle{Constraint Programming - in a nutshell}
\begin{itemize}
\item Declarative description of problems with
\begin{itemize}
\item {\em Variables} which range over (finite) sets of values
\item {\em Constraints} over subsets of variables which restrict possible value combinations
\item A {\em solution} is a value assignment which satisfies all constraints
\end{itemize}

\item Constraint propagation/reasoning
\begin{itemize}
\item Removing inconsistent values for variables
\item Detect failure if constraint can not be satisfied
\item Interaction of constraints via shared variables
\item Incomplete
\end{itemize}

\item Search
\begin{itemize}
\item User controlled assignment of values to variables
\item Each step triggers constraint propagation 
\end{itemize}
\item Different domains require/allow different methods
\end{itemize}
\end{frame}

\begin{frame}
  \frametitle{Constraint Programming is Different}
  \begin{itemize}
  \item Declarative Programming
    \begin{itemize}
    \item Concentrate on what you want
      \item Not how to get there
      \item Program != Algorithm
      \item Program = Model
    \end{itemize}
    \item Applied to Combinatorial Problems
      \begin{itemize}
        \item No complete polynomial algorithms known (exist?)
        \item CP less ad-hoc than heuristics
        \item Models can evolve
  \end{itemize}
  \end{itemize}
  \end{frame}
    
\begin{frame}
  \frametitle{A Subtractive Process}
  \begin{textblock}{4}(8,-3)
    \includegraphics[width=4cm]{../introduction/images/stages}
  \end{textblock}
  \vfill
  \begin{quote}
    ``Oh, bosh, as Mr. Ruskin says. Sculpture, per se, is the simplest thing in the world. All you have to do is to take a big chunk of marble and a hammer and chisel, make up your mind what you are about to create and chip off all the marble you don't want.''-Paris Gaulois.
  \end{quote}
  
  {\tiny Source: \url{https://quoteinvestigator.com/2014/06/22/chip-away/}}
\end{frame}

\begin{frame}
\frametitle{Basic Process}
\begin{center}
\begin{tikzpicture}
\node[shape=rectangle,fill=pantone127-4] (problem) at (2,5) {Problem};
\node (human) at (2,4) {Human};
\node[shape=rectangle,fill=pantone127-4] (model) at (2,3) {Model};
\node (solver) at (2,2) {Constraint Solver/Search};
\node[shape=rectangle,fill=pantone157-8] (solution) at (2,1) {Solution};
\draw[-] (problem) -- (human);
\draw[->] (human) -- (model);
\draw[-] (model) -- (solver);
\draw[->] (solver) -- (solution);
\end{tikzpicture}
\end{center}
\end{frame}

\begin{frame}
\frametitle{More Realistic}
\begin{center}
\begin{tikzpicture}[xscale=4.0]
\node[shape=rectangle,fill=pantone127-4] (problem) at (2,5) {Problem};
\node (human) at (2,4) {Human};
\node[shape=rectangle,fill=pantone127-4] (model) at (2,3) {Model};
\node (solver) at (2,2) {Constraint Solver/Search};
\node[shape=rectangle,fill=pantone24-8] (hangs) at (1,1) {Hangs};
\node[shape=rectangle,fill=pantone157-8] (solution) at (2,1) {Solution};
\node[shape=rectangle,fill=pantone24-8] (wrong) at (3,1) {Wrong Solution};
\draw[-] (problem) -- (human);
\draw[->] (human) -- (model);
\draw[-] (model) -- (solver);
\draw[->] (solver) -- (hangs);
\draw[->] (solver) -- (solution);
\draw[->] (solver) -- (wrong);
\draw[->] (hangs) -- (1,4) -- (human);
\draw[->] (wrong) -- (3,4) -- (human);
\end{tikzpicture}
\end{center}
\end{frame}

\begin{frame}
\frametitle{Dual Role of Model}
\begin{itemize}
\item Allows Human to Express Problem
\begin{itemize}
\item Close to Problem Domain
\item Constraints as Abstractions
\end{itemize}
\item Allows Solver to Execute
\begin{itemize}
\item Variables as Communication Mechanism
\item Constraints as Algorithms
\end{itemize}
\end{itemize}
\end{frame}

\begin{frame}
\frametitle{Modelling Frameworks}
\begin{itemize}
\item \href{https://www.minizinc.org/}{MiniZinc} (NICTA, Monash University, Australia)
\item \href{https://github.com/eomahony/Numberjack}{NumberJack} (Insight, Ireland) 
\item \href{https://www-users.york.ac.uk/peter.nightingale/savilerow//}{EssencePrime/SavilleRow} (UK) 
\item \href{https://cpmpy.readthedocs.io/en/latest/}{CPMpy} (KU Leuven)
\item Allow use of multiple back-end solvers
\item Compile model into variants for each solver
\item A priori solver independent model(CP, MIP, SAT)
\end{itemize}
\end{frame} 

\begin{frame}
\frametitle{Framework Process}
\begin{center}
\begin{tikzpicture}[xscale=2]
\node[shape=rectangle,fill=pantone127-4] (problem) at (2.5,5) {Problem};
\node (human) at (2.5,4) {Human};
\node[shape=rectangle,fill=pantone127-4] (model) at (2.5,3) {Model};
\node (compile) at (2.5,2) {Compile/Reformulate};
\node (solver1) at (1,1) {CP};
\node (solver2) at (2,1) {MIP};
\node (solver3) at (3,1) {SAT};
\node (solver4) at (4,1) {Other};
\node[shape=rectangle,fill=pantone157-8] (solution1) at (1,0) {Solution};
\node[shape=rectangle,fill=pantone157-8] (solution2) at (2,0) {Solution};
\node[shape=rectangle,fill=pantone157-8] (solution3) at (3,0) {Solution};
\node[shape=rectangle,fill=pantone157-8] (solution4) at (4,0) {Solution};
\draw[-] (problem) -- (human);
\draw[->] (human) -- (model);
\draw[-] (model) -- (compile);
\draw[-] (compile) -- (solver1);
\draw[-] (compile) -- (solver2);
\draw[-] (compile) -- (solver3);
\draw[-] (compile) -- (solver4);
\draw[->] (solver1) -- (solution1);
\draw[->] (solver2) -- (solution2);
\draw[->] (solver3) -- (solution3);
\draw[->] (solver4) -- (solution4);
\end{tikzpicture}
\end{center}
\end{frame}

% \begin{frame}
  % \frametitle{Do It Now!}
  % \begin{itemize}
  % \item Download and install Minizinc
  % \item \url{https://www.minizinc.org/}
  % \end{itemize}
% \end{frame}


\begin{frame}
\frametitle{Why use Puzzles as Examples?}
\begin{itemize}
\item Easy to understand the problem
\item Solvable by hand without specialized knowledge
\item Possible to compare automated to manual solving process
\end{itemize}

\begin{quote}
The puzzle, though inanimate, is presented as a solvable problem without lasting negative consequences, a very low-risk low-reward situation. By being a puzzle, the object is attempting to convince the user that it must be completed.
\end{quote}
{\tiny Source: \href{https://www.everydayrhetoric.com/2020/03/the-rhetoric-of-puzzles/}{Every Day Rhetoric}}
\end{frame}


\mode<all>{
\part{Basic Constraint Propagation}
}
\begin{frame}
\partpage
\end{frame}
\mode<all>{
\input{../sendmore/body}
}

\mode<all>{
\part{Global Constraints}
}
\begin{frame}
\partpage
\end{frame}
\mode<all>{
\input{../sudoku/body}
}

\clearpage
\mode<all>{
\part{Customizing Search}
}
\begin{frame}
\partpage
\end{frame}
\mode<all>{
\input{../nqueen/body}
}


%% \clearpage
%% \mode<all>{
%%   \part{Problem Decomposition}
%% }
%% \begin{frame}
%%   \partpage
%% \end{frame}
%% \mode<all>{
%%   \input{../party/body}
%%   }

%% \clearpage
%% \mode<all>{
%%   \part{More Global Constraints}
%% }
%% \begin{frame}
%%   \partpage
%% \end{frame}
%% \mode<all>{
%%   \input{../car/body}
%%   }

%% \clearpage
%% \mode<all>{
%%   \part{Methodology}
%% }
%% \begin{frame}
%%   \partpage
%% \end{frame}
%% \mode<all>{
%%   \input{../methodology/body}
%%   }

\clearpage
\mode<all>{
\part{What is missing?}
}
\begin{frame}
\partpage
\end{frame}

\begin{frame}
\frametitle{Many Specialized Topics}
\begin{itemize}
\item How to design efficient core engine
\item Hybrids with LP/MIP tools
\item Hybrids with SAT
\item Symmetry breaking
\item Use of MDD/BDD to encode sets of solutions
\item High level modelling tools
\item Debugging/visualization
\end{itemize}
\end{frame}

\begin{frame}
\frametitle{Reformulation}
\begin{itemize}
\item Just because the user has modelled it this way, it doesn't mean we have to solve it that way
\begin{itemize}
\item Replace some constraint(s) by other, equivalent constraints
\item Because we don't have that constraint in our system
\item For performance
\end{itemize}
\end{itemize}
\end{frame}

\begin{frame}
\frametitle{Learning}
\begin{itemize}
\item While solving the problem we can learn how to strengthen the model/search 
\begin{itemize}
\item Understand which constraints/method contribute to propagation and change schedule
\item Learn no-good constraints by explaining failure
\item Adapt search strategy based on search experience
\end{itemize}
\end{itemize}
\end{frame}

%% \begin{frame}
%% \frametitle{Refined Process}
%% \begin{center}
%% \begin{tikzpicture}
%% \node[shape=rectangle,fill=pantone127-4] (problem) at (3,7) {Problem};
%% \node (human) at (3,6) {Human};
%% \node[shape=rectangle,fill=pantone127-4] (model) at (3,5) {User Model};
%% \node (reformulation) at (3,4) {Reformulation};
%% \node[shape=rectangle,fill=pantone127-4] (implementation) at (3,3) {Implementation Model};
%% \node (solver) at (3,2) {Constraint Solver/Search};
%% \node[shape=rectangle,fill=pantone157-8] (solution) at (3,1) {Solution};
%% \draw[-] (problem) -- (human);
%% \draw[->] (human) -- (model);
%% \draw[-] (model) -- (reformulation);
%% \draw[->] (reformulation) -- (implementation);
%% \draw[-] (implementation) -- (solver);
%% \draw[->] (solver) -- (0,2) -- (0,3) -- (implementation);
%% \draw[->] (solver) -- (solution);
%% \end{tikzpicture}
%% \end{center}
%% \end{frame}

%\begin{frame}
%\frametitle{What is CP actually used for?}
%\url{http://hsimonis.wordpress.com}
%\includegraphics[width=10cm]{../introduction/images/blog}
%\end{frame}

%% \begin{frame}
%% \frametitle{Exercises}
%% \begin{itemize}
%% \item Install Constraint Solver of your Choice
%% \item MiniZinc (MiningZinc) will be used in later sessions
%% \item Run Sendmoremoney, Sudoku, Queens examples
%% \item Can you choose which propagator to use for alldifferent?
%% \item Can you use different search strategies?
%% \end{itemize}
%% \end{frame}

\begin{frame}
\frametitle{More Learning Resources}
\begin{itemize}
\item Survey of Methods, Resources, and Formats for Teaching Constraint Programming
\begin{itemize}
\item by Tejas Santanam, Helmut Simonis
\item \url{https://doi.org/10.48550/arXiv.2403.12717}
\item Based on survey of community for WTCP 2023
\item \url{https://hsimonis.github.io/WTCP2023/}
\end{itemize}
\end{itemize}
\end{frame}


%%   }

%% \clearpage
%% \mode<all>{
%%   \part{More Global Constraints}
%% }
%% \begin{frame}
%%   \partpage
%% \end{frame}
%% \mode<all>{
%%   \begin{frame}
\frametitle{Objectives}
\begin{itemize}
\item Overview of Core Constraint Programming
\item Three Main Concepts
\begin{itemize}
\item Constraint Propagation
\item Global Constraints
\item Customizing Search
\end{itemize}
%\item Get Some Experience with MiniZinc
\item Topics will be treated in more detail in later parts of the school
\item Based on Examples, not Formal Description
\end{itemize}
\end{frame}

\begin{frame}
\frametitle{Outline}
\begin{itemize}
\item Why Constraint Programming?
\item Constraint Propagation
\item Global Constraints
\item Customizing Search
%\item Choices in Modelling
%\item Decomposition
%\item More Global Constraints
%\item Methodology
%\item What is missing?
\end{itemize}
\end{frame}

% \begin{frame}
  % \frametitle{Using MiniZinc IDE}
  % \begin{itemize}
  % \item Developed in the Australian NICTA project
  % \item Maintained by Monash University
  % \item Modelling tool with multiple back-end solvers
  % \item Available from \url{https://www.minizinc.org/}
  % \end{itemize}
% \end{frame}


% \begin{frame}
% \frametitle{Examples in ECLiPSe}
% \begin{itemize}
% \item Open sourced constraint programming language
% \item Development goes back to 1985
% \item ECRC, ICL, IC-Parc, PTL, Cisco
% \item \url{https://eclipseclp.org/}
% \item Specialities
% \begin{itemize}
% \item Develop new solvers for specific domains
% \item Integration with MIP
% \end{itemize}
% \item Not included in bundled MiniZinc IDE
% \item Specialized visualization tools used here
  % \begin{itemize}
  % \item CP-Viz, Simonis et al. 2010
    % \end{itemize}
% \end{itemize}
% \end{frame}

\begin{frame}
\frametitle{Tutorial Based on ECLiPSe ELearning Course}
\begin{itemize}
\item Self-study course in constraint programming
\item Supported by Cisco Systems and Silicon Valley Community Foundation
\item Multi-media format, video lectures, slides, handout etc
\item {\small \url{https://eclipseclp.org/ELearning/index.html}}
\end{itemize}
\end{frame}

\begin{frame}
\frametitle{Also Part of CRT-AI Constraint Week}
\begin{itemize}
\item Annual one week course on CP and Optimization in Ireland
\item Part of national training program for PhD students in AI
\item \url{https://www.crt-ai.ie/}
\end{itemize}
\end{frame}



\begin{frame}
\frametitle{Constraint Programming - in a nutshell}
\begin{itemize}
\item Declarative description of problems with
\begin{itemize}
\item {\em Variables} which range over (finite) sets of values
\item {\em Constraints} over subsets of variables which restrict possible value combinations
\item A {\em solution} is a value assignment which satisfies all constraints
\end{itemize}

\item Constraint propagation/reasoning
\begin{itemize}
\item Removing inconsistent values for variables
\item Detect failure if constraint can not be satisfied
\item Interaction of constraints via shared variables
\item Incomplete
\end{itemize}

\item Search
\begin{itemize}
\item User controlled assignment of values to variables
\item Each step triggers constraint propagation 
\end{itemize}
\item Different domains require/allow different methods
\end{itemize}
\end{frame}

\begin{frame}
  \frametitle{Constraint Programming is Different}
  \begin{itemize}
  \item Declarative Programming
    \begin{itemize}
    \item Concentrate on what you want
      \item Not how to get there
      \item Program != Algorithm
      \item Program = Model
    \end{itemize}
    \item Applied to Combinatorial Problems
      \begin{itemize}
        \item No complete polynomial algorithms known (exist?)
        \item CP less ad-hoc than heuristics
        \item Models can evolve
  \end{itemize}
  \end{itemize}
  \end{frame}
    
\begin{frame}
  \frametitle{A Subtractive Process}
  \begin{textblock}{4}(8,-3)
    \includegraphics[width=4cm]{../introduction/images/stages}
  \end{textblock}
  \vfill
  \begin{quote}
    ``Oh, bosh, as Mr. Ruskin says. Sculpture, per se, is the simplest thing in the world. All you have to do is to take a big chunk of marble and a hammer and chisel, make up your mind what you are about to create and chip off all the marble you don't want.''-Paris Gaulois.
  \end{quote}
  
  {\tiny Source: \url{https://quoteinvestigator.com/2014/06/22/chip-away/}}
\end{frame}

\begin{frame}
\frametitle{Basic Process}
\begin{center}
\begin{tikzpicture}
\node[shape=rectangle,fill=pantone127-4] (problem) at (2,5) {Problem};
\node (human) at (2,4) {Human};
\node[shape=rectangle,fill=pantone127-4] (model) at (2,3) {Model};
\node (solver) at (2,2) {Constraint Solver/Search};
\node[shape=rectangle,fill=pantone157-8] (solution) at (2,1) {Solution};
\draw[-] (problem) -- (human);
\draw[->] (human) -- (model);
\draw[-] (model) -- (solver);
\draw[->] (solver) -- (solution);
\end{tikzpicture}
\end{center}
\end{frame}

\begin{frame}
\frametitle{More Realistic}
\begin{center}
\begin{tikzpicture}[xscale=4.0]
\node[shape=rectangle,fill=pantone127-4] (problem) at (2,5) {Problem};
\node (human) at (2,4) {Human};
\node[shape=rectangle,fill=pantone127-4] (model) at (2,3) {Model};
\node (solver) at (2,2) {Constraint Solver/Search};
\node[shape=rectangle,fill=pantone24-8] (hangs) at (1,1) {Hangs};
\node[shape=rectangle,fill=pantone157-8] (solution) at (2,1) {Solution};
\node[shape=rectangle,fill=pantone24-8] (wrong) at (3,1) {Wrong Solution};
\draw[-] (problem) -- (human);
\draw[->] (human) -- (model);
\draw[-] (model) -- (solver);
\draw[->] (solver) -- (hangs);
\draw[->] (solver) -- (solution);
\draw[->] (solver) -- (wrong);
\draw[->] (hangs) -- (1,4) -- (human);
\draw[->] (wrong) -- (3,4) -- (human);
\end{tikzpicture}
\end{center}
\end{frame}

\begin{frame}
\frametitle{Dual Role of Model}
\begin{itemize}
\item Allows Human to Express Problem
\begin{itemize}
\item Close to Problem Domain
\item Constraints as Abstractions
\end{itemize}
\item Allows Solver to Execute
\begin{itemize}
\item Variables as Communication Mechanism
\item Constraints as Algorithms
\end{itemize}
\end{itemize}
\end{frame}

\begin{frame}
\frametitle{Modelling Frameworks}
\begin{itemize}
\item \href{https://www.minizinc.org/}{MiniZinc} (NICTA, Monash University, Australia)
\item \href{https://github.com/eomahony/Numberjack}{NumberJack} (Insight, Ireland) 
\item \href{https://www-users.york.ac.uk/peter.nightingale/savilerow//}{EssencePrime/SavilleRow} (UK) 
\item \href{https://cpmpy.readthedocs.io/en/latest/}{CPMpy} (KU Leuven)
\item Allow use of multiple back-end solvers
\item Compile model into variants for each solver
\item A priori solver independent model(CP, MIP, SAT)
\end{itemize}
\end{frame} 

\begin{frame}
\frametitle{Framework Process}
\begin{center}
\begin{tikzpicture}[xscale=2]
\node[shape=rectangle,fill=pantone127-4] (problem) at (2.5,5) {Problem};
\node (human) at (2.5,4) {Human};
\node[shape=rectangle,fill=pantone127-4] (model) at (2.5,3) {Model};
\node (compile) at (2.5,2) {Compile/Reformulate};
\node (solver1) at (1,1) {CP};
\node (solver2) at (2,1) {MIP};
\node (solver3) at (3,1) {SAT};
\node (solver4) at (4,1) {Other};
\node[shape=rectangle,fill=pantone157-8] (solution1) at (1,0) {Solution};
\node[shape=rectangle,fill=pantone157-8] (solution2) at (2,0) {Solution};
\node[shape=rectangle,fill=pantone157-8] (solution3) at (3,0) {Solution};
\node[shape=rectangle,fill=pantone157-8] (solution4) at (4,0) {Solution};
\draw[-] (problem) -- (human);
\draw[->] (human) -- (model);
\draw[-] (model) -- (compile);
\draw[-] (compile) -- (solver1);
\draw[-] (compile) -- (solver2);
\draw[-] (compile) -- (solver3);
\draw[-] (compile) -- (solver4);
\draw[->] (solver1) -- (solution1);
\draw[->] (solver2) -- (solution2);
\draw[->] (solver3) -- (solution3);
\draw[->] (solver4) -- (solution4);
\end{tikzpicture}
\end{center}
\end{frame}

% \begin{frame}
  % \frametitle{Do It Now!}
  % \begin{itemize}
  % \item Download and install Minizinc
  % \item \url{https://www.minizinc.org/}
  % \end{itemize}
% \end{frame}


\begin{frame}
\frametitle{Why use Puzzles as Examples?}
\begin{itemize}
\item Easy to understand the problem
\item Solvable by hand without specialized knowledge
\item Possible to compare automated to manual solving process
\end{itemize}

\begin{quote}
The puzzle, though inanimate, is presented as a solvable problem without lasting negative consequences, a very low-risk low-reward situation. By being a puzzle, the object is attempting to convince the user that it must be completed.
\end{quote}
{\tiny Source: \href{https://www.everydayrhetoric.com/2020/03/the-rhetoric-of-puzzles/}{Every Day Rhetoric}}
\end{frame}


\mode<all>{
\part{Basic Constraint Propagation}
}
\begin{frame}
\partpage
\end{frame}
\mode<all>{
\input{../sendmore/body}
}

\mode<all>{
\part{Global Constraints}
}
\begin{frame}
\partpage
\end{frame}
\mode<all>{
\input{../sudoku/body}
}

\clearpage
\mode<all>{
\part{Customizing Search}
}
\begin{frame}
\partpage
\end{frame}
\mode<all>{
\input{../nqueen/body}
}


%% \clearpage
%% \mode<all>{
%%   \part{Problem Decomposition}
%% }
%% \begin{frame}
%%   \partpage
%% \end{frame}
%% \mode<all>{
%%   \input{../party/body}
%%   }

%% \clearpage
%% \mode<all>{
%%   \part{More Global Constraints}
%% }
%% \begin{frame}
%%   \partpage
%% \end{frame}
%% \mode<all>{
%%   \input{../car/body}
%%   }

%% \clearpage
%% \mode<all>{
%%   \part{Methodology}
%% }
%% \begin{frame}
%%   \partpage
%% \end{frame}
%% \mode<all>{
%%   \input{../methodology/body}
%%   }

\clearpage
\mode<all>{
\part{What is missing?}
}
\begin{frame}
\partpage
\end{frame}

\begin{frame}
\frametitle{Many Specialized Topics}
\begin{itemize}
\item How to design efficient core engine
\item Hybrids with LP/MIP tools
\item Hybrids with SAT
\item Symmetry breaking
\item Use of MDD/BDD to encode sets of solutions
\item High level modelling tools
\item Debugging/visualization
\end{itemize}
\end{frame}

\begin{frame}
\frametitle{Reformulation}
\begin{itemize}
\item Just because the user has modelled it this way, it doesn't mean we have to solve it that way
\begin{itemize}
\item Replace some constraint(s) by other, equivalent constraints
\item Because we don't have that constraint in our system
\item For performance
\end{itemize}
\end{itemize}
\end{frame}

\begin{frame}
\frametitle{Learning}
\begin{itemize}
\item While solving the problem we can learn how to strengthen the model/search 
\begin{itemize}
\item Understand which constraints/method contribute to propagation and change schedule
\item Learn no-good constraints by explaining failure
\item Adapt search strategy based on search experience
\end{itemize}
\end{itemize}
\end{frame}

%% \begin{frame}
%% \frametitle{Refined Process}
%% \begin{center}
%% \begin{tikzpicture}
%% \node[shape=rectangle,fill=pantone127-4] (problem) at (3,7) {Problem};
%% \node (human) at (3,6) {Human};
%% \node[shape=rectangle,fill=pantone127-4] (model) at (3,5) {User Model};
%% \node (reformulation) at (3,4) {Reformulation};
%% \node[shape=rectangle,fill=pantone127-4] (implementation) at (3,3) {Implementation Model};
%% \node (solver) at (3,2) {Constraint Solver/Search};
%% \node[shape=rectangle,fill=pantone157-8] (solution) at (3,1) {Solution};
%% \draw[-] (problem) -- (human);
%% \draw[->] (human) -- (model);
%% \draw[-] (model) -- (reformulation);
%% \draw[->] (reformulation) -- (implementation);
%% \draw[-] (implementation) -- (solver);
%% \draw[->] (solver) -- (0,2) -- (0,3) -- (implementation);
%% \draw[->] (solver) -- (solution);
%% \end{tikzpicture}
%% \end{center}
%% \end{frame}

%\begin{frame}
%\frametitle{What is CP actually used for?}
%\url{http://hsimonis.wordpress.com}
%\includegraphics[width=10cm]{../introduction/images/blog}
%\end{frame}

%% \begin{frame}
%% \frametitle{Exercises}
%% \begin{itemize}
%% \item Install Constraint Solver of your Choice
%% \item MiniZinc (MiningZinc) will be used in later sessions
%% \item Run Sendmoremoney, Sudoku, Queens examples
%% \item Can you choose which propagator to use for alldifferent?
%% \item Can you use different search strategies?
%% \end{itemize}
%% \end{frame}

\begin{frame}
\frametitle{More Learning Resources}
\begin{itemize}
\item Survey of Methods, Resources, and Formats for Teaching Constraint Programming
\begin{itemize}
\item by Tejas Santanam, Helmut Simonis
\item \url{https://doi.org/10.48550/arXiv.2403.12717}
\item Based on survey of community for WTCP 2023
\item \url{https://hsimonis.github.io/WTCP2023/}
\end{itemize}
\end{itemize}
\end{frame}


%%   }

%% \clearpage
%% \mode<all>{
%%   \part{Methodology}
%% }
%% \begin{frame}
%%   \partpage
%% \end{frame}
%% \mode<all>{
%%   \begin{frame}
\frametitle{Objectives}
\begin{itemize}
\item Overview of Core Constraint Programming
\item Three Main Concepts
\begin{itemize}
\item Constraint Propagation
\item Global Constraints
\item Customizing Search
\end{itemize}
%\item Get Some Experience with MiniZinc
\item Topics will be treated in more detail in later parts of the school
\item Based on Examples, not Formal Description
\end{itemize}
\end{frame}

\begin{frame}
\frametitle{Outline}
\begin{itemize}
\item Why Constraint Programming?
\item Constraint Propagation
\item Global Constraints
\item Customizing Search
%\item Choices in Modelling
%\item Decomposition
%\item More Global Constraints
%\item Methodology
%\item What is missing?
\end{itemize}
\end{frame}

% \begin{frame}
  % \frametitle{Using MiniZinc IDE}
  % \begin{itemize}
  % \item Developed in the Australian NICTA project
  % \item Maintained by Monash University
  % \item Modelling tool with multiple back-end solvers
  % \item Available from \url{https://www.minizinc.org/}
  % \end{itemize}
% \end{frame}


% \begin{frame}
% \frametitle{Examples in ECLiPSe}
% \begin{itemize}
% \item Open sourced constraint programming language
% \item Development goes back to 1985
% \item ECRC, ICL, IC-Parc, PTL, Cisco
% \item \url{https://eclipseclp.org/}
% \item Specialities
% \begin{itemize}
% \item Develop new solvers for specific domains
% \item Integration with MIP
% \end{itemize}
% \item Not included in bundled MiniZinc IDE
% \item Specialized visualization tools used here
  % \begin{itemize}
  % \item CP-Viz, Simonis et al. 2010
    % \end{itemize}
% \end{itemize}
% \end{frame}

\begin{frame}
\frametitle{Tutorial Based on ECLiPSe ELearning Course}
\begin{itemize}
\item Self-study course in constraint programming
\item Supported by Cisco Systems and Silicon Valley Community Foundation
\item Multi-media format, video lectures, slides, handout etc
\item {\small \url{https://eclipseclp.org/ELearning/index.html}}
\end{itemize}
\end{frame}

\begin{frame}
\frametitle{Also Part of CRT-AI Constraint Week}
\begin{itemize}
\item Annual one week course on CP and Optimization in Ireland
\item Part of national training program for PhD students in AI
\item \url{https://www.crt-ai.ie/}
\end{itemize}
\end{frame}



\begin{frame}
\frametitle{Constraint Programming - in a nutshell}
\begin{itemize}
\item Declarative description of problems with
\begin{itemize}
\item {\em Variables} which range over (finite) sets of values
\item {\em Constraints} over subsets of variables which restrict possible value combinations
\item A {\em solution} is a value assignment which satisfies all constraints
\end{itemize}

\item Constraint propagation/reasoning
\begin{itemize}
\item Removing inconsistent values for variables
\item Detect failure if constraint can not be satisfied
\item Interaction of constraints via shared variables
\item Incomplete
\end{itemize}

\item Search
\begin{itemize}
\item User controlled assignment of values to variables
\item Each step triggers constraint propagation 
\end{itemize}
\item Different domains require/allow different methods
\end{itemize}
\end{frame}

\begin{frame}
  \frametitle{Constraint Programming is Different}
  \begin{itemize}
  \item Declarative Programming
    \begin{itemize}
    \item Concentrate on what you want
      \item Not how to get there
      \item Program != Algorithm
      \item Program = Model
    \end{itemize}
    \item Applied to Combinatorial Problems
      \begin{itemize}
        \item No complete polynomial algorithms known (exist?)
        \item CP less ad-hoc than heuristics
        \item Models can evolve
  \end{itemize}
  \end{itemize}
  \end{frame}
    
\begin{frame}
  \frametitle{A Subtractive Process}
  \begin{textblock}{4}(8,-3)
    \includegraphics[width=4cm]{../introduction/images/stages}
  \end{textblock}
  \vfill
  \begin{quote}
    ``Oh, bosh, as Mr. Ruskin says. Sculpture, per se, is the simplest thing in the world. All you have to do is to take a big chunk of marble and a hammer and chisel, make up your mind what you are about to create and chip off all the marble you don't want.''-Paris Gaulois.
  \end{quote}
  
  {\tiny Source: \url{https://quoteinvestigator.com/2014/06/22/chip-away/}}
\end{frame}

\begin{frame}
\frametitle{Basic Process}
\begin{center}
\begin{tikzpicture}
\node[shape=rectangle,fill=pantone127-4] (problem) at (2,5) {Problem};
\node (human) at (2,4) {Human};
\node[shape=rectangle,fill=pantone127-4] (model) at (2,3) {Model};
\node (solver) at (2,2) {Constraint Solver/Search};
\node[shape=rectangle,fill=pantone157-8] (solution) at (2,1) {Solution};
\draw[-] (problem) -- (human);
\draw[->] (human) -- (model);
\draw[-] (model) -- (solver);
\draw[->] (solver) -- (solution);
\end{tikzpicture}
\end{center}
\end{frame}

\begin{frame}
\frametitle{More Realistic}
\begin{center}
\begin{tikzpicture}[xscale=4.0]
\node[shape=rectangle,fill=pantone127-4] (problem) at (2,5) {Problem};
\node (human) at (2,4) {Human};
\node[shape=rectangle,fill=pantone127-4] (model) at (2,3) {Model};
\node (solver) at (2,2) {Constraint Solver/Search};
\node[shape=rectangle,fill=pantone24-8] (hangs) at (1,1) {Hangs};
\node[shape=rectangle,fill=pantone157-8] (solution) at (2,1) {Solution};
\node[shape=rectangle,fill=pantone24-8] (wrong) at (3,1) {Wrong Solution};
\draw[-] (problem) -- (human);
\draw[->] (human) -- (model);
\draw[-] (model) -- (solver);
\draw[->] (solver) -- (hangs);
\draw[->] (solver) -- (solution);
\draw[->] (solver) -- (wrong);
\draw[->] (hangs) -- (1,4) -- (human);
\draw[->] (wrong) -- (3,4) -- (human);
\end{tikzpicture}
\end{center}
\end{frame}

\begin{frame}
\frametitle{Dual Role of Model}
\begin{itemize}
\item Allows Human to Express Problem
\begin{itemize}
\item Close to Problem Domain
\item Constraints as Abstractions
\end{itemize}
\item Allows Solver to Execute
\begin{itemize}
\item Variables as Communication Mechanism
\item Constraints as Algorithms
\end{itemize}
\end{itemize}
\end{frame}

\begin{frame}
\frametitle{Modelling Frameworks}
\begin{itemize}
\item \href{https://www.minizinc.org/}{MiniZinc} (NICTA, Monash University, Australia)
\item \href{https://github.com/eomahony/Numberjack}{NumberJack} (Insight, Ireland) 
\item \href{https://www-users.york.ac.uk/peter.nightingale/savilerow//}{EssencePrime/SavilleRow} (UK) 
\item \href{https://cpmpy.readthedocs.io/en/latest/}{CPMpy} (KU Leuven)
\item Allow use of multiple back-end solvers
\item Compile model into variants for each solver
\item A priori solver independent model(CP, MIP, SAT)
\end{itemize}
\end{frame} 

\begin{frame}
\frametitle{Framework Process}
\begin{center}
\begin{tikzpicture}[xscale=2]
\node[shape=rectangle,fill=pantone127-4] (problem) at (2.5,5) {Problem};
\node (human) at (2.5,4) {Human};
\node[shape=rectangle,fill=pantone127-4] (model) at (2.5,3) {Model};
\node (compile) at (2.5,2) {Compile/Reformulate};
\node (solver1) at (1,1) {CP};
\node (solver2) at (2,1) {MIP};
\node (solver3) at (3,1) {SAT};
\node (solver4) at (4,1) {Other};
\node[shape=rectangle,fill=pantone157-8] (solution1) at (1,0) {Solution};
\node[shape=rectangle,fill=pantone157-8] (solution2) at (2,0) {Solution};
\node[shape=rectangle,fill=pantone157-8] (solution3) at (3,0) {Solution};
\node[shape=rectangle,fill=pantone157-8] (solution4) at (4,0) {Solution};
\draw[-] (problem) -- (human);
\draw[->] (human) -- (model);
\draw[-] (model) -- (compile);
\draw[-] (compile) -- (solver1);
\draw[-] (compile) -- (solver2);
\draw[-] (compile) -- (solver3);
\draw[-] (compile) -- (solver4);
\draw[->] (solver1) -- (solution1);
\draw[->] (solver2) -- (solution2);
\draw[->] (solver3) -- (solution3);
\draw[->] (solver4) -- (solution4);
\end{tikzpicture}
\end{center}
\end{frame}

% \begin{frame}
  % \frametitle{Do It Now!}
  % \begin{itemize}
  % \item Download and install Minizinc
  % \item \url{https://www.minizinc.org/}
  % \end{itemize}
% \end{frame}


\begin{frame}
\frametitle{Why use Puzzles as Examples?}
\begin{itemize}
\item Easy to understand the problem
\item Solvable by hand without specialized knowledge
\item Possible to compare automated to manual solving process
\end{itemize}

\begin{quote}
The puzzle, though inanimate, is presented as a solvable problem without lasting negative consequences, a very low-risk low-reward situation. By being a puzzle, the object is attempting to convince the user that it must be completed.
\end{quote}
{\tiny Source: \href{https://www.everydayrhetoric.com/2020/03/the-rhetoric-of-puzzles/}{Every Day Rhetoric}}
\end{frame}


\mode<all>{
\part{Basic Constraint Propagation}
}
\begin{frame}
\partpage
\end{frame}
\mode<all>{
\input{../sendmore/body}
}

\mode<all>{
\part{Global Constraints}
}
\begin{frame}
\partpage
\end{frame}
\mode<all>{
\input{../sudoku/body}
}

\clearpage
\mode<all>{
\part{Customizing Search}
}
\begin{frame}
\partpage
\end{frame}
\mode<all>{
\input{../nqueen/body}
}


%% \clearpage
%% \mode<all>{
%%   \part{Problem Decomposition}
%% }
%% \begin{frame}
%%   \partpage
%% \end{frame}
%% \mode<all>{
%%   \input{../party/body}
%%   }

%% \clearpage
%% \mode<all>{
%%   \part{More Global Constraints}
%% }
%% \begin{frame}
%%   \partpage
%% \end{frame}
%% \mode<all>{
%%   \input{../car/body}
%%   }

%% \clearpage
%% \mode<all>{
%%   \part{Methodology}
%% }
%% \begin{frame}
%%   \partpage
%% \end{frame}
%% \mode<all>{
%%   \input{../methodology/body}
%%   }

\clearpage
\mode<all>{
\part{What is missing?}
}
\begin{frame}
\partpage
\end{frame}

\begin{frame}
\frametitle{Many Specialized Topics}
\begin{itemize}
\item How to design efficient core engine
\item Hybrids with LP/MIP tools
\item Hybrids with SAT
\item Symmetry breaking
\item Use of MDD/BDD to encode sets of solutions
\item High level modelling tools
\item Debugging/visualization
\end{itemize}
\end{frame}

\begin{frame}
\frametitle{Reformulation}
\begin{itemize}
\item Just because the user has modelled it this way, it doesn't mean we have to solve it that way
\begin{itemize}
\item Replace some constraint(s) by other, equivalent constraints
\item Because we don't have that constraint in our system
\item For performance
\end{itemize}
\end{itemize}
\end{frame}

\begin{frame}
\frametitle{Learning}
\begin{itemize}
\item While solving the problem we can learn how to strengthen the model/search 
\begin{itemize}
\item Understand which constraints/method contribute to propagation and change schedule
\item Learn no-good constraints by explaining failure
\item Adapt search strategy based on search experience
\end{itemize}
\end{itemize}
\end{frame}

%% \begin{frame}
%% \frametitle{Refined Process}
%% \begin{center}
%% \begin{tikzpicture}
%% \node[shape=rectangle,fill=pantone127-4] (problem) at (3,7) {Problem};
%% \node (human) at (3,6) {Human};
%% \node[shape=rectangle,fill=pantone127-4] (model) at (3,5) {User Model};
%% \node (reformulation) at (3,4) {Reformulation};
%% \node[shape=rectangle,fill=pantone127-4] (implementation) at (3,3) {Implementation Model};
%% \node (solver) at (3,2) {Constraint Solver/Search};
%% \node[shape=rectangle,fill=pantone157-8] (solution) at (3,1) {Solution};
%% \draw[-] (problem) -- (human);
%% \draw[->] (human) -- (model);
%% \draw[-] (model) -- (reformulation);
%% \draw[->] (reformulation) -- (implementation);
%% \draw[-] (implementation) -- (solver);
%% \draw[->] (solver) -- (0,2) -- (0,3) -- (implementation);
%% \draw[->] (solver) -- (solution);
%% \end{tikzpicture}
%% \end{center}
%% \end{frame}

%\begin{frame}
%\frametitle{What is CP actually used for?}
%\url{http://hsimonis.wordpress.com}
%\includegraphics[width=10cm]{../introduction/images/blog}
%\end{frame}

%% \begin{frame}
%% \frametitle{Exercises}
%% \begin{itemize}
%% \item Install Constraint Solver of your Choice
%% \item MiniZinc (MiningZinc) will be used in later sessions
%% \item Run Sendmoremoney, Sudoku, Queens examples
%% \item Can you choose which propagator to use for alldifferent?
%% \item Can you use different search strategies?
%% \end{itemize}
%% \end{frame}

\begin{frame}
\frametitle{More Learning Resources}
\begin{itemize}
\item Survey of Methods, Resources, and Formats for Teaching Constraint Programming
\begin{itemize}
\item by Tejas Santanam, Helmut Simonis
\item \url{https://doi.org/10.48550/arXiv.2403.12717}
\item Based on survey of community for WTCP 2023
\item \url{https://hsimonis.github.io/WTCP2023/}
\end{itemize}
\end{itemize}
\end{frame}


%%   }

\clearpage
\mode<all>{
\part{What is missing?}
}
\begin{frame}
\partpage
\end{frame}

\begin{frame}
\frametitle{Many Specialized Topics}
\begin{itemize}
\item How to design efficient core engine
\item Hybrids with LP/MIP tools
\item Hybrids with SAT
\item Symmetry breaking
\item Use of MDD/BDD to encode sets of solutions
\item High level modelling tools
\item Debugging/visualization
\end{itemize}
\end{frame}

\begin{frame}
\frametitle{Reformulation}
\begin{itemize}
\item Just because the user has modelled it this way, it doesn't mean we have to solve it that way
\begin{itemize}
\item Replace some constraint(s) by other, equivalent constraints
\item Because we don't have that constraint in our system
\item For performance
\end{itemize}
\end{itemize}
\end{frame}

\begin{frame}
\frametitle{Learning}
\begin{itemize}
\item While solving the problem we can learn how to strengthen the model/search 
\begin{itemize}
\item Understand which constraints/method contribute to propagation and change schedule
\item Learn no-good constraints by explaining failure
\item Adapt search strategy based on search experience
\end{itemize}
\end{itemize}
\end{frame}

%% \begin{frame}
%% \frametitle{Refined Process}
%% \begin{center}
%% \begin{tikzpicture}
%% \node[shape=rectangle,fill=pantone127-4] (problem) at (3,7) {Problem};
%% \node (human) at (3,6) {Human};
%% \node[shape=rectangle,fill=pantone127-4] (model) at (3,5) {User Model};
%% \node (reformulation) at (3,4) {Reformulation};
%% \node[shape=rectangle,fill=pantone127-4] (implementation) at (3,3) {Implementation Model};
%% \node (solver) at (3,2) {Constraint Solver/Search};
%% \node[shape=rectangle,fill=pantone157-8] (solution) at (3,1) {Solution};
%% \draw[-] (problem) -- (human);
%% \draw[->] (human) -- (model);
%% \draw[-] (model) -- (reformulation);
%% \draw[->] (reformulation) -- (implementation);
%% \draw[-] (implementation) -- (solver);
%% \draw[->] (solver) -- (0,2) -- (0,3) -- (implementation);
%% \draw[->] (solver) -- (solution);
%% \end{tikzpicture}
%% \end{center}
%% \end{frame}

%\begin{frame}
%\frametitle{What is CP actually used for?}
%\url{http://hsimonis.wordpress.com}
%\includegraphics[width=10cm]{../introduction/images/blog}
%\end{frame}

%% \begin{frame}
%% \frametitle{Exercises}
%% \begin{itemize}
%% \item Install Constraint Solver of your Choice
%% \item MiniZinc (MiningZinc) will be used in later sessions
%% \item Run Sendmoremoney, Sudoku, Queens examples
%% \item Can you choose which propagator to use for alldifferent?
%% \item Can you use different search strategies?
%% \end{itemize}
%% \end{frame}

\begin{frame}
\frametitle{More Learning Resources}
\begin{itemize}
\item Survey of Methods, Resources, and Formats for Teaching Constraint Programming
\begin{itemize}
\item by Tejas Santanam, Helmut Simonis
\item \url{https://doi.org/10.48550/arXiv.2403.12717}
\item Based on survey of community for WTCP 2023
\item \url{https://hsimonis.github.io/WTCP2023/}
\end{itemize}
\end{itemize}
\end{frame}


%%   }

\clearpage
\mode<all>{
\part{What is missing?}
}
\begin{frame}
\partpage
\end{frame}

\begin{frame}
\frametitle{Many Specialized Topics}
\begin{itemize}
\item How to design efficient core engine
\item Hybrids with LP/MIP tools
\item Hybrids with SAT
\item Symmetry breaking
\item Use of MDD/BDD to encode sets of solutions
\item High level modelling tools
\item Debugging/visualization
\end{itemize}
\end{frame}

\begin{frame}
\frametitle{Reformulation}
\begin{itemize}
\item Just because the user has modelled it this way, it doesn't mean we have to solve it that way
\begin{itemize}
\item Replace some constraint(s) by other, equivalent constraints
\item Because we don't have that constraint in our system
\item For performance
\end{itemize}
\end{itemize}
\end{frame}

\begin{frame}
\frametitle{Learning}
\begin{itemize}
\item While solving the problem we can learn how to strengthen the model/search 
\begin{itemize}
\item Understand which constraints/method contribute to propagation and change schedule
\item Learn no-good constraints by explaining failure
\item Adapt search strategy based on search experience
\end{itemize}
\end{itemize}
\end{frame}

%% \begin{frame}
%% \frametitle{Refined Process}
%% \begin{center}
%% \begin{tikzpicture}
%% \node[shape=rectangle,fill=pantone127-4] (problem) at (3,7) {Problem};
%% \node (human) at (3,6) {Human};
%% \node[shape=rectangle,fill=pantone127-4] (model) at (3,5) {User Model};
%% \node (reformulation) at (3,4) {Reformulation};
%% \node[shape=rectangle,fill=pantone127-4] (implementation) at (3,3) {Implementation Model};
%% \node (solver) at (3,2) {Constraint Solver/Search};
%% \node[shape=rectangle,fill=pantone157-8] (solution) at (3,1) {Solution};
%% \draw[-] (problem) -- (human);
%% \draw[->] (human) -- (model);
%% \draw[-] (model) -- (reformulation);
%% \draw[->] (reformulation) -- (implementation);
%% \draw[-] (implementation) -- (solver);
%% \draw[->] (solver) -- (0,2) -- (0,3) -- (implementation);
%% \draw[->] (solver) -- (solution);
%% \end{tikzpicture}
%% \end{center}
%% \end{frame}

%\begin{frame}
%\frametitle{What is CP actually used for?}
%\url{http://hsimonis.wordpress.com}
%\includegraphics[width=10cm]{../introduction/images/blog}
%\end{frame}

%% \begin{frame}
%% \frametitle{Exercises}
%% \begin{itemize}
%% \item Install Constraint Solver of your Choice
%% \item MiniZinc (MiningZinc) will be used in later sessions
%% \item Run Sendmoremoney, Sudoku, Queens examples
%% \item Can you choose which propagator to use for alldifferent?
%% \item Can you use different search strategies?
%% \end{itemize}
%% \end{frame}

\begin{frame}
\frametitle{More Learning Resources}
\begin{itemize}
\item Survey of Methods, Resources, and Formats for Teaching Constraint Programming
\begin{itemize}
\item by Tejas Santanam, Helmut Simonis
\item \url{https://doi.org/10.48550/arXiv.2403.12717}
\item Based on survey of community for WTCP 2023
\item \url{https://hsimonis.github.io/WTCP2023/}
\end{itemize}
\end{itemize}
\end{frame}


}


%% \clearpage
%% \mode<all>{
%%   \part{Problem Decomposition}
%% }
%% \begin{frame}
%%   \partpage
%% \end{frame}
%% \mode<all>{
%%   \begin{frame}
\frametitle{Objectives}
\begin{itemize}
\item Overview of Core Constraint Programming
\item Three Main Concepts
\begin{itemize}
\item Constraint Propagation
\item Global Constraints
\item Customizing Search
\end{itemize}
%\item Get Some Experience with MiniZinc
\item Topics will be treated in more detail in later parts of the school
\item Based on Examples, not Formal Description
\end{itemize}
\end{frame}

\begin{frame}
\frametitle{Outline}
\begin{itemize}
\item Why Constraint Programming?
\item Constraint Propagation
\item Global Constraints
\item Customizing Search
%\item Choices in Modelling
%\item Decomposition
%\item More Global Constraints
%\item Methodology
%\item What is missing?
\end{itemize}
\end{frame}

% \begin{frame}
  % \frametitle{Using MiniZinc IDE}
  % \begin{itemize}
  % \item Developed in the Australian NICTA project
  % \item Maintained by Monash University
  % \item Modelling tool with multiple back-end solvers
  % \item Available from \url{https://www.minizinc.org/}
  % \end{itemize}
% \end{frame}


% \begin{frame}
% \frametitle{Examples in ECLiPSe}
% \begin{itemize}
% \item Open sourced constraint programming language
% \item Development goes back to 1985
% \item ECRC, ICL, IC-Parc, PTL, Cisco
% \item \url{https://eclipseclp.org/}
% \item Specialities
% \begin{itemize}
% \item Develop new solvers for specific domains
% \item Integration with MIP
% \end{itemize}
% \item Not included in bundled MiniZinc IDE
% \item Specialized visualization tools used here
  % \begin{itemize}
  % \item CP-Viz, Simonis et al. 2010
    % \end{itemize}
% \end{itemize}
% \end{frame}

\begin{frame}
\frametitle{Tutorial Based on ECLiPSe ELearning Course}
\begin{itemize}
\item Self-study course in constraint programming
\item Supported by Cisco Systems and Silicon Valley Community Foundation
\item Multi-media format, video lectures, slides, handout etc
\item {\small \url{https://eclipseclp.org/ELearning/index.html}}
\end{itemize}
\end{frame}

\begin{frame}
\frametitle{Also Part of CRT-AI Constraint Week}
\begin{itemize}
\item Annual one week course on CP and Optimization in Ireland
\item Part of national training program for PhD students in AI
\item \url{https://www.crt-ai.ie/}
\end{itemize}
\end{frame}



\begin{frame}
\frametitle{Constraint Programming - in a nutshell}
\begin{itemize}
\item Declarative description of problems with
\begin{itemize}
\item {\em Variables} which range over (finite) sets of values
\item {\em Constraints} over subsets of variables which restrict possible value combinations
\item A {\em solution} is a value assignment which satisfies all constraints
\end{itemize}

\item Constraint propagation/reasoning
\begin{itemize}
\item Removing inconsistent values for variables
\item Detect failure if constraint can not be satisfied
\item Interaction of constraints via shared variables
\item Incomplete
\end{itemize}

\item Search
\begin{itemize}
\item User controlled assignment of values to variables
\item Each step triggers constraint propagation 
\end{itemize}
\item Different domains require/allow different methods
\end{itemize}
\end{frame}

\begin{frame}
  \frametitle{Constraint Programming is Different}
  \begin{itemize}
  \item Declarative Programming
    \begin{itemize}
    \item Concentrate on what you want
      \item Not how to get there
      \item Program != Algorithm
      \item Program = Model
    \end{itemize}
    \item Applied to Combinatorial Problems
      \begin{itemize}
        \item No complete polynomial algorithms known (exist?)
        \item CP less ad-hoc than heuristics
        \item Models can evolve
  \end{itemize}
  \end{itemize}
  \end{frame}
    
\begin{frame}
  \frametitle{A Subtractive Process}
  \begin{textblock}{4}(8,-3)
    \includegraphics[width=4cm]{../introduction/images/stages}
  \end{textblock}
  \vfill
  \begin{quote}
    ``Oh, bosh, as Mr. Ruskin says. Sculpture, per se, is the simplest thing in the world. All you have to do is to take a big chunk of marble and a hammer and chisel, make up your mind what you are about to create and chip off all the marble you don't want.''-Paris Gaulois.
  \end{quote}
  
  {\tiny Source: \url{https://quoteinvestigator.com/2014/06/22/chip-away/}}
\end{frame}

\begin{frame}
\frametitle{Basic Process}
\begin{center}
\begin{tikzpicture}
\node[shape=rectangle,fill=pantone127-4] (problem) at (2,5) {Problem};
\node (human) at (2,4) {Human};
\node[shape=rectangle,fill=pantone127-4] (model) at (2,3) {Model};
\node (solver) at (2,2) {Constraint Solver/Search};
\node[shape=rectangle,fill=pantone157-8] (solution) at (2,1) {Solution};
\draw[-] (problem) -- (human);
\draw[->] (human) -- (model);
\draw[-] (model) -- (solver);
\draw[->] (solver) -- (solution);
\end{tikzpicture}
\end{center}
\end{frame}

\begin{frame}
\frametitle{More Realistic}
\begin{center}
\begin{tikzpicture}[xscale=4.0]
\node[shape=rectangle,fill=pantone127-4] (problem) at (2,5) {Problem};
\node (human) at (2,4) {Human};
\node[shape=rectangle,fill=pantone127-4] (model) at (2,3) {Model};
\node (solver) at (2,2) {Constraint Solver/Search};
\node[shape=rectangle,fill=pantone24-8] (hangs) at (1,1) {Hangs};
\node[shape=rectangle,fill=pantone157-8] (solution) at (2,1) {Solution};
\node[shape=rectangle,fill=pantone24-8] (wrong) at (3,1) {Wrong Solution};
\draw[-] (problem) -- (human);
\draw[->] (human) -- (model);
\draw[-] (model) -- (solver);
\draw[->] (solver) -- (hangs);
\draw[->] (solver) -- (solution);
\draw[->] (solver) -- (wrong);
\draw[->] (hangs) -- (1,4) -- (human);
\draw[->] (wrong) -- (3,4) -- (human);
\end{tikzpicture}
\end{center}
\end{frame}

\begin{frame}
\frametitle{Dual Role of Model}
\begin{itemize}
\item Allows Human to Express Problem
\begin{itemize}
\item Close to Problem Domain
\item Constraints as Abstractions
\end{itemize}
\item Allows Solver to Execute
\begin{itemize}
\item Variables as Communication Mechanism
\item Constraints as Algorithms
\end{itemize}
\end{itemize}
\end{frame}

\begin{frame}
\frametitle{Modelling Frameworks}
\begin{itemize}
\item \href{https://www.minizinc.org/}{MiniZinc} (NICTA, Monash University, Australia)
\item \href{https://github.com/eomahony/Numberjack}{NumberJack} (Insight, Ireland) 
\item \href{https://www-users.york.ac.uk/peter.nightingale/savilerow//}{EssencePrime/SavilleRow} (UK) 
\item \href{https://cpmpy.readthedocs.io/en/latest/}{CPMpy} (KU Leuven)
\item Allow use of multiple back-end solvers
\item Compile model into variants for each solver
\item A priori solver independent model(CP, MIP, SAT)
\end{itemize}
\end{frame} 

\begin{frame}
\frametitle{Framework Process}
\begin{center}
\begin{tikzpicture}[xscale=2]
\node[shape=rectangle,fill=pantone127-4] (problem) at (2.5,5) {Problem};
\node (human) at (2.5,4) {Human};
\node[shape=rectangle,fill=pantone127-4] (model) at (2.5,3) {Model};
\node (compile) at (2.5,2) {Compile/Reformulate};
\node (solver1) at (1,1) {CP};
\node (solver2) at (2,1) {MIP};
\node (solver3) at (3,1) {SAT};
\node (solver4) at (4,1) {Other};
\node[shape=rectangle,fill=pantone157-8] (solution1) at (1,0) {Solution};
\node[shape=rectangle,fill=pantone157-8] (solution2) at (2,0) {Solution};
\node[shape=rectangle,fill=pantone157-8] (solution3) at (3,0) {Solution};
\node[shape=rectangle,fill=pantone157-8] (solution4) at (4,0) {Solution};
\draw[-] (problem) -- (human);
\draw[->] (human) -- (model);
\draw[-] (model) -- (compile);
\draw[-] (compile) -- (solver1);
\draw[-] (compile) -- (solver2);
\draw[-] (compile) -- (solver3);
\draw[-] (compile) -- (solver4);
\draw[->] (solver1) -- (solution1);
\draw[->] (solver2) -- (solution2);
\draw[->] (solver3) -- (solution3);
\draw[->] (solver4) -- (solution4);
\end{tikzpicture}
\end{center}
\end{frame}

% \begin{frame}
  % \frametitle{Do It Now!}
  % \begin{itemize}
  % \item Download and install Minizinc
  % \item \url{https://www.minizinc.org/}
  % \end{itemize}
% \end{frame}


\begin{frame}
\frametitle{Why use Puzzles as Examples?}
\begin{itemize}
\item Easy to understand the problem
\item Solvable by hand without specialized knowledge
\item Possible to compare automated to manual solving process
\end{itemize}

\begin{quote}
The puzzle, though inanimate, is presented as a solvable problem without lasting negative consequences, a very low-risk low-reward situation. By being a puzzle, the object is attempting to convince the user that it must be completed.
\end{quote}
{\tiny Source: \href{https://www.everydayrhetoric.com/2020/03/the-rhetoric-of-puzzles/}{Every Day Rhetoric}}
\end{frame}


\mode<all>{
\part{Basic Constraint Propagation}
}
\begin{frame}
\partpage
\end{frame}
\mode<all>{
\begin{frame}
\frametitle{Objectives}
\begin{itemize}
\item Overview of Core Constraint Programming
\item Three Main Concepts
\begin{itemize}
\item Constraint Propagation
\item Global Constraints
\item Customizing Search
\end{itemize}
%\item Get Some Experience with MiniZinc
\item Topics will be treated in more detail in later parts of the school
\item Based on Examples, not Formal Description
\end{itemize}
\end{frame}

\begin{frame}
\frametitle{Outline}
\begin{itemize}
\item Why Constraint Programming?
\item Constraint Propagation
\item Global Constraints
\item Customizing Search
%\item Choices in Modelling
%\item Decomposition
%\item More Global Constraints
%\item Methodology
%\item What is missing?
\end{itemize}
\end{frame}

% \begin{frame}
  % \frametitle{Using MiniZinc IDE}
  % \begin{itemize}
  % \item Developed in the Australian NICTA project
  % \item Maintained by Monash University
  % \item Modelling tool with multiple back-end solvers
  % \item Available from \url{https://www.minizinc.org/}
  % \end{itemize}
% \end{frame}


% \begin{frame}
% \frametitle{Examples in ECLiPSe}
% \begin{itemize}
% \item Open sourced constraint programming language
% \item Development goes back to 1985
% \item ECRC, ICL, IC-Parc, PTL, Cisco
% \item \url{https://eclipseclp.org/}
% \item Specialities
% \begin{itemize}
% \item Develop new solvers for specific domains
% \item Integration with MIP
% \end{itemize}
% \item Not included in bundled MiniZinc IDE
% \item Specialized visualization tools used here
  % \begin{itemize}
  % \item CP-Viz, Simonis et al. 2010
    % \end{itemize}
% \end{itemize}
% \end{frame}

\begin{frame}
\frametitle{Tutorial Based on ECLiPSe ELearning Course}
\begin{itemize}
\item Self-study course in constraint programming
\item Supported by Cisco Systems and Silicon Valley Community Foundation
\item Multi-media format, video lectures, slides, handout etc
\item {\small \url{https://eclipseclp.org/ELearning/index.html}}
\end{itemize}
\end{frame}

\begin{frame}
\frametitle{Also Part of CRT-AI Constraint Week}
\begin{itemize}
\item Annual one week course on CP and Optimization in Ireland
\item Part of national training program for PhD students in AI
\item \url{https://www.crt-ai.ie/}
\end{itemize}
\end{frame}



\begin{frame}
\frametitle{Constraint Programming - in a nutshell}
\begin{itemize}
\item Declarative description of problems with
\begin{itemize}
\item {\em Variables} which range over (finite) sets of values
\item {\em Constraints} over subsets of variables which restrict possible value combinations
\item A {\em solution} is a value assignment which satisfies all constraints
\end{itemize}

\item Constraint propagation/reasoning
\begin{itemize}
\item Removing inconsistent values for variables
\item Detect failure if constraint can not be satisfied
\item Interaction of constraints via shared variables
\item Incomplete
\end{itemize}

\item Search
\begin{itemize}
\item User controlled assignment of values to variables
\item Each step triggers constraint propagation 
\end{itemize}
\item Different domains require/allow different methods
\end{itemize}
\end{frame}

\begin{frame}
  \frametitle{Constraint Programming is Different}
  \begin{itemize}
  \item Declarative Programming
    \begin{itemize}
    \item Concentrate on what you want
      \item Not how to get there
      \item Program != Algorithm
      \item Program = Model
    \end{itemize}
    \item Applied to Combinatorial Problems
      \begin{itemize}
        \item No complete polynomial algorithms known (exist?)
        \item CP less ad-hoc than heuristics
        \item Models can evolve
  \end{itemize}
  \end{itemize}
  \end{frame}
    
\begin{frame}
  \frametitle{A Subtractive Process}
  \begin{textblock}{4}(8,-3)
    \includegraphics[width=4cm]{../introduction/images/stages}
  \end{textblock}
  \vfill
  \begin{quote}
    ``Oh, bosh, as Mr. Ruskin says. Sculpture, per se, is the simplest thing in the world. All you have to do is to take a big chunk of marble and a hammer and chisel, make up your mind what you are about to create and chip off all the marble you don't want.''-Paris Gaulois.
  \end{quote}
  
  {\tiny Source: \url{https://quoteinvestigator.com/2014/06/22/chip-away/}}
\end{frame}

\begin{frame}
\frametitle{Basic Process}
\begin{center}
\begin{tikzpicture}
\node[shape=rectangle,fill=pantone127-4] (problem) at (2,5) {Problem};
\node (human) at (2,4) {Human};
\node[shape=rectangle,fill=pantone127-4] (model) at (2,3) {Model};
\node (solver) at (2,2) {Constraint Solver/Search};
\node[shape=rectangle,fill=pantone157-8] (solution) at (2,1) {Solution};
\draw[-] (problem) -- (human);
\draw[->] (human) -- (model);
\draw[-] (model) -- (solver);
\draw[->] (solver) -- (solution);
\end{tikzpicture}
\end{center}
\end{frame}

\begin{frame}
\frametitle{More Realistic}
\begin{center}
\begin{tikzpicture}[xscale=4.0]
\node[shape=rectangle,fill=pantone127-4] (problem) at (2,5) {Problem};
\node (human) at (2,4) {Human};
\node[shape=rectangle,fill=pantone127-4] (model) at (2,3) {Model};
\node (solver) at (2,2) {Constraint Solver/Search};
\node[shape=rectangle,fill=pantone24-8] (hangs) at (1,1) {Hangs};
\node[shape=rectangle,fill=pantone157-8] (solution) at (2,1) {Solution};
\node[shape=rectangle,fill=pantone24-8] (wrong) at (3,1) {Wrong Solution};
\draw[-] (problem) -- (human);
\draw[->] (human) -- (model);
\draw[-] (model) -- (solver);
\draw[->] (solver) -- (hangs);
\draw[->] (solver) -- (solution);
\draw[->] (solver) -- (wrong);
\draw[->] (hangs) -- (1,4) -- (human);
\draw[->] (wrong) -- (3,4) -- (human);
\end{tikzpicture}
\end{center}
\end{frame}

\begin{frame}
\frametitle{Dual Role of Model}
\begin{itemize}
\item Allows Human to Express Problem
\begin{itemize}
\item Close to Problem Domain
\item Constraints as Abstractions
\end{itemize}
\item Allows Solver to Execute
\begin{itemize}
\item Variables as Communication Mechanism
\item Constraints as Algorithms
\end{itemize}
\end{itemize}
\end{frame}

\begin{frame}
\frametitle{Modelling Frameworks}
\begin{itemize}
\item \href{https://www.minizinc.org/}{MiniZinc} (NICTA, Monash University, Australia)
\item \href{https://github.com/eomahony/Numberjack}{NumberJack} (Insight, Ireland) 
\item \href{https://www-users.york.ac.uk/peter.nightingale/savilerow//}{EssencePrime/SavilleRow} (UK) 
\item \href{https://cpmpy.readthedocs.io/en/latest/}{CPMpy} (KU Leuven)
\item Allow use of multiple back-end solvers
\item Compile model into variants for each solver
\item A priori solver independent model(CP, MIP, SAT)
\end{itemize}
\end{frame} 

\begin{frame}
\frametitle{Framework Process}
\begin{center}
\begin{tikzpicture}[xscale=2]
\node[shape=rectangle,fill=pantone127-4] (problem) at (2.5,5) {Problem};
\node (human) at (2.5,4) {Human};
\node[shape=rectangle,fill=pantone127-4] (model) at (2.5,3) {Model};
\node (compile) at (2.5,2) {Compile/Reformulate};
\node (solver1) at (1,1) {CP};
\node (solver2) at (2,1) {MIP};
\node (solver3) at (3,1) {SAT};
\node (solver4) at (4,1) {Other};
\node[shape=rectangle,fill=pantone157-8] (solution1) at (1,0) {Solution};
\node[shape=rectangle,fill=pantone157-8] (solution2) at (2,0) {Solution};
\node[shape=rectangle,fill=pantone157-8] (solution3) at (3,0) {Solution};
\node[shape=rectangle,fill=pantone157-8] (solution4) at (4,0) {Solution};
\draw[-] (problem) -- (human);
\draw[->] (human) -- (model);
\draw[-] (model) -- (compile);
\draw[-] (compile) -- (solver1);
\draw[-] (compile) -- (solver2);
\draw[-] (compile) -- (solver3);
\draw[-] (compile) -- (solver4);
\draw[->] (solver1) -- (solution1);
\draw[->] (solver2) -- (solution2);
\draw[->] (solver3) -- (solution3);
\draw[->] (solver4) -- (solution4);
\end{tikzpicture}
\end{center}
\end{frame}

% \begin{frame}
  % \frametitle{Do It Now!}
  % \begin{itemize}
  % \item Download and install Minizinc
  % \item \url{https://www.minizinc.org/}
  % \end{itemize}
% \end{frame}


\begin{frame}
\frametitle{Why use Puzzles as Examples?}
\begin{itemize}
\item Easy to understand the problem
\item Solvable by hand without specialized knowledge
\item Possible to compare automated to manual solving process
\end{itemize}

\begin{quote}
The puzzle, though inanimate, is presented as a solvable problem without lasting negative consequences, a very low-risk low-reward situation. By being a puzzle, the object is attempting to convince the user that it must be completed.
\end{quote}
{\tiny Source: \href{https://www.everydayrhetoric.com/2020/03/the-rhetoric-of-puzzles/}{Every Day Rhetoric}}
\end{frame}


\mode<all>{
\part{Basic Constraint Propagation}
}
\begin{frame}
\partpage
\end{frame}
\mode<all>{
\begin{frame}
\frametitle{Objectives}
\begin{itemize}
\item Overview of Core Constraint Programming
\item Three Main Concepts
\begin{itemize}
\item Constraint Propagation
\item Global Constraints
\item Customizing Search
\end{itemize}
%\item Get Some Experience with MiniZinc
\item Topics will be treated in more detail in later parts of the school
\item Based on Examples, not Formal Description
\end{itemize}
\end{frame}

\begin{frame}
\frametitle{Outline}
\begin{itemize}
\item Why Constraint Programming?
\item Constraint Propagation
\item Global Constraints
\item Customizing Search
%\item Choices in Modelling
%\item Decomposition
%\item More Global Constraints
%\item Methodology
%\item What is missing?
\end{itemize}
\end{frame}

% \begin{frame}
  % \frametitle{Using MiniZinc IDE}
  % \begin{itemize}
  % \item Developed in the Australian NICTA project
  % \item Maintained by Monash University
  % \item Modelling tool with multiple back-end solvers
  % \item Available from \url{https://www.minizinc.org/}
  % \end{itemize}
% \end{frame}


% \begin{frame}
% \frametitle{Examples in ECLiPSe}
% \begin{itemize}
% \item Open sourced constraint programming language
% \item Development goes back to 1985
% \item ECRC, ICL, IC-Parc, PTL, Cisco
% \item \url{https://eclipseclp.org/}
% \item Specialities
% \begin{itemize}
% \item Develop new solvers for specific domains
% \item Integration with MIP
% \end{itemize}
% \item Not included in bundled MiniZinc IDE
% \item Specialized visualization tools used here
  % \begin{itemize}
  % \item CP-Viz, Simonis et al. 2010
    % \end{itemize}
% \end{itemize}
% \end{frame}

\begin{frame}
\frametitle{Tutorial Based on ECLiPSe ELearning Course}
\begin{itemize}
\item Self-study course in constraint programming
\item Supported by Cisco Systems and Silicon Valley Community Foundation
\item Multi-media format, video lectures, slides, handout etc
\item {\small \url{https://eclipseclp.org/ELearning/index.html}}
\end{itemize}
\end{frame}

\begin{frame}
\frametitle{Also Part of CRT-AI Constraint Week}
\begin{itemize}
\item Annual one week course on CP and Optimization in Ireland
\item Part of national training program for PhD students in AI
\item \url{https://www.crt-ai.ie/}
\end{itemize}
\end{frame}



\begin{frame}
\frametitle{Constraint Programming - in a nutshell}
\begin{itemize}
\item Declarative description of problems with
\begin{itemize}
\item {\em Variables} which range over (finite) sets of values
\item {\em Constraints} over subsets of variables which restrict possible value combinations
\item A {\em solution} is a value assignment which satisfies all constraints
\end{itemize}

\item Constraint propagation/reasoning
\begin{itemize}
\item Removing inconsistent values for variables
\item Detect failure if constraint can not be satisfied
\item Interaction of constraints via shared variables
\item Incomplete
\end{itemize}

\item Search
\begin{itemize}
\item User controlled assignment of values to variables
\item Each step triggers constraint propagation 
\end{itemize}
\item Different domains require/allow different methods
\end{itemize}
\end{frame}

\begin{frame}
  \frametitle{Constraint Programming is Different}
  \begin{itemize}
  \item Declarative Programming
    \begin{itemize}
    \item Concentrate on what you want
      \item Not how to get there
      \item Program != Algorithm
      \item Program = Model
    \end{itemize}
    \item Applied to Combinatorial Problems
      \begin{itemize}
        \item No complete polynomial algorithms known (exist?)
        \item CP less ad-hoc than heuristics
        \item Models can evolve
  \end{itemize}
  \end{itemize}
  \end{frame}
    
\begin{frame}
  \frametitle{A Subtractive Process}
  \begin{textblock}{4}(8,-3)
    \includegraphics[width=4cm]{../introduction/images/stages}
  \end{textblock}
  \vfill
  \begin{quote}
    ``Oh, bosh, as Mr. Ruskin says. Sculpture, per se, is the simplest thing in the world. All you have to do is to take a big chunk of marble and a hammer and chisel, make up your mind what you are about to create and chip off all the marble you don't want.''-Paris Gaulois.
  \end{quote}
  
  {\tiny Source: \url{https://quoteinvestigator.com/2014/06/22/chip-away/}}
\end{frame}

\begin{frame}
\frametitle{Basic Process}
\begin{center}
\begin{tikzpicture}
\node[shape=rectangle,fill=pantone127-4] (problem) at (2,5) {Problem};
\node (human) at (2,4) {Human};
\node[shape=rectangle,fill=pantone127-4] (model) at (2,3) {Model};
\node (solver) at (2,2) {Constraint Solver/Search};
\node[shape=rectangle,fill=pantone157-8] (solution) at (2,1) {Solution};
\draw[-] (problem) -- (human);
\draw[->] (human) -- (model);
\draw[-] (model) -- (solver);
\draw[->] (solver) -- (solution);
\end{tikzpicture}
\end{center}
\end{frame}

\begin{frame}
\frametitle{More Realistic}
\begin{center}
\begin{tikzpicture}[xscale=4.0]
\node[shape=rectangle,fill=pantone127-4] (problem) at (2,5) {Problem};
\node (human) at (2,4) {Human};
\node[shape=rectangle,fill=pantone127-4] (model) at (2,3) {Model};
\node (solver) at (2,2) {Constraint Solver/Search};
\node[shape=rectangle,fill=pantone24-8] (hangs) at (1,1) {Hangs};
\node[shape=rectangle,fill=pantone157-8] (solution) at (2,1) {Solution};
\node[shape=rectangle,fill=pantone24-8] (wrong) at (3,1) {Wrong Solution};
\draw[-] (problem) -- (human);
\draw[->] (human) -- (model);
\draw[-] (model) -- (solver);
\draw[->] (solver) -- (hangs);
\draw[->] (solver) -- (solution);
\draw[->] (solver) -- (wrong);
\draw[->] (hangs) -- (1,4) -- (human);
\draw[->] (wrong) -- (3,4) -- (human);
\end{tikzpicture}
\end{center}
\end{frame}

\begin{frame}
\frametitle{Dual Role of Model}
\begin{itemize}
\item Allows Human to Express Problem
\begin{itemize}
\item Close to Problem Domain
\item Constraints as Abstractions
\end{itemize}
\item Allows Solver to Execute
\begin{itemize}
\item Variables as Communication Mechanism
\item Constraints as Algorithms
\end{itemize}
\end{itemize}
\end{frame}

\begin{frame}
\frametitle{Modelling Frameworks}
\begin{itemize}
\item \href{https://www.minizinc.org/}{MiniZinc} (NICTA, Monash University, Australia)
\item \href{https://github.com/eomahony/Numberjack}{NumberJack} (Insight, Ireland) 
\item \href{https://www-users.york.ac.uk/peter.nightingale/savilerow//}{EssencePrime/SavilleRow} (UK) 
\item \href{https://cpmpy.readthedocs.io/en/latest/}{CPMpy} (KU Leuven)
\item Allow use of multiple back-end solvers
\item Compile model into variants for each solver
\item A priori solver independent model(CP, MIP, SAT)
\end{itemize}
\end{frame} 

\begin{frame}
\frametitle{Framework Process}
\begin{center}
\begin{tikzpicture}[xscale=2]
\node[shape=rectangle,fill=pantone127-4] (problem) at (2.5,5) {Problem};
\node (human) at (2.5,4) {Human};
\node[shape=rectangle,fill=pantone127-4] (model) at (2.5,3) {Model};
\node (compile) at (2.5,2) {Compile/Reformulate};
\node (solver1) at (1,1) {CP};
\node (solver2) at (2,1) {MIP};
\node (solver3) at (3,1) {SAT};
\node (solver4) at (4,1) {Other};
\node[shape=rectangle,fill=pantone157-8] (solution1) at (1,0) {Solution};
\node[shape=rectangle,fill=pantone157-8] (solution2) at (2,0) {Solution};
\node[shape=rectangle,fill=pantone157-8] (solution3) at (3,0) {Solution};
\node[shape=rectangle,fill=pantone157-8] (solution4) at (4,0) {Solution};
\draw[-] (problem) -- (human);
\draw[->] (human) -- (model);
\draw[-] (model) -- (compile);
\draw[-] (compile) -- (solver1);
\draw[-] (compile) -- (solver2);
\draw[-] (compile) -- (solver3);
\draw[-] (compile) -- (solver4);
\draw[->] (solver1) -- (solution1);
\draw[->] (solver2) -- (solution2);
\draw[->] (solver3) -- (solution3);
\draw[->] (solver4) -- (solution4);
\end{tikzpicture}
\end{center}
\end{frame}

% \begin{frame}
  % \frametitle{Do It Now!}
  % \begin{itemize}
  % \item Download and install Minizinc
  % \item \url{https://www.minizinc.org/}
  % \end{itemize}
% \end{frame}


\begin{frame}
\frametitle{Why use Puzzles as Examples?}
\begin{itemize}
\item Easy to understand the problem
\item Solvable by hand without specialized knowledge
\item Possible to compare automated to manual solving process
\end{itemize}

\begin{quote}
The puzzle, though inanimate, is presented as a solvable problem without lasting negative consequences, a very low-risk low-reward situation. By being a puzzle, the object is attempting to convince the user that it must be completed.
\end{quote}
{\tiny Source: \href{https://www.everydayrhetoric.com/2020/03/the-rhetoric-of-puzzles/}{Every Day Rhetoric}}
\end{frame}


\mode<all>{
\part{Basic Constraint Propagation}
}
\begin{frame}
\partpage
\end{frame}
\mode<all>{
\input{../sendmore/body}
}

\mode<all>{
\part{Global Constraints}
}
\begin{frame}
\partpage
\end{frame}
\mode<all>{
\input{../sudoku/body}
}

\clearpage
\mode<all>{
\part{Customizing Search}
}
\begin{frame}
\partpage
\end{frame}
\mode<all>{
\input{../nqueen/body}
}


%% \clearpage
%% \mode<all>{
%%   \part{Problem Decomposition}
%% }
%% \begin{frame}
%%   \partpage
%% \end{frame}
%% \mode<all>{
%%   \input{../party/body}
%%   }

%% \clearpage
%% \mode<all>{
%%   \part{More Global Constraints}
%% }
%% \begin{frame}
%%   \partpage
%% \end{frame}
%% \mode<all>{
%%   \input{../car/body}
%%   }

%% \clearpage
%% \mode<all>{
%%   \part{Methodology}
%% }
%% \begin{frame}
%%   \partpage
%% \end{frame}
%% \mode<all>{
%%   \input{../methodology/body}
%%   }

\clearpage
\mode<all>{
\part{What is missing?}
}
\begin{frame}
\partpage
\end{frame}

\begin{frame}
\frametitle{Many Specialized Topics}
\begin{itemize}
\item How to design efficient core engine
\item Hybrids with LP/MIP tools
\item Hybrids with SAT
\item Symmetry breaking
\item Use of MDD/BDD to encode sets of solutions
\item High level modelling tools
\item Debugging/visualization
\end{itemize}
\end{frame}

\begin{frame}
\frametitle{Reformulation}
\begin{itemize}
\item Just because the user has modelled it this way, it doesn't mean we have to solve it that way
\begin{itemize}
\item Replace some constraint(s) by other, equivalent constraints
\item Because we don't have that constraint in our system
\item For performance
\end{itemize}
\end{itemize}
\end{frame}

\begin{frame}
\frametitle{Learning}
\begin{itemize}
\item While solving the problem we can learn how to strengthen the model/search 
\begin{itemize}
\item Understand which constraints/method contribute to propagation and change schedule
\item Learn no-good constraints by explaining failure
\item Adapt search strategy based on search experience
\end{itemize}
\end{itemize}
\end{frame}

%% \begin{frame}
%% \frametitle{Refined Process}
%% \begin{center}
%% \begin{tikzpicture}
%% \node[shape=rectangle,fill=pantone127-4] (problem) at (3,7) {Problem};
%% \node (human) at (3,6) {Human};
%% \node[shape=rectangle,fill=pantone127-4] (model) at (3,5) {User Model};
%% \node (reformulation) at (3,4) {Reformulation};
%% \node[shape=rectangle,fill=pantone127-4] (implementation) at (3,3) {Implementation Model};
%% \node (solver) at (3,2) {Constraint Solver/Search};
%% \node[shape=rectangle,fill=pantone157-8] (solution) at (3,1) {Solution};
%% \draw[-] (problem) -- (human);
%% \draw[->] (human) -- (model);
%% \draw[-] (model) -- (reformulation);
%% \draw[->] (reformulation) -- (implementation);
%% \draw[-] (implementation) -- (solver);
%% \draw[->] (solver) -- (0,2) -- (0,3) -- (implementation);
%% \draw[->] (solver) -- (solution);
%% \end{tikzpicture}
%% \end{center}
%% \end{frame}

%\begin{frame}
%\frametitle{What is CP actually used for?}
%\url{http://hsimonis.wordpress.com}
%\includegraphics[width=10cm]{../introduction/images/blog}
%\end{frame}

%% \begin{frame}
%% \frametitle{Exercises}
%% \begin{itemize}
%% \item Install Constraint Solver of your Choice
%% \item MiniZinc (MiningZinc) will be used in later sessions
%% \item Run Sendmoremoney, Sudoku, Queens examples
%% \item Can you choose which propagator to use for alldifferent?
%% \item Can you use different search strategies?
%% \end{itemize}
%% \end{frame}

\begin{frame}
\frametitle{More Learning Resources}
\begin{itemize}
\item Survey of Methods, Resources, and Formats for Teaching Constraint Programming
\begin{itemize}
\item by Tejas Santanam, Helmut Simonis
\item \url{https://doi.org/10.48550/arXiv.2403.12717}
\item Based on survey of community for WTCP 2023
\item \url{https://hsimonis.github.io/WTCP2023/}
\end{itemize}
\end{itemize}
\end{frame}


}

\mode<all>{
\part{Global Constraints}
}
\begin{frame}
\partpage
\end{frame}
\mode<all>{
\begin{frame}
\frametitle{Objectives}
\begin{itemize}
\item Overview of Core Constraint Programming
\item Three Main Concepts
\begin{itemize}
\item Constraint Propagation
\item Global Constraints
\item Customizing Search
\end{itemize}
%\item Get Some Experience with MiniZinc
\item Topics will be treated in more detail in later parts of the school
\item Based on Examples, not Formal Description
\end{itemize}
\end{frame}

\begin{frame}
\frametitle{Outline}
\begin{itemize}
\item Why Constraint Programming?
\item Constraint Propagation
\item Global Constraints
\item Customizing Search
%\item Choices in Modelling
%\item Decomposition
%\item More Global Constraints
%\item Methodology
%\item What is missing?
\end{itemize}
\end{frame}

% \begin{frame}
  % \frametitle{Using MiniZinc IDE}
  % \begin{itemize}
  % \item Developed in the Australian NICTA project
  % \item Maintained by Monash University
  % \item Modelling tool with multiple back-end solvers
  % \item Available from \url{https://www.minizinc.org/}
  % \end{itemize}
% \end{frame}


% \begin{frame}
% \frametitle{Examples in ECLiPSe}
% \begin{itemize}
% \item Open sourced constraint programming language
% \item Development goes back to 1985
% \item ECRC, ICL, IC-Parc, PTL, Cisco
% \item \url{https://eclipseclp.org/}
% \item Specialities
% \begin{itemize}
% \item Develop new solvers for specific domains
% \item Integration with MIP
% \end{itemize}
% \item Not included in bundled MiniZinc IDE
% \item Specialized visualization tools used here
  % \begin{itemize}
  % \item CP-Viz, Simonis et al. 2010
    % \end{itemize}
% \end{itemize}
% \end{frame}

\begin{frame}
\frametitle{Tutorial Based on ECLiPSe ELearning Course}
\begin{itemize}
\item Self-study course in constraint programming
\item Supported by Cisco Systems and Silicon Valley Community Foundation
\item Multi-media format, video lectures, slides, handout etc
\item {\small \url{https://eclipseclp.org/ELearning/index.html}}
\end{itemize}
\end{frame}

\begin{frame}
\frametitle{Also Part of CRT-AI Constraint Week}
\begin{itemize}
\item Annual one week course on CP and Optimization in Ireland
\item Part of national training program for PhD students in AI
\item \url{https://www.crt-ai.ie/}
\end{itemize}
\end{frame}



\begin{frame}
\frametitle{Constraint Programming - in a nutshell}
\begin{itemize}
\item Declarative description of problems with
\begin{itemize}
\item {\em Variables} which range over (finite) sets of values
\item {\em Constraints} over subsets of variables which restrict possible value combinations
\item A {\em solution} is a value assignment which satisfies all constraints
\end{itemize}

\item Constraint propagation/reasoning
\begin{itemize}
\item Removing inconsistent values for variables
\item Detect failure if constraint can not be satisfied
\item Interaction of constraints via shared variables
\item Incomplete
\end{itemize}

\item Search
\begin{itemize}
\item User controlled assignment of values to variables
\item Each step triggers constraint propagation 
\end{itemize}
\item Different domains require/allow different methods
\end{itemize}
\end{frame}

\begin{frame}
  \frametitle{Constraint Programming is Different}
  \begin{itemize}
  \item Declarative Programming
    \begin{itemize}
    \item Concentrate on what you want
      \item Not how to get there
      \item Program != Algorithm
      \item Program = Model
    \end{itemize}
    \item Applied to Combinatorial Problems
      \begin{itemize}
        \item No complete polynomial algorithms known (exist?)
        \item CP less ad-hoc than heuristics
        \item Models can evolve
  \end{itemize}
  \end{itemize}
  \end{frame}
    
\begin{frame}
  \frametitle{A Subtractive Process}
  \begin{textblock}{4}(8,-3)
    \includegraphics[width=4cm]{../introduction/images/stages}
  \end{textblock}
  \vfill
  \begin{quote}
    ``Oh, bosh, as Mr. Ruskin says. Sculpture, per se, is the simplest thing in the world. All you have to do is to take a big chunk of marble and a hammer and chisel, make up your mind what you are about to create and chip off all the marble you don't want.''-Paris Gaulois.
  \end{quote}
  
  {\tiny Source: \url{https://quoteinvestigator.com/2014/06/22/chip-away/}}
\end{frame}

\begin{frame}
\frametitle{Basic Process}
\begin{center}
\begin{tikzpicture}
\node[shape=rectangle,fill=pantone127-4] (problem) at (2,5) {Problem};
\node (human) at (2,4) {Human};
\node[shape=rectangle,fill=pantone127-4] (model) at (2,3) {Model};
\node (solver) at (2,2) {Constraint Solver/Search};
\node[shape=rectangle,fill=pantone157-8] (solution) at (2,1) {Solution};
\draw[-] (problem) -- (human);
\draw[->] (human) -- (model);
\draw[-] (model) -- (solver);
\draw[->] (solver) -- (solution);
\end{tikzpicture}
\end{center}
\end{frame}

\begin{frame}
\frametitle{More Realistic}
\begin{center}
\begin{tikzpicture}[xscale=4.0]
\node[shape=rectangle,fill=pantone127-4] (problem) at (2,5) {Problem};
\node (human) at (2,4) {Human};
\node[shape=rectangle,fill=pantone127-4] (model) at (2,3) {Model};
\node (solver) at (2,2) {Constraint Solver/Search};
\node[shape=rectangle,fill=pantone24-8] (hangs) at (1,1) {Hangs};
\node[shape=rectangle,fill=pantone157-8] (solution) at (2,1) {Solution};
\node[shape=rectangle,fill=pantone24-8] (wrong) at (3,1) {Wrong Solution};
\draw[-] (problem) -- (human);
\draw[->] (human) -- (model);
\draw[-] (model) -- (solver);
\draw[->] (solver) -- (hangs);
\draw[->] (solver) -- (solution);
\draw[->] (solver) -- (wrong);
\draw[->] (hangs) -- (1,4) -- (human);
\draw[->] (wrong) -- (3,4) -- (human);
\end{tikzpicture}
\end{center}
\end{frame}

\begin{frame}
\frametitle{Dual Role of Model}
\begin{itemize}
\item Allows Human to Express Problem
\begin{itemize}
\item Close to Problem Domain
\item Constraints as Abstractions
\end{itemize}
\item Allows Solver to Execute
\begin{itemize}
\item Variables as Communication Mechanism
\item Constraints as Algorithms
\end{itemize}
\end{itemize}
\end{frame}

\begin{frame}
\frametitle{Modelling Frameworks}
\begin{itemize}
\item \href{https://www.minizinc.org/}{MiniZinc} (NICTA, Monash University, Australia)
\item \href{https://github.com/eomahony/Numberjack}{NumberJack} (Insight, Ireland) 
\item \href{https://www-users.york.ac.uk/peter.nightingale/savilerow//}{EssencePrime/SavilleRow} (UK) 
\item \href{https://cpmpy.readthedocs.io/en/latest/}{CPMpy} (KU Leuven)
\item Allow use of multiple back-end solvers
\item Compile model into variants for each solver
\item A priori solver independent model(CP, MIP, SAT)
\end{itemize}
\end{frame} 

\begin{frame}
\frametitle{Framework Process}
\begin{center}
\begin{tikzpicture}[xscale=2]
\node[shape=rectangle,fill=pantone127-4] (problem) at (2.5,5) {Problem};
\node (human) at (2.5,4) {Human};
\node[shape=rectangle,fill=pantone127-4] (model) at (2.5,3) {Model};
\node (compile) at (2.5,2) {Compile/Reformulate};
\node (solver1) at (1,1) {CP};
\node (solver2) at (2,1) {MIP};
\node (solver3) at (3,1) {SAT};
\node (solver4) at (4,1) {Other};
\node[shape=rectangle,fill=pantone157-8] (solution1) at (1,0) {Solution};
\node[shape=rectangle,fill=pantone157-8] (solution2) at (2,0) {Solution};
\node[shape=rectangle,fill=pantone157-8] (solution3) at (3,0) {Solution};
\node[shape=rectangle,fill=pantone157-8] (solution4) at (4,0) {Solution};
\draw[-] (problem) -- (human);
\draw[->] (human) -- (model);
\draw[-] (model) -- (compile);
\draw[-] (compile) -- (solver1);
\draw[-] (compile) -- (solver2);
\draw[-] (compile) -- (solver3);
\draw[-] (compile) -- (solver4);
\draw[->] (solver1) -- (solution1);
\draw[->] (solver2) -- (solution2);
\draw[->] (solver3) -- (solution3);
\draw[->] (solver4) -- (solution4);
\end{tikzpicture}
\end{center}
\end{frame}

% \begin{frame}
  % \frametitle{Do It Now!}
  % \begin{itemize}
  % \item Download and install Minizinc
  % \item \url{https://www.minizinc.org/}
  % \end{itemize}
% \end{frame}


\begin{frame}
\frametitle{Why use Puzzles as Examples?}
\begin{itemize}
\item Easy to understand the problem
\item Solvable by hand without specialized knowledge
\item Possible to compare automated to manual solving process
\end{itemize}

\begin{quote}
The puzzle, though inanimate, is presented as a solvable problem without lasting negative consequences, a very low-risk low-reward situation. By being a puzzle, the object is attempting to convince the user that it must be completed.
\end{quote}
{\tiny Source: \href{https://www.everydayrhetoric.com/2020/03/the-rhetoric-of-puzzles/}{Every Day Rhetoric}}
\end{frame}


\mode<all>{
\part{Basic Constraint Propagation}
}
\begin{frame}
\partpage
\end{frame}
\mode<all>{
\input{../sendmore/body}
}

\mode<all>{
\part{Global Constraints}
}
\begin{frame}
\partpage
\end{frame}
\mode<all>{
\input{../sudoku/body}
}

\clearpage
\mode<all>{
\part{Customizing Search}
}
\begin{frame}
\partpage
\end{frame}
\mode<all>{
\input{../nqueen/body}
}


%% \clearpage
%% \mode<all>{
%%   \part{Problem Decomposition}
%% }
%% \begin{frame}
%%   \partpage
%% \end{frame}
%% \mode<all>{
%%   \input{../party/body}
%%   }

%% \clearpage
%% \mode<all>{
%%   \part{More Global Constraints}
%% }
%% \begin{frame}
%%   \partpage
%% \end{frame}
%% \mode<all>{
%%   \input{../car/body}
%%   }

%% \clearpage
%% \mode<all>{
%%   \part{Methodology}
%% }
%% \begin{frame}
%%   \partpage
%% \end{frame}
%% \mode<all>{
%%   \input{../methodology/body}
%%   }

\clearpage
\mode<all>{
\part{What is missing?}
}
\begin{frame}
\partpage
\end{frame}

\begin{frame}
\frametitle{Many Specialized Topics}
\begin{itemize}
\item How to design efficient core engine
\item Hybrids with LP/MIP tools
\item Hybrids with SAT
\item Symmetry breaking
\item Use of MDD/BDD to encode sets of solutions
\item High level modelling tools
\item Debugging/visualization
\end{itemize}
\end{frame}

\begin{frame}
\frametitle{Reformulation}
\begin{itemize}
\item Just because the user has modelled it this way, it doesn't mean we have to solve it that way
\begin{itemize}
\item Replace some constraint(s) by other, equivalent constraints
\item Because we don't have that constraint in our system
\item For performance
\end{itemize}
\end{itemize}
\end{frame}

\begin{frame}
\frametitle{Learning}
\begin{itemize}
\item While solving the problem we can learn how to strengthen the model/search 
\begin{itemize}
\item Understand which constraints/method contribute to propagation and change schedule
\item Learn no-good constraints by explaining failure
\item Adapt search strategy based on search experience
\end{itemize}
\end{itemize}
\end{frame}

%% \begin{frame}
%% \frametitle{Refined Process}
%% \begin{center}
%% \begin{tikzpicture}
%% \node[shape=rectangle,fill=pantone127-4] (problem) at (3,7) {Problem};
%% \node (human) at (3,6) {Human};
%% \node[shape=rectangle,fill=pantone127-4] (model) at (3,5) {User Model};
%% \node (reformulation) at (3,4) {Reformulation};
%% \node[shape=rectangle,fill=pantone127-4] (implementation) at (3,3) {Implementation Model};
%% \node (solver) at (3,2) {Constraint Solver/Search};
%% \node[shape=rectangle,fill=pantone157-8] (solution) at (3,1) {Solution};
%% \draw[-] (problem) -- (human);
%% \draw[->] (human) -- (model);
%% \draw[-] (model) -- (reformulation);
%% \draw[->] (reformulation) -- (implementation);
%% \draw[-] (implementation) -- (solver);
%% \draw[->] (solver) -- (0,2) -- (0,3) -- (implementation);
%% \draw[->] (solver) -- (solution);
%% \end{tikzpicture}
%% \end{center}
%% \end{frame}

%\begin{frame}
%\frametitle{What is CP actually used for?}
%\url{http://hsimonis.wordpress.com}
%\includegraphics[width=10cm]{../introduction/images/blog}
%\end{frame}

%% \begin{frame}
%% \frametitle{Exercises}
%% \begin{itemize}
%% \item Install Constraint Solver of your Choice
%% \item MiniZinc (MiningZinc) will be used in later sessions
%% \item Run Sendmoremoney, Sudoku, Queens examples
%% \item Can you choose which propagator to use for alldifferent?
%% \item Can you use different search strategies?
%% \end{itemize}
%% \end{frame}

\begin{frame}
\frametitle{More Learning Resources}
\begin{itemize}
\item Survey of Methods, Resources, and Formats for Teaching Constraint Programming
\begin{itemize}
\item by Tejas Santanam, Helmut Simonis
\item \url{https://doi.org/10.48550/arXiv.2403.12717}
\item Based on survey of community for WTCP 2023
\item \url{https://hsimonis.github.io/WTCP2023/}
\end{itemize}
\end{itemize}
\end{frame}


}

\clearpage
\mode<all>{
\part{Customizing Search}
}
\begin{frame}
\partpage
\end{frame}
\mode<all>{
\begin{frame}
\frametitle{Objectives}
\begin{itemize}
\item Overview of Core Constraint Programming
\item Three Main Concepts
\begin{itemize}
\item Constraint Propagation
\item Global Constraints
\item Customizing Search
\end{itemize}
%\item Get Some Experience with MiniZinc
\item Topics will be treated in more detail in later parts of the school
\item Based on Examples, not Formal Description
\end{itemize}
\end{frame}

\begin{frame}
\frametitle{Outline}
\begin{itemize}
\item Why Constraint Programming?
\item Constraint Propagation
\item Global Constraints
\item Customizing Search
%\item Choices in Modelling
%\item Decomposition
%\item More Global Constraints
%\item Methodology
%\item What is missing?
\end{itemize}
\end{frame}

% \begin{frame}
  % \frametitle{Using MiniZinc IDE}
  % \begin{itemize}
  % \item Developed in the Australian NICTA project
  % \item Maintained by Monash University
  % \item Modelling tool with multiple back-end solvers
  % \item Available from \url{https://www.minizinc.org/}
  % \end{itemize}
% \end{frame}


% \begin{frame}
% \frametitle{Examples in ECLiPSe}
% \begin{itemize}
% \item Open sourced constraint programming language
% \item Development goes back to 1985
% \item ECRC, ICL, IC-Parc, PTL, Cisco
% \item \url{https://eclipseclp.org/}
% \item Specialities
% \begin{itemize}
% \item Develop new solvers for specific domains
% \item Integration with MIP
% \end{itemize}
% \item Not included in bundled MiniZinc IDE
% \item Specialized visualization tools used here
  % \begin{itemize}
  % \item CP-Viz, Simonis et al. 2010
    % \end{itemize}
% \end{itemize}
% \end{frame}

\begin{frame}
\frametitle{Tutorial Based on ECLiPSe ELearning Course}
\begin{itemize}
\item Self-study course in constraint programming
\item Supported by Cisco Systems and Silicon Valley Community Foundation
\item Multi-media format, video lectures, slides, handout etc
\item {\small \url{https://eclipseclp.org/ELearning/index.html}}
\end{itemize}
\end{frame}

\begin{frame}
\frametitle{Also Part of CRT-AI Constraint Week}
\begin{itemize}
\item Annual one week course on CP and Optimization in Ireland
\item Part of national training program for PhD students in AI
\item \url{https://www.crt-ai.ie/}
\end{itemize}
\end{frame}



\begin{frame}
\frametitle{Constraint Programming - in a nutshell}
\begin{itemize}
\item Declarative description of problems with
\begin{itemize}
\item {\em Variables} which range over (finite) sets of values
\item {\em Constraints} over subsets of variables which restrict possible value combinations
\item A {\em solution} is a value assignment which satisfies all constraints
\end{itemize}

\item Constraint propagation/reasoning
\begin{itemize}
\item Removing inconsistent values for variables
\item Detect failure if constraint can not be satisfied
\item Interaction of constraints via shared variables
\item Incomplete
\end{itemize}

\item Search
\begin{itemize}
\item User controlled assignment of values to variables
\item Each step triggers constraint propagation 
\end{itemize}
\item Different domains require/allow different methods
\end{itemize}
\end{frame}

\begin{frame}
  \frametitle{Constraint Programming is Different}
  \begin{itemize}
  \item Declarative Programming
    \begin{itemize}
    \item Concentrate on what you want
      \item Not how to get there
      \item Program != Algorithm
      \item Program = Model
    \end{itemize}
    \item Applied to Combinatorial Problems
      \begin{itemize}
        \item No complete polynomial algorithms known (exist?)
        \item CP less ad-hoc than heuristics
        \item Models can evolve
  \end{itemize}
  \end{itemize}
  \end{frame}
    
\begin{frame}
  \frametitle{A Subtractive Process}
  \begin{textblock}{4}(8,-3)
    \includegraphics[width=4cm]{../introduction/images/stages}
  \end{textblock}
  \vfill
  \begin{quote}
    ``Oh, bosh, as Mr. Ruskin says. Sculpture, per se, is the simplest thing in the world. All you have to do is to take a big chunk of marble and a hammer and chisel, make up your mind what you are about to create and chip off all the marble you don't want.''-Paris Gaulois.
  \end{quote}
  
  {\tiny Source: \url{https://quoteinvestigator.com/2014/06/22/chip-away/}}
\end{frame}

\begin{frame}
\frametitle{Basic Process}
\begin{center}
\begin{tikzpicture}
\node[shape=rectangle,fill=pantone127-4] (problem) at (2,5) {Problem};
\node (human) at (2,4) {Human};
\node[shape=rectangle,fill=pantone127-4] (model) at (2,3) {Model};
\node (solver) at (2,2) {Constraint Solver/Search};
\node[shape=rectangle,fill=pantone157-8] (solution) at (2,1) {Solution};
\draw[-] (problem) -- (human);
\draw[->] (human) -- (model);
\draw[-] (model) -- (solver);
\draw[->] (solver) -- (solution);
\end{tikzpicture}
\end{center}
\end{frame}

\begin{frame}
\frametitle{More Realistic}
\begin{center}
\begin{tikzpicture}[xscale=4.0]
\node[shape=rectangle,fill=pantone127-4] (problem) at (2,5) {Problem};
\node (human) at (2,4) {Human};
\node[shape=rectangle,fill=pantone127-4] (model) at (2,3) {Model};
\node (solver) at (2,2) {Constraint Solver/Search};
\node[shape=rectangle,fill=pantone24-8] (hangs) at (1,1) {Hangs};
\node[shape=rectangle,fill=pantone157-8] (solution) at (2,1) {Solution};
\node[shape=rectangle,fill=pantone24-8] (wrong) at (3,1) {Wrong Solution};
\draw[-] (problem) -- (human);
\draw[->] (human) -- (model);
\draw[-] (model) -- (solver);
\draw[->] (solver) -- (hangs);
\draw[->] (solver) -- (solution);
\draw[->] (solver) -- (wrong);
\draw[->] (hangs) -- (1,4) -- (human);
\draw[->] (wrong) -- (3,4) -- (human);
\end{tikzpicture}
\end{center}
\end{frame}

\begin{frame}
\frametitle{Dual Role of Model}
\begin{itemize}
\item Allows Human to Express Problem
\begin{itemize}
\item Close to Problem Domain
\item Constraints as Abstractions
\end{itemize}
\item Allows Solver to Execute
\begin{itemize}
\item Variables as Communication Mechanism
\item Constraints as Algorithms
\end{itemize}
\end{itemize}
\end{frame}

\begin{frame}
\frametitle{Modelling Frameworks}
\begin{itemize}
\item \href{https://www.minizinc.org/}{MiniZinc} (NICTA, Monash University, Australia)
\item \href{https://github.com/eomahony/Numberjack}{NumberJack} (Insight, Ireland) 
\item \href{https://www-users.york.ac.uk/peter.nightingale/savilerow//}{EssencePrime/SavilleRow} (UK) 
\item \href{https://cpmpy.readthedocs.io/en/latest/}{CPMpy} (KU Leuven)
\item Allow use of multiple back-end solvers
\item Compile model into variants for each solver
\item A priori solver independent model(CP, MIP, SAT)
\end{itemize}
\end{frame} 

\begin{frame}
\frametitle{Framework Process}
\begin{center}
\begin{tikzpicture}[xscale=2]
\node[shape=rectangle,fill=pantone127-4] (problem) at (2.5,5) {Problem};
\node (human) at (2.5,4) {Human};
\node[shape=rectangle,fill=pantone127-4] (model) at (2.5,3) {Model};
\node (compile) at (2.5,2) {Compile/Reformulate};
\node (solver1) at (1,1) {CP};
\node (solver2) at (2,1) {MIP};
\node (solver3) at (3,1) {SAT};
\node (solver4) at (4,1) {Other};
\node[shape=rectangle,fill=pantone157-8] (solution1) at (1,0) {Solution};
\node[shape=rectangle,fill=pantone157-8] (solution2) at (2,0) {Solution};
\node[shape=rectangle,fill=pantone157-8] (solution3) at (3,0) {Solution};
\node[shape=rectangle,fill=pantone157-8] (solution4) at (4,0) {Solution};
\draw[-] (problem) -- (human);
\draw[->] (human) -- (model);
\draw[-] (model) -- (compile);
\draw[-] (compile) -- (solver1);
\draw[-] (compile) -- (solver2);
\draw[-] (compile) -- (solver3);
\draw[-] (compile) -- (solver4);
\draw[->] (solver1) -- (solution1);
\draw[->] (solver2) -- (solution2);
\draw[->] (solver3) -- (solution3);
\draw[->] (solver4) -- (solution4);
\end{tikzpicture}
\end{center}
\end{frame}

% \begin{frame}
  % \frametitle{Do It Now!}
  % \begin{itemize}
  % \item Download and install Minizinc
  % \item \url{https://www.minizinc.org/}
  % \end{itemize}
% \end{frame}


\begin{frame}
\frametitle{Why use Puzzles as Examples?}
\begin{itemize}
\item Easy to understand the problem
\item Solvable by hand without specialized knowledge
\item Possible to compare automated to manual solving process
\end{itemize}

\begin{quote}
The puzzle, though inanimate, is presented as a solvable problem without lasting negative consequences, a very low-risk low-reward situation. By being a puzzle, the object is attempting to convince the user that it must be completed.
\end{quote}
{\tiny Source: \href{https://www.everydayrhetoric.com/2020/03/the-rhetoric-of-puzzles/}{Every Day Rhetoric}}
\end{frame}


\mode<all>{
\part{Basic Constraint Propagation}
}
\begin{frame}
\partpage
\end{frame}
\mode<all>{
\input{../sendmore/body}
}

\mode<all>{
\part{Global Constraints}
}
\begin{frame}
\partpage
\end{frame}
\mode<all>{
\input{../sudoku/body}
}

\clearpage
\mode<all>{
\part{Customizing Search}
}
\begin{frame}
\partpage
\end{frame}
\mode<all>{
\input{../nqueen/body}
}


%% \clearpage
%% \mode<all>{
%%   \part{Problem Decomposition}
%% }
%% \begin{frame}
%%   \partpage
%% \end{frame}
%% \mode<all>{
%%   \input{../party/body}
%%   }

%% \clearpage
%% \mode<all>{
%%   \part{More Global Constraints}
%% }
%% \begin{frame}
%%   \partpage
%% \end{frame}
%% \mode<all>{
%%   \input{../car/body}
%%   }

%% \clearpage
%% \mode<all>{
%%   \part{Methodology}
%% }
%% \begin{frame}
%%   \partpage
%% \end{frame}
%% \mode<all>{
%%   \input{../methodology/body}
%%   }

\clearpage
\mode<all>{
\part{What is missing?}
}
\begin{frame}
\partpage
\end{frame}

\begin{frame}
\frametitle{Many Specialized Topics}
\begin{itemize}
\item How to design efficient core engine
\item Hybrids with LP/MIP tools
\item Hybrids with SAT
\item Symmetry breaking
\item Use of MDD/BDD to encode sets of solutions
\item High level modelling tools
\item Debugging/visualization
\end{itemize}
\end{frame}

\begin{frame}
\frametitle{Reformulation}
\begin{itemize}
\item Just because the user has modelled it this way, it doesn't mean we have to solve it that way
\begin{itemize}
\item Replace some constraint(s) by other, equivalent constraints
\item Because we don't have that constraint in our system
\item For performance
\end{itemize}
\end{itemize}
\end{frame}

\begin{frame}
\frametitle{Learning}
\begin{itemize}
\item While solving the problem we can learn how to strengthen the model/search 
\begin{itemize}
\item Understand which constraints/method contribute to propagation and change schedule
\item Learn no-good constraints by explaining failure
\item Adapt search strategy based on search experience
\end{itemize}
\end{itemize}
\end{frame}

%% \begin{frame}
%% \frametitle{Refined Process}
%% \begin{center}
%% \begin{tikzpicture}
%% \node[shape=rectangle,fill=pantone127-4] (problem) at (3,7) {Problem};
%% \node (human) at (3,6) {Human};
%% \node[shape=rectangle,fill=pantone127-4] (model) at (3,5) {User Model};
%% \node (reformulation) at (3,4) {Reformulation};
%% \node[shape=rectangle,fill=pantone127-4] (implementation) at (3,3) {Implementation Model};
%% \node (solver) at (3,2) {Constraint Solver/Search};
%% \node[shape=rectangle,fill=pantone157-8] (solution) at (3,1) {Solution};
%% \draw[-] (problem) -- (human);
%% \draw[->] (human) -- (model);
%% \draw[-] (model) -- (reformulation);
%% \draw[->] (reformulation) -- (implementation);
%% \draw[-] (implementation) -- (solver);
%% \draw[->] (solver) -- (0,2) -- (0,3) -- (implementation);
%% \draw[->] (solver) -- (solution);
%% \end{tikzpicture}
%% \end{center}
%% \end{frame}

%\begin{frame}
%\frametitle{What is CP actually used for?}
%\url{http://hsimonis.wordpress.com}
%\includegraphics[width=10cm]{../introduction/images/blog}
%\end{frame}

%% \begin{frame}
%% \frametitle{Exercises}
%% \begin{itemize}
%% \item Install Constraint Solver of your Choice
%% \item MiniZinc (MiningZinc) will be used in later sessions
%% \item Run Sendmoremoney, Sudoku, Queens examples
%% \item Can you choose which propagator to use for alldifferent?
%% \item Can you use different search strategies?
%% \end{itemize}
%% \end{frame}

\begin{frame}
\frametitle{More Learning Resources}
\begin{itemize}
\item Survey of Methods, Resources, and Formats for Teaching Constraint Programming
\begin{itemize}
\item by Tejas Santanam, Helmut Simonis
\item \url{https://doi.org/10.48550/arXiv.2403.12717}
\item Based on survey of community for WTCP 2023
\item \url{https://hsimonis.github.io/WTCP2023/}
\end{itemize}
\end{itemize}
\end{frame}


}


%% \clearpage
%% \mode<all>{
%%   \part{Problem Decomposition}
%% }
%% \begin{frame}
%%   \partpage
%% \end{frame}
%% \mode<all>{
%%   \begin{frame}
\frametitle{Objectives}
\begin{itemize}
\item Overview of Core Constraint Programming
\item Three Main Concepts
\begin{itemize}
\item Constraint Propagation
\item Global Constraints
\item Customizing Search
\end{itemize}
%\item Get Some Experience with MiniZinc
\item Topics will be treated in more detail in later parts of the school
\item Based on Examples, not Formal Description
\end{itemize}
\end{frame}

\begin{frame}
\frametitle{Outline}
\begin{itemize}
\item Why Constraint Programming?
\item Constraint Propagation
\item Global Constraints
\item Customizing Search
%\item Choices in Modelling
%\item Decomposition
%\item More Global Constraints
%\item Methodology
%\item What is missing?
\end{itemize}
\end{frame}

% \begin{frame}
  % \frametitle{Using MiniZinc IDE}
  % \begin{itemize}
  % \item Developed in the Australian NICTA project
  % \item Maintained by Monash University
  % \item Modelling tool with multiple back-end solvers
  % \item Available from \url{https://www.minizinc.org/}
  % \end{itemize}
% \end{frame}


% \begin{frame}
% \frametitle{Examples in ECLiPSe}
% \begin{itemize}
% \item Open sourced constraint programming language
% \item Development goes back to 1985
% \item ECRC, ICL, IC-Parc, PTL, Cisco
% \item \url{https://eclipseclp.org/}
% \item Specialities
% \begin{itemize}
% \item Develop new solvers for specific domains
% \item Integration with MIP
% \end{itemize}
% \item Not included in bundled MiniZinc IDE
% \item Specialized visualization tools used here
  % \begin{itemize}
  % \item CP-Viz, Simonis et al. 2010
    % \end{itemize}
% \end{itemize}
% \end{frame}

\begin{frame}
\frametitle{Tutorial Based on ECLiPSe ELearning Course}
\begin{itemize}
\item Self-study course in constraint programming
\item Supported by Cisco Systems and Silicon Valley Community Foundation
\item Multi-media format, video lectures, slides, handout etc
\item {\small \url{https://eclipseclp.org/ELearning/index.html}}
\end{itemize}
\end{frame}

\begin{frame}
\frametitle{Also Part of CRT-AI Constraint Week}
\begin{itemize}
\item Annual one week course on CP and Optimization in Ireland
\item Part of national training program for PhD students in AI
\item \url{https://www.crt-ai.ie/}
\end{itemize}
\end{frame}



\begin{frame}
\frametitle{Constraint Programming - in a nutshell}
\begin{itemize}
\item Declarative description of problems with
\begin{itemize}
\item {\em Variables} which range over (finite) sets of values
\item {\em Constraints} over subsets of variables which restrict possible value combinations
\item A {\em solution} is a value assignment which satisfies all constraints
\end{itemize}

\item Constraint propagation/reasoning
\begin{itemize}
\item Removing inconsistent values for variables
\item Detect failure if constraint can not be satisfied
\item Interaction of constraints via shared variables
\item Incomplete
\end{itemize}

\item Search
\begin{itemize}
\item User controlled assignment of values to variables
\item Each step triggers constraint propagation 
\end{itemize}
\item Different domains require/allow different methods
\end{itemize}
\end{frame}

\begin{frame}
  \frametitle{Constraint Programming is Different}
  \begin{itemize}
  \item Declarative Programming
    \begin{itemize}
    \item Concentrate on what you want
      \item Not how to get there
      \item Program != Algorithm
      \item Program = Model
    \end{itemize}
    \item Applied to Combinatorial Problems
      \begin{itemize}
        \item No complete polynomial algorithms known (exist?)
        \item CP less ad-hoc than heuristics
        \item Models can evolve
  \end{itemize}
  \end{itemize}
  \end{frame}
    
\begin{frame}
  \frametitle{A Subtractive Process}
  \begin{textblock}{4}(8,-3)
    \includegraphics[width=4cm]{../introduction/images/stages}
  \end{textblock}
  \vfill
  \begin{quote}
    ``Oh, bosh, as Mr. Ruskin says. Sculpture, per se, is the simplest thing in the world. All you have to do is to take a big chunk of marble and a hammer and chisel, make up your mind what you are about to create and chip off all the marble you don't want.''-Paris Gaulois.
  \end{quote}
  
  {\tiny Source: \url{https://quoteinvestigator.com/2014/06/22/chip-away/}}
\end{frame}

\begin{frame}
\frametitle{Basic Process}
\begin{center}
\begin{tikzpicture}
\node[shape=rectangle,fill=pantone127-4] (problem) at (2,5) {Problem};
\node (human) at (2,4) {Human};
\node[shape=rectangle,fill=pantone127-4] (model) at (2,3) {Model};
\node (solver) at (2,2) {Constraint Solver/Search};
\node[shape=rectangle,fill=pantone157-8] (solution) at (2,1) {Solution};
\draw[-] (problem) -- (human);
\draw[->] (human) -- (model);
\draw[-] (model) -- (solver);
\draw[->] (solver) -- (solution);
\end{tikzpicture}
\end{center}
\end{frame}

\begin{frame}
\frametitle{More Realistic}
\begin{center}
\begin{tikzpicture}[xscale=4.0]
\node[shape=rectangle,fill=pantone127-4] (problem) at (2,5) {Problem};
\node (human) at (2,4) {Human};
\node[shape=rectangle,fill=pantone127-4] (model) at (2,3) {Model};
\node (solver) at (2,2) {Constraint Solver/Search};
\node[shape=rectangle,fill=pantone24-8] (hangs) at (1,1) {Hangs};
\node[shape=rectangle,fill=pantone157-8] (solution) at (2,1) {Solution};
\node[shape=rectangle,fill=pantone24-8] (wrong) at (3,1) {Wrong Solution};
\draw[-] (problem) -- (human);
\draw[->] (human) -- (model);
\draw[-] (model) -- (solver);
\draw[->] (solver) -- (hangs);
\draw[->] (solver) -- (solution);
\draw[->] (solver) -- (wrong);
\draw[->] (hangs) -- (1,4) -- (human);
\draw[->] (wrong) -- (3,4) -- (human);
\end{tikzpicture}
\end{center}
\end{frame}

\begin{frame}
\frametitle{Dual Role of Model}
\begin{itemize}
\item Allows Human to Express Problem
\begin{itemize}
\item Close to Problem Domain
\item Constraints as Abstractions
\end{itemize}
\item Allows Solver to Execute
\begin{itemize}
\item Variables as Communication Mechanism
\item Constraints as Algorithms
\end{itemize}
\end{itemize}
\end{frame}

\begin{frame}
\frametitle{Modelling Frameworks}
\begin{itemize}
\item \href{https://www.minizinc.org/}{MiniZinc} (NICTA, Monash University, Australia)
\item \href{https://github.com/eomahony/Numberjack}{NumberJack} (Insight, Ireland) 
\item \href{https://www-users.york.ac.uk/peter.nightingale/savilerow//}{EssencePrime/SavilleRow} (UK) 
\item \href{https://cpmpy.readthedocs.io/en/latest/}{CPMpy} (KU Leuven)
\item Allow use of multiple back-end solvers
\item Compile model into variants for each solver
\item A priori solver independent model(CP, MIP, SAT)
\end{itemize}
\end{frame} 

\begin{frame}
\frametitle{Framework Process}
\begin{center}
\begin{tikzpicture}[xscale=2]
\node[shape=rectangle,fill=pantone127-4] (problem) at (2.5,5) {Problem};
\node (human) at (2.5,4) {Human};
\node[shape=rectangle,fill=pantone127-4] (model) at (2.5,3) {Model};
\node (compile) at (2.5,2) {Compile/Reformulate};
\node (solver1) at (1,1) {CP};
\node (solver2) at (2,1) {MIP};
\node (solver3) at (3,1) {SAT};
\node (solver4) at (4,1) {Other};
\node[shape=rectangle,fill=pantone157-8] (solution1) at (1,0) {Solution};
\node[shape=rectangle,fill=pantone157-8] (solution2) at (2,0) {Solution};
\node[shape=rectangle,fill=pantone157-8] (solution3) at (3,0) {Solution};
\node[shape=rectangle,fill=pantone157-8] (solution4) at (4,0) {Solution};
\draw[-] (problem) -- (human);
\draw[->] (human) -- (model);
\draw[-] (model) -- (compile);
\draw[-] (compile) -- (solver1);
\draw[-] (compile) -- (solver2);
\draw[-] (compile) -- (solver3);
\draw[-] (compile) -- (solver4);
\draw[->] (solver1) -- (solution1);
\draw[->] (solver2) -- (solution2);
\draw[->] (solver3) -- (solution3);
\draw[->] (solver4) -- (solution4);
\end{tikzpicture}
\end{center}
\end{frame}

% \begin{frame}
  % \frametitle{Do It Now!}
  % \begin{itemize}
  % \item Download and install Minizinc
  % \item \url{https://www.minizinc.org/}
  % \end{itemize}
% \end{frame}


\begin{frame}
\frametitle{Why use Puzzles as Examples?}
\begin{itemize}
\item Easy to understand the problem
\item Solvable by hand without specialized knowledge
\item Possible to compare automated to manual solving process
\end{itemize}

\begin{quote}
The puzzle, though inanimate, is presented as a solvable problem without lasting negative consequences, a very low-risk low-reward situation. By being a puzzle, the object is attempting to convince the user that it must be completed.
\end{quote}
{\tiny Source: \href{https://www.everydayrhetoric.com/2020/03/the-rhetoric-of-puzzles/}{Every Day Rhetoric}}
\end{frame}


\mode<all>{
\part{Basic Constraint Propagation}
}
\begin{frame}
\partpage
\end{frame}
\mode<all>{
\input{../sendmore/body}
}

\mode<all>{
\part{Global Constraints}
}
\begin{frame}
\partpage
\end{frame}
\mode<all>{
\input{../sudoku/body}
}

\clearpage
\mode<all>{
\part{Customizing Search}
}
\begin{frame}
\partpage
\end{frame}
\mode<all>{
\input{../nqueen/body}
}


%% \clearpage
%% \mode<all>{
%%   \part{Problem Decomposition}
%% }
%% \begin{frame}
%%   \partpage
%% \end{frame}
%% \mode<all>{
%%   \input{../party/body}
%%   }

%% \clearpage
%% \mode<all>{
%%   \part{More Global Constraints}
%% }
%% \begin{frame}
%%   \partpage
%% \end{frame}
%% \mode<all>{
%%   \input{../car/body}
%%   }

%% \clearpage
%% \mode<all>{
%%   \part{Methodology}
%% }
%% \begin{frame}
%%   \partpage
%% \end{frame}
%% \mode<all>{
%%   \input{../methodology/body}
%%   }

\clearpage
\mode<all>{
\part{What is missing?}
}
\begin{frame}
\partpage
\end{frame}

\begin{frame}
\frametitle{Many Specialized Topics}
\begin{itemize}
\item How to design efficient core engine
\item Hybrids with LP/MIP tools
\item Hybrids with SAT
\item Symmetry breaking
\item Use of MDD/BDD to encode sets of solutions
\item High level modelling tools
\item Debugging/visualization
\end{itemize}
\end{frame}

\begin{frame}
\frametitle{Reformulation}
\begin{itemize}
\item Just because the user has modelled it this way, it doesn't mean we have to solve it that way
\begin{itemize}
\item Replace some constraint(s) by other, equivalent constraints
\item Because we don't have that constraint in our system
\item For performance
\end{itemize}
\end{itemize}
\end{frame}

\begin{frame}
\frametitle{Learning}
\begin{itemize}
\item While solving the problem we can learn how to strengthen the model/search 
\begin{itemize}
\item Understand which constraints/method contribute to propagation and change schedule
\item Learn no-good constraints by explaining failure
\item Adapt search strategy based on search experience
\end{itemize}
\end{itemize}
\end{frame}

%% \begin{frame}
%% \frametitle{Refined Process}
%% \begin{center}
%% \begin{tikzpicture}
%% \node[shape=rectangle,fill=pantone127-4] (problem) at (3,7) {Problem};
%% \node (human) at (3,6) {Human};
%% \node[shape=rectangle,fill=pantone127-4] (model) at (3,5) {User Model};
%% \node (reformulation) at (3,4) {Reformulation};
%% \node[shape=rectangle,fill=pantone127-4] (implementation) at (3,3) {Implementation Model};
%% \node (solver) at (3,2) {Constraint Solver/Search};
%% \node[shape=rectangle,fill=pantone157-8] (solution) at (3,1) {Solution};
%% \draw[-] (problem) -- (human);
%% \draw[->] (human) -- (model);
%% \draw[-] (model) -- (reformulation);
%% \draw[->] (reformulation) -- (implementation);
%% \draw[-] (implementation) -- (solver);
%% \draw[->] (solver) -- (0,2) -- (0,3) -- (implementation);
%% \draw[->] (solver) -- (solution);
%% \end{tikzpicture}
%% \end{center}
%% \end{frame}

%\begin{frame}
%\frametitle{What is CP actually used for?}
%\url{http://hsimonis.wordpress.com}
%\includegraphics[width=10cm]{../introduction/images/blog}
%\end{frame}

%% \begin{frame}
%% \frametitle{Exercises}
%% \begin{itemize}
%% \item Install Constraint Solver of your Choice
%% \item MiniZinc (MiningZinc) will be used in later sessions
%% \item Run Sendmoremoney, Sudoku, Queens examples
%% \item Can you choose which propagator to use for alldifferent?
%% \item Can you use different search strategies?
%% \end{itemize}
%% \end{frame}

\begin{frame}
\frametitle{More Learning Resources}
\begin{itemize}
\item Survey of Methods, Resources, and Formats for Teaching Constraint Programming
\begin{itemize}
\item by Tejas Santanam, Helmut Simonis
\item \url{https://doi.org/10.48550/arXiv.2403.12717}
\item Based on survey of community for WTCP 2023
\item \url{https://hsimonis.github.io/WTCP2023/}
\end{itemize}
\end{itemize}
\end{frame}


%%   }

%% \clearpage
%% \mode<all>{
%%   \part{More Global Constraints}
%% }
%% \begin{frame}
%%   \partpage
%% \end{frame}
%% \mode<all>{
%%   \begin{frame}
\frametitle{Objectives}
\begin{itemize}
\item Overview of Core Constraint Programming
\item Three Main Concepts
\begin{itemize}
\item Constraint Propagation
\item Global Constraints
\item Customizing Search
\end{itemize}
%\item Get Some Experience with MiniZinc
\item Topics will be treated in more detail in later parts of the school
\item Based on Examples, not Formal Description
\end{itemize}
\end{frame}

\begin{frame}
\frametitle{Outline}
\begin{itemize}
\item Why Constraint Programming?
\item Constraint Propagation
\item Global Constraints
\item Customizing Search
%\item Choices in Modelling
%\item Decomposition
%\item More Global Constraints
%\item Methodology
%\item What is missing?
\end{itemize}
\end{frame}

% \begin{frame}
  % \frametitle{Using MiniZinc IDE}
  % \begin{itemize}
  % \item Developed in the Australian NICTA project
  % \item Maintained by Monash University
  % \item Modelling tool with multiple back-end solvers
  % \item Available from \url{https://www.minizinc.org/}
  % \end{itemize}
% \end{frame}


% \begin{frame}
% \frametitle{Examples in ECLiPSe}
% \begin{itemize}
% \item Open sourced constraint programming language
% \item Development goes back to 1985
% \item ECRC, ICL, IC-Parc, PTL, Cisco
% \item \url{https://eclipseclp.org/}
% \item Specialities
% \begin{itemize}
% \item Develop new solvers for specific domains
% \item Integration with MIP
% \end{itemize}
% \item Not included in bundled MiniZinc IDE
% \item Specialized visualization tools used here
  % \begin{itemize}
  % \item CP-Viz, Simonis et al. 2010
    % \end{itemize}
% \end{itemize}
% \end{frame}

\begin{frame}
\frametitle{Tutorial Based on ECLiPSe ELearning Course}
\begin{itemize}
\item Self-study course in constraint programming
\item Supported by Cisco Systems and Silicon Valley Community Foundation
\item Multi-media format, video lectures, slides, handout etc
\item {\small \url{https://eclipseclp.org/ELearning/index.html}}
\end{itemize}
\end{frame}

\begin{frame}
\frametitle{Also Part of CRT-AI Constraint Week}
\begin{itemize}
\item Annual one week course on CP and Optimization in Ireland
\item Part of national training program for PhD students in AI
\item \url{https://www.crt-ai.ie/}
\end{itemize}
\end{frame}



\begin{frame}
\frametitle{Constraint Programming - in a nutshell}
\begin{itemize}
\item Declarative description of problems with
\begin{itemize}
\item {\em Variables} which range over (finite) sets of values
\item {\em Constraints} over subsets of variables which restrict possible value combinations
\item A {\em solution} is a value assignment which satisfies all constraints
\end{itemize}

\item Constraint propagation/reasoning
\begin{itemize}
\item Removing inconsistent values for variables
\item Detect failure if constraint can not be satisfied
\item Interaction of constraints via shared variables
\item Incomplete
\end{itemize}

\item Search
\begin{itemize}
\item User controlled assignment of values to variables
\item Each step triggers constraint propagation 
\end{itemize}
\item Different domains require/allow different methods
\end{itemize}
\end{frame}

\begin{frame}
  \frametitle{Constraint Programming is Different}
  \begin{itemize}
  \item Declarative Programming
    \begin{itemize}
    \item Concentrate on what you want
      \item Not how to get there
      \item Program != Algorithm
      \item Program = Model
    \end{itemize}
    \item Applied to Combinatorial Problems
      \begin{itemize}
        \item No complete polynomial algorithms known (exist?)
        \item CP less ad-hoc than heuristics
        \item Models can evolve
  \end{itemize}
  \end{itemize}
  \end{frame}
    
\begin{frame}
  \frametitle{A Subtractive Process}
  \begin{textblock}{4}(8,-3)
    \includegraphics[width=4cm]{../introduction/images/stages}
  \end{textblock}
  \vfill
  \begin{quote}
    ``Oh, bosh, as Mr. Ruskin says. Sculpture, per se, is the simplest thing in the world. All you have to do is to take a big chunk of marble and a hammer and chisel, make up your mind what you are about to create and chip off all the marble you don't want.''-Paris Gaulois.
  \end{quote}
  
  {\tiny Source: \url{https://quoteinvestigator.com/2014/06/22/chip-away/}}
\end{frame}

\begin{frame}
\frametitle{Basic Process}
\begin{center}
\begin{tikzpicture}
\node[shape=rectangle,fill=pantone127-4] (problem) at (2,5) {Problem};
\node (human) at (2,4) {Human};
\node[shape=rectangle,fill=pantone127-4] (model) at (2,3) {Model};
\node (solver) at (2,2) {Constraint Solver/Search};
\node[shape=rectangle,fill=pantone157-8] (solution) at (2,1) {Solution};
\draw[-] (problem) -- (human);
\draw[->] (human) -- (model);
\draw[-] (model) -- (solver);
\draw[->] (solver) -- (solution);
\end{tikzpicture}
\end{center}
\end{frame}

\begin{frame}
\frametitle{More Realistic}
\begin{center}
\begin{tikzpicture}[xscale=4.0]
\node[shape=rectangle,fill=pantone127-4] (problem) at (2,5) {Problem};
\node (human) at (2,4) {Human};
\node[shape=rectangle,fill=pantone127-4] (model) at (2,3) {Model};
\node (solver) at (2,2) {Constraint Solver/Search};
\node[shape=rectangle,fill=pantone24-8] (hangs) at (1,1) {Hangs};
\node[shape=rectangle,fill=pantone157-8] (solution) at (2,1) {Solution};
\node[shape=rectangle,fill=pantone24-8] (wrong) at (3,1) {Wrong Solution};
\draw[-] (problem) -- (human);
\draw[->] (human) -- (model);
\draw[-] (model) -- (solver);
\draw[->] (solver) -- (hangs);
\draw[->] (solver) -- (solution);
\draw[->] (solver) -- (wrong);
\draw[->] (hangs) -- (1,4) -- (human);
\draw[->] (wrong) -- (3,4) -- (human);
\end{tikzpicture}
\end{center}
\end{frame}

\begin{frame}
\frametitle{Dual Role of Model}
\begin{itemize}
\item Allows Human to Express Problem
\begin{itemize}
\item Close to Problem Domain
\item Constraints as Abstractions
\end{itemize}
\item Allows Solver to Execute
\begin{itemize}
\item Variables as Communication Mechanism
\item Constraints as Algorithms
\end{itemize}
\end{itemize}
\end{frame}

\begin{frame}
\frametitle{Modelling Frameworks}
\begin{itemize}
\item \href{https://www.minizinc.org/}{MiniZinc} (NICTA, Monash University, Australia)
\item \href{https://github.com/eomahony/Numberjack}{NumberJack} (Insight, Ireland) 
\item \href{https://www-users.york.ac.uk/peter.nightingale/savilerow//}{EssencePrime/SavilleRow} (UK) 
\item \href{https://cpmpy.readthedocs.io/en/latest/}{CPMpy} (KU Leuven)
\item Allow use of multiple back-end solvers
\item Compile model into variants for each solver
\item A priori solver independent model(CP, MIP, SAT)
\end{itemize}
\end{frame} 

\begin{frame}
\frametitle{Framework Process}
\begin{center}
\begin{tikzpicture}[xscale=2]
\node[shape=rectangle,fill=pantone127-4] (problem) at (2.5,5) {Problem};
\node (human) at (2.5,4) {Human};
\node[shape=rectangle,fill=pantone127-4] (model) at (2.5,3) {Model};
\node (compile) at (2.5,2) {Compile/Reformulate};
\node (solver1) at (1,1) {CP};
\node (solver2) at (2,1) {MIP};
\node (solver3) at (3,1) {SAT};
\node (solver4) at (4,1) {Other};
\node[shape=rectangle,fill=pantone157-8] (solution1) at (1,0) {Solution};
\node[shape=rectangle,fill=pantone157-8] (solution2) at (2,0) {Solution};
\node[shape=rectangle,fill=pantone157-8] (solution3) at (3,0) {Solution};
\node[shape=rectangle,fill=pantone157-8] (solution4) at (4,0) {Solution};
\draw[-] (problem) -- (human);
\draw[->] (human) -- (model);
\draw[-] (model) -- (compile);
\draw[-] (compile) -- (solver1);
\draw[-] (compile) -- (solver2);
\draw[-] (compile) -- (solver3);
\draw[-] (compile) -- (solver4);
\draw[->] (solver1) -- (solution1);
\draw[->] (solver2) -- (solution2);
\draw[->] (solver3) -- (solution3);
\draw[->] (solver4) -- (solution4);
\end{tikzpicture}
\end{center}
\end{frame}

% \begin{frame}
  % \frametitle{Do It Now!}
  % \begin{itemize}
  % \item Download and install Minizinc
  % \item \url{https://www.minizinc.org/}
  % \end{itemize}
% \end{frame}


\begin{frame}
\frametitle{Why use Puzzles as Examples?}
\begin{itemize}
\item Easy to understand the problem
\item Solvable by hand without specialized knowledge
\item Possible to compare automated to manual solving process
\end{itemize}

\begin{quote}
The puzzle, though inanimate, is presented as a solvable problem without lasting negative consequences, a very low-risk low-reward situation. By being a puzzle, the object is attempting to convince the user that it must be completed.
\end{quote}
{\tiny Source: \href{https://www.everydayrhetoric.com/2020/03/the-rhetoric-of-puzzles/}{Every Day Rhetoric}}
\end{frame}


\mode<all>{
\part{Basic Constraint Propagation}
}
\begin{frame}
\partpage
\end{frame}
\mode<all>{
\input{../sendmore/body}
}

\mode<all>{
\part{Global Constraints}
}
\begin{frame}
\partpage
\end{frame}
\mode<all>{
\input{../sudoku/body}
}

\clearpage
\mode<all>{
\part{Customizing Search}
}
\begin{frame}
\partpage
\end{frame}
\mode<all>{
\input{../nqueen/body}
}


%% \clearpage
%% \mode<all>{
%%   \part{Problem Decomposition}
%% }
%% \begin{frame}
%%   \partpage
%% \end{frame}
%% \mode<all>{
%%   \input{../party/body}
%%   }

%% \clearpage
%% \mode<all>{
%%   \part{More Global Constraints}
%% }
%% \begin{frame}
%%   \partpage
%% \end{frame}
%% \mode<all>{
%%   \input{../car/body}
%%   }

%% \clearpage
%% \mode<all>{
%%   \part{Methodology}
%% }
%% \begin{frame}
%%   \partpage
%% \end{frame}
%% \mode<all>{
%%   \input{../methodology/body}
%%   }

\clearpage
\mode<all>{
\part{What is missing?}
}
\begin{frame}
\partpage
\end{frame}

\begin{frame}
\frametitle{Many Specialized Topics}
\begin{itemize}
\item How to design efficient core engine
\item Hybrids with LP/MIP tools
\item Hybrids with SAT
\item Symmetry breaking
\item Use of MDD/BDD to encode sets of solutions
\item High level modelling tools
\item Debugging/visualization
\end{itemize}
\end{frame}

\begin{frame}
\frametitle{Reformulation}
\begin{itemize}
\item Just because the user has modelled it this way, it doesn't mean we have to solve it that way
\begin{itemize}
\item Replace some constraint(s) by other, equivalent constraints
\item Because we don't have that constraint in our system
\item For performance
\end{itemize}
\end{itemize}
\end{frame}

\begin{frame}
\frametitle{Learning}
\begin{itemize}
\item While solving the problem we can learn how to strengthen the model/search 
\begin{itemize}
\item Understand which constraints/method contribute to propagation and change schedule
\item Learn no-good constraints by explaining failure
\item Adapt search strategy based on search experience
\end{itemize}
\end{itemize}
\end{frame}

%% \begin{frame}
%% \frametitle{Refined Process}
%% \begin{center}
%% \begin{tikzpicture}
%% \node[shape=rectangle,fill=pantone127-4] (problem) at (3,7) {Problem};
%% \node (human) at (3,6) {Human};
%% \node[shape=rectangle,fill=pantone127-4] (model) at (3,5) {User Model};
%% \node (reformulation) at (3,4) {Reformulation};
%% \node[shape=rectangle,fill=pantone127-4] (implementation) at (3,3) {Implementation Model};
%% \node (solver) at (3,2) {Constraint Solver/Search};
%% \node[shape=rectangle,fill=pantone157-8] (solution) at (3,1) {Solution};
%% \draw[-] (problem) -- (human);
%% \draw[->] (human) -- (model);
%% \draw[-] (model) -- (reformulation);
%% \draw[->] (reformulation) -- (implementation);
%% \draw[-] (implementation) -- (solver);
%% \draw[->] (solver) -- (0,2) -- (0,3) -- (implementation);
%% \draw[->] (solver) -- (solution);
%% \end{tikzpicture}
%% \end{center}
%% \end{frame}

%\begin{frame}
%\frametitle{What is CP actually used for?}
%\url{http://hsimonis.wordpress.com}
%\includegraphics[width=10cm]{../introduction/images/blog}
%\end{frame}

%% \begin{frame}
%% \frametitle{Exercises}
%% \begin{itemize}
%% \item Install Constraint Solver of your Choice
%% \item MiniZinc (MiningZinc) will be used in later sessions
%% \item Run Sendmoremoney, Sudoku, Queens examples
%% \item Can you choose which propagator to use for alldifferent?
%% \item Can you use different search strategies?
%% \end{itemize}
%% \end{frame}

\begin{frame}
\frametitle{More Learning Resources}
\begin{itemize}
\item Survey of Methods, Resources, and Formats for Teaching Constraint Programming
\begin{itemize}
\item by Tejas Santanam, Helmut Simonis
\item \url{https://doi.org/10.48550/arXiv.2403.12717}
\item Based on survey of community for WTCP 2023
\item \url{https://hsimonis.github.io/WTCP2023/}
\end{itemize}
\end{itemize}
\end{frame}


%%   }

%% \clearpage
%% \mode<all>{
%%   \part{Methodology}
%% }
%% \begin{frame}
%%   \partpage
%% \end{frame}
%% \mode<all>{
%%   \begin{frame}
\frametitle{Objectives}
\begin{itemize}
\item Overview of Core Constraint Programming
\item Three Main Concepts
\begin{itemize}
\item Constraint Propagation
\item Global Constraints
\item Customizing Search
\end{itemize}
%\item Get Some Experience with MiniZinc
\item Topics will be treated in more detail in later parts of the school
\item Based on Examples, not Formal Description
\end{itemize}
\end{frame}

\begin{frame}
\frametitle{Outline}
\begin{itemize}
\item Why Constraint Programming?
\item Constraint Propagation
\item Global Constraints
\item Customizing Search
%\item Choices in Modelling
%\item Decomposition
%\item More Global Constraints
%\item Methodology
%\item What is missing?
\end{itemize}
\end{frame}

% \begin{frame}
  % \frametitle{Using MiniZinc IDE}
  % \begin{itemize}
  % \item Developed in the Australian NICTA project
  % \item Maintained by Monash University
  % \item Modelling tool with multiple back-end solvers
  % \item Available from \url{https://www.minizinc.org/}
  % \end{itemize}
% \end{frame}


% \begin{frame}
% \frametitle{Examples in ECLiPSe}
% \begin{itemize}
% \item Open sourced constraint programming language
% \item Development goes back to 1985
% \item ECRC, ICL, IC-Parc, PTL, Cisco
% \item \url{https://eclipseclp.org/}
% \item Specialities
% \begin{itemize}
% \item Develop new solvers for specific domains
% \item Integration with MIP
% \end{itemize}
% \item Not included in bundled MiniZinc IDE
% \item Specialized visualization tools used here
  % \begin{itemize}
  % \item CP-Viz, Simonis et al. 2010
    % \end{itemize}
% \end{itemize}
% \end{frame}

\begin{frame}
\frametitle{Tutorial Based on ECLiPSe ELearning Course}
\begin{itemize}
\item Self-study course in constraint programming
\item Supported by Cisco Systems and Silicon Valley Community Foundation
\item Multi-media format, video lectures, slides, handout etc
\item {\small \url{https://eclipseclp.org/ELearning/index.html}}
\end{itemize}
\end{frame}

\begin{frame}
\frametitle{Also Part of CRT-AI Constraint Week}
\begin{itemize}
\item Annual one week course on CP and Optimization in Ireland
\item Part of national training program for PhD students in AI
\item \url{https://www.crt-ai.ie/}
\end{itemize}
\end{frame}



\begin{frame}
\frametitle{Constraint Programming - in a nutshell}
\begin{itemize}
\item Declarative description of problems with
\begin{itemize}
\item {\em Variables} which range over (finite) sets of values
\item {\em Constraints} over subsets of variables which restrict possible value combinations
\item A {\em solution} is a value assignment which satisfies all constraints
\end{itemize}

\item Constraint propagation/reasoning
\begin{itemize}
\item Removing inconsistent values for variables
\item Detect failure if constraint can not be satisfied
\item Interaction of constraints via shared variables
\item Incomplete
\end{itemize}

\item Search
\begin{itemize}
\item User controlled assignment of values to variables
\item Each step triggers constraint propagation 
\end{itemize}
\item Different domains require/allow different methods
\end{itemize}
\end{frame}

\begin{frame}
  \frametitle{Constraint Programming is Different}
  \begin{itemize}
  \item Declarative Programming
    \begin{itemize}
    \item Concentrate on what you want
      \item Not how to get there
      \item Program != Algorithm
      \item Program = Model
    \end{itemize}
    \item Applied to Combinatorial Problems
      \begin{itemize}
        \item No complete polynomial algorithms known (exist?)
        \item CP less ad-hoc than heuristics
        \item Models can evolve
  \end{itemize}
  \end{itemize}
  \end{frame}
    
\begin{frame}
  \frametitle{A Subtractive Process}
  \begin{textblock}{4}(8,-3)
    \includegraphics[width=4cm]{../introduction/images/stages}
  \end{textblock}
  \vfill
  \begin{quote}
    ``Oh, bosh, as Mr. Ruskin says. Sculpture, per se, is the simplest thing in the world. All you have to do is to take a big chunk of marble and a hammer and chisel, make up your mind what you are about to create and chip off all the marble you don't want.''-Paris Gaulois.
  \end{quote}
  
  {\tiny Source: \url{https://quoteinvestigator.com/2014/06/22/chip-away/}}
\end{frame}

\begin{frame}
\frametitle{Basic Process}
\begin{center}
\begin{tikzpicture}
\node[shape=rectangle,fill=pantone127-4] (problem) at (2,5) {Problem};
\node (human) at (2,4) {Human};
\node[shape=rectangle,fill=pantone127-4] (model) at (2,3) {Model};
\node (solver) at (2,2) {Constraint Solver/Search};
\node[shape=rectangle,fill=pantone157-8] (solution) at (2,1) {Solution};
\draw[-] (problem) -- (human);
\draw[->] (human) -- (model);
\draw[-] (model) -- (solver);
\draw[->] (solver) -- (solution);
\end{tikzpicture}
\end{center}
\end{frame}

\begin{frame}
\frametitle{More Realistic}
\begin{center}
\begin{tikzpicture}[xscale=4.0]
\node[shape=rectangle,fill=pantone127-4] (problem) at (2,5) {Problem};
\node (human) at (2,4) {Human};
\node[shape=rectangle,fill=pantone127-4] (model) at (2,3) {Model};
\node (solver) at (2,2) {Constraint Solver/Search};
\node[shape=rectangle,fill=pantone24-8] (hangs) at (1,1) {Hangs};
\node[shape=rectangle,fill=pantone157-8] (solution) at (2,1) {Solution};
\node[shape=rectangle,fill=pantone24-8] (wrong) at (3,1) {Wrong Solution};
\draw[-] (problem) -- (human);
\draw[->] (human) -- (model);
\draw[-] (model) -- (solver);
\draw[->] (solver) -- (hangs);
\draw[->] (solver) -- (solution);
\draw[->] (solver) -- (wrong);
\draw[->] (hangs) -- (1,4) -- (human);
\draw[->] (wrong) -- (3,4) -- (human);
\end{tikzpicture}
\end{center}
\end{frame}

\begin{frame}
\frametitle{Dual Role of Model}
\begin{itemize}
\item Allows Human to Express Problem
\begin{itemize}
\item Close to Problem Domain
\item Constraints as Abstractions
\end{itemize}
\item Allows Solver to Execute
\begin{itemize}
\item Variables as Communication Mechanism
\item Constraints as Algorithms
\end{itemize}
\end{itemize}
\end{frame}

\begin{frame}
\frametitle{Modelling Frameworks}
\begin{itemize}
\item \href{https://www.minizinc.org/}{MiniZinc} (NICTA, Monash University, Australia)
\item \href{https://github.com/eomahony/Numberjack}{NumberJack} (Insight, Ireland) 
\item \href{https://www-users.york.ac.uk/peter.nightingale/savilerow//}{EssencePrime/SavilleRow} (UK) 
\item \href{https://cpmpy.readthedocs.io/en/latest/}{CPMpy} (KU Leuven)
\item Allow use of multiple back-end solvers
\item Compile model into variants for each solver
\item A priori solver independent model(CP, MIP, SAT)
\end{itemize}
\end{frame} 

\begin{frame}
\frametitle{Framework Process}
\begin{center}
\begin{tikzpicture}[xscale=2]
\node[shape=rectangle,fill=pantone127-4] (problem) at (2.5,5) {Problem};
\node (human) at (2.5,4) {Human};
\node[shape=rectangle,fill=pantone127-4] (model) at (2.5,3) {Model};
\node (compile) at (2.5,2) {Compile/Reformulate};
\node (solver1) at (1,1) {CP};
\node (solver2) at (2,1) {MIP};
\node (solver3) at (3,1) {SAT};
\node (solver4) at (4,1) {Other};
\node[shape=rectangle,fill=pantone157-8] (solution1) at (1,0) {Solution};
\node[shape=rectangle,fill=pantone157-8] (solution2) at (2,0) {Solution};
\node[shape=rectangle,fill=pantone157-8] (solution3) at (3,0) {Solution};
\node[shape=rectangle,fill=pantone157-8] (solution4) at (4,0) {Solution};
\draw[-] (problem) -- (human);
\draw[->] (human) -- (model);
\draw[-] (model) -- (compile);
\draw[-] (compile) -- (solver1);
\draw[-] (compile) -- (solver2);
\draw[-] (compile) -- (solver3);
\draw[-] (compile) -- (solver4);
\draw[->] (solver1) -- (solution1);
\draw[->] (solver2) -- (solution2);
\draw[->] (solver3) -- (solution3);
\draw[->] (solver4) -- (solution4);
\end{tikzpicture}
\end{center}
\end{frame}

% \begin{frame}
  % \frametitle{Do It Now!}
  % \begin{itemize}
  % \item Download and install Minizinc
  % \item \url{https://www.minizinc.org/}
  % \end{itemize}
% \end{frame}


\begin{frame}
\frametitle{Why use Puzzles as Examples?}
\begin{itemize}
\item Easy to understand the problem
\item Solvable by hand without specialized knowledge
\item Possible to compare automated to manual solving process
\end{itemize}

\begin{quote}
The puzzle, though inanimate, is presented as a solvable problem without lasting negative consequences, a very low-risk low-reward situation. By being a puzzle, the object is attempting to convince the user that it must be completed.
\end{quote}
{\tiny Source: \href{https://www.everydayrhetoric.com/2020/03/the-rhetoric-of-puzzles/}{Every Day Rhetoric}}
\end{frame}


\mode<all>{
\part{Basic Constraint Propagation}
}
\begin{frame}
\partpage
\end{frame}
\mode<all>{
\input{../sendmore/body}
}

\mode<all>{
\part{Global Constraints}
}
\begin{frame}
\partpage
\end{frame}
\mode<all>{
\input{../sudoku/body}
}

\clearpage
\mode<all>{
\part{Customizing Search}
}
\begin{frame}
\partpage
\end{frame}
\mode<all>{
\input{../nqueen/body}
}


%% \clearpage
%% \mode<all>{
%%   \part{Problem Decomposition}
%% }
%% \begin{frame}
%%   \partpage
%% \end{frame}
%% \mode<all>{
%%   \input{../party/body}
%%   }

%% \clearpage
%% \mode<all>{
%%   \part{More Global Constraints}
%% }
%% \begin{frame}
%%   \partpage
%% \end{frame}
%% \mode<all>{
%%   \input{../car/body}
%%   }

%% \clearpage
%% \mode<all>{
%%   \part{Methodology}
%% }
%% \begin{frame}
%%   \partpage
%% \end{frame}
%% \mode<all>{
%%   \input{../methodology/body}
%%   }

\clearpage
\mode<all>{
\part{What is missing?}
}
\begin{frame}
\partpage
\end{frame}

\begin{frame}
\frametitle{Many Specialized Topics}
\begin{itemize}
\item How to design efficient core engine
\item Hybrids with LP/MIP tools
\item Hybrids with SAT
\item Symmetry breaking
\item Use of MDD/BDD to encode sets of solutions
\item High level modelling tools
\item Debugging/visualization
\end{itemize}
\end{frame}

\begin{frame}
\frametitle{Reformulation}
\begin{itemize}
\item Just because the user has modelled it this way, it doesn't mean we have to solve it that way
\begin{itemize}
\item Replace some constraint(s) by other, equivalent constraints
\item Because we don't have that constraint in our system
\item For performance
\end{itemize}
\end{itemize}
\end{frame}

\begin{frame}
\frametitle{Learning}
\begin{itemize}
\item While solving the problem we can learn how to strengthen the model/search 
\begin{itemize}
\item Understand which constraints/method contribute to propagation and change schedule
\item Learn no-good constraints by explaining failure
\item Adapt search strategy based on search experience
\end{itemize}
\end{itemize}
\end{frame}

%% \begin{frame}
%% \frametitle{Refined Process}
%% \begin{center}
%% \begin{tikzpicture}
%% \node[shape=rectangle,fill=pantone127-4] (problem) at (3,7) {Problem};
%% \node (human) at (3,6) {Human};
%% \node[shape=rectangle,fill=pantone127-4] (model) at (3,5) {User Model};
%% \node (reformulation) at (3,4) {Reformulation};
%% \node[shape=rectangle,fill=pantone127-4] (implementation) at (3,3) {Implementation Model};
%% \node (solver) at (3,2) {Constraint Solver/Search};
%% \node[shape=rectangle,fill=pantone157-8] (solution) at (3,1) {Solution};
%% \draw[-] (problem) -- (human);
%% \draw[->] (human) -- (model);
%% \draw[-] (model) -- (reformulation);
%% \draw[->] (reformulation) -- (implementation);
%% \draw[-] (implementation) -- (solver);
%% \draw[->] (solver) -- (0,2) -- (0,3) -- (implementation);
%% \draw[->] (solver) -- (solution);
%% \end{tikzpicture}
%% \end{center}
%% \end{frame}

%\begin{frame}
%\frametitle{What is CP actually used for?}
%\url{http://hsimonis.wordpress.com}
%\includegraphics[width=10cm]{../introduction/images/blog}
%\end{frame}

%% \begin{frame}
%% \frametitle{Exercises}
%% \begin{itemize}
%% \item Install Constraint Solver of your Choice
%% \item MiniZinc (MiningZinc) will be used in later sessions
%% \item Run Sendmoremoney, Sudoku, Queens examples
%% \item Can you choose which propagator to use for alldifferent?
%% \item Can you use different search strategies?
%% \end{itemize}
%% \end{frame}

\begin{frame}
\frametitle{More Learning Resources}
\begin{itemize}
\item Survey of Methods, Resources, and Formats for Teaching Constraint Programming
\begin{itemize}
\item by Tejas Santanam, Helmut Simonis
\item \url{https://doi.org/10.48550/arXiv.2403.12717}
\item Based on survey of community for WTCP 2023
\item \url{https://hsimonis.github.io/WTCP2023/}
\end{itemize}
\end{itemize}
\end{frame}


%%   }

\clearpage
\mode<all>{
\part{What is missing?}
}
\begin{frame}
\partpage
\end{frame}

\begin{frame}
\frametitle{Many Specialized Topics}
\begin{itemize}
\item How to design efficient core engine
\item Hybrids with LP/MIP tools
\item Hybrids with SAT
\item Symmetry breaking
\item Use of MDD/BDD to encode sets of solutions
\item High level modelling tools
\item Debugging/visualization
\end{itemize}
\end{frame}

\begin{frame}
\frametitle{Reformulation}
\begin{itemize}
\item Just because the user has modelled it this way, it doesn't mean we have to solve it that way
\begin{itemize}
\item Replace some constraint(s) by other, equivalent constraints
\item Because we don't have that constraint in our system
\item For performance
\end{itemize}
\end{itemize}
\end{frame}

\begin{frame}
\frametitle{Learning}
\begin{itemize}
\item While solving the problem we can learn how to strengthen the model/search 
\begin{itemize}
\item Understand which constraints/method contribute to propagation and change schedule
\item Learn no-good constraints by explaining failure
\item Adapt search strategy based on search experience
\end{itemize}
\end{itemize}
\end{frame}

%% \begin{frame}
%% \frametitle{Refined Process}
%% \begin{center}
%% \begin{tikzpicture}
%% \node[shape=rectangle,fill=pantone127-4] (problem) at (3,7) {Problem};
%% \node (human) at (3,6) {Human};
%% \node[shape=rectangle,fill=pantone127-4] (model) at (3,5) {User Model};
%% \node (reformulation) at (3,4) {Reformulation};
%% \node[shape=rectangle,fill=pantone127-4] (implementation) at (3,3) {Implementation Model};
%% \node (solver) at (3,2) {Constraint Solver/Search};
%% \node[shape=rectangle,fill=pantone157-8] (solution) at (3,1) {Solution};
%% \draw[-] (problem) -- (human);
%% \draw[->] (human) -- (model);
%% \draw[-] (model) -- (reformulation);
%% \draw[->] (reformulation) -- (implementation);
%% \draw[-] (implementation) -- (solver);
%% \draw[->] (solver) -- (0,2) -- (0,3) -- (implementation);
%% \draw[->] (solver) -- (solution);
%% \end{tikzpicture}
%% \end{center}
%% \end{frame}

%\begin{frame}
%\frametitle{What is CP actually used for?}
%\url{http://hsimonis.wordpress.com}
%\includegraphics[width=10cm]{../introduction/images/blog}
%\end{frame}

%% \begin{frame}
%% \frametitle{Exercises}
%% \begin{itemize}
%% \item Install Constraint Solver of your Choice
%% \item MiniZinc (MiningZinc) will be used in later sessions
%% \item Run Sendmoremoney, Sudoku, Queens examples
%% \item Can you choose which propagator to use for alldifferent?
%% \item Can you use different search strategies?
%% \end{itemize}
%% \end{frame}

\begin{frame}
\frametitle{More Learning Resources}
\begin{itemize}
\item Survey of Methods, Resources, and Formats for Teaching Constraint Programming
\begin{itemize}
\item by Tejas Santanam, Helmut Simonis
\item \url{https://doi.org/10.48550/arXiv.2403.12717}
\item Based on survey of community for WTCP 2023
\item \url{https://hsimonis.github.io/WTCP2023/}
\end{itemize}
\end{itemize}
\end{frame}


}

\mode<all>{
\part{Global Constraints}
}
\begin{frame}
\partpage
\end{frame}
\mode<all>{
\begin{frame}
\frametitle{Objectives}
\begin{itemize}
\item Overview of Core Constraint Programming
\item Three Main Concepts
\begin{itemize}
\item Constraint Propagation
\item Global Constraints
\item Customizing Search
\end{itemize}
%\item Get Some Experience with MiniZinc
\item Topics will be treated in more detail in later parts of the school
\item Based on Examples, not Formal Description
\end{itemize}
\end{frame}

\begin{frame}
\frametitle{Outline}
\begin{itemize}
\item Why Constraint Programming?
\item Constraint Propagation
\item Global Constraints
\item Customizing Search
%\item Choices in Modelling
%\item Decomposition
%\item More Global Constraints
%\item Methodology
%\item What is missing?
\end{itemize}
\end{frame}

% \begin{frame}
  % \frametitle{Using MiniZinc IDE}
  % \begin{itemize}
  % \item Developed in the Australian NICTA project
  % \item Maintained by Monash University
  % \item Modelling tool with multiple back-end solvers
  % \item Available from \url{https://www.minizinc.org/}
  % \end{itemize}
% \end{frame}


% \begin{frame}
% \frametitle{Examples in ECLiPSe}
% \begin{itemize}
% \item Open sourced constraint programming language
% \item Development goes back to 1985
% \item ECRC, ICL, IC-Parc, PTL, Cisco
% \item \url{https://eclipseclp.org/}
% \item Specialities
% \begin{itemize}
% \item Develop new solvers for specific domains
% \item Integration with MIP
% \end{itemize}
% \item Not included in bundled MiniZinc IDE
% \item Specialized visualization tools used here
  % \begin{itemize}
  % \item CP-Viz, Simonis et al. 2010
    % \end{itemize}
% \end{itemize}
% \end{frame}

\begin{frame}
\frametitle{Tutorial Based on ECLiPSe ELearning Course}
\begin{itemize}
\item Self-study course in constraint programming
\item Supported by Cisco Systems and Silicon Valley Community Foundation
\item Multi-media format, video lectures, slides, handout etc
\item {\small \url{https://eclipseclp.org/ELearning/index.html}}
\end{itemize}
\end{frame}

\begin{frame}
\frametitle{Also Part of CRT-AI Constraint Week}
\begin{itemize}
\item Annual one week course on CP and Optimization in Ireland
\item Part of national training program for PhD students in AI
\item \url{https://www.crt-ai.ie/}
\end{itemize}
\end{frame}



\begin{frame}
\frametitle{Constraint Programming - in a nutshell}
\begin{itemize}
\item Declarative description of problems with
\begin{itemize}
\item {\em Variables} which range over (finite) sets of values
\item {\em Constraints} over subsets of variables which restrict possible value combinations
\item A {\em solution} is a value assignment which satisfies all constraints
\end{itemize}

\item Constraint propagation/reasoning
\begin{itemize}
\item Removing inconsistent values for variables
\item Detect failure if constraint can not be satisfied
\item Interaction of constraints via shared variables
\item Incomplete
\end{itemize}

\item Search
\begin{itemize}
\item User controlled assignment of values to variables
\item Each step triggers constraint propagation 
\end{itemize}
\item Different domains require/allow different methods
\end{itemize}
\end{frame}

\begin{frame}
  \frametitle{Constraint Programming is Different}
  \begin{itemize}
  \item Declarative Programming
    \begin{itemize}
    \item Concentrate on what you want
      \item Not how to get there
      \item Program != Algorithm
      \item Program = Model
    \end{itemize}
    \item Applied to Combinatorial Problems
      \begin{itemize}
        \item No complete polynomial algorithms known (exist?)
        \item CP less ad-hoc than heuristics
        \item Models can evolve
  \end{itemize}
  \end{itemize}
  \end{frame}
    
\begin{frame}
  \frametitle{A Subtractive Process}
  \begin{textblock}{4}(8,-3)
    \includegraphics[width=4cm]{../introduction/images/stages}
  \end{textblock}
  \vfill
  \begin{quote}
    ``Oh, bosh, as Mr. Ruskin says. Sculpture, per se, is the simplest thing in the world. All you have to do is to take a big chunk of marble and a hammer and chisel, make up your mind what you are about to create and chip off all the marble you don't want.''-Paris Gaulois.
  \end{quote}
  
  {\tiny Source: \url{https://quoteinvestigator.com/2014/06/22/chip-away/}}
\end{frame}

\begin{frame}
\frametitle{Basic Process}
\begin{center}
\begin{tikzpicture}
\node[shape=rectangle,fill=pantone127-4] (problem) at (2,5) {Problem};
\node (human) at (2,4) {Human};
\node[shape=rectangle,fill=pantone127-4] (model) at (2,3) {Model};
\node (solver) at (2,2) {Constraint Solver/Search};
\node[shape=rectangle,fill=pantone157-8] (solution) at (2,1) {Solution};
\draw[-] (problem) -- (human);
\draw[->] (human) -- (model);
\draw[-] (model) -- (solver);
\draw[->] (solver) -- (solution);
\end{tikzpicture}
\end{center}
\end{frame}

\begin{frame}
\frametitle{More Realistic}
\begin{center}
\begin{tikzpicture}[xscale=4.0]
\node[shape=rectangle,fill=pantone127-4] (problem) at (2,5) {Problem};
\node (human) at (2,4) {Human};
\node[shape=rectangle,fill=pantone127-4] (model) at (2,3) {Model};
\node (solver) at (2,2) {Constraint Solver/Search};
\node[shape=rectangle,fill=pantone24-8] (hangs) at (1,1) {Hangs};
\node[shape=rectangle,fill=pantone157-8] (solution) at (2,1) {Solution};
\node[shape=rectangle,fill=pantone24-8] (wrong) at (3,1) {Wrong Solution};
\draw[-] (problem) -- (human);
\draw[->] (human) -- (model);
\draw[-] (model) -- (solver);
\draw[->] (solver) -- (hangs);
\draw[->] (solver) -- (solution);
\draw[->] (solver) -- (wrong);
\draw[->] (hangs) -- (1,4) -- (human);
\draw[->] (wrong) -- (3,4) -- (human);
\end{tikzpicture}
\end{center}
\end{frame}

\begin{frame}
\frametitle{Dual Role of Model}
\begin{itemize}
\item Allows Human to Express Problem
\begin{itemize}
\item Close to Problem Domain
\item Constraints as Abstractions
\end{itemize}
\item Allows Solver to Execute
\begin{itemize}
\item Variables as Communication Mechanism
\item Constraints as Algorithms
\end{itemize}
\end{itemize}
\end{frame}

\begin{frame}
\frametitle{Modelling Frameworks}
\begin{itemize}
\item \href{https://www.minizinc.org/}{MiniZinc} (NICTA, Monash University, Australia)
\item \href{https://github.com/eomahony/Numberjack}{NumberJack} (Insight, Ireland) 
\item \href{https://www-users.york.ac.uk/peter.nightingale/savilerow//}{EssencePrime/SavilleRow} (UK) 
\item \href{https://cpmpy.readthedocs.io/en/latest/}{CPMpy} (KU Leuven)
\item Allow use of multiple back-end solvers
\item Compile model into variants for each solver
\item A priori solver independent model(CP, MIP, SAT)
\end{itemize}
\end{frame} 

\begin{frame}
\frametitle{Framework Process}
\begin{center}
\begin{tikzpicture}[xscale=2]
\node[shape=rectangle,fill=pantone127-4] (problem) at (2.5,5) {Problem};
\node (human) at (2.5,4) {Human};
\node[shape=rectangle,fill=pantone127-4] (model) at (2.5,3) {Model};
\node (compile) at (2.5,2) {Compile/Reformulate};
\node (solver1) at (1,1) {CP};
\node (solver2) at (2,1) {MIP};
\node (solver3) at (3,1) {SAT};
\node (solver4) at (4,1) {Other};
\node[shape=rectangle,fill=pantone157-8] (solution1) at (1,0) {Solution};
\node[shape=rectangle,fill=pantone157-8] (solution2) at (2,0) {Solution};
\node[shape=rectangle,fill=pantone157-8] (solution3) at (3,0) {Solution};
\node[shape=rectangle,fill=pantone157-8] (solution4) at (4,0) {Solution};
\draw[-] (problem) -- (human);
\draw[->] (human) -- (model);
\draw[-] (model) -- (compile);
\draw[-] (compile) -- (solver1);
\draw[-] (compile) -- (solver2);
\draw[-] (compile) -- (solver3);
\draw[-] (compile) -- (solver4);
\draw[->] (solver1) -- (solution1);
\draw[->] (solver2) -- (solution2);
\draw[->] (solver3) -- (solution3);
\draw[->] (solver4) -- (solution4);
\end{tikzpicture}
\end{center}
\end{frame}

% \begin{frame}
  % \frametitle{Do It Now!}
  % \begin{itemize}
  % \item Download and install Minizinc
  % \item \url{https://www.minizinc.org/}
  % \end{itemize}
% \end{frame}


\begin{frame}
\frametitle{Why use Puzzles as Examples?}
\begin{itemize}
\item Easy to understand the problem
\item Solvable by hand without specialized knowledge
\item Possible to compare automated to manual solving process
\end{itemize}

\begin{quote}
The puzzle, though inanimate, is presented as a solvable problem without lasting negative consequences, a very low-risk low-reward situation. By being a puzzle, the object is attempting to convince the user that it must be completed.
\end{quote}
{\tiny Source: \href{https://www.everydayrhetoric.com/2020/03/the-rhetoric-of-puzzles/}{Every Day Rhetoric}}
\end{frame}


\mode<all>{
\part{Basic Constraint Propagation}
}
\begin{frame}
\partpage
\end{frame}
\mode<all>{
\begin{frame}
\frametitle{Objectives}
\begin{itemize}
\item Overview of Core Constraint Programming
\item Three Main Concepts
\begin{itemize}
\item Constraint Propagation
\item Global Constraints
\item Customizing Search
\end{itemize}
%\item Get Some Experience with MiniZinc
\item Topics will be treated in more detail in later parts of the school
\item Based on Examples, not Formal Description
\end{itemize}
\end{frame}

\begin{frame}
\frametitle{Outline}
\begin{itemize}
\item Why Constraint Programming?
\item Constraint Propagation
\item Global Constraints
\item Customizing Search
%\item Choices in Modelling
%\item Decomposition
%\item More Global Constraints
%\item Methodology
%\item What is missing?
\end{itemize}
\end{frame}

% \begin{frame}
  % \frametitle{Using MiniZinc IDE}
  % \begin{itemize}
  % \item Developed in the Australian NICTA project
  % \item Maintained by Monash University
  % \item Modelling tool with multiple back-end solvers
  % \item Available from \url{https://www.minizinc.org/}
  % \end{itemize}
% \end{frame}


% \begin{frame}
% \frametitle{Examples in ECLiPSe}
% \begin{itemize}
% \item Open sourced constraint programming language
% \item Development goes back to 1985
% \item ECRC, ICL, IC-Parc, PTL, Cisco
% \item \url{https://eclipseclp.org/}
% \item Specialities
% \begin{itemize}
% \item Develop new solvers for specific domains
% \item Integration with MIP
% \end{itemize}
% \item Not included in bundled MiniZinc IDE
% \item Specialized visualization tools used here
  % \begin{itemize}
  % \item CP-Viz, Simonis et al. 2010
    % \end{itemize}
% \end{itemize}
% \end{frame}

\begin{frame}
\frametitle{Tutorial Based on ECLiPSe ELearning Course}
\begin{itemize}
\item Self-study course in constraint programming
\item Supported by Cisco Systems and Silicon Valley Community Foundation
\item Multi-media format, video lectures, slides, handout etc
\item {\small \url{https://eclipseclp.org/ELearning/index.html}}
\end{itemize}
\end{frame}

\begin{frame}
\frametitle{Also Part of CRT-AI Constraint Week}
\begin{itemize}
\item Annual one week course on CP and Optimization in Ireland
\item Part of national training program for PhD students in AI
\item \url{https://www.crt-ai.ie/}
\end{itemize}
\end{frame}



\begin{frame}
\frametitle{Constraint Programming - in a nutshell}
\begin{itemize}
\item Declarative description of problems with
\begin{itemize}
\item {\em Variables} which range over (finite) sets of values
\item {\em Constraints} over subsets of variables which restrict possible value combinations
\item A {\em solution} is a value assignment which satisfies all constraints
\end{itemize}

\item Constraint propagation/reasoning
\begin{itemize}
\item Removing inconsistent values for variables
\item Detect failure if constraint can not be satisfied
\item Interaction of constraints via shared variables
\item Incomplete
\end{itemize}

\item Search
\begin{itemize}
\item User controlled assignment of values to variables
\item Each step triggers constraint propagation 
\end{itemize}
\item Different domains require/allow different methods
\end{itemize}
\end{frame}

\begin{frame}
  \frametitle{Constraint Programming is Different}
  \begin{itemize}
  \item Declarative Programming
    \begin{itemize}
    \item Concentrate on what you want
      \item Not how to get there
      \item Program != Algorithm
      \item Program = Model
    \end{itemize}
    \item Applied to Combinatorial Problems
      \begin{itemize}
        \item No complete polynomial algorithms known (exist?)
        \item CP less ad-hoc than heuristics
        \item Models can evolve
  \end{itemize}
  \end{itemize}
  \end{frame}
    
\begin{frame}
  \frametitle{A Subtractive Process}
  \begin{textblock}{4}(8,-3)
    \includegraphics[width=4cm]{../introduction/images/stages}
  \end{textblock}
  \vfill
  \begin{quote}
    ``Oh, bosh, as Mr. Ruskin says. Sculpture, per se, is the simplest thing in the world. All you have to do is to take a big chunk of marble and a hammer and chisel, make up your mind what you are about to create and chip off all the marble you don't want.''-Paris Gaulois.
  \end{quote}
  
  {\tiny Source: \url{https://quoteinvestigator.com/2014/06/22/chip-away/}}
\end{frame}

\begin{frame}
\frametitle{Basic Process}
\begin{center}
\begin{tikzpicture}
\node[shape=rectangle,fill=pantone127-4] (problem) at (2,5) {Problem};
\node (human) at (2,4) {Human};
\node[shape=rectangle,fill=pantone127-4] (model) at (2,3) {Model};
\node (solver) at (2,2) {Constraint Solver/Search};
\node[shape=rectangle,fill=pantone157-8] (solution) at (2,1) {Solution};
\draw[-] (problem) -- (human);
\draw[->] (human) -- (model);
\draw[-] (model) -- (solver);
\draw[->] (solver) -- (solution);
\end{tikzpicture}
\end{center}
\end{frame}

\begin{frame}
\frametitle{More Realistic}
\begin{center}
\begin{tikzpicture}[xscale=4.0]
\node[shape=rectangle,fill=pantone127-4] (problem) at (2,5) {Problem};
\node (human) at (2,4) {Human};
\node[shape=rectangle,fill=pantone127-4] (model) at (2,3) {Model};
\node (solver) at (2,2) {Constraint Solver/Search};
\node[shape=rectangle,fill=pantone24-8] (hangs) at (1,1) {Hangs};
\node[shape=rectangle,fill=pantone157-8] (solution) at (2,1) {Solution};
\node[shape=rectangle,fill=pantone24-8] (wrong) at (3,1) {Wrong Solution};
\draw[-] (problem) -- (human);
\draw[->] (human) -- (model);
\draw[-] (model) -- (solver);
\draw[->] (solver) -- (hangs);
\draw[->] (solver) -- (solution);
\draw[->] (solver) -- (wrong);
\draw[->] (hangs) -- (1,4) -- (human);
\draw[->] (wrong) -- (3,4) -- (human);
\end{tikzpicture}
\end{center}
\end{frame}

\begin{frame}
\frametitle{Dual Role of Model}
\begin{itemize}
\item Allows Human to Express Problem
\begin{itemize}
\item Close to Problem Domain
\item Constraints as Abstractions
\end{itemize}
\item Allows Solver to Execute
\begin{itemize}
\item Variables as Communication Mechanism
\item Constraints as Algorithms
\end{itemize}
\end{itemize}
\end{frame}

\begin{frame}
\frametitle{Modelling Frameworks}
\begin{itemize}
\item \href{https://www.minizinc.org/}{MiniZinc} (NICTA, Monash University, Australia)
\item \href{https://github.com/eomahony/Numberjack}{NumberJack} (Insight, Ireland) 
\item \href{https://www-users.york.ac.uk/peter.nightingale/savilerow//}{EssencePrime/SavilleRow} (UK) 
\item \href{https://cpmpy.readthedocs.io/en/latest/}{CPMpy} (KU Leuven)
\item Allow use of multiple back-end solvers
\item Compile model into variants for each solver
\item A priori solver independent model(CP, MIP, SAT)
\end{itemize}
\end{frame} 

\begin{frame}
\frametitle{Framework Process}
\begin{center}
\begin{tikzpicture}[xscale=2]
\node[shape=rectangle,fill=pantone127-4] (problem) at (2.5,5) {Problem};
\node (human) at (2.5,4) {Human};
\node[shape=rectangle,fill=pantone127-4] (model) at (2.5,3) {Model};
\node (compile) at (2.5,2) {Compile/Reformulate};
\node (solver1) at (1,1) {CP};
\node (solver2) at (2,1) {MIP};
\node (solver3) at (3,1) {SAT};
\node (solver4) at (4,1) {Other};
\node[shape=rectangle,fill=pantone157-8] (solution1) at (1,0) {Solution};
\node[shape=rectangle,fill=pantone157-8] (solution2) at (2,0) {Solution};
\node[shape=rectangle,fill=pantone157-8] (solution3) at (3,0) {Solution};
\node[shape=rectangle,fill=pantone157-8] (solution4) at (4,0) {Solution};
\draw[-] (problem) -- (human);
\draw[->] (human) -- (model);
\draw[-] (model) -- (compile);
\draw[-] (compile) -- (solver1);
\draw[-] (compile) -- (solver2);
\draw[-] (compile) -- (solver3);
\draw[-] (compile) -- (solver4);
\draw[->] (solver1) -- (solution1);
\draw[->] (solver2) -- (solution2);
\draw[->] (solver3) -- (solution3);
\draw[->] (solver4) -- (solution4);
\end{tikzpicture}
\end{center}
\end{frame}

% \begin{frame}
  % \frametitle{Do It Now!}
  % \begin{itemize}
  % \item Download and install Minizinc
  % \item \url{https://www.minizinc.org/}
  % \end{itemize}
% \end{frame}


\begin{frame}
\frametitle{Why use Puzzles as Examples?}
\begin{itemize}
\item Easy to understand the problem
\item Solvable by hand without specialized knowledge
\item Possible to compare automated to manual solving process
\end{itemize}

\begin{quote}
The puzzle, though inanimate, is presented as a solvable problem without lasting negative consequences, a very low-risk low-reward situation. By being a puzzle, the object is attempting to convince the user that it must be completed.
\end{quote}
{\tiny Source: \href{https://www.everydayrhetoric.com/2020/03/the-rhetoric-of-puzzles/}{Every Day Rhetoric}}
\end{frame}


\mode<all>{
\part{Basic Constraint Propagation}
}
\begin{frame}
\partpage
\end{frame}
\mode<all>{
\input{../sendmore/body}
}

\mode<all>{
\part{Global Constraints}
}
\begin{frame}
\partpage
\end{frame}
\mode<all>{
\input{../sudoku/body}
}

\clearpage
\mode<all>{
\part{Customizing Search}
}
\begin{frame}
\partpage
\end{frame}
\mode<all>{
\input{../nqueen/body}
}


%% \clearpage
%% \mode<all>{
%%   \part{Problem Decomposition}
%% }
%% \begin{frame}
%%   \partpage
%% \end{frame}
%% \mode<all>{
%%   \input{../party/body}
%%   }

%% \clearpage
%% \mode<all>{
%%   \part{More Global Constraints}
%% }
%% \begin{frame}
%%   \partpage
%% \end{frame}
%% \mode<all>{
%%   \input{../car/body}
%%   }

%% \clearpage
%% \mode<all>{
%%   \part{Methodology}
%% }
%% \begin{frame}
%%   \partpage
%% \end{frame}
%% \mode<all>{
%%   \input{../methodology/body}
%%   }

\clearpage
\mode<all>{
\part{What is missing?}
}
\begin{frame}
\partpage
\end{frame}

\begin{frame}
\frametitle{Many Specialized Topics}
\begin{itemize}
\item How to design efficient core engine
\item Hybrids with LP/MIP tools
\item Hybrids with SAT
\item Symmetry breaking
\item Use of MDD/BDD to encode sets of solutions
\item High level modelling tools
\item Debugging/visualization
\end{itemize}
\end{frame}

\begin{frame}
\frametitle{Reformulation}
\begin{itemize}
\item Just because the user has modelled it this way, it doesn't mean we have to solve it that way
\begin{itemize}
\item Replace some constraint(s) by other, equivalent constraints
\item Because we don't have that constraint in our system
\item For performance
\end{itemize}
\end{itemize}
\end{frame}

\begin{frame}
\frametitle{Learning}
\begin{itemize}
\item While solving the problem we can learn how to strengthen the model/search 
\begin{itemize}
\item Understand which constraints/method contribute to propagation and change schedule
\item Learn no-good constraints by explaining failure
\item Adapt search strategy based on search experience
\end{itemize}
\end{itemize}
\end{frame}

%% \begin{frame}
%% \frametitle{Refined Process}
%% \begin{center}
%% \begin{tikzpicture}
%% \node[shape=rectangle,fill=pantone127-4] (problem) at (3,7) {Problem};
%% \node (human) at (3,6) {Human};
%% \node[shape=rectangle,fill=pantone127-4] (model) at (3,5) {User Model};
%% \node (reformulation) at (3,4) {Reformulation};
%% \node[shape=rectangle,fill=pantone127-4] (implementation) at (3,3) {Implementation Model};
%% \node (solver) at (3,2) {Constraint Solver/Search};
%% \node[shape=rectangle,fill=pantone157-8] (solution) at (3,1) {Solution};
%% \draw[-] (problem) -- (human);
%% \draw[->] (human) -- (model);
%% \draw[-] (model) -- (reformulation);
%% \draw[->] (reformulation) -- (implementation);
%% \draw[-] (implementation) -- (solver);
%% \draw[->] (solver) -- (0,2) -- (0,3) -- (implementation);
%% \draw[->] (solver) -- (solution);
%% \end{tikzpicture}
%% \end{center}
%% \end{frame}

%\begin{frame}
%\frametitle{What is CP actually used for?}
%\url{http://hsimonis.wordpress.com}
%\includegraphics[width=10cm]{../introduction/images/blog}
%\end{frame}

%% \begin{frame}
%% \frametitle{Exercises}
%% \begin{itemize}
%% \item Install Constraint Solver of your Choice
%% \item MiniZinc (MiningZinc) will be used in later sessions
%% \item Run Sendmoremoney, Sudoku, Queens examples
%% \item Can you choose which propagator to use for alldifferent?
%% \item Can you use different search strategies?
%% \end{itemize}
%% \end{frame}

\begin{frame}
\frametitle{More Learning Resources}
\begin{itemize}
\item Survey of Methods, Resources, and Formats for Teaching Constraint Programming
\begin{itemize}
\item by Tejas Santanam, Helmut Simonis
\item \url{https://doi.org/10.48550/arXiv.2403.12717}
\item Based on survey of community for WTCP 2023
\item \url{https://hsimonis.github.io/WTCP2023/}
\end{itemize}
\end{itemize}
\end{frame}


}

\mode<all>{
\part{Global Constraints}
}
\begin{frame}
\partpage
\end{frame}
\mode<all>{
\begin{frame}
\frametitle{Objectives}
\begin{itemize}
\item Overview of Core Constraint Programming
\item Three Main Concepts
\begin{itemize}
\item Constraint Propagation
\item Global Constraints
\item Customizing Search
\end{itemize}
%\item Get Some Experience with MiniZinc
\item Topics will be treated in more detail in later parts of the school
\item Based on Examples, not Formal Description
\end{itemize}
\end{frame}

\begin{frame}
\frametitle{Outline}
\begin{itemize}
\item Why Constraint Programming?
\item Constraint Propagation
\item Global Constraints
\item Customizing Search
%\item Choices in Modelling
%\item Decomposition
%\item More Global Constraints
%\item Methodology
%\item What is missing?
\end{itemize}
\end{frame}

% \begin{frame}
  % \frametitle{Using MiniZinc IDE}
  % \begin{itemize}
  % \item Developed in the Australian NICTA project
  % \item Maintained by Monash University
  % \item Modelling tool with multiple back-end solvers
  % \item Available from \url{https://www.minizinc.org/}
  % \end{itemize}
% \end{frame}


% \begin{frame}
% \frametitle{Examples in ECLiPSe}
% \begin{itemize}
% \item Open sourced constraint programming language
% \item Development goes back to 1985
% \item ECRC, ICL, IC-Parc, PTL, Cisco
% \item \url{https://eclipseclp.org/}
% \item Specialities
% \begin{itemize}
% \item Develop new solvers for specific domains
% \item Integration with MIP
% \end{itemize}
% \item Not included in bundled MiniZinc IDE
% \item Specialized visualization tools used here
  % \begin{itemize}
  % \item CP-Viz, Simonis et al. 2010
    % \end{itemize}
% \end{itemize}
% \end{frame}

\begin{frame}
\frametitle{Tutorial Based on ECLiPSe ELearning Course}
\begin{itemize}
\item Self-study course in constraint programming
\item Supported by Cisco Systems and Silicon Valley Community Foundation
\item Multi-media format, video lectures, slides, handout etc
\item {\small \url{https://eclipseclp.org/ELearning/index.html}}
\end{itemize}
\end{frame}

\begin{frame}
\frametitle{Also Part of CRT-AI Constraint Week}
\begin{itemize}
\item Annual one week course on CP and Optimization in Ireland
\item Part of national training program for PhD students in AI
\item \url{https://www.crt-ai.ie/}
\end{itemize}
\end{frame}



\begin{frame}
\frametitle{Constraint Programming - in a nutshell}
\begin{itemize}
\item Declarative description of problems with
\begin{itemize}
\item {\em Variables} which range over (finite) sets of values
\item {\em Constraints} over subsets of variables which restrict possible value combinations
\item A {\em solution} is a value assignment which satisfies all constraints
\end{itemize}

\item Constraint propagation/reasoning
\begin{itemize}
\item Removing inconsistent values for variables
\item Detect failure if constraint can not be satisfied
\item Interaction of constraints via shared variables
\item Incomplete
\end{itemize}

\item Search
\begin{itemize}
\item User controlled assignment of values to variables
\item Each step triggers constraint propagation 
\end{itemize}
\item Different domains require/allow different methods
\end{itemize}
\end{frame}

\begin{frame}
  \frametitle{Constraint Programming is Different}
  \begin{itemize}
  \item Declarative Programming
    \begin{itemize}
    \item Concentrate on what you want
      \item Not how to get there
      \item Program != Algorithm
      \item Program = Model
    \end{itemize}
    \item Applied to Combinatorial Problems
      \begin{itemize}
        \item No complete polynomial algorithms known (exist?)
        \item CP less ad-hoc than heuristics
        \item Models can evolve
  \end{itemize}
  \end{itemize}
  \end{frame}
    
\begin{frame}
  \frametitle{A Subtractive Process}
  \begin{textblock}{4}(8,-3)
    \includegraphics[width=4cm]{../introduction/images/stages}
  \end{textblock}
  \vfill
  \begin{quote}
    ``Oh, bosh, as Mr. Ruskin says. Sculpture, per se, is the simplest thing in the world. All you have to do is to take a big chunk of marble and a hammer and chisel, make up your mind what you are about to create and chip off all the marble you don't want.''-Paris Gaulois.
  \end{quote}
  
  {\tiny Source: \url{https://quoteinvestigator.com/2014/06/22/chip-away/}}
\end{frame}

\begin{frame}
\frametitle{Basic Process}
\begin{center}
\begin{tikzpicture}
\node[shape=rectangle,fill=pantone127-4] (problem) at (2,5) {Problem};
\node (human) at (2,4) {Human};
\node[shape=rectangle,fill=pantone127-4] (model) at (2,3) {Model};
\node (solver) at (2,2) {Constraint Solver/Search};
\node[shape=rectangle,fill=pantone157-8] (solution) at (2,1) {Solution};
\draw[-] (problem) -- (human);
\draw[->] (human) -- (model);
\draw[-] (model) -- (solver);
\draw[->] (solver) -- (solution);
\end{tikzpicture}
\end{center}
\end{frame}

\begin{frame}
\frametitle{More Realistic}
\begin{center}
\begin{tikzpicture}[xscale=4.0]
\node[shape=rectangle,fill=pantone127-4] (problem) at (2,5) {Problem};
\node (human) at (2,4) {Human};
\node[shape=rectangle,fill=pantone127-4] (model) at (2,3) {Model};
\node (solver) at (2,2) {Constraint Solver/Search};
\node[shape=rectangle,fill=pantone24-8] (hangs) at (1,1) {Hangs};
\node[shape=rectangle,fill=pantone157-8] (solution) at (2,1) {Solution};
\node[shape=rectangle,fill=pantone24-8] (wrong) at (3,1) {Wrong Solution};
\draw[-] (problem) -- (human);
\draw[->] (human) -- (model);
\draw[-] (model) -- (solver);
\draw[->] (solver) -- (hangs);
\draw[->] (solver) -- (solution);
\draw[->] (solver) -- (wrong);
\draw[->] (hangs) -- (1,4) -- (human);
\draw[->] (wrong) -- (3,4) -- (human);
\end{tikzpicture}
\end{center}
\end{frame}

\begin{frame}
\frametitle{Dual Role of Model}
\begin{itemize}
\item Allows Human to Express Problem
\begin{itemize}
\item Close to Problem Domain
\item Constraints as Abstractions
\end{itemize}
\item Allows Solver to Execute
\begin{itemize}
\item Variables as Communication Mechanism
\item Constraints as Algorithms
\end{itemize}
\end{itemize}
\end{frame}

\begin{frame}
\frametitle{Modelling Frameworks}
\begin{itemize}
\item \href{https://www.minizinc.org/}{MiniZinc} (NICTA, Monash University, Australia)
\item \href{https://github.com/eomahony/Numberjack}{NumberJack} (Insight, Ireland) 
\item \href{https://www-users.york.ac.uk/peter.nightingale/savilerow//}{EssencePrime/SavilleRow} (UK) 
\item \href{https://cpmpy.readthedocs.io/en/latest/}{CPMpy} (KU Leuven)
\item Allow use of multiple back-end solvers
\item Compile model into variants for each solver
\item A priori solver independent model(CP, MIP, SAT)
\end{itemize}
\end{frame} 

\begin{frame}
\frametitle{Framework Process}
\begin{center}
\begin{tikzpicture}[xscale=2]
\node[shape=rectangle,fill=pantone127-4] (problem) at (2.5,5) {Problem};
\node (human) at (2.5,4) {Human};
\node[shape=rectangle,fill=pantone127-4] (model) at (2.5,3) {Model};
\node (compile) at (2.5,2) {Compile/Reformulate};
\node (solver1) at (1,1) {CP};
\node (solver2) at (2,1) {MIP};
\node (solver3) at (3,1) {SAT};
\node (solver4) at (4,1) {Other};
\node[shape=rectangle,fill=pantone157-8] (solution1) at (1,0) {Solution};
\node[shape=rectangle,fill=pantone157-8] (solution2) at (2,0) {Solution};
\node[shape=rectangle,fill=pantone157-8] (solution3) at (3,0) {Solution};
\node[shape=rectangle,fill=pantone157-8] (solution4) at (4,0) {Solution};
\draw[-] (problem) -- (human);
\draw[->] (human) -- (model);
\draw[-] (model) -- (compile);
\draw[-] (compile) -- (solver1);
\draw[-] (compile) -- (solver2);
\draw[-] (compile) -- (solver3);
\draw[-] (compile) -- (solver4);
\draw[->] (solver1) -- (solution1);
\draw[->] (solver2) -- (solution2);
\draw[->] (solver3) -- (solution3);
\draw[->] (solver4) -- (solution4);
\end{tikzpicture}
\end{center}
\end{frame}

% \begin{frame}
  % \frametitle{Do It Now!}
  % \begin{itemize}
  % \item Download and install Minizinc
  % \item \url{https://www.minizinc.org/}
  % \end{itemize}
% \end{frame}


\begin{frame}
\frametitle{Why use Puzzles as Examples?}
\begin{itemize}
\item Easy to understand the problem
\item Solvable by hand without specialized knowledge
\item Possible to compare automated to manual solving process
\end{itemize}

\begin{quote}
The puzzle, though inanimate, is presented as a solvable problem without lasting negative consequences, a very low-risk low-reward situation. By being a puzzle, the object is attempting to convince the user that it must be completed.
\end{quote}
{\tiny Source: \href{https://www.everydayrhetoric.com/2020/03/the-rhetoric-of-puzzles/}{Every Day Rhetoric}}
\end{frame}


\mode<all>{
\part{Basic Constraint Propagation}
}
\begin{frame}
\partpage
\end{frame}
\mode<all>{
\input{../sendmore/body}
}

\mode<all>{
\part{Global Constraints}
}
\begin{frame}
\partpage
\end{frame}
\mode<all>{
\input{../sudoku/body}
}

\clearpage
\mode<all>{
\part{Customizing Search}
}
\begin{frame}
\partpage
\end{frame}
\mode<all>{
\input{../nqueen/body}
}


%% \clearpage
%% \mode<all>{
%%   \part{Problem Decomposition}
%% }
%% \begin{frame}
%%   \partpage
%% \end{frame}
%% \mode<all>{
%%   \input{../party/body}
%%   }

%% \clearpage
%% \mode<all>{
%%   \part{More Global Constraints}
%% }
%% \begin{frame}
%%   \partpage
%% \end{frame}
%% \mode<all>{
%%   \input{../car/body}
%%   }

%% \clearpage
%% \mode<all>{
%%   \part{Methodology}
%% }
%% \begin{frame}
%%   \partpage
%% \end{frame}
%% \mode<all>{
%%   \input{../methodology/body}
%%   }

\clearpage
\mode<all>{
\part{What is missing?}
}
\begin{frame}
\partpage
\end{frame}

\begin{frame}
\frametitle{Many Specialized Topics}
\begin{itemize}
\item How to design efficient core engine
\item Hybrids with LP/MIP tools
\item Hybrids with SAT
\item Symmetry breaking
\item Use of MDD/BDD to encode sets of solutions
\item High level modelling tools
\item Debugging/visualization
\end{itemize}
\end{frame}

\begin{frame}
\frametitle{Reformulation}
\begin{itemize}
\item Just because the user has modelled it this way, it doesn't mean we have to solve it that way
\begin{itemize}
\item Replace some constraint(s) by other, equivalent constraints
\item Because we don't have that constraint in our system
\item For performance
\end{itemize}
\end{itemize}
\end{frame}

\begin{frame}
\frametitle{Learning}
\begin{itemize}
\item While solving the problem we can learn how to strengthen the model/search 
\begin{itemize}
\item Understand which constraints/method contribute to propagation and change schedule
\item Learn no-good constraints by explaining failure
\item Adapt search strategy based on search experience
\end{itemize}
\end{itemize}
\end{frame}

%% \begin{frame}
%% \frametitle{Refined Process}
%% \begin{center}
%% \begin{tikzpicture}
%% \node[shape=rectangle,fill=pantone127-4] (problem) at (3,7) {Problem};
%% \node (human) at (3,6) {Human};
%% \node[shape=rectangle,fill=pantone127-4] (model) at (3,5) {User Model};
%% \node (reformulation) at (3,4) {Reformulation};
%% \node[shape=rectangle,fill=pantone127-4] (implementation) at (3,3) {Implementation Model};
%% \node (solver) at (3,2) {Constraint Solver/Search};
%% \node[shape=rectangle,fill=pantone157-8] (solution) at (3,1) {Solution};
%% \draw[-] (problem) -- (human);
%% \draw[->] (human) -- (model);
%% \draw[-] (model) -- (reformulation);
%% \draw[->] (reformulation) -- (implementation);
%% \draw[-] (implementation) -- (solver);
%% \draw[->] (solver) -- (0,2) -- (0,3) -- (implementation);
%% \draw[->] (solver) -- (solution);
%% \end{tikzpicture}
%% \end{center}
%% \end{frame}

%\begin{frame}
%\frametitle{What is CP actually used for?}
%\url{http://hsimonis.wordpress.com}
%\includegraphics[width=10cm]{../introduction/images/blog}
%\end{frame}

%% \begin{frame}
%% \frametitle{Exercises}
%% \begin{itemize}
%% \item Install Constraint Solver of your Choice
%% \item MiniZinc (MiningZinc) will be used in later sessions
%% \item Run Sendmoremoney, Sudoku, Queens examples
%% \item Can you choose which propagator to use for alldifferent?
%% \item Can you use different search strategies?
%% \end{itemize}
%% \end{frame}

\begin{frame}
\frametitle{More Learning Resources}
\begin{itemize}
\item Survey of Methods, Resources, and Formats for Teaching Constraint Programming
\begin{itemize}
\item by Tejas Santanam, Helmut Simonis
\item \url{https://doi.org/10.48550/arXiv.2403.12717}
\item Based on survey of community for WTCP 2023
\item \url{https://hsimonis.github.io/WTCP2023/}
\end{itemize}
\end{itemize}
\end{frame}


}

\clearpage
\mode<all>{
\part{Customizing Search}
}
\begin{frame}
\partpage
\end{frame}
\mode<all>{
\begin{frame}
\frametitle{Objectives}
\begin{itemize}
\item Overview of Core Constraint Programming
\item Three Main Concepts
\begin{itemize}
\item Constraint Propagation
\item Global Constraints
\item Customizing Search
\end{itemize}
%\item Get Some Experience with MiniZinc
\item Topics will be treated in more detail in later parts of the school
\item Based on Examples, not Formal Description
\end{itemize}
\end{frame}

\begin{frame}
\frametitle{Outline}
\begin{itemize}
\item Why Constraint Programming?
\item Constraint Propagation
\item Global Constraints
\item Customizing Search
%\item Choices in Modelling
%\item Decomposition
%\item More Global Constraints
%\item Methodology
%\item What is missing?
\end{itemize}
\end{frame}

% \begin{frame}
  % \frametitle{Using MiniZinc IDE}
  % \begin{itemize}
  % \item Developed in the Australian NICTA project
  % \item Maintained by Monash University
  % \item Modelling tool with multiple back-end solvers
  % \item Available from \url{https://www.minizinc.org/}
  % \end{itemize}
% \end{frame}


% \begin{frame}
% \frametitle{Examples in ECLiPSe}
% \begin{itemize}
% \item Open sourced constraint programming language
% \item Development goes back to 1985
% \item ECRC, ICL, IC-Parc, PTL, Cisco
% \item \url{https://eclipseclp.org/}
% \item Specialities
% \begin{itemize}
% \item Develop new solvers for specific domains
% \item Integration with MIP
% \end{itemize}
% \item Not included in bundled MiniZinc IDE
% \item Specialized visualization tools used here
  % \begin{itemize}
  % \item CP-Viz, Simonis et al. 2010
    % \end{itemize}
% \end{itemize}
% \end{frame}

\begin{frame}
\frametitle{Tutorial Based on ECLiPSe ELearning Course}
\begin{itemize}
\item Self-study course in constraint programming
\item Supported by Cisco Systems and Silicon Valley Community Foundation
\item Multi-media format, video lectures, slides, handout etc
\item {\small \url{https://eclipseclp.org/ELearning/index.html}}
\end{itemize}
\end{frame}

\begin{frame}
\frametitle{Also Part of CRT-AI Constraint Week}
\begin{itemize}
\item Annual one week course on CP and Optimization in Ireland
\item Part of national training program for PhD students in AI
\item \url{https://www.crt-ai.ie/}
\end{itemize}
\end{frame}



\begin{frame}
\frametitle{Constraint Programming - in a nutshell}
\begin{itemize}
\item Declarative description of problems with
\begin{itemize}
\item {\em Variables} which range over (finite) sets of values
\item {\em Constraints} over subsets of variables which restrict possible value combinations
\item A {\em solution} is a value assignment which satisfies all constraints
\end{itemize}

\item Constraint propagation/reasoning
\begin{itemize}
\item Removing inconsistent values for variables
\item Detect failure if constraint can not be satisfied
\item Interaction of constraints via shared variables
\item Incomplete
\end{itemize}

\item Search
\begin{itemize}
\item User controlled assignment of values to variables
\item Each step triggers constraint propagation 
\end{itemize}
\item Different domains require/allow different methods
\end{itemize}
\end{frame}

\begin{frame}
  \frametitle{Constraint Programming is Different}
  \begin{itemize}
  \item Declarative Programming
    \begin{itemize}
    \item Concentrate on what you want
      \item Not how to get there
      \item Program != Algorithm
      \item Program = Model
    \end{itemize}
    \item Applied to Combinatorial Problems
      \begin{itemize}
        \item No complete polynomial algorithms known (exist?)
        \item CP less ad-hoc than heuristics
        \item Models can evolve
  \end{itemize}
  \end{itemize}
  \end{frame}
    
\begin{frame}
  \frametitle{A Subtractive Process}
  \begin{textblock}{4}(8,-3)
    \includegraphics[width=4cm]{../introduction/images/stages}
  \end{textblock}
  \vfill
  \begin{quote}
    ``Oh, bosh, as Mr. Ruskin says. Sculpture, per se, is the simplest thing in the world. All you have to do is to take a big chunk of marble and a hammer and chisel, make up your mind what you are about to create and chip off all the marble you don't want.''-Paris Gaulois.
  \end{quote}
  
  {\tiny Source: \url{https://quoteinvestigator.com/2014/06/22/chip-away/}}
\end{frame}

\begin{frame}
\frametitle{Basic Process}
\begin{center}
\begin{tikzpicture}
\node[shape=rectangle,fill=pantone127-4] (problem) at (2,5) {Problem};
\node (human) at (2,4) {Human};
\node[shape=rectangle,fill=pantone127-4] (model) at (2,3) {Model};
\node (solver) at (2,2) {Constraint Solver/Search};
\node[shape=rectangle,fill=pantone157-8] (solution) at (2,1) {Solution};
\draw[-] (problem) -- (human);
\draw[->] (human) -- (model);
\draw[-] (model) -- (solver);
\draw[->] (solver) -- (solution);
\end{tikzpicture}
\end{center}
\end{frame}

\begin{frame}
\frametitle{More Realistic}
\begin{center}
\begin{tikzpicture}[xscale=4.0]
\node[shape=rectangle,fill=pantone127-4] (problem) at (2,5) {Problem};
\node (human) at (2,4) {Human};
\node[shape=rectangle,fill=pantone127-4] (model) at (2,3) {Model};
\node (solver) at (2,2) {Constraint Solver/Search};
\node[shape=rectangle,fill=pantone24-8] (hangs) at (1,1) {Hangs};
\node[shape=rectangle,fill=pantone157-8] (solution) at (2,1) {Solution};
\node[shape=rectangle,fill=pantone24-8] (wrong) at (3,1) {Wrong Solution};
\draw[-] (problem) -- (human);
\draw[->] (human) -- (model);
\draw[-] (model) -- (solver);
\draw[->] (solver) -- (hangs);
\draw[->] (solver) -- (solution);
\draw[->] (solver) -- (wrong);
\draw[->] (hangs) -- (1,4) -- (human);
\draw[->] (wrong) -- (3,4) -- (human);
\end{tikzpicture}
\end{center}
\end{frame}

\begin{frame}
\frametitle{Dual Role of Model}
\begin{itemize}
\item Allows Human to Express Problem
\begin{itemize}
\item Close to Problem Domain
\item Constraints as Abstractions
\end{itemize}
\item Allows Solver to Execute
\begin{itemize}
\item Variables as Communication Mechanism
\item Constraints as Algorithms
\end{itemize}
\end{itemize}
\end{frame}

\begin{frame}
\frametitle{Modelling Frameworks}
\begin{itemize}
\item \href{https://www.minizinc.org/}{MiniZinc} (NICTA, Monash University, Australia)
\item \href{https://github.com/eomahony/Numberjack}{NumberJack} (Insight, Ireland) 
\item \href{https://www-users.york.ac.uk/peter.nightingale/savilerow//}{EssencePrime/SavilleRow} (UK) 
\item \href{https://cpmpy.readthedocs.io/en/latest/}{CPMpy} (KU Leuven)
\item Allow use of multiple back-end solvers
\item Compile model into variants for each solver
\item A priori solver independent model(CP, MIP, SAT)
\end{itemize}
\end{frame} 

\begin{frame}
\frametitle{Framework Process}
\begin{center}
\begin{tikzpicture}[xscale=2]
\node[shape=rectangle,fill=pantone127-4] (problem) at (2.5,5) {Problem};
\node (human) at (2.5,4) {Human};
\node[shape=rectangle,fill=pantone127-4] (model) at (2.5,3) {Model};
\node (compile) at (2.5,2) {Compile/Reformulate};
\node (solver1) at (1,1) {CP};
\node (solver2) at (2,1) {MIP};
\node (solver3) at (3,1) {SAT};
\node (solver4) at (4,1) {Other};
\node[shape=rectangle,fill=pantone157-8] (solution1) at (1,0) {Solution};
\node[shape=rectangle,fill=pantone157-8] (solution2) at (2,0) {Solution};
\node[shape=rectangle,fill=pantone157-8] (solution3) at (3,0) {Solution};
\node[shape=rectangle,fill=pantone157-8] (solution4) at (4,0) {Solution};
\draw[-] (problem) -- (human);
\draw[->] (human) -- (model);
\draw[-] (model) -- (compile);
\draw[-] (compile) -- (solver1);
\draw[-] (compile) -- (solver2);
\draw[-] (compile) -- (solver3);
\draw[-] (compile) -- (solver4);
\draw[->] (solver1) -- (solution1);
\draw[->] (solver2) -- (solution2);
\draw[->] (solver3) -- (solution3);
\draw[->] (solver4) -- (solution4);
\end{tikzpicture}
\end{center}
\end{frame}

% \begin{frame}
  % \frametitle{Do It Now!}
  % \begin{itemize}
  % \item Download and install Minizinc
  % \item \url{https://www.minizinc.org/}
  % \end{itemize}
% \end{frame}


\begin{frame}
\frametitle{Why use Puzzles as Examples?}
\begin{itemize}
\item Easy to understand the problem
\item Solvable by hand without specialized knowledge
\item Possible to compare automated to manual solving process
\end{itemize}

\begin{quote}
The puzzle, though inanimate, is presented as a solvable problem without lasting negative consequences, a very low-risk low-reward situation. By being a puzzle, the object is attempting to convince the user that it must be completed.
\end{quote}
{\tiny Source: \href{https://www.everydayrhetoric.com/2020/03/the-rhetoric-of-puzzles/}{Every Day Rhetoric}}
\end{frame}


\mode<all>{
\part{Basic Constraint Propagation}
}
\begin{frame}
\partpage
\end{frame}
\mode<all>{
\input{../sendmore/body}
}

\mode<all>{
\part{Global Constraints}
}
\begin{frame}
\partpage
\end{frame}
\mode<all>{
\input{../sudoku/body}
}

\clearpage
\mode<all>{
\part{Customizing Search}
}
\begin{frame}
\partpage
\end{frame}
\mode<all>{
\input{../nqueen/body}
}


%% \clearpage
%% \mode<all>{
%%   \part{Problem Decomposition}
%% }
%% \begin{frame}
%%   \partpage
%% \end{frame}
%% \mode<all>{
%%   \input{../party/body}
%%   }

%% \clearpage
%% \mode<all>{
%%   \part{More Global Constraints}
%% }
%% \begin{frame}
%%   \partpage
%% \end{frame}
%% \mode<all>{
%%   \input{../car/body}
%%   }

%% \clearpage
%% \mode<all>{
%%   \part{Methodology}
%% }
%% \begin{frame}
%%   \partpage
%% \end{frame}
%% \mode<all>{
%%   \input{../methodology/body}
%%   }

\clearpage
\mode<all>{
\part{What is missing?}
}
\begin{frame}
\partpage
\end{frame}

\begin{frame}
\frametitle{Many Specialized Topics}
\begin{itemize}
\item How to design efficient core engine
\item Hybrids with LP/MIP tools
\item Hybrids with SAT
\item Symmetry breaking
\item Use of MDD/BDD to encode sets of solutions
\item High level modelling tools
\item Debugging/visualization
\end{itemize}
\end{frame}

\begin{frame}
\frametitle{Reformulation}
\begin{itemize}
\item Just because the user has modelled it this way, it doesn't mean we have to solve it that way
\begin{itemize}
\item Replace some constraint(s) by other, equivalent constraints
\item Because we don't have that constraint in our system
\item For performance
\end{itemize}
\end{itemize}
\end{frame}

\begin{frame}
\frametitle{Learning}
\begin{itemize}
\item While solving the problem we can learn how to strengthen the model/search 
\begin{itemize}
\item Understand which constraints/method contribute to propagation and change schedule
\item Learn no-good constraints by explaining failure
\item Adapt search strategy based on search experience
\end{itemize}
\end{itemize}
\end{frame}

%% \begin{frame}
%% \frametitle{Refined Process}
%% \begin{center}
%% \begin{tikzpicture}
%% \node[shape=rectangle,fill=pantone127-4] (problem) at (3,7) {Problem};
%% \node (human) at (3,6) {Human};
%% \node[shape=rectangle,fill=pantone127-4] (model) at (3,5) {User Model};
%% \node (reformulation) at (3,4) {Reformulation};
%% \node[shape=rectangle,fill=pantone127-4] (implementation) at (3,3) {Implementation Model};
%% \node (solver) at (3,2) {Constraint Solver/Search};
%% \node[shape=rectangle,fill=pantone157-8] (solution) at (3,1) {Solution};
%% \draw[-] (problem) -- (human);
%% \draw[->] (human) -- (model);
%% \draw[-] (model) -- (reformulation);
%% \draw[->] (reformulation) -- (implementation);
%% \draw[-] (implementation) -- (solver);
%% \draw[->] (solver) -- (0,2) -- (0,3) -- (implementation);
%% \draw[->] (solver) -- (solution);
%% \end{tikzpicture}
%% \end{center}
%% \end{frame}

%\begin{frame}
%\frametitle{What is CP actually used for?}
%\url{http://hsimonis.wordpress.com}
%\includegraphics[width=10cm]{../introduction/images/blog}
%\end{frame}

%% \begin{frame}
%% \frametitle{Exercises}
%% \begin{itemize}
%% \item Install Constraint Solver of your Choice
%% \item MiniZinc (MiningZinc) will be used in later sessions
%% \item Run Sendmoremoney, Sudoku, Queens examples
%% \item Can you choose which propagator to use for alldifferent?
%% \item Can you use different search strategies?
%% \end{itemize}
%% \end{frame}

\begin{frame}
\frametitle{More Learning Resources}
\begin{itemize}
\item Survey of Methods, Resources, and Formats for Teaching Constraint Programming
\begin{itemize}
\item by Tejas Santanam, Helmut Simonis
\item \url{https://doi.org/10.48550/arXiv.2403.12717}
\item Based on survey of community for WTCP 2023
\item \url{https://hsimonis.github.io/WTCP2023/}
\end{itemize}
\end{itemize}
\end{frame}


}


%% \clearpage
%% \mode<all>{
%%   \part{Problem Decomposition}
%% }
%% \begin{frame}
%%   \partpage
%% \end{frame}
%% \mode<all>{
%%   \begin{frame}
\frametitle{Objectives}
\begin{itemize}
\item Overview of Core Constraint Programming
\item Three Main Concepts
\begin{itemize}
\item Constraint Propagation
\item Global Constraints
\item Customizing Search
\end{itemize}
%\item Get Some Experience with MiniZinc
\item Topics will be treated in more detail in later parts of the school
\item Based on Examples, not Formal Description
\end{itemize}
\end{frame}

\begin{frame}
\frametitle{Outline}
\begin{itemize}
\item Why Constraint Programming?
\item Constraint Propagation
\item Global Constraints
\item Customizing Search
%\item Choices in Modelling
%\item Decomposition
%\item More Global Constraints
%\item Methodology
%\item What is missing?
\end{itemize}
\end{frame}

% \begin{frame}
  % \frametitle{Using MiniZinc IDE}
  % \begin{itemize}
  % \item Developed in the Australian NICTA project
  % \item Maintained by Monash University
  % \item Modelling tool with multiple back-end solvers
  % \item Available from \url{https://www.minizinc.org/}
  % \end{itemize}
% \end{frame}


% \begin{frame}
% \frametitle{Examples in ECLiPSe}
% \begin{itemize}
% \item Open sourced constraint programming language
% \item Development goes back to 1985
% \item ECRC, ICL, IC-Parc, PTL, Cisco
% \item \url{https://eclipseclp.org/}
% \item Specialities
% \begin{itemize}
% \item Develop new solvers for specific domains
% \item Integration with MIP
% \end{itemize}
% \item Not included in bundled MiniZinc IDE
% \item Specialized visualization tools used here
  % \begin{itemize}
  % \item CP-Viz, Simonis et al. 2010
    % \end{itemize}
% \end{itemize}
% \end{frame}

\begin{frame}
\frametitle{Tutorial Based on ECLiPSe ELearning Course}
\begin{itemize}
\item Self-study course in constraint programming
\item Supported by Cisco Systems and Silicon Valley Community Foundation
\item Multi-media format, video lectures, slides, handout etc
\item {\small \url{https://eclipseclp.org/ELearning/index.html}}
\end{itemize}
\end{frame}

\begin{frame}
\frametitle{Also Part of CRT-AI Constraint Week}
\begin{itemize}
\item Annual one week course on CP and Optimization in Ireland
\item Part of national training program for PhD students in AI
\item \url{https://www.crt-ai.ie/}
\end{itemize}
\end{frame}



\begin{frame}
\frametitle{Constraint Programming - in a nutshell}
\begin{itemize}
\item Declarative description of problems with
\begin{itemize}
\item {\em Variables} which range over (finite) sets of values
\item {\em Constraints} over subsets of variables which restrict possible value combinations
\item A {\em solution} is a value assignment which satisfies all constraints
\end{itemize}

\item Constraint propagation/reasoning
\begin{itemize}
\item Removing inconsistent values for variables
\item Detect failure if constraint can not be satisfied
\item Interaction of constraints via shared variables
\item Incomplete
\end{itemize}

\item Search
\begin{itemize}
\item User controlled assignment of values to variables
\item Each step triggers constraint propagation 
\end{itemize}
\item Different domains require/allow different methods
\end{itemize}
\end{frame}

\begin{frame}
  \frametitle{Constraint Programming is Different}
  \begin{itemize}
  \item Declarative Programming
    \begin{itemize}
    \item Concentrate on what you want
      \item Not how to get there
      \item Program != Algorithm
      \item Program = Model
    \end{itemize}
    \item Applied to Combinatorial Problems
      \begin{itemize}
        \item No complete polynomial algorithms known (exist?)
        \item CP less ad-hoc than heuristics
        \item Models can evolve
  \end{itemize}
  \end{itemize}
  \end{frame}
    
\begin{frame}
  \frametitle{A Subtractive Process}
  \begin{textblock}{4}(8,-3)
    \includegraphics[width=4cm]{../introduction/images/stages}
  \end{textblock}
  \vfill
  \begin{quote}
    ``Oh, bosh, as Mr. Ruskin says. Sculpture, per se, is the simplest thing in the world. All you have to do is to take a big chunk of marble and a hammer and chisel, make up your mind what you are about to create and chip off all the marble you don't want.''-Paris Gaulois.
  \end{quote}
  
  {\tiny Source: \url{https://quoteinvestigator.com/2014/06/22/chip-away/}}
\end{frame}

\begin{frame}
\frametitle{Basic Process}
\begin{center}
\begin{tikzpicture}
\node[shape=rectangle,fill=pantone127-4] (problem) at (2,5) {Problem};
\node (human) at (2,4) {Human};
\node[shape=rectangle,fill=pantone127-4] (model) at (2,3) {Model};
\node (solver) at (2,2) {Constraint Solver/Search};
\node[shape=rectangle,fill=pantone157-8] (solution) at (2,1) {Solution};
\draw[-] (problem) -- (human);
\draw[->] (human) -- (model);
\draw[-] (model) -- (solver);
\draw[->] (solver) -- (solution);
\end{tikzpicture}
\end{center}
\end{frame}

\begin{frame}
\frametitle{More Realistic}
\begin{center}
\begin{tikzpicture}[xscale=4.0]
\node[shape=rectangle,fill=pantone127-4] (problem) at (2,5) {Problem};
\node (human) at (2,4) {Human};
\node[shape=rectangle,fill=pantone127-4] (model) at (2,3) {Model};
\node (solver) at (2,2) {Constraint Solver/Search};
\node[shape=rectangle,fill=pantone24-8] (hangs) at (1,1) {Hangs};
\node[shape=rectangle,fill=pantone157-8] (solution) at (2,1) {Solution};
\node[shape=rectangle,fill=pantone24-8] (wrong) at (3,1) {Wrong Solution};
\draw[-] (problem) -- (human);
\draw[->] (human) -- (model);
\draw[-] (model) -- (solver);
\draw[->] (solver) -- (hangs);
\draw[->] (solver) -- (solution);
\draw[->] (solver) -- (wrong);
\draw[->] (hangs) -- (1,4) -- (human);
\draw[->] (wrong) -- (3,4) -- (human);
\end{tikzpicture}
\end{center}
\end{frame}

\begin{frame}
\frametitle{Dual Role of Model}
\begin{itemize}
\item Allows Human to Express Problem
\begin{itemize}
\item Close to Problem Domain
\item Constraints as Abstractions
\end{itemize}
\item Allows Solver to Execute
\begin{itemize}
\item Variables as Communication Mechanism
\item Constraints as Algorithms
\end{itemize}
\end{itemize}
\end{frame}

\begin{frame}
\frametitle{Modelling Frameworks}
\begin{itemize}
\item \href{https://www.minizinc.org/}{MiniZinc} (NICTA, Monash University, Australia)
\item \href{https://github.com/eomahony/Numberjack}{NumberJack} (Insight, Ireland) 
\item \href{https://www-users.york.ac.uk/peter.nightingale/savilerow//}{EssencePrime/SavilleRow} (UK) 
\item \href{https://cpmpy.readthedocs.io/en/latest/}{CPMpy} (KU Leuven)
\item Allow use of multiple back-end solvers
\item Compile model into variants for each solver
\item A priori solver independent model(CP, MIP, SAT)
\end{itemize}
\end{frame} 

\begin{frame}
\frametitle{Framework Process}
\begin{center}
\begin{tikzpicture}[xscale=2]
\node[shape=rectangle,fill=pantone127-4] (problem) at (2.5,5) {Problem};
\node (human) at (2.5,4) {Human};
\node[shape=rectangle,fill=pantone127-4] (model) at (2.5,3) {Model};
\node (compile) at (2.5,2) {Compile/Reformulate};
\node (solver1) at (1,1) {CP};
\node (solver2) at (2,1) {MIP};
\node (solver3) at (3,1) {SAT};
\node (solver4) at (4,1) {Other};
\node[shape=rectangle,fill=pantone157-8] (solution1) at (1,0) {Solution};
\node[shape=rectangle,fill=pantone157-8] (solution2) at (2,0) {Solution};
\node[shape=rectangle,fill=pantone157-8] (solution3) at (3,0) {Solution};
\node[shape=rectangle,fill=pantone157-8] (solution4) at (4,0) {Solution};
\draw[-] (problem) -- (human);
\draw[->] (human) -- (model);
\draw[-] (model) -- (compile);
\draw[-] (compile) -- (solver1);
\draw[-] (compile) -- (solver2);
\draw[-] (compile) -- (solver3);
\draw[-] (compile) -- (solver4);
\draw[->] (solver1) -- (solution1);
\draw[->] (solver2) -- (solution2);
\draw[->] (solver3) -- (solution3);
\draw[->] (solver4) -- (solution4);
\end{tikzpicture}
\end{center}
\end{frame}

% \begin{frame}
  % \frametitle{Do It Now!}
  % \begin{itemize}
  % \item Download and install Minizinc
  % \item \url{https://www.minizinc.org/}
  % \end{itemize}
% \end{frame}


\begin{frame}
\frametitle{Why use Puzzles as Examples?}
\begin{itemize}
\item Easy to understand the problem
\item Solvable by hand without specialized knowledge
\item Possible to compare automated to manual solving process
\end{itemize}

\begin{quote}
The puzzle, though inanimate, is presented as a solvable problem without lasting negative consequences, a very low-risk low-reward situation. By being a puzzle, the object is attempting to convince the user that it must be completed.
\end{quote}
{\tiny Source: \href{https://www.everydayrhetoric.com/2020/03/the-rhetoric-of-puzzles/}{Every Day Rhetoric}}
\end{frame}


\mode<all>{
\part{Basic Constraint Propagation}
}
\begin{frame}
\partpage
\end{frame}
\mode<all>{
\input{../sendmore/body}
}

\mode<all>{
\part{Global Constraints}
}
\begin{frame}
\partpage
\end{frame}
\mode<all>{
\input{../sudoku/body}
}

\clearpage
\mode<all>{
\part{Customizing Search}
}
\begin{frame}
\partpage
\end{frame}
\mode<all>{
\input{../nqueen/body}
}


%% \clearpage
%% \mode<all>{
%%   \part{Problem Decomposition}
%% }
%% \begin{frame}
%%   \partpage
%% \end{frame}
%% \mode<all>{
%%   \input{../party/body}
%%   }

%% \clearpage
%% \mode<all>{
%%   \part{More Global Constraints}
%% }
%% \begin{frame}
%%   \partpage
%% \end{frame}
%% \mode<all>{
%%   \input{../car/body}
%%   }

%% \clearpage
%% \mode<all>{
%%   \part{Methodology}
%% }
%% \begin{frame}
%%   \partpage
%% \end{frame}
%% \mode<all>{
%%   \input{../methodology/body}
%%   }

\clearpage
\mode<all>{
\part{What is missing?}
}
\begin{frame}
\partpage
\end{frame}

\begin{frame}
\frametitle{Many Specialized Topics}
\begin{itemize}
\item How to design efficient core engine
\item Hybrids with LP/MIP tools
\item Hybrids with SAT
\item Symmetry breaking
\item Use of MDD/BDD to encode sets of solutions
\item High level modelling tools
\item Debugging/visualization
\end{itemize}
\end{frame}

\begin{frame}
\frametitle{Reformulation}
\begin{itemize}
\item Just because the user has modelled it this way, it doesn't mean we have to solve it that way
\begin{itemize}
\item Replace some constraint(s) by other, equivalent constraints
\item Because we don't have that constraint in our system
\item For performance
\end{itemize}
\end{itemize}
\end{frame}

\begin{frame}
\frametitle{Learning}
\begin{itemize}
\item While solving the problem we can learn how to strengthen the model/search 
\begin{itemize}
\item Understand which constraints/method contribute to propagation and change schedule
\item Learn no-good constraints by explaining failure
\item Adapt search strategy based on search experience
\end{itemize}
\end{itemize}
\end{frame}

%% \begin{frame}
%% \frametitle{Refined Process}
%% \begin{center}
%% \begin{tikzpicture}
%% \node[shape=rectangle,fill=pantone127-4] (problem) at (3,7) {Problem};
%% \node (human) at (3,6) {Human};
%% \node[shape=rectangle,fill=pantone127-4] (model) at (3,5) {User Model};
%% \node (reformulation) at (3,4) {Reformulation};
%% \node[shape=rectangle,fill=pantone127-4] (implementation) at (3,3) {Implementation Model};
%% \node (solver) at (3,2) {Constraint Solver/Search};
%% \node[shape=rectangle,fill=pantone157-8] (solution) at (3,1) {Solution};
%% \draw[-] (problem) -- (human);
%% \draw[->] (human) -- (model);
%% \draw[-] (model) -- (reformulation);
%% \draw[->] (reformulation) -- (implementation);
%% \draw[-] (implementation) -- (solver);
%% \draw[->] (solver) -- (0,2) -- (0,3) -- (implementation);
%% \draw[->] (solver) -- (solution);
%% \end{tikzpicture}
%% \end{center}
%% \end{frame}

%\begin{frame}
%\frametitle{What is CP actually used for?}
%\url{http://hsimonis.wordpress.com}
%\includegraphics[width=10cm]{../introduction/images/blog}
%\end{frame}

%% \begin{frame}
%% \frametitle{Exercises}
%% \begin{itemize}
%% \item Install Constraint Solver of your Choice
%% \item MiniZinc (MiningZinc) will be used in later sessions
%% \item Run Sendmoremoney, Sudoku, Queens examples
%% \item Can you choose which propagator to use for alldifferent?
%% \item Can you use different search strategies?
%% \end{itemize}
%% \end{frame}

\begin{frame}
\frametitle{More Learning Resources}
\begin{itemize}
\item Survey of Methods, Resources, and Formats for Teaching Constraint Programming
\begin{itemize}
\item by Tejas Santanam, Helmut Simonis
\item \url{https://doi.org/10.48550/arXiv.2403.12717}
\item Based on survey of community for WTCP 2023
\item \url{https://hsimonis.github.io/WTCP2023/}
\end{itemize}
\end{itemize}
\end{frame}


%%   }

%% \clearpage
%% \mode<all>{
%%   \part{More Global Constraints}
%% }
%% \begin{frame}
%%   \partpage
%% \end{frame}
%% \mode<all>{
%%   \begin{frame}
\frametitle{Objectives}
\begin{itemize}
\item Overview of Core Constraint Programming
\item Three Main Concepts
\begin{itemize}
\item Constraint Propagation
\item Global Constraints
\item Customizing Search
\end{itemize}
%\item Get Some Experience with MiniZinc
\item Topics will be treated in more detail in later parts of the school
\item Based on Examples, not Formal Description
\end{itemize}
\end{frame}

\begin{frame}
\frametitle{Outline}
\begin{itemize}
\item Why Constraint Programming?
\item Constraint Propagation
\item Global Constraints
\item Customizing Search
%\item Choices in Modelling
%\item Decomposition
%\item More Global Constraints
%\item Methodology
%\item What is missing?
\end{itemize}
\end{frame}

% \begin{frame}
  % \frametitle{Using MiniZinc IDE}
  % \begin{itemize}
  % \item Developed in the Australian NICTA project
  % \item Maintained by Monash University
  % \item Modelling tool with multiple back-end solvers
  % \item Available from \url{https://www.minizinc.org/}
  % \end{itemize}
% \end{frame}


% \begin{frame}
% \frametitle{Examples in ECLiPSe}
% \begin{itemize}
% \item Open sourced constraint programming language
% \item Development goes back to 1985
% \item ECRC, ICL, IC-Parc, PTL, Cisco
% \item \url{https://eclipseclp.org/}
% \item Specialities
% \begin{itemize}
% \item Develop new solvers for specific domains
% \item Integration with MIP
% \end{itemize}
% \item Not included in bundled MiniZinc IDE
% \item Specialized visualization tools used here
  % \begin{itemize}
  % \item CP-Viz, Simonis et al. 2010
    % \end{itemize}
% \end{itemize}
% \end{frame}

\begin{frame}
\frametitle{Tutorial Based on ECLiPSe ELearning Course}
\begin{itemize}
\item Self-study course in constraint programming
\item Supported by Cisco Systems and Silicon Valley Community Foundation
\item Multi-media format, video lectures, slides, handout etc
\item {\small \url{https://eclipseclp.org/ELearning/index.html}}
\end{itemize}
\end{frame}

\begin{frame}
\frametitle{Also Part of CRT-AI Constraint Week}
\begin{itemize}
\item Annual one week course on CP and Optimization in Ireland
\item Part of national training program for PhD students in AI
\item \url{https://www.crt-ai.ie/}
\end{itemize}
\end{frame}



\begin{frame}
\frametitle{Constraint Programming - in a nutshell}
\begin{itemize}
\item Declarative description of problems with
\begin{itemize}
\item {\em Variables} which range over (finite) sets of values
\item {\em Constraints} over subsets of variables which restrict possible value combinations
\item A {\em solution} is a value assignment which satisfies all constraints
\end{itemize}

\item Constraint propagation/reasoning
\begin{itemize}
\item Removing inconsistent values for variables
\item Detect failure if constraint can not be satisfied
\item Interaction of constraints via shared variables
\item Incomplete
\end{itemize}

\item Search
\begin{itemize}
\item User controlled assignment of values to variables
\item Each step triggers constraint propagation 
\end{itemize}
\item Different domains require/allow different methods
\end{itemize}
\end{frame}

\begin{frame}
  \frametitle{Constraint Programming is Different}
  \begin{itemize}
  \item Declarative Programming
    \begin{itemize}
    \item Concentrate on what you want
      \item Not how to get there
      \item Program != Algorithm
      \item Program = Model
    \end{itemize}
    \item Applied to Combinatorial Problems
      \begin{itemize}
        \item No complete polynomial algorithms known (exist?)
        \item CP less ad-hoc than heuristics
        \item Models can evolve
  \end{itemize}
  \end{itemize}
  \end{frame}
    
\begin{frame}
  \frametitle{A Subtractive Process}
  \begin{textblock}{4}(8,-3)
    \includegraphics[width=4cm]{../introduction/images/stages}
  \end{textblock}
  \vfill
  \begin{quote}
    ``Oh, bosh, as Mr. Ruskin says. Sculpture, per se, is the simplest thing in the world. All you have to do is to take a big chunk of marble and a hammer and chisel, make up your mind what you are about to create and chip off all the marble you don't want.''-Paris Gaulois.
  \end{quote}
  
  {\tiny Source: \url{https://quoteinvestigator.com/2014/06/22/chip-away/}}
\end{frame}

\begin{frame}
\frametitle{Basic Process}
\begin{center}
\begin{tikzpicture}
\node[shape=rectangle,fill=pantone127-4] (problem) at (2,5) {Problem};
\node (human) at (2,4) {Human};
\node[shape=rectangle,fill=pantone127-4] (model) at (2,3) {Model};
\node (solver) at (2,2) {Constraint Solver/Search};
\node[shape=rectangle,fill=pantone157-8] (solution) at (2,1) {Solution};
\draw[-] (problem) -- (human);
\draw[->] (human) -- (model);
\draw[-] (model) -- (solver);
\draw[->] (solver) -- (solution);
\end{tikzpicture}
\end{center}
\end{frame}

\begin{frame}
\frametitle{More Realistic}
\begin{center}
\begin{tikzpicture}[xscale=4.0]
\node[shape=rectangle,fill=pantone127-4] (problem) at (2,5) {Problem};
\node (human) at (2,4) {Human};
\node[shape=rectangle,fill=pantone127-4] (model) at (2,3) {Model};
\node (solver) at (2,2) {Constraint Solver/Search};
\node[shape=rectangle,fill=pantone24-8] (hangs) at (1,1) {Hangs};
\node[shape=rectangle,fill=pantone157-8] (solution) at (2,1) {Solution};
\node[shape=rectangle,fill=pantone24-8] (wrong) at (3,1) {Wrong Solution};
\draw[-] (problem) -- (human);
\draw[->] (human) -- (model);
\draw[-] (model) -- (solver);
\draw[->] (solver) -- (hangs);
\draw[->] (solver) -- (solution);
\draw[->] (solver) -- (wrong);
\draw[->] (hangs) -- (1,4) -- (human);
\draw[->] (wrong) -- (3,4) -- (human);
\end{tikzpicture}
\end{center}
\end{frame}

\begin{frame}
\frametitle{Dual Role of Model}
\begin{itemize}
\item Allows Human to Express Problem
\begin{itemize}
\item Close to Problem Domain
\item Constraints as Abstractions
\end{itemize}
\item Allows Solver to Execute
\begin{itemize}
\item Variables as Communication Mechanism
\item Constraints as Algorithms
\end{itemize}
\end{itemize}
\end{frame}

\begin{frame}
\frametitle{Modelling Frameworks}
\begin{itemize}
\item \href{https://www.minizinc.org/}{MiniZinc} (NICTA, Monash University, Australia)
\item \href{https://github.com/eomahony/Numberjack}{NumberJack} (Insight, Ireland) 
\item \href{https://www-users.york.ac.uk/peter.nightingale/savilerow//}{EssencePrime/SavilleRow} (UK) 
\item \href{https://cpmpy.readthedocs.io/en/latest/}{CPMpy} (KU Leuven)
\item Allow use of multiple back-end solvers
\item Compile model into variants for each solver
\item A priori solver independent model(CP, MIP, SAT)
\end{itemize}
\end{frame} 

\begin{frame}
\frametitle{Framework Process}
\begin{center}
\begin{tikzpicture}[xscale=2]
\node[shape=rectangle,fill=pantone127-4] (problem) at (2.5,5) {Problem};
\node (human) at (2.5,4) {Human};
\node[shape=rectangle,fill=pantone127-4] (model) at (2.5,3) {Model};
\node (compile) at (2.5,2) {Compile/Reformulate};
\node (solver1) at (1,1) {CP};
\node (solver2) at (2,1) {MIP};
\node (solver3) at (3,1) {SAT};
\node (solver4) at (4,1) {Other};
\node[shape=rectangle,fill=pantone157-8] (solution1) at (1,0) {Solution};
\node[shape=rectangle,fill=pantone157-8] (solution2) at (2,0) {Solution};
\node[shape=rectangle,fill=pantone157-8] (solution3) at (3,0) {Solution};
\node[shape=rectangle,fill=pantone157-8] (solution4) at (4,0) {Solution};
\draw[-] (problem) -- (human);
\draw[->] (human) -- (model);
\draw[-] (model) -- (compile);
\draw[-] (compile) -- (solver1);
\draw[-] (compile) -- (solver2);
\draw[-] (compile) -- (solver3);
\draw[-] (compile) -- (solver4);
\draw[->] (solver1) -- (solution1);
\draw[->] (solver2) -- (solution2);
\draw[->] (solver3) -- (solution3);
\draw[->] (solver4) -- (solution4);
\end{tikzpicture}
\end{center}
\end{frame}

% \begin{frame}
  % \frametitle{Do It Now!}
  % \begin{itemize}
  % \item Download and install Minizinc
  % \item \url{https://www.minizinc.org/}
  % \end{itemize}
% \end{frame}


\begin{frame}
\frametitle{Why use Puzzles as Examples?}
\begin{itemize}
\item Easy to understand the problem
\item Solvable by hand without specialized knowledge
\item Possible to compare automated to manual solving process
\end{itemize}

\begin{quote}
The puzzle, though inanimate, is presented as a solvable problem without lasting negative consequences, a very low-risk low-reward situation. By being a puzzle, the object is attempting to convince the user that it must be completed.
\end{quote}
{\tiny Source: \href{https://www.everydayrhetoric.com/2020/03/the-rhetoric-of-puzzles/}{Every Day Rhetoric}}
\end{frame}


\mode<all>{
\part{Basic Constraint Propagation}
}
\begin{frame}
\partpage
\end{frame}
\mode<all>{
\input{../sendmore/body}
}

\mode<all>{
\part{Global Constraints}
}
\begin{frame}
\partpage
\end{frame}
\mode<all>{
\input{../sudoku/body}
}

\clearpage
\mode<all>{
\part{Customizing Search}
}
\begin{frame}
\partpage
\end{frame}
\mode<all>{
\input{../nqueen/body}
}


%% \clearpage
%% \mode<all>{
%%   \part{Problem Decomposition}
%% }
%% \begin{frame}
%%   \partpage
%% \end{frame}
%% \mode<all>{
%%   \input{../party/body}
%%   }

%% \clearpage
%% \mode<all>{
%%   \part{More Global Constraints}
%% }
%% \begin{frame}
%%   \partpage
%% \end{frame}
%% \mode<all>{
%%   \input{../car/body}
%%   }

%% \clearpage
%% \mode<all>{
%%   \part{Methodology}
%% }
%% \begin{frame}
%%   \partpage
%% \end{frame}
%% \mode<all>{
%%   \input{../methodology/body}
%%   }

\clearpage
\mode<all>{
\part{What is missing?}
}
\begin{frame}
\partpage
\end{frame}

\begin{frame}
\frametitle{Many Specialized Topics}
\begin{itemize}
\item How to design efficient core engine
\item Hybrids with LP/MIP tools
\item Hybrids with SAT
\item Symmetry breaking
\item Use of MDD/BDD to encode sets of solutions
\item High level modelling tools
\item Debugging/visualization
\end{itemize}
\end{frame}

\begin{frame}
\frametitle{Reformulation}
\begin{itemize}
\item Just because the user has modelled it this way, it doesn't mean we have to solve it that way
\begin{itemize}
\item Replace some constraint(s) by other, equivalent constraints
\item Because we don't have that constraint in our system
\item For performance
\end{itemize}
\end{itemize}
\end{frame}

\begin{frame}
\frametitle{Learning}
\begin{itemize}
\item While solving the problem we can learn how to strengthen the model/search 
\begin{itemize}
\item Understand which constraints/method contribute to propagation and change schedule
\item Learn no-good constraints by explaining failure
\item Adapt search strategy based on search experience
\end{itemize}
\end{itemize}
\end{frame}

%% \begin{frame}
%% \frametitle{Refined Process}
%% \begin{center}
%% \begin{tikzpicture}
%% \node[shape=rectangle,fill=pantone127-4] (problem) at (3,7) {Problem};
%% \node (human) at (3,6) {Human};
%% \node[shape=rectangle,fill=pantone127-4] (model) at (3,5) {User Model};
%% \node (reformulation) at (3,4) {Reformulation};
%% \node[shape=rectangle,fill=pantone127-4] (implementation) at (3,3) {Implementation Model};
%% \node (solver) at (3,2) {Constraint Solver/Search};
%% \node[shape=rectangle,fill=pantone157-8] (solution) at (3,1) {Solution};
%% \draw[-] (problem) -- (human);
%% \draw[->] (human) -- (model);
%% \draw[-] (model) -- (reformulation);
%% \draw[->] (reformulation) -- (implementation);
%% \draw[-] (implementation) -- (solver);
%% \draw[->] (solver) -- (0,2) -- (0,3) -- (implementation);
%% \draw[->] (solver) -- (solution);
%% \end{tikzpicture}
%% \end{center}
%% \end{frame}

%\begin{frame}
%\frametitle{What is CP actually used for?}
%\url{http://hsimonis.wordpress.com}
%\includegraphics[width=10cm]{../introduction/images/blog}
%\end{frame}

%% \begin{frame}
%% \frametitle{Exercises}
%% \begin{itemize}
%% \item Install Constraint Solver of your Choice
%% \item MiniZinc (MiningZinc) will be used in later sessions
%% \item Run Sendmoremoney, Sudoku, Queens examples
%% \item Can you choose which propagator to use for alldifferent?
%% \item Can you use different search strategies?
%% \end{itemize}
%% \end{frame}

\begin{frame}
\frametitle{More Learning Resources}
\begin{itemize}
\item Survey of Methods, Resources, and Formats for Teaching Constraint Programming
\begin{itemize}
\item by Tejas Santanam, Helmut Simonis
\item \url{https://doi.org/10.48550/arXiv.2403.12717}
\item Based on survey of community for WTCP 2023
\item \url{https://hsimonis.github.io/WTCP2023/}
\end{itemize}
\end{itemize}
\end{frame}


%%   }

%% \clearpage
%% \mode<all>{
%%   \part{Methodology}
%% }
%% \begin{frame}
%%   \partpage
%% \end{frame}
%% \mode<all>{
%%   \begin{frame}
\frametitle{Objectives}
\begin{itemize}
\item Overview of Core Constraint Programming
\item Three Main Concepts
\begin{itemize}
\item Constraint Propagation
\item Global Constraints
\item Customizing Search
\end{itemize}
%\item Get Some Experience with MiniZinc
\item Topics will be treated in more detail in later parts of the school
\item Based on Examples, not Formal Description
\end{itemize}
\end{frame}

\begin{frame}
\frametitle{Outline}
\begin{itemize}
\item Why Constraint Programming?
\item Constraint Propagation
\item Global Constraints
\item Customizing Search
%\item Choices in Modelling
%\item Decomposition
%\item More Global Constraints
%\item Methodology
%\item What is missing?
\end{itemize}
\end{frame}

% \begin{frame}
  % \frametitle{Using MiniZinc IDE}
  % \begin{itemize}
  % \item Developed in the Australian NICTA project
  % \item Maintained by Monash University
  % \item Modelling tool with multiple back-end solvers
  % \item Available from \url{https://www.minizinc.org/}
  % \end{itemize}
% \end{frame}


% \begin{frame}
% \frametitle{Examples in ECLiPSe}
% \begin{itemize}
% \item Open sourced constraint programming language
% \item Development goes back to 1985
% \item ECRC, ICL, IC-Parc, PTL, Cisco
% \item \url{https://eclipseclp.org/}
% \item Specialities
% \begin{itemize}
% \item Develop new solvers for specific domains
% \item Integration with MIP
% \end{itemize}
% \item Not included in bundled MiniZinc IDE
% \item Specialized visualization tools used here
  % \begin{itemize}
  % \item CP-Viz, Simonis et al. 2010
    % \end{itemize}
% \end{itemize}
% \end{frame}

\begin{frame}
\frametitle{Tutorial Based on ECLiPSe ELearning Course}
\begin{itemize}
\item Self-study course in constraint programming
\item Supported by Cisco Systems and Silicon Valley Community Foundation
\item Multi-media format, video lectures, slides, handout etc
\item {\small \url{https://eclipseclp.org/ELearning/index.html}}
\end{itemize}
\end{frame}

\begin{frame}
\frametitle{Also Part of CRT-AI Constraint Week}
\begin{itemize}
\item Annual one week course on CP and Optimization in Ireland
\item Part of national training program for PhD students in AI
\item \url{https://www.crt-ai.ie/}
\end{itemize}
\end{frame}



\begin{frame}
\frametitle{Constraint Programming - in a nutshell}
\begin{itemize}
\item Declarative description of problems with
\begin{itemize}
\item {\em Variables} which range over (finite) sets of values
\item {\em Constraints} over subsets of variables which restrict possible value combinations
\item A {\em solution} is a value assignment which satisfies all constraints
\end{itemize}

\item Constraint propagation/reasoning
\begin{itemize}
\item Removing inconsistent values for variables
\item Detect failure if constraint can not be satisfied
\item Interaction of constraints via shared variables
\item Incomplete
\end{itemize}

\item Search
\begin{itemize}
\item User controlled assignment of values to variables
\item Each step triggers constraint propagation 
\end{itemize}
\item Different domains require/allow different methods
\end{itemize}
\end{frame}

\begin{frame}
  \frametitle{Constraint Programming is Different}
  \begin{itemize}
  \item Declarative Programming
    \begin{itemize}
    \item Concentrate on what you want
      \item Not how to get there
      \item Program != Algorithm
      \item Program = Model
    \end{itemize}
    \item Applied to Combinatorial Problems
      \begin{itemize}
        \item No complete polynomial algorithms known (exist?)
        \item CP less ad-hoc than heuristics
        \item Models can evolve
  \end{itemize}
  \end{itemize}
  \end{frame}
    
\begin{frame}
  \frametitle{A Subtractive Process}
  \begin{textblock}{4}(8,-3)
    \includegraphics[width=4cm]{../introduction/images/stages}
  \end{textblock}
  \vfill
  \begin{quote}
    ``Oh, bosh, as Mr. Ruskin says. Sculpture, per se, is the simplest thing in the world. All you have to do is to take a big chunk of marble and a hammer and chisel, make up your mind what you are about to create and chip off all the marble you don't want.''-Paris Gaulois.
  \end{quote}
  
  {\tiny Source: \url{https://quoteinvestigator.com/2014/06/22/chip-away/}}
\end{frame}

\begin{frame}
\frametitle{Basic Process}
\begin{center}
\begin{tikzpicture}
\node[shape=rectangle,fill=pantone127-4] (problem) at (2,5) {Problem};
\node (human) at (2,4) {Human};
\node[shape=rectangle,fill=pantone127-4] (model) at (2,3) {Model};
\node (solver) at (2,2) {Constraint Solver/Search};
\node[shape=rectangle,fill=pantone157-8] (solution) at (2,1) {Solution};
\draw[-] (problem) -- (human);
\draw[->] (human) -- (model);
\draw[-] (model) -- (solver);
\draw[->] (solver) -- (solution);
\end{tikzpicture}
\end{center}
\end{frame}

\begin{frame}
\frametitle{More Realistic}
\begin{center}
\begin{tikzpicture}[xscale=4.0]
\node[shape=rectangle,fill=pantone127-4] (problem) at (2,5) {Problem};
\node (human) at (2,4) {Human};
\node[shape=rectangle,fill=pantone127-4] (model) at (2,3) {Model};
\node (solver) at (2,2) {Constraint Solver/Search};
\node[shape=rectangle,fill=pantone24-8] (hangs) at (1,1) {Hangs};
\node[shape=rectangle,fill=pantone157-8] (solution) at (2,1) {Solution};
\node[shape=rectangle,fill=pantone24-8] (wrong) at (3,1) {Wrong Solution};
\draw[-] (problem) -- (human);
\draw[->] (human) -- (model);
\draw[-] (model) -- (solver);
\draw[->] (solver) -- (hangs);
\draw[->] (solver) -- (solution);
\draw[->] (solver) -- (wrong);
\draw[->] (hangs) -- (1,4) -- (human);
\draw[->] (wrong) -- (3,4) -- (human);
\end{tikzpicture}
\end{center}
\end{frame}

\begin{frame}
\frametitle{Dual Role of Model}
\begin{itemize}
\item Allows Human to Express Problem
\begin{itemize}
\item Close to Problem Domain
\item Constraints as Abstractions
\end{itemize}
\item Allows Solver to Execute
\begin{itemize}
\item Variables as Communication Mechanism
\item Constraints as Algorithms
\end{itemize}
\end{itemize}
\end{frame}

\begin{frame}
\frametitle{Modelling Frameworks}
\begin{itemize}
\item \href{https://www.minizinc.org/}{MiniZinc} (NICTA, Monash University, Australia)
\item \href{https://github.com/eomahony/Numberjack}{NumberJack} (Insight, Ireland) 
\item \href{https://www-users.york.ac.uk/peter.nightingale/savilerow//}{EssencePrime/SavilleRow} (UK) 
\item \href{https://cpmpy.readthedocs.io/en/latest/}{CPMpy} (KU Leuven)
\item Allow use of multiple back-end solvers
\item Compile model into variants for each solver
\item A priori solver independent model(CP, MIP, SAT)
\end{itemize}
\end{frame} 

\begin{frame}
\frametitle{Framework Process}
\begin{center}
\begin{tikzpicture}[xscale=2]
\node[shape=rectangle,fill=pantone127-4] (problem) at (2.5,5) {Problem};
\node (human) at (2.5,4) {Human};
\node[shape=rectangle,fill=pantone127-4] (model) at (2.5,3) {Model};
\node (compile) at (2.5,2) {Compile/Reformulate};
\node (solver1) at (1,1) {CP};
\node (solver2) at (2,1) {MIP};
\node (solver3) at (3,1) {SAT};
\node (solver4) at (4,1) {Other};
\node[shape=rectangle,fill=pantone157-8] (solution1) at (1,0) {Solution};
\node[shape=rectangle,fill=pantone157-8] (solution2) at (2,0) {Solution};
\node[shape=rectangle,fill=pantone157-8] (solution3) at (3,0) {Solution};
\node[shape=rectangle,fill=pantone157-8] (solution4) at (4,0) {Solution};
\draw[-] (problem) -- (human);
\draw[->] (human) -- (model);
\draw[-] (model) -- (compile);
\draw[-] (compile) -- (solver1);
\draw[-] (compile) -- (solver2);
\draw[-] (compile) -- (solver3);
\draw[-] (compile) -- (solver4);
\draw[->] (solver1) -- (solution1);
\draw[->] (solver2) -- (solution2);
\draw[->] (solver3) -- (solution3);
\draw[->] (solver4) -- (solution4);
\end{tikzpicture}
\end{center}
\end{frame}

% \begin{frame}
  % \frametitle{Do It Now!}
  % \begin{itemize}
  % \item Download and install Minizinc
  % \item \url{https://www.minizinc.org/}
  % \end{itemize}
% \end{frame}


\begin{frame}
\frametitle{Why use Puzzles as Examples?}
\begin{itemize}
\item Easy to understand the problem
\item Solvable by hand without specialized knowledge
\item Possible to compare automated to manual solving process
\end{itemize}

\begin{quote}
The puzzle, though inanimate, is presented as a solvable problem without lasting negative consequences, a very low-risk low-reward situation. By being a puzzle, the object is attempting to convince the user that it must be completed.
\end{quote}
{\tiny Source: \href{https://www.everydayrhetoric.com/2020/03/the-rhetoric-of-puzzles/}{Every Day Rhetoric}}
\end{frame}


\mode<all>{
\part{Basic Constraint Propagation}
}
\begin{frame}
\partpage
\end{frame}
\mode<all>{
\input{../sendmore/body}
}

\mode<all>{
\part{Global Constraints}
}
\begin{frame}
\partpage
\end{frame}
\mode<all>{
\input{../sudoku/body}
}

\clearpage
\mode<all>{
\part{Customizing Search}
}
\begin{frame}
\partpage
\end{frame}
\mode<all>{
\input{../nqueen/body}
}


%% \clearpage
%% \mode<all>{
%%   \part{Problem Decomposition}
%% }
%% \begin{frame}
%%   \partpage
%% \end{frame}
%% \mode<all>{
%%   \input{../party/body}
%%   }

%% \clearpage
%% \mode<all>{
%%   \part{More Global Constraints}
%% }
%% \begin{frame}
%%   \partpage
%% \end{frame}
%% \mode<all>{
%%   \input{../car/body}
%%   }

%% \clearpage
%% \mode<all>{
%%   \part{Methodology}
%% }
%% \begin{frame}
%%   \partpage
%% \end{frame}
%% \mode<all>{
%%   \input{../methodology/body}
%%   }

\clearpage
\mode<all>{
\part{What is missing?}
}
\begin{frame}
\partpage
\end{frame}

\begin{frame}
\frametitle{Many Specialized Topics}
\begin{itemize}
\item How to design efficient core engine
\item Hybrids with LP/MIP tools
\item Hybrids with SAT
\item Symmetry breaking
\item Use of MDD/BDD to encode sets of solutions
\item High level modelling tools
\item Debugging/visualization
\end{itemize}
\end{frame}

\begin{frame}
\frametitle{Reformulation}
\begin{itemize}
\item Just because the user has modelled it this way, it doesn't mean we have to solve it that way
\begin{itemize}
\item Replace some constraint(s) by other, equivalent constraints
\item Because we don't have that constraint in our system
\item For performance
\end{itemize}
\end{itemize}
\end{frame}

\begin{frame}
\frametitle{Learning}
\begin{itemize}
\item While solving the problem we can learn how to strengthen the model/search 
\begin{itemize}
\item Understand which constraints/method contribute to propagation and change schedule
\item Learn no-good constraints by explaining failure
\item Adapt search strategy based on search experience
\end{itemize}
\end{itemize}
\end{frame}

%% \begin{frame}
%% \frametitle{Refined Process}
%% \begin{center}
%% \begin{tikzpicture}
%% \node[shape=rectangle,fill=pantone127-4] (problem) at (3,7) {Problem};
%% \node (human) at (3,6) {Human};
%% \node[shape=rectangle,fill=pantone127-4] (model) at (3,5) {User Model};
%% \node (reformulation) at (3,4) {Reformulation};
%% \node[shape=rectangle,fill=pantone127-4] (implementation) at (3,3) {Implementation Model};
%% \node (solver) at (3,2) {Constraint Solver/Search};
%% \node[shape=rectangle,fill=pantone157-8] (solution) at (3,1) {Solution};
%% \draw[-] (problem) -- (human);
%% \draw[->] (human) -- (model);
%% \draw[-] (model) -- (reformulation);
%% \draw[->] (reformulation) -- (implementation);
%% \draw[-] (implementation) -- (solver);
%% \draw[->] (solver) -- (0,2) -- (0,3) -- (implementation);
%% \draw[->] (solver) -- (solution);
%% \end{tikzpicture}
%% \end{center}
%% \end{frame}

%\begin{frame}
%\frametitle{What is CP actually used for?}
%\url{http://hsimonis.wordpress.com}
%\includegraphics[width=10cm]{../introduction/images/blog}
%\end{frame}

%% \begin{frame}
%% \frametitle{Exercises}
%% \begin{itemize}
%% \item Install Constraint Solver of your Choice
%% \item MiniZinc (MiningZinc) will be used in later sessions
%% \item Run Sendmoremoney, Sudoku, Queens examples
%% \item Can you choose which propagator to use for alldifferent?
%% \item Can you use different search strategies?
%% \end{itemize}
%% \end{frame}

\begin{frame}
\frametitle{More Learning Resources}
\begin{itemize}
\item Survey of Methods, Resources, and Formats for Teaching Constraint Programming
\begin{itemize}
\item by Tejas Santanam, Helmut Simonis
\item \url{https://doi.org/10.48550/arXiv.2403.12717}
\item Based on survey of community for WTCP 2023
\item \url{https://hsimonis.github.io/WTCP2023/}
\end{itemize}
\end{itemize}
\end{frame}


%%   }

\clearpage
\mode<all>{
\part{What is missing?}
}
\begin{frame}
\partpage
\end{frame}

\begin{frame}
\frametitle{Many Specialized Topics}
\begin{itemize}
\item How to design efficient core engine
\item Hybrids with LP/MIP tools
\item Hybrids with SAT
\item Symmetry breaking
\item Use of MDD/BDD to encode sets of solutions
\item High level modelling tools
\item Debugging/visualization
\end{itemize}
\end{frame}

\begin{frame}
\frametitle{Reformulation}
\begin{itemize}
\item Just because the user has modelled it this way, it doesn't mean we have to solve it that way
\begin{itemize}
\item Replace some constraint(s) by other, equivalent constraints
\item Because we don't have that constraint in our system
\item For performance
\end{itemize}
\end{itemize}
\end{frame}

\begin{frame}
\frametitle{Learning}
\begin{itemize}
\item While solving the problem we can learn how to strengthen the model/search 
\begin{itemize}
\item Understand which constraints/method contribute to propagation and change schedule
\item Learn no-good constraints by explaining failure
\item Adapt search strategy based on search experience
\end{itemize}
\end{itemize}
\end{frame}

%% \begin{frame}
%% \frametitle{Refined Process}
%% \begin{center}
%% \begin{tikzpicture}
%% \node[shape=rectangle,fill=pantone127-4] (problem) at (3,7) {Problem};
%% \node (human) at (3,6) {Human};
%% \node[shape=rectangle,fill=pantone127-4] (model) at (3,5) {User Model};
%% \node (reformulation) at (3,4) {Reformulation};
%% \node[shape=rectangle,fill=pantone127-4] (implementation) at (3,3) {Implementation Model};
%% \node (solver) at (3,2) {Constraint Solver/Search};
%% \node[shape=rectangle,fill=pantone157-8] (solution) at (3,1) {Solution};
%% \draw[-] (problem) -- (human);
%% \draw[->] (human) -- (model);
%% \draw[-] (model) -- (reformulation);
%% \draw[->] (reformulation) -- (implementation);
%% \draw[-] (implementation) -- (solver);
%% \draw[->] (solver) -- (0,2) -- (0,3) -- (implementation);
%% \draw[->] (solver) -- (solution);
%% \end{tikzpicture}
%% \end{center}
%% \end{frame}

%\begin{frame}
%\frametitle{What is CP actually used for?}
%\url{http://hsimonis.wordpress.com}
%\includegraphics[width=10cm]{../introduction/images/blog}
%\end{frame}

%% \begin{frame}
%% \frametitle{Exercises}
%% \begin{itemize}
%% \item Install Constraint Solver of your Choice
%% \item MiniZinc (MiningZinc) will be used in later sessions
%% \item Run Sendmoremoney, Sudoku, Queens examples
%% \item Can you choose which propagator to use for alldifferent?
%% \item Can you use different search strategies?
%% \end{itemize}
%% \end{frame}

\begin{frame}
\frametitle{More Learning Resources}
\begin{itemize}
\item Survey of Methods, Resources, and Formats for Teaching Constraint Programming
\begin{itemize}
\item by Tejas Santanam, Helmut Simonis
\item \url{https://doi.org/10.48550/arXiv.2403.12717}
\item Based on survey of community for WTCP 2023
\item \url{https://hsimonis.github.io/WTCP2023/}
\end{itemize}
\end{itemize}
\end{frame}


}

\clearpage
\mode<all>{
\part{Customizing Search}
}
\begin{frame}
\partpage
\end{frame}
\mode<all>{
\begin{frame}
\frametitle{Objectives}
\begin{itemize}
\item Overview of Core Constraint Programming
\item Three Main Concepts
\begin{itemize}
\item Constraint Propagation
\item Global Constraints
\item Customizing Search
\end{itemize}
%\item Get Some Experience with MiniZinc
\item Topics will be treated in more detail in later parts of the school
\item Based on Examples, not Formal Description
\end{itemize}
\end{frame}

\begin{frame}
\frametitle{Outline}
\begin{itemize}
\item Why Constraint Programming?
\item Constraint Propagation
\item Global Constraints
\item Customizing Search
%\item Choices in Modelling
%\item Decomposition
%\item More Global Constraints
%\item Methodology
%\item What is missing?
\end{itemize}
\end{frame}

% \begin{frame}
  % \frametitle{Using MiniZinc IDE}
  % \begin{itemize}
  % \item Developed in the Australian NICTA project
  % \item Maintained by Monash University
  % \item Modelling tool with multiple back-end solvers
  % \item Available from \url{https://www.minizinc.org/}
  % \end{itemize}
% \end{frame}


% \begin{frame}
% \frametitle{Examples in ECLiPSe}
% \begin{itemize}
% \item Open sourced constraint programming language
% \item Development goes back to 1985
% \item ECRC, ICL, IC-Parc, PTL, Cisco
% \item \url{https://eclipseclp.org/}
% \item Specialities
% \begin{itemize}
% \item Develop new solvers for specific domains
% \item Integration with MIP
% \end{itemize}
% \item Not included in bundled MiniZinc IDE
% \item Specialized visualization tools used here
  % \begin{itemize}
  % \item CP-Viz, Simonis et al. 2010
    % \end{itemize}
% \end{itemize}
% \end{frame}

\begin{frame}
\frametitle{Tutorial Based on ECLiPSe ELearning Course}
\begin{itemize}
\item Self-study course in constraint programming
\item Supported by Cisco Systems and Silicon Valley Community Foundation
\item Multi-media format, video lectures, slides, handout etc
\item {\small \url{https://eclipseclp.org/ELearning/index.html}}
\end{itemize}
\end{frame}

\begin{frame}
\frametitle{Also Part of CRT-AI Constraint Week}
\begin{itemize}
\item Annual one week course on CP and Optimization in Ireland
\item Part of national training program for PhD students in AI
\item \url{https://www.crt-ai.ie/}
\end{itemize}
\end{frame}



\begin{frame}
\frametitle{Constraint Programming - in a nutshell}
\begin{itemize}
\item Declarative description of problems with
\begin{itemize}
\item {\em Variables} which range over (finite) sets of values
\item {\em Constraints} over subsets of variables which restrict possible value combinations
\item A {\em solution} is a value assignment which satisfies all constraints
\end{itemize}

\item Constraint propagation/reasoning
\begin{itemize}
\item Removing inconsistent values for variables
\item Detect failure if constraint can not be satisfied
\item Interaction of constraints via shared variables
\item Incomplete
\end{itemize}

\item Search
\begin{itemize}
\item User controlled assignment of values to variables
\item Each step triggers constraint propagation 
\end{itemize}
\item Different domains require/allow different methods
\end{itemize}
\end{frame}

\begin{frame}
  \frametitle{Constraint Programming is Different}
  \begin{itemize}
  \item Declarative Programming
    \begin{itemize}
    \item Concentrate on what you want
      \item Not how to get there
      \item Program != Algorithm
      \item Program = Model
    \end{itemize}
    \item Applied to Combinatorial Problems
      \begin{itemize}
        \item No complete polynomial algorithms known (exist?)
        \item CP less ad-hoc than heuristics
        \item Models can evolve
  \end{itemize}
  \end{itemize}
  \end{frame}
    
\begin{frame}
  \frametitle{A Subtractive Process}
  \begin{textblock}{4}(8,-3)
    \includegraphics[width=4cm]{../introduction/images/stages}
  \end{textblock}
  \vfill
  \begin{quote}
    ``Oh, bosh, as Mr. Ruskin says. Sculpture, per se, is the simplest thing in the world. All you have to do is to take a big chunk of marble and a hammer and chisel, make up your mind what you are about to create and chip off all the marble you don't want.''-Paris Gaulois.
  \end{quote}
  
  {\tiny Source: \url{https://quoteinvestigator.com/2014/06/22/chip-away/}}
\end{frame}

\begin{frame}
\frametitle{Basic Process}
\begin{center}
\begin{tikzpicture}
\node[shape=rectangle,fill=pantone127-4] (problem) at (2,5) {Problem};
\node (human) at (2,4) {Human};
\node[shape=rectangle,fill=pantone127-4] (model) at (2,3) {Model};
\node (solver) at (2,2) {Constraint Solver/Search};
\node[shape=rectangle,fill=pantone157-8] (solution) at (2,1) {Solution};
\draw[-] (problem) -- (human);
\draw[->] (human) -- (model);
\draw[-] (model) -- (solver);
\draw[->] (solver) -- (solution);
\end{tikzpicture}
\end{center}
\end{frame}

\begin{frame}
\frametitle{More Realistic}
\begin{center}
\begin{tikzpicture}[xscale=4.0]
\node[shape=rectangle,fill=pantone127-4] (problem) at (2,5) {Problem};
\node (human) at (2,4) {Human};
\node[shape=rectangle,fill=pantone127-4] (model) at (2,3) {Model};
\node (solver) at (2,2) {Constraint Solver/Search};
\node[shape=rectangle,fill=pantone24-8] (hangs) at (1,1) {Hangs};
\node[shape=rectangle,fill=pantone157-8] (solution) at (2,1) {Solution};
\node[shape=rectangle,fill=pantone24-8] (wrong) at (3,1) {Wrong Solution};
\draw[-] (problem) -- (human);
\draw[->] (human) -- (model);
\draw[-] (model) -- (solver);
\draw[->] (solver) -- (hangs);
\draw[->] (solver) -- (solution);
\draw[->] (solver) -- (wrong);
\draw[->] (hangs) -- (1,4) -- (human);
\draw[->] (wrong) -- (3,4) -- (human);
\end{tikzpicture}
\end{center}
\end{frame}

\begin{frame}
\frametitle{Dual Role of Model}
\begin{itemize}
\item Allows Human to Express Problem
\begin{itemize}
\item Close to Problem Domain
\item Constraints as Abstractions
\end{itemize}
\item Allows Solver to Execute
\begin{itemize}
\item Variables as Communication Mechanism
\item Constraints as Algorithms
\end{itemize}
\end{itemize}
\end{frame}

\begin{frame}
\frametitle{Modelling Frameworks}
\begin{itemize}
\item \href{https://www.minizinc.org/}{MiniZinc} (NICTA, Monash University, Australia)
\item \href{https://github.com/eomahony/Numberjack}{NumberJack} (Insight, Ireland) 
\item \href{https://www-users.york.ac.uk/peter.nightingale/savilerow//}{EssencePrime/SavilleRow} (UK) 
\item \href{https://cpmpy.readthedocs.io/en/latest/}{CPMpy} (KU Leuven)
\item Allow use of multiple back-end solvers
\item Compile model into variants for each solver
\item A priori solver independent model(CP, MIP, SAT)
\end{itemize}
\end{frame} 

\begin{frame}
\frametitle{Framework Process}
\begin{center}
\begin{tikzpicture}[xscale=2]
\node[shape=rectangle,fill=pantone127-4] (problem) at (2.5,5) {Problem};
\node (human) at (2.5,4) {Human};
\node[shape=rectangle,fill=pantone127-4] (model) at (2.5,3) {Model};
\node (compile) at (2.5,2) {Compile/Reformulate};
\node (solver1) at (1,1) {CP};
\node (solver2) at (2,1) {MIP};
\node (solver3) at (3,1) {SAT};
\node (solver4) at (4,1) {Other};
\node[shape=rectangle,fill=pantone157-8] (solution1) at (1,0) {Solution};
\node[shape=rectangle,fill=pantone157-8] (solution2) at (2,0) {Solution};
\node[shape=rectangle,fill=pantone157-8] (solution3) at (3,0) {Solution};
\node[shape=rectangle,fill=pantone157-8] (solution4) at (4,0) {Solution};
\draw[-] (problem) -- (human);
\draw[->] (human) -- (model);
\draw[-] (model) -- (compile);
\draw[-] (compile) -- (solver1);
\draw[-] (compile) -- (solver2);
\draw[-] (compile) -- (solver3);
\draw[-] (compile) -- (solver4);
\draw[->] (solver1) -- (solution1);
\draw[->] (solver2) -- (solution2);
\draw[->] (solver3) -- (solution3);
\draw[->] (solver4) -- (solution4);
\end{tikzpicture}
\end{center}
\end{frame}

% \begin{frame}
  % \frametitle{Do It Now!}
  % \begin{itemize}
  % \item Download and install Minizinc
  % \item \url{https://www.minizinc.org/}
  % \end{itemize}
% \end{frame}


\begin{frame}
\frametitle{Why use Puzzles as Examples?}
\begin{itemize}
\item Easy to understand the problem
\item Solvable by hand without specialized knowledge
\item Possible to compare automated to manual solving process
\end{itemize}

\begin{quote}
The puzzle, though inanimate, is presented as a solvable problem without lasting negative consequences, a very low-risk low-reward situation. By being a puzzle, the object is attempting to convince the user that it must be completed.
\end{quote}
{\tiny Source: \href{https://www.everydayrhetoric.com/2020/03/the-rhetoric-of-puzzles/}{Every Day Rhetoric}}
\end{frame}


\mode<all>{
\part{Basic Constraint Propagation}
}
\begin{frame}
\partpage
\end{frame}
\mode<all>{
\begin{frame}
\frametitle{Objectives}
\begin{itemize}
\item Overview of Core Constraint Programming
\item Three Main Concepts
\begin{itemize}
\item Constraint Propagation
\item Global Constraints
\item Customizing Search
\end{itemize}
%\item Get Some Experience with MiniZinc
\item Topics will be treated in more detail in later parts of the school
\item Based on Examples, not Formal Description
\end{itemize}
\end{frame}

\begin{frame}
\frametitle{Outline}
\begin{itemize}
\item Why Constraint Programming?
\item Constraint Propagation
\item Global Constraints
\item Customizing Search
%\item Choices in Modelling
%\item Decomposition
%\item More Global Constraints
%\item Methodology
%\item What is missing?
\end{itemize}
\end{frame}

% \begin{frame}
  % \frametitle{Using MiniZinc IDE}
  % \begin{itemize}
  % \item Developed in the Australian NICTA project
  % \item Maintained by Monash University
  % \item Modelling tool with multiple back-end solvers
  % \item Available from \url{https://www.minizinc.org/}
  % \end{itemize}
% \end{frame}


% \begin{frame}
% \frametitle{Examples in ECLiPSe}
% \begin{itemize}
% \item Open sourced constraint programming language
% \item Development goes back to 1985
% \item ECRC, ICL, IC-Parc, PTL, Cisco
% \item \url{https://eclipseclp.org/}
% \item Specialities
% \begin{itemize}
% \item Develop new solvers for specific domains
% \item Integration with MIP
% \end{itemize}
% \item Not included in bundled MiniZinc IDE
% \item Specialized visualization tools used here
  % \begin{itemize}
  % \item CP-Viz, Simonis et al. 2010
    % \end{itemize}
% \end{itemize}
% \end{frame}

\begin{frame}
\frametitle{Tutorial Based on ECLiPSe ELearning Course}
\begin{itemize}
\item Self-study course in constraint programming
\item Supported by Cisco Systems and Silicon Valley Community Foundation
\item Multi-media format, video lectures, slides, handout etc
\item {\small \url{https://eclipseclp.org/ELearning/index.html}}
\end{itemize}
\end{frame}

\begin{frame}
\frametitle{Also Part of CRT-AI Constraint Week}
\begin{itemize}
\item Annual one week course on CP and Optimization in Ireland
\item Part of national training program for PhD students in AI
\item \url{https://www.crt-ai.ie/}
\end{itemize}
\end{frame}



\begin{frame}
\frametitle{Constraint Programming - in a nutshell}
\begin{itemize}
\item Declarative description of problems with
\begin{itemize}
\item {\em Variables} which range over (finite) sets of values
\item {\em Constraints} over subsets of variables which restrict possible value combinations
\item A {\em solution} is a value assignment which satisfies all constraints
\end{itemize}

\item Constraint propagation/reasoning
\begin{itemize}
\item Removing inconsistent values for variables
\item Detect failure if constraint can not be satisfied
\item Interaction of constraints via shared variables
\item Incomplete
\end{itemize}

\item Search
\begin{itemize}
\item User controlled assignment of values to variables
\item Each step triggers constraint propagation 
\end{itemize}
\item Different domains require/allow different methods
\end{itemize}
\end{frame}

\begin{frame}
  \frametitle{Constraint Programming is Different}
  \begin{itemize}
  \item Declarative Programming
    \begin{itemize}
    \item Concentrate on what you want
      \item Not how to get there
      \item Program != Algorithm
      \item Program = Model
    \end{itemize}
    \item Applied to Combinatorial Problems
      \begin{itemize}
        \item No complete polynomial algorithms known (exist?)
        \item CP less ad-hoc than heuristics
        \item Models can evolve
  \end{itemize}
  \end{itemize}
  \end{frame}
    
\begin{frame}
  \frametitle{A Subtractive Process}
  \begin{textblock}{4}(8,-3)
    \includegraphics[width=4cm]{../introduction/images/stages}
  \end{textblock}
  \vfill
  \begin{quote}
    ``Oh, bosh, as Mr. Ruskin says. Sculpture, per se, is the simplest thing in the world. All you have to do is to take a big chunk of marble and a hammer and chisel, make up your mind what you are about to create and chip off all the marble you don't want.''-Paris Gaulois.
  \end{quote}
  
  {\tiny Source: \url{https://quoteinvestigator.com/2014/06/22/chip-away/}}
\end{frame}

\begin{frame}
\frametitle{Basic Process}
\begin{center}
\begin{tikzpicture}
\node[shape=rectangle,fill=pantone127-4] (problem) at (2,5) {Problem};
\node (human) at (2,4) {Human};
\node[shape=rectangle,fill=pantone127-4] (model) at (2,3) {Model};
\node (solver) at (2,2) {Constraint Solver/Search};
\node[shape=rectangle,fill=pantone157-8] (solution) at (2,1) {Solution};
\draw[-] (problem) -- (human);
\draw[->] (human) -- (model);
\draw[-] (model) -- (solver);
\draw[->] (solver) -- (solution);
\end{tikzpicture}
\end{center}
\end{frame}

\begin{frame}
\frametitle{More Realistic}
\begin{center}
\begin{tikzpicture}[xscale=4.0]
\node[shape=rectangle,fill=pantone127-4] (problem) at (2,5) {Problem};
\node (human) at (2,4) {Human};
\node[shape=rectangle,fill=pantone127-4] (model) at (2,3) {Model};
\node (solver) at (2,2) {Constraint Solver/Search};
\node[shape=rectangle,fill=pantone24-8] (hangs) at (1,1) {Hangs};
\node[shape=rectangle,fill=pantone157-8] (solution) at (2,1) {Solution};
\node[shape=rectangle,fill=pantone24-8] (wrong) at (3,1) {Wrong Solution};
\draw[-] (problem) -- (human);
\draw[->] (human) -- (model);
\draw[-] (model) -- (solver);
\draw[->] (solver) -- (hangs);
\draw[->] (solver) -- (solution);
\draw[->] (solver) -- (wrong);
\draw[->] (hangs) -- (1,4) -- (human);
\draw[->] (wrong) -- (3,4) -- (human);
\end{tikzpicture}
\end{center}
\end{frame}

\begin{frame}
\frametitle{Dual Role of Model}
\begin{itemize}
\item Allows Human to Express Problem
\begin{itemize}
\item Close to Problem Domain
\item Constraints as Abstractions
\end{itemize}
\item Allows Solver to Execute
\begin{itemize}
\item Variables as Communication Mechanism
\item Constraints as Algorithms
\end{itemize}
\end{itemize}
\end{frame}

\begin{frame}
\frametitle{Modelling Frameworks}
\begin{itemize}
\item \href{https://www.minizinc.org/}{MiniZinc} (NICTA, Monash University, Australia)
\item \href{https://github.com/eomahony/Numberjack}{NumberJack} (Insight, Ireland) 
\item \href{https://www-users.york.ac.uk/peter.nightingale/savilerow//}{EssencePrime/SavilleRow} (UK) 
\item \href{https://cpmpy.readthedocs.io/en/latest/}{CPMpy} (KU Leuven)
\item Allow use of multiple back-end solvers
\item Compile model into variants for each solver
\item A priori solver independent model(CP, MIP, SAT)
\end{itemize}
\end{frame} 

\begin{frame}
\frametitle{Framework Process}
\begin{center}
\begin{tikzpicture}[xscale=2]
\node[shape=rectangle,fill=pantone127-4] (problem) at (2.5,5) {Problem};
\node (human) at (2.5,4) {Human};
\node[shape=rectangle,fill=pantone127-4] (model) at (2.5,3) {Model};
\node (compile) at (2.5,2) {Compile/Reformulate};
\node (solver1) at (1,1) {CP};
\node (solver2) at (2,1) {MIP};
\node (solver3) at (3,1) {SAT};
\node (solver4) at (4,1) {Other};
\node[shape=rectangle,fill=pantone157-8] (solution1) at (1,0) {Solution};
\node[shape=rectangle,fill=pantone157-8] (solution2) at (2,0) {Solution};
\node[shape=rectangle,fill=pantone157-8] (solution3) at (3,0) {Solution};
\node[shape=rectangle,fill=pantone157-8] (solution4) at (4,0) {Solution};
\draw[-] (problem) -- (human);
\draw[->] (human) -- (model);
\draw[-] (model) -- (compile);
\draw[-] (compile) -- (solver1);
\draw[-] (compile) -- (solver2);
\draw[-] (compile) -- (solver3);
\draw[-] (compile) -- (solver4);
\draw[->] (solver1) -- (solution1);
\draw[->] (solver2) -- (solution2);
\draw[->] (solver3) -- (solution3);
\draw[->] (solver4) -- (solution4);
\end{tikzpicture}
\end{center}
\end{frame}

% \begin{frame}
  % \frametitle{Do It Now!}
  % \begin{itemize}
  % \item Download and install Minizinc
  % \item \url{https://www.minizinc.org/}
  % \end{itemize}
% \end{frame}


\begin{frame}
\frametitle{Why use Puzzles as Examples?}
\begin{itemize}
\item Easy to understand the problem
\item Solvable by hand without specialized knowledge
\item Possible to compare automated to manual solving process
\end{itemize}

\begin{quote}
The puzzle, though inanimate, is presented as a solvable problem without lasting negative consequences, a very low-risk low-reward situation. By being a puzzle, the object is attempting to convince the user that it must be completed.
\end{quote}
{\tiny Source: \href{https://www.everydayrhetoric.com/2020/03/the-rhetoric-of-puzzles/}{Every Day Rhetoric}}
\end{frame}


\mode<all>{
\part{Basic Constraint Propagation}
}
\begin{frame}
\partpage
\end{frame}
\mode<all>{
\input{../sendmore/body}
}

\mode<all>{
\part{Global Constraints}
}
\begin{frame}
\partpage
\end{frame}
\mode<all>{
\input{../sudoku/body}
}

\clearpage
\mode<all>{
\part{Customizing Search}
}
\begin{frame}
\partpage
\end{frame}
\mode<all>{
\input{../nqueen/body}
}


%% \clearpage
%% \mode<all>{
%%   \part{Problem Decomposition}
%% }
%% \begin{frame}
%%   \partpage
%% \end{frame}
%% \mode<all>{
%%   \input{../party/body}
%%   }

%% \clearpage
%% \mode<all>{
%%   \part{More Global Constraints}
%% }
%% \begin{frame}
%%   \partpage
%% \end{frame}
%% \mode<all>{
%%   \input{../car/body}
%%   }

%% \clearpage
%% \mode<all>{
%%   \part{Methodology}
%% }
%% \begin{frame}
%%   \partpage
%% \end{frame}
%% \mode<all>{
%%   \input{../methodology/body}
%%   }

\clearpage
\mode<all>{
\part{What is missing?}
}
\begin{frame}
\partpage
\end{frame}

\begin{frame}
\frametitle{Many Specialized Topics}
\begin{itemize}
\item How to design efficient core engine
\item Hybrids with LP/MIP tools
\item Hybrids with SAT
\item Symmetry breaking
\item Use of MDD/BDD to encode sets of solutions
\item High level modelling tools
\item Debugging/visualization
\end{itemize}
\end{frame}

\begin{frame}
\frametitle{Reformulation}
\begin{itemize}
\item Just because the user has modelled it this way, it doesn't mean we have to solve it that way
\begin{itemize}
\item Replace some constraint(s) by other, equivalent constraints
\item Because we don't have that constraint in our system
\item For performance
\end{itemize}
\end{itemize}
\end{frame}

\begin{frame}
\frametitle{Learning}
\begin{itemize}
\item While solving the problem we can learn how to strengthen the model/search 
\begin{itemize}
\item Understand which constraints/method contribute to propagation and change schedule
\item Learn no-good constraints by explaining failure
\item Adapt search strategy based on search experience
\end{itemize}
\end{itemize}
\end{frame}

%% \begin{frame}
%% \frametitle{Refined Process}
%% \begin{center}
%% \begin{tikzpicture}
%% \node[shape=rectangle,fill=pantone127-4] (problem) at (3,7) {Problem};
%% \node (human) at (3,6) {Human};
%% \node[shape=rectangle,fill=pantone127-4] (model) at (3,5) {User Model};
%% \node (reformulation) at (3,4) {Reformulation};
%% \node[shape=rectangle,fill=pantone127-4] (implementation) at (3,3) {Implementation Model};
%% \node (solver) at (3,2) {Constraint Solver/Search};
%% \node[shape=rectangle,fill=pantone157-8] (solution) at (3,1) {Solution};
%% \draw[-] (problem) -- (human);
%% \draw[->] (human) -- (model);
%% \draw[-] (model) -- (reformulation);
%% \draw[->] (reformulation) -- (implementation);
%% \draw[-] (implementation) -- (solver);
%% \draw[->] (solver) -- (0,2) -- (0,3) -- (implementation);
%% \draw[->] (solver) -- (solution);
%% \end{tikzpicture}
%% \end{center}
%% \end{frame}

%\begin{frame}
%\frametitle{What is CP actually used for?}
%\url{http://hsimonis.wordpress.com}
%\includegraphics[width=10cm]{../introduction/images/blog}
%\end{frame}

%% \begin{frame}
%% \frametitle{Exercises}
%% \begin{itemize}
%% \item Install Constraint Solver of your Choice
%% \item MiniZinc (MiningZinc) will be used in later sessions
%% \item Run Sendmoremoney, Sudoku, Queens examples
%% \item Can you choose which propagator to use for alldifferent?
%% \item Can you use different search strategies?
%% \end{itemize}
%% \end{frame}

\begin{frame}
\frametitle{More Learning Resources}
\begin{itemize}
\item Survey of Methods, Resources, and Formats for Teaching Constraint Programming
\begin{itemize}
\item by Tejas Santanam, Helmut Simonis
\item \url{https://doi.org/10.48550/arXiv.2403.12717}
\item Based on survey of community for WTCP 2023
\item \url{https://hsimonis.github.io/WTCP2023/}
\end{itemize}
\end{itemize}
\end{frame}


}

\mode<all>{
\part{Global Constraints}
}
\begin{frame}
\partpage
\end{frame}
\mode<all>{
\begin{frame}
\frametitle{Objectives}
\begin{itemize}
\item Overview of Core Constraint Programming
\item Three Main Concepts
\begin{itemize}
\item Constraint Propagation
\item Global Constraints
\item Customizing Search
\end{itemize}
%\item Get Some Experience with MiniZinc
\item Topics will be treated in more detail in later parts of the school
\item Based on Examples, not Formal Description
\end{itemize}
\end{frame}

\begin{frame}
\frametitle{Outline}
\begin{itemize}
\item Why Constraint Programming?
\item Constraint Propagation
\item Global Constraints
\item Customizing Search
%\item Choices in Modelling
%\item Decomposition
%\item More Global Constraints
%\item Methodology
%\item What is missing?
\end{itemize}
\end{frame}

% \begin{frame}
  % \frametitle{Using MiniZinc IDE}
  % \begin{itemize}
  % \item Developed in the Australian NICTA project
  % \item Maintained by Monash University
  % \item Modelling tool with multiple back-end solvers
  % \item Available from \url{https://www.minizinc.org/}
  % \end{itemize}
% \end{frame}


% \begin{frame}
% \frametitle{Examples in ECLiPSe}
% \begin{itemize}
% \item Open sourced constraint programming language
% \item Development goes back to 1985
% \item ECRC, ICL, IC-Parc, PTL, Cisco
% \item \url{https://eclipseclp.org/}
% \item Specialities
% \begin{itemize}
% \item Develop new solvers for specific domains
% \item Integration with MIP
% \end{itemize}
% \item Not included in bundled MiniZinc IDE
% \item Specialized visualization tools used here
  % \begin{itemize}
  % \item CP-Viz, Simonis et al. 2010
    % \end{itemize}
% \end{itemize}
% \end{frame}

\begin{frame}
\frametitle{Tutorial Based on ECLiPSe ELearning Course}
\begin{itemize}
\item Self-study course in constraint programming
\item Supported by Cisco Systems and Silicon Valley Community Foundation
\item Multi-media format, video lectures, slides, handout etc
\item {\small \url{https://eclipseclp.org/ELearning/index.html}}
\end{itemize}
\end{frame}

\begin{frame}
\frametitle{Also Part of CRT-AI Constraint Week}
\begin{itemize}
\item Annual one week course on CP and Optimization in Ireland
\item Part of national training program for PhD students in AI
\item \url{https://www.crt-ai.ie/}
\end{itemize}
\end{frame}



\begin{frame}
\frametitle{Constraint Programming - in a nutshell}
\begin{itemize}
\item Declarative description of problems with
\begin{itemize}
\item {\em Variables} which range over (finite) sets of values
\item {\em Constraints} over subsets of variables which restrict possible value combinations
\item A {\em solution} is a value assignment which satisfies all constraints
\end{itemize}

\item Constraint propagation/reasoning
\begin{itemize}
\item Removing inconsistent values for variables
\item Detect failure if constraint can not be satisfied
\item Interaction of constraints via shared variables
\item Incomplete
\end{itemize}

\item Search
\begin{itemize}
\item User controlled assignment of values to variables
\item Each step triggers constraint propagation 
\end{itemize}
\item Different domains require/allow different methods
\end{itemize}
\end{frame}

\begin{frame}
  \frametitle{Constraint Programming is Different}
  \begin{itemize}
  \item Declarative Programming
    \begin{itemize}
    \item Concentrate on what you want
      \item Not how to get there
      \item Program != Algorithm
      \item Program = Model
    \end{itemize}
    \item Applied to Combinatorial Problems
      \begin{itemize}
        \item No complete polynomial algorithms known (exist?)
        \item CP less ad-hoc than heuristics
        \item Models can evolve
  \end{itemize}
  \end{itemize}
  \end{frame}
    
\begin{frame}
  \frametitle{A Subtractive Process}
  \begin{textblock}{4}(8,-3)
    \includegraphics[width=4cm]{../introduction/images/stages}
  \end{textblock}
  \vfill
  \begin{quote}
    ``Oh, bosh, as Mr. Ruskin says. Sculpture, per se, is the simplest thing in the world. All you have to do is to take a big chunk of marble and a hammer and chisel, make up your mind what you are about to create and chip off all the marble you don't want.''-Paris Gaulois.
  \end{quote}
  
  {\tiny Source: \url{https://quoteinvestigator.com/2014/06/22/chip-away/}}
\end{frame}

\begin{frame}
\frametitle{Basic Process}
\begin{center}
\begin{tikzpicture}
\node[shape=rectangle,fill=pantone127-4] (problem) at (2,5) {Problem};
\node (human) at (2,4) {Human};
\node[shape=rectangle,fill=pantone127-4] (model) at (2,3) {Model};
\node (solver) at (2,2) {Constraint Solver/Search};
\node[shape=rectangle,fill=pantone157-8] (solution) at (2,1) {Solution};
\draw[-] (problem) -- (human);
\draw[->] (human) -- (model);
\draw[-] (model) -- (solver);
\draw[->] (solver) -- (solution);
\end{tikzpicture}
\end{center}
\end{frame}

\begin{frame}
\frametitle{More Realistic}
\begin{center}
\begin{tikzpicture}[xscale=4.0]
\node[shape=rectangle,fill=pantone127-4] (problem) at (2,5) {Problem};
\node (human) at (2,4) {Human};
\node[shape=rectangle,fill=pantone127-4] (model) at (2,3) {Model};
\node (solver) at (2,2) {Constraint Solver/Search};
\node[shape=rectangle,fill=pantone24-8] (hangs) at (1,1) {Hangs};
\node[shape=rectangle,fill=pantone157-8] (solution) at (2,1) {Solution};
\node[shape=rectangle,fill=pantone24-8] (wrong) at (3,1) {Wrong Solution};
\draw[-] (problem) -- (human);
\draw[->] (human) -- (model);
\draw[-] (model) -- (solver);
\draw[->] (solver) -- (hangs);
\draw[->] (solver) -- (solution);
\draw[->] (solver) -- (wrong);
\draw[->] (hangs) -- (1,4) -- (human);
\draw[->] (wrong) -- (3,4) -- (human);
\end{tikzpicture}
\end{center}
\end{frame}

\begin{frame}
\frametitle{Dual Role of Model}
\begin{itemize}
\item Allows Human to Express Problem
\begin{itemize}
\item Close to Problem Domain
\item Constraints as Abstractions
\end{itemize}
\item Allows Solver to Execute
\begin{itemize}
\item Variables as Communication Mechanism
\item Constraints as Algorithms
\end{itemize}
\end{itemize}
\end{frame}

\begin{frame}
\frametitle{Modelling Frameworks}
\begin{itemize}
\item \href{https://www.minizinc.org/}{MiniZinc} (NICTA, Monash University, Australia)
\item \href{https://github.com/eomahony/Numberjack}{NumberJack} (Insight, Ireland) 
\item \href{https://www-users.york.ac.uk/peter.nightingale/savilerow//}{EssencePrime/SavilleRow} (UK) 
\item \href{https://cpmpy.readthedocs.io/en/latest/}{CPMpy} (KU Leuven)
\item Allow use of multiple back-end solvers
\item Compile model into variants for each solver
\item A priori solver independent model(CP, MIP, SAT)
\end{itemize}
\end{frame} 

\begin{frame}
\frametitle{Framework Process}
\begin{center}
\begin{tikzpicture}[xscale=2]
\node[shape=rectangle,fill=pantone127-4] (problem) at (2.5,5) {Problem};
\node (human) at (2.5,4) {Human};
\node[shape=rectangle,fill=pantone127-4] (model) at (2.5,3) {Model};
\node (compile) at (2.5,2) {Compile/Reformulate};
\node (solver1) at (1,1) {CP};
\node (solver2) at (2,1) {MIP};
\node (solver3) at (3,1) {SAT};
\node (solver4) at (4,1) {Other};
\node[shape=rectangle,fill=pantone157-8] (solution1) at (1,0) {Solution};
\node[shape=rectangle,fill=pantone157-8] (solution2) at (2,0) {Solution};
\node[shape=rectangle,fill=pantone157-8] (solution3) at (3,0) {Solution};
\node[shape=rectangle,fill=pantone157-8] (solution4) at (4,0) {Solution};
\draw[-] (problem) -- (human);
\draw[->] (human) -- (model);
\draw[-] (model) -- (compile);
\draw[-] (compile) -- (solver1);
\draw[-] (compile) -- (solver2);
\draw[-] (compile) -- (solver3);
\draw[-] (compile) -- (solver4);
\draw[->] (solver1) -- (solution1);
\draw[->] (solver2) -- (solution2);
\draw[->] (solver3) -- (solution3);
\draw[->] (solver4) -- (solution4);
\end{tikzpicture}
\end{center}
\end{frame}

% \begin{frame}
  % \frametitle{Do It Now!}
  % \begin{itemize}
  % \item Download and install Minizinc
  % \item \url{https://www.minizinc.org/}
  % \end{itemize}
% \end{frame}


\begin{frame}
\frametitle{Why use Puzzles as Examples?}
\begin{itemize}
\item Easy to understand the problem
\item Solvable by hand without specialized knowledge
\item Possible to compare automated to manual solving process
\end{itemize}

\begin{quote}
The puzzle, though inanimate, is presented as a solvable problem without lasting negative consequences, a very low-risk low-reward situation. By being a puzzle, the object is attempting to convince the user that it must be completed.
\end{quote}
{\tiny Source: \href{https://www.everydayrhetoric.com/2020/03/the-rhetoric-of-puzzles/}{Every Day Rhetoric}}
\end{frame}


\mode<all>{
\part{Basic Constraint Propagation}
}
\begin{frame}
\partpage
\end{frame}
\mode<all>{
\input{../sendmore/body}
}

\mode<all>{
\part{Global Constraints}
}
\begin{frame}
\partpage
\end{frame}
\mode<all>{
\input{../sudoku/body}
}

\clearpage
\mode<all>{
\part{Customizing Search}
}
\begin{frame}
\partpage
\end{frame}
\mode<all>{
\input{../nqueen/body}
}


%% \clearpage
%% \mode<all>{
%%   \part{Problem Decomposition}
%% }
%% \begin{frame}
%%   \partpage
%% \end{frame}
%% \mode<all>{
%%   \input{../party/body}
%%   }

%% \clearpage
%% \mode<all>{
%%   \part{More Global Constraints}
%% }
%% \begin{frame}
%%   \partpage
%% \end{frame}
%% \mode<all>{
%%   \input{../car/body}
%%   }

%% \clearpage
%% \mode<all>{
%%   \part{Methodology}
%% }
%% \begin{frame}
%%   \partpage
%% \end{frame}
%% \mode<all>{
%%   \input{../methodology/body}
%%   }

\clearpage
\mode<all>{
\part{What is missing?}
}
\begin{frame}
\partpage
\end{frame}

\begin{frame}
\frametitle{Many Specialized Topics}
\begin{itemize}
\item How to design efficient core engine
\item Hybrids with LP/MIP tools
\item Hybrids with SAT
\item Symmetry breaking
\item Use of MDD/BDD to encode sets of solutions
\item High level modelling tools
\item Debugging/visualization
\end{itemize}
\end{frame}

\begin{frame}
\frametitle{Reformulation}
\begin{itemize}
\item Just because the user has modelled it this way, it doesn't mean we have to solve it that way
\begin{itemize}
\item Replace some constraint(s) by other, equivalent constraints
\item Because we don't have that constraint in our system
\item For performance
\end{itemize}
\end{itemize}
\end{frame}

\begin{frame}
\frametitle{Learning}
\begin{itemize}
\item While solving the problem we can learn how to strengthen the model/search 
\begin{itemize}
\item Understand which constraints/method contribute to propagation and change schedule
\item Learn no-good constraints by explaining failure
\item Adapt search strategy based on search experience
\end{itemize}
\end{itemize}
\end{frame}

%% \begin{frame}
%% \frametitle{Refined Process}
%% \begin{center}
%% \begin{tikzpicture}
%% \node[shape=rectangle,fill=pantone127-4] (problem) at (3,7) {Problem};
%% \node (human) at (3,6) {Human};
%% \node[shape=rectangle,fill=pantone127-4] (model) at (3,5) {User Model};
%% \node (reformulation) at (3,4) {Reformulation};
%% \node[shape=rectangle,fill=pantone127-4] (implementation) at (3,3) {Implementation Model};
%% \node (solver) at (3,2) {Constraint Solver/Search};
%% \node[shape=rectangle,fill=pantone157-8] (solution) at (3,1) {Solution};
%% \draw[-] (problem) -- (human);
%% \draw[->] (human) -- (model);
%% \draw[-] (model) -- (reformulation);
%% \draw[->] (reformulation) -- (implementation);
%% \draw[-] (implementation) -- (solver);
%% \draw[->] (solver) -- (0,2) -- (0,3) -- (implementation);
%% \draw[->] (solver) -- (solution);
%% \end{tikzpicture}
%% \end{center}
%% \end{frame}

%\begin{frame}
%\frametitle{What is CP actually used for?}
%\url{http://hsimonis.wordpress.com}
%\includegraphics[width=10cm]{../introduction/images/blog}
%\end{frame}

%% \begin{frame}
%% \frametitle{Exercises}
%% \begin{itemize}
%% \item Install Constraint Solver of your Choice
%% \item MiniZinc (MiningZinc) will be used in later sessions
%% \item Run Sendmoremoney, Sudoku, Queens examples
%% \item Can you choose which propagator to use for alldifferent?
%% \item Can you use different search strategies?
%% \end{itemize}
%% \end{frame}

\begin{frame}
\frametitle{More Learning Resources}
\begin{itemize}
\item Survey of Methods, Resources, and Formats for Teaching Constraint Programming
\begin{itemize}
\item by Tejas Santanam, Helmut Simonis
\item \url{https://doi.org/10.48550/arXiv.2403.12717}
\item Based on survey of community for WTCP 2023
\item \url{https://hsimonis.github.io/WTCP2023/}
\end{itemize}
\end{itemize}
\end{frame}


}

\clearpage
\mode<all>{
\part{Customizing Search}
}
\begin{frame}
\partpage
\end{frame}
\mode<all>{
\begin{frame}
\frametitle{Objectives}
\begin{itemize}
\item Overview of Core Constraint Programming
\item Three Main Concepts
\begin{itemize}
\item Constraint Propagation
\item Global Constraints
\item Customizing Search
\end{itemize}
%\item Get Some Experience with MiniZinc
\item Topics will be treated in more detail in later parts of the school
\item Based on Examples, not Formal Description
\end{itemize}
\end{frame}

\begin{frame}
\frametitle{Outline}
\begin{itemize}
\item Why Constraint Programming?
\item Constraint Propagation
\item Global Constraints
\item Customizing Search
%\item Choices in Modelling
%\item Decomposition
%\item More Global Constraints
%\item Methodology
%\item What is missing?
\end{itemize}
\end{frame}

% \begin{frame}
  % \frametitle{Using MiniZinc IDE}
  % \begin{itemize}
  % \item Developed in the Australian NICTA project
  % \item Maintained by Monash University
  % \item Modelling tool with multiple back-end solvers
  % \item Available from \url{https://www.minizinc.org/}
  % \end{itemize}
% \end{frame}


% \begin{frame}
% \frametitle{Examples in ECLiPSe}
% \begin{itemize}
% \item Open sourced constraint programming language
% \item Development goes back to 1985
% \item ECRC, ICL, IC-Parc, PTL, Cisco
% \item \url{https://eclipseclp.org/}
% \item Specialities
% \begin{itemize}
% \item Develop new solvers for specific domains
% \item Integration with MIP
% \end{itemize}
% \item Not included in bundled MiniZinc IDE
% \item Specialized visualization tools used here
  % \begin{itemize}
  % \item CP-Viz, Simonis et al. 2010
    % \end{itemize}
% \end{itemize}
% \end{frame}

\begin{frame}
\frametitle{Tutorial Based on ECLiPSe ELearning Course}
\begin{itemize}
\item Self-study course in constraint programming
\item Supported by Cisco Systems and Silicon Valley Community Foundation
\item Multi-media format, video lectures, slides, handout etc
\item {\small \url{https://eclipseclp.org/ELearning/index.html}}
\end{itemize}
\end{frame}

\begin{frame}
\frametitle{Also Part of CRT-AI Constraint Week}
\begin{itemize}
\item Annual one week course on CP and Optimization in Ireland
\item Part of national training program for PhD students in AI
\item \url{https://www.crt-ai.ie/}
\end{itemize}
\end{frame}



\begin{frame}
\frametitle{Constraint Programming - in a nutshell}
\begin{itemize}
\item Declarative description of problems with
\begin{itemize}
\item {\em Variables} which range over (finite) sets of values
\item {\em Constraints} over subsets of variables which restrict possible value combinations
\item A {\em solution} is a value assignment which satisfies all constraints
\end{itemize}

\item Constraint propagation/reasoning
\begin{itemize}
\item Removing inconsistent values for variables
\item Detect failure if constraint can not be satisfied
\item Interaction of constraints via shared variables
\item Incomplete
\end{itemize}

\item Search
\begin{itemize}
\item User controlled assignment of values to variables
\item Each step triggers constraint propagation 
\end{itemize}
\item Different domains require/allow different methods
\end{itemize}
\end{frame}

\begin{frame}
  \frametitle{Constraint Programming is Different}
  \begin{itemize}
  \item Declarative Programming
    \begin{itemize}
    \item Concentrate on what you want
      \item Not how to get there
      \item Program != Algorithm
      \item Program = Model
    \end{itemize}
    \item Applied to Combinatorial Problems
      \begin{itemize}
        \item No complete polynomial algorithms known (exist?)
        \item CP less ad-hoc than heuristics
        \item Models can evolve
  \end{itemize}
  \end{itemize}
  \end{frame}
    
\begin{frame}
  \frametitle{A Subtractive Process}
  \begin{textblock}{4}(8,-3)
    \includegraphics[width=4cm]{../introduction/images/stages}
  \end{textblock}
  \vfill
  \begin{quote}
    ``Oh, bosh, as Mr. Ruskin says. Sculpture, per se, is the simplest thing in the world. All you have to do is to take a big chunk of marble and a hammer and chisel, make up your mind what you are about to create and chip off all the marble you don't want.''-Paris Gaulois.
  \end{quote}
  
  {\tiny Source: \url{https://quoteinvestigator.com/2014/06/22/chip-away/}}
\end{frame}

\begin{frame}
\frametitle{Basic Process}
\begin{center}
\begin{tikzpicture}
\node[shape=rectangle,fill=pantone127-4] (problem) at (2,5) {Problem};
\node (human) at (2,4) {Human};
\node[shape=rectangle,fill=pantone127-4] (model) at (2,3) {Model};
\node (solver) at (2,2) {Constraint Solver/Search};
\node[shape=rectangle,fill=pantone157-8] (solution) at (2,1) {Solution};
\draw[-] (problem) -- (human);
\draw[->] (human) -- (model);
\draw[-] (model) -- (solver);
\draw[->] (solver) -- (solution);
\end{tikzpicture}
\end{center}
\end{frame}

\begin{frame}
\frametitle{More Realistic}
\begin{center}
\begin{tikzpicture}[xscale=4.0]
\node[shape=rectangle,fill=pantone127-4] (problem) at (2,5) {Problem};
\node (human) at (2,4) {Human};
\node[shape=rectangle,fill=pantone127-4] (model) at (2,3) {Model};
\node (solver) at (2,2) {Constraint Solver/Search};
\node[shape=rectangle,fill=pantone24-8] (hangs) at (1,1) {Hangs};
\node[shape=rectangle,fill=pantone157-8] (solution) at (2,1) {Solution};
\node[shape=rectangle,fill=pantone24-8] (wrong) at (3,1) {Wrong Solution};
\draw[-] (problem) -- (human);
\draw[->] (human) -- (model);
\draw[-] (model) -- (solver);
\draw[->] (solver) -- (hangs);
\draw[->] (solver) -- (solution);
\draw[->] (solver) -- (wrong);
\draw[->] (hangs) -- (1,4) -- (human);
\draw[->] (wrong) -- (3,4) -- (human);
\end{tikzpicture}
\end{center}
\end{frame}

\begin{frame}
\frametitle{Dual Role of Model}
\begin{itemize}
\item Allows Human to Express Problem
\begin{itemize}
\item Close to Problem Domain
\item Constraints as Abstractions
\end{itemize}
\item Allows Solver to Execute
\begin{itemize}
\item Variables as Communication Mechanism
\item Constraints as Algorithms
\end{itemize}
\end{itemize}
\end{frame}

\begin{frame}
\frametitle{Modelling Frameworks}
\begin{itemize}
\item \href{https://www.minizinc.org/}{MiniZinc} (NICTA, Monash University, Australia)
\item \href{https://github.com/eomahony/Numberjack}{NumberJack} (Insight, Ireland) 
\item \href{https://www-users.york.ac.uk/peter.nightingale/savilerow//}{EssencePrime/SavilleRow} (UK) 
\item \href{https://cpmpy.readthedocs.io/en/latest/}{CPMpy} (KU Leuven)
\item Allow use of multiple back-end solvers
\item Compile model into variants for each solver
\item A priori solver independent model(CP, MIP, SAT)
\end{itemize}
\end{frame} 

\begin{frame}
\frametitle{Framework Process}
\begin{center}
\begin{tikzpicture}[xscale=2]
\node[shape=rectangle,fill=pantone127-4] (problem) at (2.5,5) {Problem};
\node (human) at (2.5,4) {Human};
\node[shape=rectangle,fill=pantone127-4] (model) at (2.5,3) {Model};
\node (compile) at (2.5,2) {Compile/Reformulate};
\node (solver1) at (1,1) {CP};
\node (solver2) at (2,1) {MIP};
\node (solver3) at (3,1) {SAT};
\node (solver4) at (4,1) {Other};
\node[shape=rectangle,fill=pantone157-8] (solution1) at (1,0) {Solution};
\node[shape=rectangle,fill=pantone157-8] (solution2) at (2,0) {Solution};
\node[shape=rectangle,fill=pantone157-8] (solution3) at (3,0) {Solution};
\node[shape=rectangle,fill=pantone157-8] (solution4) at (4,0) {Solution};
\draw[-] (problem) -- (human);
\draw[->] (human) -- (model);
\draw[-] (model) -- (compile);
\draw[-] (compile) -- (solver1);
\draw[-] (compile) -- (solver2);
\draw[-] (compile) -- (solver3);
\draw[-] (compile) -- (solver4);
\draw[->] (solver1) -- (solution1);
\draw[->] (solver2) -- (solution2);
\draw[->] (solver3) -- (solution3);
\draw[->] (solver4) -- (solution4);
\end{tikzpicture}
\end{center}
\end{frame}

% \begin{frame}
  % \frametitle{Do It Now!}
  % \begin{itemize}
  % \item Download and install Minizinc
  % \item \url{https://www.minizinc.org/}
  % \end{itemize}
% \end{frame}


\begin{frame}
\frametitle{Why use Puzzles as Examples?}
\begin{itemize}
\item Easy to understand the problem
\item Solvable by hand without specialized knowledge
\item Possible to compare automated to manual solving process
\end{itemize}

\begin{quote}
The puzzle, though inanimate, is presented as a solvable problem without lasting negative consequences, a very low-risk low-reward situation. By being a puzzle, the object is attempting to convince the user that it must be completed.
\end{quote}
{\tiny Source: \href{https://www.everydayrhetoric.com/2020/03/the-rhetoric-of-puzzles/}{Every Day Rhetoric}}
\end{frame}


\mode<all>{
\part{Basic Constraint Propagation}
}
\begin{frame}
\partpage
\end{frame}
\mode<all>{
\input{../sendmore/body}
}

\mode<all>{
\part{Global Constraints}
}
\begin{frame}
\partpage
\end{frame}
\mode<all>{
\input{../sudoku/body}
}

\clearpage
\mode<all>{
\part{Customizing Search}
}
\begin{frame}
\partpage
\end{frame}
\mode<all>{
\input{../nqueen/body}
}


%% \clearpage
%% \mode<all>{
%%   \part{Problem Decomposition}
%% }
%% \begin{frame}
%%   \partpage
%% \end{frame}
%% \mode<all>{
%%   \input{../party/body}
%%   }

%% \clearpage
%% \mode<all>{
%%   \part{More Global Constraints}
%% }
%% \begin{frame}
%%   \partpage
%% \end{frame}
%% \mode<all>{
%%   \input{../car/body}
%%   }

%% \clearpage
%% \mode<all>{
%%   \part{Methodology}
%% }
%% \begin{frame}
%%   \partpage
%% \end{frame}
%% \mode<all>{
%%   \input{../methodology/body}
%%   }

\clearpage
\mode<all>{
\part{What is missing?}
}
\begin{frame}
\partpage
\end{frame}

\begin{frame}
\frametitle{Many Specialized Topics}
\begin{itemize}
\item How to design efficient core engine
\item Hybrids with LP/MIP tools
\item Hybrids with SAT
\item Symmetry breaking
\item Use of MDD/BDD to encode sets of solutions
\item High level modelling tools
\item Debugging/visualization
\end{itemize}
\end{frame}

\begin{frame}
\frametitle{Reformulation}
\begin{itemize}
\item Just because the user has modelled it this way, it doesn't mean we have to solve it that way
\begin{itemize}
\item Replace some constraint(s) by other, equivalent constraints
\item Because we don't have that constraint in our system
\item For performance
\end{itemize}
\end{itemize}
\end{frame}

\begin{frame}
\frametitle{Learning}
\begin{itemize}
\item While solving the problem we can learn how to strengthen the model/search 
\begin{itemize}
\item Understand which constraints/method contribute to propagation and change schedule
\item Learn no-good constraints by explaining failure
\item Adapt search strategy based on search experience
\end{itemize}
\end{itemize}
\end{frame}

%% \begin{frame}
%% \frametitle{Refined Process}
%% \begin{center}
%% \begin{tikzpicture}
%% \node[shape=rectangle,fill=pantone127-4] (problem) at (3,7) {Problem};
%% \node (human) at (3,6) {Human};
%% \node[shape=rectangle,fill=pantone127-4] (model) at (3,5) {User Model};
%% \node (reformulation) at (3,4) {Reformulation};
%% \node[shape=rectangle,fill=pantone127-4] (implementation) at (3,3) {Implementation Model};
%% \node (solver) at (3,2) {Constraint Solver/Search};
%% \node[shape=rectangle,fill=pantone157-8] (solution) at (3,1) {Solution};
%% \draw[-] (problem) -- (human);
%% \draw[->] (human) -- (model);
%% \draw[-] (model) -- (reformulation);
%% \draw[->] (reformulation) -- (implementation);
%% \draw[-] (implementation) -- (solver);
%% \draw[->] (solver) -- (0,2) -- (0,3) -- (implementation);
%% \draw[->] (solver) -- (solution);
%% \end{tikzpicture}
%% \end{center}
%% \end{frame}

%\begin{frame}
%\frametitle{What is CP actually used for?}
%\url{http://hsimonis.wordpress.com}
%\includegraphics[width=10cm]{../introduction/images/blog}
%\end{frame}

%% \begin{frame}
%% \frametitle{Exercises}
%% \begin{itemize}
%% \item Install Constraint Solver of your Choice
%% \item MiniZinc (MiningZinc) will be used in later sessions
%% \item Run Sendmoremoney, Sudoku, Queens examples
%% \item Can you choose which propagator to use for alldifferent?
%% \item Can you use different search strategies?
%% \end{itemize}
%% \end{frame}

\begin{frame}
\frametitle{More Learning Resources}
\begin{itemize}
\item Survey of Methods, Resources, and Formats for Teaching Constraint Programming
\begin{itemize}
\item by Tejas Santanam, Helmut Simonis
\item \url{https://doi.org/10.48550/arXiv.2403.12717}
\item Based on survey of community for WTCP 2023
\item \url{https://hsimonis.github.io/WTCP2023/}
\end{itemize}
\end{itemize}
\end{frame}


}


%% \clearpage
%% \mode<all>{
%%   \part{Problem Decomposition}
%% }
%% \begin{frame}
%%   \partpage
%% \end{frame}
%% \mode<all>{
%%   \begin{frame}
\frametitle{Objectives}
\begin{itemize}
\item Overview of Core Constraint Programming
\item Three Main Concepts
\begin{itemize}
\item Constraint Propagation
\item Global Constraints
\item Customizing Search
\end{itemize}
%\item Get Some Experience with MiniZinc
\item Topics will be treated in more detail in later parts of the school
\item Based on Examples, not Formal Description
\end{itemize}
\end{frame}

\begin{frame}
\frametitle{Outline}
\begin{itemize}
\item Why Constraint Programming?
\item Constraint Propagation
\item Global Constraints
\item Customizing Search
%\item Choices in Modelling
%\item Decomposition
%\item More Global Constraints
%\item Methodology
%\item What is missing?
\end{itemize}
\end{frame}

% \begin{frame}
  % \frametitle{Using MiniZinc IDE}
  % \begin{itemize}
  % \item Developed in the Australian NICTA project
  % \item Maintained by Monash University
  % \item Modelling tool with multiple back-end solvers
  % \item Available from \url{https://www.minizinc.org/}
  % \end{itemize}
% \end{frame}


% \begin{frame}
% \frametitle{Examples in ECLiPSe}
% \begin{itemize}
% \item Open sourced constraint programming language
% \item Development goes back to 1985
% \item ECRC, ICL, IC-Parc, PTL, Cisco
% \item \url{https://eclipseclp.org/}
% \item Specialities
% \begin{itemize}
% \item Develop new solvers for specific domains
% \item Integration with MIP
% \end{itemize}
% \item Not included in bundled MiniZinc IDE
% \item Specialized visualization tools used here
  % \begin{itemize}
  % \item CP-Viz, Simonis et al. 2010
    % \end{itemize}
% \end{itemize}
% \end{frame}

\begin{frame}
\frametitle{Tutorial Based on ECLiPSe ELearning Course}
\begin{itemize}
\item Self-study course in constraint programming
\item Supported by Cisco Systems and Silicon Valley Community Foundation
\item Multi-media format, video lectures, slides, handout etc
\item {\small \url{https://eclipseclp.org/ELearning/index.html}}
\end{itemize}
\end{frame}

\begin{frame}
\frametitle{Also Part of CRT-AI Constraint Week}
\begin{itemize}
\item Annual one week course on CP and Optimization in Ireland
\item Part of national training program for PhD students in AI
\item \url{https://www.crt-ai.ie/}
\end{itemize}
\end{frame}



\begin{frame}
\frametitle{Constraint Programming - in a nutshell}
\begin{itemize}
\item Declarative description of problems with
\begin{itemize}
\item {\em Variables} which range over (finite) sets of values
\item {\em Constraints} over subsets of variables which restrict possible value combinations
\item A {\em solution} is a value assignment which satisfies all constraints
\end{itemize}

\item Constraint propagation/reasoning
\begin{itemize}
\item Removing inconsistent values for variables
\item Detect failure if constraint can not be satisfied
\item Interaction of constraints via shared variables
\item Incomplete
\end{itemize}

\item Search
\begin{itemize}
\item User controlled assignment of values to variables
\item Each step triggers constraint propagation 
\end{itemize}
\item Different domains require/allow different methods
\end{itemize}
\end{frame}

\begin{frame}
  \frametitle{Constraint Programming is Different}
  \begin{itemize}
  \item Declarative Programming
    \begin{itemize}
    \item Concentrate on what you want
      \item Not how to get there
      \item Program != Algorithm
      \item Program = Model
    \end{itemize}
    \item Applied to Combinatorial Problems
      \begin{itemize}
        \item No complete polynomial algorithms known (exist?)
        \item CP less ad-hoc than heuristics
        \item Models can evolve
  \end{itemize}
  \end{itemize}
  \end{frame}
    
\begin{frame}
  \frametitle{A Subtractive Process}
  \begin{textblock}{4}(8,-3)
    \includegraphics[width=4cm]{../introduction/images/stages}
  \end{textblock}
  \vfill
  \begin{quote}
    ``Oh, bosh, as Mr. Ruskin says. Sculpture, per se, is the simplest thing in the world. All you have to do is to take a big chunk of marble and a hammer and chisel, make up your mind what you are about to create and chip off all the marble you don't want.''-Paris Gaulois.
  \end{quote}
  
  {\tiny Source: \url{https://quoteinvestigator.com/2014/06/22/chip-away/}}
\end{frame}

\begin{frame}
\frametitle{Basic Process}
\begin{center}
\begin{tikzpicture}
\node[shape=rectangle,fill=pantone127-4] (problem) at (2,5) {Problem};
\node (human) at (2,4) {Human};
\node[shape=rectangle,fill=pantone127-4] (model) at (2,3) {Model};
\node (solver) at (2,2) {Constraint Solver/Search};
\node[shape=rectangle,fill=pantone157-8] (solution) at (2,1) {Solution};
\draw[-] (problem) -- (human);
\draw[->] (human) -- (model);
\draw[-] (model) -- (solver);
\draw[->] (solver) -- (solution);
\end{tikzpicture}
\end{center}
\end{frame}

\begin{frame}
\frametitle{More Realistic}
\begin{center}
\begin{tikzpicture}[xscale=4.0]
\node[shape=rectangle,fill=pantone127-4] (problem) at (2,5) {Problem};
\node (human) at (2,4) {Human};
\node[shape=rectangle,fill=pantone127-4] (model) at (2,3) {Model};
\node (solver) at (2,2) {Constraint Solver/Search};
\node[shape=rectangle,fill=pantone24-8] (hangs) at (1,1) {Hangs};
\node[shape=rectangle,fill=pantone157-8] (solution) at (2,1) {Solution};
\node[shape=rectangle,fill=pantone24-8] (wrong) at (3,1) {Wrong Solution};
\draw[-] (problem) -- (human);
\draw[->] (human) -- (model);
\draw[-] (model) -- (solver);
\draw[->] (solver) -- (hangs);
\draw[->] (solver) -- (solution);
\draw[->] (solver) -- (wrong);
\draw[->] (hangs) -- (1,4) -- (human);
\draw[->] (wrong) -- (3,4) -- (human);
\end{tikzpicture}
\end{center}
\end{frame}

\begin{frame}
\frametitle{Dual Role of Model}
\begin{itemize}
\item Allows Human to Express Problem
\begin{itemize}
\item Close to Problem Domain
\item Constraints as Abstractions
\end{itemize}
\item Allows Solver to Execute
\begin{itemize}
\item Variables as Communication Mechanism
\item Constraints as Algorithms
\end{itemize}
\end{itemize}
\end{frame}

\begin{frame}
\frametitle{Modelling Frameworks}
\begin{itemize}
\item \href{https://www.minizinc.org/}{MiniZinc} (NICTA, Monash University, Australia)
\item \href{https://github.com/eomahony/Numberjack}{NumberJack} (Insight, Ireland) 
\item \href{https://www-users.york.ac.uk/peter.nightingale/savilerow//}{EssencePrime/SavilleRow} (UK) 
\item \href{https://cpmpy.readthedocs.io/en/latest/}{CPMpy} (KU Leuven)
\item Allow use of multiple back-end solvers
\item Compile model into variants for each solver
\item A priori solver independent model(CP, MIP, SAT)
\end{itemize}
\end{frame} 

\begin{frame}
\frametitle{Framework Process}
\begin{center}
\begin{tikzpicture}[xscale=2]
\node[shape=rectangle,fill=pantone127-4] (problem) at (2.5,5) {Problem};
\node (human) at (2.5,4) {Human};
\node[shape=rectangle,fill=pantone127-4] (model) at (2.5,3) {Model};
\node (compile) at (2.5,2) {Compile/Reformulate};
\node (solver1) at (1,1) {CP};
\node (solver2) at (2,1) {MIP};
\node (solver3) at (3,1) {SAT};
\node (solver4) at (4,1) {Other};
\node[shape=rectangle,fill=pantone157-8] (solution1) at (1,0) {Solution};
\node[shape=rectangle,fill=pantone157-8] (solution2) at (2,0) {Solution};
\node[shape=rectangle,fill=pantone157-8] (solution3) at (3,0) {Solution};
\node[shape=rectangle,fill=pantone157-8] (solution4) at (4,0) {Solution};
\draw[-] (problem) -- (human);
\draw[->] (human) -- (model);
\draw[-] (model) -- (compile);
\draw[-] (compile) -- (solver1);
\draw[-] (compile) -- (solver2);
\draw[-] (compile) -- (solver3);
\draw[-] (compile) -- (solver4);
\draw[->] (solver1) -- (solution1);
\draw[->] (solver2) -- (solution2);
\draw[->] (solver3) -- (solution3);
\draw[->] (solver4) -- (solution4);
\end{tikzpicture}
\end{center}
\end{frame}

% \begin{frame}
  % \frametitle{Do It Now!}
  % \begin{itemize}
  % \item Download and install Minizinc
  % \item \url{https://www.minizinc.org/}
  % \end{itemize}
% \end{frame}


\begin{frame}
\frametitle{Why use Puzzles as Examples?}
\begin{itemize}
\item Easy to understand the problem
\item Solvable by hand without specialized knowledge
\item Possible to compare automated to manual solving process
\end{itemize}

\begin{quote}
The puzzle, though inanimate, is presented as a solvable problem without lasting negative consequences, a very low-risk low-reward situation. By being a puzzle, the object is attempting to convince the user that it must be completed.
\end{quote}
{\tiny Source: \href{https://www.everydayrhetoric.com/2020/03/the-rhetoric-of-puzzles/}{Every Day Rhetoric}}
\end{frame}


\mode<all>{
\part{Basic Constraint Propagation}
}
\begin{frame}
\partpage
\end{frame}
\mode<all>{
\input{../sendmore/body}
}

\mode<all>{
\part{Global Constraints}
}
\begin{frame}
\partpage
\end{frame}
\mode<all>{
\input{../sudoku/body}
}

\clearpage
\mode<all>{
\part{Customizing Search}
}
\begin{frame}
\partpage
\end{frame}
\mode<all>{
\input{../nqueen/body}
}


%% \clearpage
%% \mode<all>{
%%   \part{Problem Decomposition}
%% }
%% \begin{frame}
%%   \partpage
%% \end{frame}
%% \mode<all>{
%%   \input{../party/body}
%%   }

%% \clearpage
%% \mode<all>{
%%   \part{More Global Constraints}
%% }
%% \begin{frame}
%%   \partpage
%% \end{frame}
%% \mode<all>{
%%   \input{../car/body}
%%   }

%% \clearpage
%% \mode<all>{
%%   \part{Methodology}
%% }
%% \begin{frame}
%%   \partpage
%% \end{frame}
%% \mode<all>{
%%   \input{../methodology/body}
%%   }

\clearpage
\mode<all>{
\part{What is missing?}
}
\begin{frame}
\partpage
\end{frame}

\begin{frame}
\frametitle{Many Specialized Topics}
\begin{itemize}
\item How to design efficient core engine
\item Hybrids with LP/MIP tools
\item Hybrids with SAT
\item Symmetry breaking
\item Use of MDD/BDD to encode sets of solutions
\item High level modelling tools
\item Debugging/visualization
\end{itemize}
\end{frame}

\begin{frame}
\frametitle{Reformulation}
\begin{itemize}
\item Just because the user has modelled it this way, it doesn't mean we have to solve it that way
\begin{itemize}
\item Replace some constraint(s) by other, equivalent constraints
\item Because we don't have that constraint in our system
\item For performance
\end{itemize}
\end{itemize}
\end{frame}

\begin{frame}
\frametitle{Learning}
\begin{itemize}
\item While solving the problem we can learn how to strengthen the model/search 
\begin{itemize}
\item Understand which constraints/method contribute to propagation and change schedule
\item Learn no-good constraints by explaining failure
\item Adapt search strategy based on search experience
\end{itemize}
\end{itemize}
\end{frame}

%% \begin{frame}
%% \frametitle{Refined Process}
%% \begin{center}
%% \begin{tikzpicture}
%% \node[shape=rectangle,fill=pantone127-4] (problem) at (3,7) {Problem};
%% \node (human) at (3,6) {Human};
%% \node[shape=rectangle,fill=pantone127-4] (model) at (3,5) {User Model};
%% \node (reformulation) at (3,4) {Reformulation};
%% \node[shape=rectangle,fill=pantone127-4] (implementation) at (3,3) {Implementation Model};
%% \node (solver) at (3,2) {Constraint Solver/Search};
%% \node[shape=rectangle,fill=pantone157-8] (solution) at (3,1) {Solution};
%% \draw[-] (problem) -- (human);
%% \draw[->] (human) -- (model);
%% \draw[-] (model) -- (reformulation);
%% \draw[->] (reformulation) -- (implementation);
%% \draw[-] (implementation) -- (solver);
%% \draw[->] (solver) -- (0,2) -- (0,3) -- (implementation);
%% \draw[->] (solver) -- (solution);
%% \end{tikzpicture}
%% \end{center}
%% \end{frame}

%\begin{frame}
%\frametitle{What is CP actually used for?}
%\url{http://hsimonis.wordpress.com}
%\includegraphics[width=10cm]{../introduction/images/blog}
%\end{frame}

%% \begin{frame}
%% \frametitle{Exercises}
%% \begin{itemize}
%% \item Install Constraint Solver of your Choice
%% \item MiniZinc (MiningZinc) will be used in later sessions
%% \item Run Sendmoremoney, Sudoku, Queens examples
%% \item Can you choose which propagator to use for alldifferent?
%% \item Can you use different search strategies?
%% \end{itemize}
%% \end{frame}

\begin{frame}
\frametitle{More Learning Resources}
\begin{itemize}
\item Survey of Methods, Resources, and Formats for Teaching Constraint Programming
\begin{itemize}
\item by Tejas Santanam, Helmut Simonis
\item \url{https://doi.org/10.48550/arXiv.2403.12717}
\item Based on survey of community for WTCP 2023
\item \url{https://hsimonis.github.io/WTCP2023/}
\end{itemize}
\end{itemize}
\end{frame}


%%   }

%% \clearpage
%% \mode<all>{
%%   \part{More Global Constraints}
%% }
%% \begin{frame}
%%   \partpage
%% \end{frame}
%% \mode<all>{
%%   \begin{frame}
\frametitle{Objectives}
\begin{itemize}
\item Overview of Core Constraint Programming
\item Three Main Concepts
\begin{itemize}
\item Constraint Propagation
\item Global Constraints
\item Customizing Search
\end{itemize}
%\item Get Some Experience with MiniZinc
\item Topics will be treated in more detail in later parts of the school
\item Based on Examples, not Formal Description
\end{itemize}
\end{frame}

\begin{frame}
\frametitle{Outline}
\begin{itemize}
\item Why Constraint Programming?
\item Constraint Propagation
\item Global Constraints
\item Customizing Search
%\item Choices in Modelling
%\item Decomposition
%\item More Global Constraints
%\item Methodology
%\item What is missing?
\end{itemize}
\end{frame}

% \begin{frame}
  % \frametitle{Using MiniZinc IDE}
  % \begin{itemize}
  % \item Developed in the Australian NICTA project
  % \item Maintained by Monash University
  % \item Modelling tool with multiple back-end solvers
  % \item Available from \url{https://www.minizinc.org/}
  % \end{itemize}
% \end{frame}


% \begin{frame}
% \frametitle{Examples in ECLiPSe}
% \begin{itemize}
% \item Open sourced constraint programming language
% \item Development goes back to 1985
% \item ECRC, ICL, IC-Parc, PTL, Cisco
% \item \url{https://eclipseclp.org/}
% \item Specialities
% \begin{itemize}
% \item Develop new solvers for specific domains
% \item Integration with MIP
% \end{itemize}
% \item Not included in bundled MiniZinc IDE
% \item Specialized visualization tools used here
  % \begin{itemize}
  % \item CP-Viz, Simonis et al. 2010
    % \end{itemize}
% \end{itemize}
% \end{frame}

\begin{frame}
\frametitle{Tutorial Based on ECLiPSe ELearning Course}
\begin{itemize}
\item Self-study course in constraint programming
\item Supported by Cisco Systems and Silicon Valley Community Foundation
\item Multi-media format, video lectures, slides, handout etc
\item {\small \url{https://eclipseclp.org/ELearning/index.html}}
\end{itemize}
\end{frame}

\begin{frame}
\frametitle{Also Part of CRT-AI Constraint Week}
\begin{itemize}
\item Annual one week course on CP and Optimization in Ireland
\item Part of national training program for PhD students in AI
\item \url{https://www.crt-ai.ie/}
\end{itemize}
\end{frame}



\begin{frame}
\frametitle{Constraint Programming - in a nutshell}
\begin{itemize}
\item Declarative description of problems with
\begin{itemize}
\item {\em Variables} which range over (finite) sets of values
\item {\em Constraints} over subsets of variables which restrict possible value combinations
\item A {\em solution} is a value assignment which satisfies all constraints
\end{itemize}

\item Constraint propagation/reasoning
\begin{itemize}
\item Removing inconsistent values for variables
\item Detect failure if constraint can not be satisfied
\item Interaction of constraints via shared variables
\item Incomplete
\end{itemize}

\item Search
\begin{itemize}
\item User controlled assignment of values to variables
\item Each step triggers constraint propagation 
\end{itemize}
\item Different domains require/allow different methods
\end{itemize}
\end{frame}

\begin{frame}
  \frametitle{Constraint Programming is Different}
  \begin{itemize}
  \item Declarative Programming
    \begin{itemize}
    \item Concentrate on what you want
      \item Not how to get there
      \item Program != Algorithm
      \item Program = Model
    \end{itemize}
    \item Applied to Combinatorial Problems
      \begin{itemize}
        \item No complete polynomial algorithms known (exist?)
        \item CP less ad-hoc than heuristics
        \item Models can evolve
  \end{itemize}
  \end{itemize}
  \end{frame}
    
\begin{frame}
  \frametitle{A Subtractive Process}
  \begin{textblock}{4}(8,-3)
    \includegraphics[width=4cm]{../introduction/images/stages}
  \end{textblock}
  \vfill
  \begin{quote}
    ``Oh, bosh, as Mr. Ruskin says. Sculpture, per se, is the simplest thing in the world. All you have to do is to take a big chunk of marble and a hammer and chisel, make up your mind what you are about to create and chip off all the marble you don't want.''-Paris Gaulois.
  \end{quote}
  
  {\tiny Source: \url{https://quoteinvestigator.com/2014/06/22/chip-away/}}
\end{frame}

\begin{frame}
\frametitle{Basic Process}
\begin{center}
\begin{tikzpicture}
\node[shape=rectangle,fill=pantone127-4] (problem) at (2,5) {Problem};
\node (human) at (2,4) {Human};
\node[shape=rectangle,fill=pantone127-4] (model) at (2,3) {Model};
\node (solver) at (2,2) {Constraint Solver/Search};
\node[shape=rectangle,fill=pantone157-8] (solution) at (2,1) {Solution};
\draw[-] (problem) -- (human);
\draw[->] (human) -- (model);
\draw[-] (model) -- (solver);
\draw[->] (solver) -- (solution);
\end{tikzpicture}
\end{center}
\end{frame}

\begin{frame}
\frametitle{More Realistic}
\begin{center}
\begin{tikzpicture}[xscale=4.0]
\node[shape=rectangle,fill=pantone127-4] (problem) at (2,5) {Problem};
\node (human) at (2,4) {Human};
\node[shape=rectangle,fill=pantone127-4] (model) at (2,3) {Model};
\node (solver) at (2,2) {Constraint Solver/Search};
\node[shape=rectangle,fill=pantone24-8] (hangs) at (1,1) {Hangs};
\node[shape=rectangle,fill=pantone157-8] (solution) at (2,1) {Solution};
\node[shape=rectangle,fill=pantone24-8] (wrong) at (3,1) {Wrong Solution};
\draw[-] (problem) -- (human);
\draw[->] (human) -- (model);
\draw[-] (model) -- (solver);
\draw[->] (solver) -- (hangs);
\draw[->] (solver) -- (solution);
\draw[->] (solver) -- (wrong);
\draw[->] (hangs) -- (1,4) -- (human);
\draw[->] (wrong) -- (3,4) -- (human);
\end{tikzpicture}
\end{center}
\end{frame}

\begin{frame}
\frametitle{Dual Role of Model}
\begin{itemize}
\item Allows Human to Express Problem
\begin{itemize}
\item Close to Problem Domain
\item Constraints as Abstractions
\end{itemize}
\item Allows Solver to Execute
\begin{itemize}
\item Variables as Communication Mechanism
\item Constraints as Algorithms
\end{itemize}
\end{itemize}
\end{frame}

\begin{frame}
\frametitle{Modelling Frameworks}
\begin{itemize}
\item \href{https://www.minizinc.org/}{MiniZinc} (NICTA, Monash University, Australia)
\item \href{https://github.com/eomahony/Numberjack}{NumberJack} (Insight, Ireland) 
\item \href{https://www-users.york.ac.uk/peter.nightingale/savilerow//}{EssencePrime/SavilleRow} (UK) 
\item \href{https://cpmpy.readthedocs.io/en/latest/}{CPMpy} (KU Leuven)
\item Allow use of multiple back-end solvers
\item Compile model into variants for each solver
\item A priori solver independent model(CP, MIP, SAT)
\end{itemize}
\end{frame} 

\begin{frame}
\frametitle{Framework Process}
\begin{center}
\begin{tikzpicture}[xscale=2]
\node[shape=rectangle,fill=pantone127-4] (problem) at (2.5,5) {Problem};
\node (human) at (2.5,4) {Human};
\node[shape=rectangle,fill=pantone127-4] (model) at (2.5,3) {Model};
\node (compile) at (2.5,2) {Compile/Reformulate};
\node (solver1) at (1,1) {CP};
\node (solver2) at (2,1) {MIP};
\node (solver3) at (3,1) {SAT};
\node (solver4) at (4,1) {Other};
\node[shape=rectangle,fill=pantone157-8] (solution1) at (1,0) {Solution};
\node[shape=rectangle,fill=pantone157-8] (solution2) at (2,0) {Solution};
\node[shape=rectangle,fill=pantone157-8] (solution3) at (3,0) {Solution};
\node[shape=rectangle,fill=pantone157-8] (solution4) at (4,0) {Solution};
\draw[-] (problem) -- (human);
\draw[->] (human) -- (model);
\draw[-] (model) -- (compile);
\draw[-] (compile) -- (solver1);
\draw[-] (compile) -- (solver2);
\draw[-] (compile) -- (solver3);
\draw[-] (compile) -- (solver4);
\draw[->] (solver1) -- (solution1);
\draw[->] (solver2) -- (solution2);
\draw[->] (solver3) -- (solution3);
\draw[->] (solver4) -- (solution4);
\end{tikzpicture}
\end{center}
\end{frame}

% \begin{frame}
  % \frametitle{Do It Now!}
  % \begin{itemize}
  % \item Download and install Minizinc
  % \item \url{https://www.minizinc.org/}
  % \end{itemize}
% \end{frame}


\begin{frame}
\frametitle{Why use Puzzles as Examples?}
\begin{itemize}
\item Easy to understand the problem
\item Solvable by hand without specialized knowledge
\item Possible to compare automated to manual solving process
\end{itemize}

\begin{quote}
The puzzle, though inanimate, is presented as a solvable problem without lasting negative consequences, a very low-risk low-reward situation. By being a puzzle, the object is attempting to convince the user that it must be completed.
\end{quote}
{\tiny Source: \href{https://www.everydayrhetoric.com/2020/03/the-rhetoric-of-puzzles/}{Every Day Rhetoric}}
\end{frame}


\mode<all>{
\part{Basic Constraint Propagation}
}
\begin{frame}
\partpage
\end{frame}
\mode<all>{
\input{../sendmore/body}
}

\mode<all>{
\part{Global Constraints}
}
\begin{frame}
\partpage
\end{frame}
\mode<all>{
\input{../sudoku/body}
}

\clearpage
\mode<all>{
\part{Customizing Search}
}
\begin{frame}
\partpage
\end{frame}
\mode<all>{
\input{../nqueen/body}
}


%% \clearpage
%% \mode<all>{
%%   \part{Problem Decomposition}
%% }
%% \begin{frame}
%%   \partpage
%% \end{frame}
%% \mode<all>{
%%   \input{../party/body}
%%   }

%% \clearpage
%% \mode<all>{
%%   \part{More Global Constraints}
%% }
%% \begin{frame}
%%   \partpage
%% \end{frame}
%% \mode<all>{
%%   \input{../car/body}
%%   }

%% \clearpage
%% \mode<all>{
%%   \part{Methodology}
%% }
%% \begin{frame}
%%   \partpage
%% \end{frame}
%% \mode<all>{
%%   \input{../methodology/body}
%%   }

\clearpage
\mode<all>{
\part{What is missing?}
}
\begin{frame}
\partpage
\end{frame}

\begin{frame}
\frametitle{Many Specialized Topics}
\begin{itemize}
\item How to design efficient core engine
\item Hybrids with LP/MIP tools
\item Hybrids with SAT
\item Symmetry breaking
\item Use of MDD/BDD to encode sets of solutions
\item High level modelling tools
\item Debugging/visualization
\end{itemize}
\end{frame}

\begin{frame}
\frametitle{Reformulation}
\begin{itemize}
\item Just because the user has modelled it this way, it doesn't mean we have to solve it that way
\begin{itemize}
\item Replace some constraint(s) by other, equivalent constraints
\item Because we don't have that constraint in our system
\item For performance
\end{itemize}
\end{itemize}
\end{frame}

\begin{frame}
\frametitle{Learning}
\begin{itemize}
\item While solving the problem we can learn how to strengthen the model/search 
\begin{itemize}
\item Understand which constraints/method contribute to propagation and change schedule
\item Learn no-good constraints by explaining failure
\item Adapt search strategy based on search experience
\end{itemize}
\end{itemize}
\end{frame}

%% \begin{frame}
%% \frametitle{Refined Process}
%% \begin{center}
%% \begin{tikzpicture}
%% \node[shape=rectangle,fill=pantone127-4] (problem) at (3,7) {Problem};
%% \node (human) at (3,6) {Human};
%% \node[shape=rectangle,fill=pantone127-4] (model) at (3,5) {User Model};
%% \node (reformulation) at (3,4) {Reformulation};
%% \node[shape=rectangle,fill=pantone127-4] (implementation) at (3,3) {Implementation Model};
%% \node (solver) at (3,2) {Constraint Solver/Search};
%% \node[shape=rectangle,fill=pantone157-8] (solution) at (3,1) {Solution};
%% \draw[-] (problem) -- (human);
%% \draw[->] (human) -- (model);
%% \draw[-] (model) -- (reformulation);
%% \draw[->] (reformulation) -- (implementation);
%% \draw[-] (implementation) -- (solver);
%% \draw[->] (solver) -- (0,2) -- (0,3) -- (implementation);
%% \draw[->] (solver) -- (solution);
%% \end{tikzpicture}
%% \end{center}
%% \end{frame}

%\begin{frame}
%\frametitle{What is CP actually used for?}
%\url{http://hsimonis.wordpress.com}
%\includegraphics[width=10cm]{../introduction/images/blog}
%\end{frame}

%% \begin{frame}
%% \frametitle{Exercises}
%% \begin{itemize}
%% \item Install Constraint Solver of your Choice
%% \item MiniZinc (MiningZinc) will be used in later sessions
%% \item Run Sendmoremoney, Sudoku, Queens examples
%% \item Can you choose which propagator to use for alldifferent?
%% \item Can you use different search strategies?
%% \end{itemize}
%% \end{frame}

\begin{frame}
\frametitle{More Learning Resources}
\begin{itemize}
\item Survey of Methods, Resources, and Formats for Teaching Constraint Programming
\begin{itemize}
\item by Tejas Santanam, Helmut Simonis
\item \url{https://doi.org/10.48550/arXiv.2403.12717}
\item Based on survey of community for WTCP 2023
\item \url{https://hsimonis.github.io/WTCP2023/}
\end{itemize}
\end{itemize}
\end{frame}


%%   }

%% \clearpage
%% \mode<all>{
%%   \part{Methodology}
%% }
%% \begin{frame}
%%   \partpage
%% \end{frame}
%% \mode<all>{
%%   \begin{frame}
\frametitle{Objectives}
\begin{itemize}
\item Overview of Core Constraint Programming
\item Three Main Concepts
\begin{itemize}
\item Constraint Propagation
\item Global Constraints
\item Customizing Search
\end{itemize}
%\item Get Some Experience with MiniZinc
\item Topics will be treated in more detail in later parts of the school
\item Based on Examples, not Formal Description
\end{itemize}
\end{frame}

\begin{frame}
\frametitle{Outline}
\begin{itemize}
\item Why Constraint Programming?
\item Constraint Propagation
\item Global Constraints
\item Customizing Search
%\item Choices in Modelling
%\item Decomposition
%\item More Global Constraints
%\item Methodology
%\item What is missing?
\end{itemize}
\end{frame}

% \begin{frame}
  % \frametitle{Using MiniZinc IDE}
  % \begin{itemize}
  % \item Developed in the Australian NICTA project
  % \item Maintained by Monash University
  % \item Modelling tool with multiple back-end solvers
  % \item Available from \url{https://www.minizinc.org/}
  % \end{itemize}
% \end{frame}


% \begin{frame}
% \frametitle{Examples in ECLiPSe}
% \begin{itemize}
% \item Open sourced constraint programming language
% \item Development goes back to 1985
% \item ECRC, ICL, IC-Parc, PTL, Cisco
% \item \url{https://eclipseclp.org/}
% \item Specialities
% \begin{itemize}
% \item Develop new solvers for specific domains
% \item Integration with MIP
% \end{itemize}
% \item Not included in bundled MiniZinc IDE
% \item Specialized visualization tools used here
  % \begin{itemize}
  % \item CP-Viz, Simonis et al. 2010
    % \end{itemize}
% \end{itemize}
% \end{frame}

\begin{frame}
\frametitle{Tutorial Based on ECLiPSe ELearning Course}
\begin{itemize}
\item Self-study course in constraint programming
\item Supported by Cisco Systems and Silicon Valley Community Foundation
\item Multi-media format, video lectures, slides, handout etc
\item {\small \url{https://eclipseclp.org/ELearning/index.html}}
\end{itemize}
\end{frame}

\begin{frame}
\frametitle{Also Part of CRT-AI Constraint Week}
\begin{itemize}
\item Annual one week course on CP and Optimization in Ireland
\item Part of national training program for PhD students in AI
\item \url{https://www.crt-ai.ie/}
\end{itemize}
\end{frame}



\begin{frame}
\frametitle{Constraint Programming - in a nutshell}
\begin{itemize}
\item Declarative description of problems with
\begin{itemize}
\item {\em Variables} which range over (finite) sets of values
\item {\em Constraints} over subsets of variables which restrict possible value combinations
\item A {\em solution} is a value assignment which satisfies all constraints
\end{itemize}

\item Constraint propagation/reasoning
\begin{itemize}
\item Removing inconsistent values for variables
\item Detect failure if constraint can not be satisfied
\item Interaction of constraints via shared variables
\item Incomplete
\end{itemize}

\item Search
\begin{itemize}
\item User controlled assignment of values to variables
\item Each step triggers constraint propagation 
\end{itemize}
\item Different domains require/allow different methods
\end{itemize}
\end{frame}

\begin{frame}
  \frametitle{Constraint Programming is Different}
  \begin{itemize}
  \item Declarative Programming
    \begin{itemize}
    \item Concentrate on what you want
      \item Not how to get there
      \item Program != Algorithm
      \item Program = Model
    \end{itemize}
    \item Applied to Combinatorial Problems
      \begin{itemize}
        \item No complete polynomial algorithms known (exist?)
        \item CP less ad-hoc than heuristics
        \item Models can evolve
  \end{itemize}
  \end{itemize}
  \end{frame}
    
\begin{frame}
  \frametitle{A Subtractive Process}
  \begin{textblock}{4}(8,-3)
    \includegraphics[width=4cm]{../introduction/images/stages}
  \end{textblock}
  \vfill
  \begin{quote}
    ``Oh, bosh, as Mr. Ruskin says. Sculpture, per se, is the simplest thing in the world. All you have to do is to take a big chunk of marble and a hammer and chisel, make up your mind what you are about to create and chip off all the marble you don't want.''-Paris Gaulois.
  \end{quote}
  
  {\tiny Source: \url{https://quoteinvestigator.com/2014/06/22/chip-away/}}
\end{frame}

\begin{frame}
\frametitle{Basic Process}
\begin{center}
\begin{tikzpicture}
\node[shape=rectangle,fill=pantone127-4] (problem) at (2,5) {Problem};
\node (human) at (2,4) {Human};
\node[shape=rectangle,fill=pantone127-4] (model) at (2,3) {Model};
\node (solver) at (2,2) {Constraint Solver/Search};
\node[shape=rectangle,fill=pantone157-8] (solution) at (2,1) {Solution};
\draw[-] (problem) -- (human);
\draw[->] (human) -- (model);
\draw[-] (model) -- (solver);
\draw[->] (solver) -- (solution);
\end{tikzpicture}
\end{center}
\end{frame}

\begin{frame}
\frametitle{More Realistic}
\begin{center}
\begin{tikzpicture}[xscale=4.0]
\node[shape=rectangle,fill=pantone127-4] (problem) at (2,5) {Problem};
\node (human) at (2,4) {Human};
\node[shape=rectangle,fill=pantone127-4] (model) at (2,3) {Model};
\node (solver) at (2,2) {Constraint Solver/Search};
\node[shape=rectangle,fill=pantone24-8] (hangs) at (1,1) {Hangs};
\node[shape=rectangle,fill=pantone157-8] (solution) at (2,1) {Solution};
\node[shape=rectangle,fill=pantone24-8] (wrong) at (3,1) {Wrong Solution};
\draw[-] (problem) -- (human);
\draw[->] (human) -- (model);
\draw[-] (model) -- (solver);
\draw[->] (solver) -- (hangs);
\draw[->] (solver) -- (solution);
\draw[->] (solver) -- (wrong);
\draw[->] (hangs) -- (1,4) -- (human);
\draw[->] (wrong) -- (3,4) -- (human);
\end{tikzpicture}
\end{center}
\end{frame}

\begin{frame}
\frametitle{Dual Role of Model}
\begin{itemize}
\item Allows Human to Express Problem
\begin{itemize}
\item Close to Problem Domain
\item Constraints as Abstractions
\end{itemize}
\item Allows Solver to Execute
\begin{itemize}
\item Variables as Communication Mechanism
\item Constraints as Algorithms
\end{itemize}
\end{itemize}
\end{frame}

\begin{frame}
\frametitle{Modelling Frameworks}
\begin{itemize}
\item \href{https://www.minizinc.org/}{MiniZinc} (NICTA, Monash University, Australia)
\item \href{https://github.com/eomahony/Numberjack}{NumberJack} (Insight, Ireland) 
\item \href{https://www-users.york.ac.uk/peter.nightingale/savilerow//}{EssencePrime/SavilleRow} (UK) 
\item \href{https://cpmpy.readthedocs.io/en/latest/}{CPMpy} (KU Leuven)
\item Allow use of multiple back-end solvers
\item Compile model into variants for each solver
\item A priori solver independent model(CP, MIP, SAT)
\end{itemize}
\end{frame} 

\begin{frame}
\frametitle{Framework Process}
\begin{center}
\begin{tikzpicture}[xscale=2]
\node[shape=rectangle,fill=pantone127-4] (problem) at (2.5,5) {Problem};
\node (human) at (2.5,4) {Human};
\node[shape=rectangle,fill=pantone127-4] (model) at (2.5,3) {Model};
\node (compile) at (2.5,2) {Compile/Reformulate};
\node (solver1) at (1,1) {CP};
\node (solver2) at (2,1) {MIP};
\node (solver3) at (3,1) {SAT};
\node (solver4) at (4,1) {Other};
\node[shape=rectangle,fill=pantone157-8] (solution1) at (1,0) {Solution};
\node[shape=rectangle,fill=pantone157-8] (solution2) at (2,0) {Solution};
\node[shape=rectangle,fill=pantone157-8] (solution3) at (3,0) {Solution};
\node[shape=rectangle,fill=pantone157-8] (solution4) at (4,0) {Solution};
\draw[-] (problem) -- (human);
\draw[->] (human) -- (model);
\draw[-] (model) -- (compile);
\draw[-] (compile) -- (solver1);
\draw[-] (compile) -- (solver2);
\draw[-] (compile) -- (solver3);
\draw[-] (compile) -- (solver4);
\draw[->] (solver1) -- (solution1);
\draw[->] (solver2) -- (solution2);
\draw[->] (solver3) -- (solution3);
\draw[->] (solver4) -- (solution4);
\end{tikzpicture}
\end{center}
\end{frame}

% \begin{frame}
  % \frametitle{Do It Now!}
  % \begin{itemize}
  % \item Download and install Minizinc
  % \item \url{https://www.minizinc.org/}
  % \end{itemize}
% \end{frame}


\begin{frame}
\frametitle{Why use Puzzles as Examples?}
\begin{itemize}
\item Easy to understand the problem
\item Solvable by hand without specialized knowledge
\item Possible to compare automated to manual solving process
\end{itemize}

\begin{quote}
The puzzle, though inanimate, is presented as a solvable problem without lasting negative consequences, a very low-risk low-reward situation. By being a puzzle, the object is attempting to convince the user that it must be completed.
\end{quote}
{\tiny Source: \href{https://www.everydayrhetoric.com/2020/03/the-rhetoric-of-puzzles/}{Every Day Rhetoric}}
\end{frame}


\mode<all>{
\part{Basic Constraint Propagation}
}
\begin{frame}
\partpage
\end{frame}
\mode<all>{
\input{../sendmore/body}
}

\mode<all>{
\part{Global Constraints}
}
\begin{frame}
\partpage
\end{frame}
\mode<all>{
\input{../sudoku/body}
}

\clearpage
\mode<all>{
\part{Customizing Search}
}
\begin{frame}
\partpage
\end{frame}
\mode<all>{
\input{../nqueen/body}
}


%% \clearpage
%% \mode<all>{
%%   \part{Problem Decomposition}
%% }
%% \begin{frame}
%%   \partpage
%% \end{frame}
%% \mode<all>{
%%   \input{../party/body}
%%   }

%% \clearpage
%% \mode<all>{
%%   \part{More Global Constraints}
%% }
%% \begin{frame}
%%   \partpage
%% \end{frame}
%% \mode<all>{
%%   \input{../car/body}
%%   }

%% \clearpage
%% \mode<all>{
%%   \part{Methodology}
%% }
%% \begin{frame}
%%   \partpage
%% \end{frame}
%% \mode<all>{
%%   \input{../methodology/body}
%%   }

\clearpage
\mode<all>{
\part{What is missing?}
}
\begin{frame}
\partpage
\end{frame}

\begin{frame}
\frametitle{Many Specialized Topics}
\begin{itemize}
\item How to design efficient core engine
\item Hybrids with LP/MIP tools
\item Hybrids with SAT
\item Symmetry breaking
\item Use of MDD/BDD to encode sets of solutions
\item High level modelling tools
\item Debugging/visualization
\end{itemize}
\end{frame}

\begin{frame}
\frametitle{Reformulation}
\begin{itemize}
\item Just because the user has modelled it this way, it doesn't mean we have to solve it that way
\begin{itemize}
\item Replace some constraint(s) by other, equivalent constraints
\item Because we don't have that constraint in our system
\item For performance
\end{itemize}
\end{itemize}
\end{frame}

\begin{frame}
\frametitle{Learning}
\begin{itemize}
\item While solving the problem we can learn how to strengthen the model/search 
\begin{itemize}
\item Understand which constraints/method contribute to propagation and change schedule
\item Learn no-good constraints by explaining failure
\item Adapt search strategy based on search experience
\end{itemize}
\end{itemize}
\end{frame}

%% \begin{frame}
%% \frametitle{Refined Process}
%% \begin{center}
%% \begin{tikzpicture}
%% \node[shape=rectangle,fill=pantone127-4] (problem) at (3,7) {Problem};
%% \node (human) at (3,6) {Human};
%% \node[shape=rectangle,fill=pantone127-4] (model) at (3,5) {User Model};
%% \node (reformulation) at (3,4) {Reformulation};
%% \node[shape=rectangle,fill=pantone127-4] (implementation) at (3,3) {Implementation Model};
%% \node (solver) at (3,2) {Constraint Solver/Search};
%% \node[shape=rectangle,fill=pantone157-8] (solution) at (3,1) {Solution};
%% \draw[-] (problem) -- (human);
%% \draw[->] (human) -- (model);
%% \draw[-] (model) -- (reformulation);
%% \draw[->] (reformulation) -- (implementation);
%% \draw[-] (implementation) -- (solver);
%% \draw[->] (solver) -- (0,2) -- (0,3) -- (implementation);
%% \draw[->] (solver) -- (solution);
%% \end{tikzpicture}
%% \end{center}
%% \end{frame}

%\begin{frame}
%\frametitle{What is CP actually used for?}
%\url{http://hsimonis.wordpress.com}
%\includegraphics[width=10cm]{../introduction/images/blog}
%\end{frame}

%% \begin{frame}
%% \frametitle{Exercises}
%% \begin{itemize}
%% \item Install Constraint Solver of your Choice
%% \item MiniZinc (MiningZinc) will be used in later sessions
%% \item Run Sendmoremoney, Sudoku, Queens examples
%% \item Can you choose which propagator to use for alldifferent?
%% \item Can you use different search strategies?
%% \end{itemize}
%% \end{frame}

\begin{frame}
\frametitle{More Learning Resources}
\begin{itemize}
\item Survey of Methods, Resources, and Formats for Teaching Constraint Programming
\begin{itemize}
\item by Tejas Santanam, Helmut Simonis
\item \url{https://doi.org/10.48550/arXiv.2403.12717}
\item Based on survey of community for WTCP 2023
\item \url{https://hsimonis.github.io/WTCP2023/}
\end{itemize}
\end{itemize}
\end{frame}


%%   }

\clearpage
\mode<all>{
\part{What is missing?}
}
\begin{frame}
\partpage
\end{frame}

\begin{frame}
\frametitle{Many Specialized Topics}
\begin{itemize}
\item How to design efficient core engine
\item Hybrids with LP/MIP tools
\item Hybrids with SAT
\item Symmetry breaking
\item Use of MDD/BDD to encode sets of solutions
\item High level modelling tools
\item Debugging/visualization
\end{itemize}
\end{frame}

\begin{frame}
\frametitle{Reformulation}
\begin{itemize}
\item Just because the user has modelled it this way, it doesn't mean we have to solve it that way
\begin{itemize}
\item Replace some constraint(s) by other, equivalent constraints
\item Because we don't have that constraint in our system
\item For performance
\end{itemize}
\end{itemize}
\end{frame}

\begin{frame}
\frametitle{Learning}
\begin{itemize}
\item While solving the problem we can learn how to strengthen the model/search 
\begin{itemize}
\item Understand which constraints/method contribute to propagation and change schedule
\item Learn no-good constraints by explaining failure
\item Adapt search strategy based on search experience
\end{itemize}
\end{itemize}
\end{frame}

%% \begin{frame}
%% \frametitle{Refined Process}
%% \begin{center}
%% \begin{tikzpicture}
%% \node[shape=rectangle,fill=pantone127-4] (problem) at (3,7) {Problem};
%% \node (human) at (3,6) {Human};
%% \node[shape=rectangle,fill=pantone127-4] (model) at (3,5) {User Model};
%% \node (reformulation) at (3,4) {Reformulation};
%% \node[shape=rectangle,fill=pantone127-4] (implementation) at (3,3) {Implementation Model};
%% \node (solver) at (3,2) {Constraint Solver/Search};
%% \node[shape=rectangle,fill=pantone157-8] (solution) at (3,1) {Solution};
%% \draw[-] (problem) -- (human);
%% \draw[->] (human) -- (model);
%% \draw[-] (model) -- (reformulation);
%% \draw[->] (reformulation) -- (implementation);
%% \draw[-] (implementation) -- (solver);
%% \draw[->] (solver) -- (0,2) -- (0,3) -- (implementation);
%% \draw[->] (solver) -- (solution);
%% \end{tikzpicture}
%% \end{center}
%% \end{frame}

%\begin{frame}
%\frametitle{What is CP actually used for?}
%\url{http://hsimonis.wordpress.com}
%\includegraphics[width=10cm]{../introduction/images/blog}
%\end{frame}

%% \begin{frame}
%% \frametitle{Exercises}
%% \begin{itemize}
%% \item Install Constraint Solver of your Choice
%% \item MiniZinc (MiningZinc) will be used in later sessions
%% \item Run Sendmoremoney, Sudoku, Queens examples
%% \item Can you choose which propagator to use for alldifferent?
%% \item Can you use different search strategies?
%% \end{itemize}
%% \end{frame}

\begin{frame}
\frametitle{More Learning Resources}
\begin{itemize}
\item Survey of Methods, Resources, and Formats for Teaching Constraint Programming
\begin{itemize}
\item by Tejas Santanam, Helmut Simonis
\item \url{https://doi.org/10.48550/arXiv.2403.12717}
\item Based on survey of community for WTCP 2023
\item \url{https://hsimonis.github.io/WTCP2023/}
\end{itemize}
\end{itemize}
\end{frame}


}


%% \clearpage
%% \mode<all>{
%%   \part{Problem Decomposition}
%% }
%% \begin{frame}
%%   \partpage
%% \end{frame}
%% \mode<all>{
%%   \begin{frame}
\frametitle{Objectives}
\begin{itemize}
\item Overview of Core Constraint Programming
\item Three Main Concepts
\begin{itemize}
\item Constraint Propagation
\item Global Constraints
\item Customizing Search
\end{itemize}
%\item Get Some Experience with MiniZinc
\item Topics will be treated in more detail in later parts of the school
\item Based on Examples, not Formal Description
\end{itemize}
\end{frame}

\begin{frame}
\frametitle{Outline}
\begin{itemize}
\item Why Constraint Programming?
\item Constraint Propagation
\item Global Constraints
\item Customizing Search
%\item Choices in Modelling
%\item Decomposition
%\item More Global Constraints
%\item Methodology
%\item What is missing?
\end{itemize}
\end{frame}

% \begin{frame}
  % \frametitle{Using MiniZinc IDE}
  % \begin{itemize}
  % \item Developed in the Australian NICTA project
  % \item Maintained by Monash University
  % \item Modelling tool with multiple back-end solvers
  % \item Available from \url{https://www.minizinc.org/}
  % \end{itemize}
% \end{frame}


% \begin{frame}
% \frametitle{Examples in ECLiPSe}
% \begin{itemize}
% \item Open sourced constraint programming language
% \item Development goes back to 1985
% \item ECRC, ICL, IC-Parc, PTL, Cisco
% \item \url{https://eclipseclp.org/}
% \item Specialities
% \begin{itemize}
% \item Develop new solvers for specific domains
% \item Integration with MIP
% \end{itemize}
% \item Not included in bundled MiniZinc IDE
% \item Specialized visualization tools used here
  % \begin{itemize}
  % \item CP-Viz, Simonis et al. 2010
    % \end{itemize}
% \end{itemize}
% \end{frame}

\begin{frame}
\frametitle{Tutorial Based on ECLiPSe ELearning Course}
\begin{itemize}
\item Self-study course in constraint programming
\item Supported by Cisco Systems and Silicon Valley Community Foundation
\item Multi-media format, video lectures, slides, handout etc
\item {\small \url{https://eclipseclp.org/ELearning/index.html}}
\end{itemize}
\end{frame}

\begin{frame}
\frametitle{Also Part of CRT-AI Constraint Week}
\begin{itemize}
\item Annual one week course on CP and Optimization in Ireland
\item Part of national training program for PhD students in AI
\item \url{https://www.crt-ai.ie/}
\end{itemize}
\end{frame}



\begin{frame}
\frametitle{Constraint Programming - in a nutshell}
\begin{itemize}
\item Declarative description of problems with
\begin{itemize}
\item {\em Variables} which range over (finite) sets of values
\item {\em Constraints} over subsets of variables which restrict possible value combinations
\item A {\em solution} is a value assignment which satisfies all constraints
\end{itemize}

\item Constraint propagation/reasoning
\begin{itemize}
\item Removing inconsistent values for variables
\item Detect failure if constraint can not be satisfied
\item Interaction of constraints via shared variables
\item Incomplete
\end{itemize}

\item Search
\begin{itemize}
\item User controlled assignment of values to variables
\item Each step triggers constraint propagation 
\end{itemize}
\item Different domains require/allow different methods
\end{itemize}
\end{frame}

\begin{frame}
  \frametitle{Constraint Programming is Different}
  \begin{itemize}
  \item Declarative Programming
    \begin{itemize}
    \item Concentrate on what you want
      \item Not how to get there
      \item Program != Algorithm
      \item Program = Model
    \end{itemize}
    \item Applied to Combinatorial Problems
      \begin{itemize}
        \item No complete polynomial algorithms known (exist?)
        \item CP less ad-hoc than heuristics
        \item Models can evolve
  \end{itemize}
  \end{itemize}
  \end{frame}
    
\begin{frame}
  \frametitle{A Subtractive Process}
  \begin{textblock}{4}(8,-3)
    \includegraphics[width=4cm]{../introduction/images/stages}
  \end{textblock}
  \vfill
  \begin{quote}
    ``Oh, bosh, as Mr. Ruskin says. Sculpture, per se, is the simplest thing in the world. All you have to do is to take a big chunk of marble and a hammer and chisel, make up your mind what you are about to create and chip off all the marble you don't want.''-Paris Gaulois.
  \end{quote}
  
  {\tiny Source: \url{https://quoteinvestigator.com/2014/06/22/chip-away/}}
\end{frame}

\begin{frame}
\frametitle{Basic Process}
\begin{center}
\begin{tikzpicture}
\node[shape=rectangle,fill=pantone127-4] (problem) at (2,5) {Problem};
\node (human) at (2,4) {Human};
\node[shape=rectangle,fill=pantone127-4] (model) at (2,3) {Model};
\node (solver) at (2,2) {Constraint Solver/Search};
\node[shape=rectangle,fill=pantone157-8] (solution) at (2,1) {Solution};
\draw[-] (problem) -- (human);
\draw[->] (human) -- (model);
\draw[-] (model) -- (solver);
\draw[->] (solver) -- (solution);
\end{tikzpicture}
\end{center}
\end{frame}

\begin{frame}
\frametitle{More Realistic}
\begin{center}
\begin{tikzpicture}[xscale=4.0]
\node[shape=rectangle,fill=pantone127-4] (problem) at (2,5) {Problem};
\node (human) at (2,4) {Human};
\node[shape=rectangle,fill=pantone127-4] (model) at (2,3) {Model};
\node (solver) at (2,2) {Constraint Solver/Search};
\node[shape=rectangle,fill=pantone24-8] (hangs) at (1,1) {Hangs};
\node[shape=rectangle,fill=pantone157-8] (solution) at (2,1) {Solution};
\node[shape=rectangle,fill=pantone24-8] (wrong) at (3,1) {Wrong Solution};
\draw[-] (problem) -- (human);
\draw[->] (human) -- (model);
\draw[-] (model) -- (solver);
\draw[->] (solver) -- (hangs);
\draw[->] (solver) -- (solution);
\draw[->] (solver) -- (wrong);
\draw[->] (hangs) -- (1,4) -- (human);
\draw[->] (wrong) -- (3,4) -- (human);
\end{tikzpicture}
\end{center}
\end{frame}

\begin{frame}
\frametitle{Dual Role of Model}
\begin{itemize}
\item Allows Human to Express Problem
\begin{itemize}
\item Close to Problem Domain
\item Constraints as Abstractions
\end{itemize}
\item Allows Solver to Execute
\begin{itemize}
\item Variables as Communication Mechanism
\item Constraints as Algorithms
\end{itemize}
\end{itemize}
\end{frame}

\begin{frame}
\frametitle{Modelling Frameworks}
\begin{itemize}
\item \href{https://www.minizinc.org/}{MiniZinc} (NICTA, Monash University, Australia)
\item \href{https://github.com/eomahony/Numberjack}{NumberJack} (Insight, Ireland) 
\item \href{https://www-users.york.ac.uk/peter.nightingale/savilerow//}{EssencePrime/SavilleRow} (UK) 
\item \href{https://cpmpy.readthedocs.io/en/latest/}{CPMpy} (KU Leuven)
\item Allow use of multiple back-end solvers
\item Compile model into variants for each solver
\item A priori solver independent model(CP, MIP, SAT)
\end{itemize}
\end{frame} 

\begin{frame}
\frametitle{Framework Process}
\begin{center}
\begin{tikzpicture}[xscale=2]
\node[shape=rectangle,fill=pantone127-4] (problem) at (2.5,5) {Problem};
\node (human) at (2.5,4) {Human};
\node[shape=rectangle,fill=pantone127-4] (model) at (2.5,3) {Model};
\node (compile) at (2.5,2) {Compile/Reformulate};
\node (solver1) at (1,1) {CP};
\node (solver2) at (2,1) {MIP};
\node (solver3) at (3,1) {SAT};
\node (solver4) at (4,1) {Other};
\node[shape=rectangle,fill=pantone157-8] (solution1) at (1,0) {Solution};
\node[shape=rectangle,fill=pantone157-8] (solution2) at (2,0) {Solution};
\node[shape=rectangle,fill=pantone157-8] (solution3) at (3,0) {Solution};
\node[shape=rectangle,fill=pantone157-8] (solution4) at (4,0) {Solution};
\draw[-] (problem) -- (human);
\draw[->] (human) -- (model);
\draw[-] (model) -- (compile);
\draw[-] (compile) -- (solver1);
\draw[-] (compile) -- (solver2);
\draw[-] (compile) -- (solver3);
\draw[-] (compile) -- (solver4);
\draw[->] (solver1) -- (solution1);
\draw[->] (solver2) -- (solution2);
\draw[->] (solver3) -- (solution3);
\draw[->] (solver4) -- (solution4);
\end{tikzpicture}
\end{center}
\end{frame}

% \begin{frame}
  % \frametitle{Do It Now!}
  % \begin{itemize}
  % \item Download and install Minizinc
  % \item \url{https://www.minizinc.org/}
  % \end{itemize}
% \end{frame}


\begin{frame}
\frametitle{Why use Puzzles as Examples?}
\begin{itemize}
\item Easy to understand the problem
\item Solvable by hand without specialized knowledge
\item Possible to compare automated to manual solving process
\end{itemize}

\begin{quote}
The puzzle, though inanimate, is presented as a solvable problem without lasting negative consequences, a very low-risk low-reward situation. By being a puzzle, the object is attempting to convince the user that it must be completed.
\end{quote}
{\tiny Source: \href{https://www.everydayrhetoric.com/2020/03/the-rhetoric-of-puzzles/}{Every Day Rhetoric}}
\end{frame}


\mode<all>{
\part{Basic Constraint Propagation}
}
\begin{frame}
\partpage
\end{frame}
\mode<all>{
\begin{frame}
\frametitle{Objectives}
\begin{itemize}
\item Overview of Core Constraint Programming
\item Three Main Concepts
\begin{itemize}
\item Constraint Propagation
\item Global Constraints
\item Customizing Search
\end{itemize}
%\item Get Some Experience with MiniZinc
\item Topics will be treated in more detail in later parts of the school
\item Based on Examples, not Formal Description
\end{itemize}
\end{frame}

\begin{frame}
\frametitle{Outline}
\begin{itemize}
\item Why Constraint Programming?
\item Constraint Propagation
\item Global Constraints
\item Customizing Search
%\item Choices in Modelling
%\item Decomposition
%\item More Global Constraints
%\item Methodology
%\item What is missing?
\end{itemize}
\end{frame}

% \begin{frame}
  % \frametitle{Using MiniZinc IDE}
  % \begin{itemize}
  % \item Developed in the Australian NICTA project
  % \item Maintained by Monash University
  % \item Modelling tool with multiple back-end solvers
  % \item Available from \url{https://www.minizinc.org/}
  % \end{itemize}
% \end{frame}


% \begin{frame}
% \frametitle{Examples in ECLiPSe}
% \begin{itemize}
% \item Open sourced constraint programming language
% \item Development goes back to 1985
% \item ECRC, ICL, IC-Parc, PTL, Cisco
% \item \url{https://eclipseclp.org/}
% \item Specialities
% \begin{itemize}
% \item Develop new solvers for specific domains
% \item Integration with MIP
% \end{itemize}
% \item Not included in bundled MiniZinc IDE
% \item Specialized visualization tools used here
  % \begin{itemize}
  % \item CP-Viz, Simonis et al. 2010
    % \end{itemize}
% \end{itemize}
% \end{frame}

\begin{frame}
\frametitle{Tutorial Based on ECLiPSe ELearning Course}
\begin{itemize}
\item Self-study course in constraint programming
\item Supported by Cisco Systems and Silicon Valley Community Foundation
\item Multi-media format, video lectures, slides, handout etc
\item {\small \url{https://eclipseclp.org/ELearning/index.html}}
\end{itemize}
\end{frame}

\begin{frame}
\frametitle{Also Part of CRT-AI Constraint Week}
\begin{itemize}
\item Annual one week course on CP and Optimization in Ireland
\item Part of national training program for PhD students in AI
\item \url{https://www.crt-ai.ie/}
\end{itemize}
\end{frame}



\begin{frame}
\frametitle{Constraint Programming - in a nutshell}
\begin{itemize}
\item Declarative description of problems with
\begin{itemize}
\item {\em Variables} which range over (finite) sets of values
\item {\em Constraints} over subsets of variables which restrict possible value combinations
\item A {\em solution} is a value assignment which satisfies all constraints
\end{itemize}

\item Constraint propagation/reasoning
\begin{itemize}
\item Removing inconsistent values for variables
\item Detect failure if constraint can not be satisfied
\item Interaction of constraints via shared variables
\item Incomplete
\end{itemize}

\item Search
\begin{itemize}
\item User controlled assignment of values to variables
\item Each step triggers constraint propagation 
\end{itemize}
\item Different domains require/allow different methods
\end{itemize}
\end{frame}

\begin{frame}
  \frametitle{Constraint Programming is Different}
  \begin{itemize}
  \item Declarative Programming
    \begin{itemize}
    \item Concentrate on what you want
      \item Not how to get there
      \item Program != Algorithm
      \item Program = Model
    \end{itemize}
    \item Applied to Combinatorial Problems
      \begin{itemize}
        \item No complete polynomial algorithms known (exist?)
        \item CP less ad-hoc than heuristics
        \item Models can evolve
  \end{itemize}
  \end{itemize}
  \end{frame}
    
\begin{frame}
  \frametitle{A Subtractive Process}
  \begin{textblock}{4}(8,-3)
    \includegraphics[width=4cm]{../introduction/images/stages}
  \end{textblock}
  \vfill
  \begin{quote}
    ``Oh, bosh, as Mr. Ruskin says. Sculpture, per se, is the simplest thing in the world. All you have to do is to take a big chunk of marble and a hammer and chisel, make up your mind what you are about to create and chip off all the marble you don't want.''-Paris Gaulois.
  \end{quote}
  
  {\tiny Source: \url{https://quoteinvestigator.com/2014/06/22/chip-away/}}
\end{frame}

\begin{frame}
\frametitle{Basic Process}
\begin{center}
\begin{tikzpicture}
\node[shape=rectangle,fill=pantone127-4] (problem) at (2,5) {Problem};
\node (human) at (2,4) {Human};
\node[shape=rectangle,fill=pantone127-4] (model) at (2,3) {Model};
\node (solver) at (2,2) {Constraint Solver/Search};
\node[shape=rectangle,fill=pantone157-8] (solution) at (2,1) {Solution};
\draw[-] (problem) -- (human);
\draw[->] (human) -- (model);
\draw[-] (model) -- (solver);
\draw[->] (solver) -- (solution);
\end{tikzpicture}
\end{center}
\end{frame}

\begin{frame}
\frametitle{More Realistic}
\begin{center}
\begin{tikzpicture}[xscale=4.0]
\node[shape=rectangle,fill=pantone127-4] (problem) at (2,5) {Problem};
\node (human) at (2,4) {Human};
\node[shape=rectangle,fill=pantone127-4] (model) at (2,3) {Model};
\node (solver) at (2,2) {Constraint Solver/Search};
\node[shape=rectangle,fill=pantone24-8] (hangs) at (1,1) {Hangs};
\node[shape=rectangle,fill=pantone157-8] (solution) at (2,1) {Solution};
\node[shape=rectangle,fill=pantone24-8] (wrong) at (3,1) {Wrong Solution};
\draw[-] (problem) -- (human);
\draw[->] (human) -- (model);
\draw[-] (model) -- (solver);
\draw[->] (solver) -- (hangs);
\draw[->] (solver) -- (solution);
\draw[->] (solver) -- (wrong);
\draw[->] (hangs) -- (1,4) -- (human);
\draw[->] (wrong) -- (3,4) -- (human);
\end{tikzpicture}
\end{center}
\end{frame}

\begin{frame}
\frametitle{Dual Role of Model}
\begin{itemize}
\item Allows Human to Express Problem
\begin{itemize}
\item Close to Problem Domain
\item Constraints as Abstractions
\end{itemize}
\item Allows Solver to Execute
\begin{itemize}
\item Variables as Communication Mechanism
\item Constraints as Algorithms
\end{itemize}
\end{itemize}
\end{frame}

\begin{frame}
\frametitle{Modelling Frameworks}
\begin{itemize}
\item \href{https://www.minizinc.org/}{MiniZinc} (NICTA, Monash University, Australia)
\item \href{https://github.com/eomahony/Numberjack}{NumberJack} (Insight, Ireland) 
\item \href{https://www-users.york.ac.uk/peter.nightingale/savilerow//}{EssencePrime/SavilleRow} (UK) 
\item \href{https://cpmpy.readthedocs.io/en/latest/}{CPMpy} (KU Leuven)
\item Allow use of multiple back-end solvers
\item Compile model into variants for each solver
\item A priori solver independent model(CP, MIP, SAT)
\end{itemize}
\end{frame} 

\begin{frame}
\frametitle{Framework Process}
\begin{center}
\begin{tikzpicture}[xscale=2]
\node[shape=rectangle,fill=pantone127-4] (problem) at (2.5,5) {Problem};
\node (human) at (2.5,4) {Human};
\node[shape=rectangle,fill=pantone127-4] (model) at (2.5,3) {Model};
\node (compile) at (2.5,2) {Compile/Reformulate};
\node (solver1) at (1,1) {CP};
\node (solver2) at (2,1) {MIP};
\node (solver3) at (3,1) {SAT};
\node (solver4) at (4,1) {Other};
\node[shape=rectangle,fill=pantone157-8] (solution1) at (1,0) {Solution};
\node[shape=rectangle,fill=pantone157-8] (solution2) at (2,0) {Solution};
\node[shape=rectangle,fill=pantone157-8] (solution3) at (3,0) {Solution};
\node[shape=rectangle,fill=pantone157-8] (solution4) at (4,0) {Solution};
\draw[-] (problem) -- (human);
\draw[->] (human) -- (model);
\draw[-] (model) -- (compile);
\draw[-] (compile) -- (solver1);
\draw[-] (compile) -- (solver2);
\draw[-] (compile) -- (solver3);
\draw[-] (compile) -- (solver4);
\draw[->] (solver1) -- (solution1);
\draw[->] (solver2) -- (solution2);
\draw[->] (solver3) -- (solution3);
\draw[->] (solver4) -- (solution4);
\end{tikzpicture}
\end{center}
\end{frame}

% \begin{frame}
  % \frametitle{Do It Now!}
  % \begin{itemize}
  % \item Download and install Minizinc
  % \item \url{https://www.minizinc.org/}
  % \end{itemize}
% \end{frame}


\begin{frame}
\frametitle{Why use Puzzles as Examples?}
\begin{itemize}
\item Easy to understand the problem
\item Solvable by hand without specialized knowledge
\item Possible to compare automated to manual solving process
\end{itemize}

\begin{quote}
The puzzle, though inanimate, is presented as a solvable problem without lasting negative consequences, a very low-risk low-reward situation. By being a puzzle, the object is attempting to convince the user that it must be completed.
\end{quote}
{\tiny Source: \href{https://www.everydayrhetoric.com/2020/03/the-rhetoric-of-puzzles/}{Every Day Rhetoric}}
\end{frame}


\mode<all>{
\part{Basic Constraint Propagation}
}
\begin{frame}
\partpage
\end{frame}
\mode<all>{
\input{../sendmore/body}
}

\mode<all>{
\part{Global Constraints}
}
\begin{frame}
\partpage
\end{frame}
\mode<all>{
\input{../sudoku/body}
}

\clearpage
\mode<all>{
\part{Customizing Search}
}
\begin{frame}
\partpage
\end{frame}
\mode<all>{
\input{../nqueen/body}
}


%% \clearpage
%% \mode<all>{
%%   \part{Problem Decomposition}
%% }
%% \begin{frame}
%%   \partpage
%% \end{frame}
%% \mode<all>{
%%   \input{../party/body}
%%   }

%% \clearpage
%% \mode<all>{
%%   \part{More Global Constraints}
%% }
%% \begin{frame}
%%   \partpage
%% \end{frame}
%% \mode<all>{
%%   \input{../car/body}
%%   }

%% \clearpage
%% \mode<all>{
%%   \part{Methodology}
%% }
%% \begin{frame}
%%   \partpage
%% \end{frame}
%% \mode<all>{
%%   \input{../methodology/body}
%%   }

\clearpage
\mode<all>{
\part{What is missing?}
}
\begin{frame}
\partpage
\end{frame}

\begin{frame}
\frametitle{Many Specialized Topics}
\begin{itemize}
\item How to design efficient core engine
\item Hybrids with LP/MIP tools
\item Hybrids with SAT
\item Symmetry breaking
\item Use of MDD/BDD to encode sets of solutions
\item High level modelling tools
\item Debugging/visualization
\end{itemize}
\end{frame}

\begin{frame}
\frametitle{Reformulation}
\begin{itemize}
\item Just because the user has modelled it this way, it doesn't mean we have to solve it that way
\begin{itemize}
\item Replace some constraint(s) by other, equivalent constraints
\item Because we don't have that constraint in our system
\item For performance
\end{itemize}
\end{itemize}
\end{frame}

\begin{frame}
\frametitle{Learning}
\begin{itemize}
\item While solving the problem we can learn how to strengthen the model/search 
\begin{itemize}
\item Understand which constraints/method contribute to propagation and change schedule
\item Learn no-good constraints by explaining failure
\item Adapt search strategy based on search experience
\end{itemize}
\end{itemize}
\end{frame}

%% \begin{frame}
%% \frametitle{Refined Process}
%% \begin{center}
%% \begin{tikzpicture}
%% \node[shape=rectangle,fill=pantone127-4] (problem) at (3,7) {Problem};
%% \node (human) at (3,6) {Human};
%% \node[shape=rectangle,fill=pantone127-4] (model) at (3,5) {User Model};
%% \node (reformulation) at (3,4) {Reformulation};
%% \node[shape=rectangle,fill=pantone127-4] (implementation) at (3,3) {Implementation Model};
%% \node (solver) at (3,2) {Constraint Solver/Search};
%% \node[shape=rectangle,fill=pantone157-8] (solution) at (3,1) {Solution};
%% \draw[-] (problem) -- (human);
%% \draw[->] (human) -- (model);
%% \draw[-] (model) -- (reformulation);
%% \draw[->] (reformulation) -- (implementation);
%% \draw[-] (implementation) -- (solver);
%% \draw[->] (solver) -- (0,2) -- (0,3) -- (implementation);
%% \draw[->] (solver) -- (solution);
%% \end{tikzpicture}
%% \end{center}
%% \end{frame}

%\begin{frame}
%\frametitle{What is CP actually used for?}
%\url{http://hsimonis.wordpress.com}
%\includegraphics[width=10cm]{../introduction/images/blog}
%\end{frame}

%% \begin{frame}
%% \frametitle{Exercises}
%% \begin{itemize}
%% \item Install Constraint Solver of your Choice
%% \item MiniZinc (MiningZinc) will be used in later sessions
%% \item Run Sendmoremoney, Sudoku, Queens examples
%% \item Can you choose which propagator to use for alldifferent?
%% \item Can you use different search strategies?
%% \end{itemize}
%% \end{frame}

\begin{frame}
\frametitle{More Learning Resources}
\begin{itemize}
\item Survey of Methods, Resources, and Formats for Teaching Constraint Programming
\begin{itemize}
\item by Tejas Santanam, Helmut Simonis
\item \url{https://doi.org/10.48550/arXiv.2403.12717}
\item Based on survey of community for WTCP 2023
\item \url{https://hsimonis.github.io/WTCP2023/}
\end{itemize}
\end{itemize}
\end{frame}


}

\mode<all>{
\part{Global Constraints}
}
\begin{frame}
\partpage
\end{frame}
\mode<all>{
\begin{frame}
\frametitle{Objectives}
\begin{itemize}
\item Overview of Core Constraint Programming
\item Three Main Concepts
\begin{itemize}
\item Constraint Propagation
\item Global Constraints
\item Customizing Search
\end{itemize}
%\item Get Some Experience with MiniZinc
\item Topics will be treated in more detail in later parts of the school
\item Based on Examples, not Formal Description
\end{itemize}
\end{frame}

\begin{frame}
\frametitle{Outline}
\begin{itemize}
\item Why Constraint Programming?
\item Constraint Propagation
\item Global Constraints
\item Customizing Search
%\item Choices in Modelling
%\item Decomposition
%\item More Global Constraints
%\item Methodology
%\item What is missing?
\end{itemize}
\end{frame}

% \begin{frame}
  % \frametitle{Using MiniZinc IDE}
  % \begin{itemize}
  % \item Developed in the Australian NICTA project
  % \item Maintained by Monash University
  % \item Modelling tool with multiple back-end solvers
  % \item Available from \url{https://www.minizinc.org/}
  % \end{itemize}
% \end{frame}


% \begin{frame}
% \frametitle{Examples in ECLiPSe}
% \begin{itemize}
% \item Open sourced constraint programming language
% \item Development goes back to 1985
% \item ECRC, ICL, IC-Parc, PTL, Cisco
% \item \url{https://eclipseclp.org/}
% \item Specialities
% \begin{itemize}
% \item Develop new solvers for specific domains
% \item Integration with MIP
% \end{itemize}
% \item Not included in bundled MiniZinc IDE
% \item Specialized visualization tools used here
  % \begin{itemize}
  % \item CP-Viz, Simonis et al. 2010
    % \end{itemize}
% \end{itemize}
% \end{frame}

\begin{frame}
\frametitle{Tutorial Based on ECLiPSe ELearning Course}
\begin{itemize}
\item Self-study course in constraint programming
\item Supported by Cisco Systems and Silicon Valley Community Foundation
\item Multi-media format, video lectures, slides, handout etc
\item {\small \url{https://eclipseclp.org/ELearning/index.html}}
\end{itemize}
\end{frame}

\begin{frame}
\frametitle{Also Part of CRT-AI Constraint Week}
\begin{itemize}
\item Annual one week course on CP and Optimization in Ireland
\item Part of national training program for PhD students in AI
\item \url{https://www.crt-ai.ie/}
\end{itemize}
\end{frame}



\begin{frame}
\frametitle{Constraint Programming - in a nutshell}
\begin{itemize}
\item Declarative description of problems with
\begin{itemize}
\item {\em Variables} which range over (finite) sets of values
\item {\em Constraints} over subsets of variables which restrict possible value combinations
\item A {\em solution} is a value assignment which satisfies all constraints
\end{itemize}

\item Constraint propagation/reasoning
\begin{itemize}
\item Removing inconsistent values for variables
\item Detect failure if constraint can not be satisfied
\item Interaction of constraints via shared variables
\item Incomplete
\end{itemize}

\item Search
\begin{itemize}
\item User controlled assignment of values to variables
\item Each step triggers constraint propagation 
\end{itemize}
\item Different domains require/allow different methods
\end{itemize}
\end{frame}

\begin{frame}
  \frametitle{Constraint Programming is Different}
  \begin{itemize}
  \item Declarative Programming
    \begin{itemize}
    \item Concentrate on what you want
      \item Not how to get there
      \item Program != Algorithm
      \item Program = Model
    \end{itemize}
    \item Applied to Combinatorial Problems
      \begin{itemize}
        \item No complete polynomial algorithms known (exist?)
        \item CP less ad-hoc than heuristics
        \item Models can evolve
  \end{itemize}
  \end{itemize}
  \end{frame}
    
\begin{frame}
  \frametitle{A Subtractive Process}
  \begin{textblock}{4}(8,-3)
    \includegraphics[width=4cm]{../introduction/images/stages}
  \end{textblock}
  \vfill
  \begin{quote}
    ``Oh, bosh, as Mr. Ruskin says. Sculpture, per se, is the simplest thing in the world. All you have to do is to take a big chunk of marble and a hammer and chisel, make up your mind what you are about to create and chip off all the marble you don't want.''-Paris Gaulois.
  \end{quote}
  
  {\tiny Source: \url{https://quoteinvestigator.com/2014/06/22/chip-away/}}
\end{frame}

\begin{frame}
\frametitle{Basic Process}
\begin{center}
\begin{tikzpicture}
\node[shape=rectangle,fill=pantone127-4] (problem) at (2,5) {Problem};
\node (human) at (2,4) {Human};
\node[shape=rectangle,fill=pantone127-4] (model) at (2,3) {Model};
\node (solver) at (2,2) {Constraint Solver/Search};
\node[shape=rectangle,fill=pantone157-8] (solution) at (2,1) {Solution};
\draw[-] (problem) -- (human);
\draw[->] (human) -- (model);
\draw[-] (model) -- (solver);
\draw[->] (solver) -- (solution);
\end{tikzpicture}
\end{center}
\end{frame}

\begin{frame}
\frametitle{More Realistic}
\begin{center}
\begin{tikzpicture}[xscale=4.0]
\node[shape=rectangle,fill=pantone127-4] (problem) at (2,5) {Problem};
\node (human) at (2,4) {Human};
\node[shape=rectangle,fill=pantone127-4] (model) at (2,3) {Model};
\node (solver) at (2,2) {Constraint Solver/Search};
\node[shape=rectangle,fill=pantone24-8] (hangs) at (1,1) {Hangs};
\node[shape=rectangle,fill=pantone157-8] (solution) at (2,1) {Solution};
\node[shape=rectangle,fill=pantone24-8] (wrong) at (3,1) {Wrong Solution};
\draw[-] (problem) -- (human);
\draw[->] (human) -- (model);
\draw[-] (model) -- (solver);
\draw[->] (solver) -- (hangs);
\draw[->] (solver) -- (solution);
\draw[->] (solver) -- (wrong);
\draw[->] (hangs) -- (1,4) -- (human);
\draw[->] (wrong) -- (3,4) -- (human);
\end{tikzpicture}
\end{center}
\end{frame}

\begin{frame}
\frametitle{Dual Role of Model}
\begin{itemize}
\item Allows Human to Express Problem
\begin{itemize}
\item Close to Problem Domain
\item Constraints as Abstractions
\end{itemize}
\item Allows Solver to Execute
\begin{itemize}
\item Variables as Communication Mechanism
\item Constraints as Algorithms
\end{itemize}
\end{itemize}
\end{frame}

\begin{frame}
\frametitle{Modelling Frameworks}
\begin{itemize}
\item \href{https://www.minizinc.org/}{MiniZinc} (NICTA, Monash University, Australia)
\item \href{https://github.com/eomahony/Numberjack}{NumberJack} (Insight, Ireland) 
\item \href{https://www-users.york.ac.uk/peter.nightingale/savilerow//}{EssencePrime/SavilleRow} (UK) 
\item \href{https://cpmpy.readthedocs.io/en/latest/}{CPMpy} (KU Leuven)
\item Allow use of multiple back-end solvers
\item Compile model into variants for each solver
\item A priori solver independent model(CP, MIP, SAT)
\end{itemize}
\end{frame} 

\begin{frame}
\frametitle{Framework Process}
\begin{center}
\begin{tikzpicture}[xscale=2]
\node[shape=rectangle,fill=pantone127-4] (problem) at (2.5,5) {Problem};
\node (human) at (2.5,4) {Human};
\node[shape=rectangle,fill=pantone127-4] (model) at (2.5,3) {Model};
\node (compile) at (2.5,2) {Compile/Reformulate};
\node (solver1) at (1,1) {CP};
\node (solver2) at (2,1) {MIP};
\node (solver3) at (3,1) {SAT};
\node (solver4) at (4,1) {Other};
\node[shape=rectangle,fill=pantone157-8] (solution1) at (1,0) {Solution};
\node[shape=rectangle,fill=pantone157-8] (solution2) at (2,0) {Solution};
\node[shape=rectangle,fill=pantone157-8] (solution3) at (3,0) {Solution};
\node[shape=rectangle,fill=pantone157-8] (solution4) at (4,0) {Solution};
\draw[-] (problem) -- (human);
\draw[->] (human) -- (model);
\draw[-] (model) -- (compile);
\draw[-] (compile) -- (solver1);
\draw[-] (compile) -- (solver2);
\draw[-] (compile) -- (solver3);
\draw[-] (compile) -- (solver4);
\draw[->] (solver1) -- (solution1);
\draw[->] (solver2) -- (solution2);
\draw[->] (solver3) -- (solution3);
\draw[->] (solver4) -- (solution4);
\end{tikzpicture}
\end{center}
\end{frame}

% \begin{frame}
  % \frametitle{Do It Now!}
  % \begin{itemize}
  % \item Download and install Minizinc
  % \item \url{https://www.minizinc.org/}
  % \end{itemize}
% \end{frame}


\begin{frame}
\frametitle{Why use Puzzles as Examples?}
\begin{itemize}
\item Easy to understand the problem
\item Solvable by hand without specialized knowledge
\item Possible to compare automated to manual solving process
\end{itemize}

\begin{quote}
The puzzle, though inanimate, is presented as a solvable problem without lasting negative consequences, a very low-risk low-reward situation. By being a puzzle, the object is attempting to convince the user that it must be completed.
\end{quote}
{\tiny Source: \href{https://www.everydayrhetoric.com/2020/03/the-rhetoric-of-puzzles/}{Every Day Rhetoric}}
\end{frame}


\mode<all>{
\part{Basic Constraint Propagation}
}
\begin{frame}
\partpage
\end{frame}
\mode<all>{
\input{../sendmore/body}
}

\mode<all>{
\part{Global Constraints}
}
\begin{frame}
\partpage
\end{frame}
\mode<all>{
\input{../sudoku/body}
}

\clearpage
\mode<all>{
\part{Customizing Search}
}
\begin{frame}
\partpage
\end{frame}
\mode<all>{
\input{../nqueen/body}
}


%% \clearpage
%% \mode<all>{
%%   \part{Problem Decomposition}
%% }
%% \begin{frame}
%%   \partpage
%% \end{frame}
%% \mode<all>{
%%   \input{../party/body}
%%   }

%% \clearpage
%% \mode<all>{
%%   \part{More Global Constraints}
%% }
%% \begin{frame}
%%   \partpage
%% \end{frame}
%% \mode<all>{
%%   \input{../car/body}
%%   }

%% \clearpage
%% \mode<all>{
%%   \part{Methodology}
%% }
%% \begin{frame}
%%   \partpage
%% \end{frame}
%% \mode<all>{
%%   \input{../methodology/body}
%%   }

\clearpage
\mode<all>{
\part{What is missing?}
}
\begin{frame}
\partpage
\end{frame}

\begin{frame}
\frametitle{Many Specialized Topics}
\begin{itemize}
\item How to design efficient core engine
\item Hybrids with LP/MIP tools
\item Hybrids with SAT
\item Symmetry breaking
\item Use of MDD/BDD to encode sets of solutions
\item High level modelling tools
\item Debugging/visualization
\end{itemize}
\end{frame}

\begin{frame}
\frametitle{Reformulation}
\begin{itemize}
\item Just because the user has modelled it this way, it doesn't mean we have to solve it that way
\begin{itemize}
\item Replace some constraint(s) by other, equivalent constraints
\item Because we don't have that constraint in our system
\item For performance
\end{itemize}
\end{itemize}
\end{frame}

\begin{frame}
\frametitle{Learning}
\begin{itemize}
\item While solving the problem we can learn how to strengthen the model/search 
\begin{itemize}
\item Understand which constraints/method contribute to propagation and change schedule
\item Learn no-good constraints by explaining failure
\item Adapt search strategy based on search experience
\end{itemize}
\end{itemize}
\end{frame}

%% \begin{frame}
%% \frametitle{Refined Process}
%% \begin{center}
%% \begin{tikzpicture}
%% \node[shape=rectangle,fill=pantone127-4] (problem) at (3,7) {Problem};
%% \node (human) at (3,6) {Human};
%% \node[shape=rectangle,fill=pantone127-4] (model) at (3,5) {User Model};
%% \node (reformulation) at (3,4) {Reformulation};
%% \node[shape=rectangle,fill=pantone127-4] (implementation) at (3,3) {Implementation Model};
%% \node (solver) at (3,2) {Constraint Solver/Search};
%% \node[shape=rectangle,fill=pantone157-8] (solution) at (3,1) {Solution};
%% \draw[-] (problem) -- (human);
%% \draw[->] (human) -- (model);
%% \draw[-] (model) -- (reformulation);
%% \draw[->] (reformulation) -- (implementation);
%% \draw[-] (implementation) -- (solver);
%% \draw[->] (solver) -- (0,2) -- (0,3) -- (implementation);
%% \draw[->] (solver) -- (solution);
%% \end{tikzpicture}
%% \end{center}
%% \end{frame}

%\begin{frame}
%\frametitle{What is CP actually used for?}
%\url{http://hsimonis.wordpress.com}
%\includegraphics[width=10cm]{../introduction/images/blog}
%\end{frame}

%% \begin{frame}
%% \frametitle{Exercises}
%% \begin{itemize}
%% \item Install Constraint Solver of your Choice
%% \item MiniZinc (MiningZinc) will be used in later sessions
%% \item Run Sendmoremoney, Sudoku, Queens examples
%% \item Can you choose which propagator to use for alldifferent?
%% \item Can you use different search strategies?
%% \end{itemize}
%% \end{frame}

\begin{frame}
\frametitle{More Learning Resources}
\begin{itemize}
\item Survey of Methods, Resources, and Formats for Teaching Constraint Programming
\begin{itemize}
\item by Tejas Santanam, Helmut Simonis
\item \url{https://doi.org/10.48550/arXiv.2403.12717}
\item Based on survey of community for WTCP 2023
\item \url{https://hsimonis.github.io/WTCP2023/}
\end{itemize}
\end{itemize}
\end{frame}


}

\clearpage
\mode<all>{
\part{Customizing Search}
}
\begin{frame}
\partpage
\end{frame}
\mode<all>{
\begin{frame}
\frametitle{Objectives}
\begin{itemize}
\item Overview of Core Constraint Programming
\item Three Main Concepts
\begin{itemize}
\item Constraint Propagation
\item Global Constraints
\item Customizing Search
\end{itemize}
%\item Get Some Experience with MiniZinc
\item Topics will be treated in more detail in later parts of the school
\item Based on Examples, not Formal Description
\end{itemize}
\end{frame}

\begin{frame}
\frametitle{Outline}
\begin{itemize}
\item Why Constraint Programming?
\item Constraint Propagation
\item Global Constraints
\item Customizing Search
%\item Choices in Modelling
%\item Decomposition
%\item More Global Constraints
%\item Methodology
%\item What is missing?
\end{itemize}
\end{frame}

% \begin{frame}
  % \frametitle{Using MiniZinc IDE}
  % \begin{itemize}
  % \item Developed in the Australian NICTA project
  % \item Maintained by Monash University
  % \item Modelling tool with multiple back-end solvers
  % \item Available from \url{https://www.minizinc.org/}
  % \end{itemize}
% \end{frame}


% \begin{frame}
% \frametitle{Examples in ECLiPSe}
% \begin{itemize}
% \item Open sourced constraint programming language
% \item Development goes back to 1985
% \item ECRC, ICL, IC-Parc, PTL, Cisco
% \item \url{https://eclipseclp.org/}
% \item Specialities
% \begin{itemize}
% \item Develop new solvers for specific domains
% \item Integration with MIP
% \end{itemize}
% \item Not included in bundled MiniZinc IDE
% \item Specialized visualization tools used here
  % \begin{itemize}
  % \item CP-Viz, Simonis et al. 2010
    % \end{itemize}
% \end{itemize}
% \end{frame}

\begin{frame}
\frametitle{Tutorial Based on ECLiPSe ELearning Course}
\begin{itemize}
\item Self-study course in constraint programming
\item Supported by Cisco Systems and Silicon Valley Community Foundation
\item Multi-media format, video lectures, slides, handout etc
\item {\small \url{https://eclipseclp.org/ELearning/index.html}}
\end{itemize}
\end{frame}

\begin{frame}
\frametitle{Also Part of CRT-AI Constraint Week}
\begin{itemize}
\item Annual one week course on CP and Optimization in Ireland
\item Part of national training program for PhD students in AI
\item \url{https://www.crt-ai.ie/}
\end{itemize}
\end{frame}



\begin{frame}
\frametitle{Constraint Programming - in a nutshell}
\begin{itemize}
\item Declarative description of problems with
\begin{itemize}
\item {\em Variables} which range over (finite) sets of values
\item {\em Constraints} over subsets of variables which restrict possible value combinations
\item A {\em solution} is a value assignment which satisfies all constraints
\end{itemize}

\item Constraint propagation/reasoning
\begin{itemize}
\item Removing inconsistent values for variables
\item Detect failure if constraint can not be satisfied
\item Interaction of constraints via shared variables
\item Incomplete
\end{itemize}

\item Search
\begin{itemize}
\item User controlled assignment of values to variables
\item Each step triggers constraint propagation 
\end{itemize}
\item Different domains require/allow different methods
\end{itemize}
\end{frame}

\begin{frame}
  \frametitle{Constraint Programming is Different}
  \begin{itemize}
  \item Declarative Programming
    \begin{itemize}
    \item Concentrate on what you want
      \item Not how to get there
      \item Program != Algorithm
      \item Program = Model
    \end{itemize}
    \item Applied to Combinatorial Problems
      \begin{itemize}
        \item No complete polynomial algorithms known (exist?)
        \item CP less ad-hoc than heuristics
        \item Models can evolve
  \end{itemize}
  \end{itemize}
  \end{frame}
    
\begin{frame}
  \frametitle{A Subtractive Process}
  \begin{textblock}{4}(8,-3)
    \includegraphics[width=4cm]{../introduction/images/stages}
  \end{textblock}
  \vfill
  \begin{quote}
    ``Oh, bosh, as Mr. Ruskin says. Sculpture, per se, is the simplest thing in the world. All you have to do is to take a big chunk of marble and a hammer and chisel, make up your mind what you are about to create and chip off all the marble you don't want.''-Paris Gaulois.
  \end{quote}
  
  {\tiny Source: \url{https://quoteinvestigator.com/2014/06/22/chip-away/}}
\end{frame}

\begin{frame}
\frametitle{Basic Process}
\begin{center}
\begin{tikzpicture}
\node[shape=rectangle,fill=pantone127-4] (problem) at (2,5) {Problem};
\node (human) at (2,4) {Human};
\node[shape=rectangle,fill=pantone127-4] (model) at (2,3) {Model};
\node (solver) at (2,2) {Constraint Solver/Search};
\node[shape=rectangle,fill=pantone157-8] (solution) at (2,1) {Solution};
\draw[-] (problem) -- (human);
\draw[->] (human) -- (model);
\draw[-] (model) -- (solver);
\draw[->] (solver) -- (solution);
\end{tikzpicture}
\end{center}
\end{frame}

\begin{frame}
\frametitle{More Realistic}
\begin{center}
\begin{tikzpicture}[xscale=4.0]
\node[shape=rectangle,fill=pantone127-4] (problem) at (2,5) {Problem};
\node (human) at (2,4) {Human};
\node[shape=rectangle,fill=pantone127-4] (model) at (2,3) {Model};
\node (solver) at (2,2) {Constraint Solver/Search};
\node[shape=rectangle,fill=pantone24-8] (hangs) at (1,1) {Hangs};
\node[shape=rectangle,fill=pantone157-8] (solution) at (2,1) {Solution};
\node[shape=rectangle,fill=pantone24-8] (wrong) at (3,1) {Wrong Solution};
\draw[-] (problem) -- (human);
\draw[->] (human) -- (model);
\draw[-] (model) -- (solver);
\draw[->] (solver) -- (hangs);
\draw[->] (solver) -- (solution);
\draw[->] (solver) -- (wrong);
\draw[->] (hangs) -- (1,4) -- (human);
\draw[->] (wrong) -- (3,4) -- (human);
\end{tikzpicture}
\end{center}
\end{frame}

\begin{frame}
\frametitle{Dual Role of Model}
\begin{itemize}
\item Allows Human to Express Problem
\begin{itemize}
\item Close to Problem Domain
\item Constraints as Abstractions
\end{itemize}
\item Allows Solver to Execute
\begin{itemize}
\item Variables as Communication Mechanism
\item Constraints as Algorithms
\end{itemize}
\end{itemize}
\end{frame}

\begin{frame}
\frametitle{Modelling Frameworks}
\begin{itemize}
\item \href{https://www.minizinc.org/}{MiniZinc} (NICTA, Monash University, Australia)
\item \href{https://github.com/eomahony/Numberjack}{NumberJack} (Insight, Ireland) 
\item \href{https://www-users.york.ac.uk/peter.nightingale/savilerow//}{EssencePrime/SavilleRow} (UK) 
\item \href{https://cpmpy.readthedocs.io/en/latest/}{CPMpy} (KU Leuven)
\item Allow use of multiple back-end solvers
\item Compile model into variants for each solver
\item A priori solver independent model(CP, MIP, SAT)
\end{itemize}
\end{frame} 

\begin{frame}
\frametitle{Framework Process}
\begin{center}
\begin{tikzpicture}[xscale=2]
\node[shape=rectangle,fill=pantone127-4] (problem) at (2.5,5) {Problem};
\node (human) at (2.5,4) {Human};
\node[shape=rectangle,fill=pantone127-4] (model) at (2.5,3) {Model};
\node (compile) at (2.5,2) {Compile/Reformulate};
\node (solver1) at (1,1) {CP};
\node (solver2) at (2,1) {MIP};
\node (solver3) at (3,1) {SAT};
\node (solver4) at (4,1) {Other};
\node[shape=rectangle,fill=pantone157-8] (solution1) at (1,0) {Solution};
\node[shape=rectangle,fill=pantone157-8] (solution2) at (2,0) {Solution};
\node[shape=rectangle,fill=pantone157-8] (solution3) at (3,0) {Solution};
\node[shape=rectangle,fill=pantone157-8] (solution4) at (4,0) {Solution};
\draw[-] (problem) -- (human);
\draw[->] (human) -- (model);
\draw[-] (model) -- (compile);
\draw[-] (compile) -- (solver1);
\draw[-] (compile) -- (solver2);
\draw[-] (compile) -- (solver3);
\draw[-] (compile) -- (solver4);
\draw[->] (solver1) -- (solution1);
\draw[->] (solver2) -- (solution2);
\draw[->] (solver3) -- (solution3);
\draw[->] (solver4) -- (solution4);
\end{tikzpicture}
\end{center}
\end{frame}

% \begin{frame}
  % \frametitle{Do It Now!}
  % \begin{itemize}
  % \item Download and install Minizinc
  % \item \url{https://www.minizinc.org/}
  % \end{itemize}
% \end{frame}


\begin{frame}
\frametitle{Why use Puzzles as Examples?}
\begin{itemize}
\item Easy to understand the problem
\item Solvable by hand without specialized knowledge
\item Possible to compare automated to manual solving process
\end{itemize}

\begin{quote}
The puzzle, though inanimate, is presented as a solvable problem without lasting negative consequences, a very low-risk low-reward situation. By being a puzzle, the object is attempting to convince the user that it must be completed.
\end{quote}
{\tiny Source: \href{https://www.everydayrhetoric.com/2020/03/the-rhetoric-of-puzzles/}{Every Day Rhetoric}}
\end{frame}


\mode<all>{
\part{Basic Constraint Propagation}
}
\begin{frame}
\partpage
\end{frame}
\mode<all>{
\input{../sendmore/body}
}

\mode<all>{
\part{Global Constraints}
}
\begin{frame}
\partpage
\end{frame}
\mode<all>{
\input{../sudoku/body}
}

\clearpage
\mode<all>{
\part{Customizing Search}
}
\begin{frame}
\partpage
\end{frame}
\mode<all>{
\input{../nqueen/body}
}


%% \clearpage
%% \mode<all>{
%%   \part{Problem Decomposition}
%% }
%% \begin{frame}
%%   \partpage
%% \end{frame}
%% \mode<all>{
%%   \input{../party/body}
%%   }

%% \clearpage
%% \mode<all>{
%%   \part{More Global Constraints}
%% }
%% \begin{frame}
%%   \partpage
%% \end{frame}
%% \mode<all>{
%%   \input{../car/body}
%%   }

%% \clearpage
%% \mode<all>{
%%   \part{Methodology}
%% }
%% \begin{frame}
%%   \partpage
%% \end{frame}
%% \mode<all>{
%%   \input{../methodology/body}
%%   }

\clearpage
\mode<all>{
\part{What is missing?}
}
\begin{frame}
\partpage
\end{frame}

\begin{frame}
\frametitle{Many Specialized Topics}
\begin{itemize}
\item How to design efficient core engine
\item Hybrids with LP/MIP tools
\item Hybrids with SAT
\item Symmetry breaking
\item Use of MDD/BDD to encode sets of solutions
\item High level modelling tools
\item Debugging/visualization
\end{itemize}
\end{frame}

\begin{frame}
\frametitle{Reformulation}
\begin{itemize}
\item Just because the user has modelled it this way, it doesn't mean we have to solve it that way
\begin{itemize}
\item Replace some constraint(s) by other, equivalent constraints
\item Because we don't have that constraint in our system
\item For performance
\end{itemize}
\end{itemize}
\end{frame}

\begin{frame}
\frametitle{Learning}
\begin{itemize}
\item While solving the problem we can learn how to strengthen the model/search 
\begin{itemize}
\item Understand which constraints/method contribute to propagation and change schedule
\item Learn no-good constraints by explaining failure
\item Adapt search strategy based on search experience
\end{itemize}
\end{itemize}
\end{frame}

%% \begin{frame}
%% \frametitle{Refined Process}
%% \begin{center}
%% \begin{tikzpicture}
%% \node[shape=rectangle,fill=pantone127-4] (problem) at (3,7) {Problem};
%% \node (human) at (3,6) {Human};
%% \node[shape=rectangle,fill=pantone127-4] (model) at (3,5) {User Model};
%% \node (reformulation) at (3,4) {Reformulation};
%% \node[shape=rectangle,fill=pantone127-4] (implementation) at (3,3) {Implementation Model};
%% \node (solver) at (3,2) {Constraint Solver/Search};
%% \node[shape=rectangle,fill=pantone157-8] (solution) at (3,1) {Solution};
%% \draw[-] (problem) -- (human);
%% \draw[->] (human) -- (model);
%% \draw[-] (model) -- (reformulation);
%% \draw[->] (reformulation) -- (implementation);
%% \draw[-] (implementation) -- (solver);
%% \draw[->] (solver) -- (0,2) -- (0,3) -- (implementation);
%% \draw[->] (solver) -- (solution);
%% \end{tikzpicture}
%% \end{center}
%% \end{frame}

%\begin{frame}
%\frametitle{What is CP actually used for?}
%\url{http://hsimonis.wordpress.com}
%\includegraphics[width=10cm]{../introduction/images/blog}
%\end{frame}

%% \begin{frame}
%% \frametitle{Exercises}
%% \begin{itemize}
%% \item Install Constraint Solver of your Choice
%% \item MiniZinc (MiningZinc) will be used in later sessions
%% \item Run Sendmoremoney, Sudoku, Queens examples
%% \item Can you choose which propagator to use for alldifferent?
%% \item Can you use different search strategies?
%% \end{itemize}
%% \end{frame}

\begin{frame}
\frametitle{More Learning Resources}
\begin{itemize}
\item Survey of Methods, Resources, and Formats for Teaching Constraint Programming
\begin{itemize}
\item by Tejas Santanam, Helmut Simonis
\item \url{https://doi.org/10.48550/arXiv.2403.12717}
\item Based on survey of community for WTCP 2023
\item \url{https://hsimonis.github.io/WTCP2023/}
\end{itemize}
\end{itemize}
\end{frame}


}


%% \clearpage
%% \mode<all>{
%%   \part{Problem Decomposition}
%% }
%% \begin{frame}
%%   \partpage
%% \end{frame}
%% \mode<all>{
%%   \begin{frame}
\frametitle{Objectives}
\begin{itemize}
\item Overview of Core Constraint Programming
\item Three Main Concepts
\begin{itemize}
\item Constraint Propagation
\item Global Constraints
\item Customizing Search
\end{itemize}
%\item Get Some Experience with MiniZinc
\item Topics will be treated in more detail in later parts of the school
\item Based on Examples, not Formal Description
\end{itemize}
\end{frame}

\begin{frame}
\frametitle{Outline}
\begin{itemize}
\item Why Constraint Programming?
\item Constraint Propagation
\item Global Constraints
\item Customizing Search
%\item Choices in Modelling
%\item Decomposition
%\item More Global Constraints
%\item Methodology
%\item What is missing?
\end{itemize}
\end{frame}

% \begin{frame}
  % \frametitle{Using MiniZinc IDE}
  % \begin{itemize}
  % \item Developed in the Australian NICTA project
  % \item Maintained by Monash University
  % \item Modelling tool with multiple back-end solvers
  % \item Available from \url{https://www.minizinc.org/}
  % \end{itemize}
% \end{frame}


% \begin{frame}
% \frametitle{Examples in ECLiPSe}
% \begin{itemize}
% \item Open sourced constraint programming language
% \item Development goes back to 1985
% \item ECRC, ICL, IC-Parc, PTL, Cisco
% \item \url{https://eclipseclp.org/}
% \item Specialities
% \begin{itemize}
% \item Develop new solvers for specific domains
% \item Integration with MIP
% \end{itemize}
% \item Not included in bundled MiniZinc IDE
% \item Specialized visualization tools used here
  % \begin{itemize}
  % \item CP-Viz, Simonis et al. 2010
    % \end{itemize}
% \end{itemize}
% \end{frame}

\begin{frame}
\frametitle{Tutorial Based on ECLiPSe ELearning Course}
\begin{itemize}
\item Self-study course in constraint programming
\item Supported by Cisco Systems and Silicon Valley Community Foundation
\item Multi-media format, video lectures, slides, handout etc
\item {\small \url{https://eclipseclp.org/ELearning/index.html}}
\end{itemize}
\end{frame}

\begin{frame}
\frametitle{Also Part of CRT-AI Constraint Week}
\begin{itemize}
\item Annual one week course on CP and Optimization in Ireland
\item Part of national training program for PhD students in AI
\item \url{https://www.crt-ai.ie/}
\end{itemize}
\end{frame}



\begin{frame}
\frametitle{Constraint Programming - in a nutshell}
\begin{itemize}
\item Declarative description of problems with
\begin{itemize}
\item {\em Variables} which range over (finite) sets of values
\item {\em Constraints} over subsets of variables which restrict possible value combinations
\item A {\em solution} is a value assignment which satisfies all constraints
\end{itemize}

\item Constraint propagation/reasoning
\begin{itemize}
\item Removing inconsistent values for variables
\item Detect failure if constraint can not be satisfied
\item Interaction of constraints via shared variables
\item Incomplete
\end{itemize}

\item Search
\begin{itemize}
\item User controlled assignment of values to variables
\item Each step triggers constraint propagation 
\end{itemize}
\item Different domains require/allow different methods
\end{itemize}
\end{frame}

\begin{frame}
  \frametitle{Constraint Programming is Different}
  \begin{itemize}
  \item Declarative Programming
    \begin{itemize}
    \item Concentrate on what you want
      \item Not how to get there
      \item Program != Algorithm
      \item Program = Model
    \end{itemize}
    \item Applied to Combinatorial Problems
      \begin{itemize}
        \item No complete polynomial algorithms known (exist?)
        \item CP less ad-hoc than heuristics
        \item Models can evolve
  \end{itemize}
  \end{itemize}
  \end{frame}
    
\begin{frame}
  \frametitle{A Subtractive Process}
  \begin{textblock}{4}(8,-3)
    \includegraphics[width=4cm]{../introduction/images/stages}
  \end{textblock}
  \vfill
  \begin{quote}
    ``Oh, bosh, as Mr. Ruskin says. Sculpture, per se, is the simplest thing in the world. All you have to do is to take a big chunk of marble and a hammer and chisel, make up your mind what you are about to create and chip off all the marble you don't want.''-Paris Gaulois.
  \end{quote}
  
  {\tiny Source: \url{https://quoteinvestigator.com/2014/06/22/chip-away/}}
\end{frame}

\begin{frame}
\frametitle{Basic Process}
\begin{center}
\begin{tikzpicture}
\node[shape=rectangle,fill=pantone127-4] (problem) at (2,5) {Problem};
\node (human) at (2,4) {Human};
\node[shape=rectangle,fill=pantone127-4] (model) at (2,3) {Model};
\node (solver) at (2,2) {Constraint Solver/Search};
\node[shape=rectangle,fill=pantone157-8] (solution) at (2,1) {Solution};
\draw[-] (problem) -- (human);
\draw[->] (human) -- (model);
\draw[-] (model) -- (solver);
\draw[->] (solver) -- (solution);
\end{tikzpicture}
\end{center}
\end{frame}

\begin{frame}
\frametitle{More Realistic}
\begin{center}
\begin{tikzpicture}[xscale=4.0]
\node[shape=rectangle,fill=pantone127-4] (problem) at (2,5) {Problem};
\node (human) at (2,4) {Human};
\node[shape=rectangle,fill=pantone127-4] (model) at (2,3) {Model};
\node (solver) at (2,2) {Constraint Solver/Search};
\node[shape=rectangle,fill=pantone24-8] (hangs) at (1,1) {Hangs};
\node[shape=rectangle,fill=pantone157-8] (solution) at (2,1) {Solution};
\node[shape=rectangle,fill=pantone24-8] (wrong) at (3,1) {Wrong Solution};
\draw[-] (problem) -- (human);
\draw[->] (human) -- (model);
\draw[-] (model) -- (solver);
\draw[->] (solver) -- (hangs);
\draw[->] (solver) -- (solution);
\draw[->] (solver) -- (wrong);
\draw[->] (hangs) -- (1,4) -- (human);
\draw[->] (wrong) -- (3,4) -- (human);
\end{tikzpicture}
\end{center}
\end{frame}

\begin{frame}
\frametitle{Dual Role of Model}
\begin{itemize}
\item Allows Human to Express Problem
\begin{itemize}
\item Close to Problem Domain
\item Constraints as Abstractions
\end{itemize}
\item Allows Solver to Execute
\begin{itemize}
\item Variables as Communication Mechanism
\item Constraints as Algorithms
\end{itemize}
\end{itemize}
\end{frame}

\begin{frame}
\frametitle{Modelling Frameworks}
\begin{itemize}
\item \href{https://www.minizinc.org/}{MiniZinc} (NICTA, Monash University, Australia)
\item \href{https://github.com/eomahony/Numberjack}{NumberJack} (Insight, Ireland) 
\item \href{https://www-users.york.ac.uk/peter.nightingale/savilerow//}{EssencePrime/SavilleRow} (UK) 
\item \href{https://cpmpy.readthedocs.io/en/latest/}{CPMpy} (KU Leuven)
\item Allow use of multiple back-end solvers
\item Compile model into variants for each solver
\item A priori solver independent model(CP, MIP, SAT)
\end{itemize}
\end{frame} 

\begin{frame}
\frametitle{Framework Process}
\begin{center}
\begin{tikzpicture}[xscale=2]
\node[shape=rectangle,fill=pantone127-4] (problem) at (2.5,5) {Problem};
\node (human) at (2.5,4) {Human};
\node[shape=rectangle,fill=pantone127-4] (model) at (2.5,3) {Model};
\node (compile) at (2.5,2) {Compile/Reformulate};
\node (solver1) at (1,1) {CP};
\node (solver2) at (2,1) {MIP};
\node (solver3) at (3,1) {SAT};
\node (solver4) at (4,1) {Other};
\node[shape=rectangle,fill=pantone157-8] (solution1) at (1,0) {Solution};
\node[shape=rectangle,fill=pantone157-8] (solution2) at (2,0) {Solution};
\node[shape=rectangle,fill=pantone157-8] (solution3) at (3,0) {Solution};
\node[shape=rectangle,fill=pantone157-8] (solution4) at (4,0) {Solution};
\draw[-] (problem) -- (human);
\draw[->] (human) -- (model);
\draw[-] (model) -- (compile);
\draw[-] (compile) -- (solver1);
\draw[-] (compile) -- (solver2);
\draw[-] (compile) -- (solver3);
\draw[-] (compile) -- (solver4);
\draw[->] (solver1) -- (solution1);
\draw[->] (solver2) -- (solution2);
\draw[->] (solver3) -- (solution3);
\draw[->] (solver4) -- (solution4);
\end{tikzpicture}
\end{center}
\end{frame}

% \begin{frame}
  % \frametitle{Do It Now!}
  % \begin{itemize}
  % \item Download and install Minizinc
  % \item \url{https://www.minizinc.org/}
  % \end{itemize}
% \end{frame}


\begin{frame}
\frametitle{Why use Puzzles as Examples?}
\begin{itemize}
\item Easy to understand the problem
\item Solvable by hand without specialized knowledge
\item Possible to compare automated to manual solving process
\end{itemize}

\begin{quote}
The puzzle, though inanimate, is presented as a solvable problem without lasting negative consequences, a very low-risk low-reward situation. By being a puzzle, the object is attempting to convince the user that it must be completed.
\end{quote}
{\tiny Source: \href{https://www.everydayrhetoric.com/2020/03/the-rhetoric-of-puzzles/}{Every Day Rhetoric}}
\end{frame}


\mode<all>{
\part{Basic Constraint Propagation}
}
\begin{frame}
\partpage
\end{frame}
\mode<all>{
\input{../sendmore/body}
}

\mode<all>{
\part{Global Constraints}
}
\begin{frame}
\partpage
\end{frame}
\mode<all>{
\input{../sudoku/body}
}

\clearpage
\mode<all>{
\part{Customizing Search}
}
\begin{frame}
\partpage
\end{frame}
\mode<all>{
\input{../nqueen/body}
}


%% \clearpage
%% \mode<all>{
%%   \part{Problem Decomposition}
%% }
%% \begin{frame}
%%   \partpage
%% \end{frame}
%% \mode<all>{
%%   \input{../party/body}
%%   }

%% \clearpage
%% \mode<all>{
%%   \part{More Global Constraints}
%% }
%% \begin{frame}
%%   \partpage
%% \end{frame}
%% \mode<all>{
%%   \input{../car/body}
%%   }

%% \clearpage
%% \mode<all>{
%%   \part{Methodology}
%% }
%% \begin{frame}
%%   \partpage
%% \end{frame}
%% \mode<all>{
%%   \input{../methodology/body}
%%   }

\clearpage
\mode<all>{
\part{What is missing?}
}
\begin{frame}
\partpage
\end{frame}

\begin{frame}
\frametitle{Many Specialized Topics}
\begin{itemize}
\item How to design efficient core engine
\item Hybrids with LP/MIP tools
\item Hybrids with SAT
\item Symmetry breaking
\item Use of MDD/BDD to encode sets of solutions
\item High level modelling tools
\item Debugging/visualization
\end{itemize}
\end{frame}

\begin{frame}
\frametitle{Reformulation}
\begin{itemize}
\item Just because the user has modelled it this way, it doesn't mean we have to solve it that way
\begin{itemize}
\item Replace some constraint(s) by other, equivalent constraints
\item Because we don't have that constraint in our system
\item For performance
\end{itemize}
\end{itemize}
\end{frame}

\begin{frame}
\frametitle{Learning}
\begin{itemize}
\item While solving the problem we can learn how to strengthen the model/search 
\begin{itemize}
\item Understand which constraints/method contribute to propagation and change schedule
\item Learn no-good constraints by explaining failure
\item Adapt search strategy based on search experience
\end{itemize}
\end{itemize}
\end{frame}

%% \begin{frame}
%% \frametitle{Refined Process}
%% \begin{center}
%% \begin{tikzpicture}
%% \node[shape=rectangle,fill=pantone127-4] (problem) at (3,7) {Problem};
%% \node (human) at (3,6) {Human};
%% \node[shape=rectangle,fill=pantone127-4] (model) at (3,5) {User Model};
%% \node (reformulation) at (3,4) {Reformulation};
%% \node[shape=rectangle,fill=pantone127-4] (implementation) at (3,3) {Implementation Model};
%% \node (solver) at (3,2) {Constraint Solver/Search};
%% \node[shape=rectangle,fill=pantone157-8] (solution) at (3,1) {Solution};
%% \draw[-] (problem) -- (human);
%% \draw[->] (human) -- (model);
%% \draw[-] (model) -- (reformulation);
%% \draw[->] (reformulation) -- (implementation);
%% \draw[-] (implementation) -- (solver);
%% \draw[->] (solver) -- (0,2) -- (0,3) -- (implementation);
%% \draw[->] (solver) -- (solution);
%% \end{tikzpicture}
%% \end{center}
%% \end{frame}

%\begin{frame}
%\frametitle{What is CP actually used for?}
%\url{http://hsimonis.wordpress.com}
%\includegraphics[width=10cm]{../introduction/images/blog}
%\end{frame}

%% \begin{frame}
%% \frametitle{Exercises}
%% \begin{itemize}
%% \item Install Constraint Solver of your Choice
%% \item MiniZinc (MiningZinc) will be used in later sessions
%% \item Run Sendmoremoney, Sudoku, Queens examples
%% \item Can you choose which propagator to use for alldifferent?
%% \item Can you use different search strategies?
%% \end{itemize}
%% \end{frame}

\begin{frame}
\frametitle{More Learning Resources}
\begin{itemize}
\item Survey of Methods, Resources, and Formats for Teaching Constraint Programming
\begin{itemize}
\item by Tejas Santanam, Helmut Simonis
\item \url{https://doi.org/10.48550/arXiv.2403.12717}
\item Based on survey of community for WTCP 2023
\item \url{https://hsimonis.github.io/WTCP2023/}
\end{itemize}
\end{itemize}
\end{frame}


%%   }

%% \clearpage
%% \mode<all>{
%%   \part{More Global Constraints}
%% }
%% \begin{frame}
%%   \partpage
%% \end{frame}
%% \mode<all>{
%%   \begin{frame}
\frametitle{Objectives}
\begin{itemize}
\item Overview of Core Constraint Programming
\item Three Main Concepts
\begin{itemize}
\item Constraint Propagation
\item Global Constraints
\item Customizing Search
\end{itemize}
%\item Get Some Experience with MiniZinc
\item Topics will be treated in more detail in later parts of the school
\item Based on Examples, not Formal Description
\end{itemize}
\end{frame}

\begin{frame}
\frametitle{Outline}
\begin{itemize}
\item Why Constraint Programming?
\item Constraint Propagation
\item Global Constraints
\item Customizing Search
%\item Choices in Modelling
%\item Decomposition
%\item More Global Constraints
%\item Methodology
%\item What is missing?
\end{itemize}
\end{frame}

% \begin{frame}
  % \frametitle{Using MiniZinc IDE}
  % \begin{itemize}
  % \item Developed in the Australian NICTA project
  % \item Maintained by Monash University
  % \item Modelling tool with multiple back-end solvers
  % \item Available from \url{https://www.minizinc.org/}
  % \end{itemize}
% \end{frame}


% \begin{frame}
% \frametitle{Examples in ECLiPSe}
% \begin{itemize}
% \item Open sourced constraint programming language
% \item Development goes back to 1985
% \item ECRC, ICL, IC-Parc, PTL, Cisco
% \item \url{https://eclipseclp.org/}
% \item Specialities
% \begin{itemize}
% \item Develop new solvers for specific domains
% \item Integration with MIP
% \end{itemize}
% \item Not included in bundled MiniZinc IDE
% \item Specialized visualization tools used here
  % \begin{itemize}
  % \item CP-Viz, Simonis et al. 2010
    % \end{itemize}
% \end{itemize}
% \end{frame}

\begin{frame}
\frametitle{Tutorial Based on ECLiPSe ELearning Course}
\begin{itemize}
\item Self-study course in constraint programming
\item Supported by Cisco Systems and Silicon Valley Community Foundation
\item Multi-media format, video lectures, slides, handout etc
\item {\small \url{https://eclipseclp.org/ELearning/index.html}}
\end{itemize}
\end{frame}

\begin{frame}
\frametitle{Also Part of CRT-AI Constraint Week}
\begin{itemize}
\item Annual one week course on CP and Optimization in Ireland
\item Part of national training program for PhD students in AI
\item \url{https://www.crt-ai.ie/}
\end{itemize}
\end{frame}



\begin{frame}
\frametitle{Constraint Programming - in a nutshell}
\begin{itemize}
\item Declarative description of problems with
\begin{itemize}
\item {\em Variables} which range over (finite) sets of values
\item {\em Constraints} over subsets of variables which restrict possible value combinations
\item A {\em solution} is a value assignment which satisfies all constraints
\end{itemize}

\item Constraint propagation/reasoning
\begin{itemize}
\item Removing inconsistent values for variables
\item Detect failure if constraint can not be satisfied
\item Interaction of constraints via shared variables
\item Incomplete
\end{itemize}

\item Search
\begin{itemize}
\item User controlled assignment of values to variables
\item Each step triggers constraint propagation 
\end{itemize}
\item Different domains require/allow different methods
\end{itemize}
\end{frame}

\begin{frame}
  \frametitle{Constraint Programming is Different}
  \begin{itemize}
  \item Declarative Programming
    \begin{itemize}
    \item Concentrate on what you want
      \item Not how to get there
      \item Program != Algorithm
      \item Program = Model
    \end{itemize}
    \item Applied to Combinatorial Problems
      \begin{itemize}
        \item No complete polynomial algorithms known (exist?)
        \item CP less ad-hoc than heuristics
        \item Models can evolve
  \end{itemize}
  \end{itemize}
  \end{frame}
    
\begin{frame}
  \frametitle{A Subtractive Process}
  \begin{textblock}{4}(8,-3)
    \includegraphics[width=4cm]{../introduction/images/stages}
  \end{textblock}
  \vfill
  \begin{quote}
    ``Oh, bosh, as Mr. Ruskin says. Sculpture, per se, is the simplest thing in the world. All you have to do is to take a big chunk of marble and a hammer and chisel, make up your mind what you are about to create and chip off all the marble you don't want.''-Paris Gaulois.
  \end{quote}
  
  {\tiny Source: \url{https://quoteinvestigator.com/2014/06/22/chip-away/}}
\end{frame}

\begin{frame}
\frametitle{Basic Process}
\begin{center}
\begin{tikzpicture}
\node[shape=rectangle,fill=pantone127-4] (problem) at (2,5) {Problem};
\node (human) at (2,4) {Human};
\node[shape=rectangle,fill=pantone127-4] (model) at (2,3) {Model};
\node (solver) at (2,2) {Constraint Solver/Search};
\node[shape=rectangle,fill=pantone157-8] (solution) at (2,1) {Solution};
\draw[-] (problem) -- (human);
\draw[->] (human) -- (model);
\draw[-] (model) -- (solver);
\draw[->] (solver) -- (solution);
\end{tikzpicture}
\end{center}
\end{frame}

\begin{frame}
\frametitle{More Realistic}
\begin{center}
\begin{tikzpicture}[xscale=4.0]
\node[shape=rectangle,fill=pantone127-4] (problem) at (2,5) {Problem};
\node (human) at (2,4) {Human};
\node[shape=rectangle,fill=pantone127-4] (model) at (2,3) {Model};
\node (solver) at (2,2) {Constraint Solver/Search};
\node[shape=rectangle,fill=pantone24-8] (hangs) at (1,1) {Hangs};
\node[shape=rectangle,fill=pantone157-8] (solution) at (2,1) {Solution};
\node[shape=rectangle,fill=pantone24-8] (wrong) at (3,1) {Wrong Solution};
\draw[-] (problem) -- (human);
\draw[->] (human) -- (model);
\draw[-] (model) -- (solver);
\draw[->] (solver) -- (hangs);
\draw[->] (solver) -- (solution);
\draw[->] (solver) -- (wrong);
\draw[->] (hangs) -- (1,4) -- (human);
\draw[->] (wrong) -- (3,4) -- (human);
\end{tikzpicture}
\end{center}
\end{frame}

\begin{frame}
\frametitle{Dual Role of Model}
\begin{itemize}
\item Allows Human to Express Problem
\begin{itemize}
\item Close to Problem Domain
\item Constraints as Abstractions
\end{itemize}
\item Allows Solver to Execute
\begin{itemize}
\item Variables as Communication Mechanism
\item Constraints as Algorithms
\end{itemize}
\end{itemize}
\end{frame}

\begin{frame}
\frametitle{Modelling Frameworks}
\begin{itemize}
\item \href{https://www.minizinc.org/}{MiniZinc} (NICTA, Monash University, Australia)
\item \href{https://github.com/eomahony/Numberjack}{NumberJack} (Insight, Ireland) 
\item \href{https://www-users.york.ac.uk/peter.nightingale/savilerow//}{EssencePrime/SavilleRow} (UK) 
\item \href{https://cpmpy.readthedocs.io/en/latest/}{CPMpy} (KU Leuven)
\item Allow use of multiple back-end solvers
\item Compile model into variants for each solver
\item A priori solver independent model(CP, MIP, SAT)
\end{itemize}
\end{frame} 

\begin{frame}
\frametitle{Framework Process}
\begin{center}
\begin{tikzpicture}[xscale=2]
\node[shape=rectangle,fill=pantone127-4] (problem) at (2.5,5) {Problem};
\node (human) at (2.5,4) {Human};
\node[shape=rectangle,fill=pantone127-4] (model) at (2.5,3) {Model};
\node (compile) at (2.5,2) {Compile/Reformulate};
\node (solver1) at (1,1) {CP};
\node (solver2) at (2,1) {MIP};
\node (solver3) at (3,1) {SAT};
\node (solver4) at (4,1) {Other};
\node[shape=rectangle,fill=pantone157-8] (solution1) at (1,0) {Solution};
\node[shape=rectangle,fill=pantone157-8] (solution2) at (2,0) {Solution};
\node[shape=rectangle,fill=pantone157-8] (solution3) at (3,0) {Solution};
\node[shape=rectangle,fill=pantone157-8] (solution4) at (4,0) {Solution};
\draw[-] (problem) -- (human);
\draw[->] (human) -- (model);
\draw[-] (model) -- (compile);
\draw[-] (compile) -- (solver1);
\draw[-] (compile) -- (solver2);
\draw[-] (compile) -- (solver3);
\draw[-] (compile) -- (solver4);
\draw[->] (solver1) -- (solution1);
\draw[->] (solver2) -- (solution2);
\draw[->] (solver3) -- (solution3);
\draw[->] (solver4) -- (solution4);
\end{tikzpicture}
\end{center}
\end{frame}

% \begin{frame}
  % \frametitle{Do It Now!}
  % \begin{itemize}
  % \item Download and install Minizinc
  % \item \url{https://www.minizinc.org/}
  % \end{itemize}
% \end{frame}


\begin{frame}
\frametitle{Why use Puzzles as Examples?}
\begin{itemize}
\item Easy to understand the problem
\item Solvable by hand without specialized knowledge
\item Possible to compare automated to manual solving process
\end{itemize}

\begin{quote}
The puzzle, though inanimate, is presented as a solvable problem without lasting negative consequences, a very low-risk low-reward situation. By being a puzzle, the object is attempting to convince the user that it must be completed.
\end{quote}
{\tiny Source: \href{https://www.everydayrhetoric.com/2020/03/the-rhetoric-of-puzzles/}{Every Day Rhetoric}}
\end{frame}


\mode<all>{
\part{Basic Constraint Propagation}
}
\begin{frame}
\partpage
\end{frame}
\mode<all>{
\input{../sendmore/body}
}

\mode<all>{
\part{Global Constraints}
}
\begin{frame}
\partpage
\end{frame}
\mode<all>{
\input{../sudoku/body}
}

\clearpage
\mode<all>{
\part{Customizing Search}
}
\begin{frame}
\partpage
\end{frame}
\mode<all>{
\input{../nqueen/body}
}


%% \clearpage
%% \mode<all>{
%%   \part{Problem Decomposition}
%% }
%% \begin{frame}
%%   \partpage
%% \end{frame}
%% \mode<all>{
%%   \input{../party/body}
%%   }

%% \clearpage
%% \mode<all>{
%%   \part{More Global Constraints}
%% }
%% \begin{frame}
%%   \partpage
%% \end{frame}
%% \mode<all>{
%%   \input{../car/body}
%%   }

%% \clearpage
%% \mode<all>{
%%   \part{Methodology}
%% }
%% \begin{frame}
%%   \partpage
%% \end{frame}
%% \mode<all>{
%%   \input{../methodology/body}
%%   }

\clearpage
\mode<all>{
\part{What is missing?}
}
\begin{frame}
\partpage
\end{frame}

\begin{frame}
\frametitle{Many Specialized Topics}
\begin{itemize}
\item How to design efficient core engine
\item Hybrids with LP/MIP tools
\item Hybrids with SAT
\item Symmetry breaking
\item Use of MDD/BDD to encode sets of solutions
\item High level modelling tools
\item Debugging/visualization
\end{itemize}
\end{frame}

\begin{frame}
\frametitle{Reformulation}
\begin{itemize}
\item Just because the user has modelled it this way, it doesn't mean we have to solve it that way
\begin{itemize}
\item Replace some constraint(s) by other, equivalent constraints
\item Because we don't have that constraint in our system
\item For performance
\end{itemize}
\end{itemize}
\end{frame}

\begin{frame}
\frametitle{Learning}
\begin{itemize}
\item While solving the problem we can learn how to strengthen the model/search 
\begin{itemize}
\item Understand which constraints/method contribute to propagation and change schedule
\item Learn no-good constraints by explaining failure
\item Adapt search strategy based on search experience
\end{itemize}
\end{itemize}
\end{frame}

%% \begin{frame}
%% \frametitle{Refined Process}
%% \begin{center}
%% \begin{tikzpicture}
%% \node[shape=rectangle,fill=pantone127-4] (problem) at (3,7) {Problem};
%% \node (human) at (3,6) {Human};
%% \node[shape=rectangle,fill=pantone127-4] (model) at (3,5) {User Model};
%% \node (reformulation) at (3,4) {Reformulation};
%% \node[shape=rectangle,fill=pantone127-4] (implementation) at (3,3) {Implementation Model};
%% \node (solver) at (3,2) {Constraint Solver/Search};
%% \node[shape=rectangle,fill=pantone157-8] (solution) at (3,1) {Solution};
%% \draw[-] (problem) -- (human);
%% \draw[->] (human) -- (model);
%% \draw[-] (model) -- (reformulation);
%% \draw[->] (reformulation) -- (implementation);
%% \draw[-] (implementation) -- (solver);
%% \draw[->] (solver) -- (0,2) -- (0,3) -- (implementation);
%% \draw[->] (solver) -- (solution);
%% \end{tikzpicture}
%% \end{center}
%% \end{frame}

%\begin{frame}
%\frametitle{What is CP actually used for?}
%\url{http://hsimonis.wordpress.com}
%\includegraphics[width=10cm]{../introduction/images/blog}
%\end{frame}

%% \begin{frame}
%% \frametitle{Exercises}
%% \begin{itemize}
%% \item Install Constraint Solver of your Choice
%% \item MiniZinc (MiningZinc) will be used in later sessions
%% \item Run Sendmoremoney, Sudoku, Queens examples
%% \item Can you choose which propagator to use for alldifferent?
%% \item Can you use different search strategies?
%% \end{itemize}
%% \end{frame}

\begin{frame}
\frametitle{More Learning Resources}
\begin{itemize}
\item Survey of Methods, Resources, and Formats for Teaching Constraint Programming
\begin{itemize}
\item by Tejas Santanam, Helmut Simonis
\item \url{https://doi.org/10.48550/arXiv.2403.12717}
\item Based on survey of community for WTCP 2023
\item \url{https://hsimonis.github.io/WTCP2023/}
\end{itemize}
\end{itemize}
\end{frame}


%%   }

%% \clearpage
%% \mode<all>{
%%   \part{Methodology}
%% }
%% \begin{frame}
%%   \partpage
%% \end{frame}
%% \mode<all>{
%%   \begin{frame}
\frametitle{Objectives}
\begin{itemize}
\item Overview of Core Constraint Programming
\item Three Main Concepts
\begin{itemize}
\item Constraint Propagation
\item Global Constraints
\item Customizing Search
\end{itemize}
%\item Get Some Experience with MiniZinc
\item Topics will be treated in more detail in later parts of the school
\item Based on Examples, not Formal Description
\end{itemize}
\end{frame}

\begin{frame}
\frametitle{Outline}
\begin{itemize}
\item Why Constraint Programming?
\item Constraint Propagation
\item Global Constraints
\item Customizing Search
%\item Choices in Modelling
%\item Decomposition
%\item More Global Constraints
%\item Methodology
%\item What is missing?
\end{itemize}
\end{frame}

% \begin{frame}
  % \frametitle{Using MiniZinc IDE}
  % \begin{itemize}
  % \item Developed in the Australian NICTA project
  % \item Maintained by Monash University
  % \item Modelling tool with multiple back-end solvers
  % \item Available from \url{https://www.minizinc.org/}
  % \end{itemize}
% \end{frame}


% \begin{frame}
% \frametitle{Examples in ECLiPSe}
% \begin{itemize}
% \item Open sourced constraint programming language
% \item Development goes back to 1985
% \item ECRC, ICL, IC-Parc, PTL, Cisco
% \item \url{https://eclipseclp.org/}
% \item Specialities
% \begin{itemize}
% \item Develop new solvers for specific domains
% \item Integration with MIP
% \end{itemize}
% \item Not included in bundled MiniZinc IDE
% \item Specialized visualization tools used here
  % \begin{itemize}
  % \item CP-Viz, Simonis et al. 2010
    % \end{itemize}
% \end{itemize}
% \end{frame}

\begin{frame}
\frametitle{Tutorial Based on ECLiPSe ELearning Course}
\begin{itemize}
\item Self-study course in constraint programming
\item Supported by Cisco Systems and Silicon Valley Community Foundation
\item Multi-media format, video lectures, slides, handout etc
\item {\small \url{https://eclipseclp.org/ELearning/index.html}}
\end{itemize}
\end{frame}

\begin{frame}
\frametitle{Also Part of CRT-AI Constraint Week}
\begin{itemize}
\item Annual one week course on CP and Optimization in Ireland
\item Part of national training program for PhD students in AI
\item \url{https://www.crt-ai.ie/}
\end{itemize}
\end{frame}



\begin{frame}
\frametitle{Constraint Programming - in a nutshell}
\begin{itemize}
\item Declarative description of problems with
\begin{itemize}
\item {\em Variables} which range over (finite) sets of values
\item {\em Constraints} over subsets of variables which restrict possible value combinations
\item A {\em solution} is a value assignment which satisfies all constraints
\end{itemize}

\item Constraint propagation/reasoning
\begin{itemize}
\item Removing inconsistent values for variables
\item Detect failure if constraint can not be satisfied
\item Interaction of constraints via shared variables
\item Incomplete
\end{itemize}

\item Search
\begin{itemize}
\item User controlled assignment of values to variables
\item Each step triggers constraint propagation 
\end{itemize}
\item Different domains require/allow different methods
\end{itemize}
\end{frame}

\begin{frame}
  \frametitle{Constraint Programming is Different}
  \begin{itemize}
  \item Declarative Programming
    \begin{itemize}
    \item Concentrate on what you want
      \item Not how to get there
      \item Program != Algorithm
      \item Program = Model
    \end{itemize}
    \item Applied to Combinatorial Problems
      \begin{itemize}
        \item No complete polynomial algorithms known (exist?)
        \item CP less ad-hoc than heuristics
        \item Models can evolve
  \end{itemize}
  \end{itemize}
  \end{frame}
    
\begin{frame}
  \frametitle{A Subtractive Process}
  \begin{textblock}{4}(8,-3)
    \includegraphics[width=4cm]{../introduction/images/stages}
  \end{textblock}
  \vfill
  \begin{quote}
    ``Oh, bosh, as Mr. Ruskin says. Sculpture, per se, is the simplest thing in the world. All you have to do is to take a big chunk of marble and a hammer and chisel, make up your mind what you are about to create and chip off all the marble you don't want.''-Paris Gaulois.
  \end{quote}
  
  {\tiny Source: \url{https://quoteinvestigator.com/2014/06/22/chip-away/}}
\end{frame}

\begin{frame}
\frametitle{Basic Process}
\begin{center}
\begin{tikzpicture}
\node[shape=rectangle,fill=pantone127-4] (problem) at (2,5) {Problem};
\node (human) at (2,4) {Human};
\node[shape=rectangle,fill=pantone127-4] (model) at (2,3) {Model};
\node (solver) at (2,2) {Constraint Solver/Search};
\node[shape=rectangle,fill=pantone157-8] (solution) at (2,1) {Solution};
\draw[-] (problem) -- (human);
\draw[->] (human) -- (model);
\draw[-] (model) -- (solver);
\draw[->] (solver) -- (solution);
\end{tikzpicture}
\end{center}
\end{frame}

\begin{frame}
\frametitle{More Realistic}
\begin{center}
\begin{tikzpicture}[xscale=4.0]
\node[shape=rectangle,fill=pantone127-4] (problem) at (2,5) {Problem};
\node (human) at (2,4) {Human};
\node[shape=rectangle,fill=pantone127-4] (model) at (2,3) {Model};
\node (solver) at (2,2) {Constraint Solver/Search};
\node[shape=rectangle,fill=pantone24-8] (hangs) at (1,1) {Hangs};
\node[shape=rectangle,fill=pantone157-8] (solution) at (2,1) {Solution};
\node[shape=rectangle,fill=pantone24-8] (wrong) at (3,1) {Wrong Solution};
\draw[-] (problem) -- (human);
\draw[->] (human) -- (model);
\draw[-] (model) -- (solver);
\draw[->] (solver) -- (hangs);
\draw[->] (solver) -- (solution);
\draw[->] (solver) -- (wrong);
\draw[->] (hangs) -- (1,4) -- (human);
\draw[->] (wrong) -- (3,4) -- (human);
\end{tikzpicture}
\end{center}
\end{frame}

\begin{frame}
\frametitle{Dual Role of Model}
\begin{itemize}
\item Allows Human to Express Problem
\begin{itemize}
\item Close to Problem Domain
\item Constraints as Abstractions
\end{itemize}
\item Allows Solver to Execute
\begin{itemize}
\item Variables as Communication Mechanism
\item Constraints as Algorithms
\end{itemize}
\end{itemize}
\end{frame}

\begin{frame}
\frametitle{Modelling Frameworks}
\begin{itemize}
\item \href{https://www.minizinc.org/}{MiniZinc} (NICTA, Monash University, Australia)
\item \href{https://github.com/eomahony/Numberjack}{NumberJack} (Insight, Ireland) 
\item \href{https://www-users.york.ac.uk/peter.nightingale/savilerow//}{EssencePrime/SavilleRow} (UK) 
\item \href{https://cpmpy.readthedocs.io/en/latest/}{CPMpy} (KU Leuven)
\item Allow use of multiple back-end solvers
\item Compile model into variants for each solver
\item A priori solver independent model(CP, MIP, SAT)
\end{itemize}
\end{frame} 

\begin{frame}
\frametitle{Framework Process}
\begin{center}
\begin{tikzpicture}[xscale=2]
\node[shape=rectangle,fill=pantone127-4] (problem) at (2.5,5) {Problem};
\node (human) at (2.5,4) {Human};
\node[shape=rectangle,fill=pantone127-4] (model) at (2.5,3) {Model};
\node (compile) at (2.5,2) {Compile/Reformulate};
\node (solver1) at (1,1) {CP};
\node (solver2) at (2,1) {MIP};
\node (solver3) at (3,1) {SAT};
\node (solver4) at (4,1) {Other};
\node[shape=rectangle,fill=pantone157-8] (solution1) at (1,0) {Solution};
\node[shape=rectangle,fill=pantone157-8] (solution2) at (2,0) {Solution};
\node[shape=rectangle,fill=pantone157-8] (solution3) at (3,0) {Solution};
\node[shape=rectangle,fill=pantone157-8] (solution4) at (4,0) {Solution};
\draw[-] (problem) -- (human);
\draw[->] (human) -- (model);
\draw[-] (model) -- (compile);
\draw[-] (compile) -- (solver1);
\draw[-] (compile) -- (solver2);
\draw[-] (compile) -- (solver3);
\draw[-] (compile) -- (solver4);
\draw[->] (solver1) -- (solution1);
\draw[->] (solver2) -- (solution2);
\draw[->] (solver3) -- (solution3);
\draw[->] (solver4) -- (solution4);
\end{tikzpicture}
\end{center}
\end{frame}

% \begin{frame}
  % \frametitle{Do It Now!}
  % \begin{itemize}
  % \item Download and install Minizinc
  % \item \url{https://www.minizinc.org/}
  % \end{itemize}
% \end{frame}


\begin{frame}
\frametitle{Why use Puzzles as Examples?}
\begin{itemize}
\item Easy to understand the problem
\item Solvable by hand without specialized knowledge
\item Possible to compare automated to manual solving process
\end{itemize}

\begin{quote}
The puzzle, though inanimate, is presented as a solvable problem without lasting negative consequences, a very low-risk low-reward situation. By being a puzzle, the object is attempting to convince the user that it must be completed.
\end{quote}
{\tiny Source: \href{https://www.everydayrhetoric.com/2020/03/the-rhetoric-of-puzzles/}{Every Day Rhetoric}}
\end{frame}


\mode<all>{
\part{Basic Constraint Propagation}
}
\begin{frame}
\partpage
\end{frame}
\mode<all>{
\input{../sendmore/body}
}

\mode<all>{
\part{Global Constraints}
}
\begin{frame}
\partpage
\end{frame}
\mode<all>{
\input{../sudoku/body}
}

\clearpage
\mode<all>{
\part{Customizing Search}
}
\begin{frame}
\partpage
\end{frame}
\mode<all>{
\input{../nqueen/body}
}


%% \clearpage
%% \mode<all>{
%%   \part{Problem Decomposition}
%% }
%% \begin{frame}
%%   \partpage
%% \end{frame}
%% \mode<all>{
%%   \input{../party/body}
%%   }

%% \clearpage
%% \mode<all>{
%%   \part{More Global Constraints}
%% }
%% \begin{frame}
%%   \partpage
%% \end{frame}
%% \mode<all>{
%%   \input{../car/body}
%%   }

%% \clearpage
%% \mode<all>{
%%   \part{Methodology}
%% }
%% \begin{frame}
%%   \partpage
%% \end{frame}
%% \mode<all>{
%%   \input{../methodology/body}
%%   }

\clearpage
\mode<all>{
\part{What is missing?}
}
\begin{frame}
\partpage
\end{frame}

\begin{frame}
\frametitle{Many Specialized Topics}
\begin{itemize}
\item How to design efficient core engine
\item Hybrids with LP/MIP tools
\item Hybrids with SAT
\item Symmetry breaking
\item Use of MDD/BDD to encode sets of solutions
\item High level modelling tools
\item Debugging/visualization
\end{itemize}
\end{frame}

\begin{frame}
\frametitle{Reformulation}
\begin{itemize}
\item Just because the user has modelled it this way, it doesn't mean we have to solve it that way
\begin{itemize}
\item Replace some constraint(s) by other, equivalent constraints
\item Because we don't have that constraint in our system
\item For performance
\end{itemize}
\end{itemize}
\end{frame}

\begin{frame}
\frametitle{Learning}
\begin{itemize}
\item While solving the problem we can learn how to strengthen the model/search 
\begin{itemize}
\item Understand which constraints/method contribute to propagation and change schedule
\item Learn no-good constraints by explaining failure
\item Adapt search strategy based on search experience
\end{itemize}
\end{itemize}
\end{frame}

%% \begin{frame}
%% \frametitle{Refined Process}
%% \begin{center}
%% \begin{tikzpicture}
%% \node[shape=rectangle,fill=pantone127-4] (problem) at (3,7) {Problem};
%% \node (human) at (3,6) {Human};
%% \node[shape=rectangle,fill=pantone127-4] (model) at (3,5) {User Model};
%% \node (reformulation) at (3,4) {Reformulation};
%% \node[shape=rectangle,fill=pantone127-4] (implementation) at (3,3) {Implementation Model};
%% \node (solver) at (3,2) {Constraint Solver/Search};
%% \node[shape=rectangle,fill=pantone157-8] (solution) at (3,1) {Solution};
%% \draw[-] (problem) -- (human);
%% \draw[->] (human) -- (model);
%% \draw[-] (model) -- (reformulation);
%% \draw[->] (reformulation) -- (implementation);
%% \draw[-] (implementation) -- (solver);
%% \draw[->] (solver) -- (0,2) -- (0,3) -- (implementation);
%% \draw[->] (solver) -- (solution);
%% \end{tikzpicture}
%% \end{center}
%% \end{frame}

%\begin{frame}
%\frametitle{What is CP actually used for?}
%\url{http://hsimonis.wordpress.com}
%\includegraphics[width=10cm]{../introduction/images/blog}
%\end{frame}

%% \begin{frame}
%% \frametitle{Exercises}
%% \begin{itemize}
%% \item Install Constraint Solver of your Choice
%% \item MiniZinc (MiningZinc) will be used in later sessions
%% \item Run Sendmoremoney, Sudoku, Queens examples
%% \item Can you choose which propagator to use for alldifferent?
%% \item Can you use different search strategies?
%% \end{itemize}
%% \end{frame}

\begin{frame}
\frametitle{More Learning Resources}
\begin{itemize}
\item Survey of Methods, Resources, and Formats for Teaching Constraint Programming
\begin{itemize}
\item by Tejas Santanam, Helmut Simonis
\item \url{https://doi.org/10.48550/arXiv.2403.12717}
\item Based on survey of community for WTCP 2023
\item \url{https://hsimonis.github.io/WTCP2023/}
\end{itemize}
\end{itemize}
\end{frame}


%%   }

\clearpage
\mode<all>{
\part{What is missing?}
}
\begin{frame}
\partpage
\end{frame}

\begin{frame}
\frametitle{Many Specialized Topics}
\begin{itemize}
\item How to design efficient core engine
\item Hybrids with LP/MIP tools
\item Hybrids with SAT
\item Symmetry breaking
\item Use of MDD/BDD to encode sets of solutions
\item High level modelling tools
\item Debugging/visualization
\end{itemize}
\end{frame}

\begin{frame}
\frametitle{Reformulation}
\begin{itemize}
\item Just because the user has modelled it this way, it doesn't mean we have to solve it that way
\begin{itemize}
\item Replace some constraint(s) by other, equivalent constraints
\item Because we don't have that constraint in our system
\item For performance
\end{itemize}
\end{itemize}
\end{frame}

\begin{frame}
\frametitle{Learning}
\begin{itemize}
\item While solving the problem we can learn how to strengthen the model/search 
\begin{itemize}
\item Understand which constraints/method contribute to propagation and change schedule
\item Learn no-good constraints by explaining failure
\item Adapt search strategy based on search experience
\end{itemize}
\end{itemize}
\end{frame}

%% \begin{frame}
%% \frametitle{Refined Process}
%% \begin{center}
%% \begin{tikzpicture}
%% \node[shape=rectangle,fill=pantone127-4] (problem) at (3,7) {Problem};
%% \node (human) at (3,6) {Human};
%% \node[shape=rectangle,fill=pantone127-4] (model) at (3,5) {User Model};
%% \node (reformulation) at (3,4) {Reformulation};
%% \node[shape=rectangle,fill=pantone127-4] (implementation) at (3,3) {Implementation Model};
%% \node (solver) at (3,2) {Constraint Solver/Search};
%% \node[shape=rectangle,fill=pantone157-8] (solution) at (3,1) {Solution};
%% \draw[-] (problem) -- (human);
%% \draw[->] (human) -- (model);
%% \draw[-] (model) -- (reformulation);
%% \draw[->] (reformulation) -- (implementation);
%% \draw[-] (implementation) -- (solver);
%% \draw[->] (solver) -- (0,2) -- (0,3) -- (implementation);
%% \draw[->] (solver) -- (solution);
%% \end{tikzpicture}
%% \end{center}
%% \end{frame}

%\begin{frame}
%\frametitle{What is CP actually used for?}
%\url{http://hsimonis.wordpress.com}
%\includegraphics[width=10cm]{../introduction/images/blog}
%\end{frame}

%% \begin{frame}
%% \frametitle{Exercises}
%% \begin{itemize}
%% \item Install Constraint Solver of your Choice
%% \item MiniZinc (MiningZinc) will be used in later sessions
%% \item Run Sendmoremoney, Sudoku, Queens examples
%% \item Can you choose which propagator to use for alldifferent?
%% \item Can you use different search strategies?
%% \end{itemize}
%% \end{frame}

\begin{frame}
\frametitle{More Learning Resources}
\begin{itemize}
\item Survey of Methods, Resources, and Formats for Teaching Constraint Programming
\begin{itemize}
\item by Tejas Santanam, Helmut Simonis
\item \url{https://doi.org/10.48550/arXiv.2403.12717}
\item Based on survey of community for WTCP 2023
\item \url{https://hsimonis.github.io/WTCP2023/}
\end{itemize}
\end{itemize}
\end{frame}


%%   }

%% \clearpage
%% \mode<all>{
%%   \part{More Global Constraints}
%% }
%% \begin{frame}
%%   \partpage
%% \end{frame}
%% \mode<all>{
%%   \begin{frame}
\frametitle{Objectives}
\begin{itemize}
\item Overview of Core Constraint Programming
\item Three Main Concepts
\begin{itemize}
\item Constraint Propagation
\item Global Constraints
\item Customizing Search
\end{itemize}
%\item Get Some Experience with MiniZinc
\item Topics will be treated in more detail in later parts of the school
\item Based on Examples, not Formal Description
\end{itemize}
\end{frame}

\begin{frame}
\frametitle{Outline}
\begin{itemize}
\item Why Constraint Programming?
\item Constraint Propagation
\item Global Constraints
\item Customizing Search
%\item Choices in Modelling
%\item Decomposition
%\item More Global Constraints
%\item Methodology
%\item What is missing?
\end{itemize}
\end{frame}

% \begin{frame}
  % \frametitle{Using MiniZinc IDE}
  % \begin{itemize}
  % \item Developed in the Australian NICTA project
  % \item Maintained by Monash University
  % \item Modelling tool with multiple back-end solvers
  % \item Available from \url{https://www.minizinc.org/}
  % \end{itemize}
% \end{frame}


% \begin{frame}
% \frametitle{Examples in ECLiPSe}
% \begin{itemize}
% \item Open sourced constraint programming language
% \item Development goes back to 1985
% \item ECRC, ICL, IC-Parc, PTL, Cisco
% \item \url{https://eclipseclp.org/}
% \item Specialities
% \begin{itemize}
% \item Develop new solvers for specific domains
% \item Integration with MIP
% \end{itemize}
% \item Not included in bundled MiniZinc IDE
% \item Specialized visualization tools used here
  % \begin{itemize}
  % \item CP-Viz, Simonis et al. 2010
    % \end{itemize}
% \end{itemize}
% \end{frame}

\begin{frame}
\frametitle{Tutorial Based on ECLiPSe ELearning Course}
\begin{itemize}
\item Self-study course in constraint programming
\item Supported by Cisco Systems and Silicon Valley Community Foundation
\item Multi-media format, video lectures, slides, handout etc
\item {\small \url{https://eclipseclp.org/ELearning/index.html}}
\end{itemize}
\end{frame}

\begin{frame}
\frametitle{Also Part of CRT-AI Constraint Week}
\begin{itemize}
\item Annual one week course on CP and Optimization in Ireland
\item Part of national training program for PhD students in AI
\item \url{https://www.crt-ai.ie/}
\end{itemize}
\end{frame}



\begin{frame}
\frametitle{Constraint Programming - in a nutshell}
\begin{itemize}
\item Declarative description of problems with
\begin{itemize}
\item {\em Variables} which range over (finite) sets of values
\item {\em Constraints} over subsets of variables which restrict possible value combinations
\item A {\em solution} is a value assignment which satisfies all constraints
\end{itemize}

\item Constraint propagation/reasoning
\begin{itemize}
\item Removing inconsistent values for variables
\item Detect failure if constraint can not be satisfied
\item Interaction of constraints via shared variables
\item Incomplete
\end{itemize}

\item Search
\begin{itemize}
\item User controlled assignment of values to variables
\item Each step triggers constraint propagation 
\end{itemize}
\item Different domains require/allow different methods
\end{itemize}
\end{frame}

\begin{frame}
  \frametitle{Constraint Programming is Different}
  \begin{itemize}
  \item Declarative Programming
    \begin{itemize}
    \item Concentrate on what you want
      \item Not how to get there
      \item Program != Algorithm
      \item Program = Model
    \end{itemize}
    \item Applied to Combinatorial Problems
      \begin{itemize}
        \item No complete polynomial algorithms known (exist?)
        \item CP less ad-hoc than heuristics
        \item Models can evolve
  \end{itemize}
  \end{itemize}
  \end{frame}
    
\begin{frame}
  \frametitle{A Subtractive Process}
  \begin{textblock}{4}(8,-3)
    \includegraphics[width=4cm]{../introduction/images/stages}
  \end{textblock}
  \vfill
  \begin{quote}
    ``Oh, bosh, as Mr. Ruskin says. Sculpture, per se, is the simplest thing in the world. All you have to do is to take a big chunk of marble and a hammer and chisel, make up your mind what you are about to create and chip off all the marble you don't want.''-Paris Gaulois.
  \end{quote}
  
  {\tiny Source: \url{https://quoteinvestigator.com/2014/06/22/chip-away/}}
\end{frame}

\begin{frame}
\frametitle{Basic Process}
\begin{center}
\begin{tikzpicture}
\node[shape=rectangle,fill=pantone127-4] (problem) at (2,5) {Problem};
\node (human) at (2,4) {Human};
\node[shape=rectangle,fill=pantone127-4] (model) at (2,3) {Model};
\node (solver) at (2,2) {Constraint Solver/Search};
\node[shape=rectangle,fill=pantone157-8] (solution) at (2,1) {Solution};
\draw[-] (problem) -- (human);
\draw[->] (human) -- (model);
\draw[-] (model) -- (solver);
\draw[->] (solver) -- (solution);
\end{tikzpicture}
\end{center}
\end{frame}

\begin{frame}
\frametitle{More Realistic}
\begin{center}
\begin{tikzpicture}[xscale=4.0]
\node[shape=rectangle,fill=pantone127-4] (problem) at (2,5) {Problem};
\node (human) at (2,4) {Human};
\node[shape=rectangle,fill=pantone127-4] (model) at (2,3) {Model};
\node (solver) at (2,2) {Constraint Solver/Search};
\node[shape=rectangle,fill=pantone24-8] (hangs) at (1,1) {Hangs};
\node[shape=rectangle,fill=pantone157-8] (solution) at (2,1) {Solution};
\node[shape=rectangle,fill=pantone24-8] (wrong) at (3,1) {Wrong Solution};
\draw[-] (problem) -- (human);
\draw[->] (human) -- (model);
\draw[-] (model) -- (solver);
\draw[->] (solver) -- (hangs);
\draw[->] (solver) -- (solution);
\draw[->] (solver) -- (wrong);
\draw[->] (hangs) -- (1,4) -- (human);
\draw[->] (wrong) -- (3,4) -- (human);
\end{tikzpicture}
\end{center}
\end{frame}

\begin{frame}
\frametitle{Dual Role of Model}
\begin{itemize}
\item Allows Human to Express Problem
\begin{itemize}
\item Close to Problem Domain
\item Constraints as Abstractions
\end{itemize}
\item Allows Solver to Execute
\begin{itemize}
\item Variables as Communication Mechanism
\item Constraints as Algorithms
\end{itemize}
\end{itemize}
\end{frame}

\begin{frame}
\frametitle{Modelling Frameworks}
\begin{itemize}
\item \href{https://www.minizinc.org/}{MiniZinc} (NICTA, Monash University, Australia)
\item \href{https://github.com/eomahony/Numberjack}{NumberJack} (Insight, Ireland) 
\item \href{https://www-users.york.ac.uk/peter.nightingale/savilerow//}{EssencePrime/SavilleRow} (UK) 
\item \href{https://cpmpy.readthedocs.io/en/latest/}{CPMpy} (KU Leuven)
\item Allow use of multiple back-end solvers
\item Compile model into variants for each solver
\item A priori solver independent model(CP, MIP, SAT)
\end{itemize}
\end{frame} 

\begin{frame}
\frametitle{Framework Process}
\begin{center}
\begin{tikzpicture}[xscale=2]
\node[shape=rectangle,fill=pantone127-4] (problem) at (2.5,5) {Problem};
\node (human) at (2.5,4) {Human};
\node[shape=rectangle,fill=pantone127-4] (model) at (2.5,3) {Model};
\node (compile) at (2.5,2) {Compile/Reformulate};
\node (solver1) at (1,1) {CP};
\node (solver2) at (2,1) {MIP};
\node (solver3) at (3,1) {SAT};
\node (solver4) at (4,1) {Other};
\node[shape=rectangle,fill=pantone157-8] (solution1) at (1,0) {Solution};
\node[shape=rectangle,fill=pantone157-8] (solution2) at (2,0) {Solution};
\node[shape=rectangle,fill=pantone157-8] (solution3) at (3,0) {Solution};
\node[shape=rectangle,fill=pantone157-8] (solution4) at (4,0) {Solution};
\draw[-] (problem) -- (human);
\draw[->] (human) -- (model);
\draw[-] (model) -- (compile);
\draw[-] (compile) -- (solver1);
\draw[-] (compile) -- (solver2);
\draw[-] (compile) -- (solver3);
\draw[-] (compile) -- (solver4);
\draw[->] (solver1) -- (solution1);
\draw[->] (solver2) -- (solution2);
\draw[->] (solver3) -- (solution3);
\draw[->] (solver4) -- (solution4);
\end{tikzpicture}
\end{center}
\end{frame}

% \begin{frame}
  % \frametitle{Do It Now!}
  % \begin{itemize}
  % \item Download and install Minizinc
  % \item \url{https://www.minizinc.org/}
  % \end{itemize}
% \end{frame}


\begin{frame}
\frametitle{Why use Puzzles as Examples?}
\begin{itemize}
\item Easy to understand the problem
\item Solvable by hand without specialized knowledge
\item Possible to compare automated to manual solving process
\end{itemize}

\begin{quote}
The puzzle, though inanimate, is presented as a solvable problem without lasting negative consequences, a very low-risk low-reward situation. By being a puzzle, the object is attempting to convince the user that it must be completed.
\end{quote}
{\tiny Source: \href{https://www.everydayrhetoric.com/2020/03/the-rhetoric-of-puzzles/}{Every Day Rhetoric}}
\end{frame}


\mode<all>{
\part{Basic Constraint Propagation}
}
\begin{frame}
\partpage
\end{frame}
\mode<all>{
\begin{frame}
\frametitle{Objectives}
\begin{itemize}
\item Overview of Core Constraint Programming
\item Three Main Concepts
\begin{itemize}
\item Constraint Propagation
\item Global Constraints
\item Customizing Search
\end{itemize}
%\item Get Some Experience with MiniZinc
\item Topics will be treated in more detail in later parts of the school
\item Based on Examples, not Formal Description
\end{itemize}
\end{frame}

\begin{frame}
\frametitle{Outline}
\begin{itemize}
\item Why Constraint Programming?
\item Constraint Propagation
\item Global Constraints
\item Customizing Search
%\item Choices in Modelling
%\item Decomposition
%\item More Global Constraints
%\item Methodology
%\item What is missing?
\end{itemize}
\end{frame}

% \begin{frame}
  % \frametitle{Using MiniZinc IDE}
  % \begin{itemize}
  % \item Developed in the Australian NICTA project
  % \item Maintained by Monash University
  % \item Modelling tool with multiple back-end solvers
  % \item Available from \url{https://www.minizinc.org/}
  % \end{itemize}
% \end{frame}


% \begin{frame}
% \frametitle{Examples in ECLiPSe}
% \begin{itemize}
% \item Open sourced constraint programming language
% \item Development goes back to 1985
% \item ECRC, ICL, IC-Parc, PTL, Cisco
% \item \url{https://eclipseclp.org/}
% \item Specialities
% \begin{itemize}
% \item Develop new solvers for specific domains
% \item Integration with MIP
% \end{itemize}
% \item Not included in bundled MiniZinc IDE
% \item Specialized visualization tools used here
  % \begin{itemize}
  % \item CP-Viz, Simonis et al. 2010
    % \end{itemize}
% \end{itemize}
% \end{frame}

\begin{frame}
\frametitle{Tutorial Based on ECLiPSe ELearning Course}
\begin{itemize}
\item Self-study course in constraint programming
\item Supported by Cisco Systems and Silicon Valley Community Foundation
\item Multi-media format, video lectures, slides, handout etc
\item {\small \url{https://eclipseclp.org/ELearning/index.html}}
\end{itemize}
\end{frame}

\begin{frame}
\frametitle{Also Part of CRT-AI Constraint Week}
\begin{itemize}
\item Annual one week course on CP and Optimization in Ireland
\item Part of national training program for PhD students in AI
\item \url{https://www.crt-ai.ie/}
\end{itemize}
\end{frame}



\begin{frame}
\frametitle{Constraint Programming - in a nutshell}
\begin{itemize}
\item Declarative description of problems with
\begin{itemize}
\item {\em Variables} which range over (finite) sets of values
\item {\em Constraints} over subsets of variables which restrict possible value combinations
\item A {\em solution} is a value assignment which satisfies all constraints
\end{itemize}

\item Constraint propagation/reasoning
\begin{itemize}
\item Removing inconsistent values for variables
\item Detect failure if constraint can not be satisfied
\item Interaction of constraints via shared variables
\item Incomplete
\end{itemize}

\item Search
\begin{itemize}
\item User controlled assignment of values to variables
\item Each step triggers constraint propagation 
\end{itemize}
\item Different domains require/allow different methods
\end{itemize}
\end{frame}

\begin{frame}
  \frametitle{Constraint Programming is Different}
  \begin{itemize}
  \item Declarative Programming
    \begin{itemize}
    \item Concentrate on what you want
      \item Not how to get there
      \item Program != Algorithm
      \item Program = Model
    \end{itemize}
    \item Applied to Combinatorial Problems
      \begin{itemize}
        \item No complete polynomial algorithms known (exist?)
        \item CP less ad-hoc than heuristics
        \item Models can evolve
  \end{itemize}
  \end{itemize}
  \end{frame}
    
\begin{frame}
  \frametitle{A Subtractive Process}
  \begin{textblock}{4}(8,-3)
    \includegraphics[width=4cm]{../introduction/images/stages}
  \end{textblock}
  \vfill
  \begin{quote}
    ``Oh, bosh, as Mr. Ruskin says. Sculpture, per se, is the simplest thing in the world. All you have to do is to take a big chunk of marble and a hammer and chisel, make up your mind what you are about to create and chip off all the marble you don't want.''-Paris Gaulois.
  \end{quote}
  
  {\tiny Source: \url{https://quoteinvestigator.com/2014/06/22/chip-away/}}
\end{frame}

\begin{frame}
\frametitle{Basic Process}
\begin{center}
\begin{tikzpicture}
\node[shape=rectangle,fill=pantone127-4] (problem) at (2,5) {Problem};
\node (human) at (2,4) {Human};
\node[shape=rectangle,fill=pantone127-4] (model) at (2,3) {Model};
\node (solver) at (2,2) {Constraint Solver/Search};
\node[shape=rectangle,fill=pantone157-8] (solution) at (2,1) {Solution};
\draw[-] (problem) -- (human);
\draw[->] (human) -- (model);
\draw[-] (model) -- (solver);
\draw[->] (solver) -- (solution);
\end{tikzpicture}
\end{center}
\end{frame}

\begin{frame}
\frametitle{More Realistic}
\begin{center}
\begin{tikzpicture}[xscale=4.0]
\node[shape=rectangle,fill=pantone127-4] (problem) at (2,5) {Problem};
\node (human) at (2,4) {Human};
\node[shape=rectangle,fill=pantone127-4] (model) at (2,3) {Model};
\node (solver) at (2,2) {Constraint Solver/Search};
\node[shape=rectangle,fill=pantone24-8] (hangs) at (1,1) {Hangs};
\node[shape=rectangle,fill=pantone157-8] (solution) at (2,1) {Solution};
\node[shape=rectangle,fill=pantone24-8] (wrong) at (3,1) {Wrong Solution};
\draw[-] (problem) -- (human);
\draw[->] (human) -- (model);
\draw[-] (model) -- (solver);
\draw[->] (solver) -- (hangs);
\draw[->] (solver) -- (solution);
\draw[->] (solver) -- (wrong);
\draw[->] (hangs) -- (1,4) -- (human);
\draw[->] (wrong) -- (3,4) -- (human);
\end{tikzpicture}
\end{center}
\end{frame}

\begin{frame}
\frametitle{Dual Role of Model}
\begin{itemize}
\item Allows Human to Express Problem
\begin{itemize}
\item Close to Problem Domain
\item Constraints as Abstractions
\end{itemize}
\item Allows Solver to Execute
\begin{itemize}
\item Variables as Communication Mechanism
\item Constraints as Algorithms
\end{itemize}
\end{itemize}
\end{frame}

\begin{frame}
\frametitle{Modelling Frameworks}
\begin{itemize}
\item \href{https://www.minizinc.org/}{MiniZinc} (NICTA, Monash University, Australia)
\item \href{https://github.com/eomahony/Numberjack}{NumberJack} (Insight, Ireland) 
\item \href{https://www-users.york.ac.uk/peter.nightingale/savilerow//}{EssencePrime/SavilleRow} (UK) 
\item \href{https://cpmpy.readthedocs.io/en/latest/}{CPMpy} (KU Leuven)
\item Allow use of multiple back-end solvers
\item Compile model into variants for each solver
\item A priori solver independent model(CP, MIP, SAT)
\end{itemize}
\end{frame} 

\begin{frame}
\frametitle{Framework Process}
\begin{center}
\begin{tikzpicture}[xscale=2]
\node[shape=rectangle,fill=pantone127-4] (problem) at (2.5,5) {Problem};
\node (human) at (2.5,4) {Human};
\node[shape=rectangle,fill=pantone127-4] (model) at (2.5,3) {Model};
\node (compile) at (2.5,2) {Compile/Reformulate};
\node (solver1) at (1,1) {CP};
\node (solver2) at (2,1) {MIP};
\node (solver3) at (3,1) {SAT};
\node (solver4) at (4,1) {Other};
\node[shape=rectangle,fill=pantone157-8] (solution1) at (1,0) {Solution};
\node[shape=rectangle,fill=pantone157-8] (solution2) at (2,0) {Solution};
\node[shape=rectangle,fill=pantone157-8] (solution3) at (3,0) {Solution};
\node[shape=rectangle,fill=pantone157-8] (solution4) at (4,0) {Solution};
\draw[-] (problem) -- (human);
\draw[->] (human) -- (model);
\draw[-] (model) -- (compile);
\draw[-] (compile) -- (solver1);
\draw[-] (compile) -- (solver2);
\draw[-] (compile) -- (solver3);
\draw[-] (compile) -- (solver4);
\draw[->] (solver1) -- (solution1);
\draw[->] (solver2) -- (solution2);
\draw[->] (solver3) -- (solution3);
\draw[->] (solver4) -- (solution4);
\end{tikzpicture}
\end{center}
\end{frame}

% \begin{frame}
  % \frametitle{Do It Now!}
  % \begin{itemize}
  % \item Download and install Minizinc
  % \item \url{https://www.minizinc.org/}
  % \end{itemize}
% \end{frame}


\begin{frame}
\frametitle{Why use Puzzles as Examples?}
\begin{itemize}
\item Easy to understand the problem
\item Solvable by hand without specialized knowledge
\item Possible to compare automated to manual solving process
\end{itemize}

\begin{quote}
The puzzle, though inanimate, is presented as a solvable problem without lasting negative consequences, a very low-risk low-reward situation. By being a puzzle, the object is attempting to convince the user that it must be completed.
\end{quote}
{\tiny Source: \href{https://www.everydayrhetoric.com/2020/03/the-rhetoric-of-puzzles/}{Every Day Rhetoric}}
\end{frame}


\mode<all>{
\part{Basic Constraint Propagation}
}
\begin{frame}
\partpage
\end{frame}
\mode<all>{
\input{../sendmore/body}
}

\mode<all>{
\part{Global Constraints}
}
\begin{frame}
\partpage
\end{frame}
\mode<all>{
\input{../sudoku/body}
}

\clearpage
\mode<all>{
\part{Customizing Search}
}
\begin{frame}
\partpage
\end{frame}
\mode<all>{
\input{../nqueen/body}
}


%% \clearpage
%% \mode<all>{
%%   \part{Problem Decomposition}
%% }
%% \begin{frame}
%%   \partpage
%% \end{frame}
%% \mode<all>{
%%   \input{../party/body}
%%   }

%% \clearpage
%% \mode<all>{
%%   \part{More Global Constraints}
%% }
%% \begin{frame}
%%   \partpage
%% \end{frame}
%% \mode<all>{
%%   \input{../car/body}
%%   }

%% \clearpage
%% \mode<all>{
%%   \part{Methodology}
%% }
%% \begin{frame}
%%   \partpage
%% \end{frame}
%% \mode<all>{
%%   \input{../methodology/body}
%%   }

\clearpage
\mode<all>{
\part{What is missing?}
}
\begin{frame}
\partpage
\end{frame}

\begin{frame}
\frametitle{Many Specialized Topics}
\begin{itemize}
\item How to design efficient core engine
\item Hybrids with LP/MIP tools
\item Hybrids with SAT
\item Symmetry breaking
\item Use of MDD/BDD to encode sets of solutions
\item High level modelling tools
\item Debugging/visualization
\end{itemize}
\end{frame}

\begin{frame}
\frametitle{Reformulation}
\begin{itemize}
\item Just because the user has modelled it this way, it doesn't mean we have to solve it that way
\begin{itemize}
\item Replace some constraint(s) by other, equivalent constraints
\item Because we don't have that constraint in our system
\item For performance
\end{itemize}
\end{itemize}
\end{frame}

\begin{frame}
\frametitle{Learning}
\begin{itemize}
\item While solving the problem we can learn how to strengthen the model/search 
\begin{itemize}
\item Understand which constraints/method contribute to propagation and change schedule
\item Learn no-good constraints by explaining failure
\item Adapt search strategy based on search experience
\end{itemize}
\end{itemize}
\end{frame}

%% \begin{frame}
%% \frametitle{Refined Process}
%% \begin{center}
%% \begin{tikzpicture}
%% \node[shape=rectangle,fill=pantone127-4] (problem) at (3,7) {Problem};
%% \node (human) at (3,6) {Human};
%% \node[shape=rectangle,fill=pantone127-4] (model) at (3,5) {User Model};
%% \node (reformulation) at (3,4) {Reformulation};
%% \node[shape=rectangle,fill=pantone127-4] (implementation) at (3,3) {Implementation Model};
%% \node (solver) at (3,2) {Constraint Solver/Search};
%% \node[shape=rectangle,fill=pantone157-8] (solution) at (3,1) {Solution};
%% \draw[-] (problem) -- (human);
%% \draw[->] (human) -- (model);
%% \draw[-] (model) -- (reformulation);
%% \draw[->] (reformulation) -- (implementation);
%% \draw[-] (implementation) -- (solver);
%% \draw[->] (solver) -- (0,2) -- (0,3) -- (implementation);
%% \draw[->] (solver) -- (solution);
%% \end{tikzpicture}
%% \end{center}
%% \end{frame}

%\begin{frame}
%\frametitle{What is CP actually used for?}
%\url{http://hsimonis.wordpress.com}
%\includegraphics[width=10cm]{../introduction/images/blog}
%\end{frame}

%% \begin{frame}
%% \frametitle{Exercises}
%% \begin{itemize}
%% \item Install Constraint Solver of your Choice
%% \item MiniZinc (MiningZinc) will be used in later sessions
%% \item Run Sendmoremoney, Sudoku, Queens examples
%% \item Can you choose which propagator to use for alldifferent?
%% \item Can you use different search strategies?
%% \end{itemize}
%% \end{frame}

\begin{frame}
\frametitle{More Learning Resources}
\begin{itemize}
\item Survey of Methods, Resources, and Formats for Teaching Constraint Programming
\begin{itemize}
\item by Tejas Santanam, Helmut Simonis
\item \url{https://doi.org/10.48550/arXiv.2403.12717}
\item Based on survey of community for WTCP 2023
\item \url{https://hsimonis.github.io/WTCP2023/}
\end{itemize}
\end{itemize}
\end{frame}


}

\mode<all>{
\part{Global Constraints}
}
\begin{frame}
\partpage
\end{frame}
\mode<all>{
\begin{frame}
\frametitle{Objectives}
\begin{itemize}
\item Overview of Core Constraint Programming
\item Three Main Concepts
\begin{itemize}
\item Constraint Propagation
\item Global Constraints
\item Customizing Search
\end{itemize}
%\item Get Some Experience with MiniZinc
\item Topics will be treated in more detail in later parts of the school
\item Based on Examples, not Formal Description
\end{itemize}
\end{frame}

\begin{frame}
\frametitle{Outline}
\begin{itemize}
\item Why Constraint Programming?
\item Constraint Propagation
\item Global Constraints
\item Customizing Search
%\item Choices in Modelling
%\item Decomposition
%\item More Global Constraints
%\item Methodology
%\item What is missing?
\end{itemize}
\end{frame}

% \begin{frame}
  % \frametitle{Using MiniZinc IDE}
  % \begin{itemize}
  % \item Developed in the Australian NICTA project
  % \item Maintained by Monash University
  % \item Modelling tool with multiple back-end solvers
  % \item Available from \url{https://www.minizinc.org/}
  % \end{itemize}
% \end{frame}


% \begin{frame}
% \frametitle{Examples in ECLiPSe}
% \begin{itemize}
% \item Open sourced constraint programming language
% \item Development goes back to 1985
% \item ECRC, ICL, IC-Parc, PTL, Cisco
% \item \url{https://eclipseclp.org/}
% \item Specialities
% \begin{itemize}
% \item Develop new solvers for specific domains
% \item Integration with MIP
% \end{itemize}
% \item Not included in bundled MiniZinc IDE
% \item Specialized visualization tools used here
  % \begin{itemize}
  % \item CP-Viz, Simonis et al. 2010
    % \end{itemize}
% \end{itemize}
% \end{frame}

\begin{frame}
\frametitle{Tutorial Based on ECLiPSe ELearning Course}
\begin{itemize}
\item Self-study course in constraint programming
\item Supported by Cisco Systems and Silicon Valley Community Foundation
\item Multi-media format, video lectures, slides, handout etc
\item {\small \url{https://eclipseclp.org/ELearning/index.html}}
\end{itemize}
\end{frame}

\begin{frame}
\frametitle{Also Part of CRT-AI Constraint Week}
\begin{itemize}
\item Annual one week course on CP and Optimization in Ireland
\item Part of national training program for PhD students in AI
\item \url{https://www.crt-ai.ie/}
\end{itemize}
\end{frame}



\begin{frame}
\frametitle{Constraint Programming - in a nutshell}
\begin{itemize}
\item Declarative description of problems with
\begin{itemize}
\item {\em Variables} which range over (finite) sets of values
\item {\em Constraints} over subsets of variables which restrict possible value combinations
\item A {\em solution} is a value assignment which satisfies all constraints
\end{itemize}

\item Constraint propagation/reasoning
\begin{itemize}
\item Removing inconsistent values for variables
\item Detect failure if constraint can not be satisfied
\item Interaction of constraints via shared variables
\item Incomplete
\end{itemize}

\item Search
\begin{itemize}
\item User controlled assignment of values to variables
\item Each step triggers constraint propagation 
\end{itemize}
\item Different domains require/allow different methods
\end{itemize}
\end{frame}

\begin{frame}
  \frametitle{Constraint Programming is Different}
  \begin{itemize}
  \item Declarative Programming
    \begin{itemize}
    \item Concentrate on what you want
      \item Not how to get there
      \item Program != Algorithm
      \item Program = Model
    \end{itemize}
    \item Applied to Combinatorial Problems
      \begin{itemize}
        \item No complete polynomial algorithms known (exist?)
        \item CP less ad-hoc than heuristics
        \item Models can evolve
  \end{itemize}
  \end{itemize}
  \end{frame}
    
\begin{frame}
  \frametitle{A Subtractive Process}
  \begin{textblock}{4}(8,-3)
    \includegraphics[width=4cm]{../introduction/images/stages}
  \end{textblock}
  \vfill
  \begin{quote}
    ``Oh, bosh, as Mr. Ruskin says. Sculpture, per se, is the simplest thing in the world. All you have to do is to take a big chunk of marble and a hammer and chisel, make up your mind what you are about to create and chip off all the marble you don't want.''-Paris Gaulois.
  \end{quote}
  
  {\tiny Source: \url{https://quoteinvestigator.com/2014/06/22/chip-away/}}
\end{frame}

\begin{frame}
\frametitle{Basic Process}
\begin{center}
\begin{tikzpicture}
\node[shape=rectangle,fill=pantone127-4] (problem) at (2,5) {Problem};
\node (human) at (2,4) {Human};
\node[shape=rectangle,fill=pantone127-4] (model) at (2,3) {Model};
\node (solver) at (2,2) {Constraint Solver/Search};
\node[shape=rectangle,fill=pantone157-8] (solution) at (2,1) {Solution};
\draw[-] (problem) -- (human);
\draw[->] (human) -- (model);
\draw[-] (model) -- (solver);
\draw[->] (solver) -- (solution);
\end{tikzpicture}
\end{center}
\end{frame}

\begin{frame}
\frametitle{More Realistic}
\begin{center}
\begin{tikzpicture}[xscale=4.0]
\node[shape=rectangle,fill=pantone127-4] (problem) at (2,5) {Problem};
\node (human) at (2,4) {Human};
\node[shape=rectangle,fill=pantone127-4] (model) at (2,3) {Model};
\node (solver) at (2,2) {Constraint Solver/Search};
\node[shape=rectangle,fill=pantone24-8] (hangs) at (1,1) {Hangs};
\node[shape=rectangle,fill=pantone157-8] (solution) at (2,1) {Solution};
\node[shape=rectangle,fill=pantone24-8] (wrong) at (3,1) {Wrong Solution};
\draw[-] (problem) -- (human);
\draw[->] (human) -- (model);
\draw[-] (model) -- (solver);
\draw[->] (solver) -- (hangs);
\draw[->] (solver) -- (solution);
\draw[->] (solver) -- (wrong);
\draw[->] (hangs) -- (1,4) -- (human);
\draw[->] (wrong) -- (3,4) -- (human);
\end{tikzpicture}
\end{center}
\end{frame}

\begin{frame}
\frametitle{Dual Role of Model}
\begin{itemize}
\item Allows Human to Express Problem
\begin{itemize}
\item Close to Problem Domain
\item Constraints as Abstractions
\end{itemize}
\item Allows Solver to Execute
\begin{itemize}
\item Variables as Communication Mechanism
\item Constraints as Algorithms
\end{itemize}
\end{itemize}
\end{frame}

\begin{frame}
\frametitle{Modelling Frameworks}
\begin{itemize}
\item \href{https://www.minizinc.org/}{MiniZinc} (NICTA, Monash University, Australia)
\item \href{https://github.com/eomahony/Numberjack}{NumberJack} (Insight, Ireland) 
\item \href{https://www-users.york.ac.uk/peter.nightingale/savilerow//}{EssencePrime/SavilleRow} (UK) 
\item \href{https://cpmpy.readthedocs.io/en/latest/}{CPMpy} (KU Leuven)
\item Allow use of multiple back-end solvers
\item Compile model into variants for each solver
\item A priori solver independent model(CP, MIP, SAT)
\end{itemize}
\end{frame} 

\begin{frame}
\frametitle{Framework Process}
\begin{center}
\begin{tikzpicture}[xscale=2]
\node[shape=rectangle,fill=pantone127-4] (problem) at (2.5,5) {Problem};
\node (human) at (2.5,4) {Human};
\node[shape=rectangle,fill=pantone127-4] (model) at (2.5,3) {Model};
\node (compile) at (2.5,2) {Compile/Reformulate};
\node (solver1) at (1,1) {CP};
\node (solver2) at (2,1) {MIP};
\node (solver3) at (3,1) {SAT};
\node (solver4) at (4,1) {Other};
\node[shape=rectangle,fill=pantone157-8] (solution1) at (1,0) {Solution};
\node[shape=rectangle,fill=pantone157-8] (solution2) at (2,0) {Solution};
\node[shape=rectangle,fill=pantone157-8] (solution3) at (3,0) {Solution};
\node[shape=rectangle,fill=pantone157-8] (solution4) at (4,0) {Solution};
\draw[-] (problem) -- (human);
\draw[->] (human) -- (model);
\draw[-] (model) -- (compile);
\draw[-] (compile) -- (solver1);
\draw[-] (compile) -- (solver2);
\draw[-] (compile) -- (solver3);
\draw[-] (compile) -- (solver4);
\draw[->] (solver1) -- (solution1);
\draw[->] (solver2) -- (solution2);
\draw[->] (solver3) -- (solution3);
\draw[->] (solver4) -- (solution4);
\end{tikzpicture}
\end{center}
\end{frame}

% \begin{frame}
  % \frametitle{Do It Now!}
  % \begin{itemize}
  % \item Download and install Minizinc
  % \item \url{https://www.minizinc.org/}
  % \end{itemize}
% \end{frame}


\begin{frame}
\frametitle{Why use Puzzles as Examples?}
\begin{itemize}
\item Easy to understand the problem
\item Solvable by hand without specialized knowledge
\item Possible to compare automated to manual solving process
\end{itemize}

\begin{quote}
The puzzle, though inanimate, is presented as a solvable problem without lasting negative consequences, a very low-risk low-reward situation. By being a puzzle, the object is attempting to convince the user that it must be completed.
\end{quote}
{\tiny Source: \href{https://www.everydayrhetoric.com/2020/03/the-rhetoric-of-puzzles/}{Every Day Rhetoric}}
\end{frame}


\mode<all>{
\part{Basic Constraint Propagation}
}
\begin{frame}
\partpage
\end{frame}
\mode<all>{
\input{../sendmore/body}
}

\mode<all>{
\part{Global Constraints}
}
\begin{frame}
\partpage
\end{frame}
\mode<all>{
\input{../sudoku/body}
}

\clearpage
\mode<all>{
\part{Customizing Search}
}
\begin{frame}
\partpage
\end{frame}
\mode<all>{
\input{../nqueen/body}
}


%% \clearpage
%% \mode<all>{
%%   \part{Problem Decomposition}
%% }
%% \begin{frame}
%%   \partpage
%% \end{frame}
%% \mode<all>{
%%   \input{../party/body}
%%   }

%% \clearpage
%% \mode<all>{
%%   \part{More Global Constraints}
%% }
%% \begin{frame}
%%   \partpage
%% \end{frame}
%% \mode<all>{
%%   \input{../car/body}
%%   }

%% \clearpage
%% \mode<all>{
%%   \part{Methodology}
%% }
%% \begin{frame}
%%   \partpage
%% \end{frame}
%% \mode<all>{
%%   \input{../methodology/body}
%%   }

\clearpage
\mode<all>{
\part{What is missing?}
}
\begin{frame}
\partpage
\end{frame}

\begin{frame}
\frametitle{Many Specialized Topics}
\begin{itemize}
\item How to design efficient core engine
\item Hybrids with LP/MIP tools
\item Hybrids with SAT
\item Symmetry breaking
\item Use of MDD/BDD to encode sets of solutions
\item High level modelling tools
\item Debugging/visualization
\end{itemize}
\end{frame}

\begin{frame}
\frametitle{Reformulation}
\begin{itemize}
\item Just because the user has modelled it this way, it doesn't mean we have to solve it that way
\begin{itemize}
\item Replace some constraint(s) by other, equivalent constraints
\item Because we don't have that constraint in our system
\item For performance
\end{itemize}
\end{itemize}
\end{frame}

\begin{frame}
\frametitle{Learning}
\begin{itemize}
\item While solving the problem we can learn how to strengthen the model/search 
\begin{itemize}
\item Understand which constraints/method contribute to propagation and change schedule
\item Learn no-good constraints by explaining failure
\item Adapt search strategy based on search experience
\end{itemize}
\end{itemize}
\end{frame}

%% \begin{frame}
%% \frametitle{Refined Process}
%% \begin{center}
%% \begin{tikzpicture}
%% \node[shape=rectangle,fill=pantone127-4] (problem) at (3,7) {Problem};
%% \node (human) at (3,6) {Human};
%% \node[shape=rectangle,fill=pantone127-4] (model) at (3,5) {User Model};
%% \node (reformulation) at (3,4) {Reformulation};
%% \node[shape=rectangle,fill=pantone127-4] (implementation) at (3,3) {Implementation Model};
%% \node (solver) at (3,2) {Constraint Solver/Search};
%% \node[shape=rectangle,fill=pantone157-8] (solution) at (3,1) {Solution};
%% \draw[-] (problem) -- (human);
%% \draw[->] (human) -- (model);
%% \draw[-] (model) -- (reformulation);
%% \draw[->] (reformulation) -- (implementation);
%% \draw[-] (implementation) -- (solver);
%% \draw[->] (solver) -- (0,2) -- (0,3) -- (implementation);
%% \draw[->] (solver) -- (solution);
%% \end{tikzpicture}
%% \end{center}
%% \end{frame}

%\begin{frame}
%\frametitle{What is CP actually used for?}
%\url{http://hsimonis.wordpress.com}
%\includegraphics[width=10cm]{../introduction/images/blog}
%\end{frame}

%% \begin{frame}
%% \frametitle{Exercises}
%% \begin{itemize}
%% \item Install Constraint Solver of your Choice
%% \item MiniZinc (MiningZinc) will be used in later sessions
%% \item Run Sendmoremoney, Sudoku, Queens examples
%% \item Can you choose which propagator to use for alldifferent?
%% \item Can you use different search strategies?
%% \end{itemize}
%% \end{frame}

\begin{frame}
\frametitle{More Learning Resources}
\begin{itemize}
\item Survey of Methods, Resources, and Formats for Teaching Constraint Programming
\begin{itemize}
\item by Tejas Santanam, Helmut Simonis
\item \url{https://doi.org/10.48550/arXiv.2403.12717}
\item Based on survey of community for WTCP 2023
\item \url{https://hsimonis.github.io/WTCP2023/}
\end{itemize}
\end{itemize}
\end{frame}


}

\clearpage
\mode<all>{
\part{Customizing Search}
}
\begin{frame}
\partpage
\end{frame}
\mode<all>{
\begin{frame}
\frametitle{Objectives}
\begin{itemize}
\item Overview of Core Constraint Programming
\item Three Main Concepts
\begin{itemize}
\item Constraint Propagation
\item Global Constraints
\item Customizing Search
\end{itemize}
%\item Get Some Experience with MiniZinc
\item Topics will be treated in more detail in later parts of the school
\item Based on Examples, not Formal Description
\end{itemize}
\end{frame}

\begin{frame}
\frametitle{Outline}
\begin{itemize}
\item Why Constraint Programming?
\item Constraint Propagation
\item Global Constraints
\item Customizing Search
%\item Choices in Modelling
%\item Decomposition
%\item More Global Constraints
%\item Methodology
%\item What is missing?
\end{itemize}
\end{frame}

% \begin{frame}
  % \frametitle{Using MiniZinc IDE}
  % \begin{itemize}
  % \item Developed in the Australian NICTA project
  % \item Maintained by Monash University
  % \item Modelling tool with multiple back-end solvers
  % \item Available from \url{https://www.minizinc.org/}
  % \end{itemize}
% \end{frame}


% \begin{frame}
% \frametitle{Examples in ECLiPSe}
% \begin{itemize}
% \item Open sourced constraint programming language
% \item Development goes back to 1985
% \item ECRC, ICL, IC-Parc, PTL, Cisco
% \item \url{https://eclipseclp.org/}
% \item Specialities
% \begin{itemize}
% \item Develop new solvers for specific domains
% \item Integration with MIP
% \end{itemize}
% \item Not included in bundled MiniZinc IDE
% \item Specialized visualization tools used here
  % \begin{itemize}
  % \item CP-Viz, Simonis et al. 2010
    % \end{itemize}
% \end{itemize}
% \end{frame}

\begin{frame}
\frametitle{Tutorial Based on ECLiPSe ELearning Course}
\begin{itemize}
\item Self-study course in constraint programming
\item Supported by Cisco Systems and Silicon Valley Community Foundation
\item Multi-media format, video lectures, slides, handout etc
\item {\small \url{https://eclipseclp.org/ELearning/index.html}}
\end{itemize}
\end{frame}

\begin{frame}
\frametitle{Also Part of CRT-AI Constraint Week}
\begin{itemize}
\item Annual one week course on CP and Optimization in Ireland
\item Part of national training program for PhD students in AI
\item \url{https://www.crt-ai.ie/}
\end{itemize}
\end{frame}



\begin{frame}
\frametitle{Constraint Programming - in a nutshell}
\begin{itemize}
\item Declarative description of problems with
\begin{itemize}
\item {\em Variables} which range over (finite) sets of values
\item {\em Constraints} over subsets of variables which restrict possible value combinations
\item A {\em solution} is a value assignment which satisfies all constraints
\end{itemize}

\item Constraint propagation/reasoning
\begin{itemize}
\item Removing inconsistent values for variables
\item Detect failure if constraint can not be satisfied
\item Interaction of constraints via shared variables
\item Incomplete
\end{itemize}

\item Search
\begin{itemize}
\item User controlled assignment of values to variables
\item Each step triggers constraint propagation 
\end{itemize}
\item Different domains require/allow different methods
\end{itemize}
\end{frame}

\begin{frame}
  \frametitle{Constraint Programming is Different}
  \begin{itemize}
  \item Declarative Programming
    \begin{itemize}
    \item Concentrate on what you want
      \item Not how to get there
      \item Program != Algorithm
      \item Program = Model
    \end{itemize}
    \item Applied to Combinatorial Problems
      \begin{itemize}
        \item No complete polynomial algorithms known (exist?)
        \item CP less ad-hoc than heuristics
        \item Models can evolve
  \end{itemize}
  \end{itemize}
  \end{frame}
    
\begin{frame}
  \frametitle{A Subtractive Process}
  \begin{textblock}{4}(8,-3)
    \includegraphics[width=4cm]{../introduction/images/stages}
  \end{textblock}
  \vfill
  \begin{quote}
    ``Oh, bosh, as Mr. Ruskin says. Sculpture, per se, is the simplest thing in the world. All you have to do is to take a big chunk of marble and a hammer and chisel, make up your mind what you are about to create and chip off all the marble you don't want.''-Paris Gaulois.
  \end{quote}
  
  {\tiny Source: \url{https://quoteinvestigator.com/2014/06/22/chip-away/}}
\end{frame}

\begin{frame}
\frametitle{Basic Process}
\begin{center}
\begin{tikzpicture}
\node[shape=rectangle,fill=pantone127-4] (problem) at (2,5) {Problem};
\node (human) at (2,4) {Human};
\node[shape=rectangle,fill=pantone127-4] (model) at (2,3) {Model};
\node (solver) at (2,2) {Constraint Solver/Search};
\node[shape=rectangle,fill=pantone157-8] (solution) at (2,1) {Solution};
\draw[-] (problem) -- (human);
\draw[->] (human) -- (model);
\draw[-] (model) -- (solver);
\draw[->] (solver) -- (solution);
\end{tikzpicture}
\end{center}
\end{frame}

\begin{frame}
\frametitle{More Realistic}
\begin{center}
\begin{tikzpicture}[xscale=4.0]
\node[shape=rectangle,fill=pantone127-4] (problem) at (2,5) {Problem};
\node (human) at (2,4) {Human};
\node[shape=rectangle,fill=pantone127-4] (model) at (2,3) {Model};
\node (solver) at (2,2) {Constraint Solver/Search};
\node[shape=rectangle,fill=pantone24-8] (hangs) at (1,1) {Hangs};
\node[shape=rectangle,fill=pantone157-8] (solution) at (2,1) {Solution};
\node[shape=rectangle,fill=pantone24-8] (wrong) at (3,1) {Wrong Solution};
\draw[-] (problem) -- (human);
\draw[->] (human) -- (model);
\draw[-] (model) -- (solver);
\draw[->] (solver) -- (hangs);
\draw[->] (solver) -- (solution);
\draw[->] (solver) -- (wrong);
\draw[->] (hangs) -- (1,4) -- (human);
\draw[->] (wrong) -- (3,4) -- (human);
\end{tikzpicture}
\end{center}
\end{frame}

\begin{frame}
\frametitle{Dual Role of Model}
\begin{itemize}
\item Allows Human to Express Problem
\begin{itemize}
\item Close to Problem Domain
\item Constraints as Abstractions
\end{itemize}
\item Allows Solver to Execute
\begin{itemize}
\item Variables as Communication Mechanism
\item Constraints as Algorithms
\end{itemize}
\end{itemize}
\end{frame}

\begin{frame}
\frametitle{Modelling Frameworks}
\begin{itemize}
\item \href{https://www.minizinc.org/}{MiniZinc} (NICTA, Monash University, Australia)
\item \href{https://github.com/eomahony/Numberjack}{NumberJack} (Insight, Ireland) 
\item \href{https://www-users.york.ac.uk/peter.nightingale/savilerow//}{EssencePrime/SavilleRow} (UK) 
\item \href{https://cpmpy.readthedocs.io/en/latest/}{CPMpy} (KU Leuven)
\item Allow use of multiple back-end solvers
\item Compile model into variants for each solver
\item A priori solver independent model(CP, MIP, SAT)
\end{itemize}
\end{frame} 

\begin{frame}
\frametitle{Framework Process}
\begin{center}
\begin{tikzpicture}[xscale=2]
\node[shape=rectangle,fill=pantone127-4] (problem) at (2.5,5) {Problem};
\node (human) at (2.5,4) {Human};
\node[shape=rectangle,fill=pantone127-4] (model) at (2.5,3) {Model};
\node (compile) at (2.5,2) {Compile/Reformulate};
\node (solver1) at (1,1) {CP};
\node (solver2) at (2,1) {MIP};
\node (solver3) at (3,1) {SAT};
\node (solver4) at (4,1) {Other};
\node[shape=rectangle,fill=pantone157-8] (solution1) at (1,0) {Solution};
\node[shape=rectangle,fill=pantone157-8] (solution2) at (2,0) {Solution};
\node[shape=rectangle,fill=pantone157-8] (solution3) at (3,0) {Solution};
\node[shape=rectangle,fill=pantone157-8] (solution4) at (4,0) {Solution};
\draw[-] (problem) -- (human);
\draw[->] (human) -- (model);
\draw[-] (model) -- (compile);
\draw[-] (compile) -- (solver1);
\draw[-] (compile) -- (solver2);
\draw[-] (compile) -- (solver3);
\draw[-] (compile) -- (solver4);
\draw[->] (solver1) -- (solution1);
\draw[->] (solver2) -- (solution2);
\draw[->] (solver3) -- (solution3);
\draw[->] (solver4) -- (solution4);
\end{tikzpicture}
\end{center}
\end{frame}

% \begin{frame}
  % \frametitle{Do It Now!}
  % \begin{itemize}
  % \item Download and install Minizinc
  % \item \url{https://www.minizinc.org/}
  % \end{itemize}
% \end{frame}


\begin{frame}
\frametitle{Why use Puzzles as Examples?}
\begin{itemize}
\item Easy to understand the problem
\item Solvable by hand without specialized knowledge
\item Possible to compare automated to manual solving process
\end{itemize}

\begin{quote}
The puzzle, though inanimate, is presented as a solvable problem without lasting negative consequences, a very low-risk low-reward situation. By being a puzzle, the object is attempting to convince the user that it must be completed.
\end{quote}
{\tiny Source: \href{https://www.everydayrhetoric.com/2020/03/the-rhetoric-of-puzzles/}{Every Day Rhetoric}}
\end{frame}


\mode<all>{
\part{Basic Constraint Propagation}
}
\begin{frame}
\partpage
\end{frame}
\mode<all>{
\input{../sendmore/body}
}

\mode<all>{
\part{Global Constraints}
}
\begin{frame}
\partpage
\end{frame}
\mode<all>{
\input{../sudoku/body}
}

\clearpage
\mode<all>{
\part{Customizing Search}
}
\begin{frame}
\partpage
\end{frame}
\mode<all>{
\input{../nqueen/body}
}


%% \clearpage
%% \mode<all>{
%%   \part{Problem Decomposition}
%% }
%% \begin{frame}
%%   \partpage
%% \end{frame}
%% \mode<all>{
%%   \input{../party/body}
%%   }

%% \clearpage
%% \mode<all>{
%%   \part{More Global Constraints}
%% }
%% \begin{frame}
%%   \partpage
%% \end{frame}
%% \mode<all>{
%%   \input{../car/body}
%%   }

%% \clearpage
%% \mode<all>{
%%   \part{Methodology}
%% }
%% \begin{frame}
%%   \partpage
%% \end{frame}
%% \mode<all>{
%%   \input{../methodology/body}
%%   }

\clearpage
\mode<all>{
\part{What is missing?}
}
\begin{frame}
\partpage
\end{frame}

\begin{frame}
\frametitle{Many Specialized Topics}
\begin{itemize}
\item How to design efficient core engine
\item Hybrids with LP/MIP tools
\item Hybrids with SAT
\item Symmetry breaking
\item Use of MDD/BDD to encode sets of solutions
\item High level modelling tools
\item Debugging/visualization
\end{itemize}
\end{frame}

\begin{frame}
\frametitle{Reformulation}
\begin{itemize}
\item Just because the user has modelled it this way, it doesn't mean we have to solve it that way
\begin{itemize}
\item Replace some constraint(s) by other, equivalent constraints
\item Because we don't have that constraint in our system
\item For performance
\end{itemize}
\end{itemize}
\end{frame}

\begin{frame}
\frametitle{Learning}
\begin{itemize}
\item While solving the problem we can learn how to strengthen the model/search 
\begin{itemize}
\item Understand which constraints/method contribute to propagation and change schedule
\item Learn no-good constraints by explaining failure
\item Adapt search strategy based on search experience
\end{itemize}
\end{itemize}
\end{frame}

%% \begin{frame}
%% \frametitle{Refined Process}
%% \begin{center}
%% \begin{tikzpicture}
%% \node[shape=rectangle,fill=pantone127-4] (problem) at (3,7) {Problem};
%% \node (human) at (3,6) {Human};
%% \node[shape=rectangle,fill=pantone127-4] (model) at (3,5) {User Model};
%% \node (reformulation) at (3,4) {Reformulation};
%% \node[shape=rectangle,fill=pantone127-4] (implementation) at (3,3) {Implementation Model};
%% \node (solver) at (3,2) {Constraint Solver/Search};
%% \node[shape=rectangle,fill=pantone157-8] (solution) at (3,1) {Solution};
%% \draw[-] (problem) -- (human);
%% \draw[->] (human) -- (model);
%% \draw[-] (model) -- (reformulation);
%% \draw[->] (reformulation) -- (implementation);
%% \draw[-] (implementation) -- (solver);
%% \draw[->] (solver) -- (0,2) -- (0,3) -- (implementation);
%% \draw[->] (solver) -- (solution);
%% \end{tikzpicture}
%% \end{center}
%% \end{frame}

%\begin{frame}
%\frametitle{What is CP actually used for?}
%\url{http://hsimonis.wordpress.com}
%\includegraphics[width=10cm]{../introduction/images/blog}
%\end{frame}

%% \begin{frame}
%% \frametitle{Exercises}
%% \begin{itemize}
%% \item Install Constraint Solver of your Choice
%% \item MiniZinc (MiningZinc) will be used in later sessions
%% \item Run Sendmoremoney, Sudoku, Queens examples
%% \item Can you choose which propagator to use for alldifferent?
%% \item Can you use different search strategies?
%% \end{itemize}
%% \end{frame}

\begin{frame}
\frametitle{More Learning Resources}
\begin{itemize}
\item Survey of Methods, Resources, and Formats for Teaching Constraint Programming
\begin{itemize}
\item by Tejas Santanam, Helmut Simonis
\item \url{https://doi.org/10.48550/arXiv.2403.12717}
\item Based on survey of community for WTCP 2023
\item \url{https://hsimonis.github.io/WTCP2023/}
\end{itemize}
\end{itemize}
\end{frame}


}


%% \clearpage
%% \mode<all>{
%%   \part{Problem Decomposition}
%% }
%% \begin{frame}
%%   \partpage
%% \end{frame}
%% \mode<all>{
%%   \begin{frame}
\frametitle{Objectives}
\begin{itemize}
\item Overview of Core Constraint Programming
\item Three Main Concepts
\begin{itemize}
\item Constraint Propagation
\item Global Constraints
\item Customizing Search
\end{itemize}
%\item Get Some Experience with MiniZinc
\item Topics will be treated in more detail in later parts of the school
\item Based on Examples, not Formal Description
\end{itemize}
\end{frame}

\begin{frame}
\frametitle{Outline}
\begin{itemize}
\item Why Constraint Programming?
\item Constraint Propagation
\item Global Constraints
\item Customizing Search
%\item Choices in Modelling
%\item Decomposition
%\item More Global Constraints
%\item Methodology
%\item What is missing?
\end{itemize}
\end{frame}

% \begin{frame}
  % \frametitle{Using MiniZinc IDE}
  % \begin{itemize}
  % \item Developed in the Australian NICTA project
  % \item Maintained by Monash University
  % \item Modelling tool with multiple back-end solvers
  % \item Available from \url{https://www.minizinc.org/}
  % \end{itemize}
% \end{frame}


% \begin{frame}
% \frametitle{Examples in ECLiPSe}
% \begin{itemize}
% \item Open sourced constraint programming language
% \item Development goes back to 1985
% \item ECRC, ICL, IC-Parc, PTL, Cisco
% \item \url{https://eclipseclp.org/}
% \item Specialities
% \begin{itemize}
% \item Develop new solvers for specific domains
% \item Integration with MIP
% \end{itemize}
% \item Not included in bundled MiniZinc IDE
% \item Specialized visualization tools used here
  % \begin{itemize}
  % \item CP-Viz, Simonis et al. 2010
    % \end{itemize}
% \end{itemize}
% \end{frame}

\begin{frame}
\frametitle{Tutorial Based on ECLiPSe ELearning Course}
\begin{itemize}
\item Self-study course in constraint programming
\item Supported by Cisco Systems and Silicon Valley Community Foundation
\item Multi-media format, video lectures, slides, handout etc
\item {\small \url{https://eclipseclp.org/ELearning/index.html}}
\end{itemize}
\end{frame}

\begin{frame}
\frametitle{Also Part of CRT-AI Constraint Week}
\begin{itemize}
\item Annual one week course on CP and Optimization in Ireland
\item Part of national training program for PhD students in AI
\item \url{https://www.crt-ai.ie/}
\end{itemize}
\end{frame}



\begin{frame}
\frametitle{Constraint Programming - in a nutshell}
\begin{itemize}
\item Declarative description of problems with
\begin{itemize}
\item {\em Variables} which range over (finite) sets of values
\item {\em Constraints} over subsets of variables which restrict possible value combinations
\item A {\em solution} is a value assignment which satisfies all constraints
\end{itemize}

\item Constraint propagation/reasoning
\begin{itemize}
\item Removing inconsistent values for variables
\item Detect failure if constraint can not be satisfied
\item Interaction of constraints via shared variables
\item Incomplete
\end{itemize}

\item Search
\begin{itemize}
\item User controlled assignment of values to variables
\item Each step triggers constraint propagation 
\end{itemize}
\item Different domains require/allow different methods
\end{itemize}
\end{frame}

\begin{frame}
  \frametitle{Constraint Programming is Different}
  \begin{itemize}
  \item Declarative Programming
    \begin{itemize}
    \item Concentrate on what you want
      \item Not how to get there
      \item Program != Algorithm
      \item Program = Model
    \end{itemize}
    \item Applied to Combinatorial Problems
      \begin{itemize}
        \item No complete polynomial algorithms known (exist?)
        \item CP less ad-hoc than heuristics
        \item Models can evolve
  \end{itemize}
  \end{itemize}
  \end{frame}
    
\begin{frame}
  \frametitle{A Subtractive Process}
  \begin{textblock}{4}(8,-3)
    \includegraphics[width=4cm]{../introduction/images/stages}
  \end{textblock}
  \vfill
  \begin{quote}
    ``Oh, bosh, as Mr. Ruskin says. Sculpture, per se, is the simplest thing in the world. All you have to do is to take a big chunk of marble and a hammer and chisel, make up your mind what you are about to create and chip off all the marble you don't want.''-Paris Gaulois.
  \end{quote}
  
  {\tiny Source: \url{https://quoteinvestigator.com/2014/06/22/chip-away/}}
\end{frame}

\begin{frame}
\frametitle{Basic Process}
\begin{center}
\begin{tikzpicture}
\node[shape=rectangle,fill=pantone127-4] (problem) at (2,5) {Problem};
\node (human) at (2,4) {Human};
\node[shape=rectangle,fill=pantone127-4] (model) at (2,3) {Model};
\node (solver) at (2,2) {Constraint Solver/Search};
\node[shape=rectangle,fill=pantone157-8] (solution) at (2,1) {Solution};
\draw[-] (problem) -- (human);
\draw[->] (human) -- (model);
\draw[-] (model) -- (solver);
\draw[->] (solver) -- (solution);
\end{tikzpicture}
\end{center}
\end{frame}

\begin{frame}
\frametitle{More Realistic}
\begin{center}
\begin{tikzpicture}[xscale=4.0]
\node[shape=rectangle,fill=pantone127-4] (problem) at (2,5) {Problem};
\node (human) at (2,4) {Human};
\node[shape=rectangle,fill=pantone127-4] (model) at (2,3) {Model};
\node (solver) at (2,2) {Constraint Solver/Search};
\node[shape=rectangle,fill=pantone24-8] (hangs) at (1,1) {Hangs};
\node[shape=rectangle,fill=pantone157-8] (solution) at (2,1) {Solution};
\node[shape=rectangle,fill=pantone24-8] (wrong) at (3,1) {Wrong Solution};
\draw[-] (problem) -- (human);
\draw[->] (human) -- (model);
\draw[-] (model) -- (solver);
\draw[->] (solver) -- (hangs);
\draw[->] (solver) -- (solution);
\draw[->] (solver) -- (wrong);
\draw[->] (hangs) -- (1,4) -- (human);
\draw[->] (wrong) -- (3,4) -- (human);
\end{tikzpicture}
\end{center}
\end{frame}

\begin{frame}
\frametitle{Dual Role of Model}
\begin{itemize}
\item Allows Human to Express Problem
\begin{itemize}
\item Close to Problem Domain
\item Constraints as Abstractions
\end{itemize}
\item Allows Solver to Execute
\begin{itemize}
\item Variables as Communication Mechanism
\item Constraints as Algorithms
\end{itemize}
\end{itemize}
\end{frame}

\begin{frame}
\frametitle{Modelling Frameworks}
\begin{itemize}
\item \href{https://www.minizinc.org/}{MiniZinc} (NICTA, Monash University, Australia)
\item \href{https://github.com/eomahony/Numberjack}{NumberJack} (Insight, Ireland) 
\item \href{https://www-users.york.ac.uk/peter.nightingale/savilerow//}{EssencePrime/SavilleRow} (UK) 
\item \href{https://cpmpy.readthedocs.io/en/latest/}{CPMpy} (KU Leuven)
\item Allow use of multiple back-end solvers
\item Compile model into variants for each solver
\item A priori solver independent model(CP, MIP, SAT)
\end{itemize}
\end{frame} 

\begin{frame}
\frametitle{Framework Process}
\begin{center}
\begin{tikzpicture}[xscale=2]
\node[shape=rectangle,fill=pantone127-4] (problem) at (2.5,5) {Problem};
\node (human) at (2.5,4) {Human};
\node[shape=rectangle,fill=pantone127-4] (model) at (2.5,3) {Model};
\node (compile) at (2.5,2) {Compile/Reformulate};
\node (solver1) at (1,1) {CP};
\node (solver2) at (2,1) {MIP};
\node (solver3) at (3,1) {SAT};
\node (solver4) at (4,1) {Other};
\node[shape=rectangle,fill=pantone157-8] (solution1) at (1,0) {Solution};
\node[shape=rectangle,fill=pantone157-8] (solution2) at (2,0) {Solution};
\node[shape=rectangle,fill=pantone157-8] (solution3) at (3,0) {Solution};
\node[shape=rectangle,fill=pantone157-8] (solution4) at (4,0) {Solution};
\draw[-] (problem) -- (human);
\draw[->] (human) -- (model);
\draw[-] (model) -- (compile);
\draw[-] (compile) -- (solver1);
\draw[-] (compile) -- (solver2);
\draw[-] (compile) -- (solver3);
\draw[-] (compile) -- (solver4);
\draw[->] (solver1) -- (solution1);
\draw[->] (solver2) -- (solution2);
\draw[->] (solver3) -- (solution3);
\draw[->] (solver4) -- (solution4);
\end{tikzpicture}
\end{center}
\end{frame}

% \begin{frame}
  % \frametitle{Do It Now!}
  % \begin{itemize}
  % \item Download and install Minizinc
  % \item \url{https://www.minizinc.org/}
  % \end{itemize}
% \end{frame}


\begin{frame}
\frametitle{Why use Puzzles as Examples?}
\begin{itemize}
\item Easy to understand the problem
\item Solvable by hand without specialized knowledge
\item Possible to compare automated to manual solving process
\end{itemize}

\begin{quote}
The puzzle, though inanimate, is presented as a solvable problem without lasting negative consequences, a very low-risk low-reward situation. By being a puzzle, the object is attempting to convince the user that it must be completed.
\end{quote}
{\tiny Source: \href{https://www.everydayrhetoric.com/2020/03/the-rhetoric-of-puzzles/}{Every Day Rhetoric}}
\end{frame}


\mode<all>{
\part{Basic Constraint Propagation}
}
\begin{frame}
\partpage
\end{frame}
\mode<all>{
\input{../sendmore/body}
}

\mode<all>{
\part{Global Constraints}
}
\begin{frame}
\partpage
\end{frame}
\mode<all>{
\input{../sudoku/body}
}

\clearpage
\mode<all>{
\part{Customizing Search}
}
\begin{frame}
\partpage
\end{frame}
\mode<all>{
\input{../nqueen/body}
}


%% \clearpage
%% \mode<all>{
%%   \part{Problem Decomposition}
%% }
%% \begin{frame}
%%   \partpage
%% \end{frame}
%% \mode<all>{
%%   \input{../party/body}
%%   }

%% \clearpage
%% \mode<all>{
%%   \part{More Global Constraints}
%% }
%% \begin{frame}
%%   \partpage
%% \end{frame}
%% \mode<all>{
%%   \input{../car/body}
%%   }

%% \clearpage
%% \mode<all>{
%%   \part{Methodology}
%% }
%% \begin{frame}
%%   \partpage
%% \end{frame}
%% \mode<all>{
%%   \input{../methodology/body}
%%   }

\clearpage
\mode<all>{
\part{What is missing?}
}
\begin{frame}
\partpage
\end{frame}

\begin{frame}
\frametitle{Many Specialized Topics}
\begin{itemize}
\item How to design efficient core engine
\item Hybrids with LP/MIP tools
\item Hybrids with SAT
\item Symmetry breaking
\item Use of MDD/BDD to encode sets of solutions
\item High level modelling tools
\item Debugging/visualization
\end{itemize}
\end{frame}

\begin{frame}
\frametitle{Reformulation}
\begin{itemize}
\item Just because the user has modelled it this way, it doesn't mean we have to solve it that way
\begin{itemize}
\item Replace some constraint(s) by other, equivalent constraints
\item Because we don't have that constraint in our system
\item For performance
\end{itemize}
\end{itemize}
\end{frame}

\begin{frame}
\frametitle{Learning}
\begin{itemize}
\item While solving the problem we can learn how to strengthen the model/search 
\begin{itemize}
\item Understand which constraints/method contribute to propagation and change schedule
\item Learn no-good constraints by explaining failure
\item Adapt search strategy based on search experience
\end{itemize}
\end{itemize}
\end{frame}

%% \begin{frame}
%% \frametitle{Refined Process}
%% \begin{center}
%% \begin{tikzpicture}
%% \node[shape=rectangle,fill=pantone127-4] (problem) at (3,7) {Problem};
%% \node (human) at (3,6) {Human};
%% \node[shape=rectangle,fill=pantone127-4] (model) at (3,5) {User Model};
%% \node (reformulation) at (3,4) {Reformulation};
%% \node[shape=rectangle,fill=pantone127-4] (implementation) at (3,3) {Implementation Model};
%% \node (solver) at (3,2) {Constraint Solver/Search};
%% \node[shape=rectangle,fill=pantone157-8] (solution) at (3,1) {Solution};
%% \draw[-] (problem) -- (human);
%% \draw[->] (human) -- (model);
%% \draw[-] (model) -- (reformulation);
%% \draw[->] (reformulation) -- (implementation);
%% \draw[-] (implementation) -- (solver);
%% \draw[->] (solver) -- (0,2) -- (0,3) -- (implementation);
%% \draw[->] (solver) -- (solution);
%% \end{tikzpicture}
%% \end{center}
%% \end{frame}

%\begin{frame}
%\frametitle{What is CP actually used for?}
%\url{http://hsimonis.wordpress.com}
%\includegraphics[width=10cm]{../introduction/images/blog}
%\end{frame}

%% \begin{frame}
%% \frametitle{Exercises}
%% \begin{itemize}
%% \item Install Constraint Solver of your Choice
%% \item MiniZinc (MiningZinc) will be used in later sessions
%% \item Run Sendmoremoney, Sudoku, Queens examples
%% \item Can you choose which propagator to use for alldifferent?
%% \item Can you use different search strategies?
%% \end{itemize}
%% \end{frame}

\begin{frame}
\frametitle{More Learning Resources}
\begin{itemize}
\item Survey of Methods, Resources, and Formats for Teaching Constraint Programming
\begin{itemize}
\item by Tejas Santanam, Helmut Simonis
\item \url{https://doi.org/10.48550/arXiv.2403.12717}
\item Based on survey of community for WTCP 2023
\item \url{https://hsimonis.github.io/WTCP2023/}
\end{itemize}
\end{itemize}
\end{frame}


%%   }

%% \clearpage
%% \mode<all>{
%%   \part{More Global Constraints}
%% }
%% \begin{frame}
%%   \partpage
%% \end{frame}
%% \mode<all>{
%%   \begin{frame}
\frametitle{Objectives}
\begin{itemize}
\item Overview of Core Constraint Programming
\item Three Main Concepts
\begin{itemize}
\item Constraint Propagation
\item Global Constraints
\item Customizing Search
\end{itemize}
%\item Get Some Experience with MiniZinc
\item Topics will be treated in more detail in later parts of the school
\item Based on Examples, not Formal Description
\end{itemize}
\end{frame}

\begin{frame}
\frametitle{Outline}
\begin{itemize}
\item Why Constraint Programming?
\item Constraint Propagation
\item Global Constraints
\item Customizing Search
%\item Choices in Modelling
%\item Decomposition
%\item More Global Constraints
%\item Methodology
%\item What is missing?
\end{itemize}
\end{frame}

% \begin{frame}
  % \frametitle{Using MiniZinc IDE}
  % \begin{itemize}
  % \item Developed in the Australian NICTA project
  % \item Maintained by Monash University
  % \item Modelling tool with multiple back-end solvers
  % \item Available from \url{https://www.minizinc.org/}
  % \end{itemize}
% \end{frame}


% \begin{frame}
% \frametitle{Examples in ECLiPSe}
% \begin{itemize}
% \item Open sourced constraint programming language
% \item Development goes back to 1985
% \item ECRC, ICL, IC-Parc, PTL, Cisco
% \item \url{https://eclipseclp.org/}
% \item Specialities
% \begin{itemize}
% \item Develop new solvers for specific domains
% \item Integration with MIP
% \end{itemize}
% \item Not included in bundled MiniZinc IDE
% \item Specialized visualization tools used here
  % \begin{itemize}
  % \item CP-Viz, Simonis et al. 2010
    % \end{itemize}
% \end{itemize}
% \end{frame}

\begin{frame}
\frametitle{Tutorial Based on ECLiPSe ELearning Course}
\begin{itemize}
\item Self-study course in constraint programming
\item Supported by Cisco Systems and Silicon Valley Community Foundation
\item Multi-media format, video lectures, slides, handout etc
\item {\small \url{https://eclipseclp.org/ELearning/index.html}}
\end{itemize}
\end{frame}

\begin{frame}
\frametitle{Also Part of CRT-AI Constraint Week}
\begin{itemize}
\item Annual one week course on CP and Optimization in Ireland
\item Part of national training program for PhD students in AI
\item \url{https://www.crt-ai.ie/}
\end{itemize}
\end{frame}



\begin{frame}
\frametitle{Constraint Programming - in a nutshell}
\begin{itemize}
\item Declarative description of problems with
\begin{itemize}
\item {\em Variables} which range over (finite) sets of values
\item {\em Constraints} over subsets of variables which restrict possible value combinations
\item A {\em solution} is a value assignment which satisfies all constraints
\end{itemize}

\item Constraint propagation/reasoning
\begin{itemize}
\item Removing inconsistent values for variables
\item Detect failure if constraint can not be satisfied
\item Interaction of constraints via shared variables
\item Incomplete
\end{itemize}

\item Search
\begin{itemize}
\item User controlled assignment of values to variables
\item Each step triggers constraint propagation 
\end{itemize}
\item Different domains require/allow different methods
\end{itemize}
\end{frame}

\begin{frame}
  \frametitle{Constraint Programming is Different}
  \begin{itemize}
  \item Declarative Programming
    \begin{itemize}
    \item Concentrate on what you want
      \item Not how to get there
      \item Program != Algorithm
      \item Program = Model
    \end{itemize}
    \item Applied to Combinatorial Problems
      \begin{itemize}
        \item No complete polynomial algorithms known (exist?)
        \item CP less ad-hoc than heuristics
        \item Models can evolve
  \end{itemize}
  \end{itemize}
  \end{frame}
    
\begin{frame}
  \frametitle{A Subtractive Process}
  \begin{textblock}{4}(8,-3)
    \includegraphics[width=4cm]{../introduction/images/stages}
  \end{textblock}
  \vfill
  \begin{quote}
    ``Oh, bosh, as Mr. Ruskin says. Sculpture, per se, is the simplest thing in the world. All you have to do is to take a big chunk of marble and a hammer and chisel, make up your mind what you are about to create and chip off all the marble you don't want.''-Paris Gaulois.
  \end{quote}
  
  {\tiny Source: \url{https://quoteinvestigator.com/2014/06/22/chip-away/}}
\end{frame}

\begin{frame}
\frametitle{Basic Process}
\begin{center}
\begin{tikzpicture}
\node[shape=rectangle,fill=pantone127-4] (problem) at (2,5) {Problem};
\node (human) at (2,4) {Human};
\node[shape=rectangle,fill=pantone127-4] (model) at (2,3) {Model};
\node (solver) at (2,2) {Constraint Solver/Search};
\node[shape=rectangle,fill=pantone157-8] (solution) at (2,1) {Solution};
\draw[-] (problem) -- (human);
\draw[->] (human) -- (model);
\draw[-] (model) -- (solver);
\draw[->] (solver) -- (solution);
\end{tikzpicture}
\end{center}
\end{frame}

\begin{frame}
\frametitle{More Realistic}
\begin{center}
\begin{tikzpicture}[xscale=4.0]
\node[shape=rectangle,fill=pantone127-4] (problem) at (2,5) {Problem};
\node (human) at (2,4) {Human};
\node[shape=rectangle,fill=pantone127-4] (model) at (2,3) {Model};
\node (solver) at (2,2) {Constraint Solver/Search};
\node[shape=rectangle,fill=pantone24-8] (hangs) at (1,1) {Hangs};
\node[shape=rectangle,fill=pantone157-8] (solution) at (2,1) {Solution};
\node[shape=rectangle,fill=pantone24-8] (wrong) at (3,1) {Wrong Solution};
\draw[-] (problem) -- (human);
\draw[->] (human) -- (model);
\draw[-] (model) -- (solver);
\draw[->] (solver) -- (hangs);
\draw[->] (solver) -- (solution);
\draw[->] (solver) -- (wrong);
\draw[->] (hangs) -- (1,4) -- (human);
\draw[->] (wrong) -- (3,4) -- (human);
\end{tikzpicture}
\end{center}
\end{frame}

\begin{frame}
\frametitle{Dual Role of Model}
\begin{itemize}
\item Allows Human to Express Problem
\begin{itemize}
\item Close to Problem Domain
\item Constraints as Abstractions
\end{itemize}
\item Allows Solver to Execute
\begin{itemize}
\item Variables as Communication Mechanism
\item Constraints as Algorithms
\end{itemize}
\end{itemize}
\end{frame}

\begin{frame}
\frametitle{Modelling Frameworks}
\begin{itemize}
\item \href{https://www.minizinc.org/}{MiniZinc} (NICTA, Monash University, Australia)
\item \href{https://github.com/eomahony/Numberjack}{NumberJack} (Insight, Ireland) 
\item \href{https://www-users.york.ac.uk/peter.nightingale/savilerow//}{EssencePrime/SavilleRow} (UK) 
\item \href{https://cpmpy.readthedocs.io/en/latest/}{CPMpy} (KU Leuven)
\item Allow use of multiple back-end solvers
\item Compile model into variants for each solver
\item A priori solver independent model(CP, MIP, SAT)
\end{itemize}
\end{frame} 

\begin{frame}
\frametitle{Framework Process}
\begin{center}
\begin{tikzpicture}[xscale=2]
\node[shape=rectangle,fill=pantone127-4] (problem) at (2.5,5) {Problem};
\node (human) at (2.5,4) {Human};
\node[shape=rectangle,fill=pantone127-4] (model) at (2.5,3) {Model};
\node (compile) at (2.5,2) {Compile/Reformulate};
\node (solver1) at (1,1) {CP};
\node (solver2) at (2,1) {MIP};
\node (solver3) at (3,1) {SAT};
\node (solver4) at (4,1) {Other};
\node[shape=rectangle,fill=pantone157-8] (solution1) at (1,0) {Solution};
\node[shape=rectangle,fill=pantone157-8] (solution2) at (2,0) {Solution};
\node[shape=rectangle,fill=pantone157-8] (solution3) at (3,0) {Solution};
\node[shape=rectangle,fill=pantone157-8] (solution4) at (4,0) {Solution};
\draw[-] (problem) -- (human);
\draw[->] (human) -- (model);
\draw[-] (model) -- (compile);
\draw[-] (compile) -- (solver1);
\draw[-] (compile) -- (solver2);
\draw[-] (compile) -- (solver3);
\draw[-] (compile) -- (solver4);
\draw[->] (solver1) -- (solution1);
\draw[->] (solver2) -- (solution2);
\draw[->] (solver3) -- (solution3);
\draw[->] (solver4) -- (solution4);
\end{tikzpicture}
\end{center}
\end{frame}

% \begin{frame}
  % \frametitle{Do It Now!}
  % \begin{itemize}
  % \item Download and install Minizinc
  % \item \url{https://www.minizinc.org/}
  % \end{itemize}
% \end{frame}


\begin{frame}
\frametitle{Why use Puzzles as Examples?}
\begin{itemize}
\item Easy to understand the problem
\item Solvable by hand without specialized knowledge
\item Possible to compare automated to manual solving process
\end{itemize}

\begin{quote}
The puzzle, though inanimate, is presented as a solvable problem without lasting negative consequences, a very low-risk low-reward situation. By being a puzzle, the object is attempting to convince the user that it must be completed.
\end{quote}
{\tiny Source: \href{https://www.everydayrhetoric.com/2020/03/the-rhetoric-of-puzzles/}{Every Day Rhetoric}}
\end{frame}


\mode<all>{
\part{Basic Constraint Propagation}
}
\begin{frame}
\partpage
\end{frame}
\mode<all>{
\input{../sendmore/body}
}

\mode<all>{
\part{Global Constraints}
}
\begin{frame}
\partpage
\end{frame}
\mode<all>{
\input{../sudoku/body}
}

\clearpage
\mode<all>{
\part{Customizing Search}
}
\begin{frame}
\partpage
\end{frame}
\mode<all>{
\input{../nqueen/body}
}


%% \clearpage
%% \mode<all>{
%%   \part{Problem Decomposition}
%% }
%% \begin{frame}
%%   \partpage
%% \end{frame}
%% \mode<all>{
%%   \input{../party/body}
%%   }

%% \clearpage
%% \mode<all>{
%%   \part{More Global Constraints}
%% }
%% \begin{frame}
%%   \partpage
%% \end{frame}
%% \mode<all>{
%%   \input{../car/body}
%%   }

%% \clearpage
%% \mode<all>{
%%   \part{Methodology}
%% }
%% \begin{frame}
%%   \partpage
%% \end{frame}
%% \mode<all>{
%%   \input{../methodology/body}
%%   }

\clearpage
\mode<all>{
\part{What is missing?}
}
\begin{frame}
\partpage
\end{frame}

\begin{frame}
\frametitle{Many Specialized Topics}
\begin{itemize}
\item How to design efficient core engine
\item Hybrids with LP/MIP tools
\item Hybrids with SAT
\item Symmetry breaking
\item Use of MDD/BDD to encode sets of solutions
\item High level modelling tools
\item Debugging/visualization
\end{itemize}
\end{frame}

\begin{frame}
\frametitle{Reformulation}
\begin{itemize}
\item Just because the user has modelled it this way, it doesn't mean we have to solve it that way
\begin{itemize}
\item Replace some constraint(s) by other, equivalent constraints
\item Because we don't have that constraint in our system
\item For performance
\end{itemize}
\end{itemize}
\end{frame}

\begin{frame}
\frametitle{Learning}
\begin{itemize}
\item While solving the problem we can learn how to strengthen the model/search 
\begin{itemize}
\item Understand which constraints/method contribute to propagation and change schedule
\item Learn no-good constraints by explaining failure
\item Adapt search strategy based on search experience
\end{itemize}
\end{itemize}
\end{frame}

%% \begin{frame}
%% \frametitle{Refined Process}
%% \begin{center}
%% \begin{tikzpicture}
%% \node[shape=rectangle,fill=pantone127-4] (problem) at (3,7) {Problem};
%% \node (human) at (3,6) {Human};
%% \node[shape=rectangle,fill=pantone127-4] (model) at (3,5) {User Model};
%% \node (reformulation) at (3,4) {Reformulation};
%% \node[shape=rectangle,fill=pantone127-4] (implementation) at (3,3) {Implementation Model};
%% \node (solver) at (3,2) {Constraint Solver/Search};
%% \node[shape=rectangle,fill=pantone157-8] (solution) at (3,1) {Solution};
%% \draw[-] (problem) -- (human);
%% \draw[->] (human) -- (model);
%% \draw[-] (model) -- (reformulation);
%% \draw[->] (reformulation) -- (implementation);
%% \draw[-] (implementation) -- (solver);
%% \draw[->] (solver) -- (0,2) -- (0,3) -- (implementation);
%% \draw[->] (solver) -- (solution);
%% \end{tikzpicture}
%% \end{center}
%% \end{frame}

%\begin{frame}
%\frametitle{What is CP actually used for?}
%\url{http://hsimonis.wordpress.com}
%\includegraphics[width=10cm]{../introduction/images/blog}
%\end{frame}

%% \begin{frame}
%% \frametitle{Exercises}
%% \begin{itemize}
%% \item Install Constraint Solver of your Choice
%% \item MiniZinc (MiningZinc) will be used in later sessions
%% \item Run Sendmoremoney, Sudoku, Queens examples
%% \item Can you choose which propagator to use for alldifferent?
%% \item Can you use different search strategies?
%% \end{itemize}
%% \end{frame}

\begin{frame}
\frametitle{More Learning Resources}
\begin{itemize}
\item Survey of Methods, Resources, and Formats for Teaching Constraint Programming
\begin{itemize}
\item by Tejas Santanam, Helmut Simonis
\item \url{https://doi.org/10.48550/arXiv.2403.12717}
\item Based on survey of community for WTCP 2023
\item \url{https://hsimonis.github.io/WTCP2023/}
\end{itemize}
\end{itemize}
\end{frame}


%%   }

%% \clearpage
%% \mode<all>{
%%   \part{Methodology}
%% }
%% \begin{frame}
%%   \partpage
%% \end{frame}
%% \mode<all>{
%%   \begin{frame}
\frametitle{Objectives}
\begin{itemize}
\item Overview of Core Constraint Programming
\item Three Main Concepts
\begin{itemize}
\item Constraint Propagation
\item Global Constraints
\item Customizing Search
\end{itemize}
%\item Get Some Experience with MiniZinc
\item Topics will be treated in more detail in later parts of the school
\item Based on Examples, not Formal Description
\end{itemize}
\end{frame}

\begin{frame}
\frametitle{Outline}
\begin{itemize}
\item Why Constraint Programming?
\item Constraint Propagation
\item Global Constraints
\item Customizing Search
%\item Choices in Modelling
%\item Decomposition
%\item More Global Constraints
%\item Methodology
%\item What is missing?
\end{itemize}
\end{frame}

% \begin{frame}
  % \frametitle{Using MiniZinc IDE}
  % \begin{itemize}
  % \item Developed in the Australian NICTA project
  % \item Maintained by Monash University
  % \item Modelling tool with multiple back-end solvers
  % \item Available from \url{https://www.minizinc.org/}
  % \end{itemize}
% \end{frame}


% \begin{frame}
% \frametitle{Examples in ECLiPSe}
% \begin{itemize}
% \item Open sourced constraint programming language
% \item Development goes back to 1985
% \item ECRC, ICL, IC-Parc, PTL, Cisco
% \item \url{https://eclipseclp.org/}
% \item Specialities
% \begin{itemize}
% \item Develop new solvers for specific domains
% \item Integration with MIP
% \end{itemize}
% \item Not included in bundled MiniZinc IDE
% \item Specialized visualization tools used here
  % \begin{itemize}
  % \item CP-Viz, Simonis et al. 2010
    % \end{itemize}
% \end{itemize}
% \end{frame}

\begin{frame}
\frametitle{Tutorial Based on ECLiPSe ELearning Course}
\begin{itemize}
\item Self-study course in constraint programming
\item Supported by Cisco Systems and Silicon Valley Community Foundation
\item Multi-media format, video lectures, slides, handout etc
\item {\small \url{https://eclipseclp.org/ELearning/index.html}}
\end{itemize}
\end{frame}

\begin{frame}
\frametitle{Also Part of CRT-AI Constraint Week}
\begin{itemize}
\item Annual one week course on CP and Optimization in Ireland
\item Part of national training program for PhD students in AI
\item \url{https://www.crt-ai.ie/}
\end{itemize}
\end{frame}



\begin{frame}
\frametitle{Constraint Programming - in a nutshell}
\begin{itemize}
\item Declarative description of problems with
\begin{itemize}
\item {\em Variables} which range over (finite) sets of values
\item {\em Constraints} over subsets of variables which restrict possible value combinations
\item A {\em solution} is a value assignment which satisfies all constraints
\end{itemize}

\item Constraint propagation/reasoning
\begin{itemize}
\item Removing inconsistent values for variables
\item Detect failure if constraint can not be satisfied
\item Interaction of constraints via shared variables
\item Incomplete
\end{itemize}

\item Search
\begin{itemize}
\item User controlled assignment of values to variables
\item Each step triggers constraint propagation 
\end{itemize}
\item Different domains require/allow different methods
\end{itemize}
\end{frame}

\begin{frame}
  \frametitle{Constraint Programming is Different}
  \begin{itemize}
  \item Declarative Programming
    \begin{itemize}
    \item Concentrate on what you want
      \item Not how to get there
      \item Program != Algorithm
      \item Program = Model
    \end{itemize}
    \item Applied to Combinatorial Problems
      \begin{itemize}
        \item No complete polynomial algorithms known (exist?)
        \item CP less ad-hoc than heuristics
        \item Models can evolve
  \end{itemize}
  \end{itemize}
  \end{frame}
    
\begin{frame}
  \frametitle{A Subtractive Process}
  \begin{textblock}{4}(8,-3)
    \includegraphics[width=4cm]{../introduction/images/stages}
  \end{textblock}
  \vfill
  \begin{quote}
    ``Oh, bosh, as Mr. Ruskin says. Sculpture, per se, is the simplest thing in the world. All you have to do is to take a big chunk of marble and a hammer and chisel, make up your mind what you are about to create and chip off all the marble you don't want.''-Paris Gaulois.
  \end{quote}
  
  {\tiny Source: \url{https://quoteinvestigator.com/2014/06/22/chip-away/}}
\end{frame}

\begin{frame}
\frametitle{Basic Process}
\begin{center}
\begin{tikzpicture}
\node[shape=rectangle,fill=pantone127-4] (problem) at (2,5) {Problem};
\node (human) at (2,4) {Human};
\node[shape=rectangle,fill=pantone127-4] (model) at (2,3) {Model};
\node (solver) at (2,2) {Constraint Solver/Search};
\node[shape=rectangle,fill=pantone157-8] (solution) at (2,1) {Solution};
\draw[-] (problem) -- (human);
\draw[->] (human) -- (model);
\draw[-] (model) -- (solver);
\draw[->] (solver) -- (solution);
\end{tikzpicture}
\end{center}
\end{frame}

\begin{frame}
\frametitle{More Realistic}
\begin{center}
\begin{tikzpicture}[xscale=4.0]
\node[shape=rectangle,fill=pantone127-4] (problem) at (2,5) {Problem};
\node (human) at (2,4) {Human};
\node[shape=rectangle,fill=pantone127-4] (model) at (2,3) {Model};
\node (solver) at (2,2) {Constraint Solver/Search};
\node[shape=rectangle,fill=pantone24-8] (hangs) at (1,1) {Hangs};
\node[shape=rectangle,fill=pantone157-8] (solution) at (2,1) {Solution};
\node[shape=rectangle,fill=pantone24-8] (wrong) at (3,1) {Wrong Solution};
\draw[-] (problem) -- (human);
\draw[->] (human) -- (model);
\draw[-] (model) -- (solver);
\draw[->] (solver) -- (hangs);
\draw[->] (solver) -- (solution);
\draw[->] (solver) -- (wrong);
\draw[->] (hangs) -- (1,4) -- (human);
\draw[->] (wrong) -- (3,4) -- (human);
\end{tikzpicture}
\end{center}
\end{frame}

\begin{frame}
\frametitle{Dual Role of Model}
\begin{itemize}
\item Allows Human to Express Problem
\begin{itemize}
\item Close to Problem Domain
\item Constraints as Abstractions
\end{itemize}
\item Allows Solver to Execute
\begin{itemize}
\item Variables as Communication Mechanism
\item Constraints as Algorithms
\end{itemize}
\end{itemize}
\end{frame}

\begin{frame}
\frametitle{Modelling Frameworks}
\begin{itemize}
\item \href{https://www.minizinc.org/}{MiniZinc} (NICTA, Monash University, Australia)
\item \href{https://github.com/eomahony/Numberjack}{NumberJack} (Insight, Ireland) 
\item \href{https://www-users.york.ac.uk/peter.nightingale/savilerow//}{EssencePrime/SavilleRow} (UK) 
\item \href{https://cpmpy.readthedocs.io/en/latest/}{CPMpy} (KU Leuven)
\item Allow use of multiple back-end solvers
\item Compile model into variants for each solver
\item A priori solver independent model(CP, MIP, SAT)
\end{itemize}
\end{frame} 

\begin{frame}
\frametitle{Framework Process}
\begin{center}
\begin{tikzpicture}[xscale=2]
\node[shape=rectangle,fill=pantone127-4] (problem) at (2.5,5) {Problem};
\node (human) at (2.5,4) {Human};
\node[shape=rectangle,fill=pantone127-4] (model) at (2.5,3) {Model};
\node (compile) at (2.5,2) {Compile/Reformulate};
\node (solver1) at (1,1) {CP};
\node (solver2) at (2,1) {MIP};
\node (solver3) at (3,1) {SAT};
\node (solver4) at (4,1) {Other};
\node[shape=rectangle,fill=pantone157-8] (solution1) at (1,0) {Solution};
\node[shape=rectangle,fill=pantone157-8] (solution2) at (2,0) {Solution};
\node[shape=rectangle,fill=pantone157-8] (solution3) at (3,0) {Solution};
\node[shape=rectangle,fill=pantone157-8] (solution4) at (4,0) {Solution};
\draw[-] (problem) -- (human);
\draw[->] (human) -- (model);
\draw[-] (model) -- (compile);
\draw[-] (compile) -- (solver1);
\draw[-] (compile) -- (solver2);
\draw[-] (compile) -- (solver3);
\draw[-] (compile) -- (solver4);
\draw[->] (solver1) -- (solution1);
\draw[->] (solver2) -- (solution2);
\draw[->] (solver3) -- (solution3);
\draw[->] (solver4) -- (solution4);
\end{tikzpicture}
\end{center}
\end{frame}

% \begin{frame}
  % \frametitle{Do It Now!}
  % \begin{itemize}
  % \item Download and install Minizinc
  % \item \url{https://www.minizinc.org/}
  % \end{itemize}
% \end{frame}


\begin{frame}
\frametitle{Why use Puzzles as Examples?}
\begin{itemize}
\item Easy to understand the problem
\item Solvable by hand without specialized knowledge
\item Possible to compare automated to manual solving process
\end{itemize}

\begin{quote}
The puzzle, though inanimate, is presented as a solvable problem without lasting negative consequences, a very low-risk low-reward situation. By being a puzzle, the object is attempting to convince the user that it must be completed.
\end{quote}
{\tiny Source: \href{https://www.everydayrhetoric.com/2020/03/the-rhetoric-of-puzzles/}{Every Day Rhetoric}}
\end{frame}


\mode<all>{
\part{Basic Constraint Propagation}
}
\begin{frame}
\partpage
\end{frame}
\mode<all>{
\input{../sendmore/body}
}

\mode<all>{
\part{Global Constraints}
}
\begin{frame}
\partpage
\end{frame}
\mode<all>{
\input{../sudoku/body}
}

\clearpage
\mode<all>{
\part{Customizing Search}
}
\begin{frame}
\partpage
\end{frame}
\mode<all>{
\input{../nqueen/body}
}


%% \clearpage
%% \mode<all>{
%%   \part{Problem Decomposition}
%% }
%% \begin{frame}
%%   \partpage
%% \end{frame}
%% \mode<all>{
%%   \input{../party/body}
%%   }

%% \clearpage
%% \mode<all>{
%%   \part{More Global Constraints}
%% }
%% \begin{frame}
%%   \partpage
%% \end{frame}
%% \mode<all>{
%%   \input{../car/body}
%%   }

%% \clearpage
%% \mode<all>{
%%   \part{Methodology}
%% }
%% \begin{frame}
%%   \partpage
%% \end{frame}
%% \mode<all>{
%%   \input{../methodology/body}
%%   }

\clearpage
\mode<all>{
\part{What is missing?}
}
\begin{frame}
\partpage
\end{frame}

\begin{frame}
\frametitle{Many Specialized Topics}
\begin{itemize}
\item How to design efficient core engine
\item Hybrids with LP/MIP tools
\item Hybrids with SAT
\item Symmetry breaking
\item Use of MDD/BDD to encode sets of solutions
\item High level modelling tools
\item Debugging/visualization
\end{itemize}
\end{frame}

\begin{frame}
\frametitle{Reformulation}
\begin{itemize}
\item Just because the user has modelled it this way, it doesn't mean we have to solve it that way
\begin{itemize}
\item Replace some constraint(s) by other, equivalent constraints
\item Because we don't have that constraint in our system
\item For performance
\end{itemize}
\end{itemize}
\end{frame}

\begin{frame}
\frametitle{Learning}
\begin{itemize}
\item While solving the problem we can learn how to strengthen the model/search 
\begin{itemize}
\item Understand which constraints/method contribute to propagation and change schedule
\item Learn no-good constraints by explaining failure
\item Adapt search strategy based on search experience
\end{itemize}
\end{itemize}
\end{frame}

%% \begin{frame}
%% \frametitle{Refined Process}
%% \begin{center}
%% \begin{tikzpicture}
%% \node[shape=rectangle,fill=pantone127-4] (problem) at (3,7) {Problem};
%% \node (human) at (3,6) {Human};
%% \node[shape=rectangle,fill=pantone127-4] (model) at (3,5) {User Model};
%% \node (reformulation) at (3,4) {Reformulation};
%% \node[shape=rectangle,fill=pantone127-4] (implementation) at (3,3) {Implementation Model};
%% \node (solver) at (3,2) {Constraint Solver/Search};
%% \node[shape=rectangle,fill=pantone157-8] (solution) at (3,1) {Solution};
%% \draw[-] (problem) -- (human);
%% \draw[->] (human) -- (model);
%% \draw[-] (model) -- (reformulation);
%% \draw[->] (reformulation) -- (implementation);
%% \draw[-] (implementation) -- (solver);
%% \draw[->] (solver) -- (0,2) -- (0,3) -- (implementation);
%% \draw[->] (solver) -- (solution);
%% \end{tikzpicture}
%% \end{center}
%% \end{frame}

%\begin{frame}
%\frametitle{What is CP actually used for?}
%\url{http://hsimonis.wordpress.com}
%\includegraphics[width=10cm]{../introduction/images/blog}
%\end{frame}

%% \begin{frame}
%% \frametitle{Exercises}
%% \begin{itemize}
%% \item Install Constraint Solver of your Choice
%% \item MiniZinc (MiningZinc) will be used in later sessions
%% \item Run Sendmoremoney, Sudoku, Queens examples
%% \item Can you choose which propagator to use for alldifferent?
%% \item Can you use different search strategies?
%% \end{itemize}
%% \end{frame}

\begin{frame}
\frametitle{More Learning Resources}
\begin{itemize}
\item Survey of Methods, Resources, and Formats for Teaching Constraint Programming
\begin{itemize}
\item by Tejas Santanam, Helmut Simonis
\item \url{https://doi.org/10.48550/arXiv.2403.12717}
\item Based on survey of community for WTCP 2023
\item \url{https://hsimonis.github.io/WTCP2023/}
\end{itemize}
\end{itemize}
\end{frame}


%%   }

\clearpage
\mode<all>{
\part{What is missing?}
}
\begin{frame}
\partpage
\end{frame}

\begin{frame}
\frametitle{Many Specialized Topics}
\begin{itemize}
\item How to design efficient core engine
\item Hybrids with LP/MIP tools
\item Hybrids with SAT
\item Symmetry breaking
\item Use of MDD/BDD to encode sets of solutions
\item High level modelling tools
\item Debugging/visualization
\end{itemize}
\end{frame}

\begin{frame}
\frametitle{Reformulation}
\begin{itemize}
\item Just because the user has modelled it this way, it doesn't mean we have to solve it that way
\begin{itemize}
\item Replace some constraint(s) by other, equivalent constraints
\item Because we don't have that constraint in our system
\item For performance
\end{itemize}
\end{itemize}
\end{frame}

\begin{frame}
\frametitle{Learning}
\begin{itemize}
\item While solving the problem we can learn how to strengthen the model/search 
\begin{itemize}
\item Understand which constraints/method contribute to propagation and change schedule
\item Learn no-good constraints by explaining failure
\item Adapt search strategy based on search experience
\end{itemize}
\end{itemize}
\end{frame}

%% \begin{frame}
%% \frametitle{Refined Process}
%% \begin{center}
%% \begin{tikzpicture}
%% \node[shape=rectangle,fill=pantone127-4] (problem) at (3,7) {Problem};
%% \node (human) at (3,6) {Human};
%% \node[shape=rectangle,fill=pantone127-4] (model) at (3,5) {User Model};
%% \node (reformulation) at (3,4) {Reformulation};
%% \node[shape=rectangle,fill=pantone127-4] (implementation) at (3,3) {Implementation Model};
%% \node (solver) at (3,2) {Constraint Solver/Search};
%% \node[shape=rectangle,fill=pantone157-8] (solution) at (3,1) {Solution};
%% \draw[-] (problem) -- (human);
%% \draw[->] (human) -- (model);
%% \draw[-] (model) -- (reformulation);
%% \draw[->] (reformulation) -- (implementation);
%% \draw[-] (implementation) -- (solver);
%% \draw[->] (solver) -- (0,2) -- (0,3) -- (implementation);
%% \draw[->] (solver) -- (solution);
%% \end{tikzpicture}
%% \end{center}
%% \end{frame}

%\begin{frame}
%\frametitle{What is CP actually used for?}
%\url{http://hsimonis.wordpress.com}
%\includegraphics[width=10cm]{../introduction/images/blog}
%\end{frame}

%% \begin{frame}
%% \frametitle{Exercises}
%% \begin{itemize}
%% \item Install Constraint Solver of your Choice
%% \item MiniZinc (MiningZinc) will be used in later sessions
%% \item Run Sendmoremoney, Sudoku, Queens examples
%% \item Can you choose which propagator to use for alldifferent?
%% \item Can you use different search strategies?
%% \end{itemize}
%% \end{frame}

\begin{frame}
\frametitle{More Learning Resources}
\begin{itemize}
\item Survey of Methods, Resources, and Formats for Teaching Constraint Programming
\begin{itemize}
\item by Tejas Santanam, Helmut Simonis
\item \url{https://doi.org/10.48550/arXiv.2403.12717}
\item Based on survey of community for WTCP 2023
\item \url{https://hsimonis.github.io/WTCP2023/}
\end{itemize}
\end{itemize}
\end{frame}


%%   }

%% \clearpage
%% \mode<all>{
%%   \part{Methodology}
%% }
%% \begin{frame}
%%   \partpage
%% \end{frame}
%% \mode<all>{
%%   \begin{frame}
\frametitle{Objectives}
\begin{itemize}
\item Overview of Core Constraint Programming
\item Three Main Concepts
\begin{itemize}
\item Constraint Propagation
\item Global Constraints
\item Customizing Search
\end{itemize}
%\item Get Some Experience with MiniZinc
\item Topics will be treated in more detail in later parts of the school
\item Based on Examples, not Formal Description
\end{itemize}
\end{frame}

\begin{frame}
\frametitle{Outline}
\begin{itemize}
\item Why Constraint Programming?
\item Constraint Propagation
\item Global Constraints
\item Customizing Search
%\item Choices in Modelling
%\item Decomposition
%\item More Global Constraints
%\item Methodology
%\item What is missing?
\end{itemize}
\end{frame}

% \begin{frame}
  % \frametitle{Using MiniZinc IDE}
  % \begin{itemize}
  % \item Developed in the Australian NICTA project
  % \item Maintained by Monash University
  % \item Modelling tool with multiple back-end solvers
  % \item Available from \url{https://www.minizinc.org/}
  % \end{itemize}
% \end{frame}


% \begin{frame}
% \frametitle{Examples in ECLiPSe}
% \begin{itemize}
% \item Open sourced constraint programming language
% \item Development goes back to 1985
% \item ECRC, ICL, IC-Parc, PTL, Cisco
% \item \url{https://eclipseclp.org/}
% \item Specialities
% \begin{itemize}
% \item Develop new solvers for specific domains
% \item Integration with MIP
% \end{itemize}
% \item Not included in bundled MiniZinc IDE
% \item Specialized visualization tools used here
  % \begin{itemize}
  % \item CP-Viz, Simonis et al. 2010
    % \end{itemize}
% \end{itemize}
% \end{frame}

\begin{frame}
\frametitle{Tutorial Based on ECLiPSe ELearning Course}
\begin{itemize}
\item Self-study course in constraint programming
\item Supported by Cisco Systems and Silicon Valley Community Foundation
\item Multi-media format, video lectures, slides, handout etc
\item {\small \url{https://eclipseclp.org/ELearning/index.html}}
\end{itemize}
\end{frame}

\begin{frame}
\frametitle{Also Part of CRT-AI Constraint Week}
\begin{itemize}
\item Annual one week course on CP and Optimization in Ireland
\item Part of national training program for PhD students in AI
\item \url{https://www.crt-ai.ie/}
\end{itemize}
\end{frame}



\begin{frame}
\frametitle{Constraint Programming - in a nutshell}
\begin{itemize}
\item Declarative description of problems with
\begin{itemize}
\item {\em Variables} which range over (finite) sets of values
\item {\em Constraints} over subsets of variables which restrict possible value combinations
\item A {\em solution} is a value assignment which satisfies all constraints
\end{itemize}

\item Constraint propagation/reasoning
\begin{itemize}
\item Removing inconsistent values for variables
\item Detect failure if constraint can not be satisfied
\item Interaction of constraints via shared variables
\item Incomplete
\end{itemize}

\item Search
\begin{itemize}
\item User controlled assignment of values to variables
\item Each step triggers constraint propagation 
\end{itemize}
\item Different domains require/allow different methods
\end{itemize}
\end{frame}

\begin{frame}
  \frametitle{Constraint Programming is Different}
  \begin{itemize}
  \item Declarative Programming
    \begin{itemize}
    \item Concentrate on what you want
      \item Not how to get there
      \item Program != Algorithm
      \item Program = Model
    \end{itemize}
    \item Applied to Combinatorial Problems
      \begin{itemize}
        \item No complete polynomial algorithms known (exist?)
        \item CP less ad-hoc than heuristics
        \item Models can evolve
  \end{itemize}
  \end{itemize}
  \end{frame}
    
\begin{frame}
  \frametitle{A Subtractive Process}
  \begin{textblock}{4}(8,-3)
    \includegraphics[width=4cm]{../introduction/images/stages}
  \end{textblock}
  \vfill
  \begin{quote}
    ``Oh, bosh, as Mr. Ruskin says. Sculpture, per se, is the simplest thing in the world. All you have to do is to take a big chunk of marble and a hammer and chisel, make up your mind what you are about to create and chip off all the marble you don't want.''-Paris Gaulois.
  \end{quote}
  
  {\tiny Source: \url{https://quoteinvestigator.com/2014/06/22/chip-away/}}
\end{frame}

\begin{frame}
\frametitle{Basic Process}
\begin{center}
\begin{tikzpicture}
\node[shape=rectangle,fill=pantone127-4] (problem) at (2,5) {Problem};
\node (human) at (2,4) {Human};
\node[shape=rectangle,fill=pantone127-4] (model) at (2,3) {Model};
\node (solver) at (2,2) {Constraint Solver/Search};
\node[shape=rectangle,fill=pantone157-8] (solution) at (2,1) {Solution};
\draw[-] (problem) -- (human);
\draw[->] (human) -- (model);
\draw[-] (model) -- (solver);
\draw[->] (solver) -- (solution);
\end{tikzpicture}
\end{center}
\end{frame}

\begin{frame}
\frametitle{More Realistic}
\begin{center}
\begin{tikzpicture}[xscale=4.0]
\node[shape=rectangle,fill=pantone127-4] (problem) at (2,5) {Problem};
\node (human) at (2,4) {Human};
\node[shape=rectangle,fill=pantone127-4] (model) at (2,3) {Model};
\node (solver) at (2,2) {Constraint Solver/Search};
\node[shape=rectangle,fill=pantone24-8] (hangs) at (1,1) {Hangs};
\node[shape=rectangle,fill=pantone157-8] (solution) at (2,1) {Solution};
\node[shape=rectangle,fill=pantone24-8] (wrong) at (3,1) {Wrong Solution};
\draw[-] (problem) -- (human);
\draw[->] (human) -- (model);
\draw[-] (model) -- (solver);
\draw[->] (solver) -- (hangs);
\draw[->] (solver) -- (solution);
\draw[->] (solver) -- (wrong);
\draw[->] (hangs) -- (1,4) -- (human);
\draw[->] (wrong) -- (3,4) -- (human);
\end{tikzpicture}
\end{center}
\end{frame}

\begin{frame}
\frametitle{Dual Role of Model}
\begin{itemize}
\item Allows Human to Express Problem
\begin{itemize}
\item Close to Problem Domain
\item Constraints as Abstractions
\end{itemize}
\item Allows Solver to Execute
\begin{itemize}
\item Variables as Communication Mechanism
\item Constraints as Algorithms
\end{itemize}
\end{itemize}
\end{frame}

\begin{frame}
\frametitle{Modelling Frameworks}
\begin{itemize}
\item \href{https://www.minizinc.org/}{MiniZinc} (NICTA, Monash University, Australia)
\item \href{https://github.com/eomahony/Numberjack}{NumberJack} (Insight, Ireland) 
\item \href{https://www-users.york.ac.uk/peter.nightingale/savilerow//}{EssencePrime/SavilleRow} (UK) 
\item \href{https://cpmpy.readthedocs.io/en/latest/}{CPMpy} (KU Leuven)
\item Allow use of multiple back-end solvers
\item Compile model into variants for each solver
\item A priori solver independent model(CP, MIP, SAT)
\end{itemize}
\end{frame} 

\begin{frame}
\frametitle{Framework Process}
\begin{center}
\begin{tikzpicture}[xscale=2]
\node[shape=rectangle,fill=pantone127-4] (problem) at (2.5,5) {Problem};
\node (human) at (2.5,4) {Human};
\node[shape=rectangle,fill=pantone127-4] (model) at (2.5,3) {Model};
\node (compile) at (2.5,2) {Compile/Reformulate};
\node (solver1) at (1,1) {CP};
\node (solver2) at (2,1) {MIP};
\node (solver3) at (3,1) {SAT};
\node (solver4) at (4,1) {Other};
\node[shape=rectangle,fill=pantone157-8] (solution1) at (1,0) {Solution};
\node[shape=rectangle,fill=pantone157-8] (solution2) at (2,0) {Solution};
\node[shape=rectangle,fill=pantone157-8] (solution3) at (3,0) {Solution};
\node[shape=rectangle,fill=pantone157-8] (solution4) at (4,0) {Solution};
\draw[-] (problem) -- (human);
\draw[->] (human) -- (model);
\draw[-] (model) -- (compile);
\draw[-] (compile) -- (solver1);
\draw[-] (compile) -- (solver2);
\draw[-] (compile) -- (solver3);
\draw[-] (compile) -- (solver4);
\draw[->] (solver1) -- (solution1);
\draw[->] (solver2) -- (solution2);
\draw[->] (solver3) -- (solution3);
\draw[->] (solver4) -- (solution4);
\end{tikzpicture}
\end{center}
\end{frame}

% \begin{frame}
  % \frametitle{Do It Now!}
  % \begin{itemize}
  % \item Download and install Minizinc
  % \item \url{https://www.minizinc.org/}
  % \end{itemize}
% \end{frame}


\begin{frame}
\frametitle{Why use Puzzles as Examples?}
\begin{itemize}
\item Easy to understand the problem
\item Solvable by hand without specialized knowledge
\item Possible to compare automated to manual solving process
\end{itemize}

\begin{quote}
The puzzle, though inanimate, is presented as a solvable problem without lasting negative consequences, a very low-risk low-reward situation. By being a puzzle, the object is attempting to convince the user that it must be completed.
\end{quote}
{\tiny Source: \href{https://www.everydayrhetoric.com/2020/03/the-rhetoric-of-puzzles/}{Every Day Rhetoric}}
\end{frame}


\mode<all>{
\part{Basic Constraint Propagation}
}
\begin{frame}
\partpage
\end{frame}
\mode<all>{
\begin{frame}
\frametitle{Objectives}
\begin{itemize}
\item Overview of Core Constraint Programming
\item Three Main Concepts
\begin{itemize}
\item Constraint Propagation
\item Global Constraints
\item Customizing Search
\end{itemize}
%\item Get Some Experience with MiniZinc
\item Topics will be treated in more detail in later parts of the school
\item Based on Examples, not Formal Description
\end{itemize}
\end{frame}

\begin{frame}
\frametitle{Outline}
\begin{itemize}
\item Why Constraint Programming?
\item Constraint Propagation
\item Global Constraints
\item Customizing Search
%\item Choices in Modelling
%\item Decomposition
%\item More Global Constraints
%\item Methodology
%\item What is missing?
\end{itemize}
\end{frame}

% \begin{frame}
  % \frametitle{Using MiniZinc IDE}
  % \begin{itemize}
  % \item Developed in the Australian NICTA project
  % \item Maintained by Monash University
  % \item Modelling tool with multiple back-end solvers
  % \item Available from \url{https://www.minizinc.org/}
  % \end{itemize}
% \end{frame}


% \begin{frame}
% \frametitle{Examples in ECLiPSe}
% \begin{itemize}
% \item Open sourced constraint programming language
% \item Development goes back to 1985
% \item ECRC, ICL, IC-Parc, PTL, Cisco
% \item \url{https://eclipseclp.org/}
% \item Specialities
% \begin{itemize}
% \item Develop new solvers for specific domains
% \item Integration with MIP
% \end{itemize}
% \item Not included in bundled MiniZinc IDE
% \item Specialized visualization tools used here
  % \begin{itemize}
  % \item CP-Viz, Simonis et al. 2010
    % \end{itemize}
% \end{itemize}
% \end{frame}

\begin{frame}
\frametitle{Tutorial Based on ECLiPSe ELearning Course}
\begin{itemize}
\item Self-study course in constraint programming
\item Supported by Cisco Systems and Silicon Valley Community Foundation
\item Multi-media format, video lectures, slides, handout etc
\item {\small \url{https://eclipseclp.org/ELearning/index.html}}
\end{itemize}
\end{frame}

\begin{frame}
\frametitle{Also Part of CRT-AI Constraint Week}
\begin{itemize}
\item Annual one week course on CP and Optimization in Ireland
\item Part of national training program for PhD students in AI
\item \url{https://www.crt-ai.ie/}
\end{itemize}
\end{frame}



\begin{frame}
\frametitle{Constraint Programming - in a nutshell}
\begin{itemize}
\item Declarative description of problems with
\begin{itemize}
\item {\em Variables} which range over (finite) sets of values
\item {\em Constraints} over subsets of variables which restrict possible value combinations
\item A {\em solution} is a value assignment which satisfies all constraints
\end{itemize}

\item Constraint propagation/reasoning
\begin{itemize}
\item Removing inconsistent values for variables
\item Detect failure if constraint can not be satisfied
\item Interaction of constraints via shared variables
\item Incomplete
\end{itemize}

\item Search
\begin{itemize}
\item User controlled assignment of values to variables
\item Each step triggers constraint propagation 
\end{itemize}
\item Different domains require/allow different methods
\end{itemize}
\end{frame}

\begin{frame}
  \frametitle{Constraint Programming is Different}
  \begin{itemize}
  \item Declarative Programming
    \begin{itemize}
    \item Concentrate on what you want
      \item Not how to get there
      \item Program != Algorithm
      \item Program = Model
    \end{itemize}
    \item Applied to Combinatorial Problems
      \begin{itemize}
        \item No complete polynomial algorithms known (exist?)
        \item CP less ad-hoc than heuristics
        \item Models can evolve
  \end{itemize}
  \end{itemize}
  \end{frame}
    
\begin{frame}
  \frametitle{A Subtractive Process}
  \begin{textblock}{4}(8,-3)
    \includegraphics[width=4cm]{../introduction/images/stages}
  \end{textblock}
  \vfill
  \begin{quote}
    ``Oh, bosh, as Mr. Ruskin says. Sculpture, per se, is the simplest thing in the world. All you have to do is to take a big chunk of marble and a hammer and chisel, make up your mind what you are about to create and chip off all the marble you don't want.''-Paris Gaulois.
  \end{quote}
  
  {\tiny Source: \url{https://quoteinvestigator.com/2014/06/22/chip-away/}}
\end{frame}

\begin{frame}
\frametitle{Basic Process}
\begin{center}
\begin{tikzpicture}
\node[shape=rectangle,fill=pantone127-4] (problem) at (2,5) {Problem};
\node (human) at (2,4) {Human};
\node[shape=rectangle,fill=pantone127-4] (model) at (2,3) {Model};
\node (solver) at (2,2) {Constraint Solver/Search};
\node[shape=rectangle,fill=pantone157-8] (solution) at (2,1) {Solution};
\draw[-] (problem) -- (human);
\draw[->] (human) -- (model);
\draw[-] (model) -- (solver);
\draw[->] (solver) -- (solution);
\end{tikzpicture}
\end{center}
\end{frame}

\begin{frame}
\frametitle{More Realistic}
\begin{center}
\begin{tikzpicture}[xscale=4.0]
\node[shape=rectangle,fill=pantone127-4] (problem) at (2,5) {Problem};
\node (human) at (2,4) {Human};
\node[shape=rectangle,fill=pantone127-4] (model) at (2,3) {Model};
\node (solver) at (2,2) {Constraint Solver/Search};
\node[shape=rectangle,fill=pantone24-8] (hangs) at (1,1) {Hangs};
\node[shape=rectangle,fill=pantone157-8] (solution) at (2,1) {Solution};
\node[shape=rectangle,fill=pantone24-8] (wrong) at (3,1) {Wrong Solution};
\draw[-] (problem) -- (human);
\draw[->] (human) -- (model);
\draw[-] (model) -- (solver);
\draw[->] (solver) -- (hangs);
\draw[->] (solver) -- (solution);
\draw[->] (solver) -- (wrong);
\draw[->] (hangs) -- (1,4) -- (human);
\draw[->] (wrong) -- (3,4) -- (human);
\end{tikzpicture}
\end{center}
\end{frame}

\begin{frame}
\frametitle{Dual Role of Model}
\begin{itemize}
\item Allows Human to Express Problem
\begin{itemize}
\item Close to Problem Domain
\item Constraints as Abstractions
\end{itemize}
\item Allows Solver to Execute
\begin{itemize}
\item Variables as Communication Mechanism
\item Constraints as Algorithms
\end{itemize}
\end{itemize}
\end{frame}

\begin{frame}
\frametitle{Modelling Frameworks}
\begin{itemize}
\item \href{https://www.minizinc.org/}{MiniZinc} (NICTA, Monash University, Australia)
\item \href{https://github.com/eomahony/Numberjack}{NumberJack} (Insight, Ireland) 
\item \href{https://www-users.york.ac.uk/peter.nightingale/savilerow//}{EssencePrime/SavilleRow} (UK) 
\item \href{https://cpmpy.readthedocs.io/en/latest/}{CPMpy} (KU Leuven)
\item Allow use of multiple back-end solvers
\item Compile model into variants for each solver
\item A priori solver independent model(CP, MIP, SAT)
\end{itemize}
\end{frame} 

\begin{frame}
\frametitle{Framework Process}
\begin{center}
\begin{tikzpicture}[xscale=2]
\node[shape=rectangle,fill=pantone127-4] (problem) at (2.5,5) {Problem};
\node (human) at (2.5,4) {Human};
\node[shape=rectangle,fill=pantone127-4] (model) at (2.5,3) {Model};
\node (compile) at (2.5,2) {Compile/Reformulate};
\node (solver1) at (1,1) {CP};
\node (solver2) at (2,1) {MIP};
\node (solver3) at (3,1) {SAT};
\node (solver4) at (4,1) {Other};
\node[shape=rectangle,fill=pantone157-8] (solution1) at (1,0) {Solution};
\node[shape=rectangle,fill=pantone157-8] (solution2) at (2,0) {Solution};
\node[shape=rectangle,fill=pantone157-8] (solution3) at (3,0) {Solution};
\node[shape=rectangle,fill=pantone157-8] (solution4) at (4,0) {Solution};
\draw[-] (problem) -- (human);
\draw[->] (human) -- (model);
\draw[-] (model) -- (compile);
\draw[-] (compile) -- (solver1);
\draw[-] (compile) -- (solver2);
\draw[-] (compile) -- (solver3);
\draw[-] (compile) -- (solver4);
\draw[->] (solver1) -- (solution1);
\draw[->] (solver2) -- (solution2);
\draw[->] (solver3) -- (solution3);
\draw[->] (solver4) -- (solution4);
\end{tikzpicture}
\end{center}
\end{frame}

% \begin{frame}
  % \frametitle{Do It Now!}
  % \begin{itemize}
  % \item Download and install Minizinc
  % \item \url{https://www.minizinc.org/}
  % \end{itemize}
% \end{frame}


\begin{frame}
\frametitle{Why use Puzzles as Examples?}
\begin{itemize}
\item Easy to understand the problem
\item Solvable by hand without specialized knowledge
\item Possible to compare automated to manual solving process
\end{itemize}

\begin{quote}
The puzzle, though inanimate, is presented as a solvable problem without lasting negative consequences, a very low-risk low-reward situation. By being a puzzle, the object is attempting to convince the user that it must be completed.
\end{quote}
{\tiny Source: \href{https://www.everydayrhetoric.com/2020/03/the-rhetoric-of-puzzles/}{Every Day Rhetoric}}
\end{frame}


\mode<all>{
\part{Basic Constraint Propagation}
}
\begin{frame}
\partpage
\end{frame}
\mode<all>{
\input{../sendmore/body}
}

\mode<all>{
\part{Global Constraints}
}
\begin{frame}
\partpage
\end{frame}
\mode<all>{
\input{../sudoku/body}
}

\clearpage
\mode<all>{
\part{Customizing Search}
}
\begin{frame}
\partpage
\end{frame}
\mode<all>{
\input{../nqueen/body}
}


%% \clearpage
%% \mode<all>{
%%   \part{Problem Decomposition}
%% }
%% \begin{frame}
%%   \partpage
%% \end{frame}
%% \mode<all>{
%%   \input{../party/body}
%%   }

%% \clearpage
%% \mode<all>{
%%   \part{More Global Constraints}
%% }
%% \begin{frame}
%%   \partpage
%% \end{frame}
%% \mode<all>{
%%   \input{../car/body}
%%   }

%% \clearpage
%% \mode<all>{
%%   \part{Methodology}
%% }
%% \begin{frame}
%%   \partpage
%% \end{frame}
%% \mode<all>{
%%   \input{../methodology/body}
%%   }

\clearpage
\mode<all>{
\part{What is missing?}
}
\begin{frame}
\partpage
\end{frame}

\begin{frame}
\frametitle{Many Specialized Topics}
\begin{itemize}
\item How to design efficient core engine
\item Hybrids with LP/MIP tools
\item Hybrids with SAT
\item Symmetry breaking
\item Use of MDD/BDD to encode sets of solutions
\item High level modelling tools
\item Debugging/visualization
\end{itemize}
\end{frame}

\begin{frame}
\frametitle{Reformulation}
\begin{itemize}
\item Just because the user has modelled it this way, it doesn't mean we have to solve it that way
\begin{itemize}
\item Replace some constraint(s) by other, equivalent constraints
\item Because we don't have that constraint in our system
\item For performance
\end{itemize}
\end{itemize}
\end{frame}

\begin{frame}
\frametitle{Learning}
\begin{itemize}
\item While solving the problem we can learn how to strengthen the model/search 
\begin{itemize}
\item Understand which constraints/method contribute to propagation and change schedule
\item Learn no-good constraints by explaining failure
\item Adapt search strategy based on search experience
\end{itemize}
\end{itemize}
\end{frame}

%% \begin{frame}
%% \frametitle{Refined Process}
%% \begin{center}
%% \begin{tikzpicture}
%% \node[shape=rectangle,fill=pantone127-4] (problem) at (3,7) {Problem};
%% \node (human) at (3,6) {Human};
%% \node[shape=rectangle,fill=pantone127-4] (model) at (3,5) {User Model};
%% \node (reformulation) at (3,4) {Reformulation};
%% \node[shape=rectangle,fill=pantone127-4] (implementation) at (3,3) {Implementation Model};
%% \node (solver) at (3,2) {Constraint Solver/Search};
%% \node[shape=rectangle,fill=pantone157-8] (solution) at (3,1) {Solution};
%% \draw[-] (problem) -- (human);
%% \draw[->] (human) -- (model);
%% \draw[-] (model) -- (reformulation);
%% \draw[->] (reformulation) -- (implementation);
%% \draw[-] (implementation) -- (solver);
%% \draw[->] (solver) -- (0,2) -- (0,3) -- (implementation);
%% \draw[->] (solver) -- (solution);
%% \end{tikzpicture}
%% \end{center}
%% \end{frame}

%\begin{frame}
%\frametitle{What is CP actually used for?}
%\url{http://hsimonis.wordpress.com}
%\includegraphics[width=10cm]{../introduction/images/blog}
%\end{frame}

%% \begin{frame}
%% \frametitle{Exercises}
%% \begin{itemize}
%% \item Install Constraint Solver of your Choice
%% \item MiniZinc (MiningZinc) will be used in later sessions
%% \item Run Sendmoremoney, Sudoku, Queens examples
%% \item Can you choose which propagator to use for alldifferent?
%% \item Can you use different search strategies?
%% \end{itemize}
%% \end{frame}

\begin{frame}
\frametitle{More Learning Resources}
\begin{itemize}
\item Survey of Methods, Resources, and Formats for Teaching Constraint Programming
\begin{itemize}
\item by Tejas Santanam, Helmut Simonis
\item \url{https://doi.org/10.48550/arXiv.2403.12717}
\item Based on survey of community for WTCP 2023
\item \url{https://hsimonis.github.io/WTCP2023/}
\end{itemize}
\end{itemize}
\end{frame}


}

\mode<all>{
\part{Global Constraints}
}
\begin{frame}
\partpage
\end{frame}
\mode<all>{
\begin{frame}
\frametitle{Objectives}
\begin{itemize}
\item Overview of Core Constraint Programming
\item Three Main Concepts
\begin{itemize}
\item Constraint Propagation
\item Global Constraints
\item Customizing Search
\end{itemize}
%\item Get Some Experience with MiniZinc
\item Topics will be treated in more detail in later parts of the school
\item Based on Examples, not Formal Description
\end{itemize}
\end{frame}

\begin{frame}
\frametitle{Outline}
\begin{itemize}
\item Why Constraint Programming?
\item Constraint Propagation
\item Global Constraints
\item Customizing Search
%\item Choices in Modelling
%\item Decomposition
%\item More Global Constraints
%\item Methodology
%\item What is missing?
\end{itemize}
\end{frame}

% \begin{frame}
  % \frametitle{Using MiniZinc IDE}
  % \begin{itemize}
  % \item Developed in the Australian NICTA project
  % \item Maintained by Monash University
  % \item Modelling tool with multiple back-end solvers
  % \item Available from \url{https://www.minizinc.org/}
  % \end{itemize}
% \end{frame}


% \begin{frame}
% \frametitle{Examples in ECLiPSe}
% \begin{itemize}
% \item Open sourced constraint programming language
% \item Development goes back to 1985
% \item ECRC, ICL, IC-Parc, PTL, Cisco
% \item \url{https://eclipseclp.org/}
% \item Specialities
% \begin{itemize}
% \item Develop new solvers for specific domains
% \item Integration with MIP
% \end{itemize}
% \item Not included in bundled MiniZinc IDE
% \item Specialized visualization tools used here
  % \begin{itemize}
  % \item CP-Viz, Simonis et al. 2010
    % \end{itemize}
% \end{itemize}
% \end{frame}

\begin{frame}
\frametitle{Tutorial Based on ECLiPSe ELearning Course}
\begin{itemize}
\item Self-study course in constraint programming
\item Supported by Cisco Systems and Silicon Valley Community Foundation
\item Multi-media format, video lectures, slides, handout etc
\item {\small \url{https://eclipseclp.org/ELearning/index.html}}
\end{itemize}
\end{frame}

\begin{frame}
\frametitle{Also Part of CRT-AI Constraint Week}
\begin{itemize}
\item Annual one week course on CP and Optimization in Ireland
\item Part of national training program for PhD students in AI
\item \url{https://www.crt-ai.ie/}
\end{itemize}
\end{frame}



\begin{frame}
\frametitle{Constraint Programming - in a nutshell}
\begin{itemize}
\item Declarative description of problems with
\begin{itemize}
\item {\em Variables} which range over (finite) sets of values
\item {\em Constraints} over subsets of variables which restrict possible value combinations
\item A {\em solution} is a value assignment which satisfies all constraints
\end{itemize}

\item Constraint propagation/reasoning
\begin{itemize}
\item Removing inconsistent values for variables
\item Detect failure if constraint can not be satisfied
\item Interaction of constraints via shared variables
\item Incomplete
\end{itemize}

\item Search
\begin{itemize}
\item User controlled assignment of values to variables
\item Each step triggers constraint propagation 
\end{itemize}
\item Different domains require/allow different methods
\end{itemize}
\end{frame}

\begin{frame}
  \frametitle{Constraint Programming is Different}
  \begin{itemize}
  \item Declarative Programming
    \begin{itemize}
    \item Concentrate on what you want
      \item Not how to get there
      \item Program != Algorithm
      \item Program = Model
    \end{itemize}
    \item Applied to Combinatorial Problems
      \begin{itemize}
        \item No complete polynomial algorithms known (exist?)
        \item CP less ad-hoc than heuristics
        \item Models can evolve
  \end{itemize}
  \end{itemize}
  \end{frame}
    
\begin{frame}
  \frametitle{A Subtractive Process}
  \begin{textblock}{4}(8,-3)
    \includegraphics[width=4cm]{../introduction/images/stages}
  \end{textblock}
  \vfill
  \begin{quote}
    ``Oh, bosh, as Mr. Ruskin says. Sculpture, per se, is the simplest thing in the world. All you have to do is to take a big chunk of marble and a hammer and chisel, make up your mind what you are about to create and chip off all the marble you don't want.''-Paris Gaulois.
  \end{quote}
  
  {\tiny Source: \url{https://quoteinvestigator.com/2014/06/22/chip-away/}}
\end{frame}

\begin{frame}
\frametitle{Basic Process}
\begin{center}
\begin{tikzpicture}
\node[shape=rectangle,fill=pantone127-4] (problem) at (2,5) {Problem};
\node (human) at (2,4) {Human};
\node[shape=rectangle,fill=pantone127-4] (model) at (2,3) {Model};
\node (solver) at (2,2) {Constraint Solver/Search};
\node[shape=rectangle,fill=pantone157-8] (solution) at (2,1) {Solution};
\draw[-] (problem) -- (human);
\draw[->] (human) -- (model);
\draw[-] (model) -- (solver);
\draw[->] (solver) -- (solution);
\end{tikzpicture}
\end{center}
\end{frame}

\begin{frame}
\frametitle{More Realistic}
\begin{center}
\begin{tikzpicture}[xscale=4.0]
\node[shape=rectangle,fill=pantone127-4] (problem) at (2,5) {Problem};
\node (human) at (2,4) {Human};
\node[shape=rectangle,fill=pantone127-4] (model) at (2,3) {Model};
\node (solver) at (2,2) {Constraint Solver/Search};
\node[shape=rectangle,fill=pantone24-8] (hangs) at (1,1) {Hangs};
\node[shape=rectangle,fill=pantone157-8] (solution) at (2,1) {Solution};
\node[shape=rectangle,fill=pantone24-8] (wrong) at (3,1) {Wrong Solution};
\draw[-] (problem) -- (human);
\draw[->] (human) -- (model);
\draw[-] (model) -- (solver);
\draw[->] (solver) -- (hangs);
\draw[->] (solver) -- (solution);
\draw[->] (solver) -- (wrong);
\draw[->] (hangs) -- (1,4) -- (human);
\draw[->] (wrong) -- (3,4) -- (human);
\end{tikzpicture}
\end{center}
\end{frame}

\begin{frame}
\frametitle{Dual Role of Model}
\begin{itemize}
\item Allows Human to Express Problem
\begin{itemize}
\item Close to Problem Domain
\item Constraints as Abstractions
\end{itemize}
\item Allows Solver to Execute
\begin{itemize}
\item Variables as Communication Mechanism
\item Constraints as Algorithms
\end{itemize}
\end{itemize}
\end{frame}

\begin{frame}
\frametitle{Modelling Frameworks}
\begin{itemize}
\item \href{https://www.minizinc.org/}{MiniZinc} (NICTA, Monash University, Australia)
\item \href{https://github.com/eomahony/Numberjack}{NumberJack} (Insight, Ireland) 
\item \href{https://www-users.york.ac.uk/peter.nightingale/savilerow//}{EssencePrime/SavilleRow} (UK) 
\item \href{https://cpmpy.readthedocs.io/en/latest/}{CPMpy} (KU Leuven)
\item Allow use of multiple back-end solvers
\item Compile model into variants for each solver
\item A priori solver independent model(CP, MIP, SAT)
\end{itemize}
\end{frame} 

\begin{frame}
\frametitle{Framework Process}
\begin{center}
\begin{tikzpicture}[xscale=2]
\node[shape=rectangle,fill=pantone127-4] (problem) at (2.5,5) {Problem};
\node (human) at (2.5,4) {Human};
\node[shape=rectangle,fill=pantone127-4] (model) at (2.5,3) {Model};
\node (compile) at (2.5,2) {Compile/Reformulate};
\node (solver1) at (1,1) {CP};
\node (solver2) at (2,1) {MIP};
\node (solver3) at (3,1) {SAT};
\node (solver4) at (4,1) {Other};
\node[shape=rectangle,fill=pantone157-8] (solution1) at (1,0) {Solution};
\node[shape=rectangle,fill=pantone157-8] (solution2) at (2,0) {Solution};
\node[shape=rectangle,fill=pantone157-8] (solution3) at (3,0) {Solution};
\node[shape=rectangle,fill=pantone157-8] (solution4) at (4,0) {Solution};
\draw[-] (problem) -- (human);
\draw[->] (human) -- (model);
\draw[-] (model) -- (compile);
\draw[-] (compile) -- (solver1);
\draw[-] (compile) -- (solver2);
\draw[-] (compile) -- (solver3);
\draw[-] (compile) -- (solver4);
\draw[->] (solver1) -- (solution1);
\draw[->] (solver2) -- (solution2);
\draw[->] (solver3) -- (solution3);
\draw[->] (solver4) -- (solution4);
\end{tikzpicture}
\end{center}
\end{frame}

% \begin{frame}
  % \frametitle{Do It Now!}
  % \begin{itemize}
  % \item Download and install Minizinc
  % \item \url{https://www.minizinc.org/}
  % \end{itemize}
% \end{frame}


\begin{frame}
\frametitle{Why use Puzzles as Examples?}
\begin{itemize}
\item Easy to understand the problem
\item Solvable by hand without specialized knowledge
\item Possible to compare automated to manual solving process
\end{itemize}

\begin{quote}
The puzzle, though inanimate, is presented as a solvable problem without lasting negative consequences, a very low-risk low-reward situation. By being a puzzle, the object is attempting to convince the user that it must be completed.
\end{quote}
{\tiny Source: \href{https://www.everydayrhetoric.com/2020/03/the-rhetoric-of-puzzles/}{Every Day Rhetoric}}
\end{frame}


\mode<all>{
\part{Basic Constraint Propagation}
}
\begin{frame}
\partpage
\end{frame}
\mode<all>{
\input{../sendmore/body}
}

\mode<all>{
\part{Global Constraints}
}
\begin{frame}
\partpage
\end{frame}
\mode<all>{
\input{../sudoku/body}
}

\clearpage
\mode<all>{
\part{Customizing Search}
}
\begin{frame}
\partpage
\end{frame}
\mode<all>{
\input{../nqueen/body}
}


%% \clearpage
%% \mode<all>{
%%   \part{Problem Decomposition}
%% }
%% \begin{frame}
%%   \partpage
%% \end{frame}
%% \mode<all>{
%%   \input{../party/body}
%%   }

%% \clearpage
%% \mode<all>{
%%   \part{More Global Constraints}
%% }
%% \begin{frame}
%%   \partpage
%% \end{frame}
%% \mode<all>{
%%   \input{../car/body}
%%   }

%% \clearpage
%% \mode<all>{
%%   \part{Methodology}
%% }
%% \begin{frame}
%%   \partpage
%% \end{frame}
%% \mode<all>{
%%   \input{../methodology/body}
%%   }

\clearpage
\mode<all>{
\part{What is missing?}
}
\begin{frame}
\partpage
\end{frame}

\begin{frame}
\frametitle{Many Specialized Topics}
\begin{itemize}
\item How to design efficient core engine
\item Hybrids with LP/MIP tools
\item Hybrids with SAT
\item Symmetry breaking
\item Use of MDD/BDD to encode sets of solutions
\item High level modelling tools
\item Debugging/visualization
\end{itemize}
\end{frame}

\begin{frame}
\frametitle{Reformulation}
\begin{itemize}
\item Just because the user has modelled it this way, it doesn't mean we have to solve it that way
\begin{itemize}
\item Replace some constraint(s) by other, equivalent constraints
\item Because we don't have that constraint in our system
\item For performance
\end{itemize}
\end{itemize}
\end{frame}

\begin{frame}
\frametitle{Learning}
\begin{itemize}
\item While solving the problem we can learn how to strengthen the model/search 
\begin{itemize}
\item Understand which constraints/method contribute to propagation and change schedule
\item Learn no-good constraints by explaining failure
\item Adapt search strategy based on search experience
\end{itemize}
\end{itemize}
\end{frame}

%% \begin{frame}
%% \frametitle{Refined Process}
%% \begin{center}
%% \begin{tikzpicture}
%% \node[shape=rectangle,fill=pantone127-4] (problem) at (3,7) {Problem};
%% \node (human) at (3,6) {Human};
%% \node[shape=rectangle,fill=pantone127-4] (model) at (3,5) {User Model};
%% \node (reformulation) at (3,4) {Reformulation};
%% \node[shape=rectangle,fill=pantone127-4] (implementation) at (3,3) {Implementation Model};
%% \node (solver) at (3,2) {Constraint Solver/Search};
%% \node[shape=rectangle,fill=pantone157-8] (solution) at (3,1) {Solution};
%% \draw[-] (problem) -- (human);
%% \draw[->] (human) -- (model);
%% \draw[-] (model) -- (reformulation);
%% \draw[->] (reformulation) -- (implementation);
%% \draw[-] (implementation) -- (solver);
%% \draw[->] (solver) -- (0,2) -- (0,3) -- (implementation);
%% \draw[->] (solver) -- (solution);
%% \end{tikzpicture}
%% \end{center}
%% \end{frame}

%\begin{frame}
%\frametitle{What is CP actually used for?}
%\url{http://hsimonis.wordpress.com}
%\includegraphics[width=10cm]{../introduction/images/blog}
%\end{frame}

%% \begin{frame}
%% \frametitle{Exercises}
%% \begin{itemize}
%% \item Install Constraint Solver of your Choice
%% \item MiniZinc (MiningZinc) will be used in later sessions
%% \item Run Sendmoremoney, Sudoku, Queens examples
%% \item Can you choose which propagator to use for alldifferent?
%% \item Can you use different search strategies?
%% \end{itemize}
%% \end{frame}

\begin{frame}
\frametitle{More Learning Resources}
\begin{itemize}
\item Survey of Methods, Resources, and Formats for Teaching Constraint Programming
\begin{itemize}
\item by Tejas Santanam, Helmut Simonis
\item \url{https://doi.org/10.48550/arXiv.2403.12717}
\item Based on survey of community for WTCP 2023
\item \url{https://hsimonis.github.io/WTCP2023/}
\end{itemize}
\end{itemize}
\end{frame}


}

\clearpage
\mode<all>{
\part{Customizing Search}
}
\begin{frame}
\partpage
\end{frame}
\mode<all>{
\begin{frame}
\frametitle{Objectives}
\begin{itemize}
\item Overview of Core Constraint Programming
\item Three Main Concepts
\begin{itemize}
\item Constraint Propagation
\item Global Constraints
\item Customizing Search
\end{itemize}
%\item Get Some Experience with MiniZinc
\item Topics will be treated in more detail in later parts of the school
\item Based on Examples, not Formal Description
\end{itemize}
\end{frame}

\begin{frame}
\frametitle{Outline}
\begin{itemize}
\item Why Constraint Programming?
\item Constraint Propagation
\item Global Constraints
\item Customizing Search
%\item Choices in Modelling
%\item Decomposition
%\item More Global Constraints
%\item Methodology
%\item What is missing?
\end{itemize}
\end{frame}

% \begin{frame}
  % \frametitle{Using MiniZinc IDE}
  % \begin{itemize}
  % \item Developed in the Australian NICTA project
  % \item Maintained by Monash University
  % \item Modelling tool with multiple back-end solvers
  % \item Available from \url{https://www.minizinc.org/}
  % \end{itemize}
% \end{frame}


% \begin{frame}
% \frametitle{Examples in ECLiPSe}
% \begin{itemize}
% \item Open sourced constraint programming language
% \item Development goes back to 1985
% \item ECRC, ICL, IC-Parc, PTL, Cisco
% \item \url{https://eclipseclp.org/}
% \item Specialities
% \begin{itemize}
% \item Develop new solvers for specific domains
% \item Integration with MIP
% \end{itemize}
% \item Not included in bundled MiniZinc IDE
% \item Specialized visualization tools used here
  % \begin{itemize}
  % \item CP-Viz, Simonis et al. 2010
    % \end{itemize}
% \end{itemize}
% \end{frame}

\begin{frame}
\frametitle{Tutorial Based on ECLiPSe ELearning Course}
\begin{itemize}
\item Self-study course in constraint programming
\item Supported by Cisco Systems and Silicon Valley Community Foundation
\item Multi-media format, video lectures, slides, handout etc
\item {\small \url{https://eclipseclp.org/ELearning/index.html}}
\end{itemize}
\end{frame}

\begin{frame}
\frametitle{Also Part of CRT-AI Constraint Week}
\begin{itemize}
\item Annual one week course on CP and Optimization in Ireland
\item Part of national training program for PhD students in AI
\item \url{https://www.crt-ai.ie/}
\end{itemize}
\end{frame}



\begin{frame}
\frametitle{Constraint Programming - in a nutshell}
\begin{itemize}
\item Declarative description of problems with
\begin{itemize}
\item {\em Variables} which range over (finite) sets of values
\item {\em Constraints} over subsets of variables which restrict possible value combinations
\item A {\em solution} is a value assignment which satisfies all constraints
\end{itemize}

\item Constraint propagation/reasoning
\begin{itemize}
\item Removing inconsistent values for variables
\item Detect failure if constraint can not be satisfied
\item Interaction of constraints via shared variables
\item Incomplete
\end{itemize}

\item Search
\begin{itemize}
\item User controlled assignment of values to variables
\item Each step triggers constraint propagation 
\end{itemize}
\item Different domains require/allow different methods
\end{itemize}
\end{frame}

\begin{frame}
  \frametitle{Constraint Programming is Different}
  \begin{itemize}
  \item Declarative Programming
    \begin{itemize}
    \item Concentrate on what you want
      \item Not how to get there
      \item Program != Algorithm
      \item Program = Model
    \end{itemize}
    \item Applied to Combinatorial Problems
      \begin{itemize}
        \item No complete polynomial algorithms known (exist?)
        \item CP less ad-hoc than heuristics
        \item Models can evolve
  \end{itemize}
  \end{itemize}
  \end{frame}
    
\begin{frame}
  \frametitle{A Subtractive Process}
  \begin{textblock}{4}(8,-3)
    \includegraphics[width=4cm]{../introduction/images/stages}
  \end{textblock}
  \vfill
  \begin{quote}
    ``Oh, bosh, as Mr. Ruskin says. Sculpture, per se, is the simplest thing in the world. All you have to do is to take a big chunk of marble and a hammer and chisel, make up your mind what you are about to create and chip off all the marble you don't want.''-Paris Gaulois.
  \end{quote}
  
  {\tiny Source: \url{https://quoteinvestigator.com/2014/06/22/chip-away/}}
\end{frame}

\begin{frame}
\frametitle{Basic Process}
\begin{center}
\begin{tikzpicture}
\node[shape=rectangle,fill=pantone127-4] (problem) at (2,5) {Problem};
\node (human) at (2,4) {Human};
\node[shape=rectangle,fill=pantone127-4] (model) at (2,3) {Model};
\node (solver) at (2,2) {Constraint Solver/Search};
\node[shape=rectangle,fill=pantone157-8] (solution) at (2,1) {Solution};
\draw[-] (problem) -- (human);
\draw[->] (human) -- (model);
\draw[-] (model) -- (solver);
\draw[->] (solver) -- (solution);
\end{tikzpicture}
\end{center}
\end{frame}

\begin{frame}
\frametitle{More Realistic}
\begin{center}
\begin{tikzpicture}[xscale=4.0]
\node[shape=rectangle,fill=pantone127-4] (problem) at (2,5) {Problem};
\node (human) at (2,4) {Human};
\node[shape=rectangle,fill=pantone127-4] (model) at (2,3) {Model};
\node (solver) at (2,2) {Constraint Solver/Search};
\node[shape=rectangle,fill=pantone24-8] (hangs) at (1,1) {Hangs};
\node[shape=rectangle,fill=pantone157-8] (solution) at (2,1) {Solution};
\node[shape=rectangle,fill=pantone24-8] (wrong) at (3,1) {Wrong Solution};
\draw[-] (problem) -- (human);
\draw[->] (human) -- (model);
\draw[-] (model) -- (solver);
\draw[->] (solver) -- (hangs);
\draw[->] (solver) -- (solution);
\draw[->] (solver) -- (wrong);
\draw[->] (hangs) -- (1,4) -- (human);
\draw[->] (wrong) -- (3,4) -- (human);
\end{tikzpicture}
\end{center}
\end{frame}

\begin{frame}
\frametitle{Dual Role of Model}
\begin{itemize}
\item Allows Human to Express Problem
\begin{itemize}
\item Close to Problem Domain
\item Constraints as Abstractions
\end{itemize}
\item Allows Solver to Execute
\begin{itemize}
\item Variables as Communication Mechanism
\item Constraints as Algorithms
\end{itemize}
\end{itemize}
\end{frame}

\begin{frame}
\frametitle{Modelling Frameworks}
\begin{itemize}
\item \href{https://www.minizinc.org/}{MiniZinc} (NICTA, Monash University, Australia)
\item \href{https://github.com/eomahony/Numberjack}{NumberJack} (Insight, Ireland) 
\item \href{https://www-users.york.ac.uk/peter.nightingale/savilerow//}{EssencePrime/SavilleRow} (UK) 
\item \href{https://cpmpy.readthedocs.io/en/latest/}{CPMpy} (KU Leuven)
\item Allow use of multiple back-end solvers
\item Compile model into variants for each solver
\item A priori solver independent model(CP, MIP, SAT)
\end{itemize}
\end{frame} 

\begin{frame}
\frametitle{Framework Process}
\begin{center}
\begin{tikzpicture}[xscale=2]
\node[shape=rectangle,fill=pantone127-4] (problem) at (2.5,5) {Problem};
\node (human) at (2.5,4) {Human};
\node[shape=rectangle,fill=pantone127-4] (model) at (2.5,3) {Model};
\node (compile) at (2.5,2) {Compile/Reformulate};
\node (solver1) at (1,1) {CP};
\node (solver2) at (2,1) {MIP};
\node (solver3) at (3,1) {SAT};
\node (solver4) at (4,1) {Other};
\node[shape=rectangle,fill=pantone157-8] (solution1) at (1,0) {Solution};
\node[shape=rectangle,fill=pantone157-8] (solution2) at (2,0) {Solution};
\node[shape=rectangle,fill=pantone157-8] (solution3) at (3,0) {Solution};
\node[shape=rectangle,fill=pantone157-8] (solution4) at (4,0) {Solution};
\draw[-] (problem) -- (human);
\draw[->] (human) -- (model);
\draw[-] (model) -- (compile);
\draw[-] (compile) -- (solver1);
\draw[-] (compile) -- (solver2);
\draw[-] (compile) -- (solver3);
\draw[-] (compile) -- (solver4);
\draw[->] (solver1) -- (solution1);
\draw[->] (solver2) -- (solution2);
\draw[->] (solver3) -- (solution3);
\draw[->] (solver4) -- (solution4);
\end{tikzpicture}
\end{center}
\end{frame}

% \begin{frame}
  % \frametitle{Do It Now!}
  % \begin{itemize}
  % \item Download and install Minizinc
  % \item \url{https://www.minizinc.org/}
  % \end{itemize}
% \end{frame}


\begin{frame}
\frametitle{Why use Puzzles as Examples?}
\begin{itemize}
\item Easy to understand the problem
\item Solvable by hand without specialized knowledge
\item Possible to compare automated to manual solving process
\end{itemize}

\begin{quote}
The puzzle, though inanimate, is presented as a solvable problem without lasting negative consequences, a very low-risk low-reward situation. By being a puzzle, the object is attempting to convince the user that it must be completed.
\end{quote}
{\tiny Source: \href{https://www.everydayrhetoric.com/2020/03/the-rhetoric-of-puzzles/}{Every Day Rhetoric}}
\end{frame}


\mode<all>{
\part{Basic Constraint Propagation}
}
\begin{frame}
\partpage
\end{frame}
\mode<all>{
\input{../sendmore/body}
}

\mode<all>{
\part{Global Constraints}
}
\begin{frame}
\partpage
\end{frame}
\mode<all>{
\input{../sudoku/body}
}

\clearpage
\mode<all>{
\part{Customizing Search}
}
\begin{frame}
\partpage
\end{frame}
\mode<all>{
\input{../nqueen/body}
}


%% \clearpage
%% \mode<all>{
%%   \part{Problem Decomposition}
%% }
%% \begin{frame}
%%   \partpage
%% \end{frame}
%% \mode<all>{
%%   \input{../party/body}
%%   }

%% \clearpage
%% \mode<all>{
%%   \part{More Global Constraints}
%% }
%% \begin{frame}
%%   \partpage
%% \end{frame}
%% \mode<all>{
%%   \input{../car/body}
%%   }

%% \clearpage
%% \mode<all>{
%%   \part{Methodology}
%% }
%% \begin{frame}
%%   \partpage
%% \end{frame}
%% \mode<all>{
%%   \input{../methodology/body}
%%   }

\clearpage
\mode<all>{
\part{What is missing?}
}
\begin{frame}
\partpage
\end{frame}

\begin{frame}
\frametitle{Many Specialized Topics}
\begin{itemize}
\item How to design efficient core engine
\item Hybrids with LP/MIP tools
\item Hybrids with SAT
\item Symmetry breaking
\item Use of MDD/BDD to encode sets of solutions
\item High level modelling tools
\item Debugging/visualization
\end{itemize}
\end{frame}

\begin{frame}
\frametitle{Reformulation}
\begin{itemize}
\item Just because the user has modelled it this way, it doesn't mean we have to solve it that way
\begin{itemize}
\item Replace some constraint(s) by other, equivalent constraints
\item Because we don't have that constraint in our system
\item For performance
\end{itemize}
\end{itemize}
\end{frame}

\begin{frame}
\frametitle{Learning}
\begin{itemize}
\item While solving the problem we can learn how to strengthen the model/search 
\begin{itemize}
\item Understand which constraints/method contribute to propagation and change schedule
\item Learn no-good constraints by explaining failure
\item Adapt search strategy based on search experience
\end{itemize}
\end{itemize}
\end{frame}

%% \begin{frame}
%% \frametitle{Refined Process}
%% \begin{center}
%% \begin{tikzpicture}
%% \node[shape=rectangle,fill=pantone127-4] (problem) at (3,7) {Problem};
%% \node (human) at (3,6) {Human};
%% \node[shape=rectangle,fill=pantone127-4] (model) at (3,5) {User Model};
%% \node (reformulation) at (3,4) {Reformulation};
%% \node[shape=rectangle,fill=pantone127-4] (implementation) at (3,3) {Implementation Model};
%% \node (solver) at (3,2) {Constraint Solver/Search};
%% \node[shape=rectangle,fill=pantone157-8] (solution) at (3,1) {Solution};
%% \draw[-] (problem) -- (human);
%% \draw[->] (human) -- (model);
%% \draw[-] (model) -- (reformulation);
%% \draw[->] (reformulation) -- (implementation);
%% \draw[-] (implementation) -- (solver);
%% \draw[->] (solver) -- (0,2) -- (0,3) -- (implementation);
%% \draw[->] (solver) -- (solution);
%% \end{tikzpicture}
%% \end{center}
%% \end{frame}

%\begin{frame}
%\frametitle{What is CP actually used for?}
%\url{http://hsimonis.wordpress.com}
%\includegraphics[width=10cm]{../introduction/images/blog}
%\end{frame}

%% \begin{frame}
%% \frametitle{Exercises}
%% \begin{itemize}
%% \item Install Constraint Solver of your Choice
%% \item MiniZinc (MiningZinc) will be used in later sessions
%% \item Run Sendmoremoney, Sudoku, Queens examples
%% \item Can you choose which propagator to use for alldifferent?
%% \item Can you use different search strategies?
%% \end{itemize}
%% \end{frame}

\begin{frame}
\frametitle{More Learning Resources}
\begin{itemize}
\item Survey of Methods, Resources, and Formats for Teaching Constraint Programming
\begin{itemize}
\item by Tejas Santanam, Helmut Simonis
\item \url{https://doi.org/10.48550/arXiv.2403.12717}
\item Based on survey of community for WTCP 2023
\item \url{https://hsimonis.github.io/WTCP2023/}
\end{itemize}
\end{itemize}
\end{frame}


}


%% \clearpage
%% \mode<all>{
%%   \part{Problem Decomposition}
%% }
%% \begin{frame}
%%   \partpage
%% \end{frame}
%% \mode<all>{
%%   \begin{frame}
\frametitle{Objectives}
\begin{itemize}
\item Overview of Core Constraint Programming
\item Three Main Concepts
\begin{itemize}
\item Constraint Propagation
\item Global Constraints
\item Customizing Search
\end{itemize}
%\item Get Some Experience with MiniZinc
\item Topics will be treated in more detail in later parts of the school
\item Based on Examples, not Formal Description
\end{itemize}
\end{frame}

\begin{frame}
\frametitle{Outline}
\begin{itemize}
\item Why Constraint Programming?
\item Constraint Propagation
\item Global Constraints
\item Customizing Search
%\item Choices in Modelling
%\item Decomposition
%\item More Global Constraints
%\item Methodology
%\item What is missing?
\end{itemize}
\end{frame}

% \begin{frame}
  % \frametitle{Using MiniZinc IDE}
  % \begin{itemize}
  % \item Developed in the Australian NICTA project
  % \item Maintained by Monash University
  % \item Modelling tool with multiple back-end solvers
  % \item Available from \url{https://www.minizinc.org/}
  % \end{itemize}
% \end{frame}


% \begin{frame}
% \frametitle{Examples in ECLiPSe}
% \begin{itemize}
% \item Open sourced constraint programming language
% \item Development goes back to 1985
% \item ECRC, ICL, IC-Parc, PTL, Cisco
% \item \url{https://eclipseclp.org/}
% \item Specialities
% \begin{itemize}
% \item Develop new solvers for specific domains
% \item Integration with MIP
% \end{itemize}
% \item Not included in bundled MiniZinc IDE
% \item Specialized visualization tools used here
  % \begin{itemize}
  % \item CP-Viz, Simonis et al. 2010
    % \end{itemize}
% \end{itemize}
% \end{frame}

\begin{frame}
\frametitle{Tutorial Based on ECLiPSe ELearning Course}
\begin{itemize}
\item Self-study course in constraint programming
\item Supported by Cisco Systems and Silicon Valley Community Foundation
\item Multi-media format, video lectures, slides, handout etc
\item {\small \url{https://eclipseclp.org/ELearning/index.html}}
\end{itemize}
\end{frame}

\begin{frame}
\frametitle{Also Part of CRT-AI Constraint Week}
\begin{itemize}
\item Annual one week course on CP and Optimization in Ireland
\item Part of national training program for PhD students in AI
\item \url{https://www.crt-ai.ie/}
\end{itemize}
\end{frame}



\begin{frame}
\frametitle{Constraint Programming - in a nutshell}
\begin{itemize}
\item Declarative description of problems with
\begin{itemize}
\item {\em Variables} which range over (finite) sets of values
\item {\em Constraints} over subsets of variables which restrict possible value combinations
\item A {\em solution} is a value assignment which satisfies all constraints
\end{itemize}

\item Constraint propagation/reasoning
\begin{itemize}
\item Removing inconsistent values for variables
\item Detect failure if constraint can not be satisfied
\item Interaction of constraints via shared variables
\item Incomplete
\end{itemize}

\item Search
\begin{itemize}
\item User controlled assignment of values to variables
\item Each step triggers constraint propagation 
\end{itemize}
\item Different domains require/allow different methods
\end{itemize}
\end{frame}

\begin{frame}
  \frametitle{Constraint Programming is Different}
  \begin{itemize}
  \item Declarative Programming
    \begin{itemize}
    \item Concentrate on what you want
      \item Not how to get there
      \item Program != Algorithm
      \item Program = Model
    \end{itemize}
    \item Applied to Combinatorial Problems
      \begin{itemize}
        \item No complete polynomial algorithms known (exist?)
        \item CP less ad-hoc than heuristics
        \item Models can evolve
  \end{itemize}
  \end{itemize}
  \end{frame}
    
\begin{frame}
  \frametitle{A Subtractive Process}
  \begin{textblock}{4}(8,-3)
    \includegraphics[width=4cm]{../introduction/images/stages}
  \end{textblock}
  \vfill
  \begin{quote}
    ``Oh, bosh, as Mr. Ruskin says. Sculpture, per se, is the simplest thing in the world. All you have to do is to take a big chunk of marble and a hammer and chisel, make up your mind what you are about to create and chip off all the marble you don't want.''-Paris Gaulois.
  \end{quote}
  
  {\tiny Source: \url{https://quoteinvestigator.com/2014/06/22/chip-away/}}
\end{frame}

\begin{frame}
\frametitle{Basic Process}
\begin{center}
\begin{tikzpicture}
\node[shape=rectangle,fill=pantone127-4] (problem) at (2,5) {Problem};
\node (human) at (2,4) {Human};
\node[shape=rectangle,fill=pantone127-4] (model) at (2,3) {Model};
\node (solver) at (2,2) {Constraint Solver/Search};
\node[shape=rectangle,fill=pantone157-8] (solution) at (2,1) {Solution};
\draw[-] (problem) -- (human);
\draw[->] (human) -- (model);
\draw[-] (model) -- (solver);
\draw[->] (solver) -- (solution);
\end{tikzpicture}
\end{center}
\end{frame}

\begin{frame}
\frametitle{More Realistic}
\begin{center}
\begin{tikzpicture}[xscale=4.0]
\node[shape=rectangle,fill=pantone127-4] (problem) at (2,5) {Problem};
\node (human) at (2,4) {Human};
\node[shape=rectangle,fill=pantone127-4] (model) at (2,3) {Model};
\node (solver) at (2,2) {Constraint Solver/Search};
\node[shape=rectangle,fill=pantone24-8] (hangs) at (1,1) {Hangs};
\node[shape=rectangle,fill=pantone157-8] (solution) at (2,1) {Solution};
\node[shape=rectangle,fill=pantone24-8] (wrong) at (3,1) {Wrong Solution};
\draw[-] (problem) -- (human);
\draw[->] (human) -- (model);
\draw[-] (model) -- (solver);
\draw[->] (solver) -- (hangs);
\draw[->] (solver) -- (solution);
\draw[->] (solver) -- (wrong);
\draw[->] (hangs) -- (1,4) -- (human);
\draw[->] (wrong) -- (3,4) -- (human);
\end{tikzpicture}
\end{center}
\end{frame}

\begin{frame}
\frametitle{Dual Role of Model}
\begin{itemize}
\item Allows Human to Express Problem
\begin{itemize}
\item Close to Problem Domain
\item Constraints as Abstractions
\end{itemize}
\item Allows Solver to Execute
\begin{itemize}
\item Variables as Communication Mechanism
\item Constraints as Algorithms
\end{itemize}
\end{itemize}
\end{frame}

\begin{frame}
\frametitle{Modelling Frameworks}
\begin{itemize}
\item \href{https://www.minizinc.org/}{MiniZinc} (NICTA, Monash University, Australia)
\item \href{https://github.com/eomahony/Numberjack}{NumberJack} (Insight, Ireland) 
\item \href{https://www-users.york.ac.uk/peter.nightingale/savilerow//}{EssencePrime/SavilleRow} (UK) 
\item \href{https://cpmpy.readthedocs.io/en/latest/}{CPMpy} (KU Leuven)
\item Allow use of multiple back-end solvers
\item Compile model into variants for each solver
\item A priori solver independent model(CP, MIP, SAT)
\end{itemize}
\end{frame} 

\begin{frame}
\frametitle{Framework Process}
\begin{center}
\begin{tikzpicture}[xscale=2]
\node[shape=rectangle,fill=pantone127-4] (problem) at (2.5,5) {Problem};
\node (human) at (2.5,4) {Human};
\node[shape=rectangle,fill=pantone127-4] (model) at (2.5,3) {Model};
\node (compile) at (2.5,2) {Compile/Reformulate};
\node (solver1) at (1,1) {CP};
\node (solver2) at (2,1) {MIP};
\node (solver3) at (3,1) {SAT};
\node (solver4) at (4,1) {Other};
\node[shape=rectangle,fill=pantone157-8] (solution1) at (1,0) {Solution};
\node[shape=rectangle,fill=pantone157-8] (solution2) at (2,0) {Solution};
\node[shape=rectangle,fill=pantone157-8] (solution3) at (3,0) {Solution};
\node[shape=rectangle,fill=pantone157-8] (solution4) at (4,0) {Solution};
\draw[-] (problem) -- (human);
\draw[->] (human) -- (model);
\draw[-] (model) -- (compile);
\draw[-] (compile) -- (solver1);
\draw[-] (compile) -- (solver2);
\draw[-] (compile) -- (solver3);
\draw[-] (compile) -- (solver4);
\draw[->] (solver1) -- (solution1);
\draw[->] (solver2) -- (solution2);
\draw[->] (solver3) -- (solution3);
\draw[->] (solver4) -- (solution4);
\end{tikzpicture}
\end{center}
\end{frame}

% \begin{frame}
  % \frametitle{Do It Now!}
  % \begin{itemize}
  % \item Download and install Minizinc
  % \item \url{https://www.minizinc.org/}
  % \end{itemize}
% \end{frame}


\begin{frame}
\frametitle{Why use Puzzles as Examples?}
\begin{itemize}
\item Easy to understand the problem
\item Solvable by hand without specialized knowledge
\item Possible to compare automated to manual solving process
\end{itemize}

\begin{quote}
The puzzle, though inanimate, is presented as a solvable problem without lasting negative consequences, a very low-risk low-reward situation. By being a puzzle, the object is attempting to convince the user that it must be completed.
\end{quote}
{\tiny Source: \href{https://www.everydayrhetoric.com/2020/03/the-rhetoric-of-puzzles/}{Every Day Rhetoric}}
\end{frame}


\mode<all>{
\part{Basic Constraint Propagation}
}
\begin{frame}
\partpage
\end{frame}
\mode<all>{
\input{../sendmore/body}
}

\mode<all>{
\part{Global Constraints}
}
\begin{frame}
\partpage
\end{frame}
\mode<all>{
\input{../sudoku/body}
}

\clearpage
\mode<all>{
\part{Customizing Search}
}
\begin{frame}
\partpage
\end{frame}
\mode<all>{
\input{../nqueen/body}
}


%% \clearpage
%% \mode<all>{
%%   \part{Problem Decomposition}
%% }
%% \begin{frame}
%%   \partpage
%% \end{frame}
%% \mode<all>{
%%   \input{../party/body}
%%   }

%% \clearpage
%% \mode<all>{
%%   \part{More Global Constraints}
%% }
%% \begin{frame}
%%   \partpage
%% \end{frame}
%% \mode<all>{
%%   \input{../car/body}
%%   }

%% \clearpage
%% \mode<all>{
%%   \part{Methodology}
%% }
%% \begin{frame}
%%   \partpage
%% \end{frame}
%% \mode<all>{
%%   \input{../methodology/body}
%%   }

\clearpage
\mode<all>{
\part{What is missing?}
}
\begin{frame}
\partpage
\end{frame}

\begin{frame}
\frametitle{Many Specialized Topics}
\begin{itemize}
\item How to design efficient core engine
\item Hybrids with LP/MIP tools
\item Hybrids with SAT
\item Symmetry breaking
\item Use of MDD/BDD to encode sets of solutions
\item High level modelling tools
\item Debugging/visualization
\end{itemize}
\end{frame}

\begin{frame}
\frametitle{Reformulation}
\begin{itemize}
\item Just because the user has modelled it this way, it doesn't mean we have to solve it that way
\begin{itemize}
\item Replace some constraint(s) by other, equivalent constraints
\item Because we don't have that constraint in our system
\item For performance
\end{itemize}
\end{itemize}
\end{frame}

\begin{frame}
\frametitle{Learning}
\begin{itemize}
\item While solving the problem we can learn how to strengthen the model/search 
\begin{itemize}
\item Understand which constraints/method contribute to propagation and change schedule
\item Learn no-good constraints by explaining failure
\item Adapt search strategy based on search experience
\end{itemize}
\end{itemize}
\end{frame}

%% \begin{frame}
%% \frametitle{Refined Process}
%% \begin{center}
%% \begin{tikzpicture}
%% \node[shape=rectangle,fill=pantone127-4] (problem) at (3,7) {Problem};
%% \node (human) at (3,6) {Human};
%% \node[shape=rectangle,fill=pantone127-4] (model) at (3,5) {User Model};
%% \node (reformulation) at (3,4) {Reformulation};
%% \node[shape=rectangle,fill=pantone127-4] (implementation) at (3,3) {Implementation Model};
%% \node (solver) at (3,2) {Constraint Solver/Search};
%% \node[shape=rectangle,fill=pantone157-8] (solution) at (3,1) {Solution};
%% \draw[-] (problem) -- (human);
%% \draw[->] (human) -- (model);
%% \draw[-] (model) -- (reformulation);
%% \draw[->] (reformulation) -- (implementation);
%% \draw[-] (implementation) -- (solver);
%% \draw[->] (solver) -- (0,2) -- (0,3) -- (implementation);
%% \draw[->] (solver) -- (solution);
%% \end{tikzpicture}
%% \end{center}
%% \end{frame}

%\begin{frame}
%\frametitle{What is CP actually used for?}
%\url{http://hsimonis.wordpress.com}
%\includegraphics[width=10cm]{../introduction/images/blog}
%\end{frame}

%% \begin{frame}
%% \frametitle{Exercises}
%% \begin{itemize}
%% \item Install Constraint Solver of your Choice
%% \item MiniZinc (MiningZinc) will be used in later sessions
%% \item Run Sendmoremoney, Sudoku, Queens examples
%% \item Can you choose which propagator to use for alldifferent?
%% \item Can you use different search strategies?
%% \end{itemize}
%% \end{frame}

\begin{frame}
\frametitle{More Learning Resources}
\begin{itemize}
\item Survey of Methods, Resources, and Formats for Teaching Constraint Programming
\begin{itemize}
\item by Tejas Santanam, Helmut Simonis
\item \url{https://doi.org/10.48550/arXiv.2403.12717}
\item Based on survey of community for WTCP 2023
\item \url{https://hsimonis.github.io/WTCP2023/}
\end{itemize}
\end{itemize}
\end{frame}


%%   }

%% \clearpage
%% \mode<all>{
%%   \part{More Global Constraints}
%% }
%% \begin{frame}
%%   \partpage
%% \end{frame}
%% \mode<all>{
%%   \begin{frame}
\frametitle{Objectives}
\begin{itemize}
\item Overview of Core Constraint Programming
\item Three Main Concepts
\begin{itemize}
\item Constraint Propagation
\item Global Constraints
\item Customizing Search
\end{itemize}
%\item Get Some Experience with MiniZinc
\item Topics will be treated in more detail in later parts of the school
\item Based on Examples, not Formal Description
\end{itemize}
\end{frame}

\begin{frame}
\frametitle{Outline}
\begin{itemize}
\item Why Constraint Programming?
\item Constraint Propagation
\item Global Constraints
\item Customizing Search
%\item Choices in Modelling
%\item Decomposition
%\item More Global Constraints
%\item Methodology
%\item What is missing?
\end{itemize}
\end{frame}

% \begin{frame}
  % \frametitle{Using MiniZinc IDE}
  % \begin{itemize}
  % \item Developed in the Australian NICTA project
  % \item Maintained by Monash University
  % \item Modelling tool with multiple back-end solvers
  % \item Available from \url{https://www.minizinc.org/}
  % \end{itemize}
% \end{frame}


% \begin{frame}
% \frametitle{Examples in ECLiPSe}
% \begin{itemize}
% \item Open sourced constraint programming language
% \item Development goes back to 1985
% \item ECRC, ICL, IC-Parc, PTL, Cisco
% \item \url{https://eclipseclp.org/}
% \item Specialities
% \begin{itemize}
% \item Develop new solvers for specific domains
% \item Integration with MIP
% \end{itemize}
% \item Not included in bundled MiniZinc IDE
% \item Specialized visualization tools used here
  % \begin{itemize}
  % \item CP-Viz, Simonis et al. 2010
    % \end{itemize}
% \end{itemize}
% \end{frame}

\begin{frame}
\frametitle{Tutorial Based on ECLiPSe ELearning Course}
\begin{itemize}
\item Self-study course in constraint programming
\item Supported by Cisco Systems and Silicon Valley Community Foundation
\item Multi-media format, video lectures, slides, handout etc
\item {\small \url{https://eclipseclp.org/ELearning/index.html}}
\end{itemize}
\end{frame}

\begin{frame}
\frametitle{Also Part of CRT-AI Constraint Week}
\begin{itemize}
\item Annual one week course on CP and Optimization in Ireland
\item Part of national training program for PhD students in AI
\item \url{https://www.crt-ai.ie/}
\end{itemize}
\end{frame}



\begin{frame}
\frametitle{Constraint Programming - in a nutshell}
\begin{itemize}
\item Declarative description of problems with
\begin{itemize}
\item {\em Variables} which range over (finite) sets of values
\item {\em Constraints} over subsets of variables which restrict possible value combinations
\item A {\em solution} is a value assignment which satisfies all constraints
\end{itemize}

\item Constraint propagation/reasoning
\begin{itemize}
\item Removing inconsistent values for variables
\item Detect failure if constraint can not be satisfied
\item Interaction of constraints via shared variables
\item Incomplete
\end{itemize}

\item Search
\begin{itemize}
\item User controlled assignment of values to variables
\item Each step triggers constraint propagation 
\end{itemize}
\item Different domains require/allow different methods
\end{itemize}
\end{frame}

\begin{frame}
  \frametitle{Constraint Programming is Different}
  \begin{itemize}
  \item Declarative Programming
    \begin{itemize}
    \item Concentrate on what you want
      \item Not how to get there
      \item Program != Algorithm
      \item Program = Model
    \end{itemize}
    \item Applied to Combinatorial Problems
      \begin{itemize}
        \item No complete polynomial algorithms known (exist?)
        \item CP less ad-hoc than heuristics
        \item Models can evolve
  \end{itemize}
  \end{itemize}
  \end{frame}
    
\begin{frame}
  \frametitle{A Subtractive Process}
  \begin{textblock}{4}(8,-3)
    \includegraphics[width=4cm]{../introduction/images/stages}
  \end{textblock}
  \vfill
  \begin{quote}
    ``Oh, bosh, as Mr. Ruskin says. Sculpture, per se, is the simplest thing in the world. All you have to do is to take a big chunk of marble and a hammer and chisel, make up your mind what you are about to create and chip off all the marble you don't want.''-Paris Gaulois.
  \end{quote}
  
  {\tiny Source: \url{https://quoteinvestigator.com/2014/06/22/chip-away/}}
\end{frame}

\begin{frame}
\frametitle{Basic Process}
\begin{center}
\begin{tikzpicture}
\node[shape=rectangle,fill=pantone127-4] (problem) at (2,5) {Problem};
\node (human) at (2,4) {Human};
\node[shape=rectangle,fill=pantone127-4] (model) at (2,3) {Model};
\node (solver) at (2,2) {Constraint Solver/Search};
\node[shape=rectangle,fill=pantone157-8] (solution) at (2,1) {Solution};
\draw[-] (problem) -- (human);
\draw[->] (human) -- (model);
\draw[-] (model) -- (solver);
\draw[->] (solver) -- (solution);
\end{tikzpicture}
\end{center}
\end{frame}

\begin{frame}
\frametitle{More Realistic}
\begin{center}
\begin{tikzpicture}[xscale=4.0]
\node[shape=rectangle,fill=pantone127-4] (problem) at (2,5) {Problem};
\node (human) at (2,4) {Human};
\node[shape=rectangle,fill=pantone127-4] (model) at (2,3) {Model};
\node (solver) at (2,2) {Constraint Solver/Search};
\node[shape=rectangle,fill=pantone24-8] (hangs) at (1,1) {Hangs};
\node[shape=rectangle,fill=pantone157-8] (solution) at (2,1) {Solution};
\node[shape=rectangle,fill=pantone24-8] (wrong) at (3,1) {Wrong Solution};
\draw[-] (problem) -- (human);
\draw[->] (human) -- (model);
\draw[-] (model) -- (solver);
\draw[->] (solver) -- (hangs);
\draw[->] (solver) -- (solution);
\draw[->] (solver) -- (wrong);
\draw[->] (hangs) -- (1,4) -- (human);
\draw[->] (wrong) -- (3,4) -- (human);
\end{tikzpicture}
\end{center}
\end{frame}

\begin{frame}
\frametitle{Dual Role of Model}
\begin{itemize}
\item Allows Human to Express Problem
\begin{itemize}
\item Close to Problem Domain
\item Constraints as Abstractions
\end{itemize}
\item Allows Solver to Execute
\begin{itemize}
\item Variables as Communication Mechanism
\item Constraints as Algorithms
\end{itemize}
\end{itemize}
\end{frame}

\begin{frame}
\frametitle{Modelling Frameworks}
\begin{itemize}
\item \href{https://www.minizinc.org/}{MiniZinc} (NICTA, Monash University, Australia)
\item \href{https://github.com/eomahony/Numberjack}{NumberJack} (Insight, Ireland) 
\item \href{https://www-users.york.ac.uk/peter.nightingale/savilerow//}{EssencePrime/SavilleRow} (UK) 
\item \href{https://cpmpy.readthedocs.io/en/latest/}{CPMpy} (KU Leuven)
\item Allow use of multiple back-end solvers
\item Compile model into variants for each solver
\item A priori solver independent model(CP, MIP, SAT)
\end{itemize}
\end{frame} 

\begin{frame}
\frametitle{Framework Process}
\begin{center}
\begin{tikzpicture}[xscale=2]
\node[shape=rectangle,fill=pantone127-4] (problem) at (2.5,5) {Problem};
\node (human) at (2.5,4) {Human};
\node[shape=rectangle,fill=pantone127-4] (model) at (2.5,3) {Model};
\node (compile) at (2.5,2) {Compile/Reformulate};
\node (solver1) at (1,1) {CP};
\node (solver2) at (2,1) {MIP};
\node (solver3) at (3,1) {SAT};
\node (solver4) at (4,1) {Other};
\node[shape=rectangle,fill=pantone157-8] (solution1) at (1,0) {Solution};
\node[shape=rectangle,fill=pantone157-8] (solution2) at (2,0) {Solution};
\node[shape=rectangle,fill=pantone157-8] (solution3) at (3,0) {Solution};
\node[shape=rectangle,fill=pantone157-8] (solution4) at (4,0) {Solution};
\draw[-] (problem) -- (human);
\draw[->] (human) -- (model);
\draw[-] (model) -- (compile);
\draw[-] (compile) -- (solver1);
\draw[-] (compile) -- (solver2);
\draw[-] (compile) -- (solver3);
\draw[-] (compile) -- (solver4);
\draw[->] (solver1) -- (solution1);
\draw[->] (solver2) -- (solution2);
\draw[->] (solver3) -- (solution3);
\draw[->] (solver4) -- (solution4);
\end{tikzpicture}
\end{center}
\end{frame}

% \begin{frame}
  % \frametitle{Do It Now!}
  % \begin{itemize}
  % \item Download and install Minizinc
  % \item \url{https://www.minizinc.org/}
  % \end{itemize}
% \end{frame}


\begin{frame}
\frametitle{Why use Puzzles as Examples?}
\begin{itemize}
\item Easy to understand the problem
\item Solvable by hand without specialized knowledge
\item Possible to compare automated to manual solving process
\end{itemize}

\begin{quote}
The puzzle, though inanimate, is presented as a solvable problem without lasting negative consequences, a very low-risk low-reward situation. By being a puzzle, the object is attempting to convince the user that it must be completed.
\end{quote}
{\tiny Source: \href{https://www.everydayrhetoric.com/2020/03/the-rhetoric-of-puzzles/}{Every Day Rhetoric}}
\end{frame}


\mode<all>{
\part{Basic Constraint Propagation}
}
\begin{frame}
\partpage
\end{frame}
\mode<all>{
\input{../sendmore/body}
}

\mode<all>{
\part{Global Constraints}
}
\begin{frame}
\partpage
\end{frame}
\mode<all>{
\input{../sudoku/body}
}

\clearpage
\mode<all>{
\part{Customizing Search}
}
\begin{frame}
\partpage
\end{frame}
\mode<all>{
\input{../nqueen/body}
}


%% \clearpage
%% \mode<all>{
%%   \part{Problem Decomposition}
%% }
%% \begin{frame}
%%   \partpage
%% \end{frame}
%% \mode<all>{
%%   \input{../party/body}
%%   }

%% \clearpage
%% \mode<all>{
%%   \part{More Global Constraints}
%% }
%% \begin{frame}
%%   \partpage
%% \end{frame}
%% \mode<all>{
%%   \input{../car/body}
%%   }

%% \clearpage
%% \mode<all>{
%%   \part{Methodology}
%% }
%% \begin{frame}
%%   \partpage
%% \end{frame}
%% \mode<all>{
%%   \input{../methodology/body}
%%   }

\clearpage
\mode<all>{
\part{What is missing?}
}
\begin{frame}
\partpage
\end{frame}

\begin{frame}
\frametitle{Many Specialized Topics}
\begin{itemize}
\item How to design efficient core engine
\item Hybrids with LP/MIP tools
\item Hybrids with SAT
\item Symmetry breaking
\item Use of MDD/BDD to encode sets of solutions
\item High level modelling tools
\item Debugging/visualization
\end{itemize}
\end{frame}

\begin{frame}
\frametitle{Reformulation}
\begin{itemize}
\item Just because the user has modelled it this way, it doesn't mean we have to solve it that way
\begin{itemize}
\item Replace some constraint(s) by other, equivalent constraints
\item Because we don't have that constraint in our system
\item For performance
\end{itemize}
\end{itemize}
\end{frame}

\begin{frame}
\frametitle{Learning}
\begin{itemize}
\item While solving the problem we can learn how to strengthen the model/search 
\begin{itemize}
\item Understand which constraints/method contribute to propagation and change schedule
\item Learn no-good constraints by explaining failure
\item Adapt search strategy based on search experience
\end{itemize}
\end{itemize}
\end{frame}

%% \begin{frame}
%% \frametitle{Refined Process}
%% \begin{center}
%% \begin{tikzpicture}
%% \node[shape=rectangle,fill=pantone127-4] (problem) at (3,7) {Problem};
%% \node (human) at (3,6) {Human};
%% \node[shape=rectangle,fill=pantone127-4] (model) at (3,5) {User Model};
%% \node (reformulation) at (3,4) {Reformulation};
%% \node[shape=rectangle,fill=pantone127-4] (implementation) at (3,3) {Implementation Model};
%% \node (solver) at (3,2) {Constraint Solver/Search};
%% \node[shape=rectangle,fill=pantone157-8] (solution) at (3,1) {Solution};
%% \draw[-] (problem) -- (human);
%% \draw[->] (human) -- (model);
%% \draw[-] (model) -- (reformulation);
%% \draw[->] (reformulation) -- (implementation);
%% \draw[-] (implementation) -- (solver);
%% \draw[->] (solver) -- (0,2) -- (0,3) -- (implementation);
%% \draw[->] (solver) -- (solution);
%% \end{tikzpicture}
%% \end{center}
%% \end{frame}

%\begin{frame}
%\frametitle{What is CP actually used for?}
%\url{http://hsimonis.wordpress.com}
%\includegraphics[width=10cm]{../introduction/images/blog}
%\end{frame}

%% \begin{frame}
%% \frametitle{Exercises}
%% \begin{itemize}
%% \item Install Constraint Solver of your Choice
%% \item MiniZinc (MiningZinc) will be used in later sessions
%% \item Run Sendmoremoney, Sudoku, Queens examples
%% \item Can you choose which propagator to use for alldifferent?
%% \item Can you use different search strategies?
%% \end{itemize}
%% \end{frame}

\begin{frame}
\frametitle{More Learning Resources}
\begin{itemize}
\item Survey of Methods, Resources, and Formats for Teaching Constraint Programming
\begin{itemize}
\item by Tejas Santanam, Helmut Simonis
\item \url{https://doi.org/10.48550/arXiv.2403.12717}
\item Based on survey of community for WTCP 2023
\item \url{https://hsimonis.github.io/WTCP2023/}
\end{itemize}
\end{itemize}
\end{frame}


%%   }

%% \clearpage
%% \mode<all>{
%%   \part{Methodology}
%% }
%% \begin{frame}
%%   \partpage
%% \end{frame}
%% \mode<all>{
%%   \begin{frame}
\frametitle{Objectives}
\begin{itemize}
\item Overview of Core Constraint Programming
\item Three Main Concepts
\begin{itemize}
\item Constraint Propagation
\item Global Constraints
\item Customizing Search
\end{itemize}
%\item Get Some Experience with MiniZinc
\item Topics will be treated in more detail in later parts of the school
\item Based on Examples, not Formal Description
\end{itemize}
\end{frame}

\begin{frame}
\frametitle{Outline}
\begin{itemize}
\item Why Constraint Programming?
\item Constraint Propagation
\item Global Constraints
\item Customizing Search
%\item Choices in Modelling
%\item Decomposition
%\item More Global Constraints
%\item Methodology
%\item What is missing?
\end{itemize}
\end{frame}

% \begin{frame}
  % \frametitle{Using MiniZinc IDE}
  % \begin{itemize}
  % \item Developed in the Australian NICTA project
  % \item Maintained by Monash University
  % \item Modelling tool with multiple back-end solvers
  % \item Available from \url{https://www.minizinc.org/}
  % \end{itemize}
% \end{frame}


% \begin{frame}
% \frametitle{Examples in ECLiPSe}
% \begin{itemize}
% \item Open sourced constraint programming language
% \item Development goes back to 1985
% \item ECRC, ICL, IC-Parc, PTL, Cisco
% \item \url{https://eclipseclp.org/}
% \item Specialities
% \begin{itemize}
% \item Develop new solvers for specific domains
% \item Integration with MIP
% \end{itemize}
% \item Not included in bundled MiniZinc IDE
% \item Specialized visualization tools used here
  % \begin{itemize}
  % \item CP-Viz, Simonis et al. 2010
    % \end{itemize}
% \end{itemize}
% \end{frame}

\begin{frame}
\frametitle{Tutorial Based on ECLiPSe ELearning Course}
\begin{itemize}
\item Self-study course in constraint programming
\item Supported by Cisco Systems and Silicon Valley Community Foundation
\item Multi-media format, video lectures, slides, handout etc
\item {\small \url{https://eclipseclp.org/ELearning/index.html}}
\end{itemize}
\end{frame}

\begin{frame}
\frametitle{Also Part of CRT-AI Constraint Week}
\begin{itemize}
\item Annual one week course on CP and Optimization in Ireland
\item Part of national training program for PhD students in AI
\item \url{https://www.crt-ai.ie/}
\end{itemize}
\end{frame}



\begin{frame}
\frametitle{Constraint Programming - in a nutshell}
\begin{itemize}
\item Declarative description of problems with
\begin{itemize}
\item {\em Variables} which range over (finite) sets of values
\item {\em Constraints} over subsets of variables which restrict possible value combinations
\item A {\em solution} is a value assignment which satisfies all constraints
\end{itemize}

\item Constraint propagation/reasoning
\begin{itemize}
\item Removing inconsistent values for variables
\item Detect failure if constraint can not be satisfied
\item Interaction of constraints via shared variables
\item Incomplete
\end{itemize}

\item Search
\begin{itemize}
\item User controlled assignment of values to variables
\item Each step triggers constraint propagation 
\end{itemize}
\item Different domains require/allow different methods
\end{itemize}
\end{frame}

\begin{frame}
  \frametitle{Constraint Programming is Different}
  \begin{itemize}
  \item Declarative Programming
    \begin{itemize}
    \item Concentrate on what you want
      \item Not how to get there
      \item Program != Algorithm
      \item Program = Model
    \end{itemize}
    \item Applied to Combinatorial Problems
      \begin{itemize}
        \item No complete polynomial algorithms known (exist?)
        \item CP less ad-hoc than heuristics
        \item Models can evolve
  \end{itemize}
  \end{itemize}
  \end{frame}
    
\begin{frame}
  \frametitle{A Subtractive Process}
  \begin{textblock}{4}(8,-3)
    \includegraphics[width=4cm]{../introduction/images/stages}
  \end{textblock}
  \vfill
  \begin{quote}
    ``Oh, bosh, as Mr. Ruskin says. Sculpture, per se, is the simplest thing in the world. All you have to do is to take a big chunk of marble and a hammer and chisel, make up your mind what you are about to create and chip off all the marble you don't want.''-Paris Gaulois.
  \end{quote}
  
  {\tiny Source: \url{https://quoteinvestigator.com/2014/06/22/chip-away/}}
\end{frame}

\begin{frame}
\frametitle{Basic Process}
\begin{center}
\begin{tikzpicture}
\node[shape=rectangle,fill=pantone127-4] (problem) at (2,5) {Problem};
\node (human) at (2,4) {Human};
\node[shape=rectangle,fill=pantone127-4] (model) at (2,3) {Model};
\node (solver) at (2,2) {Constraint Solver/Search};
\node[shape=rectangle,fill=pantone157-8] (solution) at (2,1) {Solution};
\draw[-] (problem) -- (human);
\draw[->] (human) -- (model);
\draw[-] (model) -- (solver);
\draw[->] (solver) -- (solution);
\end{tikzpicture}
\end{center}
\end{frame}

\begin{frame}
\frametitle{More Realistic}
\begin{center}
\begin{tikzpicture}[xscale=4.0]
\node[shape=rectangle,fill=pantone127-4] (problem) at (2,5) {Problem};
\node (human) at (2,4) {Human};
\node[shape=rectangle,fill=pantone127-4] (model) at (2,3) {Model};
\node (solver) at (2,2) {Constraint Solver/Search};
\node[shape=rectangle,fill=pantone24-8] (hangs) at (1,1) {Hangs};
\node[shape=rectangle,fill=pantone157-8] (solution) at (2,1) {Solution};
\node[shape=rectangle,fill=pantone24-8] (wrong) at (3,1) {Wrong Solution};
\draw[-] (problem) -- (human);
\draw[->] (human) -- (model);
\draw[-] (model) -- (solver);
\draw[->] (solver) -- (hangs);
\draw[->] (solver) -- (solution);
\draw[->] (solver) -- (wrong);
\draw[->] (hangs) -- (1,4) -- (human);
\draw[->] (wrong) -- (3,4) -- (human);
\end{tikzpicture}
\end{center}
\end{frame}

\begin{frame}
\frametitle{Dual Role of Model}
\begin{itemize}
\item Allows Human to Express Problem
\begin{itemize}
\item Close to Problem Domain
\item Constraints as Abstractions
\end{itemize}
\item Allows Solver to Execute
\begin{itemize}
\item Variables as Communication Mechanism
\item Constraints as Algorithms
\end{itemize}
\end{itemize}
\end{frame}

\begin{frame}
\frametitle{Modelling Frameworks}
\begin{itemize}
\item \href{https://www.minizinc.org/}{MiniZinc} (NICTA, Monash University, Australia)
\item \href{https://github.com/eomahony/Numberjack}{NumberJack} (Insight, Ireland) 
\item \href{https://www-users.york.ac.uk/peter.nightingale/savilerow//}{EssencePrime/SavilleRow} (UK) 
\item \href{https://cpmpy.readthedocs.io/en/latest/}{CPMpy} (KU Leuven)
\item Allow use of multiple back-end solvers
\item Compile model into variants for each solver
\item A priori solver independent model(CP, MIP, SAT)
\end{itemize}
\end{frame} 

\begin{frame}
\frametitle{Framework Process}
\begin{center}
\begin{tikzpicture}[xscale=2]
\node[shape=rectangle,fill=pantone127-4] (problem) at (2.5,5) {Problem};
\node (human) at (2.5,4) {Human};
\node[shape=rectangle,fill=pantone127-4] (model) at (2.5,3) {Model};
\node (compile) at (2.5,2) {Compile/Reformulate};
\node (solver1) at (1,1) {CP};
\node (solver2) at (2,1) {MIP};
\node (solver3) at (3,1) {SAT};
\node (solver4) at (4,1) {Other};
\node[shape=rectangle,fill=pantone157-8] (solution1) at (1,0) {Solution};
\node[shape=rectangle,fill=pantone157-8] (solution2) at (2,0) {Solution};
\node[shape=rectangle,fill=pantone157-8] (solution3) at (3,0) {Solution};
\node[shape=rectangle,fill=pantone157-8] (solution4) at (4,0) {Solution};
\draw[-] (problem) -- (human);
\draw[->] (human) -- (model);
\draw[-] (model) -- (compile);
\draw[-] (compile) -- (solver1);
\draw[-] (compile) -- (solver2);
\draw[-] (compile) -- (solver3);
\draw[-] (compile) -- (solver4);
\draw[->] (solver1) -- (solution1);
\draw[->] (solver2) -- (solution2);
\draw[->] (solver3) -- (solution3);
\draw[->] (solver4) -- (solution4);
\end{tikzpicture}
\end{center}
\end{frame}

% \begin{frame}
  % \frametitle{Do It Now!}
  % \begin{itemize}
  % \item Download and install Minizinc
  % \item \url{https://www.minizinc.org/}
  % \end{itemize}
% \end{frame}


\begin{frame}
\frametitle{Why use Puzzles as Examples?}
\begin{itemize}
\item Easy to understand the problem
\item Solvable by hand without specialized knowledge
\item Possible to compare automated to manual solving process
\end{itemize}

\begin{quote}
The puzzle, though inanimate, is presented as a solvable problem without lasting negative consequences, a very low-risk low-reward situation. By being a puzzle, the object is attempting to convince the user that it must be completed.
\end{quote}
{\tiny Source: \href{https://www.everydayrhetoric.com/2020/03/the-rhetoric-of-puzzles/}{Every Day Rhetoric}}
\end{frame}


\mode<all>{
\part{Basic Constraint Propagation}
}
\begin{frame}
\partpage
\end{frame}
\mode<all>{
\input{../sendmore/body}
}

\mode<all>{
\part{Global Constraints}
}
\begin{frame}
\partpage
\end{frame}
\mode<all>{
\input{../sudoku/body}
}

\clearpage
\mode<all>{
\part{Customizing Search}
}
\begin{frame}
\partpage
\end{frame}
\mode<all>{
\input{../nqueen/body}
}


%% \clearpage
%% \mode<all>{
%%   \part{Problem Decomposition}
%% }
%% \begin{frame}
%%   \partpage
%% \end{frame}
%% \mode<all>{
%%   \input{../party/body}
%%   }

%% \clearpage
%% \mode<all>{
%%   \part{More Global Constraints}
%% }
%% \begin{frame}
%%   \partpage
%% \end{frame}
%% \mode<all>{
%%   \input{../car/body}
%%   }

%% \clearpage
%% \mode<all>{
%%   \part{Methodology}
%% }
%% \begin{frame}
%%   \partpage
%% \end{frame}
%% \mode<all>{
%%   \input{../methodology/body}
%%   }

\clearpage
\mode<all>{
\part{What is missing?}
}
\begin{frame}
\partpage
\end{frame}

\begin{frame}
\frametitle{Many Specialized Topics}
\begin{itemize}
\item How to design efficient core engine
\item Hybrids with LP/MIP tools
\item Hybrids with SAT
\item Symmetry breaking
\item Use of MDD/BDD to encode sets of solutions
\item High level modelling tools
\item Debugging/visualization
\end{itemize}
\end{frame}

\begin{frame}
\frametitle{Reformulation}
\begin{itemize}
\item Just because the user has modelled it this way, it doesn't mean we have to solve it that way
\begin{itemize}
\item Replace some constraint(s) by other, equivalent constraints
\item Because we don't have that constraint in our system
\item For performance
\end{itemize}
\end{itemize}
\end{frame}

\begin{frame}
\frametitle{Learning}
\begin{itemize}
\item While solving the problem we can learn how to strengthen the model/search 
\begin{itemize}
\item Understand which constraints/method contribute to propagation and change schedule
\item Learn no-good constraints by explaining failure
\item Adapt search strategy based on search experience
\end{itemize}
\end{itemize}
\end{frame}

%% \begin{frame}
%% \frametitle{Refined Process}
%% \begin{center}
%% \begin{tikzpicture}
%% \node[shape=rectangle,fill=pantone127-4] (problem) at (3,7) {Problem};
%% \node (human) at (3,6) {Human};
%% \node[shape=rectangle,fill=pantone127-4] (model) at (3,5) {User Model};
%% \node (reformulation) at (3,4) {Reformulation};
%% \node[shape=rectangle,fill=pantone127-4] (implementation) at (3,3) {Implementation Model};
%% \node (solver) at (3,2) {Constraint Solver/Search};
%% \node[shape=rectangle,fill=pantone157-8] (solution) at (3,1) {Solution};
%% \draw[-] (problem) -- (human);
%% \draw[->] (human) -- (model);
%% \draw[-] (model) -- (reformulation);
%% \draw[->] (reformulation) -- (implementation);
%% \draw[-] (implementation) -- (solver);
%% \draw[->] (solver) -- (0,2) -- (0,3) -- (implementation);
%% \draw[->] (solver) -- (solution);
%% \end{tikzpicture}
%% \end{center}
%% \end{frame}

%\begin{frame}
%\frametitle{What is CP actually used for?}
%\url{http://hsimonis.wordpress.com}
%\includegraphics[width=10cm]{../introduction/images/blog}
%\end{frame}

%% \begin{frame}
%% \frametitle{Exercises}
%% \begin{itemize}
%% \item Install Constraint Solver of your Choice
%% \item MiniZinc (MiningZinc) will be used in later sessions
%% \item Run Sendmoremoney, Sudoku, Queens examples
%% \item Can you choose which propagator to use for alldifferent?
%% \item Can you use different search strategies?
%% \end{itemize}
%% \end{frame}

\begin{frame}
\frametitle{More Learning Resources}
\begin{itemize}
\item Survey of Methods, Resources, and Formats for Teaching Constraint Programming
\begin{itemize}
\item by Tejas Santanam, Helmut Simonis
\item \url{https://doi.org/10.48550/arXiv.2403.12717}
\item Based on survey of community for WTCP 2023
\item \url{https://hsimonis.github.io/WTCP2023/}
\end{itemize}
\end{itemize}
\end{frame}


%%   }

\clearpage
\mode<all>{
\part{What is missing?}
}
\begin{frame}
\partpage
\end{frame}

\begin{frame}
\frametitle{Many Specialized Topics}
\begin{itemize}
\item How to design efficient core engine
\item Hybrids with LP/MIP tools
\item Hybrids with SAT
\item Symmetry breaking
\item Use of MDD/BDD to encode sets of solutions
\item High level modelling tools
\item Debugging/visualization
\end{itemize}
\end{frame}

\begin{frame}
\frametitle{Reformulation}
\begin{itemize}
\item Just because the user has modelled it this way, it doesn't mean we have to solve it that way
\begin{itemize}
\item Replace some constraint(s) by other, equivalent constraints
\item Because we don't have that constraint in our system
\item For performance
\end{itemize}
\end{itemize}
\end{frame}

\begin{frame}
\frametitle{Learning}
\begin{itemize}
\item While solving the problem we can learn how to strengthen the model/search 
\begin{itemize}
\item Understand which constraints/method contribute to propagation and change schedule
\item Learn no-good constraints by explaining failure
\item Adapt search strategy based on search experience
\end{itemize}
\end{itemize}
\end{frame}

%% \begin{frame}
%% \frametitle{Refined Process}
%% \begin{center}
%% \begin{tikzpicture}
%% \node[shape=rectangle,fill=pantone127-4] (problem) at (3,7) {Problem};
%% \node (human) at (3,6) {Human};
%% \node[shape=rectangle,fill=pantone127-4] (model) at (3,5) {User Model};
%% \node (reformulation) at (3,4) {Reformulation};
%% \node[shape=rectangle,fill=pantone127-4] (implementation) at (3,3) {Implementation Model};
%% \node (solver) at (3,2) {Constraint Solver/Search};
%% \node[shape=rectangle,fill=pantone157-8] (solution) at (3,1) {Solution};
%% \draw[-] (problem) -- (human);
%% \draw[->] (human) -- (model);
%% \draw[-] (model) -- (reformulation);
%% \draw[->] (reformulation) -- (implementation);
%% \draw[-] (implementation) -- (solver);
%% \draw[->] (solver) -- (0,2) -- (0,3) -- (implementation);
%% \draw[->] (solver) -- (solution);
%% \end{tikzpicture}
%% \end{center}
%% \end{frame}

%\begin{frame}
%\frametitle{What is CP actually used for?}
%\url{http://hsimonis.wordpress.com}
%\includegraphics[width=10cm]{../introduction/images/blog}
%\end{frame}

%% \begin{frame}
%% \frametitle{Exercises}
%% \begin{itemize}
%% \item Install Constraint Solver of your Choice
%% \item MiniZinc (MiningZinc) will be used in later sessions
%% \item Run Sendmoremoney, Sudoku, Queens examples
%% \item Can you choose which propagator to use for alldifferent?
%% \item Can you use different search strategies?
%% \end{itemize}
%% \end{frame}

\begin{frame}
\frametitle{More Learning Resources}
\begin{itemize}
\item Survey of Methods, Resources, and Formats for Teaching Constraint Programming
\begin{itemize}
\item by Tejas Santanam, Helmut Simonis
\item \url{https://doi.org/10.48550/arXiv.2403.12717}
\item Based on survey of community for WTCP 2023
\item \url{https://hsimonis.github.io/WTCP2023/}
\end{itemize}
\end{itemize}
\end{frame}


%%   }

\clearpage
\mode<all>{
\part{What is missing?}
}
\begin{frame}
\partpage
\end{frame}

\begin{frame}
\frametitle{Many Specialized Topics}
\begin{itemize}
\item How to design efficient core engine
\item Hybrids with LP/MIP tools
\item Hybrids with SAT
\item Symmetry breaking
\item Use of MDD/BDD to encode sets of solutions
\item High level modelling tools
\item Debugging/visualization
\end{itemize}
\end{frame}

\begin{frame}
\frametitle{Reformulation}
\begin{itemize}
\item Just because the user has modelled it this way, it doesn't mean we have to solve it that way
\begin{itemize}
\item Replace some constraint(s) by other, equivalent constraints
\item Because we don't have that constraint in our system
\item For performance
\end{itemize}
\end{itemize}
\end{frame}

\begin{frame}
\frametitle{Learning}
\begin{itemize}
\item While solving the problem we can learn how to strengthen the model/search 
\begin{itemize}
\item Understand which constraints/method contribute to propagation and change schedule
\item Learn no-good constraints by explaining failure
\item Adapt search strategy based on search experience
\end{itemize}
\end{itemize}
\end{frame}

%% \begin{frame}
%% \frametitle{Refined Process}
%% \begin{center}
%% \begin{tikzpicture}
%% \node[shape=rectangle,fill=pantone127-4] (problem) at (3,7) {Problem};
%% \node (human) at (3,6) {Human};
%% \node[shape=rectangle,fill=pantone127-4] (model) at (3,5) {User Model};
%% \node (reformulation) at (3,4) {Reformulation};
%% \node[shape=rectangle,fill=pantone127-4] (implementation) at (3,3) {Implementation Model};
%% \node (solver) at (3,2) {Constraint Solver/Search};
%% \node[shape=rectangle,fill=pantone157-8] (solution) at (3,1) {Solution};
%% \draw[-] (problem) -- (human);
%% \draw[->] (human) -- (model);
%% \draw[-] (model) -- (reformulation);
%% \draw[->] (reformulation) -- (implementation);
%% \draw[-] (implementation) -- (solver);
%% \draw[->] (solver) -- (0,2) -- (0,3) -- (implementation);
%% \draw[->] (solver) -- (solution);
%% \end{tikzpicture}
%% \end{center}
%% \end{frame}

%\begin{frame}
%\frametitle{What is CP actually used for?}
%\url{http://hsimonis.wordpress.com}
%\includegraphics[width=10cm]{../introduction/images/blog}
%\end{frame}

%% \begin{frame}
%% \frametitle{Exercises}
%% \begin{itemize}
%% \item Install Constraint Solver of your Choice
%% \item MiniZinc (MiningZinc) will be used in later sessions
%% \item Run Sendmoremoney, Sudoku, Queens examples
%% \item Can you choose which propagator to use for alldifferent?
%% \item Can you use different search strategies?
%% \end{itemize}
%% \end{frame}

\begin{frame}
\frametitle{More Learning Resources}
\begin{itemize}
\item Survey of Methods, Resources, and Formats for Teaching Constraint Programming
\begin{itemize}
\item by Tejas Santanam, Helmut Simonis
\item \url{https://doi.org/10.48550/arXiv.2403.12717}
\item Based on survey of community for WTCP 2023
\item \url{https://hsimonis.github.io/WTCP2023/}
\end{itemize}
\end{itemize}
\end{frame}


%%   }

%% \clearpage
%% \mode<all>{
%%   \part{More Global Constraints}
%% }
%% \begin{frame}
%%   \partpage
%% \end{frame}
%% \mode<all>{
%%   \begin{frame}
\frametitle{Objectives}
\begin{itemize}
\item Overview of Core Constraint Programming
\item Three Main Concepts
\begin{itemize}
\item Constraint Propagation
\item Global Constraints
\item Customizing Search
\end{itemize}
%\item Get Some Experience with MiniZinc
\item Topics will be treated in more detail in later parts of the school
\item Based on Examples, not Formal Description
\end{itemize}
\end{frame}

\begin{frame}
\frametitle{Outline}
\begin{itemize}
\item Why Constraint Programming?
\item Constraint Propagation
\item Global Constraints
\item Customizing Search
%\item Choices in Modelling
%\item Decomposition
%\item More Global Constraints
%\item Methodology
%\item What is missing?
\end{itemize}
\end{frame}

% \begin{frame}
  % \frametitle{Using MiniZinc IDE}
  % \begin{itemize}
  % \item Developed in the Australian NICTA project
  % \item Maintained by Monash University
  % \item Modelling tool with multiple back-end solvers
  % \item Available from \url{https://www.minizinc.org/}
  % \end{itemize}
% \end{frame}


% \begin{frame}
% \frametitle{Examples in ECLiPSe}
% \begin{itemize}
% \item Open sourced constraint programming language
% \item Development goes back to 1985
% \item ECRC, ICL, IC-Parc, PTL, Cisco
% \item \url{https://eclipseclp.org/}
% \item Specialities
% \begin{itemize}
% \item Develop new solvers for specific domains
% \item Integration with MIP
% \end{itemize}
% \item Not included in bundled MiniZinc IDE
% \item Specialized visualization tools used here
  % \begin{itemize}
  % \item CP-Viz, Simonis et al. 2010
    % \end{itemize}
% \end{itemize}
% \end{frame}

\begin{frame}
\frametitle{Tutorial Based on ECLiPSe ELearning Course}
\begin{itemize}
\item Self-study course in constraint programming
\item Supported by Cisco Systems and Silicon Valley Community Foundation
\item Multi-media format, video lectures, slides, handout etc
\item {\small \url{https://eclipseclp.org/ELearning/index.html}}
\end{itemize}
\end{frame}

\begin{frame}
\frametitle{Also Part of CRT-AI Constraint Week}
\begin{itemize}
\item Annual one week course on CP and Optimization in Ireland
\item Part of national training program for PhD students in AI
\item \url{https://www.crt-ai.ie/}
\end{itemize}
\end{frame}



\begin{frame}
\frametitle{Constraint Programming - in a nutshell}
\begin{itemize}
\item Declarative description of problems with
\begin{itemize}
\item {\em Variables} which range over (finite) sets of values
\item {\em Constraints} over subsets of variables which restrict possible value combinations
\item A {\em solution} is a value assignment which satisfies all constraints
\end{itemize}

\item Constraint propagation/reasoning
\begin{itemize}
\item Removing inconsistent values for variables
\item Detect failure if constraint can not be satisfied
\item Interaction of constraints via shared variables
\item Incomplete
\end{itemize}

\item Search
\begin{itemize}
\item User controlled assignment of values to variables
\item Each step triggers constraint propagation 
\end{itemize}
\item Different domains require/allow different methods
\end{itemize}
\end{frame}

\begin{frame}
  \frametitle{Constraint Programming is Different}
  \begin{itemize}
  \item Declarative Programming
    \begin{itemize}
    \item Concentrate on what you want
      \item Not how to get there
      \item Program != Algorithm
      \item Program = Model
    \end{itemize}
    \item Applied to Combinatorial Problems
      \begin{itemize}
        \item No complete polynomial algorithms known (exist?)
        \item CP less ad-hoc than heuristics
        \item Models can evolve
  \end{itemize}
  \end{itemize}
  \end{frame}
    
\begin{frame}
  \frametitle{A Subtractive Process}
  \begin{textblock}{4}(8,-3)
    \includegraphics[width=4cm]{../introduction/images/stages}
  \end{textblock}
  \vfill
  \begin{quote}
    ``Oh, bosh, as Mr. Ruskin says. Sculpture, per se, is the simplest thing in the world. All you have to do is to take a big chunk of marble and a hammer and chisel, make up your mind what you are about to create and chip off all the marble you don't want.''-Paris Gaulois.
  \end{quote}
  
  {\tiny Source: \url{https://quoteinvestigator.com/2014/06/22/chip-away/}}
\end{frame}

\begin{frame}
\frametitle{Basic Process}
\begin{center}
\begin{tikzpicture}
\node[shape=rectangle,fill=pantone127-4] (problem) at (2,5) {Problem};
\node (human) at (2,4) {Human};
\node[shape=rectangle,fill=pantone127-4] (model) at (2,3) {Model};
\node (solver) at (2,2) {Constraint Solver/Search};
\node[shape=rectangle,fill=pantone157-8] (solution) at (2,1) {Solution};
\draw[-] (problem) -- (human);
\draw[->] (human) -- (model);
\draw[-] (model) -- (solver);
\draw[->] (solver) -- (solution);
\end{tikzpicture}
\end{center}
\end{frame}

\begin{frame}
\frametitle{More Realistic}
\begin{center}
\begin{tikzpicture}[xscale=4.0]
\node[shape=rectangle,fill=pantone127-4] (problem) at (2,5) {Problem};
\node (human) at (2,4) {Human};
\node[shape=rectangle,fill=pantone127-4] (model) at (2,3) {Model};
\node (solver) at (2,2) {Constraint Solver/Search};
\node[shape=rectangle,fill=pantone24-8] (hangs) at (1,1) {Hangs};
\node[shape=rectangle,fill=pantone157-8] (solution) at (2,1) {Solution};
\node[shape=rectangle,fill=pantone24-8] (wrong) at (3,1) {Wrong Solution};
\draw[-] (problem) -- (human);
\draw[->] (human) -- (model);
\draw[-] (model) -- (solver);
\draw[->] (solver) -- (hangs);
\draw[->] (solver) -- (solution);
\draw[->] (solver) -- (wrong);
\draw[->] (hangs) -- (1,4) -- (human);
\draw[->] (wrong) -- (3,4) -- (human);
\end{tikzpicture}
\end{center}
\end{frame}

\begin{frame}
\frametitle{Dual Role of Model}
\begin{itemize}
\item Allows Human to Express Problem
\begin{itemize}
\item Close to Problem Domain
\item Constraints as Abstractions
\end{itemize}
\item Allows Solver to Execute
\begin{itemize}
\item Variables as Communication Mechanism
\item Constraints as Algorithms
\end{itemize}
\end{itemize}
\end{frame}

\begin{frame}
\frametitle{Modelling Frameworks}
\begin{itemize}
\item \href{https://www.minizinc.org/}{MiniZinc} (NICTA, Monash University, Australia)
\item \href{https://github.com/eomahony/Numberjack}{NumberJack} (Insight, Ireland) 
\item \href{https://www-users.york.ac.uk/peter.nightingale/savilerow//}{EssencePrime/SavilleRow} (UK) 
\item \href{https://cpmpy.readthedocs.io/en/latest/}{CPMpy} (KU Leuven)
\item Allow use of multiple back-end solvers
\item Compile model into variants for each solver
\item A priori solver independent model(CP, MIP, SAT)
\end{itemize}
\end{frame} 

\begin{frame}
\frametitle{Framework Process}
\begin{center}
\begin{tikzpicture}[xscale=2]
\node[shape=rectangle,fill=pantone127-4] (problem) at (2.5,5) {Problem};
\node (human) at (2.5,4) {Human};
\node[shape=rectangle,fill=pantone127-4] (model) at (2.5,3) {Model};
\node (compile) at (2.5,2) {Compile/Reformulate};
\node (solver1) at (1,1) {CP};
\node (solver2) at (2,1) {MIP};
\node (solver3) at (3,1) {SAT};
\node (solver4) at (4,1) {Other};
\node[shape=rectangle,fill=pantone157-8] (solution1) at (1,0) {Solution};
\node[shape=rectangle,fill=pantone157-8] (solution2) at (2,0) {Solution};
\node[shape=rectangle,fill=pantone157-8] (solution3) at (3,0) {Solution};
\node[shape=rectangle,fill=pantone157-8] (solution4) at (4,0) {Solution};
\draw[-] (problem) -- (human);
\draw[->] (human) -- (model);
\draw[-] (model) -- (compile);
\draw[-] (compile) -- (solver1);
\draw[-] (compile) -- (solver2);
\draw[-] (compile) -- (solver3);
\draw[-] (compile) -- (solver4);
\draw[->] (solver1) -- (solution1);
\draw[->] (solver2) -- (solution2);
\draw[->] (solver3) -- (solution3);
\draw[->] (solver4) -- (solution4);
\end{tikzpicture}
\end{center}
\end{frame}

% \begin{frame}
  % \frametitle{Do It Now!}
  % \begin{itemize}
  % \item Download and install Minizinc
  % \item \url{https://www.minizinc.org/}
  % \end{itemize}
% \end{frame}


\begin{frame}
\frametitle{Why use Puzzles as Examples?}
\begin{itemize}
\item Easy to understand the problem
\item Solvable by hand without specialized knowledge
\item Possible to compare automated to manual solving process
\end{itemize}

\begin{quote}
The puzzle, though inanimate, is presented as a solvable problem without lasting negative consequences, a very low-risk low-reward situation. By being a puzzle, the object is attempting to convince the user that it must be completed.
\end{quote}
{\tiny Source: \href{https://www.everydayrhetoric.com/2020/03/the-rhetoric-of-puzzles/}{Every Day Rhetoric}}
\end{frame}


\mode<all>{
\part{Basic Constraint Propagation}
}
\begin{frame}
\partpage
\end{frame}
\mode<all>{
\begin{frame}
\frametitle{Objectives}
\begin{itemize}
\item Overview of Core Constraint Programming
\item Three Main Concepts
\begin{itemize}
\item Constraint Propagation
\item Global Constraints
\item Customizing Search
\end{itemize}
%\item Get Some Experience with MiniZinc
\item Topics will be treated in more detail in later parts of the school
\item Based on Examples, not Formal Description
\end{itemize}
\end{frame}

\begin{frame}
\frametitle{Outline}
\begin{itemize}
\item Why Constraint Programming?
\item Constraint Propagation
\item Global Constraints
\item Customizing Search
%\item Choices in Modelling
%\item Decomposition
%\item More Global Constraints
%\item Methodology
%\item What is missing?
\end{itemize}
\end{frame}

% \begin{frame}
  % \frametitle{Using MiniZinc IDE}
  % \begin{itemize}
  % \item Developed in the Australian NICTA project
  % \item Maintained by Monash University
  % \item Modelling tool with multiple back-end solvers
  % \item Available from \url{https://www.minizinc.org/}
  % \end{itemize}
% \end{frame}


% \begin{frame}
% \frametitle{Examples in ECLiPSe}
% \begin{itemize}
% \item Open sourced constraint programming language
% \item Development goes back to 1985
% \item ECRC, ICL, IC-Parc, PTL, Cisco
% \item \url{https://eclipseclp.org/}
% \item Specialities
% \begin{itemize}
% \item Develop new solvers for specific domains
% \item Integration with MIP
% \end{itemize}
% \item Not included in bundled MiniZinc IDE
% \item Specialized visualization tools used here
  % \begin{itemize}
  % \item CP-Viz, Simonis et al. 2010
    % \end{itemize}
% \end{itemize}
% \end{frame}

\begin{frame}
\frametitle{Tutorial Based on ECLiPSe ELearning Course}
\begin{itemize}
\item Self-study course in constraint programming
\item Supported by Cisco Systems and Silicon Valley Community Foundation
\item Multi-media format, video lectures, slides, handout etc
\item {\small \url{https://eclipseclp.org/ELearning/index.html}}
\end{itemize}
\end{frame}

\begin{frame}
\frametitle{Also Part of CRT-AI Constraint Week}
\begin{itemize}
\item Annual one week course on CP and Optimization in Ireland
\item Part of national training program for PhD students in AI
\item \url{https://www.crt-ai.ie/}
\end{itemize}
\end{frame}



\begin{frame}
\frametitle{Constraint Programming - in a nutshell}
\begin{itemize}
\item Declarative description of problems with
\begin{itemize}
\item {\em Variables} which range over (finite) sets of values
\item {\em Constraints} over subsets of variables which restrict possible value combinations
\item A {\em solution} is a value assignment which satisfies all constraints
\end{itemize}

\item Constraint propagation/reasoning
\begin{itemize}
\item Removing inconsistent values for variables
\item Detect failure if constraint can not be satisfied
\item Interaction of constraints via shared variables
\item Incomplete
\end{itemize}

\item Search
\begin{itemize}
\item User controlled assignment of values to variables
\item Each step triggers constraint propagation 
\end{itemize}
\item Different domains require/allow different methods
\end{itemize}
\end{frame}

\begin{frame}
  \frametitle{Constraint Programming is Different}
  \begin{itemize}
  \item Declarative Programming
    \begin{itemize}
    \item Concentrate on what you want
      \item Not how to get there
      \item Program != Algorithm
      \item Program = Model
    \end{itemize}
    \item Applied to Combinatorial Problems
      \begin{itemize}
        \item No complete polynomial algorithms known (exist?)
        \item CP less ad-hoc than heuristics
        \item Models can evolve
  \end{itemize}
  \end{itemize}
  \end{frame}
    
\begin{frame}
  \frametitle{A Subtractive Process}
  \begin{textblock}{4}(8,-3)
    \includegraphics[width=4cm]{../introduction/images/stages}
  \end{textblock}
  \vfill
  \begin{quote}
    ``Oh, bosh, as Mr. Ruskin says. Sculpture, per se, is the simplest thing in the world. All you have to do is to take a big chunk of marble and a hammer and chisel, make up your mind what you are about to create and chip off all the marble you don't want.''-Paris Gaulois.
  \end{quote}
  
  {\tiny Source: \url{https://quoteinvestigator.com/2014/06/22/chip-away/}}
\end{frame}

\begin{frame}
\frametitle{Basic Process}
\begin{center}
\begin{tikzpicture}
\node[shape=rectangle,fill=pantone127-4] (problem) at (2,5) {Problem};
\node (human) at (2,4) {Human};
\node[shape=rectangle,fill=pantone127-4] (model) at (2,3) {Model};
\node (solver) at (2,2) {Constraint Solver/Search};
\node[shape=rectangle,fill=pantone157-8] (solution) at (2,1) {Solution};
\draw[-] (problem) -- (human);
\draw[->] (human) -- (model);
\draw[-] (model) -- (solver);
\draw[->] (solver) -- (solution);
\end{tikzpicture}
\end{center}
\end{frame}

\begin{frame}
\frametitle{More Realistic}
\begin{center}
\begin{tikzpicture}[xscale=4.0]
\node[shape=rectangle,fill=pantone127-4] (problem) at (2,5) {Problem};
\node (human) at (2,4) {Human};
\node[shape=rectangle,fill=pantone127-4] (model) at (2,3) {Model};
\node (solver) at (2,2) {Constraint Solver/Search};
\node[shape=rectangle,fill=pantone24-8] (hangs) at (1,1) {Hangs};
\node[shape=rectangle,fill=pantone157-8] (solution) at (2,1) {Solution};
\node[shape=rectangle,fill=pantone24-8] (wrong) at (3,1) {Wrong Solution};
\draw[-] (problem) -- (human);
\draw[->] (human) -- (model);
\draw[-] (model) -- (solver);
\draw[->] (solver) -- (hangs);
\draw[->] (solver) -- (solution);
\draw[->] (solver) -- (wrong);
\draw[->] (hangs) -- (1,4) -- (human);
\draw[->] (wrong) -- (3,4) -- (human);
\end{tikzpicture}
\end{center}
\end{frame}

\begin{frame}
\frametitle{Dual Role of Model}
\begin{itemize}
\item Allows Human to Express Problem
\begin{itemize}
\item Close to Problem Domain
\item Constraints as Abstractions
\end{itemize}
\item Allows Solver to Execute
\begin{itemize}
\item Variables as Communication Mechanism
\item Constraints as Algorithms
\end{itemize}
\end{itemize}
\end{frame}

\begin{frame}
\frametitle{Modelling Frameworks}
\begin{itemize}
\item \href{https://www.minizinc.org/}{MiniZinc} (NICTA, Monash University, Australia)
\item \href{https://github.com/eomahony/Numberjack}{NumberJack} (Insight, Ireland) 
\item \href{https://www-users.york.ac.uk/peter.nightingale/savilerow//}{EssencePrime/SavilleRow} (UK) 
\item \href{https://cpmpy.readthedocs.io/en/latest/}{CPMpy} (KU Leuven)
\item Allow use of multiple back-end solvers
\item Compile model into variants for each solver
\item A priori solver independent model(CP, MIP, SAT)
\end{itemize}
\end{frame} 

\begin{frame}
\frametitle{Framework Process}
\begin{center}
\begin{tikzpicture}[xscale=2]
\node[shape=rectangle,fill=pantone127-4] (problem) at (2.5,5) {Problem};
\node (human) at (2.5,4) {Human};
\node[shape=rectangle,fill=pantone127-4] (model) at (2.5,3) {Model};
\node (compile) at (2.5,2) {Compile/Reformulate};
\node (solver1) at (1,1) {CP};
\node (solver2) at (2,1) {MIP};
\node (solver3) at (3,1) {SAT};
\node (solver4) at (4,1) {Other};
\node[shape=rectangle,fill=pantone157-8] (solution1) at (1,0) {Solution};
\node[shape=rectangle,fill=pantone157-8] (solution2) at (2,0) {Solution};
\node[shape=rectangle,fill=pantone157-8] (solution3) at (3,0) {Solution};
\node[shape=rectangle,fill=pantone157-8] (solution4) at (4,0) {Solution};
\draw[-] (problem) -- (human);
\draw[->] (human) -- (model);
\draw[-] (model) -- (compile);
\draw[-] (compile) -- (solver1);
\draw[-] (compile) -- (solver2);
\draw[-] (compile) -- (solver3);
\draw[-] (compile) -- (solver4);
\draw[->] (solver1) -- (solution1);
\draw[->] (solver2) -- (solution2);
\draw[->] (solver3) -- (solution3);
\draw[->] (solver4) -- (solution4);
\end{tikzpicture}
\end{center}
\end{frame}

% \begin{frame}
  % \frametitle{Do It Now!}
  % \begin{itemize}
  % \item Download and install Minizinc
  % \item \url{https://www.minizinc.org/}
  % \end{itemize}
% \end{frame}


\begin{frame}
\frametitle{Why use Puzzles as Examples?}
\begin{itemize}
\item Easy to understand the problem
\item Solvable by hand without specialized knowledge
\item Possible to compare automated to manual solving process
\end{itemize}

\begin{quote}
The puzzle, though inanimate, is presented as a solvable problem without lasting negative consequences, a very low-risk low-reward situation. By being a puzzle, the object is attempting to convince the user that it must be completed.
\end{quote}
{\tiny Source: \href{https://www.everydayrhetoric.com/2020/03/the-rhetoric-of-puzzles/}{Every Day Rhetoric}}
\end{frame}


\mode<all>{
\part{Basic Constraint Propagation}
}
\begin{frame}
\partpage
\end{frame}
\mode<all>{
\begin{frame}
\frametitle{Objectives}
\begin{itemize}
\item Overview of Core Constraint Programming
\item Three Main Concepts
\begin{itemize}
\item Constraint Propagation
\item Global Constraints
\item Customizing Search
\end{itemize}
%\item Get Some Experience with MiniZinc
\item Topics will be treated in more detail in later parts of the school
\item Based on Examples, not Formal Description
\end{itemize}
\end{frame}

\begin{frame}
\frametitle{Outline}
\begin{itemize}
\item Why Constraint Programming?
\item Constraint Propagation
\item Global Constraints
\item Customizing Search
%\item Choices in Modelling
%\item Decomposition
%\item More Global Constraints
%\item Methodology
%\item What is missing?
\end{itemize}
\end{frame}

% \begin{frame}
  % \frametitle{Using MiniZinc IDE}
  % \begin{itemize}
  % \item Developed in the Australian NICTA project
  % \item Maintained by Monash University
  % \item Modelling tool with multiple back-end solvers
  % \item Available from \url{https://www.minizinc.org/}
  % \end{itemize}
% \end{frame}


% \begin{frame}
% \frametitle{Examples in ECLiPSe}
% \begin{itemize}
% \item Open sourced constraint programming language
% \item Development goes back to 1985
% \item ECRC, ICL, IC-Parc, PTL, Cisco
% \item \url{https://eclipseclp.org/}
% \item Specialities
% \begin{itemize}
% \item Develop new solvers for specific domains
% \item Integration with MIP
% \end{itemize}
% \item Not included in bundled MiniZinc IDE
% \item Specialized visualization tools used here
  % \begin{itemize}
  % \item CP-Viz, Simonis et al. 2010
    % \end{itemize}
% \end{itemize}
% \end{frame}

\begin{frame}
\frametitle{Tutorial Based on ECLiPSe ELearning Course}
\begin{itemize}
\item Self-study course in constraint programming
\item Supported by Cisco Systems and Silicon Valley Community Foundation
\item Multi-media format, video lectures, slides, handout etc
\item {\small \url{https://eclipseclp.org/ELearning/index.html}}
\end{itemize}
\end{frame}

\begin{frame}
\frametitle{Also Part of CRT-AI Constraint Week}
\begin{itemize}
\item Annual one week course on CP and Optimization in Ireland
\item Part of national training program for PhD students in AI
\item \url{https://www.crt-ai.ie/}
\end{itemize}
\end{frame}



\begin{frame}
\frametitle{Constraint Programming - in a nutshell}
\begin{itemize}
\item Declarative description of problems with
\begin{itemize}
\item {\em Variables} which range over (finite) sets of values
\item {\em Constraints} over subsets of variables which restrict possible value combinations
\item A {\em solution} is a value assignment which satisfies all constraints
\end{itemize}

\item Constraint propagation/reasoning
\begin{itemize}
\item Removing inconsistent values for variables
\item Detect failure if constraint can not be satisfied
\item Interaction of constraints via shared variables
\item Incomplete
\end{itemize}

\item Search
\begin{itemize}
\item User controlled assignment of values to variables
\item Each step triggers constraint propagation 
\end{itemize}
\item Different domains require/allow different methods
\end{itemize}
\end{frame}

\begin{frame}
  \frametitle{Constraint Programming is Different}
  \begin{itemize}
  \item Declarative Programming
    \begin{itemize}
    \item Concentrate on what you want
      \item Not how to get there
      \item Program != Algorithm
      \item Program = Model
    \end{itemize}
    \item Applied to Combinatorial Problems
      \begin{itemize}
        \item No complete polynomial algorithms known (exist?)
        \item CP less ad-hoc than heuristics
        \item Models can evolve
  \end{itemize}
  \end{itemize}
  \end{frame}
    
\begin{frame}
  \frametitle{A Subtractive Process}
  \begin{textblock}{4}(8,-3)
    \includegraphics[width=4cm]{../introduction/images/stages}
  \end{textblock}
  \vfill
  \begin{quote}
    ``Oh, bosh, as Mr. Ruskin says. Sculpture, per se, is the simplest thing in the world. All you have to do is to take a big chunk of marble and a hammer and chisel, make up your mind what you are about to create and chip off all the marble you don't want.''-Paris Gaulois.
  \end{quote}
  
  {\tiny Source: \url{https://quoteinvestigator.com/2014/06/22/chip-away/}}
\end{frame}

\begin{frame}
\frametitle{Basic Process}
\begin{center}
\begin{tikzpicture}
\node[shape=rectangle,fill=pantone127-4] (problem) at (2,5) {Problem};
\node (human) at (2,4) {Human};
\node[shape=rectangle,fill=pantone127-4] (model) at (2,3) {Model};
\node (solver) at (2,2) {Constraint Solver/Search};
\node[shape=rectangle,fill=pantone157-8] (solution) at (2,1) {Solution};
\draw[-] (problem) -- (human);
\draw[->] (human) -- (model);
\draw[-] (model) -- (solver);
\draw[->] (solver) -- (solution);
\end{tikzpicture}
\end{center}
\end{frame}

\begin{frame}
\frametitle{More Realistic}
\begin{center}
\begin{tikzpicture}[xscale=4.0]
\node[shape=rectangle,fill=pantone127-4] (problem) at (2,5) {Problem};
\node (human) at (2,4) {Human};
\node[shape=rectangle,fill=pantone127-4] (model) at (2,3) {Model};
\node (solver) at (2,2) {Constraint Solver/Search};
\node[shape=rectangle,fill=pantone24-8] (hangs) at (1,1) {Hangs};
\node[shape=rectangle,fill=pantone157-8] (solution) at (2,1) {Solution};
\node[shape=rectangle,fill=pantone24-8] (wrong) at (3,1) {Wrong Solution};
\draw[-] (problem) -- (human);
\draw[->] (human) -- (model);
\draw[-] (model) -- (solver);
\draw[->] (solver) -- (hangs);
\draw[->] (solver) -- (solution);
\draw[->] (solver) -- (wrong);
\draw[->] (hangs) -- (1,4) -- (human);
\draw[->] (wrong) -- (3,4) -- (human);
\end{tikzpicture}
\end{center}
\end{frame}

\begin{frame}
\frametitle{Dual Role of Model}
\begin{itemize}
\item Allows Human to Express Problem
\begin{itemize}
\item Close to Problem Domain
\item Constraints as Abstractions
\end{itemize}
\item Allows Solver to Execute
\begin{itemize}
\item Variables as Communication Mechanism
\item Constraints as Algorithms
\end{itemize}
\end{itemize}
\end{frame}

\begin{frame}
\frametitle{Modelling Frameworks}
\begin{itemize}
\item \href{https://www.minizinc.org/}{MiniZinc} (NICTA, Monash University, Australia)
\item \href{https://github.com/eomahony/Numberjack}{NumberJack} (Insight, Ireland) 
\item \href{https://www-users.york.ac.uk/peter.nightingale/savilerow//}{EssencePrime/SavilleRow} (UK) 
\item \href{https://cpmpy.readthedocs.io/en/latest/}{CPMpy} (KU Leuven)
\item Allow use of multiple back-end solvers
\item Compile model into variants for each solver
\item A priori solver independent model(CP, MIP, SAT)
\end{itemize}
\end{frame} 

\begin{frame}
\frametitle{Framework Process}
\begin{center}
\begin{tikzpicture}[xscale=2]
\node[shape=rectangle,fill=pantone127-4] (problem) at (2.5,5) {Problem};
\node (human) at (2.5,4) {Human};
\node[shape=rectangle,fill=pantone127-4] (model) at (2.5,3) {Model};
\node (compile) at (2.5,2) {Compile/Reformulate};
\node (solver1) at (1,1) {CP};
\node (solver2) at (2,1) {MIP};
\node (solver3) at (3,1) {SAT};
\node (solver4) at (4,1) {Other};
\node[shape=rectangle,fill=pantone157-8] (solution1) at (1,0) {Solution};
\node[shape=rectangle,fill=pantone157-8] (solution2) at (2,0) {Solution};
\node[shape=rectangle,fill=pantone157-8] (solution3) at (3,0) {Solution};
\node[shape=rectangle,fill=pantone157-8] (solution4) at (4,0) {Solution};
\draw[-] (problem) -- (human);
\draw[->] (human) -- (model);
\draw[-] (model) -- (compile);
\draw[-] (compile) -- (solver1);
\draw[-] (compile) -- (solver2);
\draw[-] (compile) -- (solver3);
\draw[-] (compile) -- (solver4);
\draw[->] (solver1) -- (solution1);
\draw[->] (solver2) -- (solution2);
\draw[->] (solver3) -- (solution3);
\draw[->] (solver4) -- (solution4);
\end{tikzpicture}
\end{center}
\end{frame}

% \begin{frame}
  % \frametitle{Do It Now!}
  % \begin{itemize}
  % \item Download and install Minizinc
  % \item \url{https://www.minizinc.org/}
  % \end{itemize}
% \end{frame}


\begin{frame}
\frametitle{Why use Puzzles as Examples?}
\begin{itemize}
\item Easy to understand the problem
\item Solvable by hand without specialized knowledge
\item Possible to compare automated to manual solving process
\end{itemize}

\begin{quote}
The puzzle, though inanimate, is presented as a solvable problem without lasting negative consequences, a very low-risk low-reward situation. By being a puzzle, the object is attempting to convince the user that it must be completed.
\end{quote}
{\tiny Source: \href{https://www.everydayrhetoric.com/2020/03/the-rhetoric-of-puzzles/}{Every Day Rhetoric}}
\end{frame}


\mode<all>{
\part{Basic Constraint Propagation}
}
\begin{frame}
\partpage
\end{frame}
\mode<all>{
\input{../sendmore/body}
}

\mode<all>{
\part{Global Constraints}
}
\begin{frame}
\partpage
\end{frame}
\mode<all>{
\input{../sudoku/body}
}

\clearpage
\mode<all>{
\part{Customizing Search}
}
\begin{frame}
\partpage
\end{frame}
\mode<all>{
\input{../nqueen/body}
}


%% \clearpage
%% \mode<all>{
%%   \part{Problem Decomposition}
%% }
%% \begin{frame}
%%   \partpage
%% \end{frame}
%% \mode<all>{
%%   \input{../party/body}
%%   }

%% \clearpage
%% \mode<all>{
%%   \part{More Global Constraints}
%% }
%% \begin{frame}
%%   \partpage
%% \end{frame}
%% \mode<all>{
%%   \input{../car/body}
%%   }

%% \clearpage
%% \mode<all>{
%%   \part{Methodology}
%% }
%% \begin{frame}
%%   \partpage
%% \end{frame}
%% \mode<all>{
%%   \input{../methodology/body}
%%   }

\clearpage
\mode<all>{
\part{What is missing?}
}
\begin{frame}
\partpage
\end{frame}

\begin{frame}
\frametitle{Many Specialized Topics}
\begin{itemize}
\item How to design efficient core engine
\item Hybrids with LP/MIP tools
\item Hybrids with SAT
\item Symmetry breaking
\item Use of MDD/BDD to encode sets of solutions
\item High level modelling tools
\item Debugging/visualization
\end{itemize}
\end{frame}

\begin{frame}
\frametitle{Reformulation}
\begin{itemize}
\item Just because the user has modelled it this way, it doesn't mean we have to solve it that way
\begin{itemize}
\item Replace some constraint(s) by other, equivalent constraints
\item Because we don't have that constraint in our system
\item For performance
\end{itemize}
\end{itemize}
\end{frame}

\begin{frame}
\frametitle{Learning}
\begin{itemize}
\item While solving the problem we can learn how to strengthen the model/search 
\begin{itemize}
\item Understand which constraints/method contribute to propagation and change schedule
\item Learn no-good constraints by explaining failure
\item Adapt search strategy based on search experience
\end{itemize}
\end{itemize}
\end{frame}

%% \begin{frame}
%% \frametitle{Refined Process}
%% \begin{center}
%% \begin{tikzpicture}
%% \node[shape=rectangle,fill=pantone127-4] (problem) at (3,7) {Problem};
%% \node (human) at (3,6) {Human};
%% \node[shape=rectangle,fill=pantone127-4] (model) at (3,5) {User Model};
%% \node (reformulation) at (3,4) {Reformulation};
%% \node[shape=rectangle,fill=pantone127-4] (implementation) at (3,3) {Implementation Model};
%% \node (solver) at (3,2) {Constraint Solver/Search};
%% \node[shape=rectangle,fill=pantone157-8] (solution) at (3,1) {Solution};
%% \draw[-] (problem) -- (human);
%% \draw[->] (human) -- (model);
%% \draw[-] (model) -- (reformulation);
%% \draw[->] (reformulation) -- (implementation);
%% \draw[-] (implementation) -- (solver);
%% \draw[->] (solver) -- (0,2) -- (0,3) -- (implementation);
%% \draw[->] (solver) -- (solution);
%% \end{tikzpicture}
%% \end{center}
%% \end{frame}

%\begin{frame}
%\frametitle{What is CP actually used for?}
%\url{http://hsimonis.wordpress.com}
%\includegraphics[width=10cm]{../introduction/images/blog}
%\end{frame}

%% \begin{frame}
%% \frametitle{Exercises}
%% \begin{itemize}
%% \item Install Constraint Solver of your Choice
%% \item MiniZinc (MiningZinc) will be used in later sessions
%% \item Run Sendmoremoney, Sudoku, Queens examples
%% \item Can you choose which propagator to use for alldifferent?
%% \item Can you use different search strategies?
%% \end{itemize}
%% \end{frame}

\begin{frame}
\frametitle{More Learning Resources}
\begin{itemize}
\item Survey of Methods, Resources, and Formats for Teaching Constraint Programming
\begin{itemize}
\item by Tejas Santanam, Helmut Simonis
\item \url{https://doi.org/10.48550/arXiv.2403.12717}
\item Based on survey of community for WTCP 2023
\item \url{https://hsimonis.github.io/WTCP2023/}
\end{itemize}
\end{itemize}
\end{frame}


}

\mode<all>{
\part{Global Constraints}
}
\begin{frame}
\partpage
\end{frame}
\mode<all>{
\begin{frame}
\frametitle{Objectives}
\begin{itemize}
\item Overview of Core Constraint Programming
\item Three Main Concepts
\begin{itemize}
\item Constraint Propagation
\item Global Constraints
\item Customizing Search
\end{itemize}
%\item Get Some Experience with MiniZinc
\item Topics will be treated in more detail in later parts of the school
\item Based on Examples, not Formal Description
\end{itemize}
\end{frame}

\begin{frame}
\frametitle{Outline}
\begin{itemize}
\item Why Constraint Programming?
\item Constraint Propagation
\item Global Constraints
\item Customizing Search
%\item Choices in Modelling
%\item Decomposition
%\item More Global Constraints
%\item Methodology
%\item What is missing?
\end{itemize}
\end{frame}

% \begin{frame}
  % \frametitle{Using MiniZinc IDE}
  % \begin{itemize}
  % \item Developed in the Australian NICTA project
  % \item Maintained by Monash University
  % \item Modelling tool with multiple back-end solvers
  % \item Available from \url{https://www.minizinc.org/}
  % \end{itemize}
% \end{frame}


% \begin{frame}
% \frametitle{Examples in ECLiPSe}
% \begin{itemize}
% \item Open sourced constraint programming language
% \item Development goes back to 1985
% \item ECRC, ICL, IC-Parc, PTL, Cisco
% \item \url{https://eclipseclp.org/}
% \item Specialities
% \begin{itemize}
% \item Develop new solvers for specific domains
% \item Integration with MIP
% \end{itemize}
% \item Not included in bundled MiniZinc IDE
% \item Specialized visualization tools used here
  % \begin{itemize}
  % \item CP-Viz, Simonis et al. 2010
    % \end{itemize}
% \end{itemize}
% \end{frame}

\begin{frame}
\frametitle{Tutorial Based on ECLiPSe ELearning Course}
\begin{itemize}
\item Self-study course in constraint programming
\item Supported by Cisco Systems and Silicon Valley Community Foundation
\item Multi-media format, video lectures, slides, handout etc
\item {\small \url{https://eclipseclp.org/ELearning/index.html}}
\end{itemize}
\end{frame}

\begin{frame}
\frametitle{Also Part of CRT-AI Constraint Week}
\begin{itemize}
\item Annual one week course on CP and Optimization in Ireland
\item Part of national training program for PhD students in AI
\item \url{https://www.crt-ai.ie/}
\end{itemize}
\end{frame}



\begin{frame}
\frametitle{Constraint Programming - in a nutshell}
\begin{itemize}
\item Declarative description of problems with
\begin{itemize}
\item {\em Variables} which range over (finite) sets of values
\item {\em Constraints} over subsets of variables which restrict possible value combinations
\item A {\em solution} is a value assignment which satisfies all constraints
\end{itemize}

\item Constraint propagation/reasoning
\begin{itemize}
\item Removing inconsistent values for variables
\item Detect failure if constraint can not be satisfied
\item Interaction of constraints via shared variables
\item Incomplete
\end{itemize}

\item Search
\begin{itemize}
\item User controlled assignment of values to variables
\item Each step triggers constraint propagation 
\end{itemize}
\item Different domains require/allow different methods
\end{itemize}
\end{frame}

\begin{frame}
  \frametitle{Constraint Programming is Different}
  \begin{itemize}
  \item Declarative Programming
    \begin{itemize}
    \item Concentrate on what you want
      \item Not how to get there
      \item Program != Algorithm
      \item Program = Model
    \end{itemize}
    \item Applied to Combinatorial Problems
      \begin{itemize}
        \item No complete polynomial algorithms known (exist?)
        \item CP less ad-hoc than heuristics
        \item Models can evolve
  \end{itemize}
  \end{itemize}
  \end{frame}
    
\begin{frame}
  \frametitle{A Subtractive Process}
  \begin{textblock}{4}(8,-3)
    \includegraphics[width=4cm]{../introduction/images/stages}
  \end{textblock}
  \vfill
  \begin{quote}
    ``Oh, bosh, as Mr. Ruskin says. Sculpture, per se, is the simplest thing in the world. All you have to do is to take a big chunk of marble and a hammer and chisel, make up your mind what you are about to create and chip off all the marble you don't want.''-Paris Gaulois.
  \end{quote}
  
  {\tiny Source: \url{https://quoteinvestigator.com/2014/06/22/chip-away/}}
\end{frame}

\begin{frame}
\frametitle{Basic Process}
\begin{center}
\begin{tikzpicture}
\node[shape=rectangle,fill=pantone127-4] (problem) at (2,5) {Problem};
\node (human) at (2,4) {Human};
\node[shape=rectangle,fill=pantone127-4] (model) at (2,3) {Model};
\node (solver) at (2,2) {Constraint Solver/Search};
\node[shape=rectangle,fill=pantone157-8] (solution) at (2,1) {Solution};
\draw[-] (problem) -- (human);
\draw[->] (human) -- (model);
\draw[-] (model) -- (solver);
\draw[->] (solver) -- (solution);
\end{tikzpicture}
\end{center}
\end{frame}

\begin{frame}
\frametitle{More Realistic}
\begin{center}
\begin{tikzpicture}[xscale=4.0]
\node[shape=rectangle,fill=pantone127-4] (problem) at (2,5) {Problem};
\node (human) at (2,4) {Human};
\node[shape=rectangle,fill=pantone127-4] (model) at (2,3) {Model};
\node (solver) at (2,2) {Constraint Solver/Search};
\node[shape=rectangle,fill=pantone24-8] (hangs) at (1,1) {Hangs};
\node[shape=rectangle,fill=pantone157-8] (solution) at (2,1) {Solution};
\node[shape=rectangle,fill=pantone24-8] (wrong) at (3,1) {Wrong Solution};
\draw[-] (problem) -- (human);
\draw[->] (human) -- (model);
\draw[-] (model) -- (solver);
\draw[->] (solver) -- (hangs);
\draw[->] (solver) -- (solution);
\draw[->] (solver) -- (wrong);
\draw[->] (hangs) -- (1,4) -- (human);
\draw[->] (wrong) -- (3,4) -- (human);
\end{tikzpicture}
\end{center}
\end{frame}

\begin{frame}
\frametitle{Dual Role of Model}
\begin{itemize}
\item Allows Human to Express Problem
\begin{itemize}
\item Close to Problem Domain
\item Constraints as Abstractions
\end{itemize}
\item Allows Solver to Execute
\begin{itemize}
\item Variables as Communication Mechanism
\item Constraints as Algorithms
\end{itemize}
\end{itemize}
\end{frame}

\begin{frame}
\frametitle{Modelling Frameworks}
\begin{itemize}
\item \href{https://www.minizinc.org/}{MiniZinc} (NICTA, Monash University, Australia)
\item \href{https://github.com/eomahony/Numberjack}{NumberJack} (Insight, Ireland) 
\item \href{https://www-users.york.ac.uk/peter.nightingale/savilerow//}{EssencePrime/SavilleRow} (UK) 
\item \href{https://cpmpy.readthedocs.io/en/latest/}{CPMpy} (KU Leuven)
\item Allow use of multiple back-end solvers
\item Compile model into variants for each solver
\item A priori solver independent model(CP, MIP, SAT)
\end{itemize}
\end{frame} 

\begin{frame}
\frametitle{Framework Process}
\begin{center}
\begin{tikzpicture}[xscale=2]
\node[shape=rectangle,fill=pantone127-4] (problem) at (2.5,5) {Problem};
\node (human) at (2.5,4) {Human};
\node[shape=rectangle,fill=pantone127-4] (model) at (2.5,3) {Model};
\node (compile) at (2.5,2) {Compile/Reformulate};
\node (solver1) at (1,1) {CP};
\node (solver2) at (2,1) {MIP};
\node (solver3) at (3,1) {SAT};
\node (solver4) at (4,1) {Other};
\node[shape=rectangle,fill=pantone157-8] (solution1) at (1,0) {Solution};
\node[shape=rectangle,fill=pantone157-8] (solution2) at (2,0) {Solution};
\node[shape=rectangle,fill=pantone157-8] (solution3) at (3,0) {Solution};
\node[shape=rectangle,fill=pantone157-8] (solution4) at (4,0) {Solution};
\draw[-] (problem) -- (human);
\draw[->] (human) -- (model);
\draw[-] (model) -- (compile);
\draw[-] (compile) -- (solver1);
\draw[-] (compile) -- (solver2);
\draw[-] (compile) -- (solver3);
\draw[-] (compile) -- (solver4);
\draw[->] (solver1) -- (solution1);
\draw[->] (solver2) -- (solution2);
\draw[->] (solver3) -- (solution3);
\draw[->] (solver4) -- (solution4);
\end{tikzpicture}
\end{center}
\end{frame}

% \begin{frame}
  % \frametitle{Do It Now!}
  % \begin{itemize}
  % \item Download and install Minizinc
  % \item \url{https://www.minizinc.org/}
  % \end{itemize}
% \end{frame}


\begin{frame}
\frametitle{Why use Puzzles as Examples?}
\begin{itemize}
\item Easy to understand the problem
\item Solvable by hand without specialized knowledge
\item Possible to compare automated to manual solving process
\end{itemize}

\begin{quote}
The puzzle, though inanimate, is presented as a solvable problem without lasting negative consequences, a very low-risk low-reward situation. By being a puzzle, the object is attempting to convince the user that it must be completed.
\end{quote}
{\tiny Source: \href{https://www.everydayrhetoric.com/2020/03/the-rhetoric-of-puzzles/}{Every Day Rhetoric}}
\end{frame}


\mode<all>{
\part{Basic Constraint Propagation}
}
\begin{frame}
\partpage
\end{frame}
\mode<all>{
\input{../sendmore/body}
}

\mode<all>{
\part{Global Constraints}
}
\begin{frame}
\partpage
\end{frame}
\mode<all>{
\input{../sudoku/body}
}

\clearpage
\mode<all>{
\part{Customizing Search}
}
\begin{frame}
\partpage
\end{frame}
\mode<all>{
\input{../nqueen/body}
}


%% \clearpage
%% \mode<all>{
%%   \part{Problem Decomposition}
%% }
%% \begin{frame}
%%   \partpage
%% \end{frame}
%% \mode<all>{
%%   \input{../party/body}
%%   }

%% \clearpage
%% \mode<all>{
%%   \part{More Global Constraints}
%% }
%% \begin{frame}
%%   \partpage
%% \end{frame}
%% \mode<all>{
%%   \input{../car/body}
%%   }

%% \clearpage
%% \mode<all>{
%%   \part{Methodology}
%% }
%% \begin{frame}
%%   \partpage
%% \end{frame}
%% \mode<all>{
%%   \input{../methodology/body}
%%   }

\clearpage
\mode<all>{
\part{What is missing?}
}
\begin{frame}
\partpage
\end{frame}

\begin{frame}
\frametitle{Many Specialized Topics}
\begin{itemize}
\item How to design efficient core engine
\item Hybrids with LP/MIP tools
\item Hybrids with SAT
\item Symmetry breaking
\item Use of MDD/BDD to encode sets of solutions
\item High level modelling tools
\item Debugging/visualization
\end{itemize}
\end{frame}

\begin{frame}
\frametitle{Reformulation}
\begin{itemize}
\item Just because the user has modelled it this way, it doesn't mean we have to solve it that way
\begin{itemize}
\item Replace some constraint(s) by other, equivalent constraints
\item Because we don't have that constraint in our system
\item For performance
\end{itemize}
\end{itemize}
\end{frame}

\begin{frame}
\frametitle{Learning}
\begin{itemize}
\item While solving the problem we can learn how to strengthen the model/search 
\begin{itemize}
\item Understand which constraints/method contribute to propagation and change schedule
\item Learn no-good constraints by explaining failure
\item Adapt search strategy based on search experience
\end{itemize}
\end{itemize}
\end{frame}

%% \begin{frame}
%% \frametitle{Refined Process}
%% \begin{center}
%% \begin{tikzpicture}
%% \node[shape=rectangle,fill=pantone127-4] (problem) at (3,7) {Problem};
%% \node (human) at (3,6) {Human};
%% \node[shape=rectangle,fill=pantone127-4] (model) at (3,5) {User Model};
%% \node (reformulation) at (3,4) {Reformulation};
%% \node[shape=rectangle,fill=pantone127-4] (implementation) at (3,3) {Implementation Model};
%% \node (solver) at (3,2) {Constraint Solver/Search};
%% \node[shape=rectangle,fill=pantone157-8] (solution) at (3,1) {Solution};
%% \draw[-] (problem) -- (human);
%% \draw[->] (human) -- (model);
%% \draw[-] (model) -- (reformulation);
%% \draw[->] (reformulation) -- (implementation);
%% \draw[-] (implementation) -- (solver);
%% \draw[->] (solver) -- (0,2) -- (0,3) -- (implementation);
%% \draw[->] (solver) -- (solution);
%% \end{tikzpicture}
%% \end{center}
%% \end{frame}

%\begin{frame}
%\frametitle{What is CP actually used for?}
%\url{http://hsimonis.wordpress.com}
%\includegraphics[width=10cm]{../introduction/images/blog}
%\end{frame}

%% \begin{frame}
%% \frametitle{Exercises}
%% \begin{itemize}
%% \item Install Constraint Solver of your Choice
%% \item MiniZinc (MiningZinc) will be used in later sessions
%% \item Run Sendmoremoney, Sudoku, Queens examples
%% \item Can you choose which propagator to use for alldifferent?
%% \item Can you use different search strategies?
%% \end{itemize}
%% \end{frame}

\begin{frame}
\frametitle{More Learning Resources}
\begin{itemize}
\item Survey of Methods, Resources, and Formats for Teaching Constraint Programming
\begin{itemize}
\item by Tejas Santanam, Helmut Simonis
\item \url{https://doi.org/10.48550/arXiv.2403.12717}
\item Based on survey of community for WTCP 2023
\item \url{https://hsimonis.github.io/WTCP2023/}
\end{itemize}
\end{itemize}
\end{frame}


}

\clearpage
\mode<all>{
\part{Customizing Search}
}
\begin{frame}
\partpage
\end{frame}
\mode<all>{
\begin{frame}
\frametitle{Objectives}
\begin{itemize}
\item Overview of Core Constraint Programming
\item Three Main Concepts
\begin{itemize}
\item Constraint Propagation
\item Global Constraints
\item Customizing Search
\end{itemize}
%\item Get Some Experience with MiniZinc
\item Topics will be treated in more detail in later parts of the school
\item Based on Examples, not Formal Description
\end{itemize}
\end{frame}

\begin{frame}
\frametitle{Outline}
\begin{itemize}
\item Why Constraint Programming?
\item Constraint Propagation
\item Global Constraints
\item Customizing Search
%\item Choices in Modelling
%\item Decomposition
%\item More Global Constraints
%\item Methodology
%\item What is missing?
\end{itemize}
\end{frame}

% \begin{frame}
  % \frametitle{Using MiniZinc IDE}
  % \begin{itemize}
  % \item Developed in the Australian NICTA project
  % \item Maintained by Monash University
  % \item Modelling tool with multiple back-end solvers
  % \item Available from \url{https://www.minizinc.org/}
  % \end{itemize}
% \end{frame}


% \begin{frame}
% \frametitle{Examples in ECLiPSe}
% \begin{itemize}
% \item Open sourced constraint programming language
% \item Development goes back to 1985
% \item ECRC, ICL, IC-Parc, PTL, Cisco
% \item \url{https://eclipseclp.org/}
% \item Specialities
% \begin{itemize}
% \item Develop new solvers for specific domains
% \item Integration with MIP
% \end{itemize}
% \item Not included in bundled MiniZinc IDE
% \item Specialized visualization tools used here
  % \begin{itemize}
  % \item CP-Viz, Simonis et al. 2010
    % \end{itemize}
% \end{itemize}
% \end{frame}

\begin{frame}
\frametitle{Tutorial Based on ECLiPSe ELearning Course}
\begin{itemize}
\item Self-study course in constraint programming
\item Supported by Cisco Systems and Silicon Valley Community Foundation
\item Multi-media format, video lectures, slides, handout etc
\item {\small \url{https://eclipseclp.org/ELearning/index.html}}
\end{itemize}
\end{frame}

\begin{frame}
\frametitle{Also Part of CRT-AI Constraint Week}
\begin{itemize}
\item Annual one week course on CP and Optimization in Ireland
\item Part of national training program for PhD students in AI
\item \url{https://www.crt-ai.ie/}
\end{itemize}
\end{frame}



\begin{frame}
\frametitle{Constraint Programming - in a nutshell}
\begin{itemize}
\item Declarative description of problems with
\begin{itemize}
\item {\em Variables} which range over (finite) sets of values
\item {\em Constraints} over subsets of variables which restrict possible value combinations
\item A {\em solution} is a value assignment which satisfies all constraints
\end{itemize}

\item Constraint propagation/reasoning
\begin{itemize}
\item Removing inconsistent values for variables
\item Detect failure if constraint can not be satisfied
\item Interaction of constraints via shared variables
\item Incomplete
\end{itemize}

\item Search
\begin{itemize}
\item User controlled assignment of values to variables
\item Each step triggers constraint propagation 
\end{itemize}
\item Different domains require/allow different methods
\end{itemize}
\end{frame}

\begin{frame}
  \frametitle{Constraint Programming is Different}
  \begin{itemize}
  \item Declarative Programming
    \begin{itemize}
    \item Concentrate on what you want
      \item Not how to get there
      \item Program != Algorithm
      \item Program = Model
    \end{itemize}
    \item Applied to Combinatorial Problems
      \begin{itemize}
        \item No complete polynomial algorithms known (exist?)
        \item CP less ad-hoc than heuristics
        \item Models can evolve
  \end{itemize}
  \end{itemize}
  \end{frame}
    
\begin{frame}
  \frametitle{A Subtractive Process}
  \begin{textblock}{4}(8,-3)
    \includegraphics[width=4cm]{../introduction/images/stages}
  \end{textblock}
  \vfill
  \begin{quote}
    ``Oh, bosh, as Mr. Ruskin says. Sculpture, per se, is the simplest thing in the world. All you have to do is to take a big chunk of marble and a hammer and chisel, make up your mind what you are about to create and chip off all the marble you don't want.''-Paris Gaulois.
  \end{quote}
  
  {\tiny Source: \url{https://quoteinvestigator.com/2014/06/22/chip-away/}}
\end{frame}

\begin{frame}
\frametitle{Basic Process}
\begin{center}
\begin{tikzpicture}
\node[shape=rectangle,fill=pantone127-4] (problem) at (2,5) {Problem};
\node (human) at (2,4) {Human};
\node[shape=rectangle,fill=pantone127-4] (model) at (2,3) {Model};
\node (solver) at (2,2) {Constraint Solver/Search};
\node[shape=rectangle,fill=pantone157-8] (solution) at (2,1) {Solution};
\draw[-] (problem) -- (human);
\draw[->] (human) -- (model);
\draw[-] (model) -- (solver);
\draw[->] (solver) -- (solution);
\end{tikzpicture}
\end{center}
\end{frame}

\begin{frame}
\frametitle{More Realistic}
\begin{center}
\begin{tikzpicture}[xscale=4.0]
\node[shape=rectangle,fill=pantone127-4] (problem) at (2,5) {Problem};
\node (human) at (2,4) {Human};
\node[shape=rectangle,fill=pantone127-4] (model) at (2,3) {Model};
\node (solver) at (2,2) {Constraint Solver/Search};
\node[shape=rectangle,fill=pantone24-8] (hangs) at (1,1) {Hangs};
\node[shape=rectangle,fill=pantone157-8] (solution) at (2,1) {Solution};
\node[shape=rectangle,fill=pantone24-8] (wrong) at (3,1) {Wrong Solution};
\draw[-] (problem) -- (human);
\draw[->] (human) -- (model);
\draw[-] (model) -- (solver);
\draw[->] (solver) -- (hangs);
\draw[->] (solver) -- (solution);
\draw[->] (solver) -- (wrong);
\draw[->] (hangs) -- (1,4) -- (human);
\draw[->] (wrong) -- (3,4) -- (human);
\end{tikzpicture}
\end{center}
\end{frame}

\begin{frame}
\frametitle{Dual Role of Model}
\begin{itemize}
\item Allows Human to Express Problem
\begin{itemize}
\item Close to Problem Domain
\item Constraints as Abstractions
\end{itemize}
\item Allows Solver to Execute
\begin{itemize}
\item Variables as Communication Mechanism
\item Constraints as Algorithms
\end{itemize}
\end{itemize}
\end{frame}

\begin{frame}
\frametitle{Modelling Frameworks}
\begin{itemize}
\item \href{https://www.minizinc.org/}{MiniZinc} (NICTA, Monash University, Australia)
\item \href{https://github.com/eomahony/Numberjack}{NumberJack} (Insight, Ireland) 
\item \href{https://www-users.york.ac.uk/peter.nightingale/savilerow//}{EssencePrime/SavilleRow} (UK) 
\item \href{https://cpmpy.readthedocs.io/en/latest/}{CPMpy} (KU Leuven)
\item Allow use of multiple back-end solvers
\item Compile model into variants for each solver
\item A priori solver independent model(CP, MIP, SAT)
\end{itemize}
\end{frame} 

\begin{frame}
\frametitle{Framework Process}
\begin{center}
\begin{tikzpicture}[xscale=2]
\node[shape=rectangle,fill=pantone127-4] (problem) at (2.5,5) {Problem};
\node (human) at (2.5,4) {Human};
\node[shape=rectangle,fill=pantone127-4] (model) at (2.5,3) {Model};
\node (compile) at (2.5,2) {Compile/Reformulate};
\node (solver1) at (1,1) {CP};
\node (solver2) at (2,1) {MIP};
\node (solver3) at (3,1) {SAT};
\node (solver4) at (4,1) {Other};
\node[shape=rectangle,fill=pantone157-8] (solution1) at (1,0) {Solution};
\node[shape=rectangle,fill=pantone157-8] (solution2) at (2,0) {Solution};
\node[shape=rectangle,fill=pantone157-8] (solution3) at (3,0) {Solution};
\node[shape=rectangle,fill=pantone157-8] (solution4) at (4,0) {Solution};
\draw[-] (problem) -- (human);
\draw[->] (human) -- (model);
\draw[-] (model) -- (compile);
\draw[-] (compile) -- (solver1);
\draw[-] (compile) -- (solver2);
\draw[-] (compile) -- (solver3);
\draw[-] (compile) -- (solver4);
\draw[->] (solver1) -- (solution1);
\draw[->] (solver2) -- (solution2);
\draw[->] (solver3) -- (solution3);
\draw[->] (solver4) -- (solution4);
\end{tikzpicture}
\end{center}
\end{frame}

% \begin{frame}
  % \frametitle{Do It Now!}
  % \begin{itemize}
  % \item Download and install Minizinc
  % \item \url{https://www.minizinc.org/}
  % \end{itemize}
% \end{frame}


\begin{frame}
\frametitle{Why use Puzzles as Examples?}
\begin{itemize}
\item Easy to understand the problem
\item Solvable by hand without specialized knowledge
\item Possible to compare automated to manual solving process
\end{itemize}

\begin{quote}
The puzzle, though inanimate, is presented as a solvable problem without lasting negative consequences, a very low-risk low-reward situation. By being a puzzle, the object is attempting to convince the user that it must be completed.
\end{quote}
{\tiny Source: \href{https://www.everydayrhetoric.com/2020/03/the-rhetoric-of-puzzles/}{Every Day Rhetoric}}
\end{frame}


\mode<all>{
\part{Basic Constraint Propagation}
}
\begin{frame}
\partpage
\end{frame}
\mode<all>{
\input{../sendmore/body}
}

\mode<all>{
\part{Global Constraints}
}
\begin{frame}
\partpage
\end{frame}
\mode<all>{
\input{../sudoku/body}
}

\clearpage
\mode<all>{
\part{Customizing Search}
}
\begin{frame}
\partpage
\end{frame}
\mode<all>{
\input{../nqueen/body}
}


%% \clearpage
%% \mode<all>{
%%   \part{Problem Decomposition}
%% }
%% \begin{frame}
%%   \partpage
%% \end{frame}
%% \mode<all>{
%%   \input{../party/body}
%%   }

%% \clearpage
%% \mode<all>{
%%   \part{More Global Constraints}
%% }
%% \begin{frame}
%%   \partpage
%% \end{frame}
%% \mode<all>{
%%   \input{../car/body}
%%   }

%% \clearpage
%% \mode<all>{
%%   \part{Methodology}
%% }
%% \begin{frame}
%%   \partpage
%% \end{frame}
%% \mode<all>{
%%   \input{../methodology/body}
%%   }

\clearpage
\mode<all>{
\part{What is missing?}
}
\begin{frame}
\partpage
\end{frame}

\begin{frame}
\frametitle{Many Specialized Topics}
\begin{itemize}
\item How to design efficient core engine
\item Hybrids with LP/MIP tools
\item Hybrids with SAT
\item Symmetry breaking
\item Use of MDD/BDD to encode sets of solutions
\item High level modelling tools
\item Debugging/visualization
\end{itemize}
\end{frame}

\begin{frame}
\frametitle{Reformulation}
\begin{itemize}
\item Just because the user has modelled it this way, it doesn't mean we have to solve it that way
\begin{itemize}
\item Replace some constraint(s) by other, equivalent constraints
\item Because we don't have that constraint in our system
\item For performance
\end{itemize}
\end{itemize}
\end{frame}

\begin{frame}
\frametitle{Learning}
\begin{itemize}
\item While solving the problem we can learn how to strengthen the model/search 
\begin{itemize}
\item Understand which constraints/method contribute to propagation and change schedule
\item Learn no-good constraints by explaining failure
\item Adapt search strategy based on search experience
\end{itemize}
\end{itemize}
\end{frame}

%% \begin{frame}
%% \frametitle{Refined Process}
%% \begin{center}
%% \begin{tikzpicture}
%% \node[shape=rectangle,fill=pantone127-4] (problem) at (3,7) {Problem};
%% \node (human) at (3,6) {Human};
%% \node[shape=rectangle,fill=pantone127-4] (model) at (3,5) {User Model};
%% \node (reformulation) at (3,4) {Reformulation};
%% \node[shape=rectangle,fill=pantone127-4] (implementation) at (3,3) {Implementation Model};
%% \node (solver) at (3,2) {Constraint Solver/Search};
%% \node[shape=rectangle,fill=pantone157-8] (solution) at (3,1) {Solution};
%% \draw[-] (problem) -- (human);
%% \draw[->] (human) -- (model);
%% \draw[-] (model) -- (reformulation);
%% \draw[->] (reformulation) -- (implementation);
%% \draw[-] (implementation) -- (solver);
%% \draw[->] (solver) -- (0,2) -- (0,3) -- (implementation);
%% \draw[->] (solver) -- (solution);
%% \end{tikzpicture}
%% \end{center}
%% \end{frame}

%\begin{frame}
%\frametitle{What is CP actually used for?}
%\url{http://hsimonis.wordpress.com}
%\includegraphics[width=10cm]{../introduction/images/blog}
%\end{frame}

%% \begin{frame}
%% \frametitle{Exercises}
%% \begin{itemize}
%% \item Install Constraint Solver of your Choice
%% \item MiniZinc (MiningZinc) will be used in later sessions
%% \item Run Sendmoremoney, Sudoku, Queens examples
%% \item Can you choose which propagator to use for alldifferent?
%% \item Can you use different search strategies?
%% \end{itemize}
%% \end{frame}

\begin{frame}
\frametitle{More Learning Resources}
\begin{itemize}
\item Survey of Methods, Resources, and Formats for Teaching Constraint Programming
\begin{itemize}
\item by Tejas Santanam, Helmut Simonis
\item \url{https://doi.org/10.48550/arXiv.2403.12717}
\item Based on survey of community for WTCP 2023
\item \url{https://hsimonis.github.io/WTCP2023/}
\end{itemize}
\end{itemize}
\end{frame}


}


%% \clearpage
%% \mode<all>{
%%   \part{Problem Decomposition}
%% }
%% \begin{frame}
%%   \partpage
%% \end{frame}
%% \mode<all>{
%%   \begin{frame}
\frametitle{Objectives}
\begin{itemize}
\item Overview of Core Constraint Programming
\item Three Main Concepts
\begin{itemize}
\item Constraint Propagation
\item Global Constraints
\item Customizing Search
\end{itemize}
%\item Get Some Experience with MiniZinc
\item Topics will be treated in more detail in later parts of the school
\item Based on Examples, not Formal Description
\end{itemize}
\end{frame}

\begin{frame}
\frametitle{Outline}
\begin{itemize}
\item Why Constraint Programming?
\item Constraint Propagation
\item Global Constraints
\item Customizing Search
%\item Choices in Modelling
%\item Decomposition
%\item More Global Constraints
%\item Methodology
%\item What is missing?
\end{itemize}
\end{frame}

% \begin{frame}
  % \frametitle{Using MiniZinc IDE}
  % \begin{itemize}
  % \item Developed in the Australian NICTA project
  % \item Maintained by Monash University
  % \item Modelling tool with multiple back-end solvers
  % \item Available from \url{https://www.minizinc.org/}
  % \end{itemize}
% \end{frame}


% \begin{frame}
% \frametitle{Examples in ECLiPSe}
% \begin{itemize}
% \item Open sourced constraint programming language
% \item Development goes back to 1985
% \item ECRC, ICL, IC-Parc, PTL, Cisco
% \item \url{https://eclipseclp.org/}
% \item Specialities
% \begin{itemize}
% \item Develop new solvers for specific domains
% \item Integration with MIP
% \end{itemize}
% \item Not included in bundled MiniZinc IDE
% \item Specialized visualization tools used here
  % \begin{itemize}
  % \item CP-Viz, Simonis et al. 2010
    % \end{itemize}
% \end{itemize}
% \end{frame}

\begin{frame}
\frametitle{Tutorial Based on ECLiPSe ELearning Course}
\begin{itemize}
\item Self-study course in constraint programming
\item Supported by Cisco Systems and Silicon Valley Community Foundation
\item Multi-media format, video lectures, slides, handout etc
\item {\small \url{https://eclipseclp.org/ELearning/index.html}}
\end{itemize}
\end{frame}

\begin{frame}
\frametitle{Also Part of CRT-AI Constraint Week}
\begin{itemize}
\item Annual one week course on CP and Optimization in Ireland
\item Part of national training program for PhD students in AI
\item \url{https://www.crt-ai.ie/}
\end{itemize}
\end{frame}



\begin{frame}
\frametitle{Constraint Programming - in a nutshell}
\begin{itemize}
\item Declarative description of problems with
\begin{itemize}
\item {\em Variables} which range over (finite) sets of values
\item {\em Constraints} over subsets of variables which restrict possible value combinations
\item A {\em solution} is a value assignment which satisfies all constraints
\end{itemize}

\item Constraint propagation/reasoning
\begin{itemize}
\item Removing inconsistent values for variables
\item Detect failure if constraint can not be satisfied
\item Interaction of constraints via shared variables
\item Incomplete
\end{itemize}

\item Search
\begin{itemize}
\item User controlled assignment of values to variables
\item Each step triggers constraint propagation 
\end{itemize}
\item Different domains require/allow different methods
\end{itemize}
\end{frame}

\begin{frame}
  \frametitle{Constraint Programming is Different}
  \begin{itemize}
  \item Declarative Programming
    \begin{itemize}
    \item Concentrate on what you want
      \item Not how to get there
      \item Program != Algorithm
      \item Program = Model
    \end{itemize}
    \item Applied to Combinatorial Problems
      \begin{itemize}
        \item No complete polynomial algorithms known (exist?)
        \item CP less ad-hoc than heuristics
        \item Models can evolve
  \end{itemize}
  \end{itemize}
  \end{frame}
    
\begin{frame}
  \frametitle{A Subtractive Process}
  \begin{textblock}{4}(8,-3)
    \includegraphics[width=4cm]{../introduction/images/stages}
  \end{textblock}
  \vfill
  \begin{quote}
    ``Oh, bosh, as Mr. Ruskin says. Sculpture, per se, is the simplest thing in the world. All you have to do is to take a big chunk of marble and a hammer and chisel, make up your mind what you are about to create and chip off all the marble you don't want.''-Paris Gaulois.
  \end{quote}
  
  {\tiny Source: \url{https://quoteinvestigator.com/2014/06/22/chip-away/}}
\end{frame}

\begin{frame}
\frametitle{Basic Process}
\begin{center}
\begin{tikzpicture}
\node[shape=rectangle,fill=pantone127-4] (problem) at (2,5) {Problem};
\node (human) at (2,4) {Human};
\node[shape=rectangle,fill=pantone127-4] (model) at (2,3) {Model};
\node (solver) at (2,2) {Constraint Solver/Search};
\node[shape=rectangle,fill=pantone157-8] (solution) at (2,1) {Solution};
\draw[-] (problem) -- (human);
\draw[->] (human) -- (model);
\draw[-] (model) -- (solver);
\draw[->] (solver) -- (solution);
\end{tikzpicture}
\end{center}
\end{frame}

\begin{frame}
\frametitle{More Realistic}
\begin{center}
\begin{tikzpicture}[xscale=4.0]
\node[shape=rectangle,fill=pantone127-4] (problem) at (2,5) {Problem};
\node (human) at (2,4) {Human};
\node[shape=rectangle,fill=pantone127-4] (model) at (2,3) {Model};
\node (solver) at (2,2) {Constraint Solver/Search};
\node[shape=rectangle,fill=pantone24-8] (hangs) at (1,1) {Hangs};
\node[shape=rectangle,fill=pantone157-8] (solution) at (2,1) {Solution};
\node[shape=rectangle,fill=pantone24-8] (wrong) at (3,1) {Wrong Solution};
\draw[-] (problem) -- (human);
\draw[->] (human) -- (model);
\draw[-] (model) -- (solver);
\draw[->] (solver) -- (hangs);
\draw[->] (solver) -- (solution);
\draw[->] (solver) -- (wrong);
\draw[->] (hangs) -- (1,4) -- (human);
\draw[->] (wrong) -- (3,4) -- (human);
\end{tikzpicture}
\end{center}
\end{frame}

\begin{frame}
\frametitle{Dual Role of Model}
\begin{itemize}
\item Allows Human to Express Problem
\begin{itemize}
\item Close to Problem Domain
\item Constraints as Abstractions
\end{itemize}
\item Allows Solver to Execute
\begin{itemize}
\item Variables as Communication Mechanism
\item Constraints as Algorithms
\end{itemize}
\end{itemize}
\end{frame}

\begin{frame}
\frametitle{Modelling Frameworks}
\begin{itemize}
\item \href{https://www.minizinc.org/}{MiniZinc} (NICTA, Monash University, Australia)
\item \href{https://github.com/eomahony/Numberjack}{NumberJack} (Insight, Ireland) 
\item \href{https://www-users.york.ac.uk/peter.nightingale/savilerow//}{EssencePrime/SavilleRow} (UK) 
\item \href{https://cpmpy.readthedocs.io/en/latest/}{CPMpy} (KU Leuven)
\item Allow use of multiple back-end solvers
\item Compile model into variants for each solver
\item A priori solver independent model(CP, MIP, SAT)
\end{itemize}
\end{frame} 

\begin{frame}
\frametitle{Framework Process}
\begin{center}
\begin{tikzpicture}[xscale=2]
\node[shape=rectangle,fill=pantone127-4] (problem) at (2.5,5) {Problem};
\node (human) at (2.5,4) {Human};
\node[shape=rectangle,fill=pantone127-4] (model) at (2.5,3) {Model};
\node (compile) at (2.5,2) {Compile/Reformulate};
\node (solver1) at (1,1) {CP};
\node (solver2) at (2,1) {MIP};
\node (solver3) at (3,1) {SAT};
\node (solver4) at (4,1) {Other};
\node[shape=rectangle,fill=pantone157-8] (solution1) at (1,0) {Solution};
\node[shape=rectangle,fill=pantone157-8] (solution2) at (2,0) {Solution};
\node[shape=rectangle,fill=pantone157-8] (solution3) at (3,0) {Solution};
\node[shape=rectangle,fill=pantone157-8] (solution4) at (4,0) {Solution};
\draw[-] (problem) -- (human);
\draw[->] (human) -- (model);
\draw[-] (model) -- (compile);
\draw[-] (compile) -- (solver1);
\draw[-] (compile) -- (solver2);
\draw[-] (compile) -- (solver3);
\draw[-] (compile) -- (solver4);
\draw[->] (solver1) -- (solution1);
\draw[->] (solver2) -- (solution2);
\draw[->] (solver3) -- (solution3);
\draw[->] (solver4) -- (solution4);
\end{tikzpicture}
\end{center}
\end{frame}

% \begin{frame}
  % \frametitle{Do It Now!}
  % \begin{itemize}
  % \item Download and install Minizinc
  % \item \url{https://www.minizinc.org/}
  % \end{itemize}
% \end{frame}


\begin{frame}
\frametitle{Why use Puzzles as Examples?}
\begin{itemize}
\item Easy to understand the problem
\item Solvable by hand without specialized knowledge
\item Possible to compare automated to manual solving process
\end{itemize}

\begin{quote}
The puzzle, though inanimate, is presented as a solvable problem without lasting negative consequences, a very low-risk low-reward situation. By being a puzzle, the object is attempting to convince the user that it must be completed.
\end{quote}
{\tiny Source: \href{https://www.everydayrhetoric.com/2020/03/the-rhetoric-of-puzzles/}{Every Day Rhetoric}}
\end{frame}


\mode<all>{
\part{Basic Constraint Propagation}
}
\begin{frame}
\partpage
\end{frame}
\mode<all>{
\input{../sendmore/body}
}

\mode<all>{
\part{Global Constraints}
}
\begin{frame}
\partpage
\end{frame}
\mode<all>{
\input{../sudoku/body}
}

\clearpage
\mode<all>{
\part{Customizing Search}
}
\begin{frame}
\partpage
\end{frame}
\mode<all>{
\input{../nqueen/body}
}


%% \clearpage
%% \mode<all>{
%%   \part{Problem Decomposition}
%% }
%% \begin{frame}
%%   \partpage
%% \end{frame}
%% \mode<all>{
%%   \input{../party/body}
%%   }

%% \clearpage
%% \mode<all>{
%%   \part{More Global Constraints}
%% }
%% \begin{frame}
%%   \partpage
%% \end{frame}
%% \mode<all>{
%%   \input{../car/body}
%%   }

%% \clearpage
%% \mode<all>{
%%   \part{Methodology}
%% }
%% \begin{frame}
%%   \partpage
%% \end{frame}
%% \mode<all>{
%%   \input{../methodology/body}
%%   }

\clearpage
\mode<all>{
\part{What is missing?}
}
\begin{frame}
\partpage
\end{frame}

\begin{frame}
\frametitle{Many Specialized Topics}
\begin{itemize}
\item How to design efficient core engine
\item Hybrids with LP/MIP tools
\item Hybrids with SAT
\item Symmetry breaking
\item Use of MDD/BDD to encode sets of solutions
\item High level modelling tools
\item Debugging/visualization
\end{itemize}
\end{frame}

\begin{frame}
\frametitle{Reformulation}
\begin{itemize}
\item Just because the user has modelled it this way, it doesn't mean we have to solve it that way
\begin{itemize}
\item Replace some constraint(s) by other, equivalent constraints
\item Because we don't have that constraint in our system
\item For performance
\end{itemize}
\end{itemize}
\end{frame}

\begin{frame}
\frametitle{Learning}
\begin{itemize}
\item While solving the problem we can learn how to strengthen the model/search 
\begin{itemize}
\item Understand which constraints/method contribute to propagation and change schedule
\item Learn no-good constraints by explaining failure
\item Adapt search strategy based on search experience
\end{itemize}
\end{itemize}
\end{frame}

%% \begin{frame}
%% \frametitle{Refined Process}
%% \begin{center}
%% \begin{tikzpicture}
%% \node[shape=rectangle,fill=pantone127-4] (problem) at (3,7) {Problem};
%% \node (human) at (3,6) {Human};
%% \node[shape=rectangle,fill=pantone127-4] (model) at (3,5) {User Model};
%% \node (reformulation) at (3,4) {Reformulation};
%% \node[shape=rectangle,fill=pantone127-4] (implementation) at (3,3) {Implementation Model};
%% \node (solver) at (3,2) {Constraint Solver/Search};
%% \node[shape=rectangle,fill=pantone157-8] (solution) at (3,1) {Solution};
%% \draw[-] (problem) -- (human);
%% \draw[->] (human) -- (model);
%% \draw[-] (model) -- (reformulation);
%% \draw[->] (reformulation) -- (implementation);
%% \draw[-] (implementation) -- (solver);
%% \draw[->] (solver) -- (0,2) -- (0,3) -- (implementation);
%% \draw[->] (solver) -- (solution);
%% \end{tikzpicture}
%% \end{center}
%% \end{frame}

%\begin{frame}
%\frametitle{What is CP actually used for?}
%\url{http://hsimonis.wordpress.com}
%\includegraphics[width=10cm]{../introduction/images/blog}
%\end{frame}

%% \begin{frame}
%% \frametitle{Exercises}
%% \begin{itemize}
%% \item Install Constraint Solver of your Choice
%% \item MiniZinc (MiningZinc) will be used in later sessions
%% \item Run Sendmoremoney, Sudoku, Queens examples
%% \item Can you choose which propagator to use for alldifferent?
%% \item Can you use different search strategies?
%% \end{itemize}
%% \end{frame}

\begin{frame}
\frametitle{More Learning Resources}
\begin{itemize}
\item Survey of Methods, Resources, and Formats for Teaching Constraint Programming
\begin{itemize}
\item by Tejas Santanam, Helmut Simonis
\item \url{https://doi.org/10.48550/arXiv.2403.12717}
\item Based on survey of community for WTCP 2023
\item \url{https://hsimonis.github.io/WTCP2023/}
\end{itemize}
\end{itemize}
\end{frame}


%%   }

%% \clearpage
%% \mode<all>{
%%   \part{More Global Constraints}
%% }
%% \begin{frame}
%%   \partpage
%% \end{frame}
%% \mode<all>{
%%   \begin{frame}
\frametitle{Objectives}
\begin{itemize}
\item Overview of Core Constraint Programming
\item Three Main Concepts
\begin{itemize}
\item Constraint Propagation
\item Global Constraints
\item Customizing Search
\end{itemize}
%\item Get Some Experience with MiniZinc
\item Topics will be treated in more detail in later parts of the school
\item Based on Examples, not Formal Description
\end{itemize}
\end{frame}

\begin{frame}
\frametitle{Outline}
\begin{itemize}
\item Why Constraint Programming?
\item Constraint Propagation
\item Global Constraints
\item Customizing Search
%\item Choices in Modelling
%\item Decomposition
%\item More Global Constraints
%\item Methodology
%\item What is missing?
\end{itemize}
\end{frame}

% \begin{frame}
  % \frametitle{Using MiniZinc IDE}
  % \begin{itemize}
  % \item Developed in the Australian NICTA project
  % \item Maintained by Monash University
  % \item Modelling tool with multiple back-end solvers
  % \item Available from \url{https://www.minizinc.org/}
  % \end{itemize}
% \end{frame}


% \begin{frame}
% \frametitle{Examples in ECLiPSe}
% \begin{itemize}
% \item Open sourced constraint programming language
% \item Development goes back to 1985
% \item ECRC, ICL, IC-Parc, PTL, Cisco
% \item \url{https://eclipseclp.org/}
% \item Specialities
% \begin{itemize}
% \item Develop new solvers for specific domains
% \item Integration with MIP
% \end{itemize}
% \item Not included in bundled MiniZinc IDE
% \item Specialized visualization tools used here
  % \begin{itemize}
  % \item CP-Viz, Simonis et al. 2010
    % \end{itemize}
% \end{itemize}
% \end{frame}

\begin{frame}
\frametitle{Tutorial Based on ECLiPSe ELearning Course}
\begin{itemize}
\item Self-study course in constraint programming
\item Supported by Cisco Systems and Silicon Valley Community Foundation
\item Multi-media format, video lectures, slides, handout etc
\item {\small \url{https://eclipseclp.org/ELearning/index.html}}
\end{itemize}
\end{frame}

\begin{frame}
\frametitle{Also Part of CRT-AI Constraint Week}
\begin{itemize}
\item Annual one week course on CP and Optimization in Ireland
\item Part of national training program for PhD students in AI
\item \url{https://www.crt-ai.ie/}
\end{itemize}
\end{frame}



\begin{frame}
\frametitle{Constraint Programming - in a nutshell}
\begin{itemize}
\item Declarative description of problems with
\begin{itemize}
\item {\em Variables} which range over (finite) sets of values
\item {\em Constraints} over subsets of variables which restrict possible value combinations
\item A {\em solution} is a value assignment which satisfies all constraints
\end{itemize}

\item Constraint propagation/reasoning
\begin{itemize}
\item Removing inconsistent values for variables
\item Detect failure if constraint can not be satisfied
\item Interaction of constraints via shared variables
\item Incomplete
\end{itemize}

\item Search
\begin{itemize}
\item User controlled assignment of values to variables
\item Each step triggers constraint propagation 
\end{itemize}
\item Different domains require/allow different methods
\end{itemize}
\end{frame}

\begin{frame}
  \frametitle{Constraint Programming is Different}
  \begin{itemize}
  \item Declarative Programming
    \begin{itemize}
    \item Concentrate on what you want
      \item Not how to get there
      \item Program != Algorithm
      \item Program = Model
    \end{itemize}
    \item Applied to Combinatorial Problems
      \begin{itemize}
        \item No complete polynomial algorithms known (exist?)
        \item CP less ad-hoc than heuristics
        \item Models can evolve
  \end{itemize}
  \end{itemize}
  \end{frame}
    
\begin{frame}
  \frametitle{A Subtractive Process}
  \begin{textblock}{4}(8,-3)
    \includegraphics[width=4cm]{../introduction/images/stages}
  \end{textblock}
  \vfill
  \begin{quote}
    ``Oh, bosh, as Mr. Ruskin says. Sculpture, per se, is the simplest thing in the world. All you have to do is to take a big chunk of marble and a hammer and chisel, make up your mind what you are about to create and chip off all the marble you don't want.''-Paris Gaulois.
  \end{quote}
  
  {\tiny Source: \url{https://quoteinvestigator.com/2014/06/22/chip-away/}}
\end{frame}

\begin{frame}
\frametitle{Basic Process}
\begin{center}
\begin{tikzpicture}
\node[shape=rectangle,fill=pantone127-4] (problem) at (2,5) {Problem};
\node (human) at (2,4) {Human};
\node[shape=rectangle,fill=pantone127-4] (model) at (2,3) {Model};
\node (solver) at (2,2) {Constraint Solver/Search};
\node[shape=rectangle,fill=pantone157-8] (solution) at (2,1) {Solution};
\draw[-] (problem) -- (human);
\draw[->] (human) -- (model);
\draw[-] (model) -- (solver);
\draw[->] (solver) -- (solution);
\end{tikzpicture}
\end{center}
\end{frame}

\begin{frame}
\frametitle{More Realistic}
\begin{center}
\begin{tikzpicture}[xscale=4.0]
\node[shape=rectangle,fill=pantone127-4] (problem) at (2,5) {Problem};
\node (human) at (2,4) {Human};
\node[shape=rectangle,fill=pantone127-4] (model) at (2,3) {Model};
\node (solver) at (2,2) {Constraint Solver/Search};
\node[shape=rectangle,fill=pantone24-8] (hangs) at (1,1) {Hangs};
\node[shape=rectangle,fill=pantone157-8] (solution) at (2,1) {Solution};
\node[shape=rectangle,fill=pantone24-8] (wrong) at (3,1) {Wrong Solution};
\draw[-] (problem) -- (human);
\draw[->] (human) -- (model);
\draw[-] (model) -- (solver);
\draw[->] (solver) -- (hangs);
\draw[->] (solver) -- (solution);
\draw[->] (solver) -- (wrong);
\draw[->] (hangs) -- (1,4) -- (human);
\draw[->] (wrong) -- (3,4) -- (human);
\end{tikzpicture}
\end{center}
\end{frame}

\begin{frame}
\frametitle{Dual Role of Model}
\begin{itemize}
\item Allows Human to Express Problem
\begin{itemize}
\item Close to Problem Domain
\item Constraints as Abstractions
\end{itemize}
\item Allows Solver to Execute
\begin{itemize}
\item Variables as Communication Mechanism
\item Constraints as Algorithms
\end{itemize}
\end{itemize}
\end{frame}

\begin{frame}
\frametitle{Modelling Frameworks}
\begin{itemize}
\item \href{https://www.minizinc.org/}{MiniZinc} (NICTA, Monash University, Australia)
\item \href{https://github.com/eomahony/Numberjack}{NumberJack} (Insight, Ireland) 
\item \href{https://www-users.york.ac.uk/peter.nightingale/savilerow//}{EssencePrime/SavilleRow} (UK) 
\item \href{https://cpmpy.readthedocs.io/en/latest/}{CPMpy} (KU Leuven)
\item Allow use of multiple back-end solvers
\item Compile model into variants for each solver
\item A priori solver independent model(CP, MIP, SAT)
\end{itemize}
\end{frame} 

\begin{frame}
\frametitle{Framework Process}
\begin{center}
\begin{tikzpicture}[xscale=2]
\node[shape=rectangle,fill=pantone127-4] (problem) at (2.5,5) {Problem};
\node (human) at (2.5,4) {Human};
\node[shape=rectangle,fill=pantone127-4] (model) at (2.5,3) {Model};
\node (compile) at (2.5,2) {Compile/Reformulate};
\node (solver1) at (1,1) {CP};
\node (solver2) at (2,1) {MIP};
\node (solver3) at (3,1) {SAT};
\node (solver4) at (4,1) {Other};
\node[shape=rectangle,fill=pantone157-8] (solution1) at (1,0) {Solution};
\node[shape=rectangle,fill=pantone157-8] (solution2) at (2,0) {Solution};
\node[shape=rectangle,fill=pantone157-8] (solution3) at (3,0) {Solution};
\node[shape=rectangle,fill=pantone157-8] (solution4) at (4,0) {Solution};
\draw[-] (problem) -- (human);
\draw[->] (human) -- (model);
\draw[-] (model) -- (compile);
\draw[-] (compile) -- (solver1);
\draw[-] (compile) -- (solver2);
\draw[-] (compile) -- (solver3);
\draw[-] (compile) -- (solver4);
\draw[->] (solver1) -- (solution1);
\draw[->] (solver2) -- (solution2);
\draw[->] (solver3) -- (solution3);
\draw[->] (solver4) -- (solution4);
\end{tikzpicture}
\end{center}
\end{frame}

% \begin{frame}
  % \frametitle{Do It Now!}
  % \begin{itemize}
  % \item Download and install Minizinc
  % \item \url{https://www.minizinc.org/}
  % \end{itemize}
% \end{frame}


\begin{frame}
\frametitle{Why use Puzzles as Examples?}
\begin{itemize}
\item Easy to understand the problem
\item Solvable by hand without specialized knowledge
\item Possible to compare automated to manual solving process
\end{itemize}

\begin{quote}
The puzzle, though inanimate, is presented as a solvable problem without lasting negative consequences, a very low-risk low-reward situation. By being a puzzle, the object is attempting to convince the user that it must be completed.
\end{quote}
{\tiny Source: \href{https://www.everydayrhetoric.com/2020/03/the-rhetoric-of-puzzles/}{Every Day Rhetoric}}
\end{frame}


\mode<all>{
\part{Basic Constraint Propagation}
}
\begin{frame}
\partpage
\end{frame}
\mode<all>{
\input{../sendmore/body}
}

\mode<all>{
\part{Global Constraints}
}
\begin{frame}
\partpage
\end{frame}
\mode<all>{
\input{../sudoku/body}
}

\clearpage
\mode<all>{
\part{Customizing Search}
}
\begin{frame}
\partpage
\end{frame}
\mode<all>{
\input{../nqueen/body}
}


%% \clearpage
%% \mode<all>{
%%   \part{Problem Decomposition}
%% }
%% \begin{frame}
%%   \partpage
%% \end{frame}
%% \mode<all>{
%%   \input{../party/body}
%%   }

%% \clearpage
%% \mode<all>{
%%   \part{More Global Constraints}
%% }
%% \begin{frame}
%%   \partpage
%% \end{frame}
%% \mode<all>{
%%   \input{../car/body}
%%   }

%% \clearpage
%% \mode<all>{
%%   \part{Methodology}
%% }
%% \begin{frame}
%%   \partpage
%% \end{frame}
%% \mode<all>{
%%   \input{../methodology/body}
%%   }

\clearpage
\mode<all>{
\part{What is missing?}
}
\begin{frame}
\partpage
\end{frame}

\begin{frame}
\frametitle{Many Specialized Topics}
\begin{itemize}
\item How to design efficient core engine
\item Hybrids with LP/MIP tools
\item Hybrids with SAT
\item Symmetry breaking
\item Use of MDD/BDD to encode sets of solutions
\item High level modelling tools
\item Debugging/visualization
\end{itemize}
\end{frame}

\begin{frame}
\frametitle{Reformulation}
\begin{itemize}
\item Just because the user has modelled it this way, it doesn't mean we have to solve it that way
\begin{itemize}
\item Replace some constraint(s) by other, equivalent constraints
\item Because we don't have that constraint in our system
\item For performance
\end{itemize}
\end{itemize}
\end{frame}

\begin{frame}
\frametitle{Learning}
\begin{itemize}
\item While solving the problem we can learn how to strengthen the model/search 
\begin{itemize}
\item Understand which constraints/method contribute to propagation and change schedule
\item Learn no-good constraints by explaining failure
\item Adapt search strategy based on search experience
\end{itemize}
\end{itemize}
\end{frame}

%% \begin{frame}
%% \frametitle{Refined Process}
%% \begin{center}
%% \begin{tikzpicture}
%% \node[shape=rectangle,fill=pantone127-4] (problem) at (3,7) {Problem};
%% \node (human) at (3,6) {Human};
%% \node[shape=rectangle,fill=pantone127-4] (model) at (3,5) {User Model};
%% \node (reformulation) at (3,4) {Reformulation};
%% \node[shape=rectangle,fill=pantone127-4] (implementation) at (3,3) {Implementation Model};
%% \node (solver) at (3,2) {Constraint Solver/Search};
%% \node[shape=rectangle,fill=pantone157-8] (solution) at (3,1) {Solution};
%% \draw[-] (problem) -- (human);
%% \draw[->] (human) -- (model);
%% \draw[-] (model) -- (reformulation);
%% \draw[->] (reformulation) -- (implementation);
%% \draw[-] (implementation) -- (solver);
%% \draw[->] (solver) -- (0,2) -- (0,3) -- (implementation);
%% \draw[->] (solver) -- (solution);
%% \end{tikzpicture}
%% \end{center}
%% \end{frame}

%\begin{frame}
%\frametitle{What is CP actually used for?}
%\url{http://hsimonis.wordpress.com}
%\includegraphics[width=10cm]{../introduction/images/blog}
%\end{frame}

%% \begin{frame}
%% \frametitle{Exercises}
%% \begin{itemize}
%% \item Install Constraint Solver of your Choice
%% \item MiniZinc (MiningZinc) will be used in later sessions
%% \item Run Sendmoremoney, Sudoku, Queens examples
%% \item Can you choose which propagator to use for alldifferent?
%% \item Can you use different search strategies?
%% \end{itemize}
%% \end{frame}

\begin{frame}
\frametitle{More Learning Resources}
\begin{itemize}
\item Survey of Methods, Resources, and Formats for Teaching Constraint Programming
\begin{itemize}
\item by Tejas Santanam, Helmut Simonis
\item \url{https://doi.org/10.48550/arXiv.2403.12717}
\item Based on survey of community for WTCP 2023
\item \url{https://hsimonis.github.io/WTCP2023/}
\end{itemize}
\end{itemize}
\end{frame}


%%   }

%% \clearpage
%% \mode<all>{
%%   \part{Methodology}
%% }
%% \begin{frame}
%%   \partpage
%% \end{frame}
%% \mode<all>{
%%   \begin{frame}
\frametitle{Objectives}
\begin{itemize}
\item Overview of Core Constraint Programming
\item Three Main Concepts
\begin{itemize}
\item Constraint Propagation
\item Global Constraints
\item Customizing Search
\end{itemize}
%\item Get Some Experience with MiniZinc
\item Topics will be treated in more detail in later parts of the school
\item Based on Examples, not Formal Description
\end{itemize}
\end{frame}

\begin{frame}
\frametitle{Outline}
\begin{itemize}
\item Why Constraint Programming?
\item Constraint Propagation
\item Global Constraints
\item Customizing Search
%\item Choices in Modelling
%\item Decomposition
%\item More Global Constraints
%\item Methodology
%\item What is missing?
\end{itemize}
\end{frame}

% \begin{frame}
  % \frametitle{Using MiniZinc IDE}
  % \begin{itemize}
  % \item Developed in the Australian NICTA project
  % \item Maintained by Monash University
  % \item Modelling tool with multiple back-end solvers
  % \item Available from \url{https://www.minizinc.org/}
  % \end{itemize}
% \end{frame}


% \begin{frame}
% \frametitle{Examples in ECLiPSe}
% \begin{itemize}
% \item Open sourced constraint programming language
% \item Development goes back to 1985
% \item ECRC, ICL, IC-Parc, PTL, Cisco
% \item \url{https://eclipseclp.org/}
% \item Specialities
% \begin{itemize}
% \item Develop new solvers for specific domains
% \item Integration with MIP
% \end{itemize}
% \item Not included in bundled MiniZinc IDE
% \item Specialized visualization tools used here
  % \begin{itemize}
  % \item CP-Viz, Simonis et al. 2010
    % \end{itemize}
% \end{itemize}
% \end{frame}

\begin{frame}
\frametitle{Tutorial Based on ECLiPSe ELearning Course}
\begin{itemize}
\item Self-study course in constraint programming
\item Supported by Cisco Systems and Silicon Valley Community Foundation
\item Multi-media format, video lectures, slides, handout etc
\item {\small \url{https://eclipseclp.org/ELearning/index.html}}
\end{itemize}
\end{frame}

\begin{frame}
\frametitle{Also Part of CRT-AI Constraint Week}
\begin{itemize}
\item Annual one week course on CP and Optimization in Ireland
\item Part of national training program for PhD students in AI
\item \url{https://www.crt-ai.ie/}
\end{itemize}
\end{frame}



\begin{frame}
\frametitle{Constraint Programming - in a nutshell}
\begin{itemize}
\item Declarative description of problems with
\begin{itemize}
\item {\em Variables} which range over (finite) sets of values
\item {\em Constraints} over subsets of variables which restrict possible value combinations
\item A {\em solution} is a value assignment which satisfies all constraints
\end{itemize}

\item Constraint propagation/reasoning
\begin{itemize}
\item Removing inconsistent values for variables
\item Detect failure if constraint can not be satisfied
\item Interaction of constraints via shared variables
\item Incomplete
\end{itemize}

\item Search
\begin{itemize}
\item User controlled assignment of values to variables
\item Each step triggers constraint propagation 
\end{itemize}
\item Different domains require/allow different methods
\end{itemize}
\end{frame}

\begin{frame}
  \frametitle{Constraint Programming is Different}
  \begin{itemize}
  \item Declarative Programming
    \begin{itemize}
    \item Concentrate on what you want
      \item Not how to get there
      \item Program != Algorithm
      \item Program = Model
    \end{itemize}
    \item Applied to Combinatorial Problems
      \begin{itemize}
        \item No complete polynomial algorithms known (exist?)
        \item CP less ad-hoc than heuristics
        \item Models can evolve
  \end{itemize}
  \end{itemize}
  \end{frame}
    
\begin{frame}
  \frametitle{A Subtractive Process}
  \begin{textblock}{4}(8,-3)
    \includegraphics[width=4cm]{../introduction/images/stages}
  \end{textblock}
  \vfill
  \begin{quote}
    ``Oh, bosh, as Mr. Ruskin says. Sculpture, per se, is the simplest thing in the world. All you have to do is to take a big chunk of marble and a hammer and chisel, make up your mind what you are about to create and chip off all the marble you don't want.''-Paris Gaulois.
  \end{quote}
  
  {\tiny Source: \url{https://quoteinvestigator.com/2014/06/22/chip-away/}}
\end{frame}

\begin{frame}
\frametitle{Basic Process}
\begin{center}
\begin{tikzpicture}
\node[shape=rectangle,fill=pantone127-4] (problem) at (2,5) {Problem};
\node (human) at (2,4) {Human};
\node[shape=rectangle,fill=pantone127-4] (model) at (2,3) {Model};
\node (solver) at (2,2) {Constraint Solver/Search};
\node[shape=rectangle,fill=pantone157-8] (solution) at (2,1) {Solution};
\draw[-] (problem) -- (human);
\draw[->] (human) -- (model);
\draw[-] (model) -- (solver);
\draw[->] (solver) -- (solution);
\end{tikzpicture}
\end{center}
\end{frame}

\begin{frame}
\frametitle{More Realistic}
\begin{center}
\begin{tikzpicture}[xscale=4.0]
\node[shape=rectangle,fill=pantone127-4] (problem) at (2,5) {Problem};
\node (human) at (2,4) {Human};
\node[shape=rectangle,fill=pantone127-4] (model) at (2,3) {Model};
\node (solver) at (2,2) {Constraint Solver/Search};
\node[shape=rectangle,fill=pantone24-8] (hangs) at (1,1) {Hangs};
\node[shape=rectangle,fill=pantone157-8] (solution) at (2,1) {Solution};
\node[shape=rectangle,fill=pantone24-8] (wrong) at (3,1) {Wrong Solution};
\draw[-] (problem) -- (human);
\draw[->] (human) -- (model);
\draw[-] (model) -- (solver);
\draw[->] (solver) -- (hangs);
\draw[->] (solver) -- (solution);
\draw[->] (solver) -- (wrong);
\draw[->] (hangs) -- (1,4) -- (human);
\draw[->] (wrong) -- (3,4) -- (human);
\end{tikzpicture}
\end{center}
\end{frame}

\begin{frame}
\frametitle{Dual Role of Model}
\begin{itemize}
\item Allows Human to Express Problem
\begin{itemize}
\item Close to Problem Domain
\item Constraints as Abstractions
\end{itemize}
\item Allows Solver to Execute
\begin{itemize}
\item Variables as Communication Mechanism
\item Constraints as Algorithms
\end{itemize}
\end{itemize}
\end{frame}

\begin{frame}
\frametitle{Modelling Frameworks}
\begin{itemize}
\item \href{https://www.minizinc.org/}{MiniZinc} (NICTA, Monash University, Australia)
\item \href{https://github.com/eomahony/Numberjack}{NumberJack} (Insight, Ireland) 
\item \href{https://www-users.york.ac.uk/peter.nightingale/savilerow//}{EssencePrime/SavilleRow} (UK) 
\item \href{https://cpmpy.readthedocs.io/en/latest/}{CPMpy} (KU Leuven)
\item Allow use of multiple back-end solvers
\item Compile model into variants for each solver
\item A priori solver independent model(CP, MIP, SAT)
\end{itemize}
\end{frame} 

\begin{frame}
\frametitle{Framework Process}
\begin{center}
\begin{tikzpicture}[xscale=2]
\node[shape=rectangle,fill=pantone127-4] (problem) at (2.5,5) {Problem};
\node (human) at (2.5,4) {Human};
\node[shape=rectangle,fill=pantone127-4] (model) at (2.5,3) {Model};
\node (compile) at (2.5,2) {Compile/Reformulate};
\node (solver1) at (1,1) {CP};
\node (solver2) at (2,1) {MIP};
\node (solver3) at (3,1) {SAT};
\node (solver4) at (4,1) {Other};
\node[shape=rectangle,fill=pantone157-8] (solution1) at (1,0) {Solution};
\node[shape=rectangle,fill=pantone157-8] (solution2) at (2,0) {Solution};
\node[shape=rectangle,fill=pantone157-8] (solution3) at (3,0) {Solution};
\node[shape=rectangle,fill=pantone157-8] (solution4) at (4,0) {Solution};
\draw[-] (problem) -- (human);
\draw[->] (human) -- (model);
\draw[-] (model) -- (compile);
\draw[-] (compile) -- (solver1);
\draw[-] (compile) -- (solver2);
\draw[-] (compile) -- (solver3);
\draw[-] (compile) -- (solver4);
\draw[->] (solver1) -- (solution1);
\draw[->] (solver2) -- (solution2);
\draw[->] (solver3) -- (solution3);
\draw[->] (solver4) -- (solution4);
\end{tikzpicture}
\end{center}
\end{frame}

% \begin{frame}
  % \frametitle{Do It Now!}
  % \begin{itemize}
  % \item Download and install Minizinc
  % \item \url{https://www.minizinc.org/}
  % \end{itemize}
% \end{frame}


\begin{frame}
\frametitle{Why use Puzzles as Examples?}
\begin{itemize}
\item Easy to understand the problem
\item Solvable by hand without specialized knowledge
\item Possible to compare automated to manual solving process
\end{itemize}

\begin{quote}
The puzzle, though inanimate, is presented as a solvable problem without lasting negative consequences, a very low-risk low-reward situation. By being a puzzle, the object is attempting to convince the user that it must be completed.
\end{quote}
{\tiny Source: \href{https://www.everydayrhetoric.com/2020/03/the-rhetoric-of-puzzles/}{Every Day Rhetoric}}
\end{frame}


\mode<all>{
\part{Basic Constraint Propagation}
}
\begin{frame}
\partpage
\end{frame}
\mode<all>{
\input{../sendmore/body}
}

\mode<all>{
\part{Global Constraints}
}
\begin{frame}
\partpage
\end{frame}
\mode<all>{
\input{../sudoku/body}
}

\clearpage
\mode<all>{
\part{Customizing Search}
}
\begin{frame}
\partpage
\end{frame}
\mode<all>{
\input{../nqueen/body}
}


%% \clearpage
%% \mode<all>{
%%   \part{Problem Decomposition}
%% }
%% \begin{frame}
%%   \partpage
%% \end{frame}
%% \mode<all>{
%%   \input{../party/body}
%%   }

%% \clearpage
%% \mode<all>{
%%   \part{More Global Constraints}
%% }
%% \begin{frame}
%%   \partpage
%% \end{frame}
%% \mode<all>{
%%   \input{../car/body}
%%   }

%% \clearpage
%% \mode<all>{
%%   \part{Methodology}
%% }
%% \begin{frame}
%%   \partpage
%% \end{frame}
%% \mode<all>{
%%   \input{../methodology/body}
%%   }

\clearpage
\mode<all>{
\part{What is missing?}
}
\begin{frame}
\partpage
\end{frame}

\begin{frame}
\frametitle{Many Specialized Topics}
\begin{itemize}
\item How to design efficient core engine
\item Hybrids with LP/MIP tools
\item Hybrids with SAT
\item Symmetry breaking
\item Use of MDD/BDD to encode sets of solutions
\item High level modelling tools
\item Debugging/visualization
\end{itemize}
\end{frame}

\begin{frame}
\frametitle{Reformulation}
\begin{itemize}
\item Just because the user has modelled it this way, it doesn't mean we have to solve it that way
\begin{itemize}
\item Replace some constraint(s) by other, equivalent constraints
\item Because we don't have that constraint in our system
\item For performance
\end{itemize}
\end{itemize}
\end{frame}

\begin{frame}
\frametitle{Learning}
\begin{itemize}
\item While solving the problem we can learn how to strengthen the model/search 
\begin{itemize}
\item Understand which constraints/method contribute to propagation and change schedule
\item Learn no-good constraints by explaining failure
\item Adapt search strategy based on search experience
\end{itemize}
\end{itemize}
\end{frame}

%% \begin{frame}
%% \frametitle{Refined Process}
%% \begin{center}
%% \begin{tikzpicture}
%% \node[shape=rectangle,fill=pantone127-4] (problem) at (3,7) {Problem};
%% \node (human) at (3,6) {Human};
%% \node[shape=rectangle,fill=pantone127-4] (model) at (3,5) {User Model};
%% \node (reformulation) at (3,4) {Reformulation};
%% \node[shape=rectangle,fill=pantone127-4] (implementation) at (3,3) {Implementation Model};
%% \node (solver) at (3,2) {Constraint Solver/Search};
%% \node[shape=rectangle,fill=pantone157-8] (solution) at (3,1) {Solution};
%% \draw[-] (problem) -- (human);
%% \draw[->] (human) -- (model);
%% \draw[-] (model) -- (reformulation);
%% \draw[->] (reformulation) -- (implementation);
%% \draw[-] (implementation) -- (solver);
%% \draw[->] (solver) -- (0,2) -- (0,3) -- (implementation);
%% \draw[->] (solver) -- (solution);
%% \end{tikzpicture}
%% \end{center}
%% \end{frame}

%\begin{frame}
%\frametitle{What is CP actually used for?}
%\url{http://hsimonis.wordpress.com}
%\includegraphics[width=10cm]{../introduction/images/blog}
%\end{frame}

%% \begin{frame}
%% \frametitle{Exercises}
%% \begin{itemize}
%% \item Install Constraint Solver of your Choice
%% \item MiniZinc (MiningZinc) will be used in later sessions
%% \item Run Sendmoremoney, Sudoku, Queens examples
%% \item Can you choose which propagator to use for alldifferent?
%% \item Can you use different search strategies?
%% \end{itemize}
%% \end{frame}

\begin{frame}
\frametitle{More Learning Resources}
\begin{itemize}
\item Survey of Methods, Resources, and Formats for Teaching Constraint Programming
\begin{itemize}
\item by Tejas Santanam, Helmut Simonis
\item \url{https://doi.org/10.48550/arXiv.2403.12717}
\item Based on survey of community for WTCP 2023
\item \url{https://hsimonis.github.io/WTCP2023/}
\end{itemize}
\end{itemize}
\end{frame}


%%   }

\clearpage
\mode<all>{
\part{What is missing?}
}
\begin{frame}
\partpage
\end{frame}

\begin{frame}
\frametitle{Many Specialized Topics}
\begin{itemize}
\item How to design efficient core engine
\item Hybrids with LP/MIP tools
\item Hybrids with SAT
\item Symmetry breaking
\item Use of MDD/BDD to encode sets of solutions
\item High level modelling tools
\item Debugging/visualization
\end{itemize}
\end{frame}

\begin{frame}
\frametitle{Reformulation}
\begin{itemize}
\item Just because the user has modelled it this way, it doesn't mean we have to solve it that way
\begin{itemize}
\item Replace some constraint(s) by other, equivalent constraints
\item Because we don't have that constraint in our system
\item For performance
\end{itemize}
\end{itemize}
\end{frame}

\begin{frame}
\frametitle{Learning}
\begin{itemize}
\item While solving the problem we can learn how to strengthen the model/search 
\begin{itemize}
\item Understand which constraints/method contribute to propagation and change schedule
\item Learn no-good constraints by explaining failure
\item Adapt search strategy based on search experience
\end{itemize}
\end{itemize}
\end{frame}

%% \begin{frame}
%% \frametitle{Refined Process}
%% \begin{center}
%% \begin{tikzpicture}
%% \node[shape=rectangle,fill=pantone127-4] (problem) at (3,7) {Problem};
%% \node (human) at (3,6) {Human};
%% \node[shape=rectangle,fill=pantone127-4] (model) at (3,5) {User Model};
%% \node (reformulation) at (3,4) {Reformulation};
%% \node[shape=rectangle,fill=pantone127-4] (implementation) at (3,3) {Implementation Model};
%% \node (solver) at (3,2) {Constraint Solver/Search};
%% \node[shape=rectangle,fill=pantone157-8] (solution) at (3,1) {Solution};
%% \draw[-] (problem) -- (human);
%% \draw[->] (human) -- (model);
%% \draw[-] (model) -- (reformulation);
%% \draw[->] (reformulation) -- (implementation);
%% \draw[-] (implementation) -- (solver);
%% \draw[->] (solver) -- (0,2) -- (0,3) -- (implementation);
%% \draw[->] (solver) -- (solution);
%% \end{tikzpicture}
%% \end{center}
%% \end{frame}

%\begin{frame}
%\frametitle{What is CP actually used for?}
%\url{http://hsimonis.wordpress.com}
%\includegraphics[width=10cm]{../introduction/images/blog}
%\end{frame}

%% \begin{frame}
%% \frametitle{Exercises}
%% \begin{itemize}
%% \item Install Constraint Solver of your Choice
%% \item MiniZinc (MiningZinc) will be used in later sessions
%% \item Run Sendmoremoney, Sudoku, Queens examples
%% \item Can you choose which propagator to use for alldifferent?
%% \item Can you use different search strategies?
%% \end{itemize}
%% \end{frame}

\begin{frame}
\frametitle{More Learning Resources}
\begin{itemize}
\item Survey of Methods, Resources, and Formats for Teaching Constraint Programming
\begin{itemize}
\item by Tejas Santanam, Helmut Simonis
\item \url{https://doi.org/10.48550/arXiv.2403.12717}
\item Based on survey of community for WTCP 2023
\item \url{https://hsimonis.github.io/WTCP2023/}
\end{itemize}
\end{itemize}
\end{frame}


}

\mode<all>{
\part{Global Constraints}
}
\begin{frame}
\partpage
\end{frame}
\mode<all>{
\begin{frame}
\frametitle{Objectives}
\begin{itemize}
\item Overview of Core Constraint Programming
\item Three Main Concepts
\begin{itemize}
\item Constraint Propagation
\item Global Constraints
\item Customizing Search
\end{itemize}
%\item Get Some Experience with MiniZinc
\item Topics will be treated in more detail in later parts of the school
\item Based on Examples, not Formal Description
\end{itemize}
\end{frame}

\begin{frame}
\frametitle{Outline}
\begin{itemize}
\item Why Constraint Programming?
\item Constraint Propagation
\item Global Constraints
\item Customizing Search
%\item Choices in Modelling
%\item Decomposition
%\item More Global Constraints
%\item Methodology
%\item What is missing?
\end{itemize}
\end{frame}

% \begin{frame}
  % \frametitle{Using MiniZinc IDE}
  % \begin{itemize}
  % \item Developed in the Australian NICTA project
  % \item Maintained by Monash University
  % \item Modelling tool with multiple back-end solvers
  % \item Available from \url{https://www.minizinc.org/}
  % \end{itemize}
% \end{frame}


% \begin{frame}
% \frametitle{Examples in ECLiPSe}
% \begin{itemize}
% \item Open sourced constraint programming language
% \item Development goes back to 1985
% \item ECRC, ICL, IC-Parc, PTL, Cisco
% \item \url{https://eclipseclp.org/}
% \item Specialities
% \begin{itemize}
% \item Develop new solvers for specific domains
% \item Integration with MIP
% \end{itemize}
% \item Not included in bundled MiniZinc IDE
% \item Specialized visualization tools used here
  % \begin{itemize}
  % \item CP-Viz, Simonis et al. 2010
    % \end{itemize}
% \end{itemize}
% \end{frame}

\begin{frame}
\frametitle{Tutorial Based on ECLiPSe ELearning Course}
\begin{itemize}
\item Self-study course in constraint programming
\item Supported by Cisco Systems and Silicon Valley Community Foundation
\item Multi-media format, video lectures, slides, handout etc
\item {\small \url{https://eclipseclp.org/ELearning/index.html}}
\end{itemize}
\end{frame}

\begin{frame}
\frametitle{Also Part of CRT-AI Constraint Week}
\begin{itemize}
\item Annual one week course on CP and Optimization in Ireland
\item Part of national training program for PhD students in AI
\item \url{https://www.crt-ai.ie/}
\end{itemize}
\end{frame}



\begin{frame}
\frametitle{Constraint Programming - in a nutshell}
\begin{itemize}
\item Declarative description of problems with
\begin{itemize}
\item {\em Variables} which range over (finite) sets of values
\item {\em Constraints} over subsets of variables which restrict possible value combinations
\item A {\em solution} is a value assignment which satisfies all constraints
\end{itemize}

\item Constraint propagation/reasoning
\begin{itemize}
\item Removing inconsistent values for variables
\item Detect failure if constraint can not be satisfied
\item Interaction of constraints via shared variables
\item Incomplete
\end{itemize}

\item Search
\begin{itemize}
\item User controlled assignment of values to variables
\item Each step triggers constraint propagation 
\end{itemize}
\item Different domains require/allow different methods
\end{itemize}
\end{frame}

\begin{frame}
  \frametitle{Constraint Programming is Different}
  \begin{itemize}
  \item Declarative Programming
    \begin{itemize}
    \item Concentrate on what you want
      \item Not how to get there
      \item Program != Algorithm
      \item Program = Model
    \end{itemize}
    \item Applied to Combinatorial Problems
      \begin{itemize}
        \item No complete polynomial algorithms known (exist?)
        \item CP less ad-hoc than heuristics
        \item Models can evolve
  \end{itemize}
  \end{itemize}
  \end{frame}
    
\begin{frame}
  \frametitle{A Subtractive Process}
  \begin{textblock}{4}(8,-3)
    \includegraphics[width=4cm]{../introduction/images/stages}
  \end{textblock}
  \vfill
  \begin{quote}
    ``Oh, bosh, as Mr. Ruskin says. Sculpture, per se, is the simplest thing in the world. All you have to do is to take a big chunk of marble and a hammer and chisel, make up your mind what you are about to create and chip off all the marble you don't want.''-Paris Gaulois.
  \end{quote}
  
  {\tiny Source: \url{https://quoteinvestigator.com/2014/06/22/chip-away/}}
\end{frame}

\begin{frame}
\frametitle{Basic Process}
\begin{center}
\begin{tikzpicture}
\node[shape=rectangle,fill=pantone127-4] (problem) at (2,5) {Problem};
\node (human) at (2,4) {Human};
\node[shape=rectangle,fill=pantone127-4] (model) at (2,3) {Model};
\node (solver) at (2,2) {Constraint Solver/Search};
\node[shape=rectangle,fill=pantone157-8] (solution) at (2,1) {Solution};
\draw[-] (problem) -- (human);
\draw[->] (human) -- (model);
\draw[-] (model) -- (solver);
\draw[->] (solver) -- (solution);
\end{tikzpicture}
\end{center}
\end{frame}

\begin{frame}
\frametitle{More Realistic}
\begin{center}
\begin{tikzpicture}[xscale=4.0]
\node[shape=rectangle,fill=pantone127-4] (problem) at (2,5) {Problem};
\node (human) at (2,4) {Human};
\node[shape=rectangle,fill=pantone127-4] (model) at (2,3) {Model};
\node (solver) at (2,2) {Constraint Solver/Search};
\node[shape=rectangle,fill=pantone24-8] (hangs) at (1,1) {Hangs};
\node[shape=rectangle,fill=pantone157-8] (solution) at (2,1) {Solution};
\node[shape=rectangle,fill=pantone24-8] (wrong) at (3,1) {Wrong Solution};
\draw[-] (problem) -- (human);
\draw[->] (human) -- (model);
\draw[-] (model) -- (solver);
\draw[->] (solver) -- (hangs);
\draw[->] (solver) -- (solution);
\draw[->] (solver) -- (wrong);
\draw[->] (hangs) -- (1,4) -- (human);
\draw[->] (wrong) -- (3,4) -- (human);
\end{tikzpicture}
\end{center}
\end{frame}

\begin{frame}
\frametitle{Dual Role of Model}
\begin{itemize}
\item Allows Human to Express Problem
\begin{itemize}
\item Close to Problem Domain
\item Constraints as Abstractions
\end{itemize}
\item Allows Solver to Execute
\begin{itemize}
\item Variables as Communication Mechanism
\item Constraints as Algorithms
\end{itemize}
\end{itemize}
\end{frame}

\begin{frame}
\frametitle{Modelling Frameworks}
\begin{itemize}
\item \href{https://www.minizinc.org/}{MiniZinc} (NICTA, Monash University, Australia)
\item \href{https://github.com/eomahony/Numberjack}{NumberJack} (Insight, Ireland) 
\item \href{https://www-users.york.ac.uk/peter.nightingale/savilerow//}{EssencePrime/SavilleRow} (UK) 
\item \href{https://cpmpy.readthedocs.io/en/latest/}{CPMpy} (KU Leuven)
\item Allow use of multiple back-end solvers
\item Compile model into variants for each solver
\item A priori solver independent model(CP, MIP, SAT)
\end{itemize}
\end{frame} 

\begin{frame}
\frametitle{Framework Process}
\begin{center}
\begin{tikzpicture}[xscale=2]
\node[shape=rectangle,fill=pantone127-4] (problem) at (2.5,5) {Problem};
\node (human) at (2.5,4) {Human};
\node[shape=rectangle,fill=pantone127-4] (model) at (2.5,3) {Model};
\node (compile) at (2.5,2) {Compile/Reformulate};
\node (solver1) at (1,1) {CP};
\node (solver2) at (2,1) {MIP};
\node (solver3) at (3,1) {SAT};
\node (solver4) at (4,1) {Other};
\node[shape=rectangle,fill=pantone157-8] (solution1) at (1,0) {Solution};
\node[shape=rectangle,fill=pantone157-8] (solution2) at (2,0) {Solution};
\node[shape=rectangle,fill=pantone157-8] (solution3) at (3,0) {Solution};
\node[shape=rectangle,fill=pantone157-8] (solution4) at (4,0) {Solution};
\draw[-] (problem) -- (human);
\draw[->] (human) -- (model);
\draw[-] (model) -- (compile);
\draw[-] (compile) -- (solver1);
\draw[-] (compile) -- (solver2);
\draw[-] (compile) -- (solver3);
\draw[-] (compile) -- (solver4);
\draw[->] (solver1) -- (solution1);
\draw[->] (solver2) -- (solution2);
\draw[->] (solver3) -- (solution3);
\draw[->] (solver4) -- (solution4);
\end{tikzpicture}
\end{center}
\end{frame}

% \begin{frame}
  % \frametitle{Do It Now!}
  % \begin{itemize}
  % \item Download and install Minizinc
  % \item \url{https://www.minizinc.org/}
  % \end{itemize}
% \end{frame}


\begin{frame}
\frametitle{Why use Puzzles as Examples?}
\begin{itemize}
\item Easy to understand the problem
\item Solvable by hand without specialized knowledge
\item Possible to compare automated to manual solving process
\end{itemize}

\begin{quote}
The puzzle, though inanimate, is presented as a solvable problem without lasting negative consequences, a very low-risk low-reward situation. By being a puzzle, the object is attempting to convince the user that it must be completed.
\end{quote}
{\tiny Source: \href{https://www.everydayrhetoric.com/2020/03/the-rhetoric-of-puzzles/}{Every Day Rhetoric}}
\end{frame}


\mode<all>{
\part{Basic Constraint Propagation}
}
\begin{frame}
\partpage
\end{frame}
\mode<all>{
\begin{frame}
\frametitle{Objectives}
\begin{itemize}
\item Overview of Core Constraint Programming
\item Three Main Concepts
\begin{itemize}
\item Constraint Propagation
\item Global Constraints
\item Customizing Search
\end{itemize}
%\item Get Some Experience with MiniZinc
\item Topics will be treated in more detail in later parts of the school
\item Based on Examples, not Formal Description
\end{itemize}
\end{frame}

\begin{frame}
\frametitle{Outline}
\begin{itemize}
\item Why Constraint Programming?
\item Constraint Propagation
\item Global Constraints
\item Customizing Search
%\item Choices in Modelling
%\item Decomposition
%\item More Global Constraints
%\item Methodology
%\item What is missing?
\end{itemize}
\end{frame}

% \begin{frame}
  % \frametitle{Using MiniZinc IDE}
  % \begin{itemize}
  % \item Developed in the Australian NICTA project
  % \item Maintained by Monash University
  % \item Modelling tool with multiple back-end solvers
  % \item Available from \url{https://www.minizinc.org/}
  % \end{itemize}
% \end{frame}


% \begin{frame}
% \frametitle{Examples in ECLiPSe}
% \begin{itemize}
% \item Open sourced constraint programming language
% \item Development goes back to 1985
% \item ECRC, ICL, IC-Parc, PTL, Cisco
% \item \url{https://eclipseclp.org/}
% \item Specialities
% \begin{itemize}
% \item Develop new solvers for specific domains
% \item Integration with MIP
% \end{itemize}
% \item Not included in bundled MiniZinc IDE
% \item Specialized visualization tools used here
  % \begin{itemize}
  % \item CP-Viz, Simonis et al. 2010
    % \end{itemize}
% \end{itemize}
% \end{frame}

\begin{frame}
\frametitle{Tutorial Based on ECLiPSe ELearning Course}
\begin{itemize}
\item Self-study course in constraint programming
\item Supported by Cisco Systems and Silicon Valley Community Foundation
\item Multi-media format, video lectures, slides, handout etc
\item {\small \url{https://eclipseclp.org/ELearning/index.html}}
\end{itemize}
\end{frame}

\begin{frame}
\frametitle{Also Part of CRT-AI Constraint Week}
\begin{itemize}
\item Annual one week course on CP and Optimization in Ireland
\item Part of national training program for PhD students in AI
\item \url{https://www.crt-ai.ie/}
\end{itemize}
\end{frame}



\begin{frame}
\frametitle{Constraint Programming - in a nutshell}
\begin{itemize}
\item Declarative description of problems with
\begin{itemize}
\item {\em Variables} which range over (finite) sets of values
\item {\em Constraints} over subsets of variables which restrict possible value combinations
\item A {\em solution} is a value assignment which satisfies all constraints
\end{itemize}

\item Constraint propagation/reasoning
\begin{itemize}
\item Removing inconsistent values for variables
\item Detect failure if constraint can not be satisfied
\item Interaction of constraints via shared variables
\item Incomplete
\end{itemize}

\item Search
\begin{itemize}
\item User controlled assignment of values to variables
\item Each step triggers constraint propagation 
\end{itemize}
\item Different domains require/allow different methods
\end{itemize}
\end{frame}

\begin{frame}
  \frametitle{Constraint Programming is Different}
  \begin{itemize}
  \item Declarative Programming
    \begin{itemize}
    \item Concentrate on what you want
      \item Not how to get there
      \item Program != Algorithm
      \item Program = Model
    \end{itemize}
    \item Applied to Combinatorial Problems
      \begin{itemize}
        \item No complete polynomial algorithms known (exist?)
        \item CP less ad-hoc than heuristics
        \item Models can evolve
  \end{itemize}
  \end{itemize}
  \end{frame}
    
\begin{frame}
  \frametitle{A Subtractive Process}
  \begin{textblock}{4}(8,-3)
    \includegraphics[width=4cm]{../introduction/images/stages}
  \end{textblock}
  \vfill
  \begin{quote}
    ``Oh, bosh, as Mr. Ruskin says. Sculpture, per se, is the simplest thing in the world. All you have to do is to take a big chunk of marble and a hammer and chisel, make up your mind what you are about to create and chip off all the marble you don't want.''-Paris Gaulois.
  \end{quote}
  
  {\tiny Source: \url{https://quoteinvestigator.com/2014/06/22/chip-away/}}
\end{frame}

\begin{frame}
\frametitle{Basic Process}
\begin{center}
\begin{tikzpicture}
\node[shape=rectangle,fill=pantone127-4] (problem) at (2,5) {Problem};
\node (human) at (2,4) {Human};
\node[shape=rectangle,fill=pantone127-4] (model) at (2,3) {Model};
\node (solver) at (2,2) {Constraint Solver/Search};
\node[shape=rectangle,fill=pantone157-8] (solution) at (2,1) {Solution};
\draw[-] (problem) -- (human);
\draw[->] (human) -- (model);
\draw[-] (model) -- (solver);
\draw[->] (solver) -- (solution);
\end{tikzpicture}
\end{center}
\end{frame}

\begin{frame}
\frametitle{More Realistic}
\begin{center}
\begin{tikzpicture}[xscale=4.0]
\node[shape=rectangle,fill=pantone127-4] (problem) at (2,5) {Problem};
\node (human) at (2,4) {Human};
\node[shape=rectangle,fill=pantone127-4] (model) at (2,3) {Model};
\node (solver) at (2,2) {Constraint Solver/Search};
\node[shape=rectangle,fill=pantone24-8] (hangs) at (1,1) {Hangs};
\node[shape=rectangle,fill=pantone157-8] (solution) at (2,1) {Solution};
\node[shape=rectangle,fill=pantone24-8] (wrong) at (3,1) {Wrong Solution};
\draw[-] (problem) -- (human);
\draw[->] (human) -- (model);
\draw[-] (model) -- (solver);
\draw[->] (solver) -- (hangs);
\draw[->] (solver) -- (solution);
\draw[->] (solver) -- (wrong);
\draw[->] (hangs) -- (1,4) -- (human);
\draw[->] (wrong) -- (3,4) -- (human);
\end{tikzpicture}
\end{center}
\end{frame}

\begin{frame}
\frametitle{Dual Role of Model}
\begin{itemize}
\item Allows Human to Express Problem
\begin{itemize}
\item Close to Problem Domain
\item Constraints as Abstractions
\end{itemize}
\item Allows Solver to Execute
\begin{itemize}
\item Variables as Communication Mechanism
\item Constraints as Algorithms
\end{itemize}
\end{itemize}
\end{frame}

\begin{frame}
\frametitle{Modelling Frameworks}
\begin{itemize}
\item \href{https://www.minizinc.org/}{MiniZinc} (NICTA, Monash University, Australia)
\item \href{https://github.com/eomahony/Numberjack}{NumberJack} (Insight, Ireland) 
\item \href{https://www-users.york.ac.uk/peter.nightingale/savilerow//}{EssencePrime/SavilleRow} (UK) 
\item \href{https://cpmpy.readthedocs.io/en/latest/}{CPMpy} (KU Leuven)
\item Allow use of multiple back-end solvers
\item Compile model into variants for each solver
\item A priori solver independent model(CP, MIP, SAT)
\end{itemize}
\end{frame} 

\begin{frame}
\frametitle{Framework Process}
\begin{center}
\begin{tikzpicture}[xscale=2]
\node[shape=rectangle,fill=pantone127-4] (problem) at (2.5,5) {Problem};
\node (human) at (2.5,4) {Human};
\node[shape=rectangle,fill=pantone127-4] (model) at (2.5,3) {Model};
\node (compile) at (2.5,2) {Compile/Reformulate};
\node (solver1) at (1,1) {CP};
\node (solver2) at (2,1) {MIP};
\node (solver3) at (3,1) {SAT};
\node (solver4) at (4,1) {Other};
\node[shape=rectangle,fill=pantone157-8] (solution1) at (1,0) {Solution};
\node[shape=rectangle,fill=pantone157-8] (solution2) at (2,0) {Solution};
\node[shape=rectangle,fill=pantone157-8] (solution3) at (3,0) {Solution};
\node[shape=rectangle,fill=pantone157-8] (solution4) at (4,0) {Solution};
\draw[-] (problem) -- (human);
\draw[->] (human) -- (model);
\draw[-] (model) -- (compile);
\draw[-] (compile) -- (solver1);
\draw[-] (compile) -- (solver2);
\draw[-] (compile) -- (solver3);
\draw[-] (compile) -- (solver4);
\draw[->] (solver1) -- (solution1);
\draw[->] (solver2) -- (solution2);
\draw[->] (solver3) -- (solution3);
\draw[->] (solver4) -- (solution4);
\end{tikzpicture}
\end{center}
\end{frame}

% \begin{frame}
  % \frametitle{Do It Now!}
  % \begin{itemize}
  % \item Download and install Minizinc
  % \item \url{https://www.minizinc.org/}
  % \end{itemize}
% \end{frame}


\begin{frame}
\frametitle{Why use Puzzles as Examples?}
\begin{itemize}
\item Easy to understand the problem
\item Solvable by hand without specialized knowledge
\item Possible to compare automated to manual solving process
\end{itemize}

\begin{quote}
The puzzle, though inanimate, is presented as a solvable problem without lasting negative consequences, a very low-risk low-reward situation. By being a puzzle, the object is attempting to convince the user that it must be completed.
\end{quote}
{\tiny Source: \href{https://www.everydayrhetoric.com/2020/03/the-rhetoric-of-puzzles/}{Every Day Rhetoric}}
\end{frame}


\mode<all>{
\part{Basic Constraint Propagation}
}
\begin{frame}
\partpage
\end{frame}
\mode<all>{
\input{../sendmore/body}
}

\mode<all>{
\part{Global Constraints}
}
\begin{frame}
\partpage
\end{frame}
\mode<all>{
\input{../sudoku/body}
}

\clearpage
\mode<all>{
\part{Customizing Search}
}
\begin{frame}
\partpage
\end{frame}
\mode<all>{
\input{../nqueen/body}
}


%% \clearpage
%% \mode<all>{
%%   \part{Problem Decomposition}
%% }
%% \begin{frame}
%%   \partpage
%% \end{frame}
%% \mode<all>{
%%   \input{../party/body}
%%   }

%% \clearpage
%% \mode<all>{
%%   \part{More Global Constraints}
%% }
%% \begin{frame}
%%   \partpage
%% \end{frame}
%% \mode<all>{
%%   \input{../car/body}
%%   }

%% \clearpage
%% \mode<all>{
%%   \part{Methodology}
%% }
%% \begin{frame}
%%   \partpage
%% \end{frame}
%% \mode<all>{
%%   \input{../methodology/body}
%%   }

\clearpage
\mode<all>{
\part{What is missing?}
}
\begin{frame}
\partpage
\end{frame}

\begin{frame}
\frametitle{Many Specialized Topics}
\begin{itemize}
\item How to design efficient core engine
\item Hybrids with LP/MIP tools
\item Hybrids with SAT
\item Symmetry breaking
\item Use of MDD/BDD to encode sets of solutions
\item High level modelling tools
\item Debugging/visualization
\end{itemize}
\end{frame}

\begin{frame}
\frametitle{Reformulation}
\begin{itemize}
\item Just because the user has modelled it this way, it doesn't mean we have to solve it that way
\begin{itemize}
\item Replace some constraint(s) by other, equivalent constraints
\item Because we don't have that constraint in our system
\item For performance
\end{itemize}
\end{itemize}
\end{frame}

\begin{frame}
\frametitle{Learning}
\begin{itemize}
\item While solving the problem we can learn how to strengthen the model/search 
\begin{itemize}
\item Understand which constraints/method contribute to propagation and change schedule
\item Learn no-good constraints by explaining failure
\item Adapt search strategy based on search experience
\end{itemize}
\end{itemize}
\end{frame}

%% \begin{frame}
%% \frametitle{Refined Process}
%% \begin{center}
%% \begin{tikzpicture}
%% \node[shape=rectangle,fill=pantone127-4] (problem) at (3,7) {Problem};
%% \node (human) at (3,6) {Human};
%% \node[shape=rectangle,fill=pantone127-4] (model) at (3,5) {User Model};
%% \node (reformulation) at (3,4) {Reformulation};
%% \node[shape=rectangle,fill=pantone127-4] (implementation) at (3,3) {Implementation Model};
%% \node (solver) at (3,2) {Constraint Solver/Search};
%% \node[shape=rectangle,fill=pantone157-8] (solution) at (3,1) {Solution};
%% \draw[-] (problem) -- (human);
%% \draw[->] (human) -- (model);
%% \draw[-] (model) -- (reformulation);
%% \draw[->] (reformulation) -- (implementation);
%% \draw[-] (implementation) -- (solver);
%% \draw[->] (solver) -- (0,2) -- (0,3) -- (implementation);
%% \draw[->] (solver) -- (solution);
%% \end{tikzpicture}
%% \end{center}
%% \end{frame}

%\begin{frame}
%\frametitle{What is CP actually used for?}
%\url{http://hsimonis.wordpress.com}
%\includegraphics[width=10cm]{../introduction/images/blog}
%\end{frame}

%% \begin{frame}
%% \frametitle{Exercises}
%% \begin{itemize}
%% \item Install Constraint Solver of your Choice
%% \item MiniZinc (MiningZinc) will be used in later sessions
%% \item Run Sendmoremoney, Sudoku, Queens examples
%% \item Can you choose which propagator to use for alldifferent?
%% \item Can you use different search strategies?
%% \end{itemize}
%% \end{frame}

\begin{frame}
\frametitle{More Learning Resources}
\begin{itemize}
\item Survey of Methods, Resources, and Formats for Teaching Constraint Programming
\begin{itemize}
\item by Tejas Santanam, Helmut Simonis
\item \url{https://doi.org/10.48550/arXiv.2403.12717}
\item Based on survey of community for WTCP 2023
\item \url{https://hsimonis.github.io/WTCP2023/}
\end{itemize}
\end{itemize}
\end{frame}


}

\mode<all>{
\part{Global Constraints}
}
\begin{frame}
\partpage
\end{frame}
\mode<all>{
\begin{frame}
\frametitle{Objectives}
\begin{itemize}
\item Overview of Core Constraint Programming
\item Three Main Concepts
\begin{itemize}
\item Constraint Propagation
\item Global Constraints
\item Customizing Search
\end{itemize}
%\item Get Some Experience with MiniZinc
\item Topics will be treated in more detail in later parts of the school
\item Based on Examples, not Formal Description
\end{itemize}
\end{frame}

\begin{frame}
\frametitle{Outline}
\begin{itemize}
\item Why Constraint Programming?
\item Constraint Propagation
\item Global Constraints
\item Customizing Search
%\item Choices in Modelling
%\item Decomposition
%\item More Global Constraints
%\item Methodology
%\item What is missing?
\end{itemize}
\end{frame}

% \begin{frame}
  % \frametitle{Using MiniZinc IDE}
  % \begin{itemize}
  % \item Developed in the Australian NICTA project
  % \item Maintained by Monash University
  % \item Modelling tool with multiple back-end solvers
  % \item Available from \url{https://www.minizinc.org/}
  % \end{itemize}
% \end{frame}


% \begin{frame}
% \frametitle{Examples in ECLiPSe}
% \begin{itemize}
% \item Open sourced constraint programming language
% \item Development goes back to 1985
% \item ECRC, ICL, IC-Parc, PTL, Cisco
% \item \url{https://eclipseclp.org/}
% \item Specialities
% \begin{itemize}
% \item Develop new solvers for specific domains
% \item Integration with MIP
% \end{itemize}
% \item Not included in bundled MiniZinc IDE
% \item Specialized visualization tools used here
  % \begin{itemize}
  % \item CP-Viz, Simonis et al. 2010
    % \end{itemize}
% \end{itemize}
% \end{frame}

\begin{frame}
\frametitle{Tutorial Based on ECLiPSe ELearning Course}
\begin{itemize}
\item Self-study course in constraint programming
\item Supported by Cisco Systems and Silicon Valley Community Foundation
\item Multi-media format, video lectures, slides, handout etc
\item {\small \url{https://eclipseclp.org/ELearning/index.html}}
\end{itemize}
\end{frame}

\begin{frame}
\frametitle{Also Part of CRT-AI Constraint Week}
\begin{itemize}
\item Annual one week course on CP and Optimization in Ireland
\item Part of national training program for PhD students in AI
\item \url{https://www.crt-ai.ie/}
\end{itemize}
\end{frame}



\begin{frame}
\frametitle{Constraint Programming - in a nutshell}
\begin{itemize}
\item Declarative description of problems with
\begin{itemize}
\item {\em Variables} which range over (finite) sets of values
\item {\em Constraints} over subsets of variables which restrict possible value combinations
\item A {\em solution} is a value assignment which satisfies all constraints
\end{itemize}

\item Constraint propagation/reasoning
\begin{itemize}
\item Removing inconsistent values for variables
\item Detect failure if constraint can not be satisfied
\item Interaction of constraints via shared variables
\item Incomplete
\end{itemize}

\item Search
\begin{itemize}
\item User controlled assignment of values to variables
\item Each step triggers constraint propagation 
\end{itemize}
\item Different domains require/allow different methods
\end{itemize}
\end{frame}

\begin{frame}
  \frametitle{Constraint Programming is Different}
  \begin{itemize}
  \item Declarative Programming
    \begin{itemize}
    \item Concentrate on what you want
      \item Not how to get there
      \item Program != Algorithm
      \item Program = Model
    \end{itemize}
    \item Applied to Combinatorial Problems
      \begin{itemize}
        \item No complete polynomial algorithms known (exist?)
        \item CP less ad-hoc than heuristics
        \item Models can evolve
  \end{itemize}
  \end{itemize}
  \end{frame}
    
\begin{frame}
  \frametitle{A Subtractive Process}
  \begin{textblock}{4}(8,-3)
    \includegraphics[width=4cm]{../introduction/images/stages}
  \end{textblock}
  \vfill
  \begin{quote}
    ``Oh, bosh, as Mr. Ruskin says. Sculpture, per se, is the simplest thing in the world. All you have to do is to take a big chunk of marble and a hammer and chisel, make up your mind what you are about to create and chip off all the marble you don't want.''-Paris Gaulois.
  \end{quote}
  
  {\tiny Source: \url{https://quoteinvestigator.com/2014/06/22/chip-away/}}
\end{frame}

\begin{frame}
\frametitle{Basic Process}
\begin{center}
\begin{tikzpicture}
\node[shape=rectangle,fill=pantone127-4] (problem) at (2,5) {Problem};
\node (human) at (2,4) {Human};
\node[shape=rectangle,fill=pantone127-4] (model) at (2,3) {Model};
\node (solver) at (2,2) {Constraint Solver/Search};
\node[shape=rectangle,fill=pantone157-8] (solution) at (2,1) {Solution};
\draw[-] (problem) -- (human);
\draw[->] (human) -- (model);
\draw[-] (model) -- (solver);
\draw[->] (solver) -- (solution);
\end{tikzpicture}
\end{center}
\end{frame}

\begin{frame}
\frametitle{More Realistic}
\begin{center}
\begin{tikzpicture}[xscale=4.0]
\node[shape=rectangle,fill=pantone127-4] (problem) at (2,5) {Problem};
\node (human) at (2,4) {Human};
\node[shape=rectangle,fill=pantone127-4] (model) at (2,3) {Model};
\node (solver) at (2,2) {Constraint Solver/Search};
\node[shape=rectangle,fill=pantone24-8] (hangs) at (1,1) {Hangs};
\node[shape=rectangle,fill=pantone157-8] (solution) at (2,1) {Solution};
\node[shape=rectangle,fill=pantone24-8] (wrong) at (3,1) {Wrong Solution};
\draw[-] (problem) -- (human);
\draw[->] (human) -- (model);
\draw[-] (model) -- (solver);
\draw[->] (solver) -- (hangs);
\draw[->] (solver) -- (solution);
\draw[->] (solver) -- (wrong);
\draw[->] (hangs) -- (1,4) -- (human);
\draw[->] (wrong) -- (3,4) -- (human);
\end{tikzpicture}
\end{center}
\end{frame}

\begin{frame}
\frametitle{Dual Role of Model}
\begin{itemize}
\item Allows Human to Express Problem
\begin{itemize}
\item Close to Problem Domain
\item Constraints as Abstractions
\end{itemize}
\item Allows Solver to Execute
\begin{itemize}
\item Variables as Communication Mechanism
\item Constraints as Algorithms
\end{itemize}
\end{itemize}
\end{frame}

\begin{frame}
\frametitle{Modelling Frameworks}
\begin{itemize}
\item \href{https://www.minizinc.org/}{MiniZinc} (NICTA, Monash University, Australia)
\item \href{https://github.com/eomahony/Numberjack}{NumberJack} (Insight, Ireland) 
\item \href{https://www-users.york.ac.uk/peter.nightingale/savilerow//}{EssencePrime/SavilleRow} (UK) 
\item \href{https://cpmpy.readthedocs.io/en/latest/}{CPMpy} (KU Leuven)
\item Allow use of multiple back-end solvers
\item Compile model into variants for each solver
\item A priori solver independent model(CP, MIP, SAT)
\end{itemize}
\end{frame} 

\begin{frame}
\frametitle{Framework Process}
\begin{center}
\begin{tikzpicture}[xscale=2]
\node[shape=rectangle,fill=pantone127-4] (problem) at (2.5,5) {Problem};
\node (human) at (2.5,4) {Human};
\node[shape=rectangle,fill=pantone127-4] (model) at (2.5,3) {Model};
\node (compile) at (2.5,2) {Compile/Reformulate};
\node (solver1) at (1,1) {CP};
\node (solver2) at (2,1) {MIP};
\node (solver3) at (3,1) {SAT};
\node (solver4) at (4,1) {Other};
\node[shape=rectangle,fill=pantone157-8] (solution1) at (1,0) {Solution};
\node[shape=rectangle,fill=pantone157-8] (solution2) at (2,0) {Solution};
\node[shape=rectangle,fill=pantone157-8] (solution3) at (3,0) {Solution};
\node[shape=rectangle,fill=pantone157-8] (solution4) at (4,0) {Solution};
\draw[-] (problem) -- (human);
\draw[->] (human) -- (model);
\draw[-] (model) -- (compile);
\draw[-] (compile) -- (solver1);
\draw[-] (compile) -- (solver2);
\draw[-] (compile) -- (solver3);
\draw[-] (compile) -- (solver4);
\draw[->] (solver1) -- (solution1);
\draw[->] (solver2) -- (solution2);
\draw[->] (solver3) -- (solution3);
\draw[->] (solver4) -- (solution4);
\end{tikzpicture}
\end{center}
\end{frame}

% \begin{frame}
  % \frametitle{Do It Now!}
  % \begin{itemize}
  % \item Download and install Minizinc
  % \item \url{https://www.minizinc.org/}
  % \end{itemize}
% \end{frame}


\begin{frame}
\frametitle{Why use Puzzles as Examples?}
\begin{itemize}
\item Easy to understand the problem
\item Solvable by hand without specialized knowledge
\item Possible to compare automated to manual solving process
\end{itemize}

\begin{quote}
The puzzle, though inanimate, is presented as a solvable problem without lasting negative consequences, a very low-risk low-reward situation. By being a puzzle, the object is attempting to convince the user that it must be completed.
\end{quote}
{\tiny Source: \href{https://www.everydayrhetoric.com/2020/03/the-rhetoric-of-puzzles/}{Every Day Rhetoric}}
\end{frame}


\mode<all>{
\part{Basic Constraint Propagation}
}
\begin{frame}
\partpage
\end{frame}
\mode<all>{
\input{../sendmore/body}
}

\mode<all>{
\part{Global Constraints}
}
\begin{frame}
\partpage
\end{frame}
\mode<all>{
\input{../sudoku/body}
}

\clearpage
\mode<all>{
\part{Customizing Search}
}
\begin{frame}
\partpage
\end{frame}
\mode<all>{
\input{../nqueen/body}
}


%% \clearpage
%% \mode<all>{
%%   \part{Problem Decomposition}
%% }
%% \begin{frame}
%%   \partpage
%% \end{frame}
%% \mode<all>{
%%   \input{../party/body}
%%   }

%% \clearpage
%% \mode<all>{
%%   \part{More Global Constraints}
%% }
%% \begin{frame}
%%   \partpage
%% \end{frame}
%% \mode<all>{
%%   \input{../car/body}
%%   }

%% \clearpage
%% \mode<all>{
%%   \part{Methodology}
%% }
%% \begin{frame}
%%   \partpage
%% \end{frame}
%% \mode<all>{
%%   \input{../methodology/body}
%%   }

\clearpage
\mode<all>{
\part{What is missing?}
}
\begin{frame}
\partpage
\end{frame}

\begin{frame}
\frametitle{Many Specialized Topics}
\begin{itemize}
\item How to design efficient core engine
\item Hybrids with LP/MIP tools
\item Hybrids with SAT
\item Symmetry breaking
\item Use of MDD/BDD to encode sets of solutions
\item High level modelling tools
\item Debugging/visualization
\end{itemize}
\end{frame}

\begin{frame}
\frametitle{Reformulation}
\begin{itemize}
\item Just because the user has modelled it this way, it doesn't mean we have to solve it that way
\begin{itemize}
\item Replace some constraint(s) by other, equivalent constraints
\item Because we don't have that constraint in our system
\item For performance
\end{itemize}
\end{itemize}
\end{frame}

\begin{frame}
\frametitle{Learning}
\begin{itemize}
\item While solving the problem we can learn how to strengthen the model/search 
\begin{itemize}
\item Understand which constraints/method contribute to propagation and change schedule
\item Learn no-good constraints by explaining failure
\item Adapt search strategy based on search experience
\end{itemize}
\end{itemize}
\end{frame}

%% \begin{frame}
%% \frametitle{Refined Process}
%% \begin{center}
%% \begin{tikzpicture}
%% \node[shape=rectangle,fill=pantone127-4] (problem) at (3,7) {Problem};
%% \node (human) at (3,6) {Human};
%% \node[shape=rectangle,fill=pantone127-4] (model) at (3,5) {User Model};
%% \node (reformulation) at (3,4) {Reformulation};
%% \node[shape=rectangle,fill=pantone127-4] (implementation) at (3,3) {Implementation Model};
%% \node (solver) at (3,2) {Constraint Solver/Search};
%% \node[shape=rectangle,fill=pantone157-8] (solution) at (3,1) {Solution};
%% \draw[-] (problem) -- (human);
%% \draw[->] (human) -- (model);
%% \draw[-] (model) -- (reformulation);
%% \draw[->] (reformulation) -- (implementation);
%% \draw[-] (implementation) -- (solver);
%% \draw[->] (solver) -- (0,2) -- (0,3) -- (implementation);
%% \draw[->] (solver) -- (solution);
%% \end{tikzpicture}
%% \end{center}
%% \end{frame}

%\begin{frame}
%\frametitle{What is CP actually used for?}
%\url{http://hsimonis.wordpress.com}
%\includegraphics[width=10cm]{../introduction/images/blog}
%\end{frame}

%% \begin{frame}
%% \frametitle{Exercises}
%% \begin{itemize}
%% \item Install Constraint Solver of your Choice
%% \item MiniZinc (MiningZinc) will be used in later sessions
%% \item Run Sendmoremoney, Sudoku, Queens examples
%% \item Can you choose which propagator to use for alldifferent?
%% \item Can you use different search strategies?
%% \end{itemize}
%% \end{frame}

\begin{frame}
\frametitle{More Learning Resources}
\begin{itemize}
\item Survey of Methods, Resources, and Formats for Teaching Constraint Programming
\begin{itemize}
\item by Tejas Santanam, Helmut Simonis
\item \url{https://doi.org/10.48550/arXiv.2403.12717}
\item Based on survey of community for WTCP 2023
\item \url{https://hsimonis.github.io/WTCP2023/}
\end{itemize}
\end{itemize}
\end{frame}


}

\clearpage
\mode<all>{
\part{Customizing Search}
}
\begin{frame}
\partpage
\end{frame}
\mode<all>{
\begin{frame}
\frametitle{Objectives}
\begin{itemize}
\item Overview of Core Constraint Programming
\item Three Main Concepts
\begin{itemize}
\item Constraint Propagation
\item Global Constraints
\item Customizing Search
\end{itemize}
%\item Get Some Experience with MiniZinc
\item Topics will be treated in more detail in later parts of the school
\item Based on Examples, not Formal Description
\end{itemize}
\end{frame}

\begin{frame}
\frametitle{Outline}
\begin{itemize}
\item Why Constraint Programming?
\item Constraint Propagation
\item Global Constraints
\item Customizing Search
%\item Choices in Modelling
%\item Decomposition
%\item More Global Constraints
%\item Methodology
%\item What is missing?
\end{itemize}
\end{frame}

% \begin{frame}
  % \frametitle{Using MiniZinc IDE}
  % \begin{itemize}
  % \item Developed in the Australian NICTA project
  % \item Maintained by Monash University
  % \item Modelling tool with multiple back-end solvers
  % \item Available from \url{https://www.minizinc.org/}
  % \end{itemize}
% \end{frame}


% \begin{frame}
% \frametitle{Examples in ECLiPSe}
% \begin{itemize}
% \item Open sourced constraint programming language
% \item Development goes back to 1985
% \item ECRC, ICL, IC-Parc, PTL, Cisco
% \item \url{https://eclipseclp.org/}
% \item Specialities
% \begin{itemize}
% \item Develop new solvers for specific domains
% \item Integration with MIP
% \end{itemize}
% \item Not included in bundled MiniZinc IDE
% \item Specialized visualization tools used here
  % \begin{itemize}
  % \item CP-Viz, Simonis et al. 2010
    % \end{itemize}
% \end{itemize}
% \end{frame}

\begin{frame}
\frametitle{Tutorial Based on ECLiPSe ELearning Course}
\begin{itemize}
\item Self-study course in constraint programming
\item Supported by Cisco Systems and Silicon Valley Community Foundation
\item Multi-media format, video lectures, slides, handout etc
\item {\small \url{https://eclipseclp.org/ELearning/index.html}}
\end{itemize}
\end{frame}

\begin{frame}
\frametitle{Also Part of CRT-AI Constraint Week}
\begin{itemize}
\item Annual one week course on CP and Optimization in Ireland
\item Part of national training program for PhD students in AI
\item \url{https://www.crt-ai.ie/}
\end{itemize}
\end{frame}



\begin{frame}
\frametitle{Constraint Programming - in a nutshell}
\begin{itemize}
\item Declarative description of problems with
\begin{itemize}
\item {\em Variables} which range over (finite) sets of values
\item {\em Constraints} over subsets of variables which restrict possible value combinations
\item A {\em solution} is a value assignment which satisfies all constraints
\end{itemize}

\item Constraint propagation/reasoning
\begin{itemize}
\item Removing inconsistent values for variables
\item Detect failure if constraint can not be satisfied
\item Interaction of constraints via shared variables
\item Incomplete
\end{itemize}

\item Search
\begin{itemize}
\item User controlled assignment of values to variables
\item Each step triggers constraint propagation 
\end{itemize}
\item Different domains require/allow different methods
\end{itemize}
\end{frame}

\begin{frame}
  \frametitle{Constraint Programming is Different}
  \begin{itemize}
  \item Declarative Programming
    \begin{itemize}
    \item Concentrate on what you want
      \item Not how to get there
      \item Program != Algorithm
      \item Program = Model
    \end{itemize}
    \item Applied to Combinatorial Problems
      \begin{itemize}
        \item No complete polynomial algorithms known (exist?)
        \item CP less ad-hoc than heuristics
        \item Models can evolve
  \end{itemize}
  \end{itemize}
  \end{frame}
    
\begin{frame}
  \frametitle{A Subtractive Process}
  \begin{textblock}{4}(8,-3)
    \includegraphics[width=4cm]{../introduction/images/stages}
  \end{textblock}
  \vfill
  \begin{quote}
    ``Oh, bosh, as Mr. Ruskin says. Sculpture, per se, is the simplest thing in the world. All you have to do is to take a big chunk of marble and a hammer and chisel, make up your mind what you are about to create and chip off all the marble you don't want.''-Paris Gaulois.
  \end{quote}
  
  {\tiny Source: \url{https://quoteinvestigator.com/2014/06/22/chip-away/}}
\end{frame}

\begin{frame}
\frametitle{Basic Process}
\begin{center}
\begin{tikzpicture}
\node[shape=rectangle,fill=pantone127-4] (problem) at (2,5) {Problem};
\node (human) at (2,4) {Human};
\node[shape=rectangle,fill=pantone127-4] (model) at (2,3) {Model};
\node (solver) at (2,2) {Constraint Solver/Search};
\node[shape=rectangle,fill=pantone157-8] (solution) at (2,1) {Solution};
\draw[-] (problem) -- (human);
\draw[->] (human) -- (model);
\draw[-] (model) -- (solver);
\draw[->] (solver) -- (solution);
\end{tikzpicture}
\end{center}
\end{frame}

\begin{frame}
\frametitle{More Realistic}
\begin{center}
\begin{tikzpicture}[xscale=4.0]
\node[shape=rectangle,fill=pantone127-4] (problem) at (2,5) {Problem};
\node (human) at (2,4) {Human};
\node[shape=rectangle,fill=pantone127-4] (model) at (2,3) {Model};
\node (solver) at (2,2) {Constraint Solver/Search};
\node[shape=rectangle,fill=pantone24-8] (hangs) at (1,1) {Hangs};
\node[shape=rectangle,fill=pantone157-8] (solution) at (2,1) {Solution};
\node[shape=rectangle,fill=pantone24-8] (wrong) at (3,1) {Wrong Solution};
\draw[-] (problem) -- (human);
\draw[->] (human) -- (model);
\draw[-] (model) -- (solver);
\draw[->] (solver) -- (hangs);
\draw[->] (solver) -- (solution);
\draw[->] (solver) -- (wrong);
\draw[->] (hangs) -- (1,4) -- (human);
\draw[->] (wrong) -- (3,4) -- (human);
\end{tikzpicture}
\end{center}
\end{frame}

\begin{frame}
\frametitle{Dual Role of Model}
\begin{itemize}
\item Allows Human to Express Problem
\begin{itemize}
\item Close to Problem Domain
\item Constraints as Abstractions
\end{itemize}
\item Allows Solver to Execute
\begin{itemize}
\item Variables as Communication Mechanism
\item Constraints as Algorithms
\end{itemize}
\end{itemize}
\end{frame}

\begin{frame}
\frametitle{Modelling Frameworks}
\begin{itemize}
\item \href{https://www.minizinc.org/}{MiniZinc} (NICTA, Monash University, Australia)
\item \href{https://github.com/eomahony/Numberjack}{NumberJack} (Insight, Ireland) 
\item \href{https://www-users.york.ac.uk/peter.nightingale/savilerow//}{EssencePrime/SavilleRow} (UK) 
\item \href{https://cpmpy.readthedocs.io/en/latest/}{CPMpy} (KU Leuven)
\item Allow use of multiple back-end solvers
\item Compile model into variants for each solver
\item A priori solver independent model(CP, MIP, SAT)
\end{itemize}
\end{frame} 

\begin{frame}
\frametitle{Framework Process}
\begin{center}
\begin{tikzpicture}[xscale=2]
\node[shape=rectangle,fill=pantone127-4] (problem) at (2.5,5) {Problem};
\node (human) at (2.5,4) {Human};
\node[shape=rectangle,fill=pantone127-4] (model) at (2.5,3) {Model};
\node (compile) at (2.5,2) {Compile/Reformulate};
\node (solver1) at (1,1) {CP};
\node (solver2) at (2,1) {MIP};
\node (solver3) at (3,1) {SAT};
\node (solver4) at (4,1) {Other};
\node[shape=rectangle,fill=pantone157-8] (solution1) at (1,0) {Solution};
\node[shape=rectangle,fill=pantone157-8] (solution2) at (2,0) {Solution};
\node[shape=rectangle,fill=pantone157-8] (solution3) at (3,0) {Solution};
\node[shape=rectangle,fill=pantone157-8] (solution4) at (4,0) {Solution};
\draw[-] (problem) -- (human);
\draw[->] (human) -- (model);
\draw[-] (model) -- (compile);
\draw[-] (compile) -- (solver1);
\draw[-] (compile) -- (solver2);
\draw[-] (compile) -- (solver3);
\draw[-] (compile) -- (solver4);
\draw[->] (solver1) -- (solution1);
\draw[->] (solver2) -- (solution2);
\draw[->] (solver3) -- (solution3);
\draw[->] (solver4) -- (solution4);
\end{tikzpicture}
\end{center}
\end{frame}

% \begin{frame}
  % \frametitle{Do It Now!}
  % \begin{itemize}
  % \item Download and install Minizinc
  % \item \url{https://www.minizinc.org/}
  % \end{itemize}
% \end{frame}


\begin{frame}
\frametitle{Why use Puzzles as Examples?}
\begin{itemize}
\item Easy to understand the problem
\item Solvable by hand without specialized knowledge
\item Possible to compare automated to manual solving process
\end{itemize}

\begin{quote}
The puzzle, though inanimate, is presented as a solvable problem without lasting negative consequences, a very low-risk low-reward situation. By being a puzzle, the object is attempting to convince the user that it must be completed.
\end{quote}
{\tiny Source: \href{https://www.everydayrhetoric.com/2020/03/the-rhetoric-of-puzzles/}{Every Day Rhetoric}}
\end{frame}


\mode<all>{
\part{Basic Constraint Propagation}
}
\begin{frame}
\partpage
\end{frame}
\mode<all>{
\input{../sendmore/body}
}

\mode<all>{
\part{Global Constraints}
}
\begin{frame}
\partpage
\end{frame}
\mode<all>{
\input{../sudoku/body}
}

\clearpage
\mode<all>{
\part{Customizing Search}
}
\begin{frame}
\partpage
\end{frame}
\mode<all>{
\input{../nqueen/body}
}


%% \clearpage
%% \mode<all>{
%%   \part{Problem Decomposition}
%% }
%% \begin{frame}
%%   \partpage
%% \end{frame}
%% \mode<all>{
%%   \input{../party/body}
%%   }

%% \clearpage
%% \mode<all>{
%%   \part{More Global Constraints}
%% }
%% \begin{frame}
%%   \partpage
%% \end{frame}
%% \mode<all>{
%%   \input{../car/body}
%%   }

%% \clearpage
%% \mode<all>{
%%   \part{Methodology}
%% }
%% \begin{frame}
%%   \partpage
%% \end{frame}
%% \mode<all>{
%%   \input{../methodology/body}
%%   }

\clearpage
\mode<all>{
\part{What is missing?}
}
\begin{frame}
\partpage
\end{frame}

\begin{frame}
\frametitle{Many Specialized Topics}
\begin{itemize}
\item How to design efficient core engine
\item Hybrids with LP/MIP tools
\item Hybrids with SAT
\item Symmetry breaking
\item Use of MDD/BDD to encode sets of solutions
\item High level modelling tools
\item Debugging/visualization
\end{itemize}
\end{frame}

\begin{frame}
\frametitle{Reformulation}
\begin{itemize}
\item Just because the user has modelled it this way, it doesn't mean we have to solve it that way
\begin{itemize}
\item Replace some constraint(s) by other, equivalent constraints
\item Because we don't have that constraint in our system
\item For performance
\end{itemize}
\end{itemize}
\end{frame}

\begin{frame}
\frametitle{Learning}
\begin{itemize}
\item While solving the problem we can learn how to strengthen the model/search 
\begin{itemize}
\item Understand which constraints/method contribute to propagation and change schedule
\item Learn no-good constraints by explaining failure
\item Adapt search strategy based on search experience
\end{itemize}
\end{itemize}
\end{frame}

%% \begin{frame}
%% \frametitle{Refined Process}
%% \begin{center}
%% \begin{tikzpicture}
%% \node[shape=rectangle,fill=pantone127-4] (problem) at (3,7) {Problem};
%% \node (human) at (3,6) {Human};
%% \node[shape=rectangle,fill=pantone127-4] (model) at (3,5) {User Model};
%% \node (reformulation) at (3,4) {Reformulation};
%% \node[shape=rectangle,fill=pantone127-4] (implementation) at (3,3) {Implementation Model};
%% \node (solver) at (3,2) {Constraint Solver/Search};
%% \node[shape=rectangle,fill=pantone157-8] (solution) at (3,1) {Solution};
%% \draw[-] (problem) -- (human);
%% \draw[->] (human) -- (model);
%% \draw[-] (model) -- (reformulation);
%% \draw[->] (reformulation) -- (implementation);
%% \draw[-] (implementation) -- (solver);
%% \draw[->] (solver) -- (0,2) -- (0,3) -- (implementation);
%% \draw[->] (solver) -- (solution);
%% \end{tikzpicture}
%% \end{center}
%% \end{frame}

%\begin{frame}
%\frametitle{What is CP actually used for?}
%\url{http://hsimonis.wordpress.com}
%\includegraphics[width=10cm]{../introduction/images/blog}
%\end{frame}

%% \begin{frame}
%% \frametitle{Exercises}
%% \begin{itemize}
%% \item Install Constraint Solver of your Choice
%% \item MiniZinc (MiningZinc) will be used in later sessions
%% \item Run Sendmoremoney, Sudoku, Queens examples
%% \item Can you choose which propagator to use for alldifferent?
%% \item Can you use different search strategies?
%% \end{itemize}
%% \end{frame}

\begin{frame}
\frametitle{More Learning Resources}
\begin{itemize}
\item Survey of Methods, Resources, and Formats for Teaching Constraint Programming
\begin{itemize}
\item by Tejas Santanam, Helmut Simonis
\item \url{https://doi.org/10.48550/arXiv.2403.12717}
\item Based on survey of community for WTCP 2023
\item \url{https://hsimonis.github.io/WTCP2023/}
\end{itemize}
\end{itemize}
\end{frame}


}


%% \clearpage
%% \mode<all>{
%%   \part{Problem Decomposition}
%% }
%% \begin{frame}
%%   \partpage
%% \end{frame}
%% \mode<all>{
%%   \begin{frame}
\frametitle{Objectives}
\begin{itemize}
\item Overview of Core Constraint Programming
\item Three Main Concepts
\begin{itemize}
\item Constraint Propagation
\item Global Constraints
\item Customizing Search
\end{itemize}
%\item Get Some Experience with MiniZinc
\item Topics will be treated in more detail in later parts of the school
\item Based on Examples, not Formal Description
\end{itemize}
\end{frame}

\begin{frame}
\frametitle{Outline}
\begin{itemize}
\item Why Constraint Programming?
\item Constraint Propagation
\item Global Constraints
\item Customizing Search
%\item Choices in Modelling
%\item Decomposition
%\item More Global Constraints
%\item Methodology
%\item What is missing?
\end{itemize}
\end{frame}

% \begin{frame}
  % \frametitle{Using MiniZinc IDE}
  % \begin{itemize}
  % \item Developed in the Australian NICTA project
  % \item Maintained by Monash University
  % \item Modelling tool with multiple back-end solvers
  % \item Available from \url{https://www.minizinc.org/}
  % \end{itemize}
% \end{frame}


% \begin{frame}
% \frametitle{Examples in ECLiPSe}
% \begin{itemize}
% \item Open sourced constraint programming language
% \item Development goes back to 1985
% \item ECRC, ICL, IC-Parc, PTL, Cisco
% \item \url{https://eclipseclp.org/}
% \item Specialities
% \begin{itemize}
% \item Develop new solvers for specific domains
% \item Integration with MIP
% \end{itemize}
% \item Not included in bundled MiniZinc IDE
% \item Specialized visualization tools used here
  % \begin{itemize}
  % \item CP-Viz, Simonis et al. 2010
    % \end{itemize}
% \end{itemize}
% \end{frame}

\begin{frame}
\frametitle{Tutorial Based on ECLiPSe ELearning Course}
\begin{itemize}
\item Self-study course in constraint programming
\item Supported by Cisco Systems and Silicon Valley Community Foundation
\item Multi-media format, video lectures, slides, handout etc
\item {\small \url{https://eclipseclp.org/ELearning/index.html}}
\end{itemize}
\end{frame}

\begin{frame}
\frametitle{Also Part of CRT-AI Constraint Week}
\begin{itemize}
\item Annual one week course on CP and Optimization in Ireland
\item Part of national training program for PhD students in AI
\item \url{https://www.crt-ai.ie/}
\end{itemize}
\end{frame}



\begin{frame}
\frametitle{Constraint Programming - in a nutshell}
\begin{itemize}
\item Declarative description of problems with
\begin{itemize}
\item {\em Variables} which range over (finite) sets of values
\item {\em Constraints} over subsets of variables which restrict possible value combinations
\item A {\em solution} is a value assignment which satisfies all constraints
\end{itemize}

\item Constraint propagation/reasoning
\begin{itemize}
\item Removing inconsistent values for variables
\item Detect failure if constraint can not be satisfied
\item Interaction of constraints via shared variables
\item Incomplete
\end{itemize}

\item Search
\begin{itemize}
\item User controlled assignment of values to variables
\item Each step triggers constraint propagation 
\end{itemize}
\item Different domains require/allow different methods
\end{itemize}
\end{frame}

\begin{frame}
  \frametitle{Constraint Programming is Different}
  \begin{itemize}
  \item Declarative Programming
    \begin{itemize}
    \item Concentrate on what you want
      \item Not how to get there
      \item Program != Algorithm
      \item Program = Model
    \end{itemize}
    \item Applied to Combinatorial Problems
      \begin{itemize}
        \item No complete polynomial algorithms known (exist?)
        \item CP less ad-hoc than heuristics
        \item Models can evolve
  \end{itemize}
  \end{itemize}
  \end{frame}
    
\begin{frame}
  \frametitle{A Subtractive Process}
  \begin{textblock}{4}(8,-3)
    \includegraphics[width=4cm]{../introduction/images/stages}
  \end{textblock}
  \vfill
  \begin{quote}
    ``Oh, bosh, as Mr. Ruskin says. Sculpture, per se, is the simplest thing in the world. All you have to do is to take a big chunk of marble and a hammer and chisel, make up your mind what you are about to create and chip off all the marble you don't want.''-Paris Gaulois.
  \end{quote}
  
  {\tiny Source: \url{https://quoteinvestigator.com/2014/06/22/chip-away/}}
\end{frame}

\begin{frame}
\frametitle{Basic Process}
\begin{center}
\begin{tikzpicture}
\node[shape=rectangle,fill=pantone127-4] (problem) at (2,5) {Problem};
\node (human) at (2,4) {Human};
\node[shape=rectangle,fill=pantone127-4] (model) at (2,3) {Model};
\node (solver) at (2,2) {Constraint Solver/Search};
\node[shape=rectangle,fill=pantone157-8] (solution) at (2,1) {Solution};
\draw[-] (problem) -- (human);
\draw[->] (human) -- (model);
\draw[-] (model) -- (solver);
\draw[->] (solver) -- (solution);
\end{tikzpicture}
\end{center}
\end{frame}

\begin{frame}
\frametitle{More Realistic}
\begin{center}
\begin{tikzpicture}[xscale=4.0]
\node[shape=rectangle,fill=pantone127-4] (problem) at (2,5) {Problem};
\node (human) at (2,4) {Human};
\node[shape=rectangle,fill=pantone127-4] (model) at (2,3) {Model};
\node (solver) at (2,2) {Constraint Solver/Search};
\node[shape=rectangle,fill=pantone24-8] (hangs) at (1,1) {Hangs};
\node[shape=rectangle,fill=pantone157-8] (solution) at (2,1) {Solution};
\node[shape=rectangle,fill=pantone24-8] (wrong) at (3,1) {Wrong Solution};
\draw[-] (problem) -- (human);
\draw[->] (human) -- (model);
\draw[-] (model) -- (solver);
\draw[->] (solver) -- (hangs);
\draw[->] (solver) -- (solution);
\draw[->] (solver) -- (wrong);
\draw[->] (hangs) -- (1,4) -- (human);
\draw[->] (wrong) -- (3,4) -- (human);
\end{tikzpicture}
\end{center}
\end{frame}

\begin{frame}
\frametitle{Dual Role of Model}
\begin{itemize}
\item Allows Human to Express Problem
\begin{itemize}
\item Close to Problem Domain
\item Constraints as Abstractions
\end{itemize}
\item Allows Solver to Execute
\begin{itemize}
\item Variables as Communication Mechanism
\item Constraints as Algorithms
\end{itemize}
\end{itemize}
\end{frame}

\begin{frame}
\frametitle{Modelling Frameworks}
\begin{itemize}
\item \href{https://www.minizinc.org/}{MiniZinc} (NICTA, Monash University, Australia)
\item \href{https://github.com/eomahony/Numberjack}{NumberJack} (Insight, Ireland) 
\item \href{https://www-users.york.ac.uk/peter.nightingale/savilerow//}{EssencePrime/SavilleRow} (UK) 
\item \href{https://cpmpy.readthedocs.io/en/latest/}{CPMpy} (KU Leuven)
\item Allow use of multiple back-end solvers
\item Compile model into variants for each solver
\item A priori solver independent model(CP, MIP, SAT)
\end{itemize}
\end{frame} 

\begin{frame}
\frametitle{Framework Process}
\begin{center}
\begin{tikzpicture}[xscale=2]
\node[shape=rectangle,fill=pantone127-4] (problem) at (2.5,5) {Problem};
\node (human) at (2.5,4) {Human};
\node[shape=rectangle,fill=pantone127-4] (model) at (2.5,3) {Model};
\node (compile) at (2.5,2) {Compile/Reformulate};
\node (solver1) at (1,1) {CP};
\node (solver2) at (2,1) {MIP};
\node (solver3) at (3,1) {SAT};
\node (solver4) at (4,1) {Other};
\node[shape=rectangle,fill=pantone157-8] (solution1) at (1,0) {Solution};
\node[shape=rectangle,fill=pantone157-8] (solution2) at (2,0) {Solution};
\node[shape=rectangle,fill=pantone157-8] (solution3) at (3,0) {Solution};
\node[shape=rectangle,fill=pantone157-8] (solution4) at (4,0) {Solution};
\draw[-] (problem) -- (human);
\draw[->] (human) -- (model);
\draw[-] (model) -- (compile);
\draw[-] (compile) -- (solver1);
\draw[-] (compile) -- (solver2);
\draw[-] (compile) -- (solver3);
\draw[-] (compile) -- (solver4);
\draw[->] (solver1) -- (solution1);
\draw[->] (solver2) -- (solution2);
\draw[->] (solver3) -- (solution3);
\draw[->] (solver4) -- (solution4);
\end{tikzpicture}
\end{center}
\end{frame}

% \begin{frame}
  % \frametitle{Do It Now!}
  % \begin{itemize}
  % \item Download and install Minizinc
  % \item \url{https://www.minizinc.org/}
  % \end{itemize}
% \end{frame}


\begin{frame}
\frametitle{Why use Puzzles as Examples?}
\begin{itemize}
\item Easy to understand the problem
\item Solvable by hand without specialized knowledge
\item Possible to compare automated to manual solving process
\end{itemize}

\begin{quote}
The puzzle, though inanimate, is presented as a solvable problem without lasting negative consequences, a very low-risk low-reward situation. By being a puzzle, the object is attempting to convince the user that it must be completed.
\end{quote}
{\tiny Source: \href{https://www.everydayrhetoric.com/2020/03/the-rhetoric-of-puzzles/}{Every Day Rhetoric}}
\end{frame}


\mode<all>{
\part{Basic Constraint Propagation}
}
\begin{frame}
\partpage
\end{frame}
\mode<all>{
\input{../sendmore/body}
}

\mode<all>{
\part{Global Constraints}
}
\begin{frame}
\partpage
\end{frame}
\mode<all>{
\input{../sudoku/body}
}

\clearpage
\mode<all>{
\part{Customizing Search}
}
\begin{frame}
\partpage
\end{frame}
\mode<all>{
\input{../nqueen/body}
}


%% \clearpage
%% \mode<all>{
%%   \part{Problem Decomposition}
%% }
%% \begin{frame}
%%   \partpage
%% \end{frame}
%% \mode<all>{
%%   \input{../party/body}
%%   }

%% \clearpage
%% \mode<all>{
%%   \part{More Global Constraints}
%% }
%% \begin{frame}
%%   \partpage
%% \end{frame}
%% \mode<all>{
%%   \input{../car/body}
%%   }

%% \clearpage
%% \mode<all>{
%%   \part{Methodology}
%% }
%% \begin{frame}
%%   \partpage
%% \end{frame}
%% \mode<all>{
%%   \input{../methodology/body}
%%   }

\clearpage
\mode<all>{
\part{What is missing?}
}
\begin{frame}
\partpage
\end{frame}

\begin{frame}
\frametitle{Many Specialized Topics}
\begin{itemize}
\item How to design efficient core engine
\item Hybrids with LP/MIP tools
\item Hybrids with SAT
\item Symmetry breaking
\item Use of MDD/BDD to encode sets of solutions
\item High level modelling tools
\item Debugging/visualization
\end{itemize}
\end{frame}

\begin{frame}
\frametitle{Reformulation}
\begin{itemize}
\item Just because the user has modelled it this way, it doesn't mean we have to solve it that way
\begin{itemize}
\item Replace some constraint(s) by other, equivalent constraints
\item Because we don't have that constraint in our system
\item For performance
\end{itemize}
\end{itemize}
\end{frame}

\begin{frame}
\frametitle{Learning}
\begin{itemize}
\item While solving the problem we can learn how to strengthen the model/search 
\begin{itemize}
\item Understand which constraints/method contribute to propagation and change schedule
\item Learn no-good constraints by explaining failure
\item Adapt search strategy based on search experience
\end{itemize}
\end{itemize}
\end{frame}

%% \begin{frame}
%% \frametitle{Refined Process}
%% \begin{center}
%% \begin{tikzpicture}
%% \node[shape=rectangle,fill=pantone127-4] (problem) at (3,7) {Problem};
%% \node (human) at (3,6) {Human};
%% \node[shape=rectangle,fill=pantone127-4] (model) at (3,5) {User Model};
%% \node (reformulation) at (3,4) {Reformulation};
%% \node[shape=rectangle,fill=pantone127-4] (implementation) at (3,3) {Implementation Model};
%% \node (solver) at (3,2) {Constraint Solver/Search};
%% \node[shape=rectangle,fill=pantone157-8] (solution) at (3,1) {Solution};
%% \draw[-] (problem) -- (human);
%% \draw[->] (human) -- (model);
%% \draw[-] (model) -- (reformulation);
%% \draw[->] (reformulation) -- (implementation);
%% \draw[-] (implementation) -- (solver);
%% \draw[->] (solver) -- (0,2) -- (0,3) -- (implementation);
%% \draw[->] (solver) -- (solution);
%% \end{tikzpicture}
%% \end{center}
%% \end{frame}

%\begin{frame}
%\frametitle{What is CP actually used for?}
%\url{http://hsimonis.wordpress.com}
%\includegraphics[width=10cm]{../introduction/images/blog}
%\end{frame}

%% \begin{frame}
%% \frametitle{Exercises}
%% \begin{itemize}
%% \item Install Constraint Solver of your Choice
%% \item MiniZinc (MiningZinc) will be used in later sessions
%% \item Run Sendmoremoney, Sudoku, Queens examples
%% \item Can you choose which propagator to use for alldifferent?
%% \item Can you use different search strategies?
%% \end{itemize}
%% \end{frame}

\begin{frame}
\frametitle{More Learning Resources}
\begin{itemize}
\item Survey of Methods, Resources, and Formats for Teaching Constraint Programming
\begin{itemize}
\item by Tejas Santanam, Helmut Simonis
\item \url{https://doi.org/10.48550/arXiv.2403.12717}
\item Based on survey of community for WTCP 2023
\item \url{https://hsimonis.github.io/WTCP2023/}
\end{itemize}
\end{itemize}
\end{frame}


%%   }

%% \clearpage
%% \mode<all>{
%%   \part{More Global Constraints}
%% }
%% \begin{frame}
%%   \partpage
%% \end{frame}
%% \mode<all>{
%%   \begin{frame}
\frametitle{Objectives}
\begin{itemize}
\item Overview of Core Constraint Programming
\item Three Main Concepts
\begin{itemize}
\item Constraint Propagation
\item Global Constraints
\item Customizing Search
\end{itemize}
%\item Get Some Experience with MiniZinc
\item Topics will be treated in more detail in later parts of the school
\item Based on Examples, not Formal Description
\end{itemize}
\end{frame}

\begin{frame}
\frametitle{Outline}
\begin{itemize}
\item Why Constraint Programming?
\item Constraint Propagation
\item Global Constraints
\item Customizing Search
%\item Choices in Modelling
%\item Decomposition
%\item More Global Constraints
%\item Methodology
%\item What is missing?
\end{itemize}
\end{frame}

% \begin{frame}
  % \frametitle{Using MiniZinc IDE}
  % \begin{itemize}
  % \item Developed in the Australian NICTA project
  % \item Maintained by Monash University
  % \item Modelling tool with multiple back-end solvers
  % \item Available from \url{https://www.minizinc.org/}
  % \end{itemize}
% \end{frame}


% \begin{frame}
% \frametitle{Examples in ECLiPSe}
% \begin{itemize}
% \item Open sourced constraint programming language
% \item Development goes back to 1985
% \item ECRC, ICL, IC-Parc, PTL, Cisco
% \item \url{https://eclipseclp.org/}
% \item Specialities
% \begin{itemize}
% \item Develop new solvers for specific domains
% \item Integration with MIP
% \end{itemize}
% \item Not included in bundled MiniZinc IDE
% \item Specialized visualization tools used here
  % \begin{itemize}
  % \item CP-Viz, Simonis et al. 2010
    % \end{itemize}
% \end{itemize}
% \end{frame}

\begin{frame}
\frametitle{Tutorial Based on ECLiPSe ELearning Course}
\begin{itemize}
\item Self-study course in constraint programming
\item Supported by Cisco Systems and Silicon Valley Community Foundation
\item Multi-media format, video lectures, slides, handout etc
\item {\small \url{https://eclipseclp.org/ELearning/index.html}}
\end{itemize}
\end{frame}

\begin{frame}
\frametitle{Also Part of CRT-AI Constraint Week}
\begin{itemize}
\item Annual one week course on CP and Optimization in Ireland
\item Part of national training program for PhD students in AI
\item \url{https://www.crt-ai.ie/}
\end{itemize}
\end{frame}



\begin{frame}
\frametitle{Constraint Programming - in a nutshell}
\begin{itemize}
\item Declarative description of problems with
\begin{itemize}
\item {\em Variables} which range over (finite) sets of values
\item {\em Constraints} over subsets of variables which restrict possible value combinations
\item A {\em solution} is a value assignment which satisfies all constraints
\end{itemize}

\item Constraint propagation/reasoning
\begin{itemize}
\item Removing inconsistent values for variables
\item Detect failure if constraint can not be satisfied
\item Interaction of constraints via shared variables
\item Incomplete
\end{itemize}

\item Search
\begin{itemize}
\item User controlled assignment of values to variables
\item Each step triggers constraint propagation 
\end{itemize}
\item Different domains require/allow different methods
\end{itemize}
\end{frame}

\begin{frame}
  \frametitle{Constraint Programming is Different}
  \begin{itemize}
  \item Declarative Programming
    \begin{itemize}
    \item Concentrate on what you want
      \item Not how to get there
      \item Program != Algorithm
      \item Program = Model
    \end{itemize}
    \item Applied to Combinatorial Problems
      \begin{itemize}
        \item No complete polynomial algorithms known (exist?)
        \item CP less ad-hoc than heuristics
        \item Models can evolve
  \end{itemize}
  \end{itemize}
  \end{frame}
    
\begin{frame}
  \frametitle{A Subtractive Process}
  \begin{textblock}{4}(8,-3)
    \includegraphics[width=4cm]{../introduction/images/stages}
  \end{textblock}
  \vfill
  \begin{quote}
    ``Oh, bosh, as Mr. Ruskin says. Sculpture, per se, is the simplest thing in the world. All you have to do is to take a big chunk of marble and a hammer and chisel, make up your mind what you are about to create and chip off all the marble you don't want.''-Paris Gaulois.
  \end{quote}
  
  {\tiny Source: \url{https://quoteinvestigator.com/2014/06/22/chip-away/}}
\end{frame}

\begin{frame}
\frametitle{Basic Process}
\begin{center}
\begin{tikzpicture}
\node[shape=rectangle,fill=pantone127-4] (problem) at (2,5) {Problem};
\node (human) at (2,4) {Human};
\node[shape=rectangle,fill=pantone127-4] (model) at (2,3) {Model};
\node (solver) at (2,2) {Constraint Solver/Search};
\node[shape=rectangle,fill=pantone157-8] (solution) at (2,1) {Solution};
\draw[-] (problem) -- (human);
\draw[->] (human) -- (model);
\draw[-] (model) -- (solver);
\draw[->] (solver) -- (solution);
\end{tikzpicture}
\end{center}
\end{frame}

\begin{frame}
\frametitle{More Realistic}
\begin{center}
\begin{tikzpicture}[xscale=4.0]
\node[shape=rectangle,fill=pantone127-4] (problem) at (2,5) {Problem};
\node (human) at (2,4) {Human};
\node[shape=rectangle,fill=pantone127-4] (model) at (2,3) {Model};
\node (solver) at (2,2) {Constraint Solver/Search};
\node[shape=rectangle,fill=pantone24-8] (hangs) at (1,1) {Hangs};
\node[shape=rectangle,fill=pantone157-8] (solution) at (2,1) {Solution};
\node[shape=rectangle,fill=pantone24-8] (wrong) at (3,1) {Wrong Solution};
\draw[-] (problem) -- (human);
\draw[->] (human) -- (model);
\draw[-] (model) -- (solver);
\draw[->] (solver) -- (hangs);
\draw[->] (solver) -- (solution);
\draw[->] (solver) -- (wrong);
\draw[->] (hangs) -- (1,4) -- (human);
\draw[->] (wrong) -- (3,4) -- (human);
\end{tikzpicture}
\end{center}
\end{frame}

\begin{frame}
\frametitle{Dual Role of Model}
\begin{itemize}
\item Allows Human to Express Problem
\begin{itemize}
\item Close to Problem Domain
\item Constraints as Abstractions
\end{itemize}
\item Allows Solver to Execute
\begin{itemize}
\item Variables as Communication Mechanism
\item Constraints as Algorithms
\end{itemize}
\end{itemize}
\end{frame}

\begin{frame}
\frametitle{Modelling Frameworks}
\begin{itemize}
\item \href{https://www.minizinc.org/}{MiniZinc} (NICTA, Monash University, Australia)
\item \href{https://github.com/eomahony/Numberjack}{NumberJack} (Insight, Ireland) 
\item \href{https://www-users.york.ac.uk/peter.nightingale/savilerow//}{EssencePrime/SavilleRow} (UK) 
\item \href{https://cpmpy.readthedocs.io/en/latest/}{CPMpy} (KU Leuven)
\item Allow use of multiple back-end solvers
\item Compile model into variants for each solver
\item A priori solver independent model(CP, MIP, SAT)
\end{itemize}
\end{frame} 

\begin{frame}
\frametitle{Framework Process}
\begin{center}
\begin{tikzpicture}[xscale=2]
\node[shape=rectangle,fill=pantone127-4] (problem) at (2.5,5) {Problem};
\node (human) at (2.5,4) {Human};
\node[shape=rectangle,fill=pantone127-4] (model) at (2.5,3) {Model};
\node (compile) at (2.5,2) {Compile/Reformulate};
\node (solver1) at (1,1) {CP};
\node (solver2) at (2,1) {MIP};
\node (solver3) at (3,1) {SAT};
\node (solver4) at (4,1) {Other};
\node[shape=rectangle,fill=pantone157-8] (solution1) at (1,0) {Solution};
\node[shape=rectangle,fill=pantone157-8] (solution2) at (2,0) {Solution};
\node[shape=rectangle,fill=pantone157-8] (solution3) at (3,0) {Solution};
\node[shape=rectangle,fill=pantone157-8] (solution4) at (4,0) {Solution};
\draw[-] (problem) -- (human);
\draw[->] (human) -- (model);
\draw[-] (model) -- (compile);
\draw[-] (compile) -- (solver1);
\draw[-] (compile) -- (solver2);
\draw[-] (compile) -- (solver3);
\draw[-] (compile) -- (solver4);
\draw[->] (solver1) -- (solution1);
\draw[->] (solver2) -- (solution2);
\draw[->] (solver3) -- (solution3);
\draw[->] (solver4) -- (solution4);
\end{tikzpicture}
\end{center}
\end{frame}

% \begin{frame}
  % \frametitle{Do It Now!}
  % \begin{itemize}
  % \item Download and install Minizinc
  % \item \url{https://www.minizinc.org/}
  % \end{itemize}
% \end{frame}


\begin{frame}
\frametitle{Why use Puzzles as Examples?}
\begin{itemize}
\item Easy to understand the problem
\item Solvable by hand without specialized knowledge
\item Possible to compare automated to manual solving process
\end{itemize}

\begin{quote}
The puzzle, though inanimate, is presented as a solvable problem without lasting negative consequences, a very low-risk low-reward situation. By being a puzzle, the object is attempting to convince the user that it must be completed.
\end{quote}
{\tiny Source: \href{https://www.everydayrhetoric.com/2020/03/the-rhetoric-of-puzzles/}{Every Day Rhetoric}}
\end{frame}


\mode<all>{
\part{Basic Constraint Propagation}
}
\begin{frame}
\partpage
\end{frame}
\mode<all>{
\input{../sendmore/body}
}

\mode<all>{
\part{Global Constraints}
}
\begin{frame}
\partpage
\end{frame}
\mode<all>{
\input{../sudoku/body}
}

\clearpage
\mode<all>{
\part{Customizing Search}
}
\begin{frame}
\partpage
\end{frame}
\mode<all>{
\input{../nqueen/body}
}


%% \clearpage
%% \mode<all>{
%%   \part{Problem Decomposition}
%% }
%% \begin{frame}
%%   \partpage
%% \end{frame}
%% \mode<all>{
%%   \input{../party/body}
%%   }

%% \clearpage
%% \mode<all>{
%%   \part{More Global Constraints}
%% }
%% \begin{frame}
%%   \partpage
%% \end{frame}
%% \mode<all>{
%%   \input{../car/body}
%%   }

%% \clearpage
%% \mode<all>{
%%   \part{Methodology}
%% }
%% \begin{frame}
%%   \partpage
%% \end{frame}
%% \mode<all>{
%%   \input{../methodology/body}
%%   }

\clearpage
\mode<all>{
\part{What is missing?}
}
\begin{frame}
\partpage
\end{frame}

\begin{frame}
\frametitle{Many Specialized Topics}
\begin{itemize}
\item How to design efficient core engine
\item Hybrids with LP/MIP tools
\item Hybrids with SAT
\item Symmetry breaking
\item Use of MDD/BDD to encode sets of solutions
\item High level modelling tools
\item Debugging/visualization
\end{itemize}
\end{frame}

\begin{frame}
\frametitle{Reformulation}
\begin{itemize}
\item Just because the user has modelled it this way, it doesn't mean we have to solve it that way
\begin{itemize}
\item Replace some constraint(s) by other, equivalent constraints
\item Because we don't have that constraint in our system
\item For performance
\end{itemize}
\end{itemize}
\end{frame}

\begin{frame}
\frametitle{Learning}
\begin{itemize}
\item While solving the problem we can learn how to strengthen the model/search 
\begin{itemize}
\item Understand which constraints/method contribute to propagation and change schedule
\item Learn no-good constraints by explaining failure
\item Adapt search strategy based on search experience
\end{itemize}
\end{itemize}
\end{frame}

%% \begin{frame}
%% \frametitle{Refined Process}
%% \begin{center}
%% \begin{tikzpicture}
%% \node[shape=rectangle,fill=pantone127-4] (problem) at (3,7) {Problem};
%% \node (human) at (3,6) {Human};
%% \node[shape=rectangle,fill=pantone127-4] (model) at (3,5) {User Model};
%% \node (reformulation) at (3,4) {Reformulation};
%% \node[shape=rectangle,fill=pantone127-4] (implementation) at (3,3) {Implementation Model};
%% \node (solver) at (3,2) {Constraint Solver/Search};
%% \node[shape=rectangle,fill=pantone157-8] (solution) at (3,1) {Solution};
%% \draw[-] (problem) -- (human);
%% \draw[->] (human) -- (model);
%% \draw[-] (model) -- (reformulation);
%% \draw[->] (reformulation) -- (implementation);
%% \draw[-] (implementation) -- (solver);
%% \draw[->] (solver) -- (0,2) -- (0,3) -- (implementation);
%% \draw[->] (solver) -- (solution);
%% \end{tikzpicture}
%% \end{center}
%% \end{frame}

%\begin{frame}
%\frametitle{What is CP actually used for?}
%\url{http://hsimonis.wordpress.com}
%\includegraphics[width=10cm]{../introduction/images/blog}
%\end{frame}

%% \begin{frame}
%% \frametitle{Exercises}
%% \begin{itemize}
%% \item Install Constraint Solver of your Choice
%% \item MiniZinc (MiningZinc) will be used in later sessions
%% \item Run Sendmoremoney, Sudoku, Queens examples
%% \item Can you choose which propagator to use for alldifferent?
%% \item Can you use different search strategies?
%% \end{itemize}
%% \end{frame}

\begin{frame}
\frametitle{More Learning Resources}
\begin{itemize}
\item Survey of Methods, Resources, and Formats for Teaching Constraint Programming
\begin{itemize}
\item by Tejas Santanam, Helmut Simonis
\item \url{https://doi.org/10.48550/arXiv.2403.12717}
\item Based on survey of community for WTCP 2023
\item \url{https://hsimonis.github.io/WTCP2023/}
\end{itemize}
\end{itemize}
\end{frame}


%%   }

%% \clearpage
%% \mode<all>{
%%   \part{Methodology}
%% }
%% \begin{frame}
%%   \partpage
%% \end{frame}
%% \mode<all>{
%%   \begin{frame}
\frametitle{Objectives}
\begin{itemize}
\item Overview of Core Constraint Programming
\item Three Main Concepts
\begin{itemize}
\item Constraint Propagation
\item Global Constraints
\item Customizing Search
\end{itemize}
%\item Get Some Experience with MiniZinc
\item Topics will be treated in more detail in later parts of the school
\item Based on Examples, not Formal Description
\end{itemize}
\end{frame}

\begin{frame}
\frametitle{Outline}
\begin{itemize}
\item Why Constraint Programming?
\item Constraint Propagation
\item Global Constraints
\item Customizing Search
%\item Choices in Modelling
%\item Decomposition
%\item More Global Constraints
%\item Methodology
%\item What is missing?
\end{itemize}
\end{frame}

% \begin{frame}
  % \frametitle{Using MiniZinc IDE}
  % \begin{itemize}
  % \item Developed in the Australian NICTA project
  % \item Maintained by Monash University
  % \item Modelling tool with multiple back-end solvers
  % \item Available from \url{https://www.minizinc.org/}
  % \end{itemize}
% \end{frame}


% \begin{frame}
% \frametitle{Examples in ECLiPSe}
% \begin{itemize}
% \item Open sourced constraint programming language
% \item Development goes back to 1985
% \item ECRC, ICL, IC-Parc, PTL, Cisco
% \item \url{https://eclipseclp.org/}
% \item Specialities
% \begin{itemize}
% \item Develop new solvers for specific domains
% \item Integration with MIP
% \end{itemize}
% \item Not included in bundled MiniZinc IDE
% \item Specialized visualization tools used here
  % \begin{itemize}
  % \item CP-Viz, Simonis et al. 2010
    % \end{itemize}
% \end{itemize}
% \end{frame}

\begin{frame}
\frametitle{Tutorial Based on ECLiPSe ELearning Course}
\begin{itemize}
\item Self-study course in constraint programming
\item Supported by Cisco Systems and Silicon Valley Community Foundation
\item Multi-media format, video lectures, slides, handout etc
\item {\small \url{https://eclipseclp.org/ELearning/index.html}}
\end{itemize}
\end{frame}

\begin{frame}
\frametitle{Also Part of CRT-AI Constraint Week}
\begin{itemize}
\item Annual one week course on CP and Optimization in Ireland
\item Part of national training program for PhD students in AI
\item \url{https://www.crt-ai.ie/}
\end{itemize}
\end{frame}



\begin{frame}
\frametitle{Constraint Programming - in a nutshell}
\begin{itemize}
\item Declarative description of problems with
\begin{itemize}
\item {\em Variables} which range over (finite) sets of values
\item {\em Constraints} over subsets of variables which restrict possible value combinations
\item A {\em solution} is a value assignment which satisfies all constraints
\end{itemize}

\item Constraint propagation/reasoning
\begin{itemize}
\item Removing inconsistent values for variables
\item Detect failure if constraint can not be satisfied
\item Interaction of constraints via shared variables
\item Incomplete
\end{itemize}

\item Search
\begin{itemize}
\item User controlled assignment of values to variables
\item Each step triggers constraint propagation 
\end{itemize}
\item Different domains require/allow different methods
\end{itemize}
\end{frame}

\begin{frame}
  \frametitle{Constraint Programming is Different}
  \begin{itemize}
  \item Declarative Programming
    \begin{itemize}
    \item Concentrate on what you want
      \item Not how to get there
      \item Program != Algorithm
      \item Program = Model
    \end{itemize}
    \item Applied to Combinatorial Problems
      \begin{itemize}
        \item No complete polynomial algorithms known (exist?)
        \item CP less ad-hoc than heuristics
        \item Models can evolve
  \end{itemize}
  \end{itemize}
  \end{frame}
    
\begin{frame}
  \frametitle{A Subtractive Process}
  \begin{textblock}{4}(8,-3)
    \includegraphics[width=4cm]{../introduction/images/stages}
  \end{textblock}
  \vfill
  \begin{quote}
    ``Oh, bosh, as Mr. Ruskin says. Sculpture, per se, is the simplest thing in the world. All you have to do is to take a big chunk of marble and a hammer and chisel, make up your mind what you are about to create and chip off all the marble you don't want.''-Paris Gaulois.
  \end{quote}
  
  {\tiny Source: \url{https://quoteinvestigator.com/2014/06/22/chip-away/}}
\end{frame}

\begin{frame}
\frametitle{Basic Process}
\begin{center}
\begin{tikzpicture}
\node[shape=rectangle,fill=pantone127-4] (problem) at (2,5) {Problem};
\node (human) at (2,4) {Human};
\node[shape=rectangle,fill=pantone127-4] (model) at (2,3) {Model};
\node (solver) at (2,2) {Constraint Solver/Search};
\node[shape=rectangle,fill=pantone157-8] (solution) at (2,1) {Solution};
\draw[-] (problem) -- (human);
\draw[->] (human) -- (model);
\draw[-] (model) -- (solver);
\draw[->] (solver) -- (solution);
\end{tikzpicture}
\end{center}
\end{frame}

\begin{frame}
\frametitle{More Realistic}
\begin{center}
\begin{tikzpicture}[xscale=4.0]
\node[shape=rectangle,fill=pantone127-4] (problem) at (2,5) {Problem};
\node (human) at (2,4) {Human};
\node[shape=rectangle,fill=pantone127-4] (model) at (2,3) {Model};
\node (solver) at (2,2) {Constraint Solver/Search};
\node[shape=rectangle,fill=pantone24-8] (hangs) at (1,1) {Hangs};
\node[shape=rectangle,fill=pantone157-8] (solution) at (2,1) {Solution};
\node[shape=rectangle,fill=pantone24-8] (wrong) at (3,1) {Wrong Solution};
\draw[-] (problem) -- (human);
\draw[->] (human) -- (model);
\draw[-] (model) -- (solver);
\draw[->] (solver) -- (hangs);
\draw[->] (solver) -- (solution);
\draw[->] (solver) -- (wrong);
\draw[->] (hangs) -- (1,4) -- (human);
\draw[->] (wrong) -- (3,4) -- (human);
\end{tikzpicture}
\end{center}
\end{frame}

\begin{frame}
\frametitle{Dual Role of Model}
\begin{itemize}
\item Allows Human to Express Problem
\begin{itemize}
\item Close to Problem Domain
\item Constraints as Abstractions
\end{itemize}
\item Allows Solver to Execute
\begin{itemize}
\item Variables as Communication Mechanism
\item Constraints as Algorithms
\end{itemize}
\end{itemize}
\end{frame}

\begin{frame}
\frametitle{Modelling Frameworks}
\begin{itemize}
\item \href{https://www.minizinc.org/}{MiniZinc} (NICTA, Monash University, Australia)
\item \href{https://github.com/eomahony/Numberjack}{NumberJack} (Insight, Ireland) 
\item \href{https://www-users.york.ac.uk/peter.nightingale/savilerow//}{EssencePrime/SavilleRow} (UK) 
\item \href{https://cpmpy.readthedocs.io/en/latest/}{CPMpy} (KU Leuven)
\item Allow use of multiple back-end solvers
\item Compile model into variants for each solver
\item A priori solver independent model(CP, MIP, SAT)
\end{itemize}
\end{frame} 

\begin{frame}
\frametitle{Framework Process}
\begin{center}
\begin{tikzpicture}[xscale=2]
\node[shape=rectangle,fill=pantone127-4] (problem) at (2.5,5) {Problem};
\node (human) at (2.5,4) {Human};
\node[shape=rectangle,fill=pantone127-4] (model) at (2.5,3) {Model};
\node (compile) at (2.5,2) {Compile/Reformulate};
\node (solver1) at (1,1) {CP};
\node (solver2) at (2,1) {MIP};
\node (solver3) at (3,1) {SAT};
\node (solver4) at (4,1) {Other};
\node[shape=rectangle,fill=pantone157-8] (solution1) at (1,0) {Solution};
\node[shape=rectangle,fill=pantone157-8] (solution2) at (2,0) {Solution};
\node[shape=rectangle,fill=pantone157-8] (solution3) at (3,0) {Solution};
\node[shape=rectangle,fill=pantone157-8] (solution4) at (4,0) {Solution};
\draw[-] (problem) -- (human);
\draw[->] (human) -- (model);
\draw[-] (model) -- (compile);
\draw[-] (compile) -- (solver1);
\draw[-] (compile) -- (solver2);
\draw[-] (compile) -- (solver3);
\draw[-] (compile) -- (solver4);
\draw[->] (solver1) -- (solution1);
\draw[->] (solver2) -- (solution2);
\draw[->] (solver3) -- (solution3);
\draw[->] (solver4) -- (solution4);
\end{tikzpicture}
\end{center}
\end{frame}

% \begin{frame}
  % \frametitle{Do It Now!}
  % \begin{itemize}
  % \item Download and install Minizinc
  % \item \url{https://www.minizinc.org/}
  % \end{itemize}
% \end{frame}


\begin{frame}
\frametitle{Why use Puzzles as Examples?}
\begin{itemize}
\item Easy to understand the problem
\item Solvable by hand without specialized knowledge
\item Possible to compare automated to manual solving process
\end{itemize}

\begin{quote}
The puzzle, though inanimate, is presented as a solvable problem without lasting negative consequences, a very low-risk low-reward situation. By being a puzzle, the object is attempting to convince the user that it must be completed.
\end{quote}
{\tiny Source: \href{https://www.everydayrhetoric.com/2020/03/the-rhetoric-of-puzzles/}{Every Day Rhetoric}}
\end{frame}


\mode<all>{
\part{Basic Constraint Propagation}
}
\begin{frame}
\partpage
\end{frame}
\mode<all>{
\input{../sendmore/body}
}

\mode<all>{
\part{Global Constraints}
}
\begin{frame}
\partpage
\end{frame}
\mode<all>{
\input{../sudoku/body}
}

\clearpage
\mode<all>{
\part{Customizing Search}
}
\begin{frame}
\partpage
\end{frame}
\mode<all>{
\input{../nqueen/body}
}


%% \clearpage
%% \mode<all>{
%%   \part{Problem Decomposition}
%% }
%% \begin{frame}
%%   \partpage
%% \end{frame}
%% \mode<all>{
%%   \input{../party/body}
%%   }

%% \clearpage
%% \mode<all>{
%%   \part{More Global Constraints}
%% }
%% \begin{frame}
%%   \partpage
%% \end{frame}
%% \mode<all>{
%%   \input{../car/body}
%%   }

%% \clearpage
%% \mode<all>{
%%   \part{Methodology}
%% }
%% \begin{frame}
%%   \partpage
%% \end{frame}
%% \mode<all>{
%%   \input{../methodology/body}
%%   }

\clearpage
\mode<all>{
\part{What is missing?}
}
\begin{frame}
\partpage
\end{frame}

\begin{frame}
\frametitle{Many Specialized Topics}
\begin{itemize}
\item How to design efficient core engine
\item Hybrids with LP/MIP tools
\item Hybrids with SAT
\item Symmetry breaking
\item Use of MDD/BDD to encode sets of solutions
\item High level modelling tools
\item Debugging/visualization
\end{itemize}
\end{frame}

\begin{frame}
\frametitle{Reformulation}
\begin{itemize}
\item Just because the user has modelled it this way, it doesn't mean we have to solve it that way
\begin{itemize}
\item Replace some constraint(s) by other, equivalent constraints
\item Because we don't have that constraint in our system
\item For performance
\end{itemize}
\end{itemize}
\end{frame}

\begin{frame}
\frametitle{Learning}
\begin{itemize}
\item While solving the problem we can learn how to strengthen the model/search 
\begin{itemize}
\item Understand which constraints/method contribute to propagation and change schedule
\item Learn no-good constraints by explaining failure
\item Adapt search strategy based on search experience
\end{itemize}
\end{itemize}
\end{frame}

%% \begin{frame}
%% \frametitle{Refined Process}
%% \begin{center}
%% \begin{tikzpicture}
%% \node[shape=rectangle,fill=pantone127-4] (problem) at (3,7) {Problem};
%% \node (human) at (3,6) {Human};
%% \node[shape=rectangle,fill=pantone127-4] (model) at (3,5) {User Model};
%% \node (reformulation) at (3,4) {Reformulation};
%% \node[shape=rectangle,fill=pantone127-4] (implementation) at (3,3) {Implementation Model};
%% \node (solver) at (3,2) {Constraint Solver/Search};
%% \node[shape=rectangle,fill=pantone157-8] (solution) at (3,1) {Solution};
%% \draw[-] (problem) -- (human);
%% \draw[->] (human) -- (model);
%% \draw[-] (model) -- (reformulation);
%% \draw[->] (reformulation) -- (implementation);
%% \draw[-] (implementation) -- (solver);
%% \draw[->] (solver) -- (0,2) -- (0,3) -- (implementation);
%% \draw[->] (solver) -- (solution);
%% \end{tikzpicture}
%% \end{center}
%% \end{frame}

%\begin{frame}
%\frametitle{What is CP actually used for?}
%\url{http://hsimonis.wordpress.com}
%\includegraphics[width=10cm]{../introduction/images/blog}
%\end{frame}

%% \begin{frame}
%% \frametitle{Exercises}
%% \begin{itemize}
%% \item Install Constraint Solver of your Choice
%% \item MiniZinc (MiningZinc) will be used in later sessions
%% \item Run Sendmoremoney, Sudoku, Queens examples
%% \item Can you choose which propagator to use for alldifferent?
%% \item Can you use different search strategies?
%% \end{itemize}
%% \end{frame}

\begin{frame}
\frametitle{More Learning Resources}
\begin{itemize}
\item Survey of Methods, Resources, and Formats for Teaching Constraint Programming
\begin{itemize}
\item by Tejas Santanam, Helmut Simonis
\item \url{https://doi.org/10.48550/arXiv.2403.12717}
\item Based on survey of community for WTCP 2023
\item \url{https://hsimonis.github.io/WTCP2023/}
\end{itemize}
\end{itemize}
\end{frame}


%%   }

\clearpage
\mode<all>{
\part{What is missing?}
}
\begin{frame}
\partpage
\end{frame}

\begin{frame}
\frametitle{Many Specialized Topics}
\begin{itemize}
\item How to design efficient core engine
\item Hybrids with LP/MIP tools
\item Hybrids with SAT
\item Symmetry breaking
\item Use of MDD/BDD to encode sets of solutions
\item High level modelling tools
\item Debugging/visualization
\end{itemize}
\end{frame}

\begin{frame}
\frametitle{Reformulation}
\begin{itemize}
\item Just because the user has modelled it this way, it doesn't mean we have to solve it that way
\begin{itemize}
\item Replace some constraint(s) by other, equivalent constraints
\item Because we don't have that constraint in our system
\item For performance
\end{itemize}
\end{itemize}
\end{frame}

\begin{frame}
\frametitle{Learning}
\begin{itemize}
\item While solving the problem we can learn how to strengthen the model/search 
\begin{itemize}
\item Understand which constraints/method contribute to propagation and change schedule
\item Learn no-good constraints by explaining failure
\item Adapt search strategy based on search experience
\end{itemize}
\end{itemize}
\end{frame}

%% \begin{frame}
%% \frametitle{Refined Process}
%% \begin{center}
%% \begin{tikzpicture}
%% \node[shape=rectangle,fill=pantone127-4] (problem) at (3,7) {Problem};
%% \node (human) at (3,6) {Human};
%% \node[shape=rectangle,fill=pantone127-4] (model) at (3,5) {User Model};
%% \node (reformulation) at (3,4) {Reformulation};
%% \node[shape=rectangle,fill=pantone127-4] (implementation) at (3,3) {Implementation Model};
%% \node (solver) at (3,2) {Constraint Solver/Search};
%% \node[shape=rectangle,fill=pantone157-8] (solution) at (3,1) {Solution};
%% \draw[-] (problem) -- (human);
%% \draw[->] (human) -- (model);
%% \draw[-] (model) -- (reformulation);
%% \draw[->] (reformulation) -- (implementation);
%% \draw[-] (implementation) -- (solver);
%% \draw[->] (solver) -- (0,2) -- (0,3) -- (implementation);
%% \draw[->] (solver) -- (solution);
%% \end{tikzpicture}
%% \end{center}
%% \end{frame}

%\begin{frame}
%\frametitle{What is CP actually used for?}
%\url{http://hsimonis.wordpress.com}
%\includegraphics[width=10cm]{../introduction/images/blog}
%\end{frame}

%% \begin{frame}
%% \frametitle{Exercises}
%% \begin{itemize}
%% \item Install Constraint Solver of your Choice
%% \item MiniZinc (MiningZinc) will be used in later sessions
%% \item Run Sendmoremoney, Sudoku, Queens examples
%% \item Can you choose which propagator to use for alldifferent?
%% \item Can you use different search strategies?
%% \end{itemize}
%% \end{frame}

\begin{frame}
\frametitle{More Learning Resources}
\begin{itemize}
\item Survey of Methods, Resources, and Formats for Teaching Constraint Programming
\begin{itemize}
\item by Tejas Santanam, Helmut Simonis
\item \url{https://doi.org/10.48550/arXiv.2403.12717}
\item Based on survey of community for WTCP 2023
\item \url{https://hsimonis.github.io/WTCP2023/}
\end{itemize}
\end{itemize}
\end{frame}


}

\clearpage
\mode<all>{
\part{Customizing Search}
}
\begin{frame}
\partpage
\end{frame}
\mode<all>{
\begin{frame}
\frametitle{Objectives}
\begin{itemize}
\item Overview of Core Constraint Programming
\item Three Main Concepts
\begin{itemize}
\item Constraint Propagation
\item Global Constraints
\item Customizing Search
\end{itemize}
%\item Get Some Experience with MiniZinc
\item Topics will be treated in more detail in later parts of the school
\item Based on Examples, not Formal Description
\end{itemize}
\end{frame}

\begin{frame}
\frametitle{Outline}
\begin{itemize}
\item Why Constraint Programming?
\item Constraint Propagation
\item Global Constraints
\item Customizing Search
%\item Choices in Modelling
%\item Decomposition
%\item More Global Constraints
%\item Methodology
%\item What is missing?
\end{itemize}
\end{frame}

% \begin{frame}
  % \frametitle{Using MiniZinc IDE}
  % \begin{itemize}
  % \item Developed in the Australian NICTA project
  % \item Maintained by Monash University
  % \item Modelling tool with multiple back-end solvers
  % \item Available from \url{https://www.minizinc.org/}
  % \end{itemize}
% \end{frame}


% \begin{frame}
% \frametitle{Examples in ECLiPSe}
% \begin{itemize}
% \item Open sourced constraint programming language
% \item Development goes back to 1985
% \item ECRC, ICL, IC-Parc, PTL, Cisco
% \item \url{https://eclipseclp.org/}
% \item Specialities
% \begin{itemize}
% \item Develop new solvers for specific domains
% \item Integration with MIP
% \end{itemize}
% \item Not included in bundled MiniZinc IDE
% \item Specialized visualization tools used here
  % \begin{itemize}
  % \item CP-Viz, Simonis et al. 2010
    % \end{itemize}
% \end{itemize}
% \end{frame}

\begin{frame}
\frametitle{Tutorial Based on ECLiPSe ELearning Course}
\begin{itemize}
\item Self-study course in constraint programming
\item Supported by Cisco Systems and Silicon Valley Community Foundation
\item Multi-media format, video lectures, slides, handout etc
\item {\small \url{https://eclipseclp.org/ELearning/index.html}}
\end{itemize}
\end{frame}

\begin{frame}
\frametitle{Also Part of CRT-AI Constraint Week}
\begin{itemize}
\item Annual one week course on CP and Optimization in Ireland
\item Part of national training program for PhD students in AI
\item \url{https://www.crt-ai.ie/}
\end{itemize}
\end{frame}



\begin{frame}
\frametitle{Constraint Programming - in a nutshell}
\begin{itemize}
\item Declarative description of problems with
\begin{itemize}
\item {\em Variables} which range over (finite) sets of values
\item {\em Constraints} over subsets of variables which restrict possible value combinations
\item A {\em solution} is a value assignment which satisfies all constraints
\end{itemize}

\item Constraint propagation/reasoning
\begin{itemize}
\item Removing inconsistent values for variables
\item Detect failure if constraint can not be satisfied
\item Interaction of constraints via shared variables
\item Incomplete
\end{itemize}

\item Search
\begin{itemize}
\item User controlled assignment of values to variables
\item Each step triggers constraint propagation 
\end{itemize}
\item Different domains require/allow different methods
\end{itemize}
\end{frame}

\begin{frame}
  \frametitle{Constraint Programming is Different}
  \begin{itemize}
  \item Declarative Programming
    \begin{itemize}
    \item Concentrate on what you want
      \item Not how to get there
      \item Program != Algorithm
      \item Program = Model
    \end{itemize}
    \item Applied to Combinatorial Problems
      \begin{itemize}
        \item No complete polynomial algorithms known (exist?)
        \item CP less ad-hoc than heuristics
        \item Models can evolve
  \end{itemize}
  \end{itemize}
  \end{frame}
    
\begin{frame}
  \frametitle{A Subtractive Process}
  \begin{textblock}{4}(8,-3)
    \includegraphics[width=4cm]{../introduction/images/stages}
  \end{textblock}
  \vfill
  \begin{quote}
    ``Oh, bosh, as Mr. Ruskin says. Sculpture, per se, is the simplest thing in the world. All you have to do is to take a big chunk of marble and a hammer and chisel, make up your mind what you are about to create and chip off all the marble you don't want.''-Paris Gaulois.
  \end{quote}
  
  {\tiny Source: \url{https://quoteinvestigator.com/2014/06/22/chip-away/}}
\end{frame}

\begin{frame}
\frametitle{Basic Process}
\begin{center}
\begin{tikzpicture}
\node[shape=rectangle,fill=pantone127-4] (problem) at (2,5) {Problem};
\node (human) at (2,4) {Human};
\node[shape=rectangle,fill=pantone127-4] (model) at (2,3) {Model};
\node (solver) at (2,2) {Constraint Solver/Search};
\node[shape=rectangle,fill=pantone157-8] (solution) at (2,1) {Solution};
\draw[-] (problem) -- (human);
\draw[->] (human) -- (model);
\draw[-] (model) -- (solver);
\draw[->] (solver) -- (solution);
\end{tikzpicture}
\end{center}
\end{frame}

\begin{frame}
\frametitle{More Realistic}
\begin{center}
\begin{tikzpicture}[xscale=4.0]
\node[shape=rectangle,fill=pantone127-4] (problem) at (2,5) {Problem};
\node (human) at (2,4) {Human};
\node[shape=rectangle,fill=pantone127-4] (model) at (2,3) {Model};
\node (solver) at (2,2) {Constraint Solver/Search};
\node[shape=rectangle,fill=pantone24-8] (hangs) at (1,1) {Hangs};
\node[shape=rectangle,fill=pantone157-8] (solution) at (2,1) {Solution};
\node[shape=rectangle,fill=pantone24-8] (wrong) at (3,1) {Wrong Solution};
\draw[-] (problem) -- (human);
\draw[->] (human) -- (model);
\draw[-] (model) -- (solver);
\draw[->] (solver) -- (hangs);
\draw[->] (solver) -- (solution);
\draw[->] (solver) -- (wrong);
\draw[->] (hangs) -- (1,4) -- (human);
\draw[->] (wrong) -- (3,4) -- (human);
\end{tikzpicture}
\end{center}
\end{frame}

\begin{frame}
\frametitle{Dual Role of Model}
\begin{itemize}
\item Allows Human to Express Problem
\begin{itemize}
\item Close to Problem Domain
\item Constraints as Abstractions
\end{itemize}
\item Allows Solver to Execute
\begin{itemize}
\item Variables as Communication Mechanism
\item Constraints as Algorithms
\end{itemize}
\end{itemize}
\end{frame}

\begin{frame}
\frametitle{Modelling Frameworks}
\begin{itemize}
\item \href{https://www.minizinc.org/}{MiniZinc} (NICTA, Monash University, Australia)
\item \href{https://github.com/eomahony/Numberjack}{NumberJack} (Insight, Ireland) 
\item \href{https://www-users.york.ac.uk/peter.nightingale/savilerow//}{EssencePrime/SavilleRow} (UK) 
\item \href{https://cpmpy.readthedocs.io/en/latest/}{CPMpy} (KU Leuven)
\item Allow use of multiple back-end solvers
\item Compile model into variants for each solver
\item A priori solver independent model(CP, MIP, SAT)
\end{itemize}
\end{frame} 

\begin{frame}
\frametitle{Framework Process}
\begin{center}
\begin{tikzpicture}[xscale=2]
\node[shape=rectangle,fill=pantone127-4] (problem) at (2.5,5) {Problem};
\node (human) at (2.5,4) {Human};
\node[shape=rectangle,fill=pantone127-4] (model) at (2.5,3) {Model};
\node (compile) at (2.5,2) {Compile/Reformulate};
\node (solver1) at (1,1) {CP};
\node (solver2) at (2,1) {MIP};
\node (solver3) at (3,1) {SAT};
\node (solver4) at (4,1) {Other};
\node[shape=rectangle,fill=pantone157-8] (solution1) at (1,0) {Solution};
\node[shape=rectangle,fill=pantone157-8] (solution2) at (2,0) {Solution};
\node[shape=rectangle,fill=pantone157-8] (solution3) at (3,0) {Solution};
\node[shape=rectangle,fill=pantone157-8] (solution4) at (4,0) {Solution};
\draw[-] (problem) -- (human);
\draw[->] (human) -- (model);
\draw[-] (model) -- (compile);
\draw[-] (compile) -- (solver1);
\draw[-] (compile) -- (solver2);
\draw[-] (compile) -- (solver3);
\draw[-] (compile) -- (solver4);
\draw[->] (solver1) -- (solution1);
\draw[->] (solver2) -- (solution2);
\draw[->] (solver3) -- (solution3);
\draw[->] (solver4) -- (solution4);
\end{tikzpicture}
\end{center}
\end{frame}

% \begin{frame}
  % \frametitle{Do It Now!}
  % \begin{itemize}
  % \item Download and install Minizinc
  % \item \url{https://www.minizinc.org/}
  % \end{itemize}
% \end{frame}


\begin{frame}
\frametitle{Why use Puzzles as Examples?}
\begin{itemize}
\item Easy to understand the problem
\item Solvable by hand without specialized knowledge
\item Possible to compare automated to manual solving process
\end{itemize}

\begin{quote}
The puzzle, though inanimate, is presented as a solvable problem without lasting negative consequences, a very low-risk low-reward situation. By being a puzzle, the object is attempting to convince the user that it must be completed.
\end{quote}
{\tiny Source: \href{https://www.everydayrhetoric.com/2020/03/the-rhetoric-of-puzzles/}{Every Day Rhetoric}}
\end{frame}


\mode<all>{
\part{Basic Constraint Propagation}
}
\begin{frame}
\partpage
\end{frame}
\mode<all>{
\begin{frame}
\frametitle{Objectives}
\begin{itemize}
\item Overview of Core Constraint Programming
\item Three Main Concepts
\begin{itemize}
\item Constraint Propagation
\item Global Constraints
\item Customizing Search
\end{itemize}
%\item Get Some Experience with MiniZinc
\item Topics will be treated in more detail in later parts of the school
\item Based on Examples, not Formal Description
\end{itemize}
\end{frame}

\begin{frame}
\frametitle{Outline}
\begin{itemize}
\item Why Constraint Programming?
\item Constraint Propagation
\item Global Constraints
\item Customizing Search
%\item Choices in Modelling
%\item Decomposition
%\item More Global Constraints
%\item Methodology
%\item What is missing?
\end{itemize}
\end{frame}

% \begin{frame}
  % \frametitle{Using MiniZinc IDE}
  % \begin{itemize}
  % \item Developed in the Australian NICTA project
  % \item Maintained by Monash University
  % \item Modelling tool with multiple back-end solvers
  % \item Available from \url{https://www.minizinc.org/}
  % \end{itemize}
% \end{frame}


% \begin{frame}
% \frametitle{Examples in ECLiPSe}
% \begin{itemize}
% \item Open sourced constraint programming language
% \item Development goes back to 1985
% \item ECRC, ICL, IC-Parc, PTL, Cisco
% \item \url{https://eclipseclp.org/}
% \item Specialities
% \begin{itemize}
% \item Develop new solvers for specific domains
% \item Integration with MIP
% \end{itemize}
% \item Not included in bundled MiniZinc IDE
% \item Specialized visualization tools used here
  % \begin{itemize}
  % \item CP-Viz, Simonis et al. 2010
    % \end{itemize}
% \end{itemize}
% \end{frame}

\begin{frame}
\frametitle{Tutorial Based on ECLiPSe ELearning Course}
\begin{itemize}
\item Self-study course in constraint programming
\item Supported by Cisco Systems and Silicon Valley Community Foundation
\item Multi-media format, video lectures, slides, handout etc
\item {\small \url{https://eclipseclp.org/ELearning/index.html}}
\end{itemize}
\end{frame}

\begin{frame}
\frametitle{Also Part of CRT-AI Constraint Week}
\begin{itemize}
\item Annual one week course on CP and Optimization in Ireland
\item Part of national training program for PhD students in AI
\item \url{https://www.crt-ai.ie/}
\end{itemize}
\end{frame}



\begin{frame}
\frametitle{Constraint Programming - in a nutshell}
\begin{itemize}
\item Declarative description of problems with
\begin{itemize}
\item {\em Variables} which range over (finite) sets of values
\item {\em Constraints} over subsets of variables which restrict possible value combinations
\item A {\em solution} is a value assignment which satisfies all constraints
\end{itemize}

\item Constraint propagation/reasoning
\begin{itemize}
\item Removing inconsistent values for variables
\item Detect failure if constraint can not be satisfied
\item Interaction of constraints via shared variables
\item Incomplete
\end{itemize}

\item Search
\begin{itemize}
\item User controlled assignment of values to variables
\item Each step triggers constraint propagation 
\end{itemize}
\item Different domains require/allow different methods
\end{itemize}
\end{frame}

\begin{frame}
  \frametitle{Constraint Programming is Different}
  \begin{itemize}
  \item Declarative Programming
    \begin{itemize}
    \item Concentrate on what you want
      \item Not how to get there
      \item Program != Algorithm
      \item Program = Model
    \end{itemize}
    \item Applied to Combinatorial Problems
      \begin{itemize}
        \item No complete polynomial algorithms known (exist?)
        \item CP less ad-hoc than heuristics
        \item Models can evolve
  \end{itemize}
  \end{itemize}
  \end{frame}
    
\begin{frame}
  \frametitle{A Subtractive Process}
  \begin{textblock}{4}(8,-3)
    \includegraphics[width=4cm]{../introduction/images/stages}
  \end{textblock}
  \vfill
  \begin{quote}
    ``Oh, bosh, as Mr. Ruskin says. Sculpture, per se, is the simplest thing in the world. All you have to do is to take a big chunk of marble and a hammer and chisel, make up your mind what you are about to create and chip off all the marble you don't want.''-Paris Gaulois.
  \end{quote}
  
  {\tiny Source: \url{https://quoteinvestigator.com/2014/06/22/chip-away/}}
\end{frame}

\begin{frame}
\frametitle{Basic Process}
\begin{center}
\begin{tikzpicture}
\node[shape=rectangle,fill=pantone127-4] (problem) at (2,5) {Problem};
\node (human) at (2,4) {Human};
\node[shape=rectangle,fill=pantone127-4] (model) at (2,3) {Model};
\node (solver) at (2,2) {Constraint Solver/Search};
\node[shape=rectangle,fill=pantone157-8] (solution) at (2,1) {Solution};
\draw[-] (problem) -- (human);
\draw[->] (human) -- (model);
\draw[-] (model) -- (solver);
\draw[->] (solver) -- (solution);
\end{tikzpicture}
\end{center}
\end{frame}

\begin{frame}
\frametitle{More Realistic}
\begin{center}
\begin{tikzpicture}[xscale=4.0]
\node[shape=rectangle,fill=pantone127-4] (problem) at (2,5) {Problem};
\node (human) at (2,4) {Human};
\node[shape=rectangle,fill=pantone127-4] (model) at (2,3) {Model};
\node (solver) at (2,2) {Constraint Solver/Search};
\node[shape=rectangle,fill=pantone24-8] (hangs) at (1,1) {Hangs};
\node[shape=rectangle,fill=pantone157-8] (solution) at (2,1) {Solution};
\node[shape=rectangle,fill=pantone24-8] (wrong) at (3,1) {Wrong Solution};
\draw[-] (problem) -- (human);
\draw[->] (human) -- (model);
\draw[-] (model) -- (solver);
\draw[->] (solver) -- (hangs);
\draw[->] (solver) -- (solution);
\draw[->] (solver) -- (wrong);
\draw[->] (hangs) -- (1,4) -- (human);
\draw[->] (wrong) -- (3,4) -- (human);
\end{tikzpicture}
\end{center}
\end{frame}

\begin{frame}
\frametitle{Dual Role of Model}
\begin{itemize}
\item Allows Human to Express Problem
\begin{itemize}
\item Close to Problem Domain
\item Constraints as Abstractions
\end{itemize}
\item Allows Solver to Execute
\begin{itemize}
\item Variables as Communication Mechanism
\item Constraints as Algorithms
\end{itemize}
\end{itemize}
\end{frame}

\begin{frame}
\frametitle{Modelling Frameworks}
\begin{itemize}
\item \href{https://www.minizinc.org/}{MiniZinc} (NICTA, Monash University, Australia)
\item \href{https://github.com/eomahony/Numberjack}{NumberJack} (Insight, Ireland) 
\item \href{https://www-users.york.ac.uk/peter.nightingale/savilerow//}{EssencePrime/SavilleRow} (UK) 
\item \href{https://cpmpy.readthedocs.io/en/latest/}{CPMpy} (KU Leuven)
\item Allow use of multiple back-end solvers
\item Compile model into variants for each solver
\item A priori solver independent model(CP, MIP, SAT)
\end{itemize}
\end{frame} 

\begin{frame}
\frametitle{Framework Process}
\begin{center}
\begin{tikzpicture}[xscale=2]
\node[shape=rectangle,fill=pantone127-4] (problem) at (2.5,5) {Problem};
\node (human) at (2.5,4) {Human};
\node[shape=rectangle,fill=pantone127-4] (model) at (2.5,3) {Model};
\node (compile) at (2.5,2) {Compile/Reformulate};
\node (solver1) at (1,1) {CP};
\node (solver2) at (2,1) {MIP};
\node (solver3) at (3,1) {SAT};
\node (solver4) at (4,1) {Other};
\node[shape=rectangle,fill=pantone157-8] (solution1) at (1,0) {Solution};
\node[shape=rectangle,fill=pantone157-8] (solution2) at (2,0) {Solution};
\node[shape=rectangle,fill=pantone157-8] (solution3) at (3,0) {Solution};
\node[shape=rectangle,fill=pantone157-8] (solution4) at (4,0) {Solution};
\draw[-] (problem) -- (human);
\draw[->] (human) -- (model);
\draw[-] (model) -- (compile);
\draw[-] (compile) -- (solver1);
\draw[-] (compile) -- (solver2);
\draw[-] (compile) -- (solver3);
\draw[-] (compile) -- (solver4);
\draw[->] (solver1) -- (solution1);
\draw[->] (solver2) -- (solution2);
\draw[->] (solver3) -- (solution3);
\draw[->] (solver4) -- (solution4);
\end{tikzpicture}
\end{center}
\end{frame}

% \begin{frame}
  % \frametitle{Do It Now!}
  % \begin{itemize}
  % \item Download and install Minizinc
  % \item \url{https://www.minizinc.org/}
  % \end{itemize}
% \end{frame}


\begin{frame}
\frametitle{Why use Puzzles as Examples?}
\begin{itemize}
\item Easy to understand the problem
\item Solvable by hand without specialized knowledge
\item Possible to compare automated to manual solving process
\end{itemize}

\begin{quote}
The puzzle, though inanimate, is presented as a solvable problem without lasting negative consequences, a very low-risk low-reward situation. By being a puzzle, the object is attempting to convince the user that it must be completed.
\end{quote}
{\tiny Source: \href{https://www.everydayrhetoric.com/2020/03/the-rhetoric-of-puzzles/}{Every Day Rhetoric}}
\end{frame}


\mode<all>{
\part{Basic Constraint Propagation}
}
\begin{frame}
\partpage
\end{frame}
\mode<all>{
\input{../sendmore/body}
}

\mode<all>{
\part{Global Constraints}
}
\begin{frame}
\partpage
\end{frame}
\mode<all>{
\input{../sudoku/body}
}

\clearpage
\mode<all>{
\part{Customizing Search}
}
\begin{frame}
\partpage
\end{frame}
\mode<all>{
\input{../nqueen/body}
}


%% \clearpage
%% \mode<all>{
%%   \part{Problem Decomposition}
%% }
%% \begin{frame}
%%   \partpage
%% \end{frame}
%% \mode<all>{
%%   \input{../party/body}
%%   }

%% \clearpage
%% \mode<all>{
%%   \part{More Global Constraints}
%% }
%% \begin{frame}
%%   \partpage
%% \end{frame}
%% \mode<all>{
%%   \input{../car/body}
%%   }

%% \clearpage
%% \mode<all>{
%%   \part{Methodology}
%% }
%% \begin{frame}
%%   \partpage
%% \end{frame}
%% \mode<all>{
%%   \input{../methodology/body}
%%   }

\clearpage
\mode<all>{
\part{What is missing?}
}
\begin{frame}
\partpage
\end{frame}

\begin{frame}
\frametitle{Many Specialized Topics}
\begin{itemize}
\item How to design efficient core engine
\item Hybrids with LP/MIP tools
\item Hybrids with SAT
\item Symmetry breaking
\item Use of MDD/BDD to encode sets of solutions
\item High level modelling tools
\item Debugging/visualization
\end{itemize}
\end{frame}

\begin{frame}
\frametitle{Reformulation}
\begin{itemize}
\item Just because the user has modelled it this way, it doesn't mean we have to solve it that way
\begin{itemize}
\item Replace some constraint(s) by other, equivalent constraints
\item Because we don't have that constraint in our system
\item For performance
\end{itemize}
\end{itemize}
\end{frame}

\begin{frame}
\frametitle{Learning}
\begin{itemize}
\item While solving the problem we can learn how to strengthen the model/search 
\begin{itemize}
\item Understand which constraints/method contribute to propagation and change schedule
\item Learn no-good constraints by explaining failure
\item Adapt search strategy based on search experience
\end{itemize}
\end{itemize}
\end{frame}

%% \begin{frame}
%% \frametitle{Refined Process}
%% \begin{center}
%% \begin{tikzpicture}
%% \node[shape=rectangle,fill=pantone127-4] (problem) at (3,7) {Problem};
%% \node (human) at (3,6) {Human};
%% \node[shape=rectangle,fill=pantone127-4] (model) at (3,5) {User Model};
%% \node (reformulation) at (3,4) {Reformulation};
%% \node[shape=rectangle,fill=pantone127-4] (implementation) at (3,3) {Implementation Model};
%% \node (solver) at (3,2) {Constraint Solver/Search};
%% \node[shape=rectangle,fill=pantone157-8] (solution) at (3,1) {Solution};
%% \draw[-] (problem) -- (human);
%% \draw[->] (human) -- (model);
%% \draw[-] (model) -- (reformulation);
%% \draw[->] (reformulation) -- (implementation);
%% \draw[-] (implementation) -- (solver);
%% \draw[->] (solver) -- (0,2) -- (0,3) -- (implementation);
%% \draw[->] (solver) -- (solution);
%% \end{tikzpicture}
%% \end{center}
%% \end{frame}

%\begin{frame}
%\frametitle{What is CP actually used for?}
%\url{http://hsimonis.wordpress.com}
%\includegraphics[width=10cm]{../introduction/images/blog}
%\end{frame}

%% \begin{frame}
%% \frametitle{Exercises}
%% \begin{itemize}
%% \item Install Constraint Solver of your Choice
%% \item MiniZinc (MiningZinc) will be used in later sessions
%% \item Run Sendmoremoney, Sudoku, Queens examples
%% \item Can you choose which propagator to use for alldifferent?
%% \item Can you use different search strategies?
%% \end{itemize}
%% \end{frame}

\begin{frame}
\frametitle{More Learning Resources}
\begin{itemize}
\item Survey of Methods, Resources, and Formats for Teaching Constraint Programming
\begin{itemize}
\item by Tejas Santanam, Helmut Simonis
\item \url{https://doi.org/10.48550/arXiv.2403.12717}
\item Based on survey of community for WTCP 2023
\item \url{https://hsimonis.github.io/WTCP2023/}
\end{itemize}
\end{itemize}
\end{frame}


}

\mode<all>{
\part{Global Constraints}
}
\begin{frame}
\partpage
\end{frame}
\mode<all>{
\begin{frame}
\frametitle{Objectives}
\begin{itemize}
\item Overview of Core Constraint Programming
\item Three Main Concepts
\begin{itemize}
\item Constraint Propagation
\item Global Constraints
\item Customizing Search
\end{itemize}
%\item Get Some Experience with MiniZinc
\item Topics will be treated in more detail in later parts of the school
\item Based on Examples, not Formal Description
\end{itemize}
\end{frame}

\begin{frame}
\frametitle{Outline}
\begin{itemize}
\item Why Constraint Programming?
\item Constraint Propagation
\item Global Constraints
\item Customizing Search
%\item Choices in Modelling
%\item Decomposition
%\item More Global Constraints
%\item Methodology
%\item What is missing?
\end{itemize}
\end{frame}

% \begin{frame}
  % \frametitle{Using MiniZinc IDE}
  % \begin{itemize}
  % \item Developed in the Australian NICTA project
  % \item Maintained by Monash University
  % \item Modelling tool with multiple back-end solvers
  % \item Available from \url{https://www.minizinc.org/}
  % \end{itemize}
% \end{frame}


% \begin{frame}
% \frametitle{Examples in ECLiPSe}
% \begin{itemize}
% \item Open sourced constraint programming language
% \item Development goes back to 1985
% \item ECRC, ICL, IC-Parc, PTL, Cisco
% \item \url{https://eclipseclp.org/}
% \item Specialities
% \begin{itemize}
% \item Develop new solvers for specific domains
% \item Integration with MIP
% \end{itemize}
% \item Not included in bundled MiniZinc IDE
% \item Specialized visualization tools used here
  % \begin{itemize}
  % \item CP-Viz, Simonis et al. 2010
    % \end{itemize}
% \end{itemize}
% \end{frame}

\begin{frame}
\frametitle{Tutorial Based on ECLiPSe ELearning Course}
\begin{itemize}
\item Self-study course in constraint programming
\item Supported by Cisco Systems and Silicon Valley Community Foundation
\item Multi-media format, video lectures, slides, handout etc
\item {\small \url{https://eclipseclp.org/ELearning/index.html}}
\end{itemize}
\end{frame}

\begin{frame}
\frametitle{Also Part of CRT-AI Constraint Week}
\begin{itemize}
\item Annual one week course on CP and Optimization in Ireland
\item Part of national training program for PhD students in AI
\item \url{https://www.crt-ai.ie/}
\end{itemize}
\end{frame}



\begin{frame}
\frametitle{Constraint Programming - in a nutshell}
\begin{itemize}
\item Declarative description of problems with
\begin{itemize}
\item {\em Variables} which range over (finite) sets of values
\item {\em Constraints} over subsets of variables which restrict possible value combinations
\item A {\em solution} is a value assignment which satisfies all constraints
\end{itemize}

\item Constraint propagation/reasoning
\begin{itemize}
\item Removing inconsistent values for variables
\item Detect failure if constraint can not be satisfied
\item Interaction of constraints via shared variables
\item Incomplete
\end{itemize}

\item Search
\begin{itemize}
\item User controlled assignment of values to variables
\item Each step triggers constraint propagation 
\end{itemize}
\item Different domains require/allow different methods
\end{itemize}
\end{frame}

\begin{frame}
  \frametitle{Constraint Programming is Different}
  \begin{itemize}
  \item Declarative Programming
    \begin{itemize}
    \item Concentrate on what you want
      \item Not how to get there
      \item Program != Algorithm
      \item Program = Model
    \end{itemize}
    \item Applied to Combinatorial Problems
      \begin{itemize}
        \item No complete polynomial algorithms known (exist?)
        \item CP less ad-hoc than heuristics
        \item Models can evolve
  \end{itemize}
  \end{itemize}
  \end{frame}
    
\begin{frame}
  \frametitle{A Subtractive Process}
  \begin{textblock}{4}(8,-3)
    \includegraphics[width=4cm]{../introduction/images/stages}
  \end{textblock}
  \vfill
  \begin{quote}
    ``Oh, bosh, as Mr. Ruskin says. Sculpture, per se, is the simplest thing in the world. All you have to do is to take a big chunk of marble and a hammer and chisel, make up your mind what you are about to create and chip off all the marble you don't want.''-Paris Gaulois.
  \end{quote}
  
  {\tiny Source: \url{https://quoteinvestigator.com/2014/06/22/chip-away/}}
\end{frame}

\begin{frame}
\frametitle{Basic Process}
\begin{center}
\begin{tikzpicture}
\node[shape=rectangle,fill=pantone127-4] (problem) at (2,5) {Problem};
\node (human) at (2,4) {Human};
\node[shape=rectangle,fill=pantone127-4] (model) at (2,3) {Model};
\node (solver) at (2,2) {Constraint Solver/Search};
\node[shape=rectangle,fill=pantone157-8] (solution) at (2,1) {Solution};
\draw[-] (problem) -- (human);
\draw[->] (human) -- (model);
\draw[-] (model) -- (solver);
\draw[->] (solver) -- (solution);
\end{tikzpicture}
\end{center}
\end{frame}

\begin{frame}
\frametitle{More Realistic}
\begin{center}
\begin{tikzpicture}[xscale=4.0]
\node[shape=rectangle,fill=pantone127-4] (problem) at (2,5) {Problem};
\node (human) at (2,4) {Human};
\node[shape=rectangle,fill=pantone127-4] (model) at (2,3) {Model};
\node (solver) at (2,2) {Constraint Solver/Search};
\node[shape=rectangle,fill=pantone24-8] (hangs) at (1,1) {Hangs};
\node[shape=rectangle,fill=pantone157-8] (solution) at (2,1) {Solution};
\node[shape=rectangle,fill=pantone24-8] (wrong) at (3,1) {Wrong Solution};
\draw[-] (problem) -- (human);
\draw[->] (human) -- (model);
\draw[-] (model) -- (solver);
\draw[->] (solver) -- (hangs);
\draw[->] (solver) -- (solution);
\draw[->] (solver) -- (wrong);
\draw[->] (hangs) -- (1,4) -- (human);
\draw[->] (wrong) -- (3,4) -- (human);
\end{tikzpicture}
\end{center}
\end{frame}

\begin{frame}
\frametitle{Dual Role of Model}
\begin{itemize}
\item Allows Human to Express Problem
\begin{itemize}
\item Close to Problem Domain
\item Constraints as Abstractions
\end{itemize}
\item Allows Solver to Execute
\begin{itemize}
\item Variables as Communication Mechanism
\item Constraints as Algorithms
\end{itemize}
\end{itemize}
\end{frame}

\begin{frame}
\frametitle{Modelling Frameworks}
\begin{itemize}
\item \href{https://www.minizinc.org/}{MiniZinc} (NICTA, Monash University, Australia)
\item \href{https://github.com/eomahony/Numberjack}{NumberJack} (Insight, Ireland) 
\item \href{https://www-users.york.ac.uk/peter.nightingale/savilerow//}{EssencePrime/SavilleRow} (UK) 
\item \href{https://cpmpy.readthedocs.io/en/latest/}{CPMpy} (KU Leuven)
\item Allow use of multiple back-end solvers
\item Compile model into variants for each solver
\item A priori solver independent model(CP, MIP, SAT)
\end{itemize}
\end{frame} 

\begin{frame}
\frametitle{Framework Process}
\begin{center}
\begin{tikzpicture}[xscale=2]
\node[shape=rectangle,fill=pantone127-4] (problem) at (2.5,5) {Problem};
\node (human) at (2.5,4) {Human};
\node[shape=rectangle,fill=pantone127-4] (model) at (2.5,3) {Model};
\node (compile) at (2.5,2) {Compile/Reformulate};
\node (solver1) at (1,1) {CP};
\node (solver2) at (2,1) {MIP};
\node (solver3) at (3,1) {SAT};
\node (solver4) at (4,1) {Other};
\node[shape=rectangle,fill=pantone157-8] (solution1) at (1,0) {Solution};
\node[shape=rectangle,fill=pantone157-8] (solution2) at (2,0) {Solution};
\node[shape=rectangle,fill=pantone157-8] (solution3) at (3,0) {Solution};
\node[shape=rectangle,fill=pantone157-8] (solution4) at (4,0) {Solution};
\draw[-] (problem) -- (human);
\draw[->] (human) -- (model);
\draw[-] (model) -- (compile);
\draw[-] (compile) -- (solver1);
\draw[-] (compile) -- (solver2);
\draw[-] (compile) -- (solver3);
\draw[-] (compile) -- (solver4);
\draw[->] (solver1) -- (solution1);
\draw[->] (solver2) -- (solution2);
\draw[->] (solver3) -- (solution3);
\draw[->] (solver4) -- (solution4);
\end{tikzpicture}
\end{center}
\end{frame}

% \begin{frame}
  % \frametitle{Do It Now!}
  % \begin{itemize}
  % \item Download and install Minizinc
  % \item \url{https://www.minizinc.org/}
  % \end{itemize}
% \end{frame}


\begin{frame}
\frametitle{Why use Puzzles as Examples?}
\begin{itemize}
\item Easy to understand the problem
\item Solvable by hand without specialized knowledge
\item Possible to compare automated to manual solving process
\end{itemize}

\begin{quote}
The puzzle, though inanimate, is presented as a solvable problem without lasting negative consequences, a very low-risk low-reward situation. By being a puzzle, the object is attempting to convince the user that it must be completed.
\end{quote}
{\tiny Source: \href{https://www.everydayrhetoric.com/2020/03/the-rhetoric-of-puzzles/}{Every Day Rhetoric}}
\end{frame}


\mode<all>{
\part{Basic Constraint Propagation}
}
\begin{frame}
\partpage
\end{frame}
\mode<all>{
\input{../sendmore/body}
}

\mode<all>{
\part{Global Constraints}
}
\begin{frame}
\partpage
\end{frame}
\mode<all>{
\input{../sudoku/body}
}

\clearpage
\mode<all>{
\part{Customizing Search}
}
\begin{frame}
\partpage
\end{frame}
\mode<all>{
\input{../nqueen/body}
}


%% \clearpage
%% \mode<all>{
%%   \part{Problem Decomposition}
%% }
%% \begin{frame}
%%   \partpage
%% \end{frame}
%% \mode<all>{
%%   \input{../party/body}
%%   }

%% \clearpage
%% \mode<all>{
%%   \part{More Global Constraints}
%% }
%% \begin{frame}
%%   \partpage
%% \end{frame}
%% \mode<all>{
%%   \input{../car/body}
%%   }

%% \clearpage
%% \mode<all>{
%%   \part{Methodology}
%% }
%% \begin{frame}
%%   \partpage
%% \end{frame}
%% \mode<all>{
%%   \input{../methodology/body}
%%   }

\clearpage
\mode<all>{
\part{What is missing?}
}
\begin{frame}
\partpage
\end{frame}

\begin{frame}
\frametitle{Many Specialized Topics}
\begin{itemize}
\item How to design efficient core engine
\item Hybrids with LP/MIP tools
\item Hybrids with SAT
\item Symmetry breaking
\item Use of MDD/BDD to encode sets of solutions
\item High level modelling tools
\item Debugging/visualization
\end{itemize}
\end{frame}

\begin{frame}
\frametitle{Reformulation}
\begin{itemize}
\item Just because the user has modelled it this way, it doesn't mean we have to solve it that way
\begin{itemize}
\item Replace some constraint(s) by other, equivalent constraints
\item Because we don't have that constraint in our system
\item For performance
\end{itemize}
\end{itemize}
\end{frame}

\begin{frame}
\frametitle{Learning}
\begin{itemize}
\item While solving the problem we can learn how to strengthen the model/search 
\begin{itemize}
\item Understand which constraints/method contribute to propagation and change schedule
\item Learn no-good constraints by explaining failure
\item Adapt search strategy based on search experience
\end{itemize}
\end{itemize}
\end{frame}

%% \begin{frame}
%% \frametitle{Refined Process}
%% \begin{center}
%% \begin{tikzpicture}
%% \node[shape=rectangle,fill=pantone127-4] (problem) at (3,7) {Problem};
%% \node (human) at (3,6) {Human};
%% \node[shape=rectangle,fill=pantone127-4] (model) at (3,5) {User Model};
%% \node (reformulation) at (3,4) {Reformulation};
%% \node[shape=rectangle,fill=pantone127-4] (implementation) at (3,3) {Implementation Model};
%% \node (solver) at (3,2) {Constraint Solver/Search};
%% \node[shape=rectangle,fill=pantone157-8] (solution) at (3,1) {Solution};
%% \draw[-] (problem) -- (human);
%% \draw[->] (human) -- (model);
%% \draw[-] (model) -- (reformulation);
%% \draw[->] (reformulation) -- (implementation);
%% \draw[-] (implementation) -- (solver);
%% \draw[->] (solver) -- (0,2) -- (0,3) -- (implementation);
%% \draw[->] (solver) -- (solution);
%% \end{tikzpicture}
%% \end{center}
%% \end{frame}

%\begin{frame}
%\frametitle{What is CP actually used for?}
%\url{http://hsimonis.wordpress.com}
%\includegraphics[width=10cm]{../introduction/images/blog}
%\end{frame}

%% \begin{frame}
%% \frametitle{Exercises}
%% \begin{itemize}
%% \item Install Constraint Solver of your Choice
%% \item MiniZinc (MiningZinc) will be used in later sessions
%% \item Run Sendmoremoney, Sudoku, Queens examples
%% \item Can you choose which propagator to use for alldifferent?
%% \item Can you use different search strategies?
%% \end{itemize}
%% \end{frame}

\begin{frame}
\frametitle{More Learning Resources}
\begin{itemize}
\item Survey of Methods, Resources, and Formats for Teaching Constraint Programming
\begin{itemize}
\item by Tejas Santanam, Helmut Simonis
\item \url{https://doi.org/10.48550/arXiv.2403.12717}
\item Based on survey of community for WTCP 2023
\item \url{https://hsimonis.github.io/WTCP2023/}
\end{itemize}
\end{itemize}
\end{frame}


}

\clearpage
\mode<all>{
\part{Customizing Search}
}
\begin{frame}
\partpage
\end{frame}
\mode<all>{
\begin{frame}
\frametitle{Objectives}
\begin{itemize}
\item Overview of Core Constraint Programming
\item Three Main Concepts
\begin{itemize}
\item Constraint Propagation
\item Global Constraints
\item Customizing Search
\end{itemize}
%\item Get Some Experience with MiniZinc
\item Topics will be treated in more detail in later parts of the school
\item Based on Examples, not Formal Description
\end{itemize}
\end{frame}

\begin{frame}
\frametitle{Outline}
\begin{itemize}
\item Why Constraint Programming?
\item Constraint Propagation
\item Global Constraints
\item Customizing Search
%\item Choices in Modelling
%\item Decomposition
%\item More Global Constraints
%\item Methodology
%\item What is missing?
\end{itemize}
\end{frame}

% \begin{frame}
  % \frametitle{Using MiniZinc IDE}
  % \begin{itemize}
  % \item Developed in the Australian NICTA project
  % \item Maintained by Monash University
  % \item Modelling tool with multiple back-end solvers
  % \item Available from \url{https://www.minizinc.org/}
  % \end{itemize}
% \end{frame}


% \begin{frame}
% \frametitle{Examples in ECLiPSe}
% \begin{itemize}
% \item Open sourced constraint programming language
% \item Development goes back to 1985
% \item ECRC, ICL, IC-Parc, PTL, Cisco
% \item \url{https://eclipseclp.org/}
% \item Specialities
% \begin{itemize}
% \item Develop new solvers for specific domains
% \item Integration with MIP
% \end{itemize}
% \item Not included in bundled MiniZinc IDE
% \item Specialized visualization tools used here
  % \begin{itemize}
  % \item CP-Viz, Simonis et al. 2010
    % \end{itemize}
% \end{itemize}
% \end{frame}

\begin{frame}
\frametitle{Tutorial Based on ECLiPSe ELearning Course}
\begin{itemize}
\item Self-study course in constraint programming
\item Supported by Cisco Systems and Silicon Valley Community Foundation
\item Multi-media format, video lectures, slides, handout etc
\item {\small \url{https://eclipseclp.org/ELearning/index.html}}
\end{itemize}
\end{frame}

\begin{frame}
\frametitle{Also Part of CRT-AI Constraint Week}
\begin{itemize}
\item Annual one week course on CP and Optimization in Ireland
\item Part of national training program for PhD students in AI
\item \url{https://www.crt-ai.ie/}
\end{itemize}
\end{frame}



\begin{frame}
\frametitle{Constraint Programming - in a nutshell}
\begin{itemize}
\item Declarative description of problems with
\begin{itemize}
\item {\em Variables} which range over (finite) sets of values
\item {\em Constraints} over subsets of variables which restrict possible value combinations
\item A {\em solution} is a value assignment which satisfies all constraints
\end{itemize}

\item Constraint propagation/reasoning
\begin{itemize}
\item Removing inconsistent values for variables
\item Detect failure if constraint can not be satisfied
\item Interaction of constraints via shared variables
\item Incomplete
\end{itemize}

\item Search
\begin{itemize}
\item User controlled assignment of values to variables
\item Each step triggers constraint propagation 
\end{itemize}
\item Different domains require/allow different methods
\end{itemize}
\end{frame}

\begin{frame}
  \frametitle{Constraint Programming is Different}
  \begin{itemize}
  \item Declarative Programming
    \begin{itemize}
    \item Concentrate on what you want
      \item Not how to get there
      \item Program != Algorithm
      \item Program = Model
    \end{itemize}
    \item Applied to Combinatorial Problems
      \begin{itemize}
        \item No complete polynomial algorithms known (exist?)
        \item CP less ad-hoc than heuristics
        \item Models can evolve
  \end{itemize}
  \end{itemize}
  \end{frame}
    
\begin{frame}
  \frametitle{A Subtractive Process}
  \begin{textblock}{4}(8,-3)
    \includegraphics[width=4cm]{../introduction/images/stages}
  \end{textblock}
  \vfill
  \begin{quote}
    ``Oh, bosh, as Mr. Ruskin says. Sculpture, per se, is the simplest thing in the world. All you have to do is to take a big chunk of marble and a hammer and chisel, make up your mind what you are about to create and chip off all the marble you don't want.''-Paris Gaulois.
  \end{quote}
  
  {\tiny Source: \url{https://quoteinvestigator.com/2014/06/22/chip-away/}}
\end{frame}

\begin{frame}
\frametitle{Basic Process}
\begin{center}
\begin{tikzpicture}
\node[shape=rectangle,fill=pantone127-4] (problem) at (2,5) {Problem};
\node (human) at (2,4) {Human};
\node[shape=rectangle,fill=pantone127-4] (model) at (2,3) {Model};
\node (solver) at (2,2) {Constraint Solver/Search};
\node[shape=rectangle,fill=pantone157-8] (solution) at (2,1) {Solution};
\draw[-] (problem) -- (human);
\draw[->] (human) -- (model);
\draw[-] (model) -- (solver);
\draw[->] (solver) -- (solution);
\end{tikzpicture}
\end{center}
\end{frame}

\begin{frame}
\frametitle{More Realistic}
\begin{center}
\begin{tikzpicture}[xscale=4.0]
\node[shape=rectangle,fill=pantone127-4] (problem) at (2,5) {Problem};
\node (human) at (2,4) {Human};
\node[shape=rectangle,fill=pantone127-4] (model) at (2,3) {Model};
\node (solver) at (2,2) {Constraint Solver/Search};
\node[shape=rectangle,fill=pantone24-8] (hangs) at (1,1) {Hangs};
\node[shape=rectangle,fill=pantone157-8] (solution) at (2,1) {Solution};
\node[shape=rectangle,fill=pantone24-8] (wrong) at (3,1) {Wrong Solution};
\draw[-] (problem) -- (human);
\draw[->] (human) -- (model);
\draw[-] (model) -- (solver);
\draw[->] (solver) -- (hangs);
\draw[->] (solver) -- (solution);
\draw[->] (solver) -- (wrong);
\draw[->] (hangs) -- (1,4) -- (human);
\draw[->] (wrong) -- (3,4) -- (human);
\end{tikzpicture}
\end{center}
\end{frame}

\begin{frame}
\frametitle{Dual Role of Model}
\begin{itemize}
\item Allows Human to Express Problem
\begin{itemize}
\item Close to Problem Domain
\item Constraints as Abstractions
\end{itemize}
\item Allows Solver to Execute
\begin{itemize}
\item Variables as Communication Mechanism
\item Constraints as Algorithms
\end{itemize}
\end{itemize}
\end{frame}

\begin{frame}
\frametitle{Modelling Frameworks}
\begin{itemize}
\item \href{https://www.minizinc.org/}{MiniZinc} (NICTA, Monash University, Australia)
\item \href{https://github.com/eomahony/Numberjack}{NumberJack} (Insight, Ireland) 
\item \href{https://www-users.york.ac.uk/peter.nightingale/savilerow//}{EssencePrime/SavilleRow} (UK) 
\item \href{https://cpmpy.readthedocs.io/en/latest/}{CPMpy} (KU Leuven)
\item Allow use of multiple back-end solvers
\item Compile model into variants for each solver
\item A priori solver independent model(CP, MIP, SAT)
\end{itemize}
\end{frame} 

\begin{frame}
\frametitle{Framework Process}
\begin{center}
\begin{tikzpicture}[xscale=2]
\node[shape=rectangle,fill=pantone127-4] (problem) at (2.5,5) {Problem};
\node (human) at (2.5,4) {Human};
\node[shape=rectangle,fill=pantone127-4] (model) at (2.5,3) {Model};
\node (compile) at (2.5,2) {Compile/Reformulate};
\node (solver1) at (1,1) {CP};
\node (solver2) at (2,1) {MIP};
\node (solver3) at (3,1) {SAT};
\node (solver4) at (4,1) {Other};
\node[shape=rectangle,fill=pantone157-8] (solution1) at (1,0) {Solution};
\node[shape=rectangle,fill=pantone157-8] (solution2) at (2,0) {Solution};
\node[shape=rectangle,fill=pantone157-8] (solution3) at (3,0) {Solution};
\node[shape=rectangle,fill=pantone157-8] (solution4) at (4,0) {Solution};
\draw[-] (problem) -- (human);
\draw[->] (human) -- (model);
\draw[-] (model) -- (compile);
\draw[-] (compile) -- (solver1);
\draw[-] (compile) -- (solver2);
\draw[-] (compile) -- (solver3);
\draw[-] (compile) -- (solver4);
\draw[->] (solver1) -- (solution1);
\draw[->] (solver2) -- (solution2);
\draw[->] (solver3) -- (solution3);
\draw[->] (solver4) -- (solution4);
\end{tikzpicture}
\end{center}
\end{frame}

% \begin{frame}
  % \frametitle{Do It Now!}
  % \begin{itemize}
  % \item Download and install Minizinc
  % \item \url{https://www.minizinc.org/}
  % \end{itemize}
% \end{frame}


\begin{frame}
\frametitle{Why use Puzzles as Examples?}
\begin{itemize}
\item Easy to understand the problem
\item Solvable by hand without specialized knowledge
\item Possible to compare automated to manual solving process
\end{itemize}

\begin{quote}
The puzzle, though inanimate, is presented as a solvable problem without lasting negative consequences, a very low-risk low-reward situation. By being a puzzle, the object is attempting to convince the user that it must be completed.
\end{quote}
{\tiny Source: \href{https://www.everydayrhetoric.com/2020/03/the-rhetoric-of-puzzles/}{Every Day Rhetoric}}
\end{frame}


\mode<all>{
\part{Basic Constraint Propagation}
}
\begin{frame}
\partpage
\end{frame}
\mode<all>{
\input{../sendmore/body}
}

\mode<all>{
\part{Global Constraints}
}
\begin{frame}
\partpage
\end{frame}
\mode<all>{
\input{../sudoku/body}
}

\clearpage
\mode<all>{
\part{Customizing Search}
}
\begin{frame}
\partpage
\end{frame}
\mode<all>{
\input{../nqueen/body}
}


%% \clearpage
%% \mode<all>{
%%   \part{Problem Decomposition}
%% }
%% \begin{frame}
%%   \partpage
%% \end{frame}
%% \mode<all>{
%%   \input{../party/body}
%%   }

%% \clearpage
%% \mode<all>{
%%   \part{More Global Constraints}
%% }
%% \begin{frame}
%%   \partpage
%% \end{frame}
%% \mode<all>{
%%   \input{../car/body}
%%   }

%% \clearpage
%% \mode<all>{
%%   \part{Methodology}
%% }
%% \begin{frame}
%%   \partpage
%% \end{frame}
%% \mode<all>{
%%   \input{../methodology/body}
%%   }

\clearpage
\mode<all>{
\part{What is missing?}
}
\begin{frame}
\partpage
\end{frame}

\begin{frame}
\frametitle{Many Specialized Topics}
\begin{itemize}
\item How to design efficient core engine
\item Hybrids with LP/MIP tools
\item Hybrids with SAT
\item Symmetry breaking
\item Use of MDD/BDD to encode sets of solutions
\item High level modelling tools
\item Debugging/visualization
\end{itemize}
\end{frame}

\begin{frame}
\frametitle{Reformulation}
\begin{itemize}
\item Just because the user has modelled it this way, it doesn't mean we have to solve it that way
\begin{itemize}
\item Replace some constraint(s) by other, equivalent constraints
\item Because we don't have that constraint in our system
\item For performance
\end{itemize}
\end{itemize}
\end{frame}

\begin{frame}
\frametitle{Learning}
\begin{itemize}
\item While solving the problem we can learn how to strengthen the model/search 
\begin{itemize}
\item Understand which constraints/method contribute to propagation and change schedule
\item Learn no-good constraints by explaining failure
\item Adapt search strategy based on search experience
\end{itemize}
\end{itemize}
\end{frame}

%% \begin{frame}
%% \frametitle{Refined Process}
%% \begin{center}
%% \begin{tikzpicture}
%% \node[shape=rectangle,fill=pantone127-4] (problem) at (3,7) {Problem};
%% \node (human) at (3,6) {Human};
%% \node[shape=rectangle,fill=pantone127-4] (model) at (3,5) {User Model};
%% \node (reformulation) at (3,4) {Reformulation};
%% \node[shape=rectangle,fill=pantone127-4] (implementation) at (3,3) {Implementation Model};
%% \node (solver) at (3,2) {Constraint Solver/Search};
%% \node[shape=rectangle,fill=pantone157-8] (solution) at (3,1) {Solution};
%% \draw[-] (problem) -- (human);
%% \draw[->] (human) -- (model);
%% \draw[-] (model) -- (reformulation);
%% \draw[->] (reformulation) -- (implementation);
%% \draw[-] (implementation) -- (solver);
%% \draw[->] (solver) -- (0,2) -- (0,3) -- (implementation);
%% \draw[->] (solver) -- (solution);
%% \end{tikzpicture}
%% \end{center}
%% \end{frame}

%\begin{frame}
%\frametitle{What is CP actually used for?}
%\url{http://hsimonis.wordpress.com}
%\includegraphics[width=10cm]{../introduction/images/blog}
%\end{frame}

%% \begin{frame}
%% \frametitle{Exercises}
%% \begin{itemize}
%% \item Install Constraint Solver of your Choice
%% \item MiniZinc (MiningZinc) will be used in later sessions
%% \item Run Sendmoremoney, Sudoku, Queens examples
%% \item Can you choose which propagator to use for alldifferent?
%% \item Can you use different search strategies?
%% \end{itemize}
%% \end{frame}

\begin{frame}
\frametitle{More Learning Resources}
\begin{itemize}
\item Survey of Methods, Resources, and Formats for Teaching Constraint Programming
\begin{itemize}
\item by Tejas Santanam, Helmut Simonis
\item \url{https://doi.org/10.48550/arXiv.2403.12717}
\item Based on survey of community for WTCP 2023
\item \url{https://hsimonis.github.io/WTCP2023/}
\end{itemize}
\end{itemize}
\end{frame}


}


%% \clearpage
%% \mode<all>{
%%   \part{Problem Decomposition}
%% }
%% \begin{frame}
%%   \partpage
%% \end{frame}
%% \mode<all>{
%%   \begin{frame}
\frametitle{Objectives}
\begin{itemize}
\item Overview of Core Constraint Programming
\item Three Main Concepts
\begin{itemize}
\item Constraint Propagation
\item Global Constraints
\item Customizing Search
\end{itemize}
%\item Get Some Experience with MiniZinc
\item Topics will be treated in more detail in later parts of the school
\item Based on Examples, not Formal Description
\end{itemize}
\end{frame}

\begin{frame}
\frametitle{Outline}
\begin{itemize}
\item Why Constraint Programming?
\item Constraint Propagation
\item Global Constraints
\item Customizing Search
%\item Choices in Modelling
%\item Decomposition
%\item More Global Constraints
%\item Methodology
%\item What is missing?
\end{itemize}
\end{frame}

% \begin{frame}
  % \frametitle{Using MiniZinc IDE}
  % \begin{itemize}
  % \item Developed in the Australian NICTA project
  % \item Maintained by Monash University
  % \item Modelling tool with multiple back-end solvers
  % \item Available from \url{https://www.minizinc.org/}
  % \end{itemize}
% \end{frame}


% \begin{frame}
% \frametitle{Examples in ECLiPSe}
% \begin{itemize}
% \item Open sourced constraint programming language
% \item Development goes back to 1985
% \item ECRC, ICL, IC-Parc, PTL, Cisco
% \item \url{https://eclipseclp.org/}
% \item Specialities
% \begin{itemize}
% \item Develop new solvers for specific domains
% \item Integration with MIP
% \end{itemize}
% \item Not included in bundled MiniZinc IDE
% \item Specialized visualization tools used here
  % \begin{itemize}
  % \item CP-Viz, Simonis et al. 2010
    % \end{itemize}
% \end{itemize}
% \end{frame}

\begin{frame}
\frametitle{Tutorial Based on ECLiPSe ELearning Course}
\begin{itemize}
\item Self-study course in constraint programming
\item Supported by Cisco Systems and Silicon Valley Community Foundation
\item Multi-media format, video lectures, slides, handout etc
\item {\small \url{https://eclipseclp.org/ELearning/index.html}}
\end{itemize}
\end{frame}

\begin{frame}
\frametitle{Also Part of CRT-AI Constraint Week}
\begin{itemize}
\item Annual one week course on CP and Optimization in Ireland
\item Part of national training program for PhD students in AI
\item \url{https://www.crt-ai.ie/}
\end{itemize}
\end{frame}



\begin{frame}
\frametitle{Constraint Programming - in a nutshell}
\begin{itemize}
\item Declarative description of problems with
\begin{itemize}
\item {\em Variables} which range over (finite) sets of values
\item {\em Constraints} over subsets of variables which restrict possible value combinations
\item A {\em solution} is a value assignment which satisfies all constraints
\end{itemize}

\item Constraint propagation/reasoning
\begin{itemize}
\item Removing inconsistent values for variables
\item Detect failure if constraint can not be satisfied
\item Interaction of constraints via shared variables
\item Incomplete
\end{itemize}

\item Search
\begin{itemize}
\item User controlled assignment of values to variables
\item Each step triggers constraint propagation 
\end{itemize}
\item Different domains require/allow different methods
\end{itemize}
\end{frame}

\begin{frame}
  \frametitle{Constraint Programming is Different}
  \begin{itemize}
  \item Declarative Programming
    \begin{itemize}
    \item Concentrate on what you want
      \item Not how to get there
      \item Program != Algorithm
      \item Program = Model
    \end{itemize}
    \item Applied to Combinatorial Problems
      \begin{itemize}
        \item No complete polynomial algorithms known (exist?)
        \item CP less ad-hoc than heuristics
        \item Models can evolve
  \end{itemize}
  \end{itemize}
  \end{frame}
    
\begin{frame}
  \frametitle{A Subtractive Process}
  \begin{textblock}{4}(8,-3)
    \includegraphics[width=4cm]{../introduction/images/stages}
  \end{textblock}
  \vfill
  \begin{quote}
    ``Oh, bosh, as Mr. Ruskin says. Sculpture, per se, is the simplest thing in the world. All you have to do is to take a big chunk of marble and a hammer and chisel, make up your mind what you are about to create and chip off all the marble you don't want.''-Paris Gaulois.
  \end{quote}
  
  {\tiny Source: \url{https://quoteinvestigator.com/2014/06/22/chip-away/}}
\end{frame}

\begin{frame}
\frametitle{Basic Process}
\begin{center}
\begin{tikzpicture}
\node[shape=rectangle,fill=pantone127-4] (problem) at (2,5) {Problem};
\node (human) at (2,4) {Human};
\node[shape=rectangle,fill=pantone127-4] (model) at (2,3) {Model};
\node (solver) at (2,2) {Constraint Solver/Search};
\node[shape=rectangle,fill=pantone157-8] (solution) at (2,1) {Solution};
\draw[-] (problem) -- (human);
\draw[->] (human) -- (model);
\draw[-] (model) -- (solver);
\draw[->] (solver) -- (solution);
\end{tikzpicture}
\end{center}
\end{frame}

\begin{frame}
\frametitle{More Realistic}
\begin{center}
\begin{tikzpicture}[xscale=4.0]
\node[shape=rectangle,fill=pantone127-4] (problem) at (2,5) {Problem};
\node (human) at (2,4) {Human};
\node[shape=rectangle,fill=pantone127-4] (model) at (2,3) {Model};
\node (solver) at (2,2) {Constraint Solver/Search};
\node[shape=rectangle,fill=pantone24-8] (hangs) at (1,1) {Hangs};
\node[shape=rectangle,fill=pantone157-8] (solution) at (2,1) {Solution};
\node[shape=rectangle,fill=pantone24-8] (wrong) at (3,1) {Wrong Solution};
\draw[-] (problem) -- (human);
\draw[->] (human) -- (model);
\draw[-] (model) -- (solver);
\draw[->] (solver) -- (hangs);
\draw[->] (solver) -- (solution);
\draw[->] (solver) -- (wrong);
\draw[->] (hangs) -- (1,4) -- (human);
\draw[->] (wrong) -- (3,4) -- (human);
\end{tikzpicture}
\end{center}
\end{frame}

\begin{frame}
\frametitle{Dual Role of Model}
\begin{itemize}
\item Allows Human to Express Problem
\begin{itemize}
\item Close to Problem Domain
\item Constraints as Abstractions
\end{itemize}
\item Allows Solver to Execute
\begin{itemize}
\item Variables as Communication Mechanism
\item Constraints as Algorithms
\end{itemize}
\end{itemize}
\end{frame}

\begin{frame}
\frametitle{Modelling Frameworks}
\begin{itemize}
\item \href{https://www.minizinc.org/}{MiniZinc} (NICTA, Monash University, Australia)
\item \href{https://github.com/eomahony/Numberjack}{NumberJack} (Insight, Ireland) 
\item \href{https://www-users.york.ac.uk/peter.nightingale/savilerow//}{EssencePrime/SavilleRow} (UK) 
\item \href{https://cpmpy.readthedocs.io/en/latest/}{CPMpy} (KU Leuven)
\item Allow use of multiple back-end solvers
\item Compile model into variants for each solver
\item A priori solver independent model(CP, MIP, SAT)
\end{itemize}
\end{frame} 

\begin{frame}
\frametitle{Framework Process}
\begin{center}
\begin{tikzpicture}[xscale=2]
\node[shape=rectangle,fill=pantone127-4] (problem) at (2.5,5) {Problem};
\node (human) at (2.5,4) {Human};
\node[shape=rectangle,fill=pantone127-4] (model) at (2.5,3) {Model};
\node (compile) at (2.5,2) {Compile/Reformulate};
\node (solver1) at (1,1) {CP};
\node (solver2) at (2,1) {MIP};
\node (solver3) at (3,1) {SAT};
\node (solver4) at (4,1) {Other};
\node[shape=rectangle,fill=pantone157-8] (solution1) at (1,0) {Solution};
\node[shape=rectangle,fill=pantone157-8] (solution2) at (2,0) {Solution};
\node[shape=rectangle,fill=pantone157-8] (solution3) at (3,0) {Solution};
\node[shape=rectangle,fill=pantone157-8] (solution4) at (4,0) {Solution};
\draw[-] (problem) -- (human);
\draw[->] (human) -- (model);
\draw[-] (model) -- (compile);
\draw[-] (compile) -- (solver1);
\draw[-] (compile) -- (solver2);
\draw[-] (compile) -- (solver3);
\draw[-] (compile) -- (solver4);
\draw[->] (solver1) -- (solution1);
\draw[->] (solver2) -- (solution2);
\draw[->] (solver3) -- (solution3);
\draw[->] (solver4) -- (solution4);
\end{tikzpicture}
\end{center}
\end{frame}

% \begin{frame}
  % \frametitle{Do It Now!}
  % \begin{itemize}
  % \item Download and install Minizinc
  % \item \url{https://www.minizinc.org/}
  % \end{itemize}
% \end{frame}


\begin{frame}
\frametitle{Why use Puzzles as Examples?}
\begin{itemize}
\item Easy to understand the problem
\item Solvable by hand without specialized knowledge
\item Possible to compare automated to manual solving process
\end{itemize}

\begin{quote}
The puzzle, though inanimate, is presented as a solvable problem without lasting negative consequences, a very low-risk low-reward situation. By being a puzzle, the object is attempting to convince the user that it must be completed.
\end{quote}
{\tiny Source: \href{https://www.everydayrhetoric.com/2020/03/the-rhetoric-of-puzzles/}{Every Day Rhetoric}}
\end{frame}


\mode<all>{
\part{Basic Constraint Propagation}
}
\begin{frame}
\partpage
\end{frame}
\mode<all>{
\input{../sendmore/body}
}

\mode<all>{
\part{Global Constraints}
}
\begin{frame}
\partpage
\end{frame}
\mode<all>{
\input{../sudoku/body}
}

\clearpage
\mode<all>{
\part{Customizing Search}
}
\begin{frame}
\partpage
\end{frame}
\mode<all>{
\input{../nqueen/body}
}


%% \clearpage
%% \mode<all>{
%%   \part{Problem Decomposition}
%% }
%% \begin{frame}
%%   \partpage
%% \end{frame}
%% \mode<all>{
%%   \input{../party/body}
%%   }

%% \clearpage
%% \mode<all>{
%%   \part{More Global Constraints}
%% }
%% \begin{frame}
%%   \partpage
%% \end{frame}
%% \mode<all>{
%%   \input{../car/body}
%%   }

%% \clearpage
%% \mode<all>{
%%   \part{Methodology}
%% }
%% \begin{frame}
%%   \partpage
%% \end{frame}
%% \mode<all>{
%%   \input{../methodology/body}
%%   }

\clearpage
\mode<all>{
\part{What is missing?}
}
\begin{frame}
\partpage
\end{frame}

\begin{frame}
\frametitle{Many Specialized Topics}
\begin{itemize}
\item How to design efficient core engine
\item Hybrids with LP/MIP tools
\item Hybrids with SAT
\item Symmetry breaking
\item Use of MDD/BDD to encode sets of solutions
\item High level modelling tools
\item Debugging/visualization
\end{itemize}
\end{frame}

\begin{frame}
\frametitle{Reformulation}
\begin{itemize}
\item Just because the user has modelled it this way, it doesn't mean we have to solve it that way
\begin{itemize}
\item Replace some constraint(s) by other, equivalent constraints
\item Because we don't have that constraint in our system
\item For performance
\end{itemize}
\end{itemize}
\end{frame}

\begin{frame}
\frametitle{Learning}
\begin{itemize}
\item While solving the problem we can learn how to strengthen the model/search 
\begin{itemize}
\item Understand which constraints/method contribute to propagation and change schedule
\item Learn no-good constraints by explaining failure
\item Adapt search strategy based on search experience
\end{itemize}
\end{itemize}
\end{frame}

%% \begin{frame}
%% \frametitle{Refined Process}
%% \begin{center}
%% \begin{tikzpicture}
%% \node[shape=rectangle,fill=pantone127-4] (problem) at (3,7) {Problem};
%% \node (human) at (3,6) {Human};
%% \node[shape=rectangle,fill=pantone127-4] (model) at (3,5) {User Model};
%% \node (reformulation) at (3,4) {Reformulation};
%% \node[shape=rectangle,fill=pantone127-4] (implementation) at (3,3) {Implementation Model};
%% \node (solver) at (3,2) {Constraint Solver/Search};
%% \node[shape=rectangle,fill=pantone157-8] (solution) at (3,1) {Solution};
%% \draw[-] (problem) -- (human);
%% \draw[->] (human) -- (model);
%% \draw[-] (model) -- (reformulation);
%% \draw[->] (reformulation) -- (implementation);
%% \draw[-] (implementation) -- (solver);
%% \draw[->] (solver) -- (0,2) -- (0,3) -- (implementation);
%% \draw[->] (solver) -- (solution);
%% \end{tikzpicture}
%% \end{center}
%% \end{frame}

%\begin{frame}
%\frametitle{What is CP actually used for?}
%\url{http://hsimonis.wordpress.com}
%\includegraphics[width=10cm]{../introduction/images/blog}
%\end{frame}

%% \begin{frame}
%% \frametitle{Exercises}
%% \begin{itemize}
%% \item Install Constraint Solver of your Choice
%% \item MiniZinc (MiningZinc) will be used in later sessions
%% \item Run Sendmoremoney, Sudoku, Queens examples
%% \item Can you choose which propagator to use for alldifferent?
%% \item Can you use different search strategies?
%% \end{itemize}
%% \end{frame}

\begin{frame}
\frametitle{More Learning Resources}
\begin{itemize}
\item Survey of Methods, Resources, and Formats for Teaching Constraint Programming
\begin{itemize}
\item by Tejas Santanam, Helmut Simonis
\item \url{https://doi.org/10.48550/arXiv.2403.12717}
\item Based on survey of community for WTCP 2023
\item \url{https://hsimonis.github.io/WTCP2023/}
\end{itemize}
\end{itemize}
\end{frame}


%%   }

%% \clearpage
%% \mode<all>{
%%   \part{More Global Constraints}
%% }
%% \begin{frame}
%%   \partpage
%% \end{frame}
%% \mode<all>{
%%   \begin{frame}
\frametitle{Objectives}
\begin{itemize}
\item Overview of Core Constraint Programming
\item Three Main Concepts
\begin{itemize}
\item Constraint Propagation
\item Global Constraints
\item Customizing Search
\end{itemize}
%\item Get Some Experience with MiniZinc
\item Topics will be treated in more detail in later parts of the school
\item Based on Examples, not Formal Description
\end{itemize}
\end{frame}

\begin{frame}
\frametitle{Outline}
\begin{itemize}
\item Why Constraint Programming?
\item Constraint Propagation
\item Global Constraints
\item Customizing Search
%\item Choices in Modelling
%\item Decomposition
%\item More Global Constraints
%\item Methodology
%\item What is missing?
\end{itemize}
\end{frame}

% \begin{frame}
  % \frametitle{Using MiniZinc IDE}
  % \begin{itemize}
  % \item Developed in the Australian NICTA project
  % \item Maintained by Monash University
  % \item Modelling tool with multiple back-end solvers
  % \item Available from \url{https://www.minizinc.org/}
  % \end{itemize}
% \end{frame}


% \begin{frame}
% \frametitle{Examples in ECLiPSe}
% \begin{itemize}
% \item Open sourced constraint programming language
% \item Development goes back to 1985
% \item ECRC, ICL, IC-Parc, PTL, Cisco
% \item \url{https://eclipseclp.org/}
% \item Specialities
% \begin{itemize}
% \item Develop new solvers for specific domains
% \item Integration with MIP
% \end{itemize}
% \item Not included in bundled MiniZinc IDE
% \item Specialized visualization tools used here
  % \begin{itemize}
  % \item CP-Viz, Simonis et al. 2010
    % \end{itemize}
% \end{itemize}
% \end{frame}

\begin{frame}
\frametitle{Tutorial Based on ECLiPSe ELearning Course}
\begin{itemize}
\item Self-study course in constraint programming
\item Supported by Cisco Systems and Silicon Valley Community Foundation
\item Multi-media format, video lectures, slides, handout etc
\item {\small \url{https://eclipseclp.org/ELearning/index.html}}
\end{itemize}
\end{frame}

\begin{frame}
\frametitle{Also Part of CRT-AI Constraint Week}
\begin{itemize}
\item Annual one week course on CP and Optimization in Ireland
\item Part of national training program for PhD students in AI
\item \url{https://www.crt-ai.ie/}
\end{itemize}
\end{frame}



\begin{frame}
\frametitle{Constraint Programming - in a nutshell}
\begin{itemize}
\item Declarative description of problems with
\begin{itemize}
\item {\em Variables} which range over (finite) sets of values
\item {\em Constraints} over subsets of variables which restrict possible value combinations
\item A {\em solution} is a value assignment which satisfies all constraints
\end{itemize}

\item Constraint propagation/reasoning
\begin{itemize}
\item Removing inconsistent values for variables
\item Detect failure if constraint can not be satisfied
\item Interaction of constraints via shared variables
\item Incomplete
\end{itemize}

\item Search
\begin{itemize}
\item User controlled assignment of values to variables
\item Each step triggers constraint propagation 
\end{itemize}
\item Different domains require/allow different methods
\end{itemize}
\end{frame}

\begin{frame}
  \frametitle{Constraint Programming is Different}
  \begin{itemize}
  \item Declarative Programming
    \begin{itemize}
    \item Concentrate on what you want
      \item Not how to get there
      \item Program != Algorithm
      \item Program = Model
    \end{itemize}
    \item Applied to Combinatorial Problems
      \begin{itemize}
        \item No complete polynomial algorithms known (exist?)
        \item CP less ad-hoc than heuristics
        \item Models can evolve
  \end{itemize}
  \end{itemize}
  \end{frame}
    
\begin{frame}
  \frametitle{A Subtractive Process}
  \begin{textblock}{4}(8,-3)
    \includegraphics[width=4cm]{../introduction/images/stages}
  \end{textblock}
  \vfill
  \begin{quote}
    ``Oh, bosh, as Mr. Ruskin says. Sculpture, per se, is the simplest thing in the world. All you have to do is to take a big chunk of marble and a hammer and chisel, make up your mind what you are about to create and chip off all the marble you don't want.''-Paris Gaulois.
  \end{quote}
  
  {\tiny Source: \url{https://quoteinvestigator.com/2014/06/22/chip-away/}}
\end{frame}

\begin{frame}
\frametitle{Basic Process}
\begin{center}
\begin{tikzpicture}
\node[shape=rectangle,fill=pantone127-4] (problem) at (2,5) {Problem};
\node (human) at (2,4) {Human};
\node[shape=rectangle,fill=pantone127-4] (model) at (2,3) {Model};
\node (solver) at (2,2) {Constraint Solver/Search};
\node[shape=rectangle,fill=pantone157-8] (solution) at (2,1) {Solution};
\draw[-] (problem) -- (human);
\draw[->] (human) -- (model);
\draw[-] (model) -- (solver);
\draw[->] (solver) -- (solution);
\end{tikzpicture}
\end{center}
\end{frame}

\begin{frame}
\frametitle{More Realistic}
\begin{center}
\begin{tikzpicture}[xscale=4.0]
\node[shape=rectangle,fill=pantone127-4] (problem) at (2,5) {Problem};
\node (human) at (2,4) {Human};
\node[shape=rectangle,fill=pantone127-4] (model) at (2,3) {Model};
\node (solver) at (2,2) {Constraint Solver/Search};
\node[shape=rectangle,fill=pantone24-8] (hangs) at (1,1) {Hangs};
\node[shape=rectangle,fill=pantone157-8] (solution) at (2,1) {Solution};
\node[shape=rectangle,fill=pantone24-8] (wrong) at (3,1) {Wrong Solution};
\draw[-] (problem) -- (human);
\draw[->] (human) -- (model);
\draw[-] (model) -- (solver);
\draw[->] (solver) -- (hangs);
\draw[->] (solver) -- (solution);
\draw[->] (solver) -- (wrong);
\draw[->] (hangs) -- (1,4) -- (human);
\draw[->] (wrong) -- (3,4) -- (human);
\end{tikzpicture}
\end{center}
\end{frame}

\begin{frame}
\frametitle{Dual Role of Model}
\begin{itemize}
\item Allows Human to Express Problem
\begin{itemize}
\item Close to Problem Domain
\item Constraints as Abstractions
\end{itemize}
\item Allows Solver to Execute
\begin{itemize}
\item Variables as Communication Mechanism
\item Constraints as Algorithms
\end{itemize}
\end{itemize}
\end{frame}

\begin{frame}
\frametitle{Modelling Frameworks}
\begin{itemize}
\item \href{https://www.minizinc.org/}{MiniZinc} (NICTA, Monash University, Australia)
\item \href{https://github.com/eomahony/Numberjack}{NumberJack} (Insight, Ireland) 
\item \href{https://www-users.york.ac.uk/peter.nightingale/savilerow//}{EssencePrime/SavilleRow} (UK) 
\item \href{https://cpmpy.readthedocs.io/en/latest/}{CPMpy} (KU Leuven)
\item Allow use of multiple back-end solvers
\item Compile model into variants for each solver
\item A priori solver independent model(CP, MIP, SAT)
\end{itemize}
\end{frame} 

\begin{frame}
\frametitle{Framework Process}
\begin{center}
\begin{tikzpicture}[xscale=2]
\node[shape=rectangle,fill=pantone127-4] (problem) at (2.5,5) {Problem};
\node (human) at (2.5,4) {Human};
\node[shape=rectangle,fill=pantone127-4] (model) at (2.5,3) {Model};
\node (compile) at (2.5,2) {Compile/Reformulate};
\node (solver1) at (1,1) {CP};
\node (solver2) at (2,1) {MIP};
\node (solver3) at (3,1) {SAT};
\node (solver4) at (4,1) {Other};
\node[shape=rectangle,fill=pantone157-8] (solution1) at (1,0) {Solution};
\node[shape=rectangle,fill=pantone157-8] (solution2) at (2,0) {Solution};
\node[shape=rectangle,fill=pantone157-8] (solution3) at (3,0) {Solution};
\node[shape=rectangle,fill=pantone157-8] (solution4) at (4,0) {Solution};
\draw[-] (problem) -- (human);
\draw[->] (human) -- (model);
\draw[-] (model) -- (compile);
\draw[-] (compile) -- (solver1);
\draw[-] (compile) -- (solver2);
\draw[-] (compile) -- (solver3);
\draw[-] (compile) -- (solver4);
\draw[->] (solver1) -- (solution1);
\draw[->] (solver2) -- (solution2);
\draw[->] (solver3) -- (solution3);
\draw[->] (solver4) -- (solution4);
\end{tikzpicture}
\end{center}
\end{frame}

% \begin{frame}
  % \frametitle{Do It Now!}
  % \begin{itemize}
  % \item Download and install Minizinc
  % \item \url{https://www.minizinc.org/}
  % \end{itemize}
% \end{frame}


\begin{frame}
\frametitle{Why use Puzzles as Examples?}
\begin{itemize}
\item Easy to understand the problem
\item Solvable by hand without specialized knowledge
\item Possible to compare automated to manual solving process
\end{itemize}

\begin{quote}
The puzzle, though inanimate, is presented as a solvable problem without lasting negative consequences, a very low-risk low-reward situation. By being a puzzle, the object is attempting to convince the user that it must be completed.
\end{quote}
{\tiny Source: \href{https://www.everydayrhetoric.com/2020/03/the-rhetoric-of-puzzles/}{Every Day Rhetoric}}
\end{frame}


\mode<all>{
\part{Basic Constraint Propagation}
}
\begin{frame}
\partpage
\end{frame}
\mode<all>{
\input{../sendmore/body}
}

\mode<all>{
\part{Global Constraints}
}
\begin{frame}
\partpage
\end{frame}
\mode<all>{
\input{../sudoku/body}
}

\clearpage
\mode<all>{
\part{Customizing Search}
}
\begin{frame}
\partpage
\end{frame}
\mode<all>{
\input{../nqueen/body}
}


%% \clearpage
%% \mode<all>{
%%   \part{Problem Decomposition}
%% }
%% \begin{frame}
%%   \partpage
%% \end{frame}
%% \mode<all>{
%%   \input{../party/body}
%%   }

%% \clearpage
%% \mode<all>{
%%   \part{More Global Constraints}
%% }
%% \begin{frame}
%%   \partpage
%% \end{frame}
%% \mode<all>{
%%   \input{../car/body}
%%   }

%% \clearpage
%% \mode<all>{
%%   \part{Methodology}
%% }
%% \begin{frame}
%%   \partpage
%% \end{frame}
%% \mode<all>{
%%   \input{../methodology/body}
%%   }

\clearpage
\mode<all>{
\part{What is missing?}
}
\begin{frame}
\partpage
\end{frame}

\begin{frame}
\frametitle{Many Specialized Topics}
\begin{itemize}
\item How to design efficient core engine
\item Hybrids with LP/MIP tools
\item Hybrids with SAT
\item Symmetry breaking
\item Use of MDD/BDD to encode sets of solutions
\item High level modelling tools
\item Debugging/visualization
\end{itemize}
\end{frame}

\begin{frame}
\frametitle{Reformulation}
\begin{itemize}
\item Just because the user has modelled it this way, it doesn't mean we have to solve it that way
\begin{itemize}
\item Replace some constraint(s) by other, equivalent constraints
\item Because we don't have that constraint in our system
\item For performance
\end{itemize}
\end{itemize}
\end{frame}

\begin{frame}
\frametitle{Learning}
\begin{itemize}
\item While solving the problem we can learn how to strengthen the model/search 
\begin{itemize}
\item Understand which constraints/method contribute to propagation and change schedule
\item Learn no-good constraints by explaining failure
\item Adapt search strategy based on search experience
\end{itemize}
\end{itemize}
\end{frame}

%% \begin{frame}
%% \frametitle{Refined Process}
%% \begin{center}
%% \begin{tikzpicture}
%% \node[shape=rectangle,fill=pantone127-4] (problem) at (3,7) {Problem};
%% \node (human) at (3,6) {Human};
%% \node[shape=rectangle,fill=pantone127-4] (model) at (3,5) {User Model};
%% \node (reformulation) at (3,4) {Reformulation};
%% \node[shape=rectangle,fill=pantone127-4] (implementation) at (3,3) {Implementation Model};
%% \node (solver) at (3,2) {Constraint Solver/Search};
%% \node[shape=rectangle,fill=pantone157-8] (solution) at (3,1) {Solution};
%% \draw[-] (problem) -- (human);
%% \draw[->] (human) -- (model);
%% \draw[-] (model) -- (reformulation);
%% \draw[->] (reformulation) -- (implementation);
%% \draw[-] (implementation) -- (solver);
%% \draw[->] (solver) -- (0,2) -- (0,3) -- (implementation);
%% \draw[->] (solver) -- (solution);
%% \end{tikzpicture}
%% \end{center}
%% \end{frame}

%\begin{frame}
%\frametitle{What is CP actually used for?}
%\url{http://hsimonis.wordpress.com}
%\includegraphics[width=10cm]{../introduction/images/blog}
%\end{frame}

%% \begin{frame}
%% \frametitle{Exercises}
%% \begin{itemize}
%% \item Install Constraint Solver of your Choice
%% \item MiniZinc (MiningZinc) will be used in later sessions
%% \item Run Sendmoremoney, Sudoku, Queens examples
%% \item Can you choose which propagator to use for alldifferent?
%% \item Can you use different search strategies?
%% \end{itemize}
%% \end{frame}

\begin{frame}
\frametitle{More Learning Resources}
\begin{itemize}
\item Survey of Methods, Resources, and Formats for Teaching Constraint Programming
\begin{itemize}
\item by Tejas Santanam, Helmut Simonis
\item \url{https://doi.org/10.48550/arXiv.2403.12717}
\item Based on survey of community for WTCP 2023
\item \url{https://hsimonis.github.io/WTCP2023/}
\end{itemize}
\end{itemize}
\end{frame}


%%   }

%% \clearpage
%% \mode<all>{
%%   \part{Methodology}
%% }
%% \begin{frame}
%%   \partpage
%% \end{frame}
%% \mode<all>{
%%   \begin{frame}
\frametitle{Objectives}
\begin{itemize}
\item Overview of Core Constraint Programming
\item Three Main Concepts
\begin{itemize}
\item Constraint Propagation
\item Global Constraints
\item Customizing Search
\end{itemize}
%\item Get Some Experience with MiniZinc
\item Topics will be treated in more detail in later parts of the school
\item Based on Examples, not Formal Description
\end{itemize}
\end{frame}

\begin{frame}
\frametitle{Outline}
\begin{itemize}
\item Why Constraint Programming?
\item Constraint Propagation
\item Global Constraints
\item Customizing Search
%\item Choices in Modelling
%\item Decomposition
%\item More Global Constraints
%\item Methodology
%\item What is missing?
\end{itemize}
\end{frame}

% \begin{frame}
  % \frametitle{Using MiniZinc IDE}
  % \begin{itemize}
  % \item Developed in the Australian NICTA project
  % \item Maintained by Monash University
  % \item Modelling tool with multiple back-end solvers
  % \item Available from \url{https://www.minizinc.org/}
  % \end{itemize}
% \end{frame}


% \begin{frame}
% \frametitle{Examples in ECLiPSe}
% \begin{itemize}
% \item Open sourced constraint programming language
% \item Development goes back to 1985
% \item ECRC, ICL, IC-Parc, PTL, Cisco
% \item \url{https://eclipseclp.org/}
% \item Specialities
% \begin{itemize}
% \item Develop new solvers for specific domains
% \item Integration with MIP
% \end{itemize}
% \item Not included in bundled MiniZinc IDE
% \item Specialized visualization tools used here
  % \begin{itemize}
  % \item CP-Viz, Simonis et al. 2010
    % \end{itemize}
% \end{itemize}
% \end{frame}

\begin{frame}
\frametitle{Tutorial Based on ECLiPSe ELearning Course}
\begin{itemize}
\item Self-study course in constraint programming
\item Supported by Cisco Systems and Silicon Valley Community Foundation
\item Multi-media format, video lectures, slides, handout etc
\item {\small \url{https://eclipseclp.org/ELearning/index.html}}
\end{itemize}
\end{frame}

\begin{frame}
\frametitle{Also Part of CRT-AI Constraint Week}
\begin{itemize}
\item Annual one week course on CP and Optimization in Ireland
\item Part of national training program for PhD students in AI
\item \url{https://www.crt-ai.ie/}
\end{itemize}
\end{frame}



\begin{frame}
\frametitle{Constraint Programming - in a nutshell}
\begin{itemize}
\item Declarative description of problems with
\begin{itemize}
\item {\em Variables} which range over (finite) sets of values
\item {\em Constraints} over subsets of variables which restrict possible value combinations
\item A {\em solution} is a value assignment which satisfies all constraints
\end{itemize}

\item Constraint propagation/reasoning
\begin{itemize}
\item Removing inconsistent values for variables
\item Detect failure if constraint can not be satisfied
\item Interaction of constraints via shared variables
\item Incomplete
\end{itemize}

\item Search
\begin{itemize}
\item User controlled assignment of values to variables
\item Each step triggers constraint propagation 
\end{itemize}
\item Different domains require/allow different methods
\end{itemize}
\end{frame}

\begin{frame}
  \frametitle{Constraint Programming is Different}
  \begin{itemize}
  \item Declarative Programming
    \begin{itemize}
    \item Concentrate on what you want
      \item Not how to get there
      \item Program != Algorithm
      \item Program = Model
    \end{itemize}
    \item Applied to Combinatorial Problems
      \begin{itemize}
        \item No complete polynomial algorithms known (exist?)
        \item CP less ad-hoc than heuristics
        \item Models can evolve
  \end{itemize}
  \end{itemize}
  \end{frame}
    
\begin{frame}
  \frametitle{A Subtractive Process}
  \begin{textblock}{4}(8,-3)
    \includegraphics[width=4cm]{../introduction/images/stages}
  \end{textblock}
  \vfill
  \begin{quote}
    ``Oh, bosh, as Mr. Ruskin says. Sculpture, per se, is the simplest thing in the world. All you have to do is to take a big chunk of marble and a hammer and chisel, make up your mind what you are about to create and chip off all the marble you don't want.''-Paris Gaulois.
  \end{quote}
  
  {\tiny Source: \url{https://quoteinvestigator.com/2014/06/22/chip-away/}}
\end{frame}

\begin{frame}
\frametitle{Basic Process}
\begin{center}
\begin{tikzpicture}
\node[shape=rectangle,fill=pantone127-4] (problem) at (2,5) {Problem};
\node (human) at (2,4) {Human};
\node[shape=rectangle,fill=pantone127-4] (model) at (2,3) {Model};
\node (solver) at (2,2) {Constraint Solver/Search};
\node[shape=rectangle,fill=pantone157-8] (solution) at (2,1) {Solution};
\draw[-] (problem) -- (human);
\draw[->] (human) -- (model);
\draw[-] (model) -- (solver);
\draw[->] (solver) -- (solution);
\end{tikzpicture}
\end{center}
\end{frame}

\begin{frame}
\frametitle{More Realistic}
\begin{center}
\begin{tikzpicture}[xscale=4.0]
\node[shape=rectangle,fill=pantone127-4] (problem) at (2,5) {Problem};
\node (human) at (2,4) {Human};
\node[shape=rectangle,fill=pantone127-4] (model) at (2,3) {Model};
\node (solver) at (2,2) {Constraint Solver/Search};
\node[shape=rectangle,fill=pantone24-8] (hangs) at (1,1) {Hangs};
\node[shape=rectangle,fill=pantone157-8] (solution) at (2,1) {Solution};
\node[shape=rectangle,fill=pantone24-8] (wrong) at (3,1) {Wrong Solution};
\draw[-] (problem) -- (human);
\draw[->] (human) -- (model);
\draw[-] (model) -- (solver);
\draw[->] (solver) -- (hangs);
\draw[->] (solver) -- (solution);
\draw[->] (solver) -- (wrong);
\draw[->] (hangs) -- (1,4) -- (human);
\draw[->] (wrong) -- (3,4) -- (human);
\end{tikzpicture}
\end{center}
\end{frame}

\begin{frame}
\frametitle{Dual Role of Model}
\begin{itemize}
\item Allows Human to Express Problem
\begin{itemize}
\item Close to Problem Domain
\item Constraints as Abstractions
\end{itemize}
\item Allows Solver to Execute
\begin{itemize}
\item Variables as Communication Mechanism
\item Constraints as Algorithms
\end{itemize}
\end{itemize}
\end{frame}

\begin{frame}
\frametitle{Modelling Frameworks}
\begin{itemize}
\item \href{https://www.minizinc.org/}{MiniZinc} (NICTA, Monash University, Australia)
\item \href{https://github.com/eomahony/Numberjack}{NumberJack} (Insight, Ireland) 
\item \href{https://www-users.york.ac.uk/peter.nightingale/savilerow//}{EssencePrime/SavilleRow} (UK) 
\item \href{https://cpmpy.readthedocs.io/en/latest/}{CPMpy} (KU Leuven)
\item Allow use of multiple back-end solvers
\item Compile model into variants for each solver
\item A priori solver independent model(CP, MIP, SAT)
\end{itemize}
\end{frame} 

\begin{frame}
\frametitle{Framework Process}
\begin{center}
\begin{tikzpicture}[xscale=2]
\node[shape=rectangle,fill=pantone127-4] (problem) at (2.5,5) {Problem};
\node (human) at (2.5,4) {Human};
\node[shape=rectangle,fill=pantone127-4] (model) at (2.5,3) {Model};
\node (compile) at (2.5,2) {Compile/Reformulate};
\node (solver1) at (1,1) {CP};
\node (solver2) at (2,1) {MIP};
\node (solver3) at (3,1) {SAT};
\node (solver4) at (4,1) {Other};
\node[shape=rectangle,fill=pantone157-8] (solution1) at (1,0) {Solution};
\node[shape=rectangle,fill=pantone157-8] (solution2) at (2,0) {Solution};
\node[shape=rectangle,fill=pantone157-8] (solution3) at (3,0) {Solution};
\node[shape=rectangle,fill=pantone157-8] (solution4) at (4,0) {Solution};
\draw[-] (problem) -- (human);
\draw[->] (human) -- (model);
\draw[-] (model) -- (compile);
\draw[-] (compile) -- (solver1);
\draw[-] (compile) -- (solver2);
\draw[-] (compile) -- (solver3);
\draw[-] (compile) -- (solver4);
\draw[->] (solver1) -- (solution1);
\draw[->] (solver2) -- (solution2);
\draw[->] (solver3) -- (solution3);
\draw[->] (solver4) -- (solution4);
\end{tikzpicture}
\end{center}
\end{frame}

% \begin{frame}
  % \frametitle{Do It Now!}
  % \begin{itemize}
  % \item Download and install Minizinc
  % \item \url{https://www.minizinc.org/}
  % \end{itemize}
% \end{frame}


\begin{frame}
\frametitle{Why use Puzzles as Examples?}
\begin{itemize}
\item Easy to understand the problem
\item Solvable by hand without specialized knowledge
\item Possible to compare automated to manual solving process
\end{itemize}

\begin{quote}
The puzzle, though inanimate, is presented as a solvable problem without lasting negative consequences, a very low-risk low-reward situation. By being a puzzle, the object is attempting to convince the user that it must be completed.
\end{quote}
{\tiny Source: \href{https://www.everydayrhetoric.com/2020/03/the-rhetoric-of-puzzles/}{Every Day Rhetoric}}
\end{frame}


\mode<all>{
\part{Basic Constraint Propagation}
}
\begin{frame}
\partpage
\end{frame}
\mode<all>{
\input{../sendmore/body}
}

\mode<all>{
\part{Global Constraints}
}
\begin{frame}
\partpage
\end{frame}
\mode<all>{
\input{../sudoku/body}
}

\clearpage
\mode<all>{
\part{Customizing Search}
}
\begin{frame}
\partpage
\end{frame}
\mode<all>{
\input{../nqueen/body}
}


%% \clearpage
%% \mode<all>{
%%   \part{Problem Decomposition}
%% }
%% \begin{frame}
%%   \partpage
%% \end{frame}
%% \mode<all>{
%%   \input{../party/body}
%%   }

%% \clearpage
%% \mode<all>{
%%   \part{More Global Constraints}
%% }
%% \begin{frame}
%%   \partpage
%% \end{frame}
%% \mode<all>{
%%   \input{../car/body}
%%   }

%% \clearpage
%% \mode<all>{
%%   \part{Methodology}
%% }
%% \begin{frame}
%%   \partpage
%% \end{frame}
%% \mode<all>{
%%   \input{../methodology/body}
%%   }

\clearpage
\mode<all>{
\part{What is missing?}
}
\begin{frame}
\partpage
\end{frame}

\begin{frame}
\frametitle{Many Specialized Topics}
\begin{itemize}
\item How to design efficient core engine
\item Hybrids with LP/MIP tools
\item Hybrids with SAT
\item Symmetry breaking
\item Use of MDD/BDD to encode sets of solutions
\item High level modelling tools
\item Debugging/visualization
\end{itemize}
\end{frame}

\begin{frame}
\frametitle{Reformulation}
\begin{itemize}
\item Just because the user has modelled it this way, it doesn't mean we have to solve it that way
\begin{itemize}
\item Replace some constraint(s) by other, equivalent constraints
\item Because we don't have that constraint in our system
\item For performance
\end{itemize}
\end{itemize}
\end{frame}

\begin{frame}
\frametitle{Learning}
\begin{itemize}
\item While solving the problem we can learn how to strengthen the model/search 
\begin{itemize}
\item Understand which constraints/method contribute to propagation and change schedule
\item Learn no-good constraints by explaining failure
\item Adapt search strategy based on search experience
\end{itemize}
\end{itemize}
\end{frame}

%% \begin{frame}
%% \frametitle{Refined Process}
%% \begin{center}
%% \begin{tikzpicture}
%% \node[shape=rectangle,fill=pantone127-4] (problem) at (3,7) {Problem};
%% \node (human) at (3,6) {Human};
%% \node[shape=rectangle,fill=pantone127-4] (model) at (3,5) {User Model};
%% \node (reformulation) at (3,4) {Reformulation};
%% \node[shape=rectangle,fill=pantone127-4] (implementation) at (3,3) {Implementation Model};
%% \node (solver) at (3,2) {Constraint Solver/Search};
%% \node[shape=rectangle,fill=pantone157-8] (solution) at (3,1) {Solution};
%% \draw[-] (problem) -- (human);
%% \draw[->] (human) -- (model);
%% \draw[-] (model) -- (reformulation);
%% \draw[->] (reformulation) -- (implementation);
%% \draw[-] (implementation) -- (solver);
%% \draw[->] (solver) -- (0,2) -- (0,3) -- (implementation);
%% \draw[->] (solver) -- (solution);
%% \end{tikzpicture}
%% \end{center}
%% \end{frame}

%\begin{frame}
%\frametitle{What is CP actually used for?}
%\url{http://hsimonis.wordpress.com}
%\includegraphics[width=10cm]{../introduction/images/blog}
%\end{frame}

%% \begin{frame}
%% \frametitle{Exercises}
%% \begin{itemize}
%% \item Install Constraint Solver of your Choice
%% \item MiniZinc (MiningZinc) will be used in later sessions
%% \item Run Sendmoremoney, Sudoku, Queens examples
%% \item Can you choose which propagator to use for alldifferent?
%% \item Can you use different search strategies?
%% \end{itemize}
%% \end{frame}

\begin{frame}
\frametitle{More Learning Resources}
\begin{itemize}
\item Survey of Methods, Resources, and Formats for Teaching Constraint Programming
\begin{itemize}
\item by Tejas Santanam, Helmut Simonis
\item \url{https://doi.org/10.48550/arXiv.2403.12717}
\item Based on survey of community for WTCP 2023
\item \url{https://hsimonis.github.io/WTCP2023/}
\end{itemize}
\end{itemize}
\end{frame}


%%   }

\clearpage
\mode<all>{
\part{What is missing?}
}
\begin{frame}
\partpage
\end{frame}

\begin{frame}
\frametitle{Many Specialized Topics}
\begin{itemize}
\item How to design efficient core engine
\item Hybrids with LP/MIP tools
\item Hybrids with SAT
\item Symmetry breaking
\item Use of MDD/BDD to encode sets of solutions
\item High level modelling tools
\item Debugging/visualization
\end{itemize}
\end{frame}

\begin{frame}
\frametitle{Reformulation}
\begin{itemize}
\item Just because the user has modelled it this way, it doesn't mean we have to solve it that way
\begin{itemize}
\item Replace some constraint(s) by other, equivalent constraints
\item Because we don't have that constraint in our system
\item For performance
\end{itemize}
\end{itemize}
\end{frame}

\begin{frame}
\frametitle{Learning}
\begin{itemize}
\item While solving the problem we can learn how to strengthen the model/search 
\begin{itemize}
\item Understand which constraints/method contribute to propagation and change schedule
\item Learn no-good constraints by explaining failure
\item Adapt search strategy based on search experience
\end{itemize}
\end{itemize}
\end{frame}

%% \begin{frame}
%% \frametitle{Refined Process}
%% \begin{center}
%% \begin{tikzpicture}
%% \node[shape=rectangle,fill=pantone127-4] (problem) at (3,7) {Problem};
%% \node (human) at (3,6) {Human};
%% \node[shape=rectangle,fill=pantone127-4] (model) at (3,5) {User Model};
%% \node (reformulation) at (3,4) {Reformulation};
%% \node[shape=rectangle,fill=pantone127-4] (implementation) at (3,3) {Implementation Model};
%% \node (solver) at (3,2) {Constraint Solver/Search};
%% \node[shape=rectangle,fill=pantone157-8] (solution) at (3,1) {Solution};
%% \draw[-] (problem) -- (human);
%% \draw[->] (human) -- (model);
%% \draw[-] (model) -- (reformulation);
%% \draw[->] (reformulation) -- (implementation);
%% \draw[-] (implementation) -- (solver);
%% \draw[->] (solver) -- (0,2) -- (0,3) -- (implementation);
%% \draw[->] (solver) -- (solution);
%% \end{tikzpicture}
%% \end{center}
%% \end{frame}

%\begin{frame}
%\frametitle{What is CP actually used for?}
%\url{http://hsimonis.wordpress.com}
%\includegraphics[width=10cm]{../introduction/images/blog}
%\end{frame}

%% \begin{frame}
%% \frametitle{Exercises}
%% \begin{itemize}
%% \item Install Constraint Solver of your Choice
%% \item MiniZinc (MiningZinc) will be used in later sessions
%% \item Run Sendmoremoney, Sudoku, Queens examples
%% \item Can you choose which propagator to use for alldifferent?
%% \item Can you use different search strategies?
%% \end{itemize}
%% \end{frame}

\begin{frame}
\frametitle{More Learning Resources}
\begin{itemize}
\item Survey of Methods, Resources, and Formats for Teaching Constraint Programming
\begin{itemize}
\item by Tejas Santanam, Helmut Simonis
\item \url{https://doi.org/10.48550/arXiv.2403.12717}
\item Based on survey of community for WTCP 2023
\item \url{https://hsimonis.github.io/WTCP2023/}
\end{itemize}
\end{itemize}
\end{frame}


}


%% \clearpage
%% \mode<all>{
%%   \part{Problem Decomposition}
%% }
%% \begin{frame}
%%   \partpage
%% \end{frame}
%% \mode<all>{
%%   \begin{frame}
\frametitle{Objectives}
\begin{itemize}
\item Overview of Core Constraint Programming
\item Three Main Concepts
\begin{itemize}
\item Constraint Propagation
\item Global Constraints
\item Customizing Search
\end{itemize}
%\item Get Some Experience with MiniZinc
\item Topics will be treated in more detail in later parts of the school
\item Based on Examples, not Formal Description
\end{itemize}
\end{frame}

\begin{frame}
\frametitle{Outline}
\begin{itemize}
\item Why Constraint Programming?
\item Constraint Propagation
\item Global Constraints
\item Customizing Search
%\item Choices in Modelling
%\item Decomposition
%\item More Global Constraints
%\item Methodology
%\item What is missing?
\end{itemize}
\end{frame}

% \begin{frame}
  % \frametitle{Using MiniZinc IDE}
  % \begin{itemize}
  % \item Developed in the Australian NICTA project
  % \item Maintained by Monash University
  % \item Modelling tool with multiple back-end solvers
  % \item Available from \url{https://www.minizinc.org/}
  % \end{itemize}
% \end{frame}


% \begin{frame}
% \frametitle{Examples in ECLiPSe}
% \begin{itemize}
% \item Open sourced constraint programming language
% \item Development goes back to 1985
% \item ECRC, ICL, IC-Parc, PTL, Cisco
% \item \url{https://eclipseclp.org/}
% \item Specialities
% \begin{itemize}
% \item Develop new solvers for specific domains
% \item Integration with MIP
% \end{itemize}
% \item Not included in bundled MiniZinc IDE
% \item Specialized visualization tools used here
  % \begin{itemize}
  % \item CP-Viz, Simonis et al. 2010
    % \end{itemize}
% \end{itemize}
% \end{frame}

\begin{frame}
\frametitle{Tutorial Based on ECLiPSe ELearning Course}
\begin{itemize}
\item Self-study course in constraint programming
\item Supported by Cisco Systems and Silicon Valley Community Foundation
\item Multi-media format, video lectures, slides, handout etc
\item {\small \url{https://eclipseclp.org/ELearning/index.html}}
\end{itemize}
\end{frame}

\begin{frame}
\frametitle{Also Part of CRT-AI Constraint Week}
\begin{itemize}
\item Annual one week course on CP and Optimization in Ireland
\item Part of national training program for PhD students in AI
\item \url{https://www.crt-ai.ie/}
\end{itemize}
\end{frame}



\begin{frame}
\frametitle{Constraint Programming - in a nutshell}
\begin{itemize}
\item Declarative description of problems with
\begin{itemize}
\item {\em Variables} which range over (finite) sets of values
\item {\em Constraints} over subsets of variables which restrict possible value combinations
\item A {\em solution} is a value assignment which satisfies all constraints
\end{itemize}

\item Constraint propagation/reasoning
\begin{itemize}
\item Removing inconsistent values for variables
\item Detect failure if constraint can not be satisfied
\item Interaction of constraints via shared variables
\item Incomplete
\end{itemize}

\item Search
\begin{itemize}
\item User controlled assignment of values to variables
\item Each step triggers constraint propagation 
\end{itemize}
\item Different domains require/allow different methods
\end{itemize}
\end{frame}

\begin{frame}
  \frametitle{Constraint Programming is Different}
  \begin{itemize}
  \item Declarative Programming
    \begin{itemize}
    \item Concentrate on what you want
      \item Not how to get there
      \item Program != Algorithm
      \item Program = Model
    \end{itemize}
    \item Applied to Combinatorial Problems
      \begin{itemize}
        \item No complete polynomial algorithms known (exist?)
        \item CP less ad-hoc than heuristics
        \item Models can evolve
  \end{itemize}
  \end{itemize}
  \end{frame}
    
\begin{frame}
  \frametitle{A Subtractive Process}
  \begin{textblock}{4}(8,-3)
    \includegraphics[width=4cm]{../introduction/images/stages}
  \end{textblock}
  \vfill
  \begin{quote}
    ``Oh, bosh, as Mr. Ruskin says. Sculpture, per se, is the simplest thing in the world. All you have to do is to take a big chunk of marble and a hammer and chisel, make up your mind what you are about to create and chip off all the marble you don't want.''-Paris Gaulois.
  \end{quote}
  
  {\tiny Source: \url{https://quoteinvestigator.com/2014/06/22/chip-away/}}
\end{frame}

\begin{frame}
\frametitle{Basic Process}
\begin{center}
\begin{tikzpicture}
\node[shape=rectangle,fill=pantone127-4] (problem) at (2,5) {Problem};
\node (human) at (2,4) {Human};
\node[shape=rectangle,fill=pantone127-4] (model) at (2,3) {Model};
\node (solver) at (2,2) {Constraint Solver/Search};
\node[shape=rectangle,fill=pantone157-8] (solution) at (2,1) {Solution};
\draw[-] (problem) -- (human);
\draw[->] (human) -- (model);
\draw[-] (model) -- (solver);
\draw[->] (solver) -- (solution);
\end{tikzpicture}
\end{center}
\end{frame}

\begin{frame}
\frametitle{More Realistic}
\begin{center}
\begin{tikzpicture}[xscale=4.0]
\node[shape=rectangle,fill=pantone127-4] (problem) at (2,5) {Problem};
\node (human) at (2,4) {Human};
\node[shape=rectangle,fill=pantone127-4] (model) at (2,3) {Model};
\node (solver) at (2,2) {Constraint Solver/Search};
\node[shape=rectangle,fill=pantone24-8] (hangs) at (1,1) {Hangs};
\node[shape=rectangle,fill=pantone157-8] (solution) at (2,1) {Solution};
\node[shape=rectangle,fill=pantone24-8] (wrong) at (3,1) {Wrong Solution};
\draw[-] (problem) -- (human);
\draw[->] (human) -- (model);
\draw[-] (model) -- (solver);
\draw[->] (solver) -- (hangs);
\draw[->] (solver) -- (solution);
\draw[->] (solver) -- (wrong);
\draw[->] (hangs) -- (1,4) -- (human);
\draw[->] (wrong) -- (3,4) -- (human);
\end{tikzpicture}
\end{center}
\end{frame}

\begin{frame}
\frametitle{Dual Role of Model}
\begin{itemize}
\item Allows Human to Express Problem
\begin{itemize}
\item Close to Problem Domain
\item Constraints as Abstractions
\end{itemize}
\item Allows Solver to Execute
\begin{itemize}
\item Variables as Communication Mechanism
\item Constraints as Algorithms
\end{itemize}
\end{itemize}
\end{frame}

\begin{frame}
\frametitle{Modelling Frameworks}
\begin{itemize}
\item \href{https://www.minizinc.org/}{MiniZinc} (NICTA, Monash University, Australia)
\item \href{https://github.com/eomahony/Numberjack}{NumberJack} (Insight, Ireland) 
\item \href{https://www-users.york.ac.uk/peter.nightingale/savilerow//}{EssencePrime/SavilleRow} (UK) 
\item \href{https://cpmpy.readthedocs.io/en/latest/}{CPMpy} (KU Leuven)
\item Allow use of multiple back-end solvers
\item Compile model into variants for each solver
\item A priori solver independent model(CP, MIP, SAT)
\end{itemize}
\end{frame} 

\begin{frame}
\frametitle{Framework Process}
\begin{center}
\begin{tikzpicture}[xscale=2]
\node[shape=rectangle,fill=pantone127-4] (problem) at (2.5,5) {Problem};
\node (human) at (2.5,4) {Human};
\node[shape=rectangle,fill=pantone127-4] (model) at (2.5,3) {Model};
\node (compile) at (2.5,2) {Compile/Reformulate};
\node (solver1) at (1,1) {CP};
\node (solver2) at (2,1) {MIP};
\node (solver3) at (3,1) {SAT};
\node (solver4) at (4,1) {Other};
\node[shape=rectangle,fill=pantone157-8] (solution1) at (1,0) {Solution};
\node[shape=rectangle,fill=pantone157-8] (solution2) at (2,0) {Solution};
\node[shape=rectangle,fill=pantone157-8] (solution3) at (3,0) {Solution};
\node[shape=rectangle,fill=pantone157-8] (solution4) at (4,0) {Solution};
\draw[-] (problem) -- (human);
\draw[->] (human) -- (model);
\draw[-] (model) -- (compile);
\draw[-] (compile) -- (solver1);
\draw[-] (compile) -- (solver2);
\draw[-] (compile) -- (solver3);
\draw[-] (compile) -- (solver4);
\draw[->] (solver1) -- (solution1);
\draw[->] (solver2) -- (solution2);
\draw[->] (solver3) -- (solution3);
\draw[->] (solver4) -- (solution4);
\end{tikzpicture}
\end{center}
\end{frame}

% \begin{frame}
  % \frametitle{Do It Now!}
  % \begin{itemize}
  % \item Download and install Minizinc
  % \item \url{https://www.minizinc.org/}
  % \end{itemize}
% \end{frame}


\begin{frame}
\frametitle{Why use Puzzles as Examples?}
\begin{itemize}
\item Easy to understand the problem
\item Solvable by hand without specialized knowledge
\item Possible to compare automated to manual solving process
\end{itemize}

\begin{quote}
The puzzle, though inanimate, is presented as a solvable problem without lasting negative consequences, a very low-risk low-reward situation. By being a puzzle, the object is attempting to convince the user that it must be completed.
\end{quote}
{\tiny Source: \href{https://www.everydayrhetoric.com/2020/03/the-rhetoric-of-puzzles/}{Every Day Rhetoric}}
\end{frame}


\mode<all>{
\part{Basic Constraint Propagation}
}
\begin{frame}
\partpage
\end{frame}
\mode<all>{
\begin{frame}
\frametitle{Objectives}
\begin{itemize}
\item Overview of Core Constraint Programming
\item Three Main Concepts
\begin{itemize}
\item Constraint Propagation
\item Global Constraints
\item Customizing Search
\end{itemize}
%\item Get Some Experience with MiniZinc
\item Topics will be treated in more detail in later parts of the school
\item Based on Examples, not Formal Description
\end{itemize}
\end{frame}

\begin{frame}
\frametitle{Outline}
\begin{itemize}
\item Why Constraint Programming?
\item Constraint Propagation
\item Global Constraints
\item Customizing Search
%\item Choices in Modelling
%\item Decomposition
%\item More Global Constraints
%\item Methodology
%\item What is missing?
\end{itemize}
\end{frame}

% \begin{frame}
  % \frametitle{Using MiniZinc IDE}
  % \begin{itemize}
  % \item Developed in the Australian NICTA project
  % \item Maintained by Monash University
  % \item Modelling tool with multiple back-end solvers
  % \item Available from \url{https://www.minizinc.org/}
  % \end{itemize}
% \end{frame}


% \begin{frame}
% \frametitle{Examples in ECLiPSe}
% \begin{itemize}
% \item Open sourced constraint programming language
% \item Development goes back to 1985
% \item ECRC, ICL, IC-Parc, PTL, Cisco
% \item \url{https://eclipseclp.org/}
% \item Specialities
% \begin{itemize}
% \item Develop new solvers for specific domains
% \item Integration with MIP
% \end{itemize}
% \item Not included in bundled MiniZinc IDE
% \item Specialized visualization tools used here
  % \begin{itemize}
  % \item CP-Viz, Simonis et al. 2010
    % \end{itemize}
% \end{itemize}
% \end{frame}

\begin{frame}
\frametitle{Tutorial Based on ECLiPSe ELearning Course}
\begin{itemize}
\item Self-study course in constraint programming
\item Supported by Cisco Systems and Silicon Valley Community Foundation
\item Multi-media format, video lectures, slides, handout etc
\item {\small \url{https://eclipseclp.org/ELearning/index.html}}
\end{itemize}
\end{frame}

\begin{frame}
\frametitle{Also Part of CRT-AI Constraint Week}
\begin{itemize}
\item Annual one week course on CP and Optimization in Ireland
\item Part of national training program for PhD students in AI
\item \url{https://www.crt-ai.ie/}
\end{itemize}
\end{frame}



\begin{frame}
\frametitle{Constraint Programming - in a nutshell}
\begin{itemize}
\item Declarative description of problems with
\begin{itemize}
\item {\em Variables} which range over (finite) sets of values
\item {\em Constraints} over subsets of variables which restrict possible value combinations
\item A {\em solution} is a value assignment which satisfies all constraints
\end{itemize}

\item Constraint propagation/reasoning
\begin{itemize}
\item Removing inconsistent values for variables
\item Detect failure if constraint can not be satisfied
\item Interaction of constraints via shared variables
\item Incomplete
\end{itemize}

\item Search
\begin{itemize}
\item User controlled assignment of values to variables
\item Each step triggers constraint propagation 
\end{itemize}
\item Different domains require/allow different methods
\end{itemize}
\end{frame}

\begin{frame}
  \frametitle{Constraint Programming is Different}
  \begin{itemize}
  \item Declarative Programming
    \begin{itemize}
    \item Concentrate on what you want
      \item Not how to get there
      \item Program != Algorithm
      \item Program = Model
    \end{itemize}
    \item Applied to Combinatorial Problems
      \begin{itemize}
        \item No complete polynomial algorithms known (exist?)
        \item CP less ad-hoc than heuristics
        \item Models can evolve
  \end{itemize}
  \end{itemize}
  \end{frame}
    
\begin{frame}
  \frametitle{A Subtractive Process}
  \begin{textblock}{4}(8,-3)
    \includegraphics[width=4cm]{../introduction/images/stages}
  \end{textblock}
  \vfill
  \begin{quote}
    ``Oh, bosh, as Mr. Ruskin says. Sculpture, per se, is the simplest thing in the world. All you have to do is to take a big chunk of marble and a hammer and chisel, make up your mind what you are about to create and chip off all the marble you don't want.''-Paris Gaulois.
  \end{quote}
  
  {\tiny Source: \url{https://quoteinvestigator.com/2014/06/22/chip-away/}}
\end{frame}

\begin{frame}
\frametitle{Basic Process}
\begin{center}
\begin{tikzpicture}
\node[shape=rectangle,fill=pantone127-4] (problem) at (2,5) {Problem};
\node (human) at (2,4) {Human};
\node[shape=rectangle,fill=pantone127-4] (model) at (2,3) {Model};
\node (solver) at (2,2) {Constraint Solver/Search};
\node[shape=rectangle,fill=pantone157-8] (solution) at (2,1) {Solution};
\draw[-] (problem) -- (human);
\draw[->] (human) -- (model);
\draw[-] (model) -- (solver);
\draw[->] (solver) -- (solution);
\end{tikzpicture}
\end{center}
\end{frame}

\begin{frame}
\frametitle{More Realistic}
\begin{center}
\begin{tikzpicture}[xscale=4.0]
\node[shape=rectangle,fill=pantone127-4] (problem) at (2,5) {Problem};
\node (human) at (2,4) {Human};
\node[shape=rectangle,fill=pantone127-4] (model) at (2,3) {Model};
\node (solver) at (2,2) {Constraint Solver/Search};
\node[shape=rectangle,fill=pantone24-8] (hangs) at (1,1) {Hangs};
\node[shape=rectangle,fill=pantone157-8] (solution) at (2,1) {Solution};
\node[shape=rectangle,fill=pantone24-8] (wrong) at (3,1) {Wrong Solution};
\draw[-] (problem) -- (human);
\draw[->] (human) -- (model);
\draw[-] (model) -- (solver);
\draw[->] (solver) -- (hangs);
\draw[->] (solver) -- (solution);
\draw[->] (solver) -- (wrong);
\draw[->] (hangs) -- (1,4) -- (human);
\draw[->] (wrong) -- (3,4) -- (human);
\end{tikzpicture}
\end{center}
\end{frame}

\begin{frame}
\frametitle{Dual Role of Model}
\begin{itemize}
\item Allows Human to Express Problem
\begin{itemize}
\item Close to Problem Domain
\item Constraints as Abstractions
\end{itemize}
\item Allows Solver to Execute
\begin{itemize}
\item Variables as Communication Mechanism
\item Constraints as Algorithms
\end{itemize}
\end{itemize}
\end{frame}

\begin{frame}
\frametitle{Modelling Frameworks}
\begin{itemize}
\item \href{https://www.minizinc.org/}{MiniZinc} (NICTA, Monash University, Australia)
\item \href{https://github.com/eomahony/Numberjack}{NumberJack} (Insight, Ireland) 
\item \href{https://www-users.york.ac.uk/peter.nightingale/savilerow//}{EssencePrime/SavilleRow} (UK) 
\item \href{https://cpmpy.readthedocs.io/en/latest/}{CPMpy} (KU Leuven)
\item Allow use of multiple back-end solvers
\item Compile model into variants for each solver
\item A priori solver independent model(CP, MIP, SAT)
\end{itemize}
\end{frame} 

\begin{frame}
\frametitle{Framework Process}
\begin{center}
\begin{tikzpicture}[xscale=2]
\node[shape=rectangle,fill=pantone127-4] (problem) at (2.5,5) {Problem};
\node (human) at (2.5,4) {Human};
\node[shape=rectangle,fill=pantone127-4] (model) at (2.5,3) {Model};
\node (compile) at (2.5,2) {Compile/Reformulate};
\node (solver1) at (1,1) {CP};
\node (solver2) at (2,1) {MIP};
\node (solver3) at (3,1) {SAT};
\node (solver4) at (4,1) {Other};
\node[shape=rectangle,fill=pantone157-8] (solution1) at (1,0) {Solution};
\node[shape=rectangle,fill=pantone157-8] (solution2) at (2,0) {Solution};
\node[shape=rectangle,fill=pantone157-8] (solution3) at (3,0) {Solution};
\node[shape=rectangle,fill=pantone157-8] (solution4) at (4,0) {Solution};
\draw[-] (problem) -- (human);
\draw[->] (human) -- (model);
\draw[-] (model) -- (compile);
\draw[-] (compile) -- (solver1);
\draw[-] (compile) -- (solver2);
\draw[-] (compile) -- (solver3);
\draw[-] (compile) -- (solver4);
\draw[->] (solver1) -- (solution1);
\draw[->] (solver2) -- (solution2);
\draw[->] (solver3) -- (solution3);
\draw[->] (solver4) -- (solution4);
\end{tikzpicture}
\end{center}
\end{frame}

% \begin{frame}
  % \frametitle{Do It Now!}
  % \begin{itemize}
  % \item Download and install Minizinc
  % \item \url{https://www.minizinc.org/}
  % \end{itemize}
% \end{frame}


\begin{frame}
\frametitle{Why use Puzzles as Examples?}
\begin{itemize}
\item Easy to understand the problem
\item Solvable by hand without specialized knowledge
\item Possible to compare automated to manual solving process
\end{itemize}

\begin{quote}
The puzzle, though inanimate, is presented as a solvable problem without lasting negative consequences, a very low-risk low-reward situation. By being a puzzle, the object is attempting to convince the user that it must be completed.
\end{quote}
{\tiny Source: \href{https://www.everydayrhetoric.com/2020/03/the-rhetoric-of-puzzles/}{Every Day Rhetoric}}
\end{frame}


\mode<all>{
\part{Basic Constraint Propagation}
}
\begin{frame}
\partpage
\end{frame}
\mode<all>{
\input{../sendmore/body}
}

\mode<all>{
\part{Global Constraints}
}
\begin{frame}
\partpage
\end{frame}
\mode<all>{
\input{../sudoku/body}
}

\clearpage
\mode<all>{
\part{Customizing Search}
}
\begin{frame}
\partpage
\end{frame}
\mode<all>{
\input{../nqueen/body}
}


%% \clearpage
%% \mode<all>{
%%   \part{Problem Decomposition}
%% }
%% \begin{frame}
%%   \partpage
%% \end{frame}
%% \mode<all>{
%%   \input{../party/body}
%%   }

%% \clearpage
%% \mode<all>{
%%   \part{More Global Constraints}
%% }
%% \begin{frame}
%%   \partpage
%% \end{frame}
%% \mode<all>{
%%   \input{../car/body}
%%   }

%% \clearpage
%% \mode<all>{
%%   \part{Methodology}
%% }
%% \begin{frame}
%%   \partpage
%% \end{frame}
%% \mode<all>{
%%   \input{../methodology/body}
%%   }

\clearpage
\mode<all>{
\part{What is missing?}
}
\begin{frame}
\partpage
\end{frame}

\begin{frame}
\frametitle{Many Specialized Topics}
\begin{itemize}
\item How to design efficient core engine
\item Hybrids with LP/MIP tools
\item Hybrids with SAT
\item Symmetry breaking
\item Use of MDD/BDD to encode sets of solutions
\item High level modelling tools
\item Debugging/visualization
\end{itemize}
\end{frame}

\begin{frame}
\frametitle{Reformulation}
\begin{itemize}
\item Just because the user has modelled it this way, it doesn't mean we have to solve it that way
\begin{itemize}
\item Replace some constraint(s) by other, equivalent constraints
\item Because we don't have that constraint in our system
\item For performance
\end{itemize}
\end{itemize}
\end{frame}

\begin{frame}
\frametitle{Learning}
\begin{itemize}
\item While solving the problem we can learn how to strengthen the model/search 
\begin{itemize}
\item Understand which constraints/method contribute to propagation and change schedule
\item Learn no-good constraints by explaining failure
\item Adapt search strategy based on search experience
\end{itemize}
\end{itemize}
\end{frame}

%% \begin{frame}
%% \frametitle{Refined Process}
%% \begin{center}
%% \begin{tikzpicture}
%% \node[shape=rectangle,fill=pantone127-4] (problem) at (3,7) {Problem};
%% \node (human) at (3,6) {Human};
%% \node[shape=rectangle,fill=pantone127-4] (model) at (3,5) {User Model};
%% \node (reformulation) at (3,4) {Reformulation};
%% \node[shape=rectangle,fill=pantone127-4] (implementation) at (3,3) {Implementation Model};
%% \node (solver) at (3,2) {Constraint Solver/Search};
%% \node[shape=rectangle,fill=pantone157-8] (solution) at (3,1) {Solution};
%% \draw[-] (problem) -- (human);
%% \draw[->] (human) -- (model);
%% \draw[-] (model) -- (reformulation);
%% \draw[->] (reformulation) -- (implementation);
%% \draw[-] (implementation) -- (solver);
%% \draw[->] (solver) -- (0,2) -- (0,3) -- (implementation);
%% \draw[->] (solver) -- (solution);
%% \end{tikzpicture}
%% \end{center}
%% \end{frame}

%\begin{frame}
%\frametitle{What is CP actually used for?}
%\url{http://hsimonis.wordpress.com}
%\includegraphics[width=10cm]{../introduction/images/blog}
%\end{frame}

%% \begin{frame}
%% \frametitle{Exercises}
%% \begin{itemize}
%% \item Install Constraint Solver of your Choice
%% \item MiniZinc (MiningZinc) will be used in later sessions
%% \item Run Sendmoremoney, Sudoku, Queens examples
%% \item Can you choose which propagator to use for alldifferent?
%% \item Can you use different search strategies?
%% \end{itemize}
%% \end{frame}

\begin{frame}
\frametitle{More Learning Resources}
\begin{itemize}
\item Survey of Methods, Resources, and Formats for Teaching Constraint Programming
\begin{itemize}
\item by Tejas Santanam, Helmut Simonis
\item \url{https://doi.org/10.48550/arXiv.2403.12717}
\item Based on survey of community for WTCP 2023
\item \url{https://hsimonis.github.io/WTCP2023/}
\end{itemize}
\end{itemize}
\end{frame}


}

\mode<all>{
\part{Global Constraints}
}
\begin{frame}
\partpage
\end{frame}
\mode<all>{
\begin{frame}
\frametitle{Objectives}
\begin{itemize}
\item Overview of Core Constraint Programming
\item Three Main Concepts
\begin{itemize}
\item Constraint Propagation
\item Global Constraints
\item Customizing Search
\end{itemize}
%\item Get Some Experience with MiniZinc
\item Topics will be treated in more detail in later parts of the school
\item Based on Examples, not Formal Description
\end{itemize}
\end{frame}

\begin{frame}
\frametitle{Outline}
\begin{itemize}
\item Why Constraint Programming?
\item Constraint Propagation
\item Global Constraints
\item Customizing Search
%\item Choices in Modelling
%\item Decomposition
%\item More Global Constraints
%\item Methodology
%\item What is missing?
\end{itemize}
\end{frame}

% \begin{frame}
  % \frametitle{Using MiniZinc IDE}
  % \begin{itemize}
  % \item Developed in the Australian NICTA project
  % \item Maintained by Monash University
  % \item Modelling tool with multiple back-end solvers
  % \item Available from \url{https://www.minizinc.org/}
  % \end{itemize}
% \end{frame}


% \begin{frame}
% \frametitle{Examples in ECLiPSe}
% \begin{itemize}
% \item Open sourced constraint programming language
% \item Development goes back to 1985
% \item ECRC, ICL, IC-Parc, PTL, Cisco
% \item \url{https://eclipseclp.org/}
% \item Specialities
% \begin{itemize}
% \item Develop new solvers for specific domains
% \item Integration with MIP
% \end{itemize}
% \item Not included in bundled MiniZinc IDE
% \item Specialized visualization tools used here
  % \begin{itemize}
  % \item CP-Viz, Simonis et al. 2010
    % \end{itemize}
% \end{itemize}
% \end{frame}

\begin{frame}
\frametitle{Tutorial Based on ECLiPSe ELearning Course}
\begin{itemize}
\item Self-study course in constraint programming
\item Supported by Cisco Systems and Silicon Valley Community Foundation
\item Multi-media format, video lectures, slides, handout etc
\item {\small \url{https://eclipseclp.org/ELearning/index.html}}
\end{itemize}
\end{frame}

\begin{frame}
\frametitle{Also Part of CRT-AI Constraint Week}
\begin{itemize}
\item Annual one week course on CP and Optimization in Ireland
\item Part of national training program for PhD students in AI
\item \url{https://www.crt-ai.ie/}
\end{itemize}
\end{frame}



\begin{frame}
\frametitle{Constraint Programming - in a nutshell}
\begin{itemize}
\item Declarative description of problems with
\begin{itemize}
\item {\em Variables} which range over (finite) sets of values
\item {\em Constraints} over subsets of variables which restrict possible value combinations
\item A {\em solution} is a value assignment which satisfies all constraints
\end{itemize}

\item Constraint propagation/reasoning
\begin{itemize}
\item Removing inconsistent values for variables
\item Detect failure if constraint can not be satisfied
\item Interaction of constraints via shared variables
\item Incomplete
\end{itemize}

\item Search
\begin{itemize}
\item User controlled assignment of values to variables
\item Each step triggers constraint propagation 
\end{itemize}
\item Different domains require/allow different methods
\end{itemize}
\end{frame}

\begin{frame}
  \frametitle{Constraint Programming is Different}
  \begin{itemize}
  \item Declarative Programming
    \begin{itemize}
    \item Concentrate on what you want
      \item Not how to get there
      \item Program != Algorithm
      \item Program = Model
    \end{itemize}
    \item Applied to Combinatorial Problems
      \begin{itemize}
        \item No complete polynomial algorithms known (exist?)
        \item CP less ad-hoc than heuristics
        \item Models can evolve
  \end{itemize}
  \end{itemize}
  \end{frame}
    
\begin{frame}
  \frametitle{A Subtractive Process}
  \begin{textblock}{4}(8,-3)
    \includegraphics[width=4cm]{../introduction/images/stages}
  \end{textblock}
  \vfill
  \begin{quote}
    ``Oh, bosh, as Mr. Ruskin says. Sculpture, per se, is the simplest thing in the world. All you have to do is to take a big chunk of marble and a hammer and chisel, make up your mind what you are about to create and chip off all the marble you don't want.''-Paris Gaulois.
  \end{quote}
  
  {\tiny Source: \url{https://quoteinvestigator.com/2014/06/22/chip-away/}}
\end{frame}

\begin{frame}
\frametitle{Basic Process}
\begin{center}
\begin{tikzpicture}
\node[shape=rectangle,fill=pantone127-4] (problem) at (2,5) {Problem};
\node (human) at (2,4) {Human};
\node[shape=rectangle,fill=pantone127-4] (model) at (2,3) {Model};
\node (solver) at (2,2) {Constraint Solver/Search};
\node[shape=rectangle,fill=pantone157-8] (solution) at (2,1) {Solution};
\draw[-] (problem) -- (human);
\draw[->] (human) -- (model);
\draw[-] (model) -- (solver);
\draw[->] (solver) -- (solution);
\end{tikzpicture}
\end{center}
\end{frame}

\begin{frame}
\frametitle{More Realistic}
\begin{center}
\begin{tikzpicture}[xscale=4.0]
\node[shape=rectangle,fill=pantone127-4] (problem) at (2,5) {Problem};
\node (human) at (2,4) {Human};
\node[shape=rectangle,fill=pantone127-4] (model) at (2,3) {Model};
\node (solver) at (2,2) {Constraint Solver/Search};
\node[shape=rectangle,fill=pantone24-8] (hangs) at (1,1) {Hangs};
\node[shape=rectangle,fill=pantone157-8] (solution) at (2,1) {Solution};
\node[shape=rectangle,fill=pantone24-8] (wrong) at (3,1) {Wrong Solution};
\draw[-] (problem) -- (human);
\draw[->] (human) -- (model);
\draw[-] (model) -- (solver);
\draw[->] (solver) -- (hangs);
\draw[->] (solver) -- (solution);
\draw[->] (solver) -- (wrong);
\draw[->] (hangs) -- (1,4) -- (human);
\draw[->] (wrong) -- (3,4) -- (human);
\end{tikzpicture}
\end{center}
\end{frame}

\begin{frame}
\frametitle{Dual Role of Model}
\begin{itemize}
\item Allows Human to Express Problem
\begin{itemize}
\item Close to Problem Domain
\item Constraints as Abstractions
\end{itemize}
\item Allows Solver to Execute
\begin{itemize}
\item Variables as Communication Mechanism
\item Constraints as Algorithms
\end{itemize}
\end{itemize}
\end{frame}

\begin{frame}
\frametitle{Modelling Frameworks}
\begin{itemize}
\item \href{https://www.minizinc.org/}{MiniZinc} (NICTA, Monash University, Australia)
\item \href{https://github.com/eomahony/Numberjack}{NumberJack} (Insight, Ireland) 
\item \href{https://www-users.york.ac.uk/peter.nightingale/savilerow//}{EssencePrime/SavilleRow} (UK) 
\item \href{https://cpmpy.readthedocs.io/en/latest/}{CPMpy} (KU Leuven)
\item Allow use of multiple back-end solvers
\item Compile model into variants for each solver
\item A priori solver independent model(CP, MIP, SAT)
\end{itemize}
\end{frame} 

\begin{frame}
\frametitle{Framework Process}
\begin{center}
\begin{tikzpicture}[xscale=2]
\node[shape=rectangle,fill=pantone127-4] (problem) at (2.5,5) {Problem};
\node (human) at (2.5,4) {Human};
\node[shape=rectangle,fill=pantone127-4] (model) at (2.5,3) {Model};
\node (compile) at (2.5,2) {Compile/Reformulate};
\node (solver1) at (1,1) {CP};
\node (solver2) at (2,1) {MIP};
\node (solver3) at (3,1) {SAT};
\node (solver4) at (4,1) {Other};
\node[shape=rectangle,fill=pantone157-8] (solution1) at (1,0) {Solution};
\node[shape=rectangle,fill=pantone157-8] (solution2) at (2,0) {Solution};
\node[shape=rectangle,fill=pantone157-8] (solution3) at (3,0) {Solution};
\node[shape=rectangle,fill=pantone157-8] (solution4) at (4,0) {Solution};
\draw[-] (problem) -- (human);
\draw[->] (human) -- (model);
\draw[-] (model) -- (compile);
\draw[-] (compile) -- (solver1);
\draw[-] (compile) -- (solver2);
\draw[-] (compile) -- (solver3);
\draw[-] (compile) -- (solver4);
\draw[->] (solver1) -- (solution1);
\draw[->] (solver2) -- (solution2);
\draw[->] (solver3) -- (solution3);
\draw[->] (solver4) -- (solution4);
\end{tikzpicture}
\end{center}
\end{frame}

% \begin{frame}
  % \frametitle{Do It Now!}
  % \begin{itemize}
  % \item Download and install Minizinc
  % \item \url{https://www.minizinc.org/}
  % \end{itemize}
% \end{frame}


\begin{frame}
\frametitle{Why use Puzzles as Examples?}
\begin{itemize}
\item Easy to understand the problem
\item Solvable by hand without specialized knowledge
\item Possible to compare automated to manual solving process
\end{itemize}

\begin{quote}
The puzzle, though inanimate, is presented as a solvable problem without lasting negative consequences, a very low-risk low-reward situation. By being a puzzle, the object is attempting to convince the user that it must be completed.
\end{quote}
{\tiny Source: \href{https://www.everydayrhetoric.com/2020/03/the-rhetoric-of-puzzles/}{Every Day Rhetoric}}
\end{frame}


\mode<all>{
\part{Basic Constraint Propagation}
}
\begin{frame}
\partpage
\end{frame}
\mode<all>{
\input{../sendmore/body}
}

\mode<all>{
\part{Global Constraints}
}
\begin{frame}
\partpage
\end{frame}
\mode<all>{
\input{../sudoku/body}
}

\clearpage
\mode<all>{
\part{Customizing Search}
}
\begin{frame}
\partpage
\end{frame}
\mode<all>{
\input{../nqueen/body}
}


%% \clearpage
%% \mode<all>{
%%   \part{Problem Decomposition}
%% }
%% \begin{frame}
%%   \partpage
%% \end{frame}
%% \mode<all>{
%%   \input{../party/body}
%%   }

%% \clearpage
%% \mode<all>{
%%   \part{More Global Constraints}
%% }
%% \begin{frame}
%%   \partpage
%% \end{frame}
%% \mode<all>{
%%   \input{../car/body}
%%   }

%% \clearpage
%% \mode<all>{
%%   \part{Methodology}
%% }
%% \begin{frame}
%%   \partpage
%% \end{frame}
%% \mode<all>{
%%   \input{../methodology/body}
%%   }

\clearpage
\mode<all>{
\part{What is missing?}
}
\begin{frame}
\partpage
\end{frame}

\begin{frame}
\frametitle{Many Specialized Topics}
\begin{itemize}
\item How to design efficient core engine
\item Hybrids with LP/MIP tools
\item Hybrids with SAT
\item Symmetry breaking
\item Use of MDD/BDD to encode sets of solutions
\item High level modelling tools
\item Debugging/visualization
\end{itemize}
\end{frame}

\begin{frame}
\frametitle{Reformulation}
\begin{itemize}
\item Just because the user has modelled it this way, it doesn't mean we have to solve it that way
\begin{itemize}
\item Replace some constraint(s) by other, equivalent constraints
\item Because we don't have that constraint in our system
\item For performance
\end{itemize}
\end{itemize}
\end{frame}

\begin{frame}
\frametitle{Learning}
\begin{itemize}
\item While solving the problem we can learn how to strengthen the model/search 
\begin{itemize}
\item Understand which constraints/method contribute to propagation and change schedule
\item Learn no-good constraints by explaining failure
\item Adapt search strategy based on search experience
\end{itemize}
\end{itemize}
\end{frame}

%% \begin{frame}
%% \frametitle{Refined Process}
%% \begin{center}
%% \begin{tikzpicture}
%% \node[shape=rectangle,fill=pantone127-4] (problem) at (3,7) {Problem};
%% \node (human) at (3,6) {Human};
%% \node[shape=rectangle,fill=pantone127-4] (model) at (3,5) {User Model};
%% \node (reformulation) at (3,4) {Reformulation};
%% \node[shape=rectangle,fill=pantone127-4] (implementation) at (3,3) {Implementation Model};
%% \node (solver) at (3,2) {Constraint Solver/Search};
%% \node[shape=rectangle,fill=pantone157-8] (solution) at (3,1) {Solution};
%% \draw[-] (problem) -- (human);
%% \draw[->] (human) -- (model);
%% \draw[-] (model) -- (reformulation);
%% \draw[->] (reformulation) -- (implementation);
%% \draw[-] (implementation) -- (solver);
%% \draw[->] (solver) -- (0,2) -- (0,3) -- (implementation);
%% \draw[->] (solver) -- (solution);
%% \end{tikzpicture}
%% \end{center}
%% \end{frame}

%\begin{frame}
%\frametitle{What is CP actually used for?}
%\url{http://hsimonis.wordpress.com}
%\includegraphics[width=10cm]{../introduction/images/blog}
%\end{frame}

%% \begin{frame}
%% \frametitle{Exercises}
%% \begin{itemize}
%% \item Install Constraint Solver of your Choice
%% \item MiniZinc (MiningZinc) will be used in later sessions
%% \item Run Sendmoremoney, Sudoku, Queens examples
%% \item Can you choose which propagator to use for alldifferent?
%% \item Can you use different search strategies?
%% \end{itemize}
%% \end{frame}

\begin{frame}
\frametitle{More Learning Resources}
\begin{itemize}
\item Survey of Methods, Resources, and Formats for Teaching Constraint Programming
\begin{itemize}
\item by Tejas Santanam, Helmut Simonis
\item \url{https://doi.org/10.48550/arXiv.2403.12717}
\item Based on survey of community for WTCP 2023
\item \url{https://hsimonis.github.io/WTCP2023/}
\end{itemize}
\end{itemize}
\end{frame}


}

\clearpage
\mode<all>{
\part{Customizing Search}
}
\begin{frame}
\partpage
\end{frame}
\mode<all>{
\begin{frame}
\frametitle{Objectives}
\begin{itemize}
\item Overview of Core Constraint Programming
\item Three Main Concepts
\begin{itemize}
\item Constraint Propagation
\item Global Constraints
\item Customizing Search
\end{itemize}
%\item Get Some Experience with MiniZinc
\item Topics will be treated in more detail in later parts of the school
\item Based on Examples, not Formal Description
\end{itemize}
\end{frame}

\begin{frame}
\frametitle{Outline}
\begin{itemize}
\item Why Constraint Programming?
\item Constraint Propagation
\item Global Constraints
\item Customizing Search
%\item Choices in Modelling
%\item Decomposition
%\item More Global Constraints
%\item Methodology
%\item What is missing?
\end{itemize}
\end{frame}

% \begin{frame}
  % \frametitle{Using MiniZinc IDE}
  % \begin{itemize}
  % \item Developed in the Australian NICTA project
  % \item Maintained by Monash University
  % \item Modelling tool with multiple back-end solvers
  % \item Available from \url{https://www.minizinc.org/}
  % \end{itemize}
% \end{frame}


% \begin{frame}
% \frametitle{Examples in ECLiPSe}
% \begin{itemize}
% \item Open sourced constraint programming language
% \item Development goes back to 1985
% \item ECRC, ICL, IC-Parc, PTL, Cisco
% \item \url{https://eclipseclp.org/}
% \item Specialities
% \begin{itemize}
% \item Develop new solvers for specific domains
% \item Integration with MIP
% \end{itemize}
% \item Not included in bundled MiniZinc IDE
% \item Specialized visualization tools used here
  % \begin{itemize}
  % \item CP-Viz, Simonis et al. 2010
    % \end{itemize}
% \end{itemize}
% \end{frame}

\begin{frame}
\frametitle{Tutorial Based on ECLiPSe ELearning Course}
\begin{itemize}
\item Self-study course in constraint programming
\item Supported by Cisco Systems and Silicon Valley Community Foundation
\item Multi-media format, video lectures, slides, handout etc
\item {\small \url{https://eclipseclp.org/ELearning/index.html}}
\end{itemize}
\end{frame}

\begin{frame}
\frametitle{Also Part of CRT-AI Constraint Week}
\begin{itemize}
\item Annual one week course on CP and Optimization in Ireland
\item Part of national training program for PhD students in AI
\item \url{https://www.crt-ai.ie/}
\end{itemize}
\end{frame}



\begin{frame}
\frametitle{Constraint Programming - in a nutshell}
\begin{itemize}
\item Declarative description of problems with
\begin{itemize}
\item {\em Variables} which range over (finite) sets of values
\item {\em Constraints} over subsets of variables which restrict possible value combinations
\item A {\em solution} is a value assignment which satisfies all constraints
\end{itemize}

\item Constraint propagation/reasoning
\begin{itemize}
\item Removing inconsistent values for variables
\item Detect failure if constraint can not be satisfied
\item Interaction of constraints via shared variables
\item Incomplete
\end{itemize}

\item Search
\begin{itemize}
\item User controlled assignment of values to variables
\item Each step triggers constraint propagation 
\end{itemize}
\item Different domains require/allow different methods
\end{itemize}
\end{frame}

\begin{frame}
  \frametitle{Constraint Programming is Different}
  \begin{itemize}
  \item Declarative Programming
    \begin{itemize}
    \item Concentrate on what you want
      \item Not how to get there
      \item Program != Algorithm
      \item Program = Model
    \end{itemize}
    \item Applied to Combinatorial Problems
      \begin{itemize}
        \item No complete polynomial algorithms known (exist?)
        \item CP less ad-hoc than heuristics
        \item Models can evolve
  \end{itemize}
  \end{itemize}
  \end{frame}
    
\begin{frame}
  \frametitle{A Subtractive Process}
  \begin{textblock}{4}(8,-3)
    \includegraphics[width=4cm]{../introduction/images/stages}
  \end{textblock}
  \vfill
  \begin{quote}
    ``Oh, bosh, as Mr. Ruskin says. Sculpture, per se, is the simplest thing in the world. All you have to do is to take a big chunk of marble and a hammer and chisel, make up your mind what you are about to create and chip off all the marble you don't want.''-Paris Gaulois.
  \end{quote}
  
  {\tiny Source: \url{https://quoteinvestigator.com/2014/06/22/chip-away/}}
\end{frame}

\begin{frame}
\frametitle{Basic Process}
\begin{center}
\begin{tikzpicture}
\node[shape=rectangle,fill=pantone127-4] (problem) at (2,5) {Problem};
\node (human) at (2,4) {Human};
\node[shape=rectangle,fill=pantone127-4] (model) at (2,3) {Model};
\node (solver) at (2,2) {Constraint Solver/Search};
\node[shape=rectangle,fill=pantone157-8] (solution) at (2,1) {Solution};
\draw[-] (problem) -- (human);
\draw[->] (human) -- (model);
\draw[-] (model) -- (solver);
\draw[->] (solver) -- (solution);
\end{tikzpicture}
\end{center}
\end{frame}

\begin{frame}
\frametitle{More Realistic}
\begin{center}
\begin{tikzpicture}[xscale=4.0]
\node[shape=rectangle,fill=pantone127-4] (problem) at (2,5) {Problem};
\node (human) at (2,4) {Human};
\node[shape=rectangle,fill=pantone127-4] (model) at (2,3) {Model};
\node (solver) at (2,2) {Constraint Solver/Search};
\node[shape=rectangle,fill=pantone24-8] (hangs) at (1,1) {Hangs};
\node[shape=rectangle,fill=pantone157-8] (solution) at (2,1) {Solution};
\node[shape=rectangle,fill=pantone24-8] (wrong) at (3,1) {Wrong Solution};
\draw[-] (problem) -- (human);
\draw[->] (human) -- (model);
\draw[-] (model) -- (solver);
\draw[->] (solver) -- (hangs);
\draw[->] (solver) -- (solution);
\draw[->] (solver) -- (wrong);
\draw[->] (hangs) -- (1,4) -- (human);
\draw[->] (wrong) -- (3,4) -- (human);
\end{tikzpicture}
\end{center}
\end{frame}

\begin{frame}
\frametitle{Dual Role of Model}
\begin{itemize}
\item Allows Human to Express Problem
\begin{itemize}
\item Close to Problem Domain
\item Constraints as Abstractions
\end{itemize}
\item Allows Solver to Execute
\begin{itemize}
\item Variables as Communication Mechanism
\item Constraints as Algorithms
\end{itemize}
\end{itemize}
\end{frame}

\begin{frame}
\frametitle{Modelling Frameworks}
\begin{itemize}
\item \href{https://www.minizinc.org/}{MiniZinc} (NICTA, Monash University, Australia)
\item \href{https://github.com/eomahony/Numberjack}{NumberJack} (Insight, Ireland) 
\item \href{https://www-users.york.ac.uk/peter.nightingale/savilerow//}{EssencePrime/SavilleRow} (UK) 
\item \href{https://cpmpy.readthedocs.io/en/latest/}{CPMpy} (KU Leuven)
\item Allow use of multiple back-end solvers
\item Compile model into variants for each solver
\item A priori solver independent model(CP, MIP, SAT)
\end{itemize}
\end{frame} 

\begin{frame}
\frametitle{Framework Process}
\begin{center}
\begin{tikzpicture}[xscale=2]
\node[shape=rectangle,fill=pantone127-4] (problem) at (2.5,5) {Problem};
\node (human) at (2.5,4) {Human};
\node[shape=rectangle,fill=pantone127-4] (model) at (2.5,3) {Model};
\node (compile) at (2.5,2) {Compile/Reformulate};
\node (solver1) at (1,1) {CP};
\node (solver2) at (2,1) {MIP};
\node (solver3) at (3,1) {SAT};
\node (solver4) at (4,1) {Other};
\node[shape=rectangle,fill=pantone157-8] (solution1) at (1,0) {Solution};
\node[shape=rectangle,fill=pantone157-8] (solution2) at (2,0) {Solution};
\node[shape=rectangle,fill=pantone157-8] (solution3) at (3,0) {Solution};
\node[shape=rectangle,fill=pantone157-8] (solution4) at (4,0) {Solution};
\draw[-] (problem) -- (human);
\draw[->] (human) -- (model);
\draw[-] (model) -- (compile);
\draw[-] (compile) -- (solver1);
\draw[-] (compile) -- (solver2);
\draw[-] (compile) -- (solver3);
\draw[-] (compile) -- (solver4);
\draw[->] (solver1) -- (solution1);
\draw[->] (solver2) -- (solution2);
\draw[->] (solver3) -- (solution3);
\draw[->] (solver4) -- (solution4);
\end{tikzpicture}
\end{center}
\end{frame}

% \begin{frame}
  % \frametitle{Do It Now!}
  % \begin{itemize}
  % \item Download and install Minizinc
  % \item \url{https://www.minizinc.org/}
  % \end{itemize}
% \end{frame}


\begin{frame}
\frametitle{Why use Puzzles as Examples?}
\begin{itemize}
\item Easy to understand the problem
\item Solvable by hand without specialized knowledge
\item Possible to compare automated to manual solving process
\end{itemize}

\begin{quote}
The puzzle, though inanimate, is presented as a solvable problem without lasting negative consequences, a very low-risk low-reward situation. By being a puzzle, the object is attempting to convince the user that it must be completed.
\end{quote}
{\tiny Source: \href{https://www.everydayrhetoric.com/2020/03/the-rhetoric-of-puzzles/}{Every Day Rhetoric}}
\end{frame}


\mode<all>{
\part{Basic Constraint Propagation}
}
\begin{frame}
\partpage
\end{frame}
\mode<all>{
\input{../sendmore/body}
}

\mode<all>{
\part{Global Constraints}
}
\begin{frame}
\partpage
\end{frame}
\mode<all>{
\input{../sudoku/body}
}

\clearpage
\mode<all>{
\part{Customizing Search}
}
\begin{frame}
\partpage
\end{frame}
\mode<all>{
\input{../nqueen/body}
}


%% \clearpage
%% \mode<all>{
%%   \part{Problem Decomposition}
%% }
%% \begin{frame}
%%   \partpage
%% \end{frame}
%% \mode<all>{
%%   \input{../party/body}
%%   }

%% \clearpage
%% \mode<all>{
%%   \part{More Global Constraints}
%% }
%% \begin{frame}
%%   \partpage
%% \end{frame}
%% \mode<all>{
%%   \input{../car/body}
%%   }

%% \clearpage
%% \mode<all>{
%%   \part{Methodology}
%% }
%% \begin{frame}
%%   \partpage
%% \end{frame}
%% \mode<all>{
%%   \input{../methodology/body}
%%   }

\clearpage
\mode<all>{
\part{What is missing?}
}
\begin{frame}
\partpage
\end{frame}

\begin{frame}
\frametitle{Many Specialized Topics}
\begin{itemize}
\item How to design efficient core engine
\item Hybrids with LP/MIP tools
\item Hybrids with SAT
\item Symmetry breaking
\item Use of MDD/BDD to encode sets of solutions
\item High level modelling tools
\item Debugging/visualization
\end{itemize}
\end{frame}

\begin{frame}
\frametitle{Reformulation}
\begin{itemize}
\item Just because the user has modelled it this way, it doesn't mean we have to solve it that way
\begin{itemize}
\item Replace some constraint(s) by other, equivalent constraints
\item Because we don't have that constraint in our system
\item For performance
\end{itemize}
\end{itemize}
\end{frame}

\begin{frame}
\frametitle{Learning}
\begin{itemize}
\item While solving the problem we can learn how to strengthen the model/search 
\begin{itemize}
\item Understand which constraints/method contribute to propagation and change schedule
\item Learn no-good constraints by explaining failure
\item Adapt search strategy based on search experience
\end{itemize}
\end{itemize}
\end{frame}

%% \begin{frame}
%% \frametitle{Refined Process}
%% \begin{center}
%% \begin{tikzpicture}
%% \node[shape=rectangle,fill=pantone127-4] (problem) at (3,7) {Problem};
%% \node (human) at (3,6) {Human};
%% \node[shape=rectangle,fill=pantone127-4] (model) at (3,5) {User Model};
%% \node (reformulation) at (3,4) {Reformulation};
%% \node[shape=rectangle,fill=pantone127-4] (implementation) at (3,3) {Implementation Model};
%% \node (solver) at (3,2) {Constraint Solver/Search};
%% \node[shape=rectangle,fill=pantone157-8] (solution) at (3,1) {Solution};
%% \draw[-] (problem) -- (human);
%% \draw[->] (human) -- (model);
%% \draw[-] (model) -- (reformulation);
%% \draw[->] (reformulation) -- (implementation);
%% \draw[-] (implementation) -- (solver);
%% \draw[->] (solver) -- (0,2) -- (0,3) -- (implementation);
%% \draw[->] (solver) -- (solution);
%% \end{tikzpicture}
%% \end{center}
%% \end{frame}

%\begin{frame}
%\frametitle{What is CP actually used for?}
%\url{http://hsimonis.wordpress.com}
%\includegraphics[width=10cm]{../introduction/images/blog}
%\end{frame}

%% \begin{frame}
%% \frametitle{Exercises}
%% \begin{itemize}
%% \item Install Constraint Solver of your Choice
%% \item MiniZinc (MiningZinc) will be used in later sessions
%% \item Run Sendmoremoney, Sudoku, Queens examples
%% \item Can you choose which propagator to use for alldifferent?
%% \item Can you use different search strategies?
%% \end{itemize}
%% \end{frame}

\begin{frame}
\frametitle{More Learning Resources}
\begin{itemize}
\item Survey of Methods, Resources, and Formats for Teaching Constraint Programming
\begin{itemize}
\item by Tejas Santanam, Helmut Simonis
\item \url{https://doi.org/10.48550/arXiv.2403.12717}
\item Based on survey of community for WTCP 2023
\item \url{https://hsimonis.github.io/WTCP2023/}
\end{itemize}
\end{itemize}
\end{frame}


}


%% \clearpage
%% \mode<all>{
%%   \part{Problem Decomposition}
%% }
%% \begin{frame}
%%   \partpage
%% \end{frame}
%% \mode<all>{
%%   \begin{frame}
\frametitle{Objectives}
\begin{itemize}
\item Overview of Core Constraint Programming
\item Three Main Concepts
\begin{itemize}
\item Constraint Propagation
\item Global Constraints
\item Customizing Search
\end{itemize}
%\item Get Some Experience with MiniZinc
\item Topics will be treated in more detail in later parts of the school
\item Based on Examples, not Formal Description
\end{itemize}
\end{frame}

\begin{frame}
\frametitle{Outline}
\begin{itemize}
\item Why Constraint Programming?
\item Constraint Propagation
\item Global Constraints
\item Customizing Search
%\item Choices in Modelling
%\item Decomposition
%\item More Global Constraints
%\item Methodology
%\item What is missing?
\end{itemize}
\end{frame}

% \begin{frame}
  % \frametitle{Using MiniZinc IDE}
  % \begin{itemize}
  % \item Developed in the Australian NICTA project
  % \item Maintained by Monash University
  % \item Modelling tool with multiple back-end solvers
  % \item Available from \url{https://www.minizinc.org/}
  % \end{itemize}
% \end{frame}


% \begin{frame}
% \frametitle{Examples in ECLiPSe}
% \begin{itemize}
% \item Open sourced constraint programming language
% \item Development goes back to 1985
% \item ECRC, ICL, IC-Parc, PTL, Cisco
% \item \url{https://eclipseclp.org/}
% \item Specialities
% \begin{itemize}
% \item Develop new solvers for specific domains
% \item Integration with MIP
% \end{itemize}
% \item Not included in bundled MiniZinc IDE
% \item Specialized visualization tools used here
  % \begin{itemize}
  % \item CP-Viz, Simonis et al. 2010
    % \end{itemize}
% \end{itemize}
% \end{frame}

\begin{frame}
\frametitle{Tutorial Based on ECLiPSe ELearning Course}
\begin{itemize}
\item Self-study course in constraint programming
\item Supported by Cisco Systems and Silicon Valley Community Foundation
\item Multi-media format, video lectures, slides, handout etc
\item {\small \url{https://eclipseclp.org/ELearning/index.html}}
\end{itemize}
\end{frame}

\begin{frame}
\frametitle{Also Part of CRT-AI Constraint Week}
\begin{itemize}
\item Annual one week course on CP and Optimization in Ireland
\item Part of national training program for PhD students in AI
\item \url{https://www.crt-ai.ie/}
\end{itemize}
\end{frame}



\begin{frame}
\frametitle{Constraint Programming - in a nutshell}
\begin{itemize}
\item Declarative description of problems with
\begin{itemize}
\item {\em Variables} which range over (finite) sets of values
\item {\em Constraints} over subsets of variables which restrict possible value combinations
\item A {\em solution} is a value assignment which satisfies all constraints
\end{itemize}

\item Constraint propagation/reasoning
\begin{itemize}
\item Removing inconsistent values for variables
\item Detect failure if constraint can not be satisfied
\item Interaction of constraints via shared variables
\item Incomplete
\end{itemize}

\item Search
\begin{itemize}
\item User controlled assignment of values to variables
\item Each step triggers constraint propagation 
\end{itemize}
\item Different domains require/allow different methods
\end{itemize}
\end{frame}

\begin{frame}
  \frametitle{Constraint Programming is Different}
  \begin{itemize}
  \item Declarative Programming
    \begin{itemize}
    \item Concentrate on what you want
      \item Not how to get there
      \item Program != Algorithm
      \item Program = Model
    \end{itemize}
    \item Applied to Combinatorial Problems
      \begin{itemize}
        \item No complete polynomial algorithms known (exist?)
        \item CP less ad-hoc than heuristics
        \item Models can evolve
  \end{itemize}
  \end{itemize}
  \end{frame}
    
\begin{frame}
  \frametitle{A Subtractive Process}
  \begin{textblock}{4}(8,-3)
    \includegraphics[width=4cm]{../introduction/images/stages}
  \end{textblock}
  \vfill
  \begin{quote}
    ``Oh, bosh, as Mr. Ruskin says. Sculpture, per se, is the simplest thing in the world. All you have to do is to take a big chunk of marble and a hammer and chisel, make up your mind what you are about to create and chip off all the marble you don't want.''-Paris Gaulois.
  \end{quote}
  
  {\tiny Source: \url{https://quoteinvestigator.com/2014/06/22/chip-away/}}
\end{frame}

\begin{frame}
\frametitle{Basic Process}
\begin{center}
\begin{tikzpicture}
\node[shape=rectangle,fill=pantone127-4] (problem) at (2,5) {Problem};
\node (human) at (2,4) {Human};
\node[shape=rectangle,fill=pantone127-4] (model) at (2,3) {Model};
\node (solver) at (2,2) {Constraint Solver/Search};
\node[shape=rectangle,fill=pantone157-8] (solution) at (2,1) {Solution};
\draw[-] (problem) -- (human);
\draw[->] (human) -- (model);
\draw[-] (model) -- (solver);
\draw[->] (solver) -- (solution);
\end{tikzpicture}
\end{center}
\end{frame}

\begin{frame}
\frametitle{More Realistic}
\begin{center}
\begin{tikzpicture}[xscale=4.0]
\node[shape=rectangle,fill=pantone127-4] (problem) at (2,5) {Problem};
\node (human) at (2,4) {Human};
\node[shape=rectangle,fill=pantone127-4] (model) at (2,3) {Model};
\node (solver) at (2,2) {Constraint Solver/Search};
\node[shape=rectangle,fill=pantone24-8] (hangs) at (1,1) {Hangs};
\node[shape=rectangle,fill=pantone157-8] (solution) at (2,1) {Solution};
\node[shape=rectangle,fill=pantone24-8] (wrong) at (3,1) {Wrong Solution};
\draw[-] (problem) -- (human);
\draw[->] (human) -- (model);
\draw[-] (model) -- (solver);
\draw[->] (solver) -- (hangs);
\draw[->] (solver) -- (solution);
\draw[->] (solver) -- (wrong);
\draw[->] (hangs) -- (1,4) -- (human);
\draw[->] (wrong) -- (3,4) -- (human);
\end{tikzpicture}
\end{center}
\end{frame}

\begin{frame}
\frametitle{Dual Role of Model}
\begin{itemize}
\item Allows Human to Express Problem
\begin{itemize}
\item Close to Problem Domain
\item Constraints as Abstractions
\end{itemize}
\item Allows Solver to Execute
\begin{itemize}
\item Variables as Communication Mechanism
\item Constraints as Algorithms
\end{itemize}
\end{itemize}
\end{frame}

\begin{frame}
\frametitle{Modelling Frameworks}
\begin{itemize}
\item \href{https://www.minizinc.org/}{MiniZinc} (NICTA, Monash University, Australia)
\item \href{https://github.com/eomahony/Numberjack}{NumberJack} (Insight, Ireland) 
\item \href{https://www-users.york.ac.uk/peter.nightingale/savilerow//}{EssencePrime/SavilleRow} (UK) 
\item \href{https://cpmpy.readthedocs.io/en/latest/}{CPMpy} (KU Leuven)
\item Allow use of multiple back-end solvers
\item Compile model into variants for each solver
\item A priori solver independent model(CP, MIP, SAT)
\end{itemize}
\end{frame} 

\begin{frame}
\frametitle{Framework Process}
\begin{center}
\begin{tikzpicture}[xscale=2]
\node[shape=rectangle,fill=pantone127-4] (problem) at (2.5,5) {Problem};
\node (human) at (2.5,4) {Human};
\node[shape=rectangle,fill=pantone127-4] (model) at (2.5,3) {Model};
\node (compile) at (2.5,2) {Compile/Reformulate};
\node (solver1) at (1,1) {CP};
\node (solver2) at (2,1) {MIP};
\node (solver3) at (3,1) {SAT};
\node (solver4) at (4,1) {Other};
\node[shape=rectangle,fill=pantone157-8] (solution1) at (1,0) {Solution};
\node[shape=rectangle,fill=pantone157-8] (solution2) at (2,0) {Solution};
\node[shape=rectangle,fill=pantone157-8] (solution3) at (3,0) {Solution};
\node[shape=rectangle,fill=pantone157-8] (solution4) at (4,0) {Solution};
\draw[-] (problem) -- (human);
\draw[->] (human) -- (model);
\draw[-] (model) -- (compile);
\draw[-] (compile) -- (solver1);
\draw[-] (compile) -- (solver2);
\draw[-] (compile) -- (solver3);
\draw[-] (compile) -- (solver4);
\draw[->] (solver1) -- (solution1);
\draw[->] (solver2) -- (solution2);
\draw[->] (solver3) -- (solution3);
\draw[->] (solver4) -- (solution4);
\end{tikzpicture}
\end{center}
\end{frame}

% \begin{frame}
  % \frametitle{Do It Now!}
  % \begin{itemize}
  % \item Download and install Minizinc
  % \item \url{https://www.minizinc.org/}
  % \end{itemize}
% \end{frame}


\begin{frame}
\frametitle{Why use Puzzles as Examples?}
\begin{itemize}
\item Easy to understand the problem
\item Solvable by hand without specialized knowledge
\item Possible to compare automated to manual solving process
\end{itemize}

\begin{quote}
The puzzle, though inanimate, is presented as a solvable problem without lasting negative consequences, a very low-risk low-reward situation. By being a puzzle, the object is attempting to convince the user that it must be completed.
\end{quote}
{\tiny Source: \href{https://www.everydayrhetoric.com/2020/03/the-rhetoric-of-puzzles/}{Every Day Rhetoric}}
\end{frame}


\mode<all>{
\part{Basic Constraint Propagation}
}
\begin{frame}
\partpage
\end{frame}
\mode<all>{
\input{../sendmore/body}
}

\mode<all>{
\part{Global Constraints}
}
\begin{frame}
\partpage
\end{frame}
\mode<all>{
\input{../sudoku/body}
}

\clearpage
\mode<all>{
\part{Customizing Search}
}
\begin{frame}
\partpage
\end{frame}
\mode<all>{
\input{../nqueen/body}
}


%% \clearpage
%% \mode<all>{
%%   \part{Problem Decomposition}
%% }
%% \begin{frame}
%%   \partpage
%% \end{frame}
%% \mode<all>{
%%   \input{../party/body}
%%   }

%% \clearpage
%% \mode<all>{
%%   \part{More Global Constraints}
%% }
%% \begin{frame}
%%   \partpage
%% \end{frame}
%% \mode<all>{
%%   \input{../car/body}
%%   }

%% \clearpage
%% \mode<all>{
%%   \part{Methodology}
%% }
%% \begin{frame}
%%   \partpage
%% \end{frame}
%% \mode<all>{
%%   \input{../methodology/body}
%%   }

\clearpage
\mode<all>{
\part{What is missing?}
}
\begin{frame}
\partpage
\end{frame}

\begin{frame}
\frametitle{Many Specialized Topics}
\begin{itemize}
\item How to design efficient core engine
\item Hybrids with LP/MIP tools
\item Hybrids with SAT
\item Symmetry breaking
\item Use of MDD/BDD to encode sets of solutions
\item High level modelling tools
\item Debugging/visualization
\end{itemize}
\end{frame}

\begin{frame}
\frametitle{Reformulation}
\begin{itemize}
\item Just because the user has modelled it this way, it doesn't mean we have to solve it that way
\begin{itemize}
\item Replace some constraint(s) by other, equivalent constraints
\item Because we don't have that constraint in our system
\item For performance
\end{itemize}
\end{itemize}
\end{frame}

\begin{frame}
\frametitle{Learning}
\begin{itemize}
\item While solving the problem we can learn how to strengthen the model/search 
\begin{itemize}
\item Understand which constraints/method contribute to propagation and change schedule
\item Learn no-good constraints by explaining failure
\item Adapt search strategy based on search experience
\end{itemize}
\end{itemize}
\end{frame}

%% \begin{frame}
%% \frametitle{Refined Process}
%% \begin{center}
%% \begin{tikzpicture}
%% \node[shape=rectangle,fill=pantone127-4] (problem) at (3,7) {Problem};
%% \node (human) at (3,6) {Human};
%% \node[shape=rectangle,fill=pantone127-4] (model) at (3,5) {User Model};
%% \node (reformulation) at (3,4) {Reformulation};
%% \node[shape=rectangle,fill=pantone127-4] (implementation) at (3,3) {Implementation Model};
%% \node (solver) at (3,2) {Constraint Solver/Search};
%% \node[shape=rectangle,fill=pantone157-8] (solution) at (3,1) {Solution};
%% \draw[-] (problem) -- (human);
%% \draw[->] (human) -- (model);
%% \draw[-] (model) -- (reformulation);
%% \draw[->] (reformulation) -- (implementation);
%% \draw[-] (implementation) -- (solver);
%% \draw[->] (solver) -- (0,2) -- (0,3) -- (implementation);
%% \draw[->] (solver) -- (solution);
%% \end{tikzpicture}
%% \end{center}
%% \end{frame}

%\begin{frame}
%\frametitle{What is CP actually used for?}
%\url{http://hsimonis.wordpress.com}
%\includegraphics[width=10cm]{../introduction/images/blog}
%\end{frame}

%% \begin{frame}
%% \frametitle{Exercises}
%% \begin{itemize}
%% \item Install Constraint Solver of your Choice
%% \item MiniZinc (MiningZinc) will be used in later sessions
%% \item Run Sendmoremoney, Sudoku, Queens examples
%% \item Can you choose which propagator to use for alldifferent?
%% \item Can you use different search strategies?
%% \end{itemize}
%% \end{frame}

\begin{frame}
\frametitle{More Learning Resources}
\begin{itemize}
\item Survey of Methods, Resources, and Formats for Teaching Constraint Programming
\begin{itemize}
\item by Tejas Santanam, Helmut Simonis
\item \url{https://doi.org/10.48550/arXiv.2403.12717}
\item Based on survey of community for WTCP 2023
\item \url{https://hsimonis.github.io/WTCP2023/}
\end{itemize}
\end{itemize}
\end{frame}


%%   }

%% \clearpage
%% \mode<all>{
%%   \part{More Global Constraints}
%% }
%% \begin{frame}
%%   \partpage
%% \end{frame}
%% \mode<all>{
%%   \begin{frame}
\frametitle{Objectives}
\begin{itemize}
\item Overview of Core Constraint Programming
\item Three Main Concepts
\begin{itemize}
\item Constraint Propagation
\item Global Constraints
\item Customizing Search
\end{itemize}
%\item Get Some Experience with MiniZinc
\item Topics will be treated in more detail in later parts of the school
\item Based on Examples, not Formal Description
\end{itemize}
\end{frame}

\begin{frame}
\frametitle{Outline}
\begin{itemize}
\item Why Constraint Programming?
\item Constraint Propagation
\item Global Constraints
\item Customizing Search
%\item Choices in Modelling
%\item Decomposition
%\item More Global Constraints
%\item Methodology
%\item What is missing?
\end{itemize}
\end{frame}

% \begin{frame}
  % \frametitle{Using MiniZinc IDE}
  % \begin{itemize}
  % \item Developed in the Australian NICTA project
  % \item Maintained by Monash University
  % \item Modelling tool with multiple back-end solvers
  % \item Available from \url{https://www.minizinc.org/}
  % \end{itemize}
% \end{frame}


% \begin{frame}
% \frametitle{Examples in ECLiPSe}
% \begin{itemize}
% \item Open sourced constraint programming language
% \item Development goes back to 1985
% \item ECRC, ICL, IC-Parc, PTL, Cisco
% \item \url{https://eclipseclp.org/}
% \item Specialities
% \begin{itemize}
% \item Develop new solvers for specific domains
% \item Integration with MIP
% \end{itemize}
% \item Not included in bundled MiniZinc IDE
% \item Specialized visualization tools used here
  % \begin{itemize}
  % \item CP-Viz, Simonis et al. 2010
    % \end{itemize}
% \end{itemize}
% \end{frame}

\begin{frame}
\frametitle{Tutorial Based on ECLiPSe ELearning Course}
\begin{itemize}
\item Self-study course in constraint programming
\item Supported by Cisco Systems and Silicon Valley Community Foundation
\item Multi-media format, video lectures, slides, handout etc
\item {\small \url{https://eclipseclp.org/ELearning/index.html}}
\end{itemize}
\end{frame}

\begin{frame}
\frametitle{Also Part of CRT-AI Constraint Week}
\begin{itemize}
\item Annual one week course on CP and Optimization in Ireland
\item Part of national training program for PhD students in AI
\item \url{https://www.crt-ai.ie/}
\end{itemize}
\end{frame}



\begin{frame}
\frametitle{Constraint Programming - in a nutshell}
\begin{itemize}
\item Declarative description of problems with
\begin{itemize}
\item {\em Variables} which range over (finite) sets of values
\item {\em Constraints} over subsets of variables which restrict possible value combinations
\item A {\em solution} is a value assignment which satisfies all constraints
\end{itemize}

\item Constraint propagation/reasoning
\begin{itemize}
\item Removing inconsistent values for variables
\item Detect failure if constraint can not be satisfied
\item Interaction of constraints via shared variables
\item Incomplete
\end{itemize}

\item Search
\begin{itemize}
\item User controlled assignment of values to variables
\item Each step triggers constraint propagation 
\end{itemize}
\item Different domains require/allow different methods
\end{itemize}
\end{frame}

\begin{frame}
  \frametitle{Constraint Programming is Different}
  \begin{itemize}
  \item Declarative Programming
    \begin{itemize}
    \item Concentrate on what you want
      \item Not how to get there
      \item Program != Algorithm
      \item Program = Model
    \end{itemize}
    \item Applied to Combinatorial Problems
      \begin{itemize}
        \item No complete polynomial algorithms known (exist?)
        \item CP less ad-hoc than heuristics
        \item Models can evolve
  \end{itemize}
  \end{itemize}
  \end{frame}
    
\begin{frame}
  \frametitle{A Subtractive Process}
  \begin{textblock}{4}(8,-3)
    \includegraphics[width=4cm]{../introduction/images/stages}
  \end{textblock}
  \vfill
  \begin{quote}
    ``Oh, bosh, as Mr. Ruskin says. Sculpture, per se, is the simplest thing in the world. All you have to do is to take a big chunk of marble and a hammer and chisel, make up your mind what you are about to create and chip off all the marble you don't want.''-Paris Gaulois.
  \end{quote}
  
  {\tiny Source: \url{https://quoteinvestigator.com/2014/06/22/chip-away/}}
\end{frame}

\begin{frame}
\frametitle{Basic Process}
\begin{center}
\begin{tikzpicture}
\node[shape=rectangle,fill=pantone127-4] (problem) at (2,5) {Problem};
\node (human) at (2,4) {Human};
\node[shape=rectangle,fill=pantone127-4] (model) at (2,3) {Model};
\node (solver) at (2,2) {Constraint Solver/Search};
\node[shape=rectangle,fill=pantone157-8] (solution) at (2,1) {Solution};
\draw[-] (problem) -- (human);
\draw[->] (human) -- (model);
\draw[-] (model) -- (solver);
\draw[->] (solver) -- (solution);
\end{tikzpicture}
\end{center}
\end{frame}

\begin{frame}
\frametitle{More Realistic}
\begin{center}
\begin{tikzpicture}[xscale=4.0]
\node[shape=rectangle,fill=pantone127-4] (problem) at (2,5) {Problem};
\node (human) at (2,4) {Human};
\node[shape=rectangle,fill=pantone127-4] (model) at (2,3) {Model};
\node (solver) at (2,2) {Constraint Solver/Search};
\node[shape=rectangle,fill=pantone24-8] (hangs) at (1,1) {Hangs};
\node[shape=rectangle,fill=pantone157-8] (solution) at (2,1) {Solution};
\node[shape=rectangle,fill=pantone24-8] (wrong) at (3,1) {Wrong Solution};
\draw[-] (problem) -- (human);
\draw[->] (human) -- (model);
\draw[-] (model) -- (solver);
\draw[->] (solver) -- (hangs);
\draw[->] (solver) -- (solution);
\draw[->] (solver) -- (wrong);
\draw[->] (hangs) -- (1,4) -- (human);
\draw[->] (wrong) -- (3,4) -- (human);
\end{tikzpicture}
\end{center}
\end{frame}

\begin{frame}
\frametitle{Dual Role of Model}
\begin{itemize}
\item Allows Human to Express Problem
\begin{itemize}
\item Close to Problem Domain
\item Constraints as Abstractions
\end{itemize}
\item Allows Solver to Execute
\begin{itemize}
\item Variables as Communication Mechanism
\item Constraints as Algorithms
\end{itemize}
\end{itemize}
\end{frame}

\begin{frame}
\frametitle{Modelling Frameworks}
\begin{itemize}
\item \href{https://www.minizinc.org/}{MiniZinc} (NICTA, Monash University, Australia)
\item \href{https://github.com/eomahony/Numberjack}{NumberJack} (Insight, Ireland) 
\item \href{https://www-users.york.ac.uk/peter.nightingale/savilerow//}{EssencePrime/SavilleRow} (UK) 
\item \href{https://cpmpy.readthedocs.io/en/latest/}{CPMpy} (KU Leuven)
\item Allow use of multiple back-end solvers
\item Compile model into variants for each solver
\item A priori solver independent model(CP, MIP, SAT)
\end{itemize}
\end{frame} 

\begin{frame}
\frametitle{Framework Process}
\begin{center}
\begin{tikzpicture}[xscale=2]
\node[shape=rectangle,fill=pantone127-4] (problem) at (2.5,5) {Problem};
\node (human) at (2.5,4) {Human};
\node[shape=rectangle,fill=pantone127-4] (model) at (2.5,3) {Model};
\node (compile) at (2.5,2) {Compile/Reformulate};
\node (solver1) at (1,1) {CP};
\node (solver2) at (2,1) {MIP};
\node (solver3) at (3,1) {SAT};
\node (solver4) at (4,1) {Other};
\node[shape=rectangle,fill=pantone157-8] (solution1) at (1,0) {Solution};
\node[shape=rectangle,fill=pantone157-8] (solution2) at (2,0) {Solution};
\node[shape=rectangle,fill=pantone157-8] (solution3) at (3,0) {Solution};
\node[shape=rectangle,fill=pantone157-8] (solution4) at (4,0) {Solution};
\draw[-] (problem) -- (human);
\draw[->] (human) -- (model);
\draw[-] (model) -- (compile);
\draw[-] (compile) -- (solver1);
\draw[-] (compile) -- (solver2);
\draw[-] (compile) -- (solver3);
\draw[-] (compile) -- (solver4);
\draw[->] (solver1) -- (solution1);
\draw[->] (solver2) -- (solution2);
\draw[->] (solver3) -- (solution3);
\draw[->] (solver4) -- (solution4);
\end{tikzpicture}
\end{center}
\end{frame}

% \begin{frame}
  % \frametitle{Do It Now!}
  % \begin{itemize}
  % \item Download and install Minizinc
  % \item \url{https://www.minizinc.org/}
  % \end{itemize}
% \end{frame}


\begin{frame}
\frametitle{Why use Puzzles as Examples?}
\begin{itemize}
\item Easy to understand the problem
\item Solvable by hand without specialized knowledge
\item Possible to compare automated to manual solving process
\end{itemize}

\begin{quote}
The puzzle, though inanimate, is presented as a solvable problem without lasting negative consequences, a very low-risk low-reward situation. By being a puzzle, the object is attempting to convince the user that it must be completed.
\end{quote}
{\tiny Source: \href{https://www.everydayrhetoric.com/2020/03/the-rhetoric-of-puzzles/}{Every Day Rhetoric}}
\end{frame}


\mode<all>{
\part{Basic Constraint Propagation}
}
\begin{frame}
\partpage
\end{frame}
\mode<all>{
\input{../sendmore/body}
}

\mode<all>{
\part{Global Constraints}
}
\begin{frame}
\partpage
\end{frame}
\mode<all>{
\input{../sudoku/body}
}

\clearpage
\mode<all>{
\part{Customizing Search}
}
\begin{frame}
\partpage
\end{frame}
\mode<all>{
\input{../nqueen/body}
}


%% \clearpage
%% \mode<all>{
%%   \part{Problem Decomposition}
%% }
%% \begin{frame}
%%   \partpage
%% \end{frame}
%% \mode<all>{
%%   \input{../party/body}
%%   }

%% \clearpage
%% \mode<all>{
%%   \part{More Global Constraints}
%% }
%% \begin{frame}
%%   \partpage
%% \end{frame}
%% \mode<all>{
%%   \input{../car/body}
%%   }

%% \clearpage
%% \mode<all>{
%%   \part{Methodology}
%% }
%% \begin{frame}
%%   \partpage
%% \end{frame}
%% \mode<all>{
%%   \input{../methodology/body}
%%   }

\clearpage
\mode<all>{
\part{What is missing?}
}
\begin{frame}
\partpage
\end{frame}

\begin{frame}
\frametitle{Many Specialized Topics}
\begin{itemize}
\item How to design efficient core engine
\item Hybrids with LP/MIP tools
\item Hybrids with SAT
\item Symmetry breaking
\item Use of MDD/BDD to encode sets of solutions
\item High level modelling tools
\item Debugging/visualization
\end{itemize}
\end{frame}

\begin{frame}
\frametitle{Reformulation}
\begin{itemize}
\item Just because the user has modelled it this way, it doesn't mean we have to solve it that way
\begin{itemize}
\item Replace some constraint(s) by other, equivalent constraints
\item Because we don't have that constraint in our system
\item For performance
\end{itemize}
\end{itemize}
\end{frame}

\begin{frame}
\frametitle{Learning}
\begin{itemize}
\item While solving the problem we can learn how to strengthen the model/search 
\begin{itemize}
\item Understand which constraints/method contribute to propagation and change schedule
\item Learn no-good constraints by explaining failure
\item Adapt search strategy based on search experience
\end{itemize}
\end{itemize}
\end{frame}

%% \begin{frame}
%% \frametitle{Refined Process}
%% \begin{center}
%% \begin{tikzpicture}
%% \node[shape=rectangle,fill=pantone127-4] (problem) at (3,7) {Problem};
%% \node (human) at (3,6) {Human};
%% \node[shape=rectangle,fill=pantone127-4] (model) at (3,5) {User Model};
%% \node (reformulation) at (3,4) {Reformulation};
%% \node[shape=rectangle,fill=pantone127-4] (implementation) at (3,3) {Implementation Model};
%% \node (solver) at (3,2) {Constraint Solver/Search};
%% \node[shape=rectangle,fill=pantone157-8] (solution) at (3,1) {Solution};
%% \draw[-] (problem) -- (human);
%% \draw[->] (human) -- (model);
%% \draw[-] (model) -- (reformulation);
%% \draw[->] (reformulation) -- (implementation);
%% \draw[-] (implementation) -- (solver);
%% \draw[->] (solver) -- (0,2) -- (0,3) -- (implementation);
%% \draw[->] (solver) -- (solution);
%% \end{tikzpicture}
%% \end{center}
%% \end{frame}

%\begin{frame}
%\frametitle{What is CP actually used for?}
%\url{http://hsimonis.wordpress.com}
%\includegraphics[width=10cm]{../introduction/images/blog}
%\end{frame}

%% \begin{frame}
%% \frametitle{Exercises}
%% \begin{itemize}
%% \item Install Constraint Solver of your Choice
%% \item MiniZinc (MiningZinc) will be used in later sessions
%% \item Run Sendmoremoney, Sudoku, Queens examples
%% \item Can you choose which propagator to use for alldifferent?
%% \item Can you use different search strategies?
%% \end{itemize}
%% \end{frame}

\begin{frame}
\frametitle{More Learning Resources}
\begin{itemize}
\item Survey of Methods, Resources, and Formats for Teaching Constraint Programming
\begin{itemize}
\item by Tejas Santanam, Helmut Simonis
\item \url{https://doi.org/10.48550/arXiv.2403.12717}
\item Based on survey of community for WTCP 2023
\item \url{https://hsimonis.github.io/WTCP2023/}
\end{itemize}
\end{itemize}
\end{frame}


%%   }

%% \clearpage
%% \mode<all>{
%%   \part{Methodology}
%% }
%% \begin{frame}
%%   \partpage
%% \end{frame}
%% \mode<all>{
%%   \begin{frame}
\frametitle{Objectives}
\begin{itemize}
\item Overview of Core Constraint Programming
\item Three Main Concepts
\begin{itemize}
\item Constraint Propagation
\item Global Constraints
\item Customizing Search
\end{itemize}
%\item Get Some Experience with MiniZinc
\item Topics will be treated in more detail in later parts of the school
\item Based on Examples, not Formal Description
\end{itemize}
\end{frame}

\begin{frame}
\frametitle{Outline}
\begin{itemize}
\item Why Constraint Programming?
\item Constraint Propagation
\item Global Constraints
\item Customizing Search
%\item Choices in Modelling
%\item Decomposition
%\item More Global Constraints
%\item Methodology
%\item What is missing?
\end{itemize}
\end{frame}

% \begin{frame}
  % \frametitle{Using MiniZinc IDE}
  % \begin{itemize}
  % \item Developed in the Australian NICTA project
  % \item Maintained by Monash University
  % \item Modelling tool with multiple back-end solvers
  % \item Available from \url{https://www.minizinc.org/}
  % \end{itemize}
% \end{frame}


% \begin{frame}
% \frametitle{Examples in ECLiPSe}
% \begin{itemize}
% \item Open sourced constraint programming language
% \item Development goes back to 1985
% \item ECRC, ICL, IC-Parc, PTL, Cisco
% \item \url{https://eclipseclp.org/}
% \item Specialities
% \begin{itemize}
% \item Develop new solvers for specific domains
% \item Integration with MIP
% \end{itemize}
% \item Not included in bundled MiniZinc IDE
% \item Specialized visualization tools used here
  % \begin{itemize}
  % \item CP-Viz, Simonis et al. 2010
    % \end{itemize}
% \end{itemize}
% \end{frame}

\begin{frame}
\frametitle{Tutorial Based on ECLiPSe ELearning Course}
\begin{itemize}
\item Self-study course in constraint programming
\item Supported by Cisco Systems and Silicon Valley Community Foundation
\item Multi-media format, video lectures, slides, handout etc
\item {\small \url{https://eclipseclp.org/ELearning/index.html}}
\end{itemize}
\end{frame}

\begin{frame}
\frametitle{Also Part of CRT-AI Constraint Week}
\begin{itemize}
\item Annual one week course on CP and Optimization in Ireland
\item Part of national training program for PhD students in AI
\item \url{https://www.crt-ai.ie/}
\end{itemize}
\end{frame}



\begin{frame}
\frametitle{Constraint Programming - in a nutshell}
\begin{itemize}
\item Declarative description of problems with
\begin{itemize}
\item {\em Variables} which range over (finite) sets of values
\item {\em Constraints} over subsets of variables which restrict possible value combinations
\item A {\em solution} is a value assignment which satisfies all constraints
\end{itemize}

\item Constraint propagation/reasoning
\begin{itemize}
\item Removing inconsistent values for variables
\item Detect failure if constraint can not be satisfied
\item Interaction of constraints via shared variables
\item Incomplete
\end{itemize}

\item Search
\begin{itemize}
\item User controlled assignment of values to variables
\item Each step triggers constraint propagation 
\end{itemize}
\item Different domains require/allow different methods
\end{itemize}
\end{frame}

\begin{frame}
  \frametitle{Constraint Programming is Different}
  \begin{itemize}
  \item Declarative Programming
    \begin{itemize}
    \item Concentrate on what you want
      \item Not how to get there
      \item Program != Algorithm
      \item Program = Model
    \end{itemize}
    \item Applied to Combinatorial Problems
      \begin{itemize}
        \item No complete polynomial algorithms known (exist?)
        \item CP less ad-hoc than heuristics
        \item Models can evolve
  \end{itemize}
  \end{itemize}
  \end{frame}
    
\begin{frame}
  \frametitle{A Subtractive Process}
  \begin{textblock}{4}(8,-3)
    \includegraphics[width=4cm]{../introduction/images/stages}
  \end{textblock}
  \vfill
  \begin{quote}
    ``Oh, bosh, as Mr. Ruskin says. Sculpture, per se, is the simplest thing in the world. All you have to do is to take a big chunk of marble and a hammer and chisel, make up your mind what you are about to create and chip off all the marble you don't want.''-Paris Gaulois.
  \end{quote}
  
  {\tiny Source: \url{https://quoteinvestigator.com/2014/06/22/chip-away/}}
\end{frame}

\begin{frame}
\frametitle{Basic Process}
\begin{center}
\begin{tikzpicture}
\node[shape=rectangle,fill=pantone127-4] (problem) at (2,5) {Problem};
\node (human) at (2,4) {Human};
\node[shape=rectangle,fill=pantone127-4] (model) at (2,3) {Model};
\node (solver) at (2,2) {Constraint Solver/Search};
\node[shape=rectangle,fill=pantone157-8] (solution) at (2,1) {Solution};
\draw[-] (problem) -- (human);
\draw[->] (human) -- (model);
\draw[-] (model) -- (solver);
\draw[->] (solver) -- (solution);
\end{tikzpicture}
\end{center}
\end{frame}

\begin{frame}
\frametitle{More Realistic}
\begin{center}
\begin{tikzpicture}[xscale=4.0]
\node[shape=rectangle,fill=pantone127-4] (problem) at (2,5) {Problem};
\node (human) at (2,4) {Human};
\node[shape=rectangle,fill=pantone127-4] (model) at (2,3) {Model};
\node (solver) at (2,2) {Constraint Solver/Search};
\node[shape=rectangle,fill=pantone24-8] (hangs) at (1,1) {Hangs};
\node[shape=rectangle,fill=pantone157-8] (solution) at (2,1) {Solution};
\node[shape=rectangle,fill=pantone24-8] (wrong) at (3,1) {Wrong Solution};
\draw[-] (problem) -- (human);
\draw[->] (human) -- (model);
\draw[-] (model) -- (solver);
\draw[->] (solver) -- (hangs);
\draw[->] (solver) -- (solution);
\draw[->] (solver) -- (wrong);
\draw[->] (hangs) -- (1,4) -- (human);
\draw[->] (wrong) -- (3,4) -- (human);
\end{tikzpicture}
\end{center}
\end{frame}

\begin{frame}
\frametitle{Dual Role of Model}
\begin{itemize}
\item Allows Human to Express Problem
\begin{itemize}
\item Close to Problem Domain
\item Constraints as Abstractions
\end{itemize}
\item Allows Solver to Execute
\begin{itemize}
\item Variables as Communication Mechanism
\item Constraints as Algorithms
\end{itemize}
\end{itemize}
\end{frame}

\begin{frame}
\frametitle{Modelling Frameworks}
\begin{itemize}
\item \href{https://www.minizinc.org/}{MiniZinc} (NICTA, Monash University, Australia)
\item \href{https://github.com/eomahony/Numberjack}{NumberJack} (Insight, Ireland) 
\item \href{https://www-users.york.ac.uk/peter.nightingale/savilerow//}{EssencePrime/SavilleRow} (UK) 
\item \href{https://cpmpy.readthedocs.io/en/latest/}{CPMpy} (KU Leuven)
\item Allow use of multiple back-end solvers
\item Compile model into variants for each solver
\item A priori solver independent model(CP, MIP, SAT)
\end{itemize}
\end{frame} 

\begin{frame}
\frametitle{Framework Process}
\begin{center}
\begin{tikzpicture}[xscale=2]
\node[shape=rectangle,fill=pantone127-4] (problem) at (2.5,5) {Problem};
\node (human) at (2.5,4) {Human};
\node[shape=rectangle,fill=pantone127-4] (model) at (2.5,3) {Model};
\node (compile) at (2.5,2) {Compile/Reformulate};
\node (solver1) at (1,1) {CP};
\node (solver2) at (2,1) {MIP};
\node (solver3) at (3,1) {SAT};
\node (solver4) at (4,1) {Other};
\node[shape=rectangle,fill=pantone157-8] (solution1) at (1,0) {Solution};
\node[shape=rectangle,fill=pantone157-8] (solution2) at (2,0) {Solution};
\node[shape=rectangle,fill=pantone157-8] (solution3) at (3,0) {Solution};
\node[shape=rectangle,fill=pantone157-8] (solution4) at (4,0) {Solution};
\draw[-] (problem) -- (human);
\draw[->] (human) -- (model);
\draw[-] (model) -- (compile);
\draw[-] (compile) -- (solver1);
\draw[-] (compile) -- (solver2);
\draw[-] (compile) -- (solver3);
\draw[-] (compile) -- (solver4);
\draw[->] (solver1) -- (solution1);
\draw[->] (solver2) -- (solution2);
\draw[->] (solver3) -- (solution3);
\draw[->] (solver4) -- (solution4);
\end{tikzpicture}
\end{center}
\end{frame}

% \begin{frame}
  % \frametitle{Do It Now!}
  % \begin{itemize}
  % \item Download and install Minizinc
  % \item \url{https://www.minizinc.org/}
  % \end{itemize}
% \end{frame}


\begin{frame}
\frametitle{Why use Puzzles as Examples?}
\begin{itemize}
\item Easy to understand the problem
\item Solvable by hand without specialized knowledge
\item Possible to compare automated to manual solving process
\end{itemize}

\begin{quote}
The puzzle, though inanimate, is presented as a solvable problem without lasting negative consequences, a very low-risk low-reward situation. By being a puzzle, the object is attempting to convince the user that it must be completed.
\end{quote}
{\tiny Source: \href{https://www.everydayrhetoric.com/2020/03/the-rhetoric-of-puzzles/}{Every Day Rhetoric}}
\end{frame}


\mode<all>{
\part{Basic Constraint Propagation}
}
\begin{frame}
\partpage
\end{frame}
\mode<all>{
\input{../sendmore/body}
}

\mode<all>{
\part{Global Constraints}
}
\begin{frame}
\partpage
\end{frame}
\mode<all>{
\input{../sudoku/body}
}

\clearpage
\mode<all>{
\part{Customizing Search}
}
\begin{frame}
\partpage
\end{frame}
\mode<all>{
\input{../nqueen/body}
}


%% \clearpage
%% \mode<all>{
%%   \part{Problem Decomposition}
%% }
%% \begin{frame}
%%   \partpage
%% \end{frame}
%% \mode<all>{
%%   \input{../party/body}
%%   }

%% \clearpage
%% \mode<all>{
%%   \part{More Global Constraints}
%% }
%% \begin{frame}
%%   \partpage
%% \end{frame}
%% \mode<all>{
%%   \input{../car/body}
%%   }

%% \clearpage
%% \mode<all>{
%%   \part{Methodology}
%% }
%% \begin{frame}
%%   \partpage
%% \end{frame}
%% \mode<all>{
%%   \input{../methodology/body}
%%   }

\clearpage
\mode<all>{
\part{What is missing?}
}
\begin{frame}
\partpage
\end{frame}

\begin{frame}
\frametitle{Many Specialized Topics}
\begin{itemize}
\item How to design efficient core engine
\item Hybrids with LP/MIP tools
\item Hybrids with SAT
\item Symmetry breaking
\item Use of MDD/BDD to encode sets of solutions
\item High level modelling tools
\item Debugging/visualization
\end{itemize}
\end{frame}

\begin{frame}
\frametitle{Reformulation}
\begin{itemize}
\item Just because the user has modelled it this way, it doesn't mean we have to solve it that way
\begin{itemize}
\item Replace some constraint(s) by other, equivalent constraints
\item Because we don't have that constraint in our system
\item For performance
\end{itemize}
\end{itemize}
\end{frame}

\begin{frame}
\frametitle{Learning}
\begin{itemize}
\item While solving the problem we can learn how to strengthen the model/search 
\begin{itemize}
\item Understand which constraints/method contribute to propagation and change schedule
\item Learn no-good constraints by explaining failure
\item Adapt search strategy based on search experience
\end{itemize}
\end{itemize}
\end{frame}

%% \begin{frame}
%% \frametitle{Refined Process}
%% \begin{center}
%% \begin{tikzpicture}
%% \node[shape=rectangle,fill=pantone127-4] (problem) at (3,7) {Problem};
%% \node (human) at (3,6) {Human};
%% \node[shape=rectangle,fill=pantone127-4] (model) at (3,5) {User Model};
%% \node (reformulation) at (3,4) {Reformulation};
%% \node[shape=rectangle,fill=pantone127-4] (implementation) at (3,3) {Implementation Model};
%% \node (solver) at (3,2) {Constraint Solver/Search};
%% \node[shape=rectangle,fill=pantone157-8] (solution) at (3,1) {Solution};
%% \draw[-] (problem) -- (human);
%% \draw[->] (human) -- (model);
%% \draw[-] (model) -- (reformulation);
%% \draw[->] (reformulation) -- (implementation);
%% \draw[-] (implementation) -- (solver);
%% \draw[->] (solver) -- (0,2) -- (0,3) -- (implementation);
%% \draw[->] (solver) -- (solution);
%% \end{tikzpicture}
%% \end{center}
%% \end{frame}

%\begin{frame}
%\frametitle{What is CP actually used for?}
%\url{http://hsimonis.wordpress.com}
%\includegraphics[width=10cm]{../introduction/images/blog}
%\end{frame}

%% \begin{frame}
%% \frametitle{Exercises}
%% \begin{itemize}
%% \item Install Constraint Solver of your Choice
%% \item MiniZinc (MiningZinc) will be used in later sessions
%% \item Run Sendmoremoney, Sudoku, Queens examples
%% \item Can you choose which propagator to use for alldifferent?
%% \item Can you use different search strategies?
%% \end{itemize}
%% \end{frame}

\begin{frame}
\frametitle{More Learning Resources}
\begin{itemize}
\item Survey of Methods, Resources, and Formats for Teaching Constraint Programming
\begin{itemize}
\item by Tejas Santanam, Helmut Simonis
\item \url{https://doi.org/10.48550/arXiv.2403.12717}
\item Based on survey of community for WTCP 2023
\item \url{https://hsimonis.github.io/WTCP2023/}
\end{itemize}
\end{itemize}
\end{frame}


%%   }

\clearpage
\mode<all>{
\part{What is missing?}
}
\begin{frame}
\partpage
\end{frame}

\begin{frame}
\frametitle{Many Specialized Topics}
\begin{itemize}
\item How to design efficient core engine
\item Hybrids with LP/MIP tools
\item Hybrids with SAT
\item Symmetry breaking
\item Use of MDD/BDD to encode sets of solutions
\item High level modelling tools
\item Debugging/visualization
\end{itemize}
\end{frame}

\begin{frame}
\frametitle{Reformulation}
\begin{itemize}
\item Just because the user has modelled it this way, it doesn't mean we have to solve it that way
\begin{itemize}
\item Replace some constraint(s) by other, equivalent constraints
\item Because we don't have that constraint in our system
\item For performance
\end{itemize}
\end{itemize}
\end{frame}

\begin{frame}
\frametitle{Learning}
\begin{itemize}
\item While solving the problem we can learn how to strengthen the model/search 
\begin{itemize}
\item Understand which constraints/method contribute to propagation and change schedule
\item Learn no-good constraints by explaining failure
\item Adapt search strategy based on search experience
\end{itemize}
\end{itemize}
\end{frame}

%% \begin{frame}
%% \frametitle{Refined Process}
%% \begin{center}
%% \begin{tikzpicture}
%% \node[shape=rectangle,fill=pantone127-4] (problem) at (3,7) {Problem};
%% \node (human) at (3,6) {Human};
%% \node[shape=rectangle,fill=pantone127-4] (model) at (3,5) {User Model};
%% \node (reformulation) at (3,4) {Reformulation};
%% \node[shape=rectangle,fill=pantone127-4] (implementation) at (3,3) {Implementation Model};
%% \node (solver) at (3,2) {Constraint Solver/Search};
%% \node[shape=rectangle,fill=pantone157-8] (solution) at (3,1) {Solution};
%% \draw[-] (problem) -- (human);
%% \draw[->] (human) -- (model);
%% \draw[-] (model) -- (reformulation);
%% \draw[->] (reformulation) -- (implementation);
%% \draw[-] (implementation) -- (solver);
%% \draw[->] (solver) -- (0,2) -- (0,3) -- (implementation);
%% \draw[->] (solver) -- (solution);
%% \end{tikzpicture}
%% \end{center}
%% \end{frame}

%\begin{frame}
%\frametitle{What is CP actually used for?}
%\url{http://hsimonis.wordpress.com}
%\includegraphics[width=10cm]{../introduction/images/blog}
%\end{frame}

%% \begin{frame}
%% \frametitle{Exercises}
%% \begin{itemize}
%% \item Install Constraint Solver of your Choice
%% \item MiniZinc (MiningZinc) will be used in later sessions
%% \item Run Sendmoremoney, Sudoku, Queens examples
%% \item Can you choose which propagator to use for alldifferent?
%% \item Can you use different search strategies?
%% \end{itemize}
%% \end{frame}

\begin{frame}
\frametitle{More Learning Resources}
\begin{itemize}
\item Survey of Methods, Resources, and Formats for Teaching Constraint Programming
\begin{itemize}
\item by Tejas Santanam, Helmut Simonis
\item \url{https://doi.org/10.48550/arXiv.2403.12717}
\item Based on survey of community for WTCP 2023
\item \url{https://hsimonis.github.io/WTCP2023/}
\end{itemize}
\end{itemize}
\end{frame}


%%   }

%% \clearpage
%% \mode<all>{
%%   \part{More Global Constraints}
%% }
%% \begin{frame}
%%   \partpage
%% \end{frame}
%% \mode<all>{
%%   \begin{frame}
\frametitle{Objectives}
\begin{itemize}
\item Overview of Core Constraint Programming
\item Three Main Concepts
\begin{itemize}
\item Constraint Propagation
\item Global Constraints
\item Customizing Search
\end{itemize}
%\item Get Some Experience with MiniZinc
\item Topics will be treated in more detail in later parts of the school
\item Based on Examples, not Formal Description
\end{itemize}
\end{frame}

\begin{frame}
\frametitle{Outline}
\begin{itemize}
\item Why Constraint Programming?
\item Constraint Propagation
\item Global Constraints
\item Customizing Search
%\item Choices in Modelling
%\item Decomposition
%\item More Global Constraints
%\item Methodology
%\item What is missing?
\end{itemize}
\end{frame}

% \begin{frame}
  % \frametitle{Using MiniZinc IDE}
  % \begin{itemize}
  % \item Developed in the Australian NICTA project
  % \item Maintained by Monash University
  % \item Modelling tool with multiple back-end solvers
  % \item Available from \url{https://www.minizinc.org/}
  % \end{itemize}
% \end{frame}


% \begin{frame}
% \frametitle{Examples in ECLiPSe}
% \begin{itemize}
% \item Open sourced constraint programming language
% \item Development goes back to 1985
% \item ECRC, ICL, IC-Parc, PTL, Cisco
% \item \url{https://eclipseclp.org/}
% \item Specialities
% \begin{itemize}
% \item Develop new solvers for specific domains
% \item Integration with MIP
% \end{itemize}
% \item Not included in bundled MiniZinc IDE
% \item Specialized visualization tools used here
  % \begin{itemize}
  % \item CP-Viz, Simonis et al. 2010
    % \end{itemize}
% \end{itemize}
% \end{frame}

\begin{frame}
\frametitle{Tutorial Based on ECLiPSe ELearning Course}
\begin{itemize}
\item Self-study course in constraint programming
\item Supported by Cisco Systems and Silicon Valley Community Foundation
\item Multi-media format, video lectures, slides, handout etc
\item {\small \url{https://eclipseclp.org/ELearning/index.html}}
\end{itemize}
\end{frame}

\begin{frame}
\frametitle{Also Part of CRT-AI Constraint Week}
\begin{itemize}
\item Annual one week course on CP and Optimization in Ireland
\item Part of national training program for PhD students in AI
\item \url{https://www.crt-ai.ie/}
\end{itemize}
\end{frame}



\begin{frame}
\frametitle{Constraint Programming - in a nutshell}
\begin{itemize}
\item Declarative description of problems with
\begin{itemize}
\item {\em Variables} which range over (finite) sets of values
\item {\em Constraints} over subsets of variables which restrict possible value combinations
\item A {\em solution} is a value assignment which satisfies all constraints
\end{itemize}

\item Constraint propagation/reasoning
\begin{itemize}
\item Removing inconsistent values for variables
\item Detect failure if constraint can not be satisfied
\item Interaction of constraints via shared variables
\item Incomplete
\end{itemize}

\item Search
\begin{itemize}
\item User controlled assignment of values to variables
\item Each step triggers constraint propagation 
\end{itemize}
\item Different domains require/allow different methods
\end{itemize}
\end{frame}

\begin{frame}
  \frametitle{Constraint Programming is Different}
  \begin{itemize}
  \item Declarative Programming
    \begin{itemize}
    \item Concentrate on what you want
      \item Not how to get there
      \item Program != Algorithm
      \item Program = Model
    \end{itemize}
    \item Applied to Combinatorial Problems
      \begin{itemize}
        \item No complete polynomial algorithms known (exist?)
        \item CP less ad-hoc than heuristics
        \item Models can evolve
  \end{itemize}
  \end{itemize}
  \end{frame}
    
\begin{frame}
  \frametitle{A Subtractive Process}
  \begin{textblock}{4}(8,-3)
    \includegraphics[width=4cm]{../introduction/images/stages}
  \end{textblock}
  \vfill
  \begin{quote}
    ``Oh, bosh, as Mr. Ruskin says. Sculpture, per se, is the simplest thing in the world. All you have to do is to take a big chunk of marble and a hammer and chisel, make up your mind what you are about to create and chip off all the marble you don't want.''-Paris Gaulois.
  \end{quote}
  
  {\tiny Source: \url{https://quoteinvestigator.com/2014/06/22/chip-away/}}
\end{frame}

\begin{frame}
\frametitle{Basic Process}
\begin{center}
\begin{tikzpicture}
\node[shape=rectangle,fill=pantone127-4] (problem) at (2,5) {Problem};
\node (human) at (2,4) {Human};
\node[shape=rectangle,fill=pantone127-4] (model) at (2,3) {Model};
\node (solver) at (2,2) {Constraint Solver/Search};
\node[shape=rectangle,fill=pantone157-8] (solution) at (2,1) {Solution};
\draw[-] (problem) -- (human);
\draw[->] (human) -- (model);
\draw[-] (model) -- (solver);
\draw[->] (solver) -- (solution);
\end{tikzpicture}
\end{center}
\end{frame}

\begin{frame}
\frametitle{More Realistic}
\begin{center}
\begin{tikzpicture}[xscale=4.0]
\node[shape=rectangle,fill=pantone127-4] (problem) at (2,5) {Problem};
\node (human) at (2,4) {Human};
\node[shape=rectangle,fill=pantone127-4] (model) at (2,3) {Model};
\node (solver) at (2,2) {Constraint Solver/Search};
\node[shape=rectangle,fill=pantone24-8] (hangs) at (1,1) {Hangs};
\node[shape=rectangle,fill=pantone157-8] (solution) at (2,1) {Solution};
\node[shape=rectangle,fill=pantone24-8] (wrong) at (3,1) {Wrong Solution};
\draw[-] (problem) -- (human);
\draw[->] (human) -- (model);
\draw[-] (model) -- (solver);
\draw[->] (solver) -- (hangs);
\draw[->] (solver) -- (solution);
\draw[->] (solver) -- (wrong);
\draw[->] (hangs) -- (1,4) -- (human);
\draw[->] (wrong) -- (3,4) -- (human);
\end{tikzpicture}
\end{center}
\end{frame}

\begin{frame}
\frametitle{Dual Role of Model}
\begin{itemize}
\item Allows Human to Express Problem
\begin{itemize}
\item Close to Problem Domain
\item Constraints as Abstractions
\end{itemize}
\item Allows Solver to Execute
\begin{itemize}
\item Variables as Communication Mechanism
\item Constraints as Algorithms
\end{itemize}
\end{itemize}
\end{frame}

\begin{frame}
\frametitle{Modelling Frameworks}
\begin{itemize}
\item \href{https://www.minizinc.org/}{MiniZinc} (NICTA, Monash University, Australia)
\item \href{https://github.com/eomahony/Numberjack}{NumberJack} (Insight, Ireland) 
\item \href{https://www-users.york.ac.uk/peter.nightingale/savilerow//}{EssencePrime/SavilleRow} (UK) 
\item \href{https://cpmpy.readthedocs.io/en/latest/}{CPMpy} (KU Leuven)
\item Allow use of multiple back-end solvers
\item Compile model into variants for each solver
\item A priori solver independent model(CP, MIP, SAT)
\end{itemize}
\end{frame} 

\begin{frame}
\frametitle{Framework Process}
\begin{center}
\begin{tikzpicture}[xscale=2]
\node[shape=rectangle,fill=pantone127-4] (problem) at (2.5,5) {Problem};
\node (human) at (2.5,4) {Human};
\node[shape=rectangle,fill=pantone127-4] (model) at (2.5,3) {Model};
\node (compile) at (2.5,2) {Compile/Reformulate};
\node (solver1) at (1,1) {CP};
\node (solver2) at (2,1) {MIP};
\node (solver3) at (3,1) {SAT};
\node (solver4) at (4,1) {Other};
\node[shape=rectangle,fill=pantone157-8] (solution1) at (1,0) {Solution};
\node[shape=rectangle,fill=pantone157-8] (solution2) at (2,0) {Solution};
\node[shape=rectangle,fill=pantone157-8] (solution3) at (3,0) {Solution};
\node[shape=rectangle,fill=pantone157-8] (solution4) at (4,0) {Solution};
\draw[-] (problem) -- (human);
\draw[->] (human) -- (model);
\draw[-] (model) -- (compile);
\draw[-] (compile) -- (solver1);
\draw[-] (compile) -- (solver2);
\draw[-] (compile) -- (solver3);
\draw[-] (compile) -- (solver4);
\draw[->] (solver1) -- (solution1);
\draw[->] (solver2) -- (solution2);
\draw[->] (solver3) -- (solution3);
\draw[->] (solver4) -- (solution4);
\end{tikzpicture}
\end{center}
\end{frame}

% \begin{frame}
  % \frametitle{Do It Now!}
  % \begin{itemize}
  % \item Download and install Minizinc
  % \item \url{https://www.minizinc.org/}
  % \end{itemize}
% \end{frame}


\begin{frame}
\frametitle{Why use Puzzles as Examples?}
\begin{itemize}
\item Easy to understand the problem
\item Solvable by hand without specialized knowledge
\item Possible to compare automated to manual solving process
\end{itemize}

\begin{quote}
The puzzle, though inanimate, is presented as a solvable problem without lasting negative consequences, a very low-risk low-reward situation. By being a puzzle, the object is attempting to convince the user that it must be completed.
\end{quote}
{\tiny Source: \href{https://www.everydayrhetoric.com/2020/03/the-rhetoric-of-puzzles/}{Every Day Rhetoric}}
\end{frame}


\mode<all>{
\part{Basic Constraint Propagation}
}
\begin{frame}
\partpage
\end{frame}
\mode<all>{
\begin{frame}
\frametitle{Objectives}
\begin{itemize}
\item Overview of Core Constraint Programming
\item Three Main Concepts
\begin{itemize}
\item Constraint Propagation
\item Global Constraints
\item Customizing Search
\end{itemize}
%\item Get Some Experience with MiniZinc
\item Topics will be treated in more detail in later parts of the school
\item Based on Examples, not Formal Description
\end{itemize}
\end{frame}

\begin{frame}
\frametitle{Outline}
\begin{itemize}
\item Why Constraint Programming?
\item Constraint Propagation
\item Global Constraints
\item Customizing Search
%\item Choices in Modelling
%\item Decomposition
%\item More Global Constraints
%\item Methodology
%\item What is missing?
\end{itemize}
\end{frame}

% \begin{frame}
  % \frametitle{Using MiniZinc IDE}
  % \begin{itemize}
  % \item Developed in the Australian NICTA project
  % \item Maintained by Monash University
  % \item Modelling tool with multiple back-end solvers
  % \item Available from \url{https://www.minizinc.org/}
  % \end{itemize}
% \end{frame}


% \begin{frame}
% \frametitle{Examples in ECLiPSe}
% \begin{itemize}
% \item Open sourced constraint programming language
% \item Development goes back to 1985
% \item ECRC, ICL, IC-Parc, PTL, Cisco
% \item \url{https://eclipseclp.org/}
% \item Specialities
% \begin{itemize}
% \item Develop new solvers for specific domains
% \item Integration with MIP
% \end{itemize}
% \item Not included in bundled MiniZinc IDE
% \item Specialized visualization tools used here
  % \begin{itemize}
  % \item CP-Viz, Simonis et al. 2010
    % \end{itemize}
% \end{itemize}
% \end{frame}

\begin{frame}
\frametitle{Tutorial Based on ECLiPSe ELearning Course}
\begin{itemize}
\item Self-study course in constraint programming
\item Supported by Cisco Systems and Silicon Valley Community Foundation
\item Multi-media format, video lectures, slides, handout etc
\item {\small \url{https://eclipseclp.org/ELearning/index.html}}
\end{itemize}
\end{frame}

\begin{frame}
\frametitle{Also Part of CRT-AI Constraint Week}
\begin{itemize}
\item Annual one week course on CP and Optimization in Ireland
\item Part of national training program for PhD students in AI
\item \url{https://www.crt-ai.ie/}
\end{itemize}
\end{frame}



\begin{frame}
\frametitle{Constraint Programming - in a nutshell}
\begin{itemize}
\item Declarative description of problems with
\begin{itemize}
\item {\em Variables} which range over (finite) sets of values
\item {\em Constraints} over subsets of variables which restrict possible value combinations
\item A {\em solution} is a value assignment which satisfies all constraints
\end{itemize}

\item Constraint propagation/reasoning
\begin{itemize}
\item Removing inconsistent values for variables
\item Detect failure if constraint can not be satisfied
\item Interaction of constraints via shared variables
\item Incomplete
\end{itemize}

\item Search
\begin{itemize}
\item User controlled assignment of values to variables
\item Each step triggers constraint propagation 
\end{itemize}
\item Different domains require/allow different methods
\end{itemize}
\end{frame}

\begin{frame}
  \frametitle{Constraint Programming is Different}
  \begin{itemize}
  \item Declarative Programming
    \begin{itemize}
    \item Concentrate on what you want
      \item Not how to get there
      \item Program != Algorithm
      \item Program = Model
    \end{itemize}
    \item Applied to Combinatorial Problems
      \begin{itemize}
        \item No complete polynomial algorithms known (exist?)
        \item CP less ad-hoc than heuristics
        \item Models can evolve
  \end{itemize}
  \end{itemize}
  \end{frame}
    
\begin{frame}
  \frametitle{A Subtractive Process}
  \begin{textblock}{4}(8,-3)
    \includegraphics[width=4cm]{../introduction/images/stages}
  \end{textblock}
  \vfill
  \begin{quote}
    ``Oh, bosh, as Mr. Ruskin says. Sculpture, per se, is the simplest thing in the world. All you have to do is to take a big chunk of marble and a hammer and chisel, make up your mind what you are about to create and chip off all the marble you don't want.''-Paris Gaulois.
  \end{quote}
  
  {\tiny Source: \url{https://quoteinvestigator.com/2014/06/22/chip-away/}}
\end{frame}

\begin{frame}
\frametitle{Basic Process}
\begin{center}
\begin{tikzpicture}
\node[shape=rectangle,fill=pantone127-4] (problem) at (2,5) {Problem};
\node (human) at (2,4) {Human};
\node[shape=rectangle,fill=pantone127-4] (model) at (2,3) {Model};
\node (solver) at (2,2) {Constraint Solver/Search};
\node[shape=rectangle,fill=pantone157-8] (solution) at (2,1) {Solution};
\draw[-] (problem) -- (human);
\draw[->] (human) -- (model);
\draw[-] (model) -- (solver);
\draw[->] (solver) -- (solution);
\end{tikzpicture}
\end{center}
\end{frame}

\begin{frame}
\frametitle{More Realistic}
\begin{center}
\begin{tikzpicture}[xscale=4.0]
\node[shape=rectangle,fill=pantone127-4] (problem) at (2,5) {Problem};
\node (human) at (2,4) {Human};
\node[shape=rectangle,fill=pantone127-4] (model) at (2,3) {Model};
\node (solver) at (2,2) {Constraint Solver/Search};
\node[shape=rectangle,fill=pantone24-8] (hangs) at (1,1) {Hangs};
\node[shape=rectangle,fill=pantone157-8] (solution) at (2,1) {Solution};
\node[shape=rectangle,fill=pantone24-8] (wrong) at (3,1) {Wrong Solution};
\draw[-] (problem) -- (human);
\draw[->] (human) -- (model);
\draw[-] (model) -- (solver);
\draw[->] (solver) -- (hangs);
\draw[->] (solver) -- (solution);
\draw[->] (solver) -- (wrong);
\draw[->] (hangs) -- (1,4) -- (human);
\draw[->] (wrong) -- (3,4) -- (human);
\end{tikzpicture}
\end{center}
\end{frame}

\begin{frame}
\frametitle{Dual Role of Model}
\begin{itemize}
\item Allows Human to Express Problem
\begin{itemize}
\item Close to Problem Domain
\item Constraints as Abstractions
\end{itemize}
\item Allows Solver to Execute
\begin{itemize}
\item Variables as Communication Mechanism
\item Constraints as Algorithms
\end{itemize}
\end{itemize}
\end{frame}

\begin{frame}
\frametitle{Modelling Frameworks}
\begin{itemize}
\item \href{https://www.minizinc.org/}{MiniZinc} (NICTA, Monash University, Australia)
\item \href{https://github.com/eomahony/Numberjack}{NumberJack} (Insight, Ireland) 
\item \href{https://www-users.york.ac.uk/peter.nightingale/savilerow//}{EssencePrime/SavilleRow} (UK) 
\item \href{https://cpmpy.readthedocs.io/en/latest/}{CPMpy} (KU Leuven)
\item Allow use of multiple back-end solvers
\item Compile model into variants for each solver
\item A priori solver independent model(CP, MIP, SAT)
\end{itemize}
\end{frame} 

\begin{frame}
\frametitle{Framework Process}
\begin{center}
\begin{tikzpicture}[xscale=2]
\node[shape=rectangle,fill=pantone127-4] (problem) at (2.5,5) {Problem};
\node (human) at (2.5,4) {Human};
\node[shape=rectangle,fill=pantone127-4] (model) at (2.5,3) {Model};
\node (compile) at (2.5,2) {Compile/Reformulate};
\node (solver1) at (1,1) {CP};
\node (solver2) at (2,1) {MIP};
\node (solver3) at (3,1) {SAT};
\node (solver4) at (4,1) {Other};
\node[shape=rectangle,fill=pantone157-8] (solution1) at (1,0) {Solution};
\node[shape=rectangle,fill=pantone157-8] (solution2) at (2,0) {Solution};
\node[shape=rectangle,fill=pantone157-8] (solution3) at (3,0) {Solution};
\node[shape=rectangle,fill=pantone157-8] (solution4) at (4,0) {Solution};
\draw[-] (problem) -- (human);
\draw[->] (human) -- (model);
\draw[-] (model) -- (compile);
\draw[-] (compile) -- (solver1);
\draw[-] (compile) -- (solver2);
\draw[-] (compile) -- (solver3);
\draw[-] (compile) -- (solver4);
\draw[->] (solver1) -- (solution1);
\draw[->] (solver2) -- (solution2);
\draw[->] (solver3) -- (solution3);
\draw[->] (solver4) -- (solution4);
\end{tikzpicture}
\end{center}
\end{frame}

% \begin{frame}
  % \frametitle{Do It Now!}
  % \begin{itemize}
  % \item Download and install Minizinc
  % \item \url{https://www.minizinc.org/}
  % \end{itemize}
% \end{frame}


\begin{frame}
\frametitle{Why use Puzzles as Examples?}
\begin{itemize}
\item Easy to understand the problem
\item Solvable by hand without specialized knowledge
\item Possible to compare automated to manual solving process
\end{itemize}

\begin{quote}
The puzzle, though inanimate, is presented as a solvable problem without lasting negative consequences, a very low-risk low-reward situation. By being a puzzle, the object is attempting to convince the user that it must be completed.
\end{quote}
{\tiny Source: \href{https://www.everydayrhetoric.com/2020/03/the-rhetoric-of-puzzles/}{Every Day Rhetoric}}
\end{frame}


\mode<all>{
\part{Basic Constraint Propagation}
}
\begin{frame}
\partpage
\end{frame}
\mode<all>{
\input{../sendmore/body}
}

\mode<all>{
\part{Global Constraints}
}
\begin{frame}
\partpage
\end{frame}
\mode<all>{
\input{../sudoku/body}
}

\clearpage
\mode<all>{
\part{Customizing Search}
}
\begin{frame}
\partpage
\end{frame}
\mode<all>{
\input{../nqueen/body}
}


%% \clearpage
%% \mode<all>{
%%   \part{Problem Decomposition}
%% }
%% \begin{frame}
%%   \partpage
%% \end{frame}
%% \mode<all>{
%%   \input{../party/body}
%%   }

%% \clearpage
%% \mode<all>{
%%   \part{More Global Constraints}
%% }
%% \begin{frame}
%%   \partpage
%% \end{frame}
%% \mode<all>{
%%   \input{../car/body}
%%   }

%% \clearpage
%% \mode<all>{
%%   \part{Methodology}
%% }
%% \begin{frame}
%%   \partpage
%% \end{frame}
%% \mode<all>{
%%   \input{../methodology/body}
%%   }

\clearpage
\mode<all>{
\part{What is missing?}
}
\begin{frame}
\partpage
\end{frame}

\begin{frame}
\frametitle{Many Specialized Topics}
\begin{itemize}
\item How to design efficient core engine
\item Hybrids with LP/MIP tools
\item Hybrids with SAT
\item Symmetry breaking
\item Use of MDD/BDD to encode sets of solutions
\item High level modelling tools
\item Debugging/visualization
\end{itemize}
\end{frame}

\begin{frame}
\frametitle{Reformulation}
\begin{itemize}
\item Just because the user has modelled it this way, it doesn't mean we have to solve it that way
\begin{itemize}
\item Replace some constraint(s) by other, equivalent constraints
\item Because we don't have that constraint in our system
\item For performance
\end{itemize}
\end{itemize}
\end{frame}

\begin{frame}
\frametitle{Learning}
\begin{itemize}
\item While solving the problem we can learn how to strengthen the model/search 
\begin{itemize}
\item Understand which constraints/method contribute to propagation and change schedule
\item Learn no-good constraints by explaining failure
\item Adapt search strategy based on search experience
\end{itemize}
\end{itemize}
\end{frame}

%% \begin{frame}
%% \frametitle{Refined Process}
%% \begin{center}
%% \begin{tikzpicture}
%% \node[shape=rectangle,fill=pantone127-4] (problem) at (3,7) {Problem};
%% \node (human) at (3,6) {Human};
%% \node[shape=rectangle,fill=pantone127-4] (model) at (3,5) {User Model};
%% \node (reformulation) at (3,4) {Reformulation};
%% \node[shape=rectangle,fill=pantone127-4] (implementation) at (3,3) {Implementation Model};
%% \node (solver) at (3,2) {Constraint Solver/Search};
%% \node[shape=rectangle,fill=pantone157-8] (solution) at (3,1) {Solution};
%% \draw[-] (problem) -- (human);
%% \draw[->] (human) -- (model);
%% \draw[-] (model) -- (reformulation);
%% \draw[->] (reformulation) -- (implementation);
%% \draw[-] (implementation) -- (solver);
%% \draw[->] (solver) -- (0,2) -- (0,3) -- (implementation);
%% \draw[->] (solver) -- (solution);
%% \end{tikzpicture}
%% \end{center}
%% \end{frame}

%\begin{frame}
%\frametitle{What is CP actually used for?}
%\url{http://hsimonis.wordpress.com}
%\includegraphics[width=10cm]{../introduction/images/blog}
%\end{frame}

%% \begin{frame}
%% \frametitle{Exercises}
%% \begin{itemize}
%% \item Install Constraint Solver of your Choice
%% \item MiniZinc (MiningZinc) will be used in later sessions
%% \item Run Sendmoremoney, Sudoku, Queens examples
%% \item Can you choose which propagator to use for alldifferent?
%% \item Can you use different search strategies?
%% \end{itemize}
%% \end{frame}

\begin{frame}
\frametitle{More Learning Resources}
\begin{itemize}
\item Survey of Methods, Resources, and Formats for Teaching Constraint Programming
\begin{itemize}
\item by Tejas Santanam, Helmut Simonis
\item \url{https://doi.org/10.48550/arXiv.2403.12717}
\item Based on survey of community for WTCP 2023
\item \url{https://hsimonis.github.io/WTCP2023/}
\end{itemize}
\end{itemize}
\end{frame}


}

\mode<all>{
\part{Global Constraints}
}
\begin{frame}
\partpage
\end{frame}
\mode<all>{
\begin{frame}
\frametitle{Objectives}
\begin{itemize}
\item Overview of Core Constraint Programming
\item Three Main Concepts
\begin{itemize}
\item Constraint Propagation
\item Global Constraints
\item Customizing Search
\end{itemize}
%\item Get Some Experience with MiniZinc
\item Topics will be treated in more detail in later parts of the school
\item Based on Examples, not Formal Description
\end{itemize}
\end{frame}

\begin{frame}
\frametitle{Outline}
\begin{itemize}
\item Why Constraint Programming?
\item Constraint Propagation
\item Global Constraints
\item Customizing Search
%\item Choices in Modelling
%\item Decomposition
%\item More Global Constraints
%\item Methodology
%\item What is missing?
\end{itemize}
\end{frame}

% \begin{frame}
  % \frametitle{Using MiniZinc IDE}
  % \begin{itemize}
  % \item Developed in the Australian NICTA project
  % \item Maintained by Monash University
  % \item Modelling tool with multiple back-end solvers
  % \item Available from \url{https://www.minizinc.org/}
  % \end{itemize}
% \end{frame}


% \begin{frame}
% \frametitle{Examples in ECLiPSe}
% \begin{itemize}
% \item Open sourced constraint programming language
% \item Development goes back to 1985
% \item ECRC, ICL, IC-Parc, PTL, Cisco
% \item \url{https://eclipseclp.org/}
% \item Specialities
% \begin{itemize}
% \item Develop new solvers for specific domains
% \item Integration with MIP
% \end{itemize}
% \item Not included in bundled MiniZinc IDE
% \item Specialized visualization tools used here
  % \begin{itemize}
  % \item CP-Viz, Simonis et al. 2010
    % \end{itemize}
% \end{itemize}
% \end{frame}

\begin{frame}
\frametitle{Tutorial Based on ECLiPSe ELearning Course}
\begin{itemize}
\item Self-study course in constraint programming
\item Supported by Cisco Systems and Silicon Valley Community Foundation
\item Multi-media format, video lectures, slides, handout etc
\item {\small \url{https://eclipseclp.org/ELearning/index.html}}
\end{itemize}
\end{frame}

\begin{frame}
\frametitle{Also Part of CRT-AI Constraint Week}
\begin{itemize}
\item Annual one week course on CP and Optimization in Ireland
\item Part of national training program for PhD students in AI
\item \url{https://www.crt-ai.ie/}
\end{itemize}
\end{frame}



\begin{frame}
\frametitle{Constraint Programming - in a nutshell}
\begin{itemize}
\item Declarative description of problems with
\begin{itemize}
\item {\em Variables} which range over (finite) sets of values
\item {\em Constraints} over subsets of variables which restrict possible value combinations
\item A {\em solution} is a value assignment which satisfies all constraints
\end{itemize}

\item Constraint propagation/reasoning
\begin{itemize}
\item Removing inconsistent values for variables
\item Detect failure if constraint can not be satisfied
\item Interaction of constraints via shared variables
\item Incomplete
\end{itemize}

\item Search
\begin{itemize}
\item User controlled assignment of values to variables
\item Each step triggers constraint propagation 
\end{itemize}
\item Different domains require/allow different methods
\end{itemize}
\end{frame}

\begin{frame}
  \frametitle{Constraint Programming is Different}
  \begin{itemize}
  \item Declarative Programming
    \begin{itemize}
    \item Concentrate on what you want
      \item Not how to get there
      \item Program != Algorithm
      \item Program = Model
    \end{itemize}
    \item Applied to Combinatorial Problems
      \begin{itemize}
        \item No complete polynomial algorithms known (exist?)
        \item CP less ad-hoc than heuristics
        \item Models can evolve
  \end{itemize}
  \end{itemize}
  \end{frame}
    
\begin{frame}
  \frametitle{A Subtractive Process}
  \begin{textblock}{4}(8,-3)
    \includegraphics[width=4cm]{../introduction/images/stages}
  \end{textblock}
  \vfill
  \begin{quote}
    ``Oh, bosh, as Mr. Ruskin says. Sculpture, per se, is the simplest thing in the world. All you have to do is to take a big chunk of marble and a hammer and chisel, make up your mind what you are about to create and chip off all the marble you don't want.''-Paris Gaulois.
  \end{quote}
  
  {\tiny Source: \url{https://quoteinvestigator.com/2014/06/22/chip-away/}}
\end{frame}

\begin{frame}
\frametitle{Basic Process}
\begin{center}
\begin{tikzpicture}
\node[shape=rectangle,fill=pantone127-4] (problem) at (2,5) {Problem};
\node (human) at (2,4) {Human};
\node[shape=rectangle,fill=pantone127-4] (model) at (2,3) {Model};
\node (solver) at (2,2) {Constraint Solver/Search};
\node[shape=rectangle,fill=pantone157-8] (solution) at (2,1) {Solution};
\draw[-] (problem) -- (human);
\draw[->] (human) -- (model);
\draw[-] (model) -- (solver);
\draw[->] (solver) -- (solution);
\end{tikzpicture}
\end{center}
\end{frame}

\begin{frame}
\frametitle{More Realistic}
\begin{center}
\begin{tikzpicture}[xscale=4.0]
\node[shape=rectangle,fill=pantone127-4] (problem) at (2,5) {Problem};
\node (human) at (2,4) {Human};
\node[shape=rectangle,fill=pantone127-4] (model) at (2,3) {Model};
\node (solver) at (2,2) {Constraint Solver/Search};
\node[shape=rectangle,fill=pantone24-8] (hangs) at (1,1) {Hangs};
\node[shape=rectangle,fill=pantone157-8] (solution) at (2,1) {Solution};
\node[shape=rectangle,fill=pantone24-8] (wrong) at (3,1) {Wrong Solution};
\draw[-] (problem) -- (human);
\draw[->] (human) -- (model);
\draw[-] (model) -- (solver);
\draw[->] (solver) -- (hangs);
\draw[->] (solver) -- (solution);
\draw[->] (solver) -- (wrong);
\draw[->] (hangs) -- (1,4) -- (human);
\draw[->] (wrong) -- (3,4) -- (human);
\end{tikzpicture}
\end{center}
\end{frame}

\begin{frame}
\frametitle{Dual Role of Model}
\begin{itemize}
\item Allows Human to Express Problem
\begin{itemize}
\item Close to Problem Domain
\item Constraints as Abstractions
\end{itemize}
\item Allows Solver to Execute
\begin{itemize}
\item Variables as Communication Mechanism
\item Constraints as Algorithms
\end{itemize}
\end{itemize}
\end{frame}

\begin{frame}
\frametitle{Modelling Frameworks}
\begin{itemize}
\item \href{https://www.minizinc.org/}{MiniZinc} (NICTA, Monash University, Australia)
\item \href{https://github.com/eomahony/Numberjack}{NumberJack} (Insight, Ireland) 
\item \href{https://www-users.york.ac.uk/peter.nightingale/savilerow//}{EssencePrime/SavilleRow} (UK) 
\item \href{https://cpmpy.readthedocs.io/en/latest/}{CPMpy} (KU Leuven)
\item Allow use of multiple back-end solvers
\item Compile model into variants for each solver
\item A priori solver independent model(CP, MIP, SAT)
\end{itemize}
\end{frame} 

\begin{frame}
\frametitle{Framework Process}
\begin{center}
\begin{tikzpicture}[xscale=2]
\node[shape=rectangle,fill=pantone127-4] (problem) at (2.5,5) {Problem};
\node (human) at (2.5,4) {Human};
\node[shape=rectangle,fill=pantone127-4] (model) at (2.5,3) {Model};
\node (compile) at (2.5,2) {Compile/Reformulate};
\node (solver1) at (1,1) {CP};
\node (solver2) at (2,1) {MIP};
\node (solver3) at (3,1) {SAT};
\node (solver4) at (4,1) {Other};
\node[shape=rectangle,fill=pantone157-8] (solution1) at (1,0) {Solution};
\node[shape=rectangle,fill=pantone157-8] (solution2) at (2,0) {Solution};
\node[shape=rectangle,fill=pantone157-8] (solution3) at (3,0) {Solution};
\node[shape=rectangle,fill=pantone157-8] (solution4) at (4,0) {Solution};
\draw[-] (problem) -- (human);
\draw[->] (human) -- (model);
\draw[-] (model) -- (compile);
\draw[-] (compile) -- (solver1);
\draw[-] (compile) -- (solver2);
\draw[-] (compile) -- (solver3);
\draw[-] (compile) -- (solver4);
\draw[->] (solver1) -- (solution1);
\draw[->] (solver2) -- (solution2);
\draw[->] (solver3) -- (solution3);
\draw[->] (solver4) -- (solution4);
\end{tikzpicture}
\end{center}
\end{frame}

% \begin{frame}
  % \frametitle{Do It Now!}
  % \begin{itemize}
  % \item Download and install Minizinc
  % \item \url{https://www.minizinc.org/}
  % \end{itemize}
% \end{frame}


\begin{frame}
\frametitle{Why use Puzzles as Examples?}
\begin{itemize}
\item Easy to understand the problem
\item Solvable by hand without specialized knowledge
\item Possible to compare automated to manual solving process
\end{itemize}

\begin{quote}
The puzzle, though inanimate, is presented as a solvable problem without lasting negative consequences, a very low-risk low-reward situation. By being a puzzle, the object is attempting to convince the user that it must be completed.
\end{quote}
{\tiny Source: \href{https://www.everydayrhetoric.com/2020/03/the-rhetoric-of-puzzles/}{Every Day Rhetoric}}
\end{frame}


\mode<all>{
\part{Basic Constraint Propagation}
}
\begin{frame}
\partpage
\end{frame}
\mode<all>{
\input{../sendmore/body}
}

\mode<all>{
\part{Global Constraints}
}
\begin{frame}
\partpage
\end{frame}
\mode<all>{
\input{../sudoku/body}
}

\clearpage
\mode<all>{
\part{Customizing Search}
}
\begin{frame}
\partpage
\end{frame}
\mode<all>{
\input{../nqueen/body}
}


%% \clearpage
%% \mode<all>{
%%   \part{Problem Decomposition}
%% }
%% \begin{frame}
%%   \partpage
%% \end{frame}
%% \mode<all>{
%%   \input{../party/body}
%%   }

%% \clearpage
%% \mode<all>{
%%   \part{More Global Constraints}
%% }
%% \begin{frame}
%%   \partpage
%% \end{frame}
%% \mode<all>{
%%   \input{../car/body}
%%   }

%% \clearpage
%% \mode<all>{
%%   \part{Methodology}
%% }
%% \begin{frame}
%%   \partpage
%% \end{frame}
%% \mode<all>{
%%   \input{../methodology/body}
%%   }

\clearpage
\mode<all>{
\part{What is missing?}
}
\begin{frame}
\partpage
\end{frame}

\begin{frame}
\frametitle{Many Specialized Topics}
\begin{itemize}
\item How to design efficient core engine
\item Hybrids with LP/MIP tools
\item Hybrids with SAT
\item Symmetry breaking
\item Use of MDD/BDD to encode sets of solutions
\item High level modelling tools
\item Debugging/visualization
\end{itemize}
\end{frame}

\begin{frame}
\frametitle{Reformulation}
\begin{itemize}
\item Just because the user has modelled it this way, it doesn't mean we have to solve it that way
\begin{itemize}
\item Replace some constraint(s) by other, equivalent constraints
\item Because we don't have that constraint in our system
\item For performance
\end{itemize}
\end{itemize}
\end{frame}

\begin{frame}
\frametitle{Learning}
\begin{itemize}
\item While solving the problem we can learn how to strengthen the model/search 
\begin{itemize}
\item Understand which constraints/method contribute to propagation and change schedule
\item Learn no-good constraints by explaining failure
\item Adapt search strategy based on search experience
\end{itemize}
\end{itemize}
\end{frame}

%% \begin{frame}
%% \frametitle{Refined Process}
%% \begin{center}
%% \begin{tikzpicture}
%% \node[shape=rectangle,fill=pantone127-4] (problem) at (3,7) {Problem};
%% \node (human) at (3,6) {Human};
%% \node[shape=rectangle,fill=pantone127-4] (model) at (3,5) {User Model};
%% \node (reformulation) at (3,4) {Reformulation};
%% \node[shape=rectangle,fill=pantone127-4] (implementation) at (3,3) {Implementation Model};
%% \node (solver) at (3,2) {Constraint Solver/Search};
%% \node[shape=rectangle,fill=pantone157-8] (solution) at (3,1) {Solution};
%% \draw[-] (problem) -- (human);
%% \draw[->] (human) -- (model);
%% \draw[-] (model) -- (reformulation);
%% \draw[->] (reformulation) -- (implementation);
%% \draw[-] (implementation) -- (solver);
%% \draw[->] (solver) -- (0,2) -- (0,3) -- (implementation);
%% \draw[->] (solver) -- (solution);
%% \end{tikzpicture}
%% \end{center}
%% \end{frame}

%\begin{frame}
%\frametitle{What is CP actually used for?}
%\url{http://hsimonis.wordpress.com}
%\includegraphics[width=10cm]{../introduction/images/blog}
%\end{frame}

%% \begin{frame}
%% \frametitle{Exercises}
%% \begin{itemize}
%% \item Install Constraint Solver of your Choice
%% \item MiniZinc (MiningZinc) will be used in later sessions
%% \item Run Sendmoremoney, Sudoku, Queens examples
%% \item Can you choose which propagator to use for alldifferent?
%% \item Can you use different search strategies?
%% \end{itemize}
%% \end{frame}

\begin{frame}
\frametitle{More Learning Resources}
\begin{itemize}
\item Survey of Methods, Resources, and Formats for Teaching Constraint Programming
\begin{itemize}
\item by Tejas Santanam, Helmut Simonis
\item \url{https://doi.org/10.48550/arXiv.2403.12717}
\item Based on survey of community for WTCP 2023
\item \url{https://hsimonis.github.io/WTCP2023/}
\end{itemize}
\end{itemize}
\end{frame}


}

\clearpage
\mode<all>{
\part{Customizing Search}
}
\begin{frame}
\partpage
\end{frame}
\mode<all>{
\begin{frame}
\frametitle{Objectives}
\begin{itemize}
\item Overview of Core Constraint Programming
\item Three Main Concepts
\begin{itemize}
\item Constraint Propagation
\item Global Constraints
\item Customizing Search
\end{itemize}
%\item Get Some Experience with MiniZinc
\item Topics will be treated in more detail in later parts of the school
\item Based on Examples, not Formal Description
\end{itemize}
\end{frame}

\begin{frame}
\frametitle{Outline}
\begin{itemize}
\item Why Constraint Programming?
\item Constraint Propagation
\item Global Constraints
\item Customizing Search
%\item Choices in Modelling
%\item Decomposition
%\item More Global Constraints
%\item Methodology
%\item What is missing?
\end{itemize}
\end{frame}

% \begin{frame}
  % \frametitle{Using MiniZinc IDE}
  % \begin{itemize}
  % \item Developed in the Australian NICTA project
  % \item Maintained by Monash University
  % \item Modelling tool with multiple back-end solvers
  % \item Available from \url{https://www.minizinc.org/}
  % \end{itemize}
% \end{frame}


% \begin{frame}
% \frametitle{Examples in ECLiPSe}
% \begin{itemize}
% \item Open sourced constraint programming language
% \item Development goes back to 1985
% \item ECRC, ICL, IC-Parc, PTL, Cisco
% \item \url{https://eclipseclp.org/}
% \item Specialities
% \begin{itemize}
% \item Develop new solvers for specific domains
% \item Integration with MIP
% \end{itemize}
% \item Not included in bundled MiniZinc IDE
% \item Specialized visualization tools used here
  % \begin{itemize}
  % \item CP-Viz, Simonis et al. 2010
    % \end{itemize}
% \end{itemize}
% \end{frame}

\begin{frame}
\frametitle{Tutorial Based on ECLiPSe ELearning Course}
\begin{itemize}
\item Self-study course in constraint programming
\item Supported by Cisco Systems and Silicon Valley Community Foundation
\item Multi-media format, video lectures, slides, handout etc
\item {\small \url{https://eclipseclp.org/ELearning/index.html}}
\end{itemize}
\end{frame}

\begin{frame}
\frametitle{Also Part of CRT-AI Constraint Week}
\begin{itemize}
\item Annual one week course on CP and Optimization in Ireland
\item Part of national training program for PhD students in AI
\item \url{https://www.crt-ai.ie/}
\end{itemize}
\end{frame}



\begin{frame}
\frametitle{Constraint Programming - in a nutshell}
\begin{itemize}
\item Declarative description of problems with
\begin{itemize}
\item {\em Variables} which range over (finite) sets of values
\item {\em Constraints} over subsets of variables which restrict possible value combinations
\item A {\em solution} is a value assignment which satisfies all constraints
\end{itemize}

\item Constraint propagation/reasoning
\begin{itemize}
\item Removing inconsistent values for variables
\item Detect failure if constraint can not be satisfied
\item Interaction of constraints via shared variables
\item Incomplete
\end{itemize}

\item Search
\begin{itemize}
\item User controlled assignment of values to variables
\item Each step triggers constraint propagation 
\end{itemize}
\item Different domains require/allow different methods
\end{itemize}
\end{frame}

\begin{frame}
  \frametitle{Constraint Programming is Different}
  \begin{itemize}
  \item Declarative Programming
    \begin{itemize}
    \item Concentrate on what you want
      \item Not how to get there
      \item Program != Algorithm
      \item Program = Model
    \end{itemize}
    \item Applied to Combinatorial Problems
      \begin{itemize}
        \item No complete polynomial algorithms known (exist?)
        \item CP less ad-hoc than heuristics
        \item Models can evolve
  \end{itemize}
  \end{itemize}
  \end{frame}
    
\begin{frame}
  \frametitle{A Subtractive Process}
  \begin{textblock}{4}(8,-3)
    \includegraphics[width=4cm]{../introduction/images/stages}
  \end{textblock}
  \vfill
  \begin{quote}
    ``Oh, bosh, as Mr. Ruskin says. Sculpture, per se, is the simplest thing in the world. All you have to do is to take a big chunk of marble and a hammer and chisel, make up your mind what you are about to create and chip off all the marble you don't want.''-Paris Gaulois.
  \end{quote}
  
  {\tiny Source: \url{https://quoteinvestigator.com/2014/06/22/chip-away/}}
\end{frame}

\begin{frame}
\frametitle{Basic Process}
\begin{center}
\begin{tikzpicture}
\node[shape=rectangle,fill=pantone127-4] (problem) at (2,5) {Problem};
\node (human) at (2,4) {Human};
\node[shape=rectangle,fill=pantone127-4] (model) at (2,3) {Model};
\node (solver) at (2,2) {Constraint Solver/Search};
\node[shape=rectangle,fill=pantone157-8] (solution) at (2,1) {Solution};
\draw[-] (problem) -- (human);
\draw[->] (human) -- (model);
\draw[-] (model) -- (solver);
\draw[->] (solver) -- (solution);
\end{tikzpicture}
\end{center}
\end{frame}

\begin{frame}
\frametitle{More Realistic}
\begin{center}
\begin{tikzpicture}[xscale=4.0]
\node[shape=rectangle,fill=pantone127-4] (problem) at (2,5) {Problem};
\node (human) at (2,4) {Human};
\node[shape=rectangle,fill=pantone127-4] (model) at (2,3) {Model};
\node (solver) at (2,2) {Constraint Solver/Search};
\node[shape=rectangle,fill=pantone24-8] (hangs) at (1,1) {Hangs};
\node[shape=rectangle,fill=pantone157-8] (solution) at (2,1) {Solution};
\node[shape=rectangle,fill=pantone24-8] (wrong) at (3,1) {Wrong Solution};
\draw[-] (problem) -- (human);
\draw[->] (human) -- (model);
\draw[-] (model) -- (solver);
\draw[->] (solver) -- (hangs);
\draw[->] (solver) -- (solution);
\draw[->] (solver) -- (wrong);
\draw[->] (hangs) -- (1,4) -- (human);
\draw[->] (wrong) -- (3,4) -- (human);
\end{tikzpicture}
\end{center}
\end{frame}

\begin{frame}
\frametitle{Dual Role of Model}
\begin{itemize}
\item Allows Human to Express Problem
\begin{itemize}
\item Close to Problem Domain
\item Constraints as Abstractions
\end{itemize}
\item Allows Solver to Execute
\begin{itemize}
\item Variables as Communication Mechanism
\item Constraints as Algorithms
\end{itemize}
\end{itemize}
\end{frame}

\begin{frame}
\frametitle{Modelling Frameworks}
\begin{itemize}
\item \href{https://www.minizinc.org/}{MiniZinc} (NICTA, Monash University, Australia)
\item \href{https://github.com/eomahony/Numberjack}{NumberJack} (Insight, Ireland) 
\item \href{https://www-users.york.ac.uk/peter.nightingale/savilerow//}{EssencePrime/SavilleRow} (UK) 
\item \href{https://cpmpy.readthedocs.io/en/latest/}{CPMpy} (KU Leuven)
\item Allow use of multiple back-end solvers
\item Compile model into variants for each solver
\item A priori solver independent model(CP, MIP, SAT)
\end{itemize}
\end{frame} 

\begin{frame}
\frametitle{Framework Process}
\begin{center}
\begin{tikzpicture}[xscale=2]
\node[shape=rectangle,fill=pantone127-4] (problem) at (2.5,5) {Problem};
\node (human) at (2.5,4) {Human};
\node[shape=rectangle,fill=pantone127-4] (model) at (2.5,3) {Model};
\node (compile) at (2.5,2) {Compile/Reformulate};
\node (solver1) at (1,1) {CP};
\node (solver2) at (2,1) {MIP};
\node (solver3) at (3,1) {SAT};
\node (solver4) at (4,1) {Other};
\node[shape=rectangle,fill=pantone157-8] (solution1) at (1,0) {Solution};
\node[shape=rectangle,fill=pantone157-8] (solution2) at (2,0) {Solution};
\node[shape=rectangle,fill=pantone157-8] (solution3) at (3,0) {Solution};
\node[shape=rectangle,fill=pantone157-8] (solution4) at (4,0) {Solution};
\draw[-] (problem) -- (human);
\draw[->] (human) -- (model);
\draw[-] (model) -- (compile);
\draw[-] (compile) -- (solver1);
\draw[-] (compile) -- (solver2);
\draw[-] (compile) -- (solver3);
\draw[-] (compile) -- (solver4);
\draw[->] (solver1) -- (solution1);
\draw[->] (solver2) -- (solution2);
\draw[->] (solver3) -- (solution3);
\draw[->] (solver4) -- (solution4);
\end{tikzpicture}
\end{center}
\end{frame}

% \begin{frame}
  % \frametitle{Do It Now!}
  % \begin{itemize}
  % \item Download and install Minizinc
  % \item \url{https://www.minizinc.org/}
  % \end{itemize}
% \end{frame}


\begin{frame}
\frametitle{Why use Puzzles as Examples?}
\begin{itemize}
\item Easy to understand the problem
\item Solvable by hand without specialized knowledge
\item Possible to compare automated to manual solving process
\end{itemize}

\begin{quote}
The puzzle, though inanimate, is presented as a solvable problem without lasting negative consequences, a very low-risk low-reward situation. By being a puzzle, the object is attempting to convince the user that it must be completed.
\end{quote}
{\tiny Source: \href{https://www.everydayrhetoric.com/2020/03/the-rhetoric-of-puzzles/}{Every Day Rhetoric}}
\end{frame}


\mode<all>{
\part{Basic Constraint Propagation}
}
\begin{frame}
\partpage
\end{frame}
\mode<all>{
\input{../sendmore/body}
}

\mode<all>{
\part{Global Constraints}
}
\begin{frame}
\partpage
\end{frame}
\mode<all>{
\input{../sudoku/body}
}

\clearpage
\mode<all>{
\part{Customizing Search}
}
\begin{frame}
\partpage
\end{frame}
\mode<all>{
\input{../nqueen/body}
}


%% \clearpage
%% \mode<all>{
%%   \part{Problem Decomposition}
%% }
%% \begin{frame}
%%   \partpage
%% \end{frame}
%% \mode<all>{
%%   \input{../party/body}
%%   }

%% \clearpage
%% \mode<all>{
%%   \part{More Global Constraints}
%% }
%% \begin{frame}
%%   \partpage
%% \end{frame}
%% \mode<all>{
%%   \input{../car/body}
%%   }

%% \clearpage
%% \mode<all>{
%%   \part{Methodology}
%% }
%% \begin{frame}
%%   \partpage
%% \end{frame}
%% \mode<all>{
%%   \input{../methodology/body}
%%   }

\clearpage
\mode<all>{
\part{What is missing?}
}
\begin{frame}
\partpage
\end{frame}

\begin{frame}
\frametitle{Many Specialized Topics}
\begin{itemize}
\item How to design efficient core engine
\item Hybrids with LP/MIP tools
\item Hybrids with SAT
\item Symmetry breaking
\item Use of MDD/BDD to encode sets of solutions
\item High level modelling tools
\item Debugging/visualization
\end{itemize}
\end{frame}

\begin{frame}
\frametitle{Reformulation}
\begin{itemize}
\item Just because the user has modelled it this way, it doesn't mean we have to solve it that way
\begin{itemize}
\item Replace some constraint(s) by other, equivalent constraints
\item Because we don't have that constraint in our system
\item For performance
\end{itemize}
\end{itemize}
\end{frame}

\begin{frame}
\frametitle{Learning}
\begin{itemize}
\item While solving the problem we can learn how to strengthen the model/search 
\begin{itemize}
\item Understand which constraints/method contribute to propagation and change schedule
\item Learn no-good constraints by explaining failure
\item Adapt search strategy based on search experience
\end{itemize}
\end{itemize}
\end{frame}

%% \begin{frame}
%% \frametitle{Refined Process}
%% \begin{center}
%% \begin{tikzpicture}
%% \node[shape=rectangle,fill=pantone127-4] (problem) at (3,7) {Problem};
%% \node (human) at (3,6) {Human};
%% \node[shape=rectangle,fill=pantone127-4] (model) at (3,5) {User Model};
%% \node (reformulation) at (3,4) {Reformulation};
%% \node[shape=rectangle,fill=pantone127-4] (implementation) at (3,3) {Implementation Model};
%% \node (solver) at (3,2) {Constraint Solver/Search};
%% \node[shape=rectangle,fill=pantone157-8] (solution) at (3,1) {Solution};
%% \draw[-] (problem) -- (human);
%% \draw[->] (human) -- (model);
%% \draw[-] (model) -- (reformulation);
%% \draw[->] (reformulation) -- (implementation);
%% \draw[-] (implementation) -- (solver);
%% \draw[->] (solver) -- (0,2) -- (0,3) -- (implementation);
%% \draw[->] (solver) -- (solution);
%% \end{tikzpicture}
%% \end{center}
%% \end{frame}

%\begin{frame}
%\frametitle{What is CP actually used for?}
%\url{http://hsimonis.wordpress.com}
%\includegraphics[width=10cm]{../introduction/images/blog}
%\end{frame}

%% \begin{frame}
%% \frametitle{Exercises}
%% \begin{itemize}
%% \item Install Constraint Solver of your Choice
%% \item MiniZinc (MiningZinc) will be used in later sessions
%% \item Run Sendmoremoney, Sudoku, Queens examples
%% \item Can you choose which propagator to use for alldifferent?
%% \item Can you use different search strategies?
%% \end{itemize}
%% \end{frame}

\begin{frame}
\frametitle{More Learning Resources}
\begin{itemize}
\item Survey of Methods, Resources, and Formats for Teaching Constraint Programming
\begin{itemize}
\item by Tejas Santanam, Helmut Simonis
\item \url{https://doi.org/10.48550/arXiv.2403.12717}
\item Based on survey of community for WTCP 2023
\item \url{https://hsimonis.github.io/WTCP2023/}
\end{itemize}
\end{itemize}
\end{frame}


}


%% \clearpage
%% \mode<all>{
%%   \part{Problem Decomposition}
%% }
%% \begin{frame}
%%   \partpage
%% \end{frame}
%% \mode<all>{
%%   \begin{frame}
\frametitle{Objectives}
\begin{itemize}
\item Overview of Core Constraint Programming
\item Three Main Concepts
\begin{itemize}
\item Constraint Propagation
\item Global Constraints
\item Customizing Search
\end{itemize}
%\item Get Some Experience with MiniZinc
\item Topics will be treated in more detail in later parts of the school
\item Based on Examples, not Formal Description
\end{itemize}
\end{frame}

\begin{frame}
\frametitle{Outline}
\begin{itemize}
\item Why Constraint Programming?
\item Constraint Propagation
\item Global Constraints
\item Customizing Search
%\item Choices in Modelling
%\item Decomposition
%\item More Global Constraints
%\item Methodology
%\item What is missing?
\end{itemize}
\end{frame}

% \begin{frame}
  % \frametitle{Using MiniZinc IDE}
  % \begin{itemize}
  % \item Developed in the Australian NICTA project
  % \item Maintained by Monash University
  % \item Modelling tool with multiple back-end solvers
  % \item Available from \url{https://www.minizinc.org/}
  % \end{itemize}
% \end{frame}


% \begin{frame}
% \frametitle{Examples in ECLiPSe}
% \begin{itemize}
% \item Open sourced constraint programming language
% \item Development goes back to 1985
% \item ECRC, ICL, IC-Parc, PTL, Cisco
% \item \url{https://eclipseclp.org/}
% \item Specialities
% \begin{itemize}
% \item Develop new solvers for specific domains
% \item Integration with MIP
% \end{itemize}
% \item Not included in bundled MiniZinc IDE
% \item Specialized visualization tools used here
  % \begin{itemize}
  % \item CP-Viz, Simonis et al. 2010
    % \end{itemize}
% \end{itemize}
% \end{frame}

\begin{frame}
\frametitle{Tutorial Based on ECLiPSe ELearning Course}
\begin{itemize}
\item Self-study course in constraint programming
\item Supported by Cisco Systems and Silicon Valley Community Foundation
\item Multi-media format, video lectures, slides, handout etc
\item {\small \url{https://eclipseclp.org/ELearning/index.html}}
\end{itemize}
\end{frame}

\begin{frame}
\frametitle{Also Part of CRT-AI Constraint Week}
\begin{itemize}
\item Annual one week course on CP and Optimization in Ireland
\item Part of national training program for PhD students in AI
\item \url{https://www.crt-ai.ie/}
\end{itemize}
\end{frame}



\begin{frame}
\frametitle{Constraint Programming - in a nutshell}
\begin{itemize}
\item Declarative description of problems with
\begin{itemize}
\item {\em Variables} which range over (finite) sets of values
\item {\em Constraints} over subsets of variables which restrict possible value combinations
\item A {\em solution} is a value assignment which satisfies all constraints
\end{itemize}

\item Constraint propagation/reasoning
\begin{itemize}
\item Removing inconsistent values for variables
\item Detect failure if constraint can not be satisfied
\item Interaction of constraints via shared variables
\item Incomplete
\end{itemize}

\item Search
\begin{itemize}
\item User controlled assignment of values to variables
\item Each step triggers constraint propagation 
\end{itemize}
\item Different domains require/allow different methods
\end{itemize}
\end{frame}

\begin{frame}
  \frametitle{Constraint Programming is Different}
  \begin{itemize}
  \item Declarative Programming
    \begin{itemize}
    \item Concentrate on what you want
      \item Not how to get there
      \item Program != Algorithm
      \item Program = Model
    \end{itemize}
    \item Applied to Combinatorial Problems
      \begin{itemize}
        \item No complete polynomial algorithms known (exist?)
        \item CP less ad-hoc than heuristics
        \item Models can evolve
  \end{itemize}
  \end{itemize}
  \end{frame}
    
\begin{frame}
  \frametitle{A Subtractive Process}
  \begin{textblock}{4}(8,-3)
    \includegraphics[width=4cm]{../introduction/images/stages}
  \end{textblock}
  \vfill
  \begin{quote}
    ``Oh, bosh, as Mr. Ruskin says. Sculpture, per se, is the simplest thing in the world. All you have to do is to take a big chunk of marble and a hammer and chisel, make up your mind what you are about to create and chip off all the marble you don't want.''-Paris Gaulois.
  \end{quote}
  
  {\tiny Source: \url{https://quoteinvestigator.com/2014/06/22/chip-away/}}
\end{frame}

\begin{frame}
\frametitle{Basic Process}
\begin{center}
\begin{tikzpicture}
\node[shape=rectangle,fill=pantone127-4] (problem) at (2,5) {Problem};
\node (human) at (2,4) {Human};
\node[shape=rectangle,fill=pantone127-4] (model) at (2,3) {Model};
\node (solver) at (2,2) {Constraint Solver/Search};
\node[shape=rectangle,fill=pantone157-8] (solution) at (2,1) {Solution};
\draw[-] (problem) -- (human);
\draw[->] (human) -- (model);
\draw[-] (model) -- (solver);
\draw[->] (solver) -- (solution);
\end{tikzpicture}
\end{center}
\end{frame}

\begin{frame}
\frametitle{More Realistic}
\begin{center}
\begin{tikzpicture}[xscale=4.0]
\node[shape=rectangle,fill=pantone127-4] (problem) at (2,5) {Problem};
\node (human) at (2,4) {Human};
\node[shape=rectangle,fill=pantone127-4] (model) at (2,3) {Model};
\node (solver) at (2,2) {Constraint Solver/Search};
\node[shape=rectangle,fill=pantone24-8] (hangs) at (1,1) {Hangs};
\node[shape=rectangle,fill=pantone157-8] (solution) at (2,1) {Solution};
\node[shape=rectangle,fill=pantone24-8] (wrong) at (3,1) {Wrong Solution};
\draw[-] (problem) -- (human);
\draw[->] (human) -- (model);
\draw[-] (model) -- (solver);
\draw[->] (solver) -- (hangs);
\draw[->] (solver) -- (solution);
\draw[->] (solver) -- (wrong);
\draw[->] (hangs) -- (1,4) -- (human);
\draw[->] (wrong) -- (3,4) -- (human);
\end{tikzpicture}
\end{center}
\end{frame}

\begin{frame}
\frametitle{Dual Role of Model}
\begin{itemize}
\item Allows Human to Express Problem
\begin{itemize}
\item Close to Problem Domain
\item Constraints as Abstractions
\end{itemize}
\item Allows Solver to Execute
\begin{itemize}
\item Variables as Communication Mechanism
\item Constraints as Algorithms
\end{itemize}
\end{itemize}
\end{frame}

\begin{frame}
\frametitle{Modelling Frameworks}
\begin{itemize}
\item \href{https://www.minizinc.org/}{MiniZinc} (NICTA, Monash University, Australia)
\item \href{https://github.com/eomahony/Numberjack}{NumberJack} (Insight, Ireland) 
\item \href{https://www-users.york.ac.uk/peter.nightingale/savilerow//}{EssencePrime/SavilleRow} (UK) 
\item \href{https://cpmpy.readthedocs.io/en/latest/}{CPMpy} (KU Leuven)
\item Allow use of multiple back-end solvers
\item Compile model into variants for each solver
\item A priori solver independent model(CP, MIP, SAT)
\end{itemize}
\end{frame} 

\begin{frame}
\frametitle{Framework Process}
\begin{center}
\begin{tikzpicture}[xscale=2]
\node[shape=rectangle,fill=pantone127-4] (problem) at (2.5,5) {Problem};
\node (human) at (2.5,4) {Human};
\node[shape=rectangle,fill=pantone127-4] (model) at (2.5,3) {Model};
\node (compile) at (2.5,2) {Compile/Reformulate};
\node (solver1) at (1,1) {CP};
\node (solver2) at (2,1) {MIP};
\node (solver3) at (3,1) {SAT};
\node (solver4) at (4,1) {Other};
\node[shape=rectangle,fill=pantone157-8] (solution1) at (1,0) {Solution};
\node[shape=rectangle,fill=pantone157-8] (solution2) at (2,0) {Solution};
\node[shape=rectangle,fill=pantone157-8] (solution3) at (3,0) {Solution};
\node[shape=rectangle,fill=pantone157-8] (solution4) at (4,0) {Solution};
\draw[-] (problem) -- (human);
\draw[->] (human) -- (model);
\draw[-] (model) -- (compile);
\draw[-] (compile) -- (solver1);
\draw[-] (compile) -- (solver2);
\draw[-] (compile) -- (solver3);
\draw[-] (compile) -- (solver4);
\draw[->] (solver1) -- (solution1);
\draw[->] (solver2) -- (solution2);
\draw[->] (solver3) -- (solution3);
\draw[->] (solver4) -- (solution4);
\end{tikzpicture}
\end{center}
\end{frame}

% \begin{frame}
  % \frametitle{Do It Now!}
  % \begin{itemize}
  % \item Download and install Minizinc
  % \item \url{https://www.minizinc.org/}
  % \end{itemize}
% \end{frame}


\begin{frame}
\frametitle{Why use Puzzles as Examples?}
\begin{itemize}
\item Easy to understand the problem
\item Solvable by hand without specialized knowledge
\item Possible to compare automated to manual solving process
\end{itemize}

\begin{quote}
The puzzle, though inanimate, is presented as a solvable problem without lasting negative consequences, a very low-risk low-reward situation. By being a puzzle, the object is attempting to convince the user that it must be completed.
\end{quote}
{\tiny Source: \href{https://www.everydayrhetoric.com/2020/03/the-rhetoric-of-puzzles/}{Every Day Rhetoric}}
\end{frame}


\mode<all>{
\part{Basic Constraint Propagation}
}
\begin{frame}
\partpage
\end{frame}
\mode<all>{
\input{../sendmore/body}
}

\mode<all>{
\part{Global Constraints}
}
\begin{frame}
\partpage
\end{frame}
\mode<all>{
\input{../sudoku/body}
}

\clearpage
\mode<all>{
\part{Customizing Search}
}
\begin{frame}
\partpage
\end{frame}
\mode<all>{
\input{../nqueen/body}
}


%% \clearpage
%% \mode<all>{
%%   \part{Problem Decomposition}
%% }
%% \begin{frame}
%%   \partpage
%% \end{frame}
%% \mode<all>{
%%   \input{../party/body}
%%   }

%% \clearpage
%% \mode<all>{
%%   \part{More Global Constraints}
%% }
%% \begin{frame}
%%   \partpage
%% \end{frame}
%% \mode<all>{
%%   \input{../car/body}
%%   }

%% \clearpage
%% \mode<all>{
%%   \part{Methodology}
%% }
%% \begin{frame}
%%   \partpage
%% \end{frame}
%% \mode<all>{
%%   \input{../methodology/body}
%%   }

\clearpage
\mode<all>{
\part{What is missing?}
}
\begin{frame}
\partpage
\end{frame}

\begin{frame}
\frametitle{Many Specialized Topics}
\begin{itemize}
\item How to design efficient core engine
\item Hybrids with LP/MIP tools
\item Hybrids with SAT
\item Symmetry breaking
\item Use of MDD/BDD to encode sets of solutions
\item High level modelling tools
\item Debugging/visualization
\end{itemize}
\end{frame}

\begin{frame}
\frametitle{Reformulation}
\begin{itemize}
\item Just because the user has modelled it this way, it doesn't mean we have to solve it that way
\begin{itemize}
\item Replace some constraint(s) by other, equivalent constraints
\item Because we don't have that constraint in our system
\item For performance
\end{itemize}
\end{itemize}
\end{frame}

\begin{frame}
\frametitle{Learning}
\begin{itemize}
\item While solving the problem we can learn how to strengthen the model/search 
\begin{itemize}
\item Understand which constraints/method contribute to propagation and change schedule
\item Learn no-good constraints by explaining failure
\item Adapt search strategy based on search experience
\end{itemize}
\end{itemize}
\end{frame}

%% \begin{frame}
%% \frametitle{Refined Process}
%% \begin{center}
%% \begin{tikzpicture}
%% \node[shape=rectangle,fill=pantone127-4] (problem) at (3,7) {Problem};
%% \node (human) at (3,6) {Human};
%% \node[shape=rectangle,fill=pantone127-4] (model) at (3,5) {User Model};
%% \node (reformulation) at (3,4) {Reformulation};
%% \node[shape=rectangle,fill=pantone127-4] (implementation) at (3,3) {Implementation Model};
%% \node (solver) at (3,2) {Constraint Solver/Search};
%% \node[shape=rectangle,fill=pantone157-8] (solution) at (3,1) {Solution};
%% \draw[-] (problem) -- (human);
%% \draw[->] (human) -- (model);
%% \draw[-] (model) -- (reformulation);
%% \draw[->] (reformulation) -- (implementation);
%% \draw[-] (implementation) -- (solver);
%% \draw[->] (solver) -- (0,2) -- (0,3) -- (implementation);
%% \draw[->] (solver) -- (solution);
%% \end{tikzpicture}
%% \end{center}
%% \end{frame}

%\begin{frame}
%\frametitle{What is CP actually used for?}
%\url{http://hsimonis.wordpress.com}
%\includegraphics[width=10cm]{../introduction/images/blog}
%\end{frame}

%% \begin{frame}
%% \frametitle{Exercises}
%% \begin{itemize}
%% \item Install Constraint Solver of your Choice
%% \item MiniZinc (MiningZinc) will be used in later sessions
%% \item Run Sendmoremoney, Sudoku, Queens examples
%% \item Can you choose which propagator to use for alldifferent?
%% \item Can you use different search strategies?
%% \end{itemize}
%% \end{frame}

\begin{frame}
\frametitle{More Learning Resources}
\begin{itemize}
\item Survey of Methods, Resources, and Formats for Teaching Constraint Programming
\begin{itemize}
\item by Tejas Santanam, Helmut Simonis
\item \url{https://doi.org/10.48550/arXiv.2403.12717}
\item Based on survey of community for WTCP 2023
\item \url{https://hsimonis.github.io/WTCP2023/}
\end{itemize}
\end{itemize}
\end{frame}


%%   }

%% \clearpage
%% \mode<all>{
%%   \part{More Global Constraints}
%% }
%% \begin{frame}
%%   \partpage
%% \end{frame}
%% \mode<all>{
%%   \begin{frame}
\frametitle{Objectives}
\begin{itemize}
\item Overview of Core Constraint Programming
\item Three Main Concepts
\begin{itemize}
\item Constraint Propagation
\item Global Constraints
\item Customizing Search
\end{itemize}
%\item Get Some Experience with MiniZinc
\item Topics will be treated in more detail in later parts of the school
\item Based on Examples, not Formal Description
\end{itemize}
\end{frame}

\begin{frame}
\frametitle{Outline}
\begin{itemize}
\item Why Constraint Programming?
\item Constraint Propagation
\item Global Constraints
\item Customizing Search
%\item Choices in Modelling
%\item Decomposition
%\item More Global Constraints
%\item Methodology
%\item What is missing?
\end{itemize}
\end{frame}

% \begin{frame}
  % \frametitle{Using MiniZinc IDE}
  % \begin{itemize}
  % \item Developed in the Australian NICTA project
  % \item Maintained by Monash University
  % \item Modelling tool with multiple back-end solvers
  % \item Available from \url{https://www.minizinc.org/}
  % \end{itemize}
% \end{frame}


% \begin{frame}
% \frametitle{Examples in ECLiPSe}
% \begin{itemize}
% \item Open sourced constraint programming language
% \item Development goes back to 1985
% \item ECRC, ICL, IC-Parc, PTL, Cisco
% \item \url{https://eclipseclp.org/}
% \item Specialities
% \begin{itemize}
% \item Develop new solvers for specific domains
% \item Integration with MIP
% \end{itemize}
% \item Not included in bundled MiniZinc IDE
% \item Specialized visualization tools used here
  % \begin{itemize}
  % \item CP-Viz, Simonis et al. 2010
    % \end{itemize}
% \end{itemize}
% \end{frame}

\begin{frame}
\frametitle{Tutorial Based on ECLiPSe ELearning Course}
\begin{itemize}
\item Self-study course in constraint programming
\item Supported by Cisco Systems and Silicon Valley Community Foundation
\item Multi-media format, video lectures, slides, handout etc
\item {\small \url{https://eclipseclp.org/ELearning/index.html}}
\end{itemize}
\end{frame}

\begin{frame}
\frametitle{Also Part of CRT-AI Constraint Week}
\begin{itemize}
\item Annual one week course on CP and Optimization in Ireland
\item Part of national training program for PhD students in AI
\item \url{https://www.crt-ai.ie/}
\end{itemize}
\end{frame}



\begin{frame}
\frametitle{Constraint Programming - in a nutshell}
\begin{itemize}
\item Declarative description of problems with
\begin{itemize}
\item {\em Variables} which range over (finite) sets of values
\item {\em Constraints} over subsets of variables which restrict possible value combinations
\item A {\em solution} is a value assignment which satisfies all constraints
\end{itemize}

\item Constraint propagation/reasoning
\begin{itemize}
\item Removing inconsistent values for variables
\item Detect failure if constraint can not be satisfied
\item Interaction of constraints via shared variables
\item Incomplete
\end{itemize}

\item Search
\begin{itemize}
\item User controlled assignment of values to variables
\item Each step triggers constraint propagation 
\end{itemize}
\item Different domains require/allow different methods
\end{itemize}
\end{frame}

\begin{frame}
  \frametitle{Constraint Programming is Different}
  \begin{itemize}
  \item Declarative Programming
    \begin{itemize}
    \item Concentrate on what you want
      \item Not how to get there
      \item Program != Algorithm
      \item Program = Model
    \end{itemize}
    \item Applied to Combinatorial Problems
      \begin{itemize}
        \item No complete polynomial algorithms known (exist?)
        \item CP less ad-hoc than heuristics
        \item Models can evolve
  \end{itemize}
  \end{itemize}
  \end{frame}
    
\begin{frame}
  \frametitle{A Subtractive Process}
  \begin{textblock}{4}(8,-3)
    \includegraphics[width=4cm]{../introduction/images/stages}
  \end{textblock}
  \vfill
  \begin{quote}
    ``Oh, bosh, as Mr. Ruskin says. Sculpture, per se, is the simplest thing in the world. All you have to do is to take a big chunk of marble and a hammer and chisel, make up your mind what you are about to create and chip off all the marble you don't want.''-Paris Gaulois.
  \end{quote}
  
  {\tiny Source: \url{https://quoteinvestigator.com/2014/06/22/chip-away/}}
\end{frame}

\begin{frame}
\frametitle{Basic Process}
\begin{center}
\begin{tikzpicture}
\node[shape=rectangle,fill=pantone127-4] (problem) at (2,5) {Problem};
\node (human) at (2,4) {Human};
\node[shape=rectangle,fill=pantone127-4] (model) at (2,3) {Model};
\node (solver) at (2,2) {Constraint Solver/Search};
\node[shape=rectangle,fill=pantone157-8] (solution) at (2,1) {Solution};
\draw[-] (problem) -- (human);
\draw[->] (human) -- (model);
\draw[-] (model) -- (solver);
\draw[->] (solver) -- (solution);
\end{tikzpicture}
\end{center}
\end{frame}

\begin{frame}
\frametitle{More Realistic}
\begin{center}
\begin{tikzpicture}[xscale=4.0]
\node[shape=rectangle,fill=pantone127-4] (problem) at (2,5) {Problem};
\node (human) at (2,4) {Human};
\node[shape=rectangle,fill=pantone127-4] (model) at (2,3) {Model};
\node (solver) at (2,2) {Constraint Solver/Search};
\node[shape=rectangle,fill=pantone24-8] (hangs) at (1,1) {Hangs};
\node[shape=rectangle,fill=pantone157-8] (solution) at (2,1) {Solution};
\node[shape=rectangle,fill=pantone24-8] (wrong) at (3,1) {Wrong Solution};
\draw[-] (problem) -- (human);
\draw[->] (human) -- (model);
\draw[-] (model) -- (solver);
\draw[->] (solver) -- (hangs);
\draw[->] (solver) -- (solution);
\draw[->] (solver) -- (wrong);
\draw[->] (hangs) -- (1,4) -- (human);
\draw[->] (wrong) -- (3,4) -- (human);
\end{tikzpicture}
\end{center}
\end{frame}

\begin{frame}
\frametitle{Dual Role of Model}
\begin{itemize}
\item Allows Human to Express Problem
\begin{itemize}
\item Close to Problem Domain
\item Constraints as Abstractions
\end{itemize}
\item Allows Solver to Execute
\begin{itemize}
\item Variables as Communication Mechanism
\item Constraints as Algorithms
\end{itemize}
\end{itemize}
\end{frame}

\begin{frame}
\frametitle{Modelling Frameworks}
\begin{itemize}
\item \href{https://www.minizinc.org/}{MiniZinc} (NICTA, Monash University, Australia)
\item \href{https://github.com/eomahony/Numberjack}{NumberJack} (Insight, Ireland) 
\item \href{https://www-users.york.ac.uk/peter.nightingale/savilerow//}{EssencePrime/SavilleRow} (UK) 
\item \href{https://cpmpy.readthedocs.io/en/latest/}{CPMpy} (KU Leuven)
\item Allow use of multiple back-end solvers
\item Compile model into variants for each solver
\item A priori solver independent model(CP, MIP, SAT)
\end{itemize}
\end{frame} 

\begin{frame}
\frametitle{Framework Process}
\begin{center}
\begin{tikzpicture}[xscale=2]
\node[shape=rectangle,fill=pantone127-4] (problem) at (2.5,5) {Problem};
\node (human) at (2.5,4) {Human};
\node[shape=rectangle,fill=pantone127-4] (model) at (2.5,3) {Model};
\node (compile) at (2.5,2) {Compile/Reformulate};
\node (solver1) at (1,1) {CP};
\node (solver2) at (2,1) {MIP};
\node (solver3) at (3,1) {SAT};
\node (solver4) at (4,1) {Other};
\node[shape=rectangle,fill=pantone157-8] (solution1) at (1,0) {Solution};
\node[shape=rectangle,fill=pantone157-8] (solution2) at (2,0) {Solution};
\node[shape=rectangle,fill=pantone157-8] (solution3) at (3,0) {Solution};
\node[shape=rectangle,fill=pantone157-8] (solution4) at (4,0) {Solution};
\draw[-] (problem) -- (human);
\draw[->] (human) -- (model);
\draw[-] (model) -- (compile);
\draw[-] (compile) -- (solver1);
\draw[-] (compile) -- (solver2);
\draw[-] (compile) -- (solver3);
\draw[-] (compile) -- (solver4);
\draw[->] (solver1) -- (solution1);
\draw[->] (solver2) -- (solution2);
\draw[->] (solver3) -- (solution3);
\draw[->] (solver4) -- (solution4);
\end{tikzpicture}
\end{center}
\end{frame}

% \begin{frame}
  % \frametitle{Do It Now!}
  % \begin{itemize}
  % \item Download and install Minizinc
  % \item \url{https://www.minizinc.org/}
  % \end{itemize}
% \end{frame}


\begin{frame}
\frametitle{Why use Puzzles as Examples?}
\begin{itemize}
\item Easy to understand the problem
\item Solvable by hand without specialized knowledge
\item Possible to compare automated to manual solving process
\end{itemize}

\begin{quote}
The puzzle, though inanimate, is presented as a solvable problem without lasting negative consequences, a very low-risk low-reward situation. By being a puzzle, the object is attempting to convince the user that it must be completed.
\end{quote}
{\tiny Source: \href{https://www.everydayrhetoric.com/2020/03/the-rhetoric-of-puzzles/}{Every Day Rhetoric}}
\end{frame}


\mode<all>{
\part{Basic Constraint Propagation}
}
\begin{frame}
\partpage
\end{frame}
\mode<all>{
\input{../sendmore/body}
}

\mode<all>{
\part{Global Constraints}
}
\begin{frame}
\partpage
\end{frame}
\mode<all>{
\input{../sudoku/body}
}

\clearpage
\mode<all>{
\part{Customizing Search}
}
\begin{frame}
\partpage
\end{frame}
\mode<all>{
\input{../nqueen/body}
}


%% \clearpage
%% \mode<all>{
%%   \part{Problem Decomposition}
%% }
%% \begin{frame}
%%   \partpage
%% \end{frame}
%% \mode<all>{
%%   \input{../party/body}
%%   }

%% \clearpage
%% \mode<all>{
%%   \part{More Global Constraints}
%% }
%% \begin{frame}
%%   \partpage
%% \end{frame}
%% \mode<all>{
%%   \input{../car/body}
%%   }

%% \clearpage
%% \mode<all>{
%%   \part{Methodology}
%% }
%% \begin{frame}
%%   \partpage
%% \end{frame}
%% \mode<all>{
%%   \input{../methodology/body}
%%   }

\clearpage
\mode<all>{
\part{What is missing?}
}
\begin{frame}
\partpage
\end{frame}

\begin{frame}
\frametitle{Many Specialized Topics}
\begin{itemize}
\item How to design efficient core engine
\item Hybrids with LP/MIP tools
\item Hybrids with SAT
\item Symmetry breaking
\item Use of MDD/BDD to encode sets of solutions
\item High level modelling tools
\item Debugging/visualization
\end{itemize}
\end{frame}

\begin{frame}
\frametitle{Reformulation}
\begin{itemize}
\item Just because the user has modelled it this way, it doesn't mean we have to solve it that way
\begin{itemize}
\item Replace some constraint(s) by other, equivalent constraints
\item Because we don't have that constraint in our system
\item For performance
\end{itemize}
\end{itemize}
\end{frame}

\begin{frame}
\frametitle{Learning}
\begin{itemize}
\item While solving the problem we can learn how to strengthen the model/search 
\begin{itemize}
\item Understand which constraints/method contribute to propagation and change schedule
\item Learn no-good constraints by explaining failure
\item Adapt search strategy based on search experience
\end{itemize}
\end{itemize}
\end{frame}

%% \begin{frame}
%% \frametitle{Refined Process}
%% \begin{center}
%% \begin{tikzpicture}
%% \node[shape=rectangle,fill=pantone127-4] (problem) at (3,7) {Problem};
%% \node (human) at (3,6) {Human};
%% \node[shape=rectangle,fill=pantone127-4] (model) at (3,5) {User Model};
%% \node (reformulation) at (3,4) {Reformulation};
%% \node[shape=rectangle,fill=pantone127-4] (implementation) at (3,3) {Implementation Model};
%% \node (solver) at (3,2) {Constraint Solver/Search};
%% \node[shape=rectangle,fill=pantone157-8] (solution) at (3,1) {Solution};
%% \draw[-] (problem) -- (human);
%% \draw[->] (human) -- (model);
%% \draw[-] (model) -- (reformulation);
%% \draw[->] (reformulation) -- (implementation);
%% \draw[-] (implementation) -- (solver);
%% \draw[->] (solver) -- (0,2) -- (0,3) -- (implementation);
%% \draw[->] (solver) -- (solution);
%% \end{tikzpicture}
%% \end{center}
%% \end{frame}

%\begin{frame}
%\frametitle{What is CP actually used for?}
%\url{http://hsimonis.wordpress.com}
%\includegraphics[width=10cm]{../introduction/images/blog}
%\end{frame}

%% \begin{frame}
%% \frametitle{Exercises}
%% \begin{itemize}
%% \item Install Constraint Solver of your Choice
%% \item MiniZinc (MiningZinc) will be used in later sessions
%% \item Run Sendmoremoney, Sudoku, Queens examples
%% \item Can you choose which propagator to use for alldifferent?
%% \item Can you use different search strategies?
%% \end{itemize}
%% \end{frame}

\begin{frame}
\frametitle{More Learning Resources}
\begin{itemize}
\item Survey of Methods, Resources, and Formats for Teaching Constraint Programming
\begin{itemize}
\item by Tejas Santanam, Helmut Simonis
\item \url{https://doi.org/10.48550/arXiv.2403.12717}
\item Based on survey of community for WTCP 2023
\item \url{https://hsimonis.github.io/WTCP2023/}
\end{itemize}
\end{itemize}
\end{frame}


%%   }

%% \clearpage
%% \mode<all>{
%%   \part{Methodology}
%% }
%% \begin{frame}
%%   \partpage
%% \end{frame}
%% \mode<all>{
%%   \begin{frame}
\frametitle{Objectives}
\begin{itemize}
\item Overview of Core Constraint Programming
\item Three Main Concepts
\begin{itemize}
\item Constraint Propagation
\item Global Constraints
\item Customizing Search
\end{itemize}
%\item Get Some Experience with MiniZinc
\item Topics will be treated in more detail in later parts of the school
\item Based on Examples, not Formal Description
\end{itemize}
\end{frame}

\begin{frame}
\frametitle{Outline}
\begin{itemize}
\item Why Constraint Programming?
\item Constraint Propagation
\item Global Constraints
\item Customizing Search
%\item Choices in Modelling
%\item Decomposition
%\item More Global Constraints
%\item Methodology
%\item What is missing?
\end{itemize}
\end{frame}

% \begin{frame}
  % \frametitle{Using MiniZinc IDE}
  % \begin{itemize}
  % \item Developed in the Australian NICTA project
  % \item Maintained by Monash University
  % \item Modelling tool with multiple back-end solvers
  % \item Available from \url{https://www.minizinc.org/}
  % \end{itemize}
% \end{frame}


% \begin{frame}
% \frametitle{Examples in ECLiPSe}
% \begin{itemize}
% \item Open sourced constraint programming language
% \item Development goes back to 1985
% \item ECRC, ICL, IC-Parc, PTL, Cisco
% \item \url{https://eclipseclp.org/}
% \item Specialities
% \begin{itemize}
% \item Develop new solvers for specific domains
% \item Integration with MIP
% \end{itemize}
% \item Not included in bundled MiniZinc IDE
% \item Specialized visualization tools used here
  % \begin{itemize}
  % \item CP-Viz, Simonis et al. 2010
    % \end{itemize}
% \end{itemize}
% \end{frame}

\begin{frame}
\frametitle{Tutorial Based on ECLiPSe ELearning Course}
\begin{itemize}
\item Self-study course in constraint programming
\item Supported by Cisco Systems and Silicon Valley Community Foundation
\item Multi-media format, video lectures, slides, handout etc
\item {\small \url{https://eclipseclp.org/ELearning/index.html}}
\end{itemize}
\end{frame}

\begin{frame}
\frametitle{Also Part of CRT-AI Constraint Week}
\begin{itemize}
\item Annual one week course on CP and Optimization in Ireland
\item Part of national training program for PhD students in AI
\item \url{https://www.crt-ai.ie/}
\end{itemize}
\end{frame}



\begin{frame}
\frametitle{Constraint Programming - in a nutshell}
\begin{itemize}
\item Declarative description of problems with
\begin{itemize}
\item {\em Variables} which range over (finite) sets of values
\item {\em Constraints} over subsets of variables which restrict possible value combinations
\item A {\em solution} is a value assignment which satisfies all constraints
\end{itemize}

\item Constraint propagation/reasoning
\begin{itemize}
\item Removing inconsistent values for variables
\item Detect failure if constraint can not be satisfied
\item Interaction of constraints via shared variables
\item Incomplete
\end{itemize}

\item Search
\begin{itemize}
\item User controlled assignment of values to variables
\item Each step triggers constraint propagation 
\end{itemize}
\item Different domains require/allow different methods
\end{itemize}
\end{frame}

\begin{frame}
  \frametitle{Constraint Programming is Different}
  \begin{itemize}
  \item Declarative Programming
    \begin{itemize}
    \item Concentrate on what you want
      \item Not how to get there
      \item Program != Algorithm
      \item Program = Model
    \end{itemize}
    \item Applied to Combinatorial Problems
      \begin{itemize}
        \item No complete polynomial algorithms known (exist?)
        \item CP less ad-hoc than heuristics
        \item Models can evolve
  \end{itemize}
  \end{itemize}
  \end{frame}
    
\begin{frame}
  \frametitle{A Subtractive Process}
  \begin{textblock}{4}(8,-3)
    \includegraphics[width=4cm]{../introduction/images/stages}
  \end{textblock}
  \vfill
  \begin{quote}
    ``Oh, bosh, as Mr. Ruskin says. Sculpture, per se, is the simplest thing in the world. All you have to do is to take a big chunk of marble and a hammer and chisel, make up your mind what you are about to create and chip off all the marble you don't want.''-Paris Gaulois.
  \end{quote}
  
  {\tiny Source: \url{https://quoteinvestigator.com/2014/06/22/chip-away/}}
\end{frame}

\begin{frame}
\frametitle{Basic Process}
\begin{center}
\begin{tikzpicture}
\node[shape=rectangle,fill=pantone127-4] (problem) at (2,5) {Problem};
\node (human) at (2,4) {Human};
\node[shape=rectangle,fill=pantone127-4] (model) at (2,3) {Model};
\node (solver) at (2,2) {Constraint Solver/Search};
\node[shape=rectangle,fill=pantone157-8] (solution) at (2,1) {Solution};
\draw[-] (problem) -- (human);
\draw[->] (human) -- (model);
\draw[-] (model) -- (solver);
\draw[->] (solver) -- (solution);
\end{tikzpicture}
\end{center}
\end{frame}

\begin{frame}
\frametitle{More Realistic}
\begin{center}
\begin{tikzpicture}[xscale=4.0]
\node[shape=rectangle,fill=pantone127-4] (problem) at (2,5) {Problem};
\node (human) at (2,4) {Human};
\node[shape=rectangle,fill=pantone127-4] (model) at (2,3) {Model};
\node (solver) at (2,2) {Constraint Solver/Search};
\node[shape=rectangle,fill=pantone24-8] (hangs) at (1,1) {Hangs};
\node[shape=rectangle,fill=pantone157-8] (solution) at (2,1) {Solution};
\node[shape=rectangle,fill=pantone24-8] (wrong) at (3,1) {Wrong Solution};
\draw[-] (problem) -- (human);
\draw[->] (human) -- (model);
\draw[-] (model) -- (solver);
\draw[->] (solver) -- (hangs);
\draw[->] (solver) -- (solution);
\draw[->] (solver) -- (wrong);
\draw[->] (hangs) -- (1,4) -- (human);
\draw[->] (wrong) -- (3,4) -- (human);
\end{tikzpicture}
\end{center}
\end{frame}

\begin{frame}
\frametitle{Dual Role of Model}
\begin{itemize}
\item Allows Human to Express Problem
\begin{itemize}
\item Close to Problem Domain
\item Constraints as Abstractions
\end{itemize}
\item Allows Solver to Execute
\begin{itemize}
\item Variables as Communication Mechanism
\item Constraints as Algorithms
\end{itemize}
\end{itemize}
\end{frame}

\begin{frame}
\frametitle{Modelling Frameworks}
\begin{itemize}
\item \href{https://www.minizinc.org/}{MiniZinc} (NICTA, Monash University, Australia)
\item \href{https://github.com/eomahony/Numberjack}{NumberJack} (Insight, Ireland) 
\item \href{https://www-users.york.ac.uk/peter.nightingale/savilerow//}{EssencePrime/SavilleRow} (UK) 
\item \href{https://cpmpy.readthedocs.io/en/latest/}{CPMpy} (KU Leuven)
\item Allow use of multiple back-end solvers
\item Compile model into variants for each solver
\item A priori solver independent model(CP, MIP, SAT)
\end{itemize}
\end{frame} 

\begin{frame}
\frametitle{Framework Process}
\begin{center}
\begin{tikzpicture}[xscale=2]
\node[shape=rectangle,fill=pantone127-4] (problem) at (2.5,5) {Problem};
\node (human) at (2.5,4) {Human};
\node[shape=rectangle,fill=pantone127-4] (model) at (2.5,3) {Model};
\node (compile) at (2.5,2) {Compile/Reformulate};
\node (solver1) at (1,1) {CP};
\node (solver2) at (2,1) {MIP};
\node (solver3) at (3,1) {SAT};
\node (solver4) at (4,1) {Other};
\node[shape=rectangle,fill=pantone157-8] (solution1) at (1,0) {Solution};
\node[shape=rectangle,fill=pantone157-8] (solution2) at (2,0) {Solution};
\node[shape=rectangle,fill=pantone157-8] (solution3) at (3,0) {Solution};
\node[shape=rectangle,fill=pantone157-8] (solution4) at (4,0) {Solution};
\draw[-] (problem) -- (human);
\draw[->] (human) -- (model);
\draw[-] (model) -- (compile);
\draw[-] (compile) -- (solver1);
\draw[-] (compile) -- (solver2);
\draw[-] (compile) -- (solver3);
\draw[-] (compile) -- (solver4);
\draw[->] (solver1) -- (solution1);
\draw[->] (solver2) -- (solution2);
\draw[->] (solver3) -- (solution3);
\draw[->] (solver4) -- (solution4);
\end{tikzpicture}
\end{center}
\end{frame}

% \begin{frame}
  % \frametitle{Do It Now!}
  % \begin{itemize}
  % \item Download and install Minizinc
  % \item \url{https://www.minizinc.org/}
  % \end{itemize}
% \end{frame}


\begin{frame}
\frametitle{Why use Puzzles as Examples?}
\begin{itemize}
\item Easy to understand the problem
\item Solvable by hand without specialized knowledge
\item Possible to compare automated to manual solving process
\end{itemize}

\begin{quote}
The puzzle, though inanimate, is presented as a solvable problem without lasting negative consequences, a very low-risk low-reward situation. By being a puzzle, the object is attempting to convince the user that it must be completed.
\end{quote}
{\tiny Source: \href{https://www.everydayrhetoric.com/2020/03/the-rhetoric-of-puzzles/}{Every Day Rhetoric}}
\end{frame}


\mode<all>{
\part{Basic Constraint Propagation}
}
\begin{frame}
\partpage
\end{frame}
\mode<all>{
\input{../sendmore/body}
}

\mode<all>{
\part{Global Constraints}
}
\begin{frame}
\partpage
\end{frame}
\mode<all>{
\input{../sudoku/body}
}

\clearpage
\mode<all>{
\part{Customizing Search}
}
\begin{frame}
\partpage
\end{frame}
\mode<all>{
\input{../nqueen/body}
}


%% \clearpage
%% \mode<all>{
%%   \part{Problem Decomposition}
%% }
%% \begin{frame}
%%   \partpage
%% \end{frame}
%% \mode<all>{
%%   \input{../party/body}
%%   }

%% \clearpage
%% \mode<all>{
%%   \part{More Global Constraints}
%% }
%% \begin{frame}
%%   \partpage
%% \end{frame}
%% \mode<all>{
%%   \input{../car/body}
%%   }

%% \clearpage
%% \mode<all>{
%%   \part{Methodology}
%% }
%% \begin{frame}
%%   \partpage
%% \end{frame}
%% \mode<all>{
%%   \input{../methodology/body}
%%   }

\clearpage
\mode<all>{
\part{What is missing?}
}
\begin{frame}
\partpage
\end{frame}

\begin{frame}
\frametitle{Many Specialized Topics}
\begin{itemize}
\item How to design efficient core engine
\item Hybrids with LP/MIP tools
\item Hybrids with SAT
\item Symmetry breaking
\item Use of MDD/BDD to encode sets of solutions
\item High level modelling tools
\item Debugging/visualization
\end{itemize}
\end{frame}

\begin{frame}
\frametitle{Reformulation}
\begin{itemize}
\item Just because the user has modelled it this way, it doesn't mean we have to solve it that way
\begin{itemize}
\item Replace some constraint(s) by other, equivalent constraints
\item Because we don't have that constraint in our system
\item For performance
\end{itemize}
\end{itemize}
\end{frame}

\begin{frame}
\frametitle{Learning}
\begin{itemize}
\item While solving the problem we can learn how to strengthen the model/search 
\begin{itemize}
\item Understand which constraints/method contribute to propagation and change schedule
\item Learn no-good constraints by explaining failure
\item Adapt search strategy based on search experience
\end{itemize}
\end{itemize}
\end{frame}

%% \begin{frame}
%% \frametitle{Refined Process}
%% \begin{center}
%% \begin{tikzpicture}
%% \node[shape=rectangle,fill=pantone127-4] (problem) at (3,7) {Problem};
%% \node (human) at (3,6) {Human};
%% \node[shape=rectangle,fill=pantone127-4] (model) at (3,5) {User Model};
%% \node (reformulation) at (3,4) {Reformulation};
%% \node[shape=rectangle,fill=pantone127-4] (implementation) at (3,3) {Implementation Model};
%% \node (solver) at (3,2) {Constraint Solver/Search};
%% \node[shape=rectangle,fill=pantone157-8] (solution) at (3,1) {Solution};
%% \draw[-] (problem) -- (human);
%% \draw[->] (human) -- (model);
%% \draw[-] (model) -- (reformulation);
%% \draw[->] (reformulation) -- (implementation);
%% \draw[-] (implementation) -- (solver);
%% \draw[->] (solver) -- (0,2) -- (0,3) -- (implementation);
%% \draw[->] (solver) -- (solution);
%% \end{tikzpicture}
%% \end{center}
%% \end{frame}

%\begin{frame}
%\frametitle{What is CP actually used for?}
%\url{http://hsimonis.wordpress.com}
%\includegraphics[width=10cm]{../introduction/images/blog}
%\end{frame}

%% \begin{frame}
%% \frametitle{Exercises}
%% \begin{itemize}
%% \item Install Constraint Solver of your Choice
%% \item MiniZinc (MiningZinc) will be used in later sessions
%% \item Run Sendmoremoney, Sudoku, Queens examples
%% \item Can you choose which propagator to use for alldifferent?
%% \item Can you use different search strategies?
%% \end{itemize}
%% \end{frame}

\begin{frame}
\frametitle{More Learning Resources}
\begin{itemize}
\item Survey of Methods, Resources, and Formats for Teaching Constraint Programming
\begin{itemize}
\item by Tejas Santanam, Helmut Simonis
\item \url{https://doi.org/10.48550/arXiv.2403.12717}
\item Based on survey of community for WTCP 2023
\item \url{https://hsimonis.github.io/WTCP2023/}
\end{itemize}
\end{itemize}
\end{frame}


%%   }

\clearpage
\mode<all>{
\part{What is missing?}
}
\begin{frame}
\partpage
\end{frame}

\begin{frame}
\frametitle{Many Specialized Topics}
\begin{itemize}
\item How to design efficient core engine
\item Hybrids with LP/MIP tools
\item Hybrids with SAT
\item Symmetry breaking
\item Use of MDD/BDD to encode sets of solutions
\item High level modelling tools
\item Debugging/visualization
\end{itemize}
\end{frame}

\begin{frame}
\frametitle{Reformulation}
\begin{itemize}
\item Just because the user has modelled it this way, it doesn't mean we have to solve it that way
\begin{itemize}
\item Replace some constraint(s) by other, equivalent constraints
\item Because we don't have that constraint in our system
\item For performance
\end{itemize}
\end{itemize}
\end{frame}

\begin{frame}
\frametitle{Learning}
\begin{itemize}
\item While solving the problem we can learn how to strengthen the model/search 
\begin{itemize}
\item Understand which constraints/method contribute to propagation and change schedule
\item Learn no-good constraints by explaining failure
\item Adapt search strategy based on search experience
\end{itemize}
\end{itemize}
\end{frame}

%% \begin{frame}
%% \frametitle{Refined Process}
%% \begin{center}
%% \begin{tikzpicture}
%% \node[shape=rectangle,fill=pantone127-4] (problem) at (3,7) {Problem};
%% \node (human) at (3,6) {Human};
%% \node[shape=rectangle,fill=pantone127-4] (model) at (3,5) {User Model};
%% \node (reformulation) at (3,4) {Reformulation};
%% \node[shape=rectangle,fill=pantone127-4] (implementation) at (3,3) {Implementation Model};
%% \node (solver) at (3,2) {Constraint Solver/Search};
%% \node[shape=rectangle,fill=pantone157-8] (solution) at (3,1) {Solution};
%% \draw[-] (problem) -- (human);
%% \draw[->] (human) -- (model);
%% \draw[-] (model) -- (reformulation);
%% \draw[->] (reformulation) -- (implementation);
%% \draw[-] (implementation) -- (solver);
%% \draw[->] (solver) -- (0,2) -- (0,3) -- (implementation);
%% \draw[->] (solver) -- (solution);
%% \end{tikzpicture}
%% \end{center}
%% \end{frame}

%\begin{frame}
%\frametitle{What is CP actually used for?}
%\url{http://hsimonis.wordpress.com}
%\includegraphics[width=10cm]{../introduction/images/blog}
%\end{frame}

%% \begin{frame}
%% \frametitle{Exercises}
%% \begin{itemize}
%% \item Install Constraint Solver of your Choice
%% \item MiniZinc (MiningZinc) will be used in later sessions
%% \item Run Sendmoremoney, Sudoku, Queens examples
%% \item Can you choose which propagator to use for alldifferent?
%% \item Can you use different search strategies?
%% \end{itemize}
%% \end{frame}

\begin{frame}
\frametitle{More Learning Resources}
\begin{itemize}
\item Survey of Methods, Resources, and Formats for Teaching Constraint Programming
\begin{itemize}
\item by Tejas Santanam, Helmut Simonis
\item \url{https://doi.org/10.48550/arXiv.2403.12717}
\item Based on survey of community for WTCP 2023
\item \url{https://hsimonis.github.io/WTCP2023/}
\end{itemize}
\end{itemize}
\end{frame}


%%   }

%% \clearpage
%% \mode<all>{
%%   \part{Methodology}
%% }
%% \begin{frame}
%%   \partpage
%% \end{frame}
%% \mode<all>{
%%   \begin{frame}
\frametitle{Objectives}
\begin{itemize}
\item Overview of Core Constraint Programming
\item Three Main Concepts
\begin{itemize}
\item Constraint Propagation
\item Global Constraints
\item Customizing Search
\end{itemize}
%\item Get Some Experience with MiniZinc
\item Topics will be treated in more detail in later parts of the school
\item Based on Examples, not Formal Description
\end{itemize}
\end{frame}

\begin{frame}
\frametitle{Outline}
\begin{itemize}
\item Why Constraint Programming?
\item Constraint Propagation
\item Global Constraints
\item Customizing Search
%\item Choices in Modelling
%\item Decomposition
%\item More Global Constraints
%\item Methodology
%\item What is missing?
\end{itemize}
\end{frame}

% \begin{frame}
  % \frametitle{Using MiniZinc IDE}
  % \begin{itemize}
  % \item Developed in the Australian NICTA project
  % \item Maintained by Monash University
  % \item Modelling tool with multiple back-end solvers
  % \item Available from \url{https://www.minizinc.org/}
  % \end{itemize}
% \end{frame}


% \begin{frame}
% \frametitle{Examples in ECLiPSe}
% \begin{itemize}
% \item Open sourced constraint programming language
% \item Development goes back to 1985
% \item ECRC, ICL, IC-Parc, PTL, Cisco
% \item \url{https://eclipseclp.org/}
% \item Specialities
% \begin{itemize}
% \item Develop new solvers for specific domains
% \item Integration with MIP
% \end{itemize}
% \item Not included in bundled MiniZinc IDE
% \item Specialized visualization tools used here
  % \begin{itemize}
  % \item CP-Viz, Simonis et al. 2010
    % \end{itemize}
% \end{itemize}
% \end{frame}

\begin{frame}
\frametitle{Tutorial Based on ECLiPSe ELearning Course}
\begin{itemize}
\item Self-study course in constraint programming
\item Supported by Cisco Systems and Silicon Valley Community Foundation
\item Multi-media format, video lectures, slides, handout etc
\item {\small \url{https://eclipseclp.org/ELearning/index.html}}
\end{itemize}
\end{frame}

\begin{frame}
\frametitle{Also Part of CRT-AI Constraint Week}
\begin{itemize}
\item Annual one week course on CP and Optimization in Ireland
\item Part of national training program for PhD students in AI
\item \url{https://www.crt-ai.ie/}
\end{itemize}
\end{frame}



\begin{frame}
\frametitle{Constraint Programming - in a nutshell}
\begin{itemize}
\item Declarative description of problems with
\begin{itemize}
\item {\em Variables} which range over (finite) sets of values
\item {\em Constraints} over subsets of variables which restrict possible value combinations
\item A {\em solution} is a value assignment which satisfies all constraints
\end{itemize}

\item Constraint propagation/reasoning
\begin{itemize}
\item Removing inconsistent values for variables
\item Detect failure if constraint can not be satisfied
\item Interaction of constraints via shared variables
\item Incomplete
\end{itemize}

\item Search
\begin{itemize}
\item User controlled assignment of values to variables
\item Each step triggers constraint propagation 
\end{itemize}
\item Different domains require/allow different methods
\end{itemize}
\end{frame}

\begin{frame}
  \frametitle{Constraint Programming is Different}
  \begin{itemize}
  \item Declarative Programming
    \begin{itemize}
    \item Concentrate on what you want
      \item Not how to get there
      \item Program != Algorithm
      \item Program = Model
    \end{itemize}
    \item Applied to Combinatorial Problems
      \begin{itemize}
        \item No complete polynomial algorithms known (exist?)
        \item CP less ad-hoc than heuristics
        \item Models can evolve
  \end{itemize}
  \end{itemize}
  \end{frame}
    
\begin{frame}
  \frametitle{A Subtractive Process}
  \begin{textblock}{4}(8,-3)
    \includegraphics[width=4cm]{../introduction/images/stages}
  \end{textblock}
  \vfill
  \begin{quote}
    ``Oh, bosh, as Mr. Ruskin says. Sculpture, per se, is the simplest thing in the world. All you have to do is to take a big chunk of marble and a hammer and chisel, make up your mind what you are about to create and chip off all the marble you don't want.''-Paris Gaulois.
  \end{quote}
  
  {\tiny Source: \url{https://quoteinvestigator.com/2014/06/22/chip-away/}}
\end{frame}

\begin{frame}
\frametitle{Basic Process}
\begin{center}
\begin{tikzpicture}
\node[shape=rectangle,fill=pantone127-4] (problem) at (2,5) {Problem};
\node (human) at (2,4) {Human};
\node[shape=rectangle,fill=pantone127-4] (model) at (2,3) {Model};
\node (solver) at (2,2) {Constraint Solver/Search};
\node[shape=rectangle,fill=pantone157-8] (solution) at (2,1) {Solution};
\draw[-] (problem) -- (human);
\draw[->] (human) -- (model);
\draw[-] (model) -- (solver);
\draw[->] (solver) -- (solution);
\end{tikzpicture}
\end{center}
\end{frame}

\begin{frame}
\frametitle{More Realistic}
\begin{center}
\begin{tikzpicture}[xscale=4.0]
\node[shape=rectangle,fill=pantone127-4] (problem) at (2,5) {Problem};
\node (human) at (2,4) {Human};
\node[shape=rectangle,fill=pantone127-4] (model) at (2,3) {Model};
\node (solver) at (2,2) {Constraint Solver/Search};
\node[shape=rectangle,fill=pantone24-8] (hangs) at (1,1) {Hangs};
\node[shape=rectangle,fill=pantone157-8] (solution) at (2,1) {Solution};
\node[shape=rectangle,fill=pantone24-8] (wrong) at (3,1) {Wrong Solution};
\draw[-] (problem) -- (human);
\draw[->] (human) -- (model);
\draw[-] (model) -- (solver);
\draw[->] (solver) -- (hangs);
\draw[->] (solver) -- (solution);
\draw[->] (solver) -- (wrong);
\draw[->] (hangs) -- (1,4) -- (human);
\draw[->] (wrong) -- (3,4) -- (human);
\end{tikzpicture}
\end{center}
\end{frame}

\begin{frame}
\frametitle{Dual Role of Model}
\begin{itemize}
\item Allows Human to Express Problem
\begin{itemize}
\item Close to Problem Domain
\item Constraints as Abstractions
\end{itemize}
\item Allows Solver to Execute
\begin{itemize}
\item Variables as Communication Mechanism
\item Constraints as Algorithms
\end{itemize}
\end{itemize}
\end{frame}

\begin{frame}
\frametitle{Modelling Frameworks}
\begin{itemize}
\item \href{https://www.minizinc.org/}{MiniZinc} (NICTA, Monash University, Australia)
\item \href{https://github.com/eomahony/Numberjack}{NumberJack} (Insight, Ireland) 
\item \href{https://www-users.york.ac.uk/peter.nightingale/savilerow//}{EssencePrime/SavilleRow} (UK) 
\item \href{https://cpmpy.readthedocs.io/en/latest/}{CPMpy} (KU Leuven)
\item Allow use of multiple back-end solvers
\item Compile model into variants for each solver
\item A priori solver independent model(CP, MIP, SAT)
\end{itemize}
\end{frame} 

\begin{frame}
\frametitle{Framework Process}
\begin{center}
\begin{tikzpicture}[xscale=2]
\node[shape=rectangle,fill=pantone127-4] (problem) at (2.5,5) {Problem};
\node (human) at (2.5,4) {Human};
\node[shape=rectangle,fill=pantone127-4] (model) at (2.5,3) {Model};
\node (compile) at (2.5,2) {Compile/Reformulate};
\node (solver1) at (1,1) {CP};
\node (solver2) at (2,1) {MIP};
\node (solver3) at (3,1) {SAT};
\node (solver4) at (4,1) {Other};
\node[shape=rectangle,fill=pantone157-8] (solution1) at (1,0) {Solution};
\node[shape=rectangle,fill=pantone157-8] (solution2) at (2,0) {Solution};
\node[shape=rectangle,fill=pantone157-8] (solution3) at (3,0) {Solution};
\node[shape=rectangle,fill=pantone157-8] (solution4) at (4,0) {Solution};
\draw[-] (problem) -- (human);
\draw[->] (human) -- (model);
\draw[-] (model) -- (compile);
\draw[-] (compile) -- (solver1);
\draw[-] (compile) -- (solver2);
\draw[-] (compile) -- (solver3);
\draw[-] (compile) -- (solver4);
\draw[->] (solver1) -- (solution1);
\draw[->] (solver2) -- (solution2);
\draw[->] (solver3) -- (solution3);
\draw[->] (solver4) -- (solution4);
\end{tikzpicture}
\end{center}
\end{frame}

% \begin{frame}
  % \frametitle{Do It Now!}
  % \begin{itemize}
  % \item Download and install Minizinc
  % \item \url{https://www.minizinc.org/}
  % \end{itemize}
% \end{frame}


\begin{frame}
\frametitle{Why use Puzzles as Examples?}
\begin{itemize}
\item Easy to understand the problem
\item Solvable by hand without specialized knowledge
\item Possible to compare automated to manual solving process
\end{itemize}

\begin{quote}
The puzzle, though inanimate, is presented as a solvable problem without lasting negative consequences, a very low-risk low-reward situation. By being a puzzle, the object is attempting to convince the user that it must be completed.
\end{quote}
{\tiny Source: \href{https://www.everydayrhetoric.com/2020/03/the-rhetoric-of-puzzles/}{Every Day Rhetoric}}
\end{frame}


\mode<all>{
\part{Basic Constraint Propagation}
}
\begin{frame}
\partpage
\end{frame}
\mode<all>{
\begin{frame}
\frametitle{Objectives}
\begin{itemize}
\item Overview of Core Constraint Programming
\item Three Main Concepts
\begin{itemize}
\item Constraint Propagation
\item Global Constraints
\item Customizing Search
\end{itemize}
%\item Get Some Experience with MiniZinc
\item Topics will be treated in more detail in later parts of the school
\item Based on Examples, not Formal Description
\end{itemize}
\end{frame}

\begin{frame}
\frametitle{Outline}
\begin{itemize}
\item Why Constraint Programming?
\item Constraint Propagation
\item Global Constraints
\item Customizing Search
%\item Choices in Modelling
%\item Decomposition
%\item More Global Constraints
%\item Methodology
%\item What is missing?
\end{itemize}
\end{frame}

% \begin{frame}
  % \frametitle{Using MiniZinc IDE}
  % \begin{itemize}
  % \item Developed in the Australian NICTA project
  % \item Maintained by Monash University
  % \item Modelling tool with multiple back-end solvers
  % \item Available from \url{https://www.minizinc.org/}
  % \end{itemize}
% \end{frame}


% \begin{frame}
% \frametitle{Examples in ECLiPSe}
% \begin{itemize}
% \item Open sourced constraint programming language
% \item Development goes back to 1985
% \item ECRC, ICL, IC-Parc, PTL, Cisco
% \item \url{https://eclipseclp.org/}
% \item Specialities
% \begin{itemize}
% \item Develop new solvers for specific domains
% \item Integration with MIP
% \end{itemize}
% \item Not included in bundled MiniZinc IDE
% \item Specialized visualization tools used here
  % \begin{itemize}
  % \item CP-Viz, Simonis et al. 2010
    % \end{itemize}
% \end{itemize}
% \end{frame}

\begin{frame}
\frametitle{Tutorial Based on ECLiPSe ELearning Course}
\begin{itemize}
\item Self-study course in constraint programming
\item Supported by Cisco Systems and Silicon Valley Community Foundation
\item Multi-media format, video lectures, slides, handout etc
\item {\small \url{https://eclipseclp.org/ELearning/index.html}}
\end{itemize}
\end{frame}

\begin{frame}
\frametitle{Also Part of CRT-AI Constraint Week}
\begin{itemize}
\item Annual one week course on CP and Optimization in Ireland
\item Part of national training program for PhD students in AI
\item \url{https://www.crt-ai.ie/}
\end{itemize}
\end{frame}



\begin{frame}
\frametitle{Constraint Programming - in a nutshell}
\begin{itemize}
\item Declarative description of problems with
\begin{itemize}
\item {\em Variables} which range over (finite) sets of values
\item {\em Constraints} over subsets of variables which restrict possible value combinations
\item A {\em solution} is a value assignment which satisfies all constraints
\end{itemize}

\item Constraint propagation/reasoning
\begin{itemize}
\item Removing inconsistent values for variables
\item Detect failure if constraint can not be satisfied
\item Interaction of constraints via shared variables
\item Incomplete
\end{itemize}

\item Search
\begin{itemize}
\item User controlled assignment of values to variables
\item Each step triggers constraint propagation 
\end{itemize}
\item Different domains require/allow different methods
\end{itemize}
\end{frame}

\begin{frame}
  \frametitle{Constraint Programming is Different}
  \begin{itemize}
  \item Declarative Programming
    \begin{itemize}
    \item Concentrate on what you want
      \item Not how to get there
      \item Program != Algorithm
      \item Program = Model
    \end{itemize}
    \item Applied to Combinatorial Problems
      \begin{itemize}
        \item No complete polynomial algorithms known (exist?)
        \item CP less ad-hoc than heuristics
        \item Models can evolve
  \end{itemize}
  \end{itemize}
  \end{frame}
    
\begin{frame}
  \frametitle{A Subtractive Process}
  \begin{textblock}{4}(8,-3)
    \includegraphics[width=4cm]{../introduction/images/stages}
  \end{textblock}
  \vfill
  \begin{quote}
    ``Oh, bosh, as Mr. Ruskin says. Sculpture, per se, is the simplest thing in the world. All you have to do is to take a big chunk of marble and a hammer and chisel, make up your mind what you are about to create and chip off all the marble you don't want.''-Paris Gaulois.
  \end{quote}
  
  {\tiny Source: \url{https://quoteinvestigator.com/2014/06/22/chip-away/}}
\end{frame}

\begin{frame}
\frametitle{Basic Process}
\begin{center}
\begin{tikzpicture}
\node[shape=rectangle,fill=pantone127-4] (problem) at (2,5) {Problem};
\node (human) at (2,4) {Human};
\node[shape=rectangle,fill=pantone127-4] (model) at (2,3) {Model};
\node (solver) at (2,2) {Constraint Solver/Search};
\node[shape=rectangle,fill=pantone157-8] (solution) at (2,1) {Solution};
\draw[-] (problem) -- (human);
\draw[->] (human) -- (model);
\draw[-] (model) -- (solver);
\draw[->] (solver) -- (solution);
\end{tikzpicture}
\end{center}
\end{frame}

\begin{frame}
\frametitle{More Realistic}
\begin{center}
\begin{tikzpicture}[xscale=4.0]
\node[shape=rectangle,fill=pantone127-4] (problem) at (2,5) {Problem};
\node (human) at (2,4) {Human};
\node[shape=rectangle,fill=pantone127-4] (model) at (2,3) {Model};
\node (solver) at (2,2) {Constraint Solver/Search};
\node[shape=rectangle,fill=pantone24-8] (hangs) at (1,1) {Hangs};
\node[shape=rectangle,fill=pantone157-8] (solution) at (2,1) {Solution};
\node[shape=rectangle,fill=pantone24-8] (wrong) at (3,1) {Wrong Solution};
\draw[-] (problem) -- (human);
\draw[->] (human) -- (model);
\draw[-] (model) -- (solver);
\draw[->] (solver) -- (hangs);
\draw[->] (solver) -- (solution);
\draw[->] (solver) -- (wrong);
\draw[->] (hangs) -- (1,4) -- (human);
\draw[->] (wrong) -- (3,4) -- (human);
\end{tikzpicture}
\end{center}
\end{frame}

\begin{frame}
\frametitle{Dual Role of Model}
\begin{itemize}
\item Allows Human to Express Problem
\begin{itemize}
\item Close to Problem Domain
\item Constraints as Abstractions
\end{itemize}
\item Allows Solver to Execute
\begin{itemize}
\item Variables as Communication Mechanism
\item Constraints as Algorithms
\end{itemize}
\end{itemize}
\end{frame}

\begin{frame}
\frametitle{Modelling Frameworks}
\begin{itemize}
\item \href{https://www.minizinc.org/}{MiniZinc} (NICTA, Monash University, Australia)
\item \href{https://github.com/eomahony/Numberjack}{NumberJack} (Insight, Ireland) 
\item \href{https://www-users.york.ac.uk/peter.nightingale/savilerow//}{EssencePrime/SavilleRow} (UK) 
\item \href{https://cpmpy.readthedocs.io/en/latest/}{CPMpy} (KU Leuven)
\item Allow use of multiple back-end solvers
\item Compile model into variants for each solver
\item A priori solver independent model(CP, MIP, SAT)
\end{itemize}
\end{frame} 

\begin{frame}
\frametitle{Framework Process}
\begin{center}
\begin{tikzpicture}[xscale=2]
\node[shape=rectangle,fill=pantone127-4] (problem) at (2.5,5) {Problem};
\node (human) at (2.5,4) {Human};
\node[shape=rectangle,fill=pantone127-4] (model) at (2.5,3) {Model};
\node (compile) at (2.5,2) {Compile/Reformulate};
\node (solver1) at (1,1) {CP};
\node (solver2) at (2,1) {MIP};
\node (solver3) at (3,1) {SAT};
\node (solver4) at (4,1) {Other};
\node[shape=rectangle,fill=pantone157-8] (solution1) at (1,0) {Solution};
\node[shape=rectangle,fill=pantone157-8] (solution2) at (2,0) {Solution};
\node[shape=rectangle,fill=pantone157-8] (solution3) at (3,0) {Solution};
\node[shape=rectangle,fill=pantone157-8] (solution4) at (4,0) {Solution};
\draw[-] (problem) -- (human);
\draw[->] (human) -- (model);
\draw[-] (model) -- (compile);
\draw[-] (compile) -- (solver1);
\draw[-] (compile) -- (solver2);
\draw[-] (compile) -- (solver3);
\draw[-] (compile) -- (solver4);
\draw[->] (solver1) -- (solution1);
\draw[->] (solver2) -- (solution2);
\draw[->] (solver3) -- (solution3);
\draw[->] (solver4) -- (solution4);
\end{tikzpicture}
\end{center}
\end{frame}

% \begin{frame}
  % \frametitle{Do It Now!}
  % \begin{itemize}
  % \item Download and install Minizinc
  % \item \url{https://www.minizinc.org/}
  % \end{itemize}
% \end{frame}


\begin{frame}
\frametitle{Why use Puzzles as Examples?}
\begin{itemize}
\item Easy to understand the problem
\item Solvable by hand without specialized knowledge
\item Possible to compare automated to manual solving process
\end{itemize}

\begin{quote}
The puzzle, though inanimate, is presented as a solvable problem without lasting negative consequences, a very low-risk low-reward situation. By being a puzzle, the object is attempting to convince the user that it must be completed.
\end{quote}
{\tiny Source: \href{https://www.everydayrhetoric.com/2020/03/the-rhetoric-of-puzzles/}{Every Day Rhetoric}}
\end{frame}


\mode<all>{
\part{Basic Constraint Propagation}
}
\begin{frame}
\partpage
\end{frame}
\mode<all>{
\input{../sendmore/body}
}

\mode<all>{
\part{Global Constraints}
}
\begin{frame}
\partpage
\end{frame}
\mode<all>{
\input{../sudoku/body}
}

\clearpage
\mode<all>{
\part{Customizing Search}
}
\begin{frame}
\partpage
\end{frame}
\mode<all>{
\input{../nqueen/body}
}


%% \clearpage
%% \mode<all>{
%%   \part{Problem Decomposition}
%% }
%% \begin{frame}
%%   \partpage
%% \end{frame}
%% \mode<all>{
%%   \input{../party/body}
%%   }

%% \clearpage
%% \mode<all>{
%%   \part{More Global Constraints}
%% }
%% \begin{frame}
%%   \partpage
%% \end{frame}
%% \mode<all>{
%%   \input{../car/body}
%%   }

%% \clearpage
%% \mode<all>{
%%   \part{Methodology}
%% }
%% \begin{frame}
%%   \partpage
%% \end{frame}
%% \mode<all>{
%%   \input{../methodology/body}
%%   }

\clearpage
\mode<all>{
\part{What is missing?}
}
\begin{frame}
\partpage
\end{frame}

\begin{frame}
\frametitle{Many Specialized Topics}
\begin{itemize}
\item How to design efficient core engine
\item Hybrids with LP/MIP tools
\item Hybrids with SAT
\item Symmetry breaking
\item Use of MDD/BDD to encode sets of solutions
\item High level modelling tools
\item Debugging/visualization
\end{itemize}
\end{frame}

\begin{frame}
\frametitle{Reformulation}
\begin{itemize}
\item Just because the user has modelled it this way, it doesn't mean we have to solve it that way
\begin{itemize}
\item Replace some constraint(s) by other, equivalent constraints
\item Because we don't have that constraint in our system
\item For performance
\end{itemize}
\end{itemize}
\end{frame}

\begin{frame}
\frametitle{Learning}
\begin{itemize}
\item While solving the problem we can learn how to strengthen the model/search 
\begin{itemize}
\item Understand which constraints/method contribute to propagation and change schedule
\item Learn no-good constraints by explaining failure
\item Adapt search strategy based on search experience
\end{itemize}
\end{itemize}
\end{frame}

%% \begin{frame}
%% \frametitle{Refined Process}
%% \begin{center}
%% \begin{tikzpicture}
%% \node[shape=rectangle,fill=pantone127-4] (problem) at (3,7) {Problem};
%% \node (human) at (3,6) {Human};
%% \node[shape=rectangle,fill=pantone127-4] (model) at (3,5) {User Model};
%% \node (reformulation) at (3,4) {Reformulation};
%% \node[shape=rectangle,fill=pantone127-4] (implementation) at (3,3) {Implementation Model};
%% \node (solver) at (3,2) {Constraint Solver/Search};
%% \node[shape=rectangle,fill=pantone157-8] (solution) at (3,1) {Solution};
%% \draw[-] (problem) -- (human);
%% \draw[->] (human) -- (model);
%% \draw[-] (model) -- (reformulation);
%% \draw[->] (reformulation) -- (implementation);
%% \draw[-] (implementation) -- (solver);
%% \draw[->] (solver) -- (0,2) -- (0,3) -- (implementation);
%% \draw[->] (solver) -- (solution);
%% \end{tikzpicture}
%% \end{center}
%% \end{frame}

%\begin{frame}
%\frametitle{What is CP actually used for?}
%\url{http://hsimonis.wordpress.com}
%\includegraphics[width=10cm]{../introduction/images/blog}
%\end{frame}

%% \begin{frame}
%% \frametitle{Exercises}
%% \begin{itemize}
%% \item Install Constraint Solver of your Choice
%% \item MiniZinc (MiningZinc) will be used in later sessions
%% \item Run Sendmoremoney, Sudoku, Queens examples
%% \item Can you choose which propagator to use for alldifferent?
%% \item Can you use different search strategies?
%% \end{itemize}
%% \end{frame}

\begin{frame}
\frametitle{More Learning Resources}
\begin{itemize}
\item Survey of Methods, Resources, and Formats for Teaching Constraint Programming
\begin{itemize}
\item by Tejas Santanam, Helmut Simonis
\item \url{https://doi.org/10.48550/arXiv.2403.12717}
\item Based on survey of community for WTCP 2023
\item \url{https://hsimonis.github.io/WTCP2023/}
\end{itemize}
\end{itemize}
\end{frame}


}

\mode<all>{
\part{Global Constraints}
}
\begin{frame}
\partpage
\end{frame}
\mode<all>{
\begin{frame}
\frametitle{Objectives}
\begin{itemize}
\item Overview of Core Constraint Programming
\item Three Main Concepts
\begin{itemize}
\item Constraint Propagation
\item Global Constraints
\item Customizing Search
\end{itemize}
%\item Get Some Experience with MiniZinc
\item Topics will be treated in more detail in later parts of the school
\item Based on Examples, not Formal Description
\end{itemize}
\end{frame}

\begin{frame}
\frametitle{Outline}
\begin{itemize}
\item Why Constraint Programming?
\item Constraint Propagation
\item Global Constraints
\item Customizing Search
%\item Choices in Modelling
%\item Decomposition
%\item More Global Constraints
%\item Methodology
%\item What is missing?
\end{itemize}
\end{frame}

% \begin{frame}
  % \frametitle{Using MiniZinc IDE}
  % \begin{itemize}
  % \item Developed in the Australian NICTA project
  % \item Maintained by Monash University
  % \item Modelling tool with multiple back-end solvers
  % \item Available from \url{https://www.minizinc.org/}
  % \end{itemize}
% \end{frame}


% \begin{frame}
% \frametitle{Examples in ECLiPSe}
% \begin{itemize}
% \item Open sourced constraint programming language
% \item Development goes back to 1985
% \item ECRC, ICL, IC-Parc, PTL, Cisco
% \item \url{https://eclipseclp.org/}
% \item Specialities
% \begin{itemize}
% \item Develop new solvers for specific domains
% \item Integration with MIP
% \end{itemize}
% \item Not included in bundled MiniZinc IDE
% \item Specialized visualization tools used here
  % \begin{itemize}
  % \item CP-Viz, Simonis et al. 2010
    % \end{itemize}
% \end{itemize}
% \end{frame}

\begin{frame}
\frametitle{Tutorial Based on ECLiPSe ELearning Course}
\begin{itemize}
\item Self-study course in constraint programming
\item Supported by Cisco Systems and Silicon Valley Community Foundation
\item Multi-media format, video lectures, slides, handout etc
\item {\small \url{https://eclipseclp.org/ELearning/index.html}}
\end{itemize}
\end{frame}

\begin{frame}
\frametitle{Also Part of CRT-AI Constraint Week}
\begin{itemize}
\item Annual one week course on CP and Optimization in Ireland
\item Part of national training program for PhD students in AI
\item \url{https://www.crt-ai.ie/}
\end{itemize}
\end{frame}



\begin{frame}
\frametitle{Constraint Programming - in a nutshell}
\begin{itemize}
\item Declarative description of problems with
\begin{itemize}
\item {\em Variables} which range over (finite) sets of values
\item {\em Constraints} over subsets of variables which restrict possible value combinations
\item A {\em solution} is a value assignment which satisfies all constraints
\end{itemize}

\item Constraint propagation/reasoning
\begin{itemize}
\item Removing inconsistent values for variables
\item Detect failure if constraint can not be satisfied
\item Interaction of constraints via shared variables
\item Incomplete
\end{itemize}

\item Search
\begin{itemize}
\item User controlled assignment of values to variables
\item Each step triggers constraint propagation 
\end{itemize}
\item Different domains require/allow different methods
\end{itemize}
\end{frame}

\begin{frame}
  \frametitle{Constraint Programming is Different}
  \begin{itemize}
  \item Declarative Programming
    \begin{itemize}
    \item Concentrate on what you want
      \item Not how to get there
      \item Program != Algorithm
      \item Program = Model
    \end{itemize}
    \item Applied to Combinatorial Problems
      \begin{itemize}
        \item No complete polynomial algorithms known (exist?)
        \item CP less ad-hoc than heuristics
        \item Models can evolve
  \end{itemize}
  \end{itemize}
  \end{frame}
    
\begin{frame}
  \frametitle{A Subtractive Process}
  \begin{textblock}{4}(8,-3)
    \includegraphics[width=4cm]{../introduction/images/stages}
  \end{textblock}
  \vfill
  \begin{quote}
    ``Oh, bosh, as Mr. Ruskin says. Sculpture, per se, is the simplest thing in the world. All you have to do is to take a big chunk of marble and a hammer and chisel, make up your mind what you are about to create and chip off all the marble you don't want.''-Paris Gaulois.
  \end{quote}
  
  {\tiny Source: \url{https://quoteinvestigator.com/2014/06/22/chip-away/}}
\end{frame}

\begin{frame}
\frametitle{Basic Process}
\begin{center}
\begin{tikzpicture}
\node[shape=rectangle,fill=pantone127-4] (problem) at (2,5) {Problem};
\node (human) at (2,4) {Human};
\node[shape=rectangle,fill=pantone127-4] (model) at (2,3) {Model};
\node (solver) at (2,2) {Constraint Solver/Search};
\node[shape=rectangle,fill=pantone157-8] (solution) at (2,1) {Solution};
\draw[-] (problem) -- (human);
\draw[->] (human) -- (model);
\draw[-] (model) -- (solver);
\draw[->] (solver) -- (solution);
\end{tikzpicture}
\end{center}
\end{frame}

\begin{frame}
\frametitle{More Realistic}
\begin{center}
\begin{tikzpicture}[xscale=4.0]
\node[shape=rectangle,fill=pantone127-4] (problem) at (2,5) {Problem};
\node (human) at (2,4) {Human};
\node[shape=rectangle,fill=pantone127-4] (model) at (2,3) {Model};
\node (solver) at (2,2) {Constraint Solver/Search};
\node[shape=rectangle,fill=pantone24-8] (hangs) at (1,1) {Hangs};
\node[shape=rectangle,fill=pantone157-8] (solution) at (2,1) {Solution};
\node[shape=rectangle,fill=pantone24-8] (wrong) at (3,1) {Wrong Solution};
\draw[-] (problem) -- (human);
\draw[->] (human) -- (model);
\draw[-] (model) -- (solver);
\draw[->] (solver) -- (hangs);
\draw[->] (solver) -- (solution);
\draw[->] (solver) -- (wrong);
\draw[->] (hangs) -- (1,4) -- (human);
\draw[->] (wrong) -- (3,4) -- (human);
\end{tikzpicture}
\end{center}
\end{frame}

\begin{frame}
\frametitle{Dual Role of Model}
\begin{itemize}
\item Allows Human to Express Problem
\begin{itemize}
\item Close to Problem Domain
\item Constraints as Abstractions
\end{itemize}
\item Allows Solver to Execute
\begin{itemize}
\item Variables as Communication Mechanism
\item Constraints as Algorithms
\end{itemize}
\end{itemize}
\end{frame}

\begin{frame}
\frametitle{Modelling Frameworks}
\begin{itemize}
\item \href{https://www.minizinc.org/}{MiniZinc} (NICTA, Monash University, Australia)
\item \href{https://github.com/eomahony/Numberjack}{NumberJack} (Insight, Ireland) 
\item \href{https://www-users.york.ac.uk/peter.nightingale/savilerow//}{EssencePrime/SavilleRow} (UK) 
\item \href{https://cpmpy.readthedocs.io/en/latest/}{CPMpy} (KU Leuven)
\item Allow use of multiple back-end solvers
\item Compile model into variants for each solver
\item A priori solver independent model(CP, MIP, SAT)
\end{itemize}
\end{frame} 

\begin{frame}
\frametitle{Framework Process}
\begin{center}
\begin{tikzpicture}[xscale=2]
\node[shape=rectangle,fill=pantone127-4] (problem) at (2.5,5) {Problem};
\node (human) at (2.5,4) {Human};
\node[shape=rectangle,fill=pantone127-4] (model) at (2.5,3) {Model};
\node (compile) at (2.5,2) {Compile/Reformulate};
\node (solver1) at (1,1) {CP};
\node (solver2) at (2,1) {MIP};
\node (solver3) at (3,1) {SAT};
\node (solver4) at (4,1) {Other};
\node[shape=rectangle,fill=pantone157-8] (solution1) at (1,0) {Solution};
\node[shape=rectangle,fill=pantone157-8] (solution2) at (2,0) {Solution};
\node[shape=rectangle,fill=pantone157-8] (solution3) at (3,0) {Solution};
\node[shape=rectangle,fill=pantone157-8] (solution4) at (4,0) {Solution};
\draw[-] (problem) -- (human);
\draw[->] (human) -- (model);
\draw[-] (model) -- (compile);
\draw[-] (compile) -- (solver1);
\draw[-] (compile) -- (solver2);
\draw[-] (compile) -- (solver3);
\draw[-] (compile) -- (solver4);
\draw[->] (solver1) -- (solution1);
\draw[->] (solver2) -- (solution2);
\draw[->] (solver3) -- (solution3);
\draw[->] (solver4) -- (solution4);
\end{tikzpicture}
\end{center}
\end{frame}

% \begin{frame}
  % \frametitle{Do It Now!}
  % \begin{itemize}
  % \item Download and install Minizinc
  % \item \url{https://www.minizinc.org/}
  % \end{itemize}
% \end{frame}


\begin{frame}
\frametitle{Why use Puzzles as Examples?}
\begin{itemize}
\item Easy to understand the problem
\item Solvable by hand without specialized knowledge
\item Possible to compare automated to manual solving process
\end{itemize}

\begin{quote}
The puzzle, though inanimate, is presented as a solvable problem without lasting negative consequences, a very low-risk low-reward situation. By being a puzzle, the object is attempting to convince the user that it must be completed.
\end{quote}
{\tiny Source: \href{https://www.everydayrhetoric.com/2020/03/the-rhetoric-of-puzzles/}{Every Day Rhetoric}}
\end{frame}


\mode<all>{
\part{Basic Constraint Propagation}
}
\begin{frame}
\partpage
\end{frame}
\mode<all>{
\input{../sendmore/body}
}

\mode<all>{
\part{Global Constraints}
}
\begin{frame}
\partpage
\end{frame}
\mode<all>{
\input{../sudoku/body}
}

\clearpage
\mode<all>{
\part{Customizing Search}
}
\begin{frame}
\partpage
\end{frame}
\mode<all>{
\input{../nqueen/body}
}


%% \clearpage
%% \mode<all>{
%%   \part{Problem Decomposition}
%% }
%% \begin{frame}
%%   \partpage
%% \end{frame}
%% \mode<all>{
%%   \input{../party/body}
%%   }

%% \clearpage
%% \mode<all>{
%%   \part{More Global Constraints}
%% }
%% \begin{frame}
%%   \partpage
%% \end{frame}
%% \mode<all>{
%%   \input{../car/body}
%%   }

%% \clearpage
%% \mode<all>{
%%   \part{Methodology}
%% }
%% \begin{frame}
%%   \partpage
%% \end{frame}
%% \mode<all>{
%%   \input{../methodology/body}
%%   }

\clearpage
\mode<all>{
\part{What is missing?}
}
\begin{frame}
\partpage
\end{frame}

\begin{frame}
\frametitle{Many Specialized Topics}
\begin{itemize}
\item How to design efficient core engine
\item Hybrids with LP/MIP tools
\item Hybrids with SAT
\item Symmetry breaking
\item Use of MDD/BDD to encode sets of solutions
\item High level modelling tools
\item Debugging/visualization
\end{itemize}
\end{frame}

\begin{frame}
\frametitle{Reformulation}
\begin{itemize}
\item Just because the user has modelled it this way, it doesn't mean we have to solve it that way
\begin{itemize}
\item Replace some constraint(s) by other, equivalent constraints
\item Because we don't have that constraint in our system
\item For performance
\end{itemize}
\end{itemize}
\end{frame}

\begin{frame}
\frametitle{Learning}
\begin{itemize}
\item While solving the problem we can learn how to strengthen the model/search 
\begin{itemize}
\item Understand which constraints/method contribute to propagation and change schedule
\item Learn no-good constraints by explaining failure
\item Adapt search strategy based on search experience
\end{itemize}
\end{itemize}
\end{frame}

%% \begin{frame}
%% \frametitle{Refined Process}
%% \begin{center}
%% \begin{tikzpicture}
%% \node[shape=rectangle,fill=pantone127-4] (problem) at (3,7) {Problem};
%% \node (human) at (3,6) {Human};
%% \node[shape=rectangle,fill=pantone127-4] (model) at (3,5) {User Model};
%% \node (reformulation) at (3,4) {Reformulation};
%% \node[shape=rectangle,fill=pantone127-4] (implementation) at (3,3) {Implementation Model};
%% \node (solver) at (3,2) {Constraint Solver/Search};
%% \node[shape=rectangle,fill=pantone157-8] (solution) at (3,1) {Solution};
%% \draw[-] (problem) -- (human);
%% \draw[->] (human) -- (model);
%% \draw[-] (model) -- (reformulation);
%% \draw[->] (reformulation) -- (implementation);
%% \draw[-] (implementation) -- (solver);
%% \draw[->] (solver) -- (0,2) -- (0,3) -- (implementation);
%% \draw[->] (solver) -- (solution);
%% \end{tikzpicture}
%% \end{center}
%% \end{frame}

%\begin{frame}
%\frametitle{What is CP actually used for?}
%\url{http://hsimonis.wordpress.com}
%\includegraphics[width=10cm]{../introduction/images/blog}
%\end{frame}

%% \begin{frame}
%% \frametitle{Exercises}
%% \begin{itemize}
%% \item Install Constraint Solver of your Choice
%% \item MiniZinc (MiningZinc) will be used in later sessions
%% \item Run Sendmoremoney, Sudoku, Queens examples
%% \item Can you choose which propagator to use for alldifferent?
%% \item Can you use different search strategies?
%% \end{itemize}
%% \end{frame}

\begin{frame}
\frametitle{More Learning Resources}
\begin{itemize}
\item Survey of Methods, Resources, and Formats for Teaching Constraint Programming
\begin{itemize}
\item by Tejas Santanam, Helmut Simonis
\item \url{https://doi.org/10.48550/arXiv.2403.12717}
\item Based on survey of community for WTCP 2023
\item \url{https://hsimonis.github.io/WTCP2023/}
\end{itemize}
\end{itemize}
\end{frame}


}

\clearpage
\mode<all>{
\part{Customizing Search}
}
\begin{frame}
\partpage
\end{frame}
\mode<all>{
\begin{frame}
\frametitle{Objectives}
\begin{itemize}
\item Overview of Core Constraint Programming
\item Three Main Concepts
\begin{itemize}
\item Constraint Propagation
\item Global Constraints
\item Customizing Search
\end{itemize}
%\item Get Some Experience with MiniZinc
\item Topics will be treated in more detail in later parts of the school
\item Based on Examples, not Formal Description
\end{itemize}
\end{frame}

\begin{frame}
\frametitle{Outline}
\begin{itemize}
\item Why Constraint Programming?
\item Constraint Propagation
\item Global Constraints
\item Customizing Search
%\item Choices in Modelling
%\item Decomposition
%\item More Global Constraints
%\item Methodology
%\item What is missing?
\end{itemize}
\end{frame}

% \begin{frame}
  % \frametitle{Using MiniZinc IDE}
  % \begin{itemize}
  % \item Developed in the Australian NICTA project
  % \item Maintained by Monash University
  % \item Modelling tool with multiple back-end solvers
  % \item Available from \url{https://www.minizinc.org/}
  % \end{itemize}
% \end{frame}


% \begin{frame}
% \frametitle{Examples in ECLiPSe}
% \begin{itemize}
% \item Open sourced constraint programming language
% \item Development goes back to 1985
% \item ECRC, ICL, IC-Parc, PTL, Cisco
% \item \url{https://eclipseclp.org/}
% \item Specialities
% \begin{itemize}
% \item Develop new solvers for specific domains
% \item Integration with MIP
% \end{itemize}
% \item Not included in bundled MiniZinc IDE
% \item Specialized visualization tools used here
  % \begin{itemize}
  % \item CP-Viz, Simonis et al. 2010
    % \end{itemize}
% \end{itemize}
% \end{frame}

\begin{frame}
\frametitle{Tutorial Based on ECLiPSe ELearning Course}
\begin{itemize}
\item Self-study course in constraint programming
\item Supported by Cisco Systems and Silicon Valley Community Foundation
\item Multi-media format, video lectures, slides, handout etc
\item {\small \url{https://eclipseclp.org/ELearning/index.html}}
\end{itemize}
\end{frame}

\begin{frame}
\frametitle{Also Part of CRT-AI Constraint Week}
\begin{itemize}
\item Annual one week course on CP and Optimization in Ireland
\item Part of national training program for PhD students in AI
\item \url{https://www.crt-ai.ie/}
\end{itemize}
\end{frame}



\begin{frame}
\frametitle{Constraint Programming - in a nutshell}
\begin{itemize}
\item Declarative description of problems with
\begin{itemize}
\item {\em Variables} which range over (finite) sets of values
\item {\em Constraints} over subsets of variables which restrict possible value combinations
\item A {\em solution} is a value assignment which satisfies all constraints
\end{itemize}

\item Constraint propagation/reasoning
\begin{itemize}
\item Removing inconsistent values for variables
\item Detect failure if constraint can not be satisfied
\item Interaction of constraints via shared variables
\item Incomplete
\end{itemize}

\item Search
\begin{itemize}
\item User controlled assignment of values to variables
\item Each step triggers constraint propagation 
\end{itemize}
\item Different domains require/allow different methods
\end{itemize}
\end{frame}

\begin{frame}
  \frametitle{Constraint Programming is Different}
  \begin{itemize}
  \item Declarative Programming
    \begin{itemize}
    \item Concentrate on what you want
      \item Not how to get there
      \item Program != Algorithm
      \item Program = Model
    \end{itemize}
    \item Applied to Combinatorial Problems
      \begin{itemize}
        \item No complete polynomial algorithms known (exist?)
        \item CP less ad-hoc than heuristics
        \item Models can evolve
  \end{itemize}
  \end{itemize}
  \end{frame}
    
\begin{frame}
  \frametitle{A Subtractive Process}
  \begin{textblock}{4}(8,-3)
    \includegraphics[width=4cm]{../introduction/images/stages}
  \end{textblock}
  \vfill
  \begin{quote}
    ``Oh, bosh, as Mr. Ruskin says. Sculpture, per se, is the simplest thing in the world. All you have to do is to take a big chunk of marble and a hammer and chisel, make up your mind what you are about to create and chip off all the marble you don't want.''-Paris Gaulois.
  \end{quote}
  
  {\tiny Source: \url{https://quoteinvestigator.com/2014/06/22/chip-away/}}
\end{frame}

\begin{frame}
\frametitle{Basic Process}
\begin{center}
\begin{tikzpicture}
\node[shape=rectangle,fill=pantone127-4] (problem) at (2,5) {Problem};
\node (human) at (2,4) {Human};
\node[shape=rectangle,fill=pantone127-4] (model) at (2,3) {Model};
\node (solver) at (2,2) {Constraint Solver/Search};
\node[shape=rectangle,fill=pantone157-8] (solution) at (2,1) {Solution};
\draw[-] (problem) -- (human);
\draw[->] (human) -- (model);
\draw[-] (model) -- (solver);
\draw[->] (solver) -- (solution);
\end{tikzpicture}
\end{center}
\end{frame}

\begin{frame}
\frametitle{More Realistic}
\begin{center}
\begin{tikzpicture}[xscale=4.0]
\node[shape=rectangle,fill=pantone127-4] (problem) at (2,5) {Problem};
\node (human) at (2,4) {Human};
\node[shape=rectangle,fill=pantone127-4] (model) at (2,3) {Model};
\node (solver) at (2,2) {Constraint Solver/Search};
\node[shape=rectangle,fill=pantone24-8] (hangs) at (1,1) {Hangs};
\node[shape=rectangle,fill=pantone157-8] (solution) at (2,1) {Solution};
\node[shape=rectangle,fill=pantone24-8] (wrong) at (3,1) {Wrong Solution};
\draw[-] (problem) -- (human);
\draw[->] (human) -- (model);
\draw[-] (model) -- (solver);
\draw[->] (solver) -- (hangs);
\draw[->] (solver) -- (solution);
\draw[->] (solver) -- (wrong);
\draw[->] (hangs) -- (1,4) -- (human);
\draw[->] (wrong) -- (3,4) -- (human);
\end{tikzpicture}
\end{center}
\end{frame}

\begin{frame}
\frametitle{Dual Role of Model}
\begin{itemize}
\item Allows Human to Express Problem
\begin{itemize}
\item Close to Problem Domain
\item Constraints as Abstractions
\end{itemize}
\item Allows Solver to Execute
\begin{itemize}
\item Variables as Communication Mechanism
\item Constraints as Algorithms
\end{itemize}
\end{itemize}
\end{frame}

\begin{frame}
\frametitle{Modelling Frameworks}
\begin{itemize}
\item \href{https://www.minizinc.org/}{MiniZinc} (NICTA, Monash University, Australia)
\item \href{https://github.com/eomahony/Numberjack}{NumberJack} (Insight, Ireland) 
\item \href{https://www-users.york.ac.uk/peter.nightingale/savilerow//}{EssencePrime/SavilleRow} (UK) 
\item \href{https://cpmpy.readthedocs.io/en/latest/}{CPMpy} (KU Leuven)
\item Allow use of multiple back-end solvers
\item Compile model into variants for each solver
\item A priori solver independent model(CP, MIP, SAT)
\end{itemize}
\end{frame} 

\begin{frame}
\frametitle{Framework Process}
\begin{center}
\begin{tikzpicture}[xscale=2]
\node[shape=rectangle,fill=pantone127-4] (problem) at (2.5,5) {Problem};
\node (human) at (2.5,4) {Human};
\node[shape=rectangle,fill=pantone127-4] (model) at (2.5,3) {Model};
\node (compile) at (2.5,2) {Compile/Reformulate};
\node (solver1) at (1,1) {CP};
\node (solver2) at (2,1) {MIP};
\node (solver3) at (3,1) {SAT};
\node (solver4) at (4,1) {Other};
\node[shape=rectangle,fill=pantone157-8] (solution1) at (1,0) {Solution};
\node[shape=rectangle,fill=pantone157-8] (solution2) at (2,0) {Solution};
\node[shape=rectangle,fill=pantone157-8] (solution3) at (3,0) {Solution};
\node[shape=rectangle,fill=pantone157-8] (solution4) at (4,0) {Solution};
\draw[-] (problem) -- (human);
\draw[->] (human) -- (model);
\draw[-] (model) -- (compile);
\draw[-] (compile) -- (solver1);
\draw[-] (compile) -- (solver2);
\draw[-] (compile) -- (solver3);
\draw[-] (compile) -- (solver4);
\draw[->] (solver1) -- (solution1);
\draw[->] (solver2) -- (solution2);
\draw[->] (solver3) -- (solution3);
\draw[->] (solver4) -- (solution4);
\end{tikzpicture}
\end{center}
\end{frame}

% \begin{frame}
  % \frametitle{Do It Now!}
  % \begin{itemize}
  % \item Download and install Minizinc
  % \item \url{https://www.minizinc.org/}
  % \end{itemize}
% \end{frame}


\begin{frame}
\frametitle{Why use Puzzles as Examples?}
\begin{itemize}
\item Easy to understand the problem
\item Solvable by hand without specialized knowledge
\item Possible to compare automated to manual solving process
\end{itemize}

\begin{quote}
The puzzle, though inanimate, is presented as a solvable problem without lasting negative consequences, a very low-risk low-reward situation. By being a puzzle, the object is attempting to convince the user that it must be completed.
\end{quote}
{\tiny Source: \href{https://www.everydayrhetoric.com/2020/03/the-rhetoric-of-puzzles/}{Every Day Rhetoric}}
\end{frame}


\mode<all>{
\part{Basic Constraint Propagation}
}
\begin{frame}
\partpage
\end{frame}
\mode<all>{
\input{../sendmore/body}
}

\mode<all>{
\part{Global Constraints}
}
\begin{frame}
\partpage
\end{frame}
\mode<all>{
\input{../sudoku/body}
}

\clearpage
\mode<all>{
\part{Customizing Search}
}
\begin{frame}
\partpage
\end{frame}
\mode<all>{
\input{../nqueen/body}
}


%% \clearpage
%% \mode<all>{
%%   \part{Problem Decomposition}
%% }
%% \begin{frame}
%%   \partpage
%% \end{frame}
%% \mode<all>{
%%   \input{../party/body}
%%   }

%% \clearpage
%% \mode<all>{
%%   \part{More Global Constraints}
%% }
%% \begin{frame}
%%   \partpage
%% \end{frame}
%% \mode<all>{
%%   \input{../car/body}
%%   }

%% \clearpage
%% \mode<all>{
%%   \part{Methodology}
%% }
%% \begin{frame}
%%   \partpage
%% \end{frame}
%% \mode<all>{
%%   \input{../methodology/body}
%%   }

\clearpage
\mode<all>{
\part{What is missing?}
}
\begin{frame}
\partpage
\end{frame}

\begin{frame}
\frametitle{Many Specialized Topics}
\begin{itemize}
\item How to design efficient core engine
\item Hybrids with LP/MIP tools
\item Hybrids with SAT
\item Symmetry breaking
\item Use of MDD/BDD to encode sets of solutions
\item High level modelling tools
\item Debugging/visualization
\end{itemize}
\end{frame}

\begin{frame}
\frametitle{Reformulation}
\begin{itemize}
\item Just because the user has modelled it this way, it doesn't mean we have to solve it that way
\begin{itemize}
\item Replace some constraint(s) by other, equivalent constraints
\item Because we don't have that constraint in our system
\item For performance
\end{itemize}
\end{itemize}
\end{frame}

\begin{frame}
\frametitle{Learning}
\begin{itemize}
\item While solving the problem we can learn how to strengthen the model/search 
\begin{itemize}
\item Understand which constraints/method contribute to propagation and change schedule
\item Learn no-good constraints by explaining failure
\item Adapt search strategy based on search experience
\end{itemize}
\end{itemize}
\end{frame}

%% \begin{frame}
%% \frametitle{Refined Process}
%% \begin{center}
%% \begin{tikzpicture}
%% \node[shape=rectangle,fill=pantone127-4] (problem) at (3,7) {Problem};
%% \node (human) at (3,6) {Human};
%% \node[shape=rectangle,fill=pantone127-4] (model) at (3,5) {User Model};
%% \node (reformulation) at (3,4) {Reformulation};
%% \node[shape=rectangle,fill=pantone127-4] (implementation) at (3,3) {Implementation Model};
%% \node (solver) at (3,2) {Constraint Solver/Search};
%% \node[shape=rectangle,fill=pantone157-8] (solution) at (3,1) {Solution};
%% \draw[-] (problem) -- (human);
%% \draw[->] (human) -- (model);
%% \draw[-] (model) -- (reformulation);
%% \draw[->] (reformulation) -- (implementation);
%% \draw[-] (implementation) -- (solver);
%% \draw[->] (solver) -- (0,2) -- (0,3) -- (implementation);
%% \draw[->] (solver) -- (solution);
%% \end{tikzpicture}
%% \end{center}
%% \end{frame}

%\begin{frame}
%\frametitle{What is CP actually used for?}
%\url{http://hsimonis.wordpress.com}
%\includegraphics[width=10cm]{../introduction/images/blog}
%\end{frame}

%% \begin{frame}
%% \frametitle{Exercises}
%% \begin{itemize}
%% \item Install Constraint Solver of your Choice
%% \item MiniZinc (MiningZinc) will be used in later sessions
%% \item Run Sendmoremoney, Sudoku, Queens examples
%% \item Can you choose which propagator to use for alldifferent?
%% \item Can you use different search strategies?
%% \end{itemize}
%% \end{frame}

\begin{frame}
\frametitle{More Learning Resources}
\begin{itemize}
\item Survey of Methods, Resources, and Formats for Teaching Constraint Programming
\begin{itemize}
\item by Tejas Santanam, Helmut Simonis
\item \url{https://doi.org/10.48550/arXiv.2403.12717}
\item Based on survey of community for WTCP 2023
\item \url{https://hsimonis.github.io/WTCP2023/}
\end{itemize}
\end{itemize}
\end{frame}


}


%% \clearpage
%% \mode<all>{
%%   \part{Problem Decomposition}
%% }
%% \begin{frame}
%%   \partpage
%% \end{frame}
%% \mode<all>{
%%   \begin{frame}
\frametitle{Objectives}
\begin{itemize}
\item Overview of Core Constraint Programming
\item Three Main Concepts
\begin{itemize}
\item Constraint Propagation
\item Global Constraints
\item Customizing Search
\end{itemize}
%\item Get Some Experience with MiniZinc
\item Topics will be treated in more detail in later parts of the school
\item Based on Examples, not Formal Description
\end{itemize}
\end{frame}

\begin{frame}
\frametitle{Outline}
\begin{itemize}
\item Why Constraint Programming?
\item Constraint Propagation
\item Global Constraints
\item Customizing Search
%\item Choices in Modelling
%\item Decomposition
%\item More Global Constraints
%\item Methodology
%\item What is missing?
\end{itemize}
\end{frame}

% \begin{frame}
  % \frametitle{Using MiniZinc IDE}
  % \begin{itemize}
  % \item Developed in the Australian NICTA project
  % \item Maintained by Monash University
  % \item Modelling tool with multiple back-end solvers
  % \item Available from \url{https://www.minizinc.org/}
  % \end{itemize}
% \end{frame}


% \begin{frame}
% \frametitle{Examples in ECLiPSe}
% \begin{itemize}
% \item Open sourced constraint programming language
% \item Development goes back to 1985
% \item ECRC, ICL, IC-Parc, PTL, Cisco
% \item \url{https://eclipseclp.org/}
% \item Specialities
% \begin{itemize}
% \item Develop new solvers for specific domains
% \item Integration with MIP
% \end{itemize}
% \item Not included in bundled MiniZinc IDE
% \item Specialized visualization tools used here
  % \begin{itemize}
  % \item CP-Viz, Simonis et al. 2010
    % \end{itemize}
% \end{itemize}
% \end{frame}

\begin{frame}
\frametitle{Tutorial Based on ECLiPSe ELearning Course}
\begin{itemize}
\item Self-study course in constraint programming
\item Supported by Cisco Systems and Silicon Valley Community Foundation
\item Multi-media format, video lectures, slides, handout etc
\item {\small \url{https://eclipseclp.org/ELearning/index.html}}
\end{itemize}
\end{frame}

\begin{frame}
\frametitle{Also Part of CRT-AI Constraint Week}
\begin{itemize}
\item Annual one week course on CP and Optimization in Ireland
\item Part of national training program for PhD students in AI
\item \url{https://www.crt-ai.ie/}
\end{itemize}
\end{frame}



\begin{frame}
\frametitle{Constraint Programming - in a nutshell}
\begin{itemize}
\item Declarative description of problems with
\begin{itemize}
\item {\em Variables} which range over (finite) sets of values
\item {\em Constraints} over subsets of variables which restrict possible value combinations
\item A {\em solution} is a value assignment which satisfies all constraints
\end{itemize}

\item Constraint propagation/reasoning
\begin{itemize}
\item Removing inconsistent values for variables
\item Detect failure if constraint can not be satisfied
\item Interaction of constraints via shared variables
\item Incomplete
\end{itemize}

\item Search
\begin{itemize}
\item User controlled assignment of values to variables
\item Each step triggers constraint propagation 
\end{itemize}
\item Different domains require/allow different methods
\end{itemize}
\end{frame}

\begin{frame}
  \frametitle{Constraint Programming is Different}
  \begin{itemize}
  \item Declarative Programming
    \begin{itemize}
    \item Concentrate on what you want
      \item Not how to get there
      \item Program != Algorithm
      \item Program = Model
    \end{itemize}
    \item Applied to Combinatorial Problems
      \begin{itemize}
        \item No complete polynomial algorithms known (exist?)
        \item CP less ad-hoc than heuristics
        \item Models can evolve
  \end{itemize}
  \end{itemize}
  \end{frame}
    
\begin{frame}
  \frametitle{A Subtractive Process}
  \begin{textblock}{4}(8,-3)
    \includegraphics[width=4cm]{../introduction/images/stages}
  \end{textblock}
  \vfill
  \begin{quote}
    ``Oh, bosh, as Mr. Ruskin says. Sculpture, per se, is the simplest thing in the world. All you have to do is to take a big chunk of marble and a hammer and chisel, make up your mind what you are about to create and chip off all the marble you don't want.''-Paris Gaulois.
  \end{quote}
  
  {\tiny Source: \url{https://quoteinvestigator.com/2014/06/22/chip-away/}}
\end{frame}

\begin{frame}
\frametitle{Basic Process}
\begin{center}
\begin{tikzpicture}
\node[shape=rectangle,fill=pantone127-4] (problem) at (2,5) {Problem};
\node (human) at (2,4) {Human};
\node[shape=rectangle,fill=pantone127-4] (model) at (2,3) {Model};
\node (solver) at (2,2) {Constraint Solver/Search};
\node[shape=rectangle,fill=pantone157-8] (solution) at (2,1) {Solution};
\draw[-] (problem) -- (human);
\draw[->] (human) -- (model);
\draw[-] (model) -- (solver);
\draw[->] (solver) -- (solution);
\end{tikzpicture}
\end{center}
\end{frame}

\begin{frame}
\frametitle{More Realistic}
\begin{center}
\begin{tikzpicture}[xscale=4.0]
\node[shape=rectangle,fill=pantone127-4] (problem) at (2,5) {Problem};
\node (human) at (2,4) {Human};
\node[shape=rectangle,fill=pantone127-4] (model) at (2,3) {Model};
\node (solver) at (2,2) {Constraint Solver/Search};
\node[shape=rectangle,fill=pantone24-8] (hangs) at (1,1) {Hangs};
\node[shape=rectangle,fill=pantone157-8] (solution) at (2,1) {Solution};
\node[shape=rectangle,fill=pantone24-8] (wrong) at (3,1) {Wrong Solution};
\draw[-] (problem) -- (human);
\draw[->] (human) -- (model);
\draw[-] (model) -- (solver);
\draw[->] (solver) -- (hangs);
\draw[->] (solver) -- (solution);
\draw[->] (solver) -- (wrong);
\draw[->] (hangs) -- (1,4) -- (human);
\draw[->] (wrong) -- (3,4) -- (human);
\end{tikzpicture}
\end{center}
\end{frame}

\begin{frame}
\frametitle{Dual Role of Model}
\begin{itemize}
\item Allows Human to Express Problem
\begin{itemize}
\item Close to Problem Domain
\item Constraints as Abstractions
\end{itemize}
\item Allows Solver to Execute
\begin{itemize}
\item Variables as Communication Mechanism
\item Constraints as Algorithms
\end{itemize}
\end{itemize}
\end{frame}

\begin{frame}
\frametitle{Modelling Frameworks}
\begin{itemize}
\item \href{https://www.minizinc.org/}{MiniZinc} (NICTA, Monash University, Australia)
\item \href{https://github.com/eomahony/Numberjack}{NumberJack} (Insight, Ireland) 
\item \href{https://www-users.york.ac.uk/peter.nightingale/savilerow//}{EssencePrime/SavilleRow} (UK) 
\item \href{https://cpmpy.readthedocs.io/en/latest/}{CPMpy} (KU Leuven)
\item Allow use of multiple back-end solvers
\item Compile model into variants for each solver
\item A priori solver independent model(CP, MIP, SAT)
\end{itemize}
\end{frame} 

\begin{frame}
\frametitle{Framework Process}
\begin{center}
\begin{tikzpicture}[xscale=2]
\node[shape=rectangle,fill=pantone127-4] (problem) at (2.5,5) {Problem};
\node (human) at (2.5,4) {Human};
\node[shape=rectangle,fill=pantone127-4] (model) at (2.5,3) {Model};
\node (compile) at (2.5,2) {Compile/Reformulate};
\node (solver1) at (1,1) {CP};
\node (solver2) at (2,1) {MIP};
\node (solver3) at (3,1) {SAT};
\node (solver4) at (4,1) {Other};
\node[shape=rectangle,fill=pantone157-8] (solution1) at (1,0) {Solution};
\node[shape=rectangle,fill=pantone157-8] (solution2) at (2,0) {Solution};
\node[shape=rectangle,fill=pantone157-8] (solution3) at (3,0) {Solution};
\node[shape=rectangle,fill=pantone157-8] (solution4) at (4,0) {Solution};
\draw[-] (problem) -- (human);
\draw[->] (human) -- (model);
\draw[-] (model) -- (compile);
\draw[-] (compile) -- (solver1);
\draw[-] (compile) -- (solver2);
\draw[-] (compile) -- (solver3);
\draw[-] (compile) -- (solver4);
\draw[->] (solver1) -- (solution1);
\draw[->] (solver2) -- (solution2);
\draw[->] (solver3) -- (solution3);
\draw[->] (solver4) -- (solution4);
\end{tikzpicture}
\end{center}
\end{frame}

% \begin{frame}
  % \frametitle{Do It Now!}
  % \begin{itemize}
  % \item Download and install Minizinc
  % \item \url{https://www.minizinc.org/}
  % \end{itemize}
% \end{frame}


\begin{frame}
\frametitle{Why use Puzzles as Examples?}
\begin{itemize}
\item Easy to understand the problem
\item Solvable by hand without specialized knowledge
\item Possible to compare automated to manual solving process
\end{itemize}

\begin{quote}
The puzzle, though inanimate, is presented as a solvable problem without lasting negative consequences, a very low-risk low-reward situation. By being a puzzle, the object is attempting to convince the user that it must be completed.
\end{quote}
{\tiny Source: \href{https://www.everydayrhetoric.com/2020/03/the-rhetoric-of-puzzles/}{Every Day Rhetoric}}
\end{frame}


\mode<all>{
\part{Basic Constraint Propagation}
}
\begin{frame}
\partpage
\end{frame}
\mode<all>{
\input{../sendmore/body}
}

\mode<all>{
\part{Global Constraints}
}
\begin{frame}
\partpage
\end{frame}
\mode<all>{
\input{../sudoku/body}
}

\clearpage
\mode<all>{
\part{Customizing Search}
}
\begin{frame}
\partpage
\end{frame}
\mode<all>{
\input{../nqueen/body}
}


%% \clearpage
%% \mode<all>{
%%   \part{Problem Decomposition}
%% }
%% \begin{frame}
%%   \partpage
%% \end{frame}
%% \mode<all>{
%%   \input{../party/body}
%%   }

%% \clearpage
%% \mode<all>{
%%   \part{More Global Constraints}
%% }
%% \begin{frame}
%%   \partpage
%% \end{frame}
%% \mode<all>{
%%   \input{../car/body}
%%   }

%% \clearpage
%% \mode<all>{
%%   \part{Methodology}
%% }
%% \begin{frame}
%%   \partpage
%% \end{frame}
%% \mode<all>{
%%   \input{../methodology/body}
%%   }

\clearpage
\mode<all>{
\part{What is missing?}
}
\begin{frame}
\partpage
\end{frame}

\begin{frame}
\frametitle{Many Specialized Topics}
\begin{itemize}
\item How to design efficient core engine
\item Hybrids with LP/MIP tools
\item Hybrids with SAT
\item Symmetry breaking
\item Use of MDD/BDD to encode sets of solutions
\item High level modelling tools
\item Debugging/visualization
\end{itemize}
\end{frame}

\begin{frame}
\frametitle{Reformulation}
\begin{itemize}
\item Just because the user has modelled it this way, it doesn't mean we have to solve it that way
\begin{itemize}
\item Replace some constraint(s) by other, equivalent constraints
\item Because we don't have that constraint in our system
\item For performance
\end{itemize}
\end{itemize}
\end{frame}

\begin{frame}
\frametitle{Learning}
\begin{itemize}
\item While solving the problem we can learn how to strengthen the model/search 
\begin{itemize}
\item Understand which constraints/method contribute to propagation and change schedule
\item Learn no-good constraints by explaining failure
\item Adapt search strategy based on search experience
\end{itemize}
\end{itemize}
\end{frame}

%% \begin{frame}
%% \frametitle{Refined Process}
%% \begin{center}
%% \begin{tikzpicture}
%% \node[shape=rectangle,fill=pantone127-4] (problem) at (3,7) {Problem};
%% \node (human) at (3,6) {Human};
%% \node[shape=rectangle,fill=pantone127-4] (model) at (3,5) {User Model};
%% \node (reformulation) at (3,4) {Reformulation};
%% \node[shape=rectangle,fill=pantone127-4] (implementation) at (3,3) {Implementation Model};
%% \node (solver) at (3,2) {Constraint Solver/Search};
%% \node[shape=rectangle,fill=pantone157-8] (solution) at (3,1) {Solution};
%% \draw[-] (problem) -- (human);
%% \draw[->] (human) -- (model);
%% \draw[-] (model) -- (reformulation);
%% \draw[->] (reformulation) -- (implementation);
%% \draw[-] (implementation) -- (solver);
%% \draw[->] (solver) -- (0,2) -- (0,3) -- (implementation);
%% \draw[->] (solver) -- (solution);
%% \end{tikzpicture}
%% \end{center}
%% \end{frame}

%\begin{frame}
%\frametitle{What is CP actually used for?}
%\url{http://hsimonis.wordpress.com}
%\includegraphics[width=10cm]{../introduction/images/blog}
%\end{frame}

%% \begin{frame}
%% \frametitle{Exercises}
%% \begin{itemize}
%% \item Install Constraint Solver of your Choice
%% \item MiniZinc (MiningZinc) will be used in later sessions
%% \item Run Sendmoremoney, Sudoku, Queens examples
%% \item Can you choose which propagator to use for alldifferent?
%% \item Can you use different search strategies?
%% \end{itemize}
%% \end{frame}

\begin{frame}
\frametitle{More Learning Resources}
\begin{itemize}
\item Survey of Methods, Resources, and Formats for Teaching Constraint Programming
\begin{itemize}
\item by Tejas Santanam, Helmut Simonis
\item \url{https://doi.org/10.48550/arXiv.2403.12717}
\item Based on survey of community for WTCP 2023
\item \url{https://hsimonis.github.io/WTCP2023/}
\end{itemize}
\end{itemize}
\end{frame}


%%   }

%% \clearpage
%% \mode<all>{
%%   \part{More Global Constraints}
%% }
%% \begin{frame}
%%   \partpage
%% \end{frame}
%% \mode<all>{
%%   \begin{frame}
\frametitle{Objectives}
\begin{itemize}
\item Overview of Core Constraint Programming
\item Three Main Concepts
\begin{itemize}
\item Constraint Propagation
\item Global Constraints
\item Customizing Search
\end{itemize}
%\item Get Some Experience with MiniZinc
\item Topics will be treated in more detail in later parts of the school
\item Based on Examples, not Formal Description
\end{itemize}
\end{frame}

\begin{frame}
\frametitle{Outline}
\begin{itemize}
\item Why Constraint Programming?
\item Constraint Propagation
\item Global Constraints
\item Customizing Search
%\item Choices in Modelling
%\item Decomposition
%\item More Global Constraints
%\item Methodology
%\item What is missing?
\end{itemize}
\end{frame}

% \begin{frame}
  % \frametitle{Using MiniZinc IDE}
  % \begin{itemize}
  % \item Developed in the Australian NICTA project
  % \item Maintained by Monash University
  % \item Modelling tool with multiple back-end solvers
  % \item Available from \url{https://www.minizinc.org/}
  % \end{itemize}
% \end{frame}


% \begin{frame}
% \frametitle{Examples in ECLiPSe}
% \begin{itemize}
% \item Open sourced constraint programming language
% \item Development goes back to 1985
% \item ECRC, ICL, IC-Parc, PTL, Cisco
% \item \url{https://eclipseclp.org/}
% \item Specialities
% \begin{itemize}
% \item Develop new solvers for specific domains
% \item Integration with MIP
% \end{itemize}
% \item Not included in bundled MiniZinc IDE
% \item Specialized visualization tools used here
  % \begin{itemize}
  % \item CP-Viz, Simonis et al. 2010
    % \end{itemize}
% \end{itemize}
% \end{frame}

\begin{frame}
\frametitle{Tutorial Based on ECLiPSe ELearning Course}
\begin{itemize}
\item Self-study course in constraint programming
\item Supported by Cisco Systems and Silicon Valley Community Foundation
\item Multi-media format, video lectures, slides, handout etc
\item {\small \url{https://eclipseclp.org/ELearning/index.html}}
\end{itemize}
\end{frame}

\begin{frame}
\frametitle{Also Part of CRT-AI Constraint Week}
\begin{itemize}
\item Annual one week course on CP and Optimization in Ireland
\item Part of national training program for PhD students in AI
\item \url{https://www.crt-ai.ie/}
\end{itemize}
\end{frame}



\begin{frame}
\frametitle{Constraint Programming - in a nutshell}
\begin{itemize}
\item Declarative description of problems with
\begin{itemize}
\item {\em Variables} which range over (finite) sets of values
\item {\em Constraints} over subsets of variables which restrict possible value combinations
\item A {\em solution} is a value assignment which satisfies all constraints
\end{itemize}

\item Constraint propagation/reasoning
\begin{itemize}
\item Removing inconsistent values for variables
\item Detect failure if constraint can not be satisfied
\item Interaction of constraints via shared variables
\item Incomplete
\end{itemize}

\item Search
\begin{itemize}
\item User controlled assignment of values to variables
\item Each step triggers constraint propagation 
\end{itemize}
\item Different domains require/allow different methods
\end{itemize}
\end{frame}

\begin{frame}
  \frametitle{Constraint Programming is Different}
  \begin{itemize}
  \item Declarative Programming
    \begin{itemize}
    \item Concentrate on what you want
      \item Not how to get there
      \item Program != Algorithm
      \item Program = Model
    \end{itemize}
    \item Applied to Combinatorial Problems
      \begin{itemize}
        \item No complete polynomial algorithms known (exist?)
        \item CP less ad-hoc than heuristics
        \item Models can evolve
  \end{itemize}
  \end{itemize}
  \end{frame}
    
\begin{frame}
  \frametitle{A Subtractive Process}
  \begin{textblock}{4}(8,-3)
    \includegraphics[width=4cm]{../introduction/images/stages}
  \end{textblock}
  \vfill
  \begin{quote}
    ``Oh, bosh, as Mr. Ruskin says. Sculpture, per se, is the simplest thing in the world. All you have to do is to take a big chunk of marble and a hammer and chisel, make up your mind what you are about to create and chip off all the marble you don't want.''-Paris Gaulois.
  \end{quote}
  
  {\tiny Source: \url{https://quoteinvestigator.com/2014/06/22/chip-away/}}
\end{frame}

\begin{frame}
\frametitle{Basic Process}
\begin{center}
\begin{tikzpicture}
\node[shape=rectangle,fill=pantone127-4] (problem) at (2,5) {Problem};
\node (human) at (2,4) {Human};
\node[shape=rectangle,fill=pantone127-4] (model) at (2,3) {Model};
\node (solver) at (2,2) {Constraint Solver/Search};
\node[shape=rectangle,fill=pantone157-8] (solution) at (2,1) {Solution};
\draw[-] (problem) -- (human);
\draw[->] (human) -- (model);
\draw[-] (model) -- (solver);
\draw[->] (solver) -- (solution);
\end{tikzpicture}
\end{center}
\end{frame}

\begin{frame}
\frametitle{More Realistic}
\begin{center}
\begin{tikzpicture}[xscale=4.0]
\node[shape=rectangle,fill=pantone127-4] (problem) at (2,5) {Problem};
\node (human) at (2,4) {Human};
\node[shape=rectangle,fill=pantone127-4] (model) at (2,3) {Model};
\node (solver) at (2,2) {Constraint Solver/Search};
\node[shape=rectangle,fill=pantone24-8] (hangs) at (1,1) {Hangs};
\node[shape=rectangle,fill=pantone157-8] (solution) at (2,1) {Solution};
\node[shape=rectangle,fill=pantone24-8] (wrong) at (3,1) {Wrong Solution};
\draw[-] (problem) -- (human);
\draw[->] (human) -- (model);
\draw[-] (model) -- (solver);
\draw[->] (solver) -- (hangs);
\draw[->] (solver) -- (solution);
\draw[->] (solver) -- (wrong);
\draw[->] (hangs) -- (1,4) -- (human);
\draw[->] (wrong) -- (3,4) -- (human);
\end{tikzpicture}
\end{center}
\end{frame}

\begin{frame}
\frametitle{Dual Role of Model}
\begin{itemize}
\item Allows Human to Express Problem
\begin{itemize}
\item Close to Problem Domain
\item Constraints as Abstractions
\end{itemize}
\item Allows Solver to Execute
\begin{itemize}
\item Variables as Communication Mechanism
\item Constraints as Algorithms
\end{itemize}
\end{itemize}
\end{frame}

\begin{frame}
\frametitle{Modelling Frameworks}
\begin{itemize}
\item \href{https://www.minizinc.org/}{MiniZinc} (NICTA, Monash University, Australia)
\item \href{https://github.com/eomahony/Numberjack}{NumberJack} (Insight, Ireland) 
\item \href{https://www-users.york.ac.uk/peter.nightingale/savilerow//}{EssencePrime/SavilleRow} (UK) 
\item \href{https://cpmpy.readthedocs.io/en/latest/}{CPMpy} (KU Leuven)
\item Allow use of multiple back-end solvers
\item Compile model into variants for each solver
\item A priori solver independent model(CP, MIP, SAT)
\end{itemize}
\end{frame} 

\begin{frame}
\frametitle{Framework Process}
\begin{center}
\begin{tikzpicture}[xscale=2]
\node[shape=rectangle,fill=pantone127-4] (problem) at (2.5,5) {Problem};
\node (human) at (2.5,4) {Human};
\node[shape=rectangle,fill=pantone127-4] (model) at (2.5,3) {Model};
\node (compile) at (2.5,2) {Compile/Reformulate};
\node (solver1) at (1,1) {CP};
\node (solver2) at (2,1) {MIP};
\node (solver3) at (3,1) {SAT};
\node (solver4) at (4,1) {Other};
\node[shape=rectangle,fill=pantone157-8] (solution1) at (1,0) {Solution};
\node[shape=rectangle,fill=pantone157-8] (solution2) at (2,0) {Solution};
\node[shape=rectangle,fill=pantone157-8] (solution3) at (3,0) {Solution};
\node[shape=rectangle,fill=pantone157-8] (solution4) at (4,0) {Solution};
\draw[-] (problem) -- (human);
\draw[->] (human) -- (model);
\draw[-] (model) -- (compile);
\draw[-] (compile) -- (solver1);
\draw[-] (compile) -- (solver2);
\draw[-] (compile) -- (solver3);
\draw[-] (compile) -- (solver4);
\draw[->] (solver1) -- (solution1);
\draw[->] (solver2) -- (solution2);
\draw[->] (solver3) -- (solution3);
\draw[->] (solver4) -- (solution4);
\end{tikzpicture}
\end{center}
\end{frame}

% \begin{frame}
  % \frametitle{Do It Now!}
  % \begin{itemize}
  % \item Download and install Minizinc
  % \item \url{https://www.minizinc.org/}
  % \end{itemize}
% \end{frame}


\begin{frame}
\frametitle{Why use Puzzles as Examples?}
\begin{itemize}
\item Easy to understand the problem
\item Solvable by hand without specialized knowledge
\item Possible to compare automated to manual solving process
\end{itemize}

\begin{quote}
The puzzle, though inanimate, is presented as a solvable problem without lasting negative consequences, a very low-risk low-reward situation. By being a puzzle, the object is attempting to convince the user that it must be completed.
\end{quote}
{\tiny Source: \href{https://www.everydayrhetoric.com/2020/03/the-rhetoric-of-puzzles/}{Every Day Rhetoric}}
\end{frame}


\mode<all>{
\part{Basic Constraint Propagation}
}
\begin{frame}
\partpage
\end{frame}
\mode<all>{
\input{../sendmore/body}
}

\mode<all>{
\part{Global Constraints}
}
\begin{frame}
\partpage
\end{frame}
\mode<all>{
\input{../sudoku/body}
}

\clearpage
\mode<all>{
\part{Customizing Search}
}
\begin{frame}
\partpage
\end{frame}
\mode<all>{
\input{../nqueen/body}
}


%% \clearpage
%% \mode<all>{
%%   \part{Problem Decomposition}
%% }
%% \begin{frame}
%%   \partpage
%% \end{frame}
%% \mode<all>{
%%   \input{../party/body}
%%   }

%% \clearpage
%% \mode<all>{
%%   \part{More Global Constraints}
%% }
%% \begin{frame}
%%   \partpage
%% \end{frame}
%% \mode<all>{
%%   \input{../car/body}
%%   }

%% \clearpage
%% \mode<all>{
%%   \part{Methodology}
%% }
%% \begin{frame}
%%   \partpage
%% \end{frame}
%% \mode<all>{
%%   \input{../methodology/body}
%%   }

\clearpage
\mode<all>{
\part{What is missing?}
}
\begin{frame}
\partpage
\end{frame}

\begin{frame}
\frametitle{Many Specialized Topics}
\begin{itemize}
\item How to design efficient core engine
\item Hybrids with LP/MIP tools
\item Hybrids with SAT
\item Symmetry breaking
\item Use of MDD/BDD to encode sets of solutions
\item High level modelling tools
\item Debugging/visualization
\end{itemize}
\end{frame}

\begin{frame}
\frametitle{Reformulation}
\begin{itemize}
\item Just because the user has modelled it this way, it doesn't mean we have to solve it that way
\begin{itemize}
\item Replace some constraint(s) by other, equivalent constraints
\item Because we don't have that constraint in our system
\item For performance
\end{itemize}
\end{itemize}
\end{frame}

\begin{frame}
\frametitle{Learning}
\begin{itemize}
\item While solving the problem we can learn how to strengthen the model/search 
\begin{itemize}
\item Understand which constraints/method contribute to propagation and change schedule
\item Learn no-good constraints by explaining failure
\item Adapt search strategy based on search experience
\end{itemize}
\end{itemize}
\end{frame}

%% \begin{frame}
%% \frametitle{Refined Process}
%% \begin{center}
%% \begin{tikzpicture}
%% \node[shape=rectangle,fill=pantone127-4] (problem) at (3,7) {Problem};
%% \node (human) at (3,6) {Human};
%% \node[shape=rectangle,fill=pantone127-4] (model) at (3,5) {User Model};
%% \node (reformulation) at (3,4) {Reformulation};
%% \node[shape=rectangle,fill=pantone127-4] (implementation) at (3,3) {Implementation Model};
%% \node (solver) at (3,2) {Constraint Solver/Search};
%% \node[shape=rectangle,fill=pantone157-8] (solution) at (3,1) {Solution};
%% \draw[-] (problem) -- (human);
%% \draw[->] (human) -- (model);
%% \draw[-] (model) -- (reformulation);
%% \draw[->] (reformulation) -- (implementation);
%% \draw[-] (implementation) -- (solver);
%% \draw[->] (solver) -- (0,2) -- (0,3) -- (implementation);
%% \draw[->] (solver) -- (solution);
%% \end{tikzpicture}
%% \end{center}
%% \end{frame}

%\begin{frame}
%\frametitle{What is CP actually used for?}
%\url{http://hsimonis.wordpress.com}
%\includegraphics[width=10cm]{../introduction/images/blog}
%\end{frame}

%% \begin{frame}
%% \frametitle{Exercises}
%% \begin{itemize}
%% \item Install Constraint Solver of your Choice
%% \item MiniZinc (MiningZinc) will be used in later sessions
%% \item Run Sendmoremoney, Sudoku, Queens examples
%% \item Can you choose which propagator to use for alldifferent?
%% \item Can you use different search strategies?
%% \end{itemize}
%% \end{frame}

\begin{frame}
\frametitle{More Learning Resources}
\begin{itemize}
\item Survey of Methods, Resources, and Formats for Teaching Constraint Programming
\begin{itemize}
\item by Tejas Santanam, Helmut Simonis
\item \url{https://doi.org/10.48550/arXiv.2403.12717}
\item Based on survey of community for WTCP 2023
\item \url{https://hsimonis.github.io/WTCP2023/}
\end{itemize}
\end{itemize}
\end{frame}


%%   }

%% \clearpage
%% \mode<all>{
%%   \part{Methodology}
%% }
%% \begin{frame}
%%   \partpage
%% \end{frame}
%% \mode<all>{
%%   \begin{frame}
\frametitle{Objectives}
\begin{itemize}
\item Overview of Core Constraint Programming
\item Three Main Concepts
\begin{itemize}
\item Constraint Propagation
\item Global Constraints
\item Customizing Search
\end{itemize}
%\item Get Some Experience with MiniZinc
\item Topics will be treated in more detail in later parts of the school
\item Based on Examples, not Formal Description
\end{itemize}
\end{frame}

\begin{frame}
\frametitle{Outline}
\begin{itemize}
\item Why Constraint Programming?
\item Constraint Propagation
\item Global Constraints
\item Customizing Search
%\item Choices in Modelling
%\item Decomposition
%\item More Global Constraints
%\item Methodology
%\item What is missing?
\end{itemize}
\end{frame}

% \begin{frame}
  % \frametitle{Using MiniZinc IDE}
  % \begin{itemize}
  % \item Developed in the Australian NICTA project
  % \item Maintained by Monash University
  % \item Modelling tool with multiple back-end solvers
  % \item Available from \url{https://www.minizinc.org/}
  % \end{itemize}
% \end{frame}


% \begin{frame}
% \frametitle{Examples in ECLiPSe}
% \begin{itemize}
% \item Open sourced constraint programming language
% \item Development goes back to 1985
% \item ECRC, ICL, IC-Parc, PTL, Cisco
% \item \url{https://eclipseclp.org/}
% \item Specialities
% \begin{itemize}
% \item Develop new solvers for specific domains
% \item Integration with MIP
% \end{itemize}
% \item Not included in bundled MiniZinc IDE
% \item Specialized visualization tools used here
  % \begin{itemize}
  % \item CP-Viz, Simonis et al. 2010
    % \end{itemize}
% \end{itemize}
% \end{frame}

\begin{frame}
\frametitle{Tutorial Based on ECLiPSe ELearning Course}
\begin{itemize}
\item Self-study course in constraint programming
\item Supported by Cisco Systems and Silicon Valley Community Foundation
\item Multi-media format, video lectures, slides, handout etc
\item {\small \url{https://eclipseclp.org/ELearning/index.html}}
\end{itemize}
\end{frame}

\begin{frame}
\frametitle{Also Part of CRT-AI Constraint Week}
\begin{itemize}
\item Annual one week course on CP and Optimization in Ireland
\item Part of national training program for PhD students in AI
\item \url{https://www.crt-ai.ie/}
\end{itemize}
\end{frame}



\begin{frame}
\frametitle{Constraint Programming - in a nutshell}
\begin{itemize}
\item Declarative description of problems with
\begin{itemize}
\item {\em Variables} which range over (finite) sets of values
\item {\em Constraints} over subsets of variables which restrict possible value combinations
\item A {\em solution} is a value assignment which satisfies all constraints
\end{itemize}

\item Constraint propagation/reasoning
\begin{itemize}
\item Removing inconsistent values for variables
\item Detect failure if constraint can not be satisfied
\item Interaction of constraints via shared variables
\item Incomplete
\end{itemize}

\item Search
\begin{itemize}
\item User controlled assignment of values to variables
\item Each step triggers constraint propagation 
\end{itemize}
\item Different domains require/allow different methods
\end{itemize}
\end{frame}

\begin{frame}
  \frametitle{Constraint Programming is Different}
  \begin{itemize}
  \item Declarative Programming
    \begin{itemize}
    \item Concentrate on what you want
      \item Not how to get there
      \item Program != Algorithm
      \item Program = Model
    \end{itemize}
    \item Applied to Combinatorial Problems
      \begin{itemize}
        \item No complete polynomial algorithms known (exist?)
        \item CP less ad-hoc than heuristics
        \item Models can evolve
  \end{itemize}
  \end{itemize}
  \end{frame}
    
\begin{frame}
  \frametitle{A Subtractive Process}
  \begin{textblock}{4}(8,-3)
    \includegraphics[width=4cm]{../introduction/images/stages}
  \end{textblock}
  \vfill
  \begin{quote}
    ``Oh, bosh, as Mr. Ruskin says. Sculpture, per se, is the simplest thing in the world. All you have to do is to take a big chunk of marble and a hammer and chisel, make up your mind what you are about to create and chip off all the marble you don't want.''-Paris Gaulois.
  \end{quote}
  
  {\tiny Source: \url{https://quoteinvestigator.com/2014/06/22/chip-away/}}
\end{frame}

\begin{frame}
\frametitle{Basic Process}
\begin{center}
\begin{tikzpicture}
\node[shape=rectangle,fill=pantone127-4] (problem) at (2,5) {Problem};
\node (human) at (2,4) {Human};
\node[shape=rectangle,fill=pantone127-4] (model) at (2,3) {Model};
\node (solver) at (2,2) {Constraint Solver/Search};
\node[shape=rectangle,fill=pantone157-8] (solution) at (2,1) {Solution};
\draw[-] (problem) -- (human);
\draw[->] (human) -- (model);
\draw[-] (model) -- (solver);
\draw[->] (solver) -- (solution);
\end{tikzpicture}
\end{center}
\end{frame}

\begin{frame}
\frametitle{More Realistic}
\begin{center}
\begin{tikzpicture}[xscale=4.0]
\node[shape=rectangle,fill=pantone127-4] (problem) at (2,5) {Problem};
\node (human) at (2,4) {Human};
\node[shape=rectangle,fill=pantone127-4] (model) at (2,3) {Model};
\node (solver) at (2,2) {Constraint Solver/Search};
\node[shape=rectangle,fill=pantone24-8] (hangs) at (1,1) {Hangs};
\node[shape=rectangle,fill=pantone157-8] (solution) at (2,1) {Solution};
\node[shape=rectangle,fill=pantone24-8] (wrong) at (3,1) {Wrong Solution};
\draw[-] (problem) -- (human);
\draw[->] (human) -- (model);
\draw[-] (model) -- (solver);
\draw[->] (solver) -- (hangs);
\draw[->] (solver) -- (solution);
\draw[->] (solver) -- (wrong);
\draw[->] (hangs) -- (1,4) -- (human);
\draw[->] (wrong) -- (3,4) -- (human);
\end{tikzpicture}
\end{center}
\end{frame}

\begin{frame}
\frametitle{Dual Role of Model}
\begin{itemize}
\item Allows Human to Express Problem
\begin{itemize}
\item Close to Problem Domain
\item Constraints as Abstractions
\end{itemize}
\item Allows Solver to Execute
\begin{itemize}
\item Variables as Communication Mechanism
\item Constraints as Algorithms
\end{itemize}
\end{itemize}
\end{frame}

\begin{frame}
\frametitle{Modelling Frameworks}
\begin{itemize}
\item \href{https://www.minizinc.org/}{MiniZinc} (NICTA, Monash University, Australia)
\item \href{https://github.com/eomahony/Numberjack}{NumberJack} (Insight, Ireland) 
\item \href{https://www-users.york.ac.uk/peter.nightingale/savilerow//}{EssencePrime/SavilleRow} (UK) 
\item \href{https://cpmpy.readthedocs.io/en/latest/}{CPMpy} (KU Leuven)
\item Allow use of multiple back-end solvers
\item Compile model into variants for each solver
\item A priori solver independent model(CP, MIP, SAT)
\end{itemize}
\end{frame} 

\begin{frame}
\frametitle{Framework Process}
\begin{center}
\begin{tikzpicture}[xscale=2]
\node[shape=rectangle,fill=pantone127-4] (problem) at (2.5,5) {Problem};
\node (human) at (2.5,4) {Human};
\node[shape=rectangle,fill=pantone127-4] (model) at (2.5,3) {Model};
\node (compile) at (2.5,2) {Compile/Reformulate};
\node (solver1) at (1,1) {CP};
\node (solver2) at (2,1) {MIP};
\node (solver3) at (3,1) {SAT};
\node (solver4) at (4,1) {Other};
\node[shape=rectangle,fill=pantone157-8] (solution1) at (1,0) {Solution};
\node[shape=rectangle,fill=pantone157-8] (solution2) at (2,0) {Solution};
\node[shape=rectangle,fill=pantone157-8] (solution3) at (3,0) {Solution};
\node[shape=rectangle,fill=pantone157-8] (solution4) at (4,0) {Solution};
\draw[-] (problem) -- (human);
\draw[->] (human) -- (model);
\draw[-] (model) -- (compile);
\draw[-] (compile) -- (solver1);
\draw[-] (compile) -- (solver2);
\draw[-] (compile) -- (solver3);
\draw[-] (compile) -- (solver4);
\draw[->] (solver1) -- (solution1);
\draw[->] (solver2) -- (solution2);
\draw[->] (solver3) -- (solution3);
\draw[->] (solver4) -- (solution4);
\end{tikzpicture}
\end{center}
\end{frame}

% \begin{frame}
  % \frametitle{Do It Now!}
  % \begin{itemize}
  % \item Download and install Minizinc
  % \item \url{https://www.minizinc.org/}
  % \end{itemize}
% \end{frame}


\begin{frame}
\frametitle{Why use Puzzles as Examples?}
\begin{itemize}
\item Easy to understand the problem
\item Solvable by hand without specialized knowledge
\item Possible to compare automated to manual solving process
\end{itemize}

\begin{quote}
The puzzle, though inanimate, is presented as a solvable problem without lasting negative consequences, a very low-risk low-reward situation. By being a puzzle, the object is attempting to convince the user that it must be completed.
\end{quote}
{\tiny Source: \href{https://www.everydayrhetoric.com/2020/03/the-rhetoric-of-puzzles/}{Every Day Rhetoric}}
\end{frame}


\mode<all>{
\part{Basic Constraint Propagation}
}
\begin{frame}
\partpage
\end{frame}
\mode<all>{
\input{../sendmore/body}
}

\mode<all>{
\part{Global Constraints}
}
\begin{frame}
\partpage
\end{frame}
\mode<all>{
\input{../sudoku/body}
}

\clearpage
\mode<all>{
\part{Customizing Search}
}
\begin{frame}
\partpage
\end{frame}
\mode<all>{
\input{../nqueen/body}
}


%% \clearpage
%% \mode<all>{
%%   \part{Problem Decomposition}
%% }
%% \begin{frame}
%%   \partpage
%% \end{frame}
%% \mode<all>{
%%   \input{../party/body}
%%   }

%% \clearpage
%% \mode<all>{
%%   \part{More Global Constraints}
%% }
%% \begin{frame}
%%   \partpage
%% \end{frame}
%% \mode<all>{
%%   \input{../car/body}
%%   }

%% \clearpage
%% \mode<all>{
%%   \part{Methodology}
%% }
%% \begin{frame}
%%   \partpage
%% \end{frame}
%% \mode<all>{
%%   \input{../methodology/body}
%%   }

\clearpage
\mode<all>{
\part{What is missing?}
}
\begin{frame}
\partpage
\end{frame}

\begin{frame}
\frametitle{Many Specialized Topics}
\begin{itemize}
\item How to design efficient core engine
\item Hybrids with LP/MIP tools
\item Hybrids with SAT
\item Symmetry breaking
\item Use of MDD/BDD to encode sets of solutions
\item High level modelling tools
\item Debugging/visualization
\end{itemize}
\end{frame}

\begin{frame}
\frametitle{Reformulation}
\begin{itemize}
\item Just because the user has modelled it this way, it doesn't mean we have to solve it that way
\begin{itemize}
\item Replace some constraint(s) by other, equivalent constraints
\item Because we don't have that constraint in our system
\item For performance
\end{itemize}
\end{itemize}
\end{frame}

\begin{frame}
\frametitle{Learning}
\begin{itemize}
\item While solving the problem we can learn how to strengthen the model/search 
\begin{itemize}
\item Understand which constraints/method contribute to propagation and change schedule
\item Learn no-good constraints by explaining failure
\item Adapt search strategy based on search experience
\end{itemize}
\end{itemize}
\end{frame}

%% \begin{frame}
%% \frametitle{Refined Process}
%% \begin{center}
%% \begin{tikzpicture}
%% \node[shape=rectangle,fill=pantone127-4] (problem) at (3,7) {Problem};
%% \node (human) at (3,6) {Human};
%% \node[shape=rectangle,fill=pantone127-4] (model) at (3,5) {User Model};
%% \node (reformulation) at (3,4) {Reformulation};
%% \node[shape=rectangle,fill=pantone127-4] (implementation) at (3,3) {Implementation Model};
%% \node (solver) at (3,2) {Constraint Solver/Search};
%% \node[shape=rectangle,fill=pantone157-8] (solution) at (3,1) {Solution};
%% \draw[-] (problem) -- (human);
%% \draw[->] (human) -- (model);
%% \draw[-] (model) -- (reformulation);
%% \draw[->] (reformulation) -- (implementation);
%% \draw[-] (implementation) -- (solver);
%% \draw[->] (solver) -- (0,2) -- (0,3) -- (implementation);
%% \draw[->] (solver) -- (solution);
%% \end{tikzpicture}
%% \end{center}
%% \end{frame}

%\begin{frame}
%\frametitle{What is CP actually used for?}
%\url{http://hsimonis.wordpress.com}
%\includegraphics[width=10cm]{../introduction/images/blog}
%\end{frame}

%% \begin{frame}
%% \frametitle{Exercises}
%% \begin{itemize}
%% \item Install Constraint Solver of your Choice
%% \item MiniZinc (MiningZinc) will be used in later sessions
%% \item Run Sendmoremoney, Sudoku, Queens examples
%% \item Can you choose which propagator to use for alldifferent?
%% \item Can you use different search strategies?
%% \end{itemize}
%% \end{frame}

\begin{frame}
\frametitle{More Learning Resources}
\begin{itemize}
\item Survey of Methods, Resources, and Formats for Teaching Constraint Programming
\begin{itemize}
\item by Tejas Santanam, Helmut Simonis
\item \url{https://doi.org/10.48550/arXiv.2403.12717}
\item Based on survey of community for WTCP 2023
\item \url{https://hsimonis.github.io/WTCP2023/}
\end{itemize}
\end{itemize}
\end{frame}


%%   }

\clearpage
\mode<all>{
\part{What is missing?}
}
\begin{frame}
\partpage
\end{frame}

\begin{frame}
\frametitle{Many Specialized Topics}
\begin{itemize}
\item How to design efficient core engine
\item Hybrids with LP/MIP tools
\item Hybrids with SAT
\item Symmetry breaking
\item Use of MDD/BDD to encode sets of solutions
\item High level modelling tools
\item Debugging/visualization
\end{itemize}
\end{frame}

\begin{frame}
\frametitle{Reformulation}
\begin{itemize}
\item Just because the user has modelled it this way, it doesn't mean we have to solve it that way
\begin{itemize}
\item Replace some constraint(s) by other, equivalent constraints
\item Because we don't have that constraint in our system
\item For performance
\end{itemize}
\end{itemize}
\end{frame}

\begin{frame}
\frametitle{Learning}
\begin{itemize}
\item While solving the problem we can learn how to strengthen the model/search 
\begin{itemize}
\item Understand which constraints/method contribute to propagation and change schedule
\item Learn no-good constraints by explaining failure
\item Adapt search strategy based on search experience
\end{itemize}
\end{itemize}
\end{frame}

%% \begin{frame}
%% \frametitle{Refined Process}
%% \begin{center}
%% \begin{tikzpicture}
%% \node[shape=rectangle,fill=pantone127-4] (problem) at (3,7) {Problem};
%% \node (human) at (3,6) {Human};
%% \node[shape=rectangle,fill=pantone127-4] (model) at (3,5) {User Model};
%% \node (reformulation) at (3,4) {Reformulation};
%% \node[shape=rectangle,fill=pantone127-4] (implementation) at (3,3) {Implementation Model};
%% \node (solver) at (3,2) {Constraint Solver/Search};
%% \node[shape=rectangle,fill=pantone157-8] (solution) at (3,1) {Solution};
%% \draw[-] (problem) -- (human);
%% \draw[->] (human) -- (model);
%% \draw[-] (model) -- (reformulation);
%% \draw[->] (reformulation) -- (implementation);
%% \draw[-] (implementation) -- (solver);
%% \draw[->] (solver) -- (0,2) -- (0,3) -- (implementation);
%% \draw[->] (solver) -- (solution);
%% \end{tikzpicture}
%% \end{center}
%% \end{frame}

%\begin{frame}
%\frametitle{What is CP actually used for?}
%\url{http://hsimonis.wordpress.com}
%\includegraphics[width=10cm]{../introduction/images/blog}
%\end{frame}

%% \begin{frame}
%% \frametitle{Exercises}
%% \begin{itemize}
%% \item Install Constraint Solver of your Choice
%% \item MiniZinc (MiningZinc) will be used in later sessions
%% \item Run Sendmoremoney, Sudoku, Queens examples
%% \item Can you choose which propagator to use for alldifferent?
%% \item Can you use different search strategies?
%% \end{itemize}
%% \end{frame}

\begin{frame}
\frametitle{More Learning Resources}
\begin{itemize}
\item Survey of Methods, Resources, and Formats for Teaching Constraint Programming
\begin{itemize}
\item by Tejas Santanam, Helmut Simonis
\item \url{https://doi.org/10.48550/arXiv.2403.12717}
\item Based on survey of community for WTCP 2023
\item \url{https://hsimonis.github.io/WTCP2023/}
\end{itemize}
\end{itemize}
\end{frame}


%%   }

\clearpage
\mode<all>{
\part{What is missing?}
}
\begin{frame}
\partpage
\end{frame}

\begin{frame}
\frametitle{Many Specialized Topics}
\begin{itemize}
\item How to design efficient core engine
\item Hybrids with LP/MIP tools
\item Hybrids with SAT
\item Symmetry breaking
\item Use of MDD/BDD to encode sets of solutions
\item High level modelling tools
\item Debugging/visualization
\end{itemize}
\end{frame}

\begin{frame}
\frametitle{Reformulation}
\begin{itemize}
\item Just because the user has modelled it this way, it doesn't mean we have to solve it that way
\begin{itemize}
\item Replace some constraint(s) by other, equivalent constraints
\item Because we don't have that constraint in our system
\item For performance
\end{itemize}
\end{itemize}
\end{frame}

\begin{frame}
\frametitle{Learning}
\begin{itemize}
\item While solving the problem we can learn how to strengthen the model/search 
\begin{itemize}
\item Understand which constraints/method contribute to propagation and change schedule
\item Learn no-good constraints by explaining failure
\item Adapt search strategy based on search experience
\end{itemize}
\end{itemize}
\end{frame}

%% \begin{frame}
%% \frametitle{Refined Process}
%% \begin{center}
%% \begin{tikzpicture}
%% \node[shape=rectangle,fill=pantone127-4] (problem) at (3,7) {Problem};
%% \node (human) at (3,6) {Human};
%% \node[shape=rectangle,fill=pantone127-4] (model) at (3,5) {User Model};
%% \node (reformulation) at (3,4) {Reformulation};
%% \node[shape=rectangle,fill=pantone127-4] (implementation) at (3,3) {Implementation Model};
%% \node (solver) at (3,2) {Constraint Solver/Search};
%% \node[shape=rectangle,fill=pantone157-8] (solution) at (3,1) {Solution};
%% \draw[-] (problem) -- (human);
%% \draw[->] (human) -- (model);
%% \draw[-] (model) -- (reformulation);
%% \draw[->] (reformulation) -- (implementation);
%% \draw[-] (implementation) -- (solver);
%% \draw[->] (solver) -- (0,2) -- (0,3) -- (implementation);
%% \draw[->] (solver) -- (solution);
%% \end{tikzpicture}
%% \end{center}
%% \end{frame}

%\begin{frame}
%\frametitle{What is CP actually used for?}
%\url{http://hsimonis.wordpress.com}
%\includegraphics[width=10cm]{../introduction/images/blog}
%\end{frame}

%% \begin{frame}
%% \frametitle{Exercises}
%% \begin{itemize}
%% \item Install Constraint Solver of your Choice
%% \item MiniZinc (MiningZinc) will be used in later sessions
%% \item Run Sendmoremoney, Sudoku, Queens examples
%% \item Can you choose which propagator to use for alldifferent?
%% \item Can you use different search strategies?
%% \end{itemize}
%% \end{frame}

\begin{frame}
\frametitle{More Learning Resources}
\begin{itemize}
\item Survey of Methods, Resources, and Formats for Teaching Constraint Programming
\begin{itemize}
\item by Tejas Santanam, Helmut Simonis
\item \url{https://doi.org/10.48550/arXiv.2403.12717}
\item Based on survey of community for WTCP 2023
\item \url{https://hsimonis.github.io/WTCP2023/}
\end{itemize}
\end{itemize}
\end{frame}


%%   }

%% \clearpage
%% \mode<all>{
%%   \part{Methodology}
%% }
%% \begin{frame}
%%   \partpage
%% \end{frame}
%% \mode<all>{
%%   \begin{frame}
\frametitle{Objectives}
\begin{itemize}
\item Overview of Core Constraint Programming
\item Three Main Concepts
\begin{itemize}
\item Constraint Propagation
\item Global Constraints
\item Customizing Search
\end{itemize}
%\item Get Some Experience with MiniZinc
\item Topics will be treated in more detail in later parts of the school
\item Based on Examples, not Formal Description
\end{itemize}
\end{frame}

\begin{frame}
\frametitle{Outline}
\begin{itemize}
\item Why Constraint Programming?
\item Constraint Propagation
\item Global Constraints
\item Customizing Search
%\item Choices in Modelling
%\item Decomposition
%\item More Global Constraints
%\item Methodology
%\item What is missing?
\end{itemize}
\end{frame}

% \begin{frame}
  % \frametitle{Using MiniZinc IDE}
  % \begin{itemize}
  % \item Developed in the Australian NICTA project
  % \item Maintained by Monash University
  % \item Modelling tool with multiple back-end solvers
  % \item Available from \url{https://www.minizinc.org/}
  % \end{itemize}
% \end{frame}


% \begin{frame}
% \frametitle{Examples in ECLiPSe}
% \begin{itemize}
% \item Open sourced constraint programming language
% \item Development goes back to 1985
% \item ECRC, ICL, IC-Parc, PTL, Cisco
% \item \url{https://eclipseclp.org/}
% \item Specialities
% \begin{itemize}
% \item Develop new solvers for specific domains
% \item Integration with MIP
% \end{itemize}
% \item Not included in bundled MiniZinc IDE
% \item Specialized visualization tools used here
  % \begin{itemize}
  % \item CP-Viz, Simonis et al. 2010
    % \end{itemize}
% \end{itemize}
% \end{frame}

\begin{frame}
\frametitle{Tutorial Based on ECLiPSe ELearning Course}
\begin{itemize}
\item Self-study course in constraint programming
\item Supported by Cisco Systems and Silicon Valley Community Foundation
\item Multi-media format, video lectures, slides, handout etc
\item {\small \url{https://eclipseclp.org/ELearning/index.html}}
\end{itemize}
\end{frame}

\begin{frame}
\frametitle{Also Part of CRT-AI Constraint Week}
\begin{itemize}
\item Annual one week course on CP and Optimization in Ireland
\item Part of national training program for PhD students in AI
\item \url{https://www.crt-ai.ie/}
\end{itemize}
\end{frame}



\begin{frame}
\frametitle{Constraint Programming - in a nutshell}
\begin{itemize}
\item Declarative description of problems with
\begin{itemize}
\item {\em Variables} which range over (finite) sets of values
\item {\em Constraints} over subsets of variables which restrict possible value combinations
\item A {\em solution} is a value assignment which satisfies all constraints
\end{itemize}

\item Constraint propagation/reasoning
\begin{itemize}
\item Removing inconsistent values for variables
\item Detect failure if constraint can not be satisfied
\item Interaction of constraints via shared variables
\item Incomplete
\end{itemize}

\item Search
\begin{itemize}
\item User controlled assignment of values to variables
\item Each step triggers constraint propagation 
\end{itemize}
\item Different domains require/allow different methods
\end{itemize}
\end{frame}

\begin{frame}
  \frametitle{Constraint Programming is Different}
  \begin{itemize}
  \item Declarative Programming
    \begin{itemize}
    \item Concentrate on what you want
      \item Not how to get there
      \item Program != Algorithm
      \item Program = Model
    \end{itemize}
    \item Applied to Combinatorial Problems
      \begin{itemize}
        \item No complete polynomial algorithms known (exist?)
        \item CP less ad-hoc than heuristics
        \item Models can evolve
  \end{itemize}
  \end{itemize}
  \end{frame}
    
\begin{frame}
  \frametitle{A Subtractive Process}
  \begin{textblock}{4}(8,-3)
    \includegraphics[width=4cm]{../introduction/images/stages}
  \end{textblock}
  \vfill
  \begin{quote}
    ``Oh, bosh, as Mr. Ruskin says. Sculpture, per se, is the simplest thing in the world. All you have to do is to take a big chunk of marble and a hammer and chisel, make up your mind what you are about to create and chip off all the marble you don't want.''-Paris Gaulois.
  \end{quote}
  
  {\tiny Source: \url{https://quoteinvestigator.com/2014/06/22/chip-away/}}
\end{frame}

\begin{frame}
\frametitle{Basic Process}
\begin{center}
\begin{tikzpicture}
\node[shape=rectangle,fill=pantone127-4] (problem) at (2,5) {Problem};
\node (human) at (2,4) {Human};
\node[shape=rectangle,fill=pantone127-4] (model) at (2,3) {Model};
\node (solver) at (2,2) {Constraint Solver/Search};
\node[shape=rectangle,fill=pantone157-8] (solution) at (2,1) {Solution};
\draw[-] (problem) -- (human);
\draw[->] (human) -- (model);
\draw[-] (model) -- (solver);
\draw[->] (solver) -- (solution);
\end{tikzpicture}
\end{center}
\end{frame}

\begin{frame}
\frametitle{More Realistic}
\begin{center}
\begin{tikzpicture}[xscale=4.0]
\node[shape=rectangle,fill=pantone127-4] (problem) at (2,5) {Problem};
\node (human) at (2,4) {Human};
\node[shape=rectangle,fill=pantone127-4] (model) at (2,3) {Model};
\node (solver) at (2,2) {Constraint Solver/Search};
\node[shape=rectangle,fill=pantone24-8] (hangs) at (1,1) {Hangs};
\node[shape=rectangle,fill=pantone157-8] (solution) at (2,1) {Solution};
\node[shape=rectangle,fill=pantone24-8] (wrong) at (3,1) {Wrong Solution};
\draw[-] (problem) -- (human);
\draw[->] (human) -- (model);
\draw[-] (model) -- (solver);
\draw[->] (solver) -- (hangs);
\draw[->] (solver) -- (solution);
\draw[->] (solver) -- (wrong);
\draw[->] (hangs) -- (1,4) -- (human);
\draw[->] (wrong) -- (3,4) -- (human);
\end{tikzpicture}
\end{center}
\end{frame}

\begin{frame}
\frametitle{Dual Role of Model}
\begin{itemize}
\item Allows Human to Express Problem
\begin{itemize}
\item Close to Problem Domain
\item Constraints as Abstractions
\end{itemize}
\item Allows Solver to Execute
\begin{itemize}
\item Variables as Communication Mechanism
\item Constraints as Algorithms
\end{itemize}
\end{itemize}
\end{frame}

\begin{frame}
\frametitle{Modelling Frameworks}
\begin{itemize}
\item \href{https://www.minizinc.org/}{MiniZinc} (NICTA, Monash University, Australia)
\item \href{https://github.com/eomahony/Numberjack}{NumberJack} (Insight, Ireland) 
\item \href{https://www-users.york.ac.uk/peter.nightingale/savilerow//}{EssencePrime/SavilleRow} (UK) 
\item \href{https://cpmpy.readthedocs.io/en/latest/}{CPMpy} (KU Leuven)
\item Allow use of multiple back-end solvers
\item Compile model into variants for each solver
\item A priori solver independent model(CP, MIP, SAT)
\end{itemize}
\end{frame} 

\begin{frame}
\frametitle{Framework Process}
\begin{center}
\begin{tikzpicture}[xscale=2]
\node[shape=rectangle,fill=pantone127-4] (problem) at (2.5,5) {Problem};
\node (human) at (2.5,4) {Human};
\node[shape=rectangle,fill=pantone127-4] (model) at (2.5,3) {Model};
\node (compile) at (2.5,2) {Compile/Reformulate};
\node (solver1) at (1,1) {CP};
\node (solver2) at (2,1) {MIP};
\node (solver3) at (3,1) {SAT};
\node (solver4) at (4,1) {Other};
\node[shape=rectangle,fill=pantone157-8] (solution1) at (1,0) {Solution};
\node[shape=rectangle,fill=pantone157-8] (solution2) at (2,0) {Solution};
\node[shape=rectangle,fill=pantone157-8] (solution3) at (3,0) {Solution};
\node[shape=rectangle,fill=pantone157-8] (solution4) at (4,0) {Solution};
\draw[-] (problem) -- (human);
\draw[->] (human) -- (model);
\draw[-] (model) -- (compile);
\draw[-] (compile) -- (solver1);
\draw[-] (compile) -- (solver2);
\draw[-] (compile) -- (solver3);
\draw[-] (compile) -- (solver4);
\draw[->] (solver1) -- (solution1);
\draw[->] (solver2) -- (solution2);
\draw[->] (solver3) -- (solution3);
\draw[->] (solver4) -- (solution4);
\end{tikzpicture}
\end{center}
\end{frame}

% \begin{frame}
  % \frametitle{Do It Now!}
  % \begin{itemize}
  % \item Download and install Minizinc
  % \item \url{https://www.minizinc.org/}
  % \end{itemize}
% \end{frame}


\begin{frame}
\frametitle{Why use Puzzles as Examples?}
\begin{itemize}
\item Easy to understand the problem
\item Solvable by hand without specialized knowledge
\item Possible to compare automated to manual solving process
\end{itemize}

\begin{quote}
The puzzle, though inanimate, is presented as a solvable problem without lasting negative consequences, a very low-risk low-reward situation. By being a puzzle, the object is attempting to convince the user that it must be completed.
\end{quote}
{\tiny Source: \href{https://www.everydayrhetoric.com/2020/03/the-rhetoric-of-puzzles/}{Every Day Rhetoric}}
\end{frame}


\mode<all>{
\part{Basic Constraint Propagation}
}
\begin{frame}
\partpage
\end{frame}
\mode<all>{
\begin{frame}
\frametitle{Objectives}
\begin{itemize}
\item Overview of Core Constraint Programming
\item Three Main Concepts
\begin{itemize}
\item Constraint Propagation
\item Global Constraints
\item Customizing Search
\end{itemize}
%\item Get Some Experience with MiniZinc
\item Topics will be treated in more detail in later parts of the school
\item Based on Examples, not Formal Description
\end{itemize}
\end{frame}

\begin{frame}
\frametitle{Outline}
\begin{itemize}
\item Why Constraint Programming?
\item Constraint Propagation
\item Global Constraints
\item Customizing Search
%\item Choices in Modelling
%\item Decomposition
%\item More Global Constraints
%\item Methodology
%\item What is missing?
\end{itemize}
\end{frame}

% \begin{frame}
  % \frametitle{Using MiniZinc IDE}
  % \begin{itemize}
  % \item Developed in the Australian NICTA project
  % \item Maintained by Monash University
  % \item Modelling tool with multiple back-end solvers
  % \item Available from \url{https://www.minizinc.org/}
  % \end{itemize}
% \end{frame}


% \begin{frame}
% \frametitle{Examples in ECLiPSe}
% \begin{itemize}
% \item Open sourced constraint programming language
% \item Development goes back to 1985
% \item ECRC, ICL, IC-Parc, PTL, Cisco
% \item \url{https://eclipseclp.org/}
% \item Specialities
% \begin{itemize}
% \item Develop new solvers for specific domains
% \item Integration with MIP
% \end{itemize}
% \item Not included in bundled MiniZinc IDE
% \item Specialized visualization tools used here
  % \begin{itemize}
  % \item CP-Viz, Simonis et al. 2010
    % \end{itemize}
% \end{itemize}
% \end{frame}

\begin{frame}
\frametitle{Tutorial Based on ECLiPSe ELearning Course}
\begin{itemize}
\item Self-study course in constraint programming
\item Supported by Cisco Systems and Silicon Valley Community Foundation
\item Multi-media format, video lectures, slides, handout etc
\item {\small \url{https://eclipseclp.org/ELearning/index.html}}
\end{itemize}
\end{frame}

\begin{frame}
\frametitle{Also Part of CRT-AI Constraint Week}
\begin{itemize}
\item Annual one week course on CP and Optimization in Ireland
\item Part of national training program for PhD students in AI
\item \url{https://www.crt-ai.ie/}
\end{itemize}
\end{frame}



\begin{frame}
\frametitle{Constraint Programming - in a nutshell}
\begin{itemize}
\item Declarative description of problems with
\begin{itemize}
\item {\em Variables} which range over (finite) sets of values
\item {\em Constraints} over subsets of variables which restrict possible value combinations
\item A {\em solution} is a value assignment which satisfies all constraints
\end{itemize}

\item Constraint propagation/reasoning
\begin{itemize}
\item Removing inconsistent values for variables
\item Detect failure if constraint can not be satisfied
\item Interaction of constraints via shared variables
\item Incomplete
\end{itemize}

\item Search
\begin{itemize}
\item User controlled assignment of values to variables
\item Each step triggers constraint propagation 
\end{itemize}
\item Different domains require/allow different methods
\end{itemize}
\end{frame}

\begin{frame}
  \frametitle{Constraint Programming is Different}
  \begin{itemize}
  \item Declarative Programming
    \begin{itemize}
    \item Concentrate on what you want
      \item Not how to get there
      \item Program != Algorithm
      \item Program = Model
    \end{itemize}
    \item Applied to Combinatorial Problems
      \begin{itemize}
        \item No complete polynomial algorithms known (exist?)
        \item CP less ad-hoc than heuristics
        \item Models can evolve
  \end{itemize}
  \end{itemize}
  \end{frame}
    
\begin{frame}
  \frametitle{A Subtractive Process}
  \begin{textblock}{4}(8,-3)
    \includegraphics[width=4cm]{../introduction/images/stages}
  \end{textblock}
  \vfill
  \begin{quote}
    ``Oh, bosh, as Mr. Ruskin says. Sculpture, per se, is the simplest thing in the world. All you have to do is to take a big chunk of marble and a hammer and chisel, make up your mind what you are about to create and chip off all the marble you don't want.''-Paris Gaulois.
  \end{quote}
  
  {\tiny Source: \url{https://quoteinvestigator.com/2014/06/22/chip-away/}}
\end{frame}

\begin{frame}
\frametitle{Basic Process}
\begin{center}
\begin{tikzpicture}
\node[shape=rectangle,fill=pantone127-4] (problem) at (2,5) {Problem};
\node (human) at (2,4) {Human};
\node[shape=rectangle,fill=pantone127-4] (model) at (2,3) {Model};
\node (solver) at (2,2) {Constraint Solver/Search};
\node[shape=rectangle,fill=pantone157-8] (solution) at (2,1) {Solution};
\draw[-] (problem) -- (human);
\draw[->] (human) -- (model);
\draw[-] (model) -- (solver);
\draw[->] (solver) -- (solution);
\end{tikzpicture}
\end{center}
\end{frame}

\begin{frame}
\frametitle{More Realistic}
\begin{center}
\begin{tikzpicture}[xscale=4.0]
\node[shape=rectangle,fill=pantone127-4] (problem) at (2,5) {Problem};
\node (human) at (2,4) {Human};
\node[shape=rectangle,fill=pantone127-4] (model) at (2,3) {Model};
\node (solver) at (2,2) {Constraint Solver/Search};
\node[shape=rectangle,fill=pantone24-8] (hangs) at (1,1) {Hangs};
\node[shape=rectangle,fill=pantone157-8] (solution) at (2,1) {Solution};
\node[shape=rectangle,fill=pantone24-8] (wrong) at (3,1) {Wrong Solution};
\draw[-] (problem) -- (human);
\draw[->] (human) -- (model);
\draw[-] (model) -- (solver);
\draw[->] (solver) -- (hangs);
\draw[->] (solver) -- (solution);
\draw[->] (solver) -- (wrong);
\draw[->] (hangs) -- (1,4) -- (human);
\draw[->] (wrong) -- (3,4) -- (human);
\end{tikzpicture}
\end{center}
\end{frame}

\begin{frame}
\frametitle{Dual Role of Model}
\begin{itemize}
\item Allows Human to Express Problem
\begin{itemize}
\item Close to Problem Domain
\item Constraints as Abstractions
\end{itemize}
\item Allows Solver to Execute
\begin{itemize}
\item Variables as Communication Mechanism
\item Constraints as Algorithms
\end{itemize}
\end{itemize}
\end{frame}

\begin{frame}
\frametitle{Modelling Frameworks}
\begin{itemize}
\item \href{https://www.minizinc.org/}{MiniZinc} (NICTA, Monash University, Australia)
\item \href{https://github.com/eomahony/Numberjack}{NumberJack} (Insight, Ireland) 
\item \href{https://www-users.york.ac.uk/peter.nightingale/savilerow//}{EssencePrime/SavilleRow} (UK) 
\item \href{https://cpmpy.readthedocs.io/en/latest/}{CPMpy} (KU Leuven)
\item Allow use of multiple back-end solvers
\item Compile model into variants for each solver
\item A priori solver independent model(CP, MIP, SAT)
\end{itemize}
\end{frame} 

\begin{frame}
\frametitle{Framework Process}
\begin{center}
\begin{tikzpicture}[xscale=2]
\node[shape=rectangle,fill=pantone127-4] (problem) at (2.5,5) {Problem};
\node (human) at (2.5,4) {Human};
\node[shape=rectangle,fill=pantone127-4] (model) at (2.5,3) {Model};
\node (compile) at (2.5,2) {Compile/Reformulate};
\node (solver1) at (1,1) {CP};
\node (solver2) at (2,1) {MIP};
\node (solver3) at (3,1) {SAT};
\node (solver4) at (4,1) {Other};
\node[shape=rectangle,fill=pantone157-8] (solution1) at (1,0) {Solution};
\node[shape=rectangle,fill=pantone157-8] (solution2) at (2,0) {Solution};
\node[shape=rectangle,fill=pantone157-8] (solution3) at (3,0) {Solution};
\node[shape=rectangle,fill=pantone157-8] (solution4) at (4,0) {Solution};
\draw[-] (problem) -- (human);
\draw[->] (human) -- (model);
\draw[-] (model) -- (compile);
\draw[-] (compile) -- (solver1);
\draw[-] (compile) -- (solver2);
\draw[-] (compile) -- (solver3);
\draw[-] (compile) -- (solver4);
\draw[->] (solver1) -- (solution1);
\draw[->] (solver2) -- (solution2);
\draw[->] (solver3) -- (solution3);
\draw[->] (solver4) -- (solution4);
\end{tikzpicture}
\end{center}
\end{frame}

% \begin{frame}
  % \frametitle{Do It Now!}
  % \begin{itemize}
  % \item Download and install Minizinc
  % \item \url{https://www.minizinc.org/}
  % \end{itemize}
% \end{frame}


\begin{frame}
\frametitle{Why use Puzzles as Examples?}
\begin{itemize}
\item Easy to understand the problem
\item Solvable by hand without specialized knowledge
\item Possible to compare automated to manual solving process
\end{itemize}

\begin{quote}
The puzzle, though inanimate, is presented as a solvable problem without lasting negative consequences, a very low-risk low-reward situation. By being a puzzle, the object is attempting to convince the user that it must be completed.
\end{quote}
{\tiny Source: \href{https://www.everydayrhetoric.com/2020/03/the-rhetoric-of-puzzles/}{Every Day Rhetoric}}
\end{frame}


\mode<all>{
\part{Basic Constraint Propagation}
}
\begin{frame}
\partpage
\end{frame}
\mode<all>{
\begin{frame}
\frametitle{Objectives}
\begin{itemize}
\item Overview of Core Constraint Programming
\item Three Main Concepts
\begin{itemize}
\item Constraint Propagation
\item Global Constraints
\item Customizing Search
\end{itemize}
%\item Get Some Experience with MiniZinc
\item Topics will be treated in more detail in later parts of the school
\item Based on Examples, not Formal Description
\end{itemize}
\end{frame}

\begin{frame}
\frametitle{Outline}
\begin{itemize}
\item Why Constraint Programming?
\item Constraint Propagation
\item Global Constraints
\item Customizing Search
%\item Choices in Modelling
%\item Decomposition
%\item More Global Constraints
%\item Methodology
%\item What is missing?
\end{itemize}
\end{frame}

% \begin{frame}
  % \frametitle{Using MiniZinc IDE}
  % \begin{itemize}
  % \item Developed in the Australian NICTA project
  % \item Maintained by Monash University
  % \item Modelling tool with multiple back-end solvers
  % \item Available from \url{https://www.minizinc.org/}
  % \end{itemize}
% \end{frame}


% \begin{frame}
% \frametitle{Examples in ECLiPSe}
% \begin{itemize}
% \item Open sourced constraint programming language
% \item Development goes back to 1985
% \item ECRC, ICL, IC-Parc, PTL, Cisco
% \item \url{https://eclipseclp.org/}
% \item Specialities
% \begin{itemize}
% \item Develop new solvers for specific domains
% \item Integration with MIP
% \end{itemize}
% \item Not included in bundled MiniZinc IDE
% \item Specialized visualization tools used here
  % \begin{itemize}
  % \item CP-Viz, Simonis et al. 2010
    % \end{itemize}
% \end{itemize}
% \end{frame}

\begin{frame}
\frametitle{Tutorial Based on ECLiPSe ELearning Course}
\begin{itemize}
\item Self-study course in constraint programming
\item Supported by Cisco Systems and Silicon Valley Community Foundation
\item Multi-media format, video lectures, slides, handout etc
\item {\small \url{https://eclipseclp.org/ELearning/index.html}}
\end{itemize}
\end{frame}

\begin{frame}
\frametitle{Also Part of CRT-AI Constraint Week}
\begin{itemize}
\item Annual one week course on CP and Optimization in Ireland
\item Part of national training program for PhD students in AI
\item \url{https://www.crt-ai.ie/}
\end{itemize}
\end{frame}



\begin{frame}
\frametitle{Constraint Programming - in a nutshell}
\begin{itemize}
\item Declarative description of problems with
\begin{itemize}
\item {\em Variables} which range over (finite) sets of values
\item {\em Constraints} over subsets of variables which restrict possible value combinations
\item A {\em solution} is a value assignment which satisfies all constraints
\end{itemize}

\item Constraint propagation/reasoning
\begin{itemize}
\item Removing inconsistent values for variables
\item Detect failure if constraint can not be satisfied
\item Interaction of constraints via shared variables
\item Incomplete
\end{itemize}

\item Search
\begin{itemize}
\item User controlled assignment of values to variables
\item Each step triggers constraint propagation 
\end{itemize}
\item Different domains require/allow different methods
\end{itemize}
\end{frame}

\begin{frame}
  \frametitle{Constraint Programming is Different}
  \begin{itemize}
  \item Declarative Programming
    \begin{itemize}
    \item Concentrate on what you want
      \item Not how to get there
      \item Program != Algorithm
      \item Program = Model
    \end{itemize}
    \item Applied to Combinatorial Problems
      \begin{itemize}
        \item No complete polynomial algorithms known (exist?)
        \item CP less ad-hoc than heuristics
        \item Models can evolve
  \end{itemize}
  \end{itemize}
  \end{frame}
    
\begin{frame}
  \frametitle{A Subtractive Process}
  \begin{textblock}{4}(8,-3)
    \includegraphics[width=4cm]{../introduction/images/stages}
  \end{textblock}
  \vfill
  \begin{quote}
    ``Oh, bosh, as Mr. Ruskin says. Sculpture, per se, is the simplest thing in the world. All you have to do is to take a big chunk of marble and a hammer and chisel, make up your mind what you are about to create and chip off all the marble you don't want.''-Paris Gaulois.
  \end{quote}
  
  {\tiny Source: \url{https://quoteinvestigator.com/2014/06/22/chip-away/}}
\end{frame}

\begin{frame}
\frametitle{Basic Process}
\begin{center}
\begin{tikzpicture}
\node[shape=rectangle,fill=pantone127-4] (problem) at (2,5) {Problem};
\node (human) at (2,4) {Human};
\node[shape=rectangle,fill=pantone127-4] (model) at (2,3) {Model};
\node (solver) at (2,2) {Constraint Solver/Search};
\node[shape=rectangle,fill=pantone157-8] (solution) at (2,1) {Solution};
\draw[-] (problem) -- (human);
\draw[->] (human) -- (model);
\draw[-] (model) -- (solver);
\draw[->] (solver) -- (solution);
\end{tikzpicture}
\end{center}
\end{frame}

\begin{frame}
\frametitle{More Realistic}
\begin{center}
\begin{tikzpicture}[xscale=4.0]
\node[shape=rectangle,fill=pantone127-4] (problem) at (2,5) {Problem};
\node (human) at (2,4) {Human};
\node[shape=rectangle,fill=pantone127-4] (model) at (2,3) {Model};
\node (solver) at (2,2) {Constraint Solver/Search};
\node[shape=rectangle,fill=pantone24-8] (hangs) at (1,1) {Hangs};
\node[shape=rectangle,fill=pantone157-8] (solution) at (2,1) {Solution};
\node[shape=rectangle,fill=pantone24-8] (wrong) at (3,1) {Wrong Solution};
\draw[-] (problem) -- (human);
\draw[->] (human) -- (model);
\draw[-] (model) -- (solver);
\draw[->] (solver) -- (hangs);
\draw[->] (solver) -- (solution);
\draw[->] (solver) -- (wrong);
\draw[->] (hangs) -- (1,4) -- (human);
\draw[->] (wrong) -- (3,4) -- (human);
\end{tikzpicture}
\end{center}
\end{frame}

\begin{frame}
\frametitle{Dual Role of Model}
\begin{itemize}
\item Allows Human to Express Problem
\begin{itemize}
\item Close to Problem Domain
\item Constraints as Abstractions
\end{itemize}
\item Allows Solver to Execute
\begin{itemize}
\item Variables as Communication Mechanism
\item Constraints as Algorithms
\end{itemize}
\end{itemize}
\end{frame}

\begin{frame}
\frametitle{Modelling Frameworks}
\begin{itemize}
\item \href{https://www.minizinc.org/}{MiniZinc} (NICTA, Monash University, Australia)
\item \href{https://github.com/eomahony/Numberjack}{NumberJack} (Insight, Ireland) 
\item \href{https://www-users.york.ac.uk/peter.nightingale/savilerow//}{EssencePrime/SavilleRow} (UK) 
\item \href{https://cpmpy.readthedocs.io/en/latest/}{CPMpy} (KU Leuven)
\item Allow use of multiple back-end solvers
\item Compile model into variants for each solver
\item A priori solver independent model(CP, MIP, SAT)
\end{itemize}
\end{frame} 

\begin{frame}
\frametitle{Framework Process}
\begin{center}
\begin{tikzpicture}[xscale=2]
\node[shape=rectangle,fill=pantone127-4] (problem) at (2.5,5) {Problem};
\node (human) at (2.5,4) {Human};
\node[shape=rectangle,fill=pantone127-4] (model) at (2.5,3) {Model};
\node (compile) at (2.5,2) {Compile/Reformulate};
\node (solver1) at (1,1) {CP};
\node (solver2) at (2,1) {MIP};
\node (solver3) at (3,1) {SAT};
\node (solver4) at (4,1) {Other};
\node[shape=rectangle,fill=pantone157-8] (solution1) at (1,0) {Solution};
\node[shape=rectangle,fill=pantone157-8] (solution2) at (2,0) {Solution};
\node[shape=rectangle,fill=pantone157-8] (solution3) at (3,0) {Solution};
\node[shape=rectangle,fill=pantone157-8] (solution4) at (4,0) {Solution};
\draw[-] (problem) -- (human);
\draw[->] (human) -- (model);
\draw[-] (model) -- (compile);
\draw[-] (compile) -- (solver1);
\draw[-] (compile) -- (solver2);
\draw[-] (compile) -- (solver3);
\draw[-] (compile) -- (solver4);
\draw[->] (solver1) -- (solution1);
\draw[->] (solver2) -- (solution2);
\draw[->] (solver3) -- (solution3);
\draw[->] (solver4) -- (solution4);
\end{tikzpicture}
\end{center}
\end{frame}

% \begin{frame}
  % \frametitle{Do It Now!}
  % \begin{itemize}
  % \item Download and install Minizinc
  % \item \url{https://www.minizinc.org/}
  % \end{itemize}
% \end{frame}


\begin{frame}
\frametitle{Why use Puzzles as Examples?}
\begin{itemize}
\item Easy to understand the problem
\item Solvable by hand without specialized knowledge
\item Possible to compare automated to manual solving process
\end{itemize}

\begin{quote}
The puzzle, though inanimate, is presented as a solvable problem without lasting negative consequences, a very low-risk low-reward situation. By being a puzzle, the object is attempting to convince the user that it must be completed.
\end{quote}
{\tiny Source: \href{https://www.everydayrhetoric.com/2020/03/the-rhetoric-of-puzzles/}{Every Day Rhetoric}}
\end{frame}


\mode<all>{
\part{Basic Constraint Propagation}
}
\begin{frame}
\partpage
\end{frame}
\mode<all>{
\input{../sendmore/body}
}

\mode<all>{
\part{Global Constraints}
}
\begin{frame}
\partpage
\end{frame}
\mode<all>{
\input{../sudoku/body}
}

\clearpage
\mode<all>{
\part{Customizing Search}
}
\begin{frame}
\partpage
\end{frame}
\mode<all>{
\input{../nqueen/body}
}


%% \clearpage
%% \mode<all>{
%%   \part{Problem Decomposition}
%% }
%% \begin{frame}
%%   \partpage
%% \end{frame}
%% \mode<all>{
%%   \input{../party/body}
%%   }

%% \clearpage
%% \mode<all>{
%%   \part{More Global Constraints}
%% }
%% \begin{frame}
%%   \partpage
%% \end{frame}
%% \mode<all>{
%%   \input{../car/body}
%%   }

%% \clearpage
%% \mode<all>{
%%   \part{Methodology}
%% }
%% \begin{frame}
%%   \partpage
%% \end{frame}
%% \mode<all>{
%%   \input{../methodology/body}
%%   }

\clearpage
\mode<all>{
\part{What is missing?}
}
\begin{frame}
\partpage
\end{frame}

\begin{frame}
\frametitle{Many Specialized Topics}
\begin{itemize}
\item How to design efficient core engine
\item Hybrids with LP/MIP tools
\item Hybrids with SAT
\item Symmetry breaking
\item Use of MDD/BDD to encode sets of solutions
\item High level modelling tools
\item Debugging/visualization
\end{itemize}
\end{frame}

\begin{frame}
\frametitle{Reformulation}
\begin{itemize}
\item Just because the user has modelled it this way, it doesn't mean we have to solve it that way
\begin{itemize}
\item Replace some constraint(s) by other, equivalent constraints
\item Because we don't have that constraint in our system
\item For performance
\end{itemize}
\end{itemize}
\end{frame}

\begin{frame}
\frametitle{Learning}
\begin{itemize}
\item While solving the problem we can learn how to strengthen the model/search 
\begin{itemize}
\item Understand which constraints/method contribute to propagation and change schedule
\item Learn no-good constraints by explaining failure
\item Adapt search strategy based on search experience
\end{itemize}
\end{itemize}
\end{frame}

%% \begin{frame}
%% \frametitle{Refined Process}
%% \begin{center}
%% \begin{tikzpicture}
%% \node[shape=rectangle,fill=pantone127-4] (problem) at (3,7) {Problem};
%% \node (human) at (3,6) {Human};
%% \node[shape=rectangle,fill=pantone127-4] (model) at (3,5) {User Model};
%% \node (reformulation) at (3,4) {Reformulation};
%% \node[shape=rectangle,fill=pantone127-4] (implementation) at (3,3) {Implementation Model};
%% \node (solver) at (3,2) {Constraint Solver/Search};
%% \node[shape=rectangle,fill=pantone157-8] (solution) at (3,1) {Solution};
%% \draw[-] (problem) -- (human);
%% \draw[->] (human) -- (model);
%% \draw[-] (model) -- (reformulation);
%% \draw[->] (reformulation) -- (implementation);
%% \draw[-] (implementation) -- (solver);
%% \draw[->] (solver) -- (0,2) -- (0,3) -- (implementation);
%% \draw[->] (solver) -- (solution);
%% \end{tikzpicture}
%% \end{center}
%% \end{frame}

%\begin{frame}
%\frametitle{What is CP actually used for?}
%\url{http://hsimonis.wordpress.com}
%\includegraphics[width=10cm]{../introduction/images/blog}
%\end{frame}

%% \begin{frame}
%% \frametitle{Exercises}
%% \begin{itemize}
%% \item Install Constraint Solver of your Choice
%% \item MiniZinc (MiningZinc) will be used in later sessions
%% \item Run Sendmoremoney, Sudoku, Queens examples
%% \item Can you choose which propagator to use for alldifferent?
%% \item Can you use different search strategies?
%% \end{itemize}
%% \end{frame}

\begin{frame}
\frametitle{More Learning Resources}
\begin{itemize}
\item Survey of Methods, Resources, and Formats for Teaching Constraint Programming
\begin{itemize}
\item by Tejas Santanam, Helmut Simonis
\item \url{https://doi.org/10.48550/arXiv.2403.12717}
\item Based on survey of community for WTCP 2023
\item \url{https://hsimonis.github.io/WTCP2023/}
\end{itemize}
\end{itemize}
\end{frame}


}

\mode<all>{
\part{Global Constraints}
}
\begin{frame}
\partpage
\end{frame}
\mode<all>{
\begin{frame}
\frametitle{Objectives}
\begin{itemize}
\item Overview of Core Constraint Programming
\item Three Main Concepts
\begin{itemize}
\item Constraint Propagation
\item Global Constraints
\item Customizing Search
\end{itemize}
%\item Get Some Experience with MiniZinc
\item Topics will be treated in more detail in later parts of the school
\item Based on Examples, not Formal Description
\end{itemize}
\end{frame}

\begin{frame}
\frametitle{Outline}
\begin{itemize}
\item Why Constraint Programming?
\item Constraint Propagation
\item Global Constraints
\item Customizing Search
%\item Choices in Modelling
%\item Decomposition
%\item More Global Constraints
%\item Methodology
%\item What is missing?
\end{itemize}
\end{frame}

% \begin{frame}
  % \frametitle{Using MiniZinc IDE}
  % \begin{itemize}
  % \item Developed in the Australian NICTA project
  % \item Maintained by Monash University
  % \item Modelling tool with multiple back-end solvers
  % \item Available from \url{https://www.minizinc.org/}
  % \end{itemize}
% \end{frame}


% \begin{frame}
% \frametitle{Examples in ECLiPSe}
% \begin{itemize}
% \item Open sourced constraint programming language
% \item Development goes back to 1985
% \item ECRC, ICL, IC-Parc, PTL, Cisco
% \item \url{https://eclipseclp.org/}
% \item Specialities
% \begin{itemize}
% \item Develop new solvers for specific domains
% \item Integration with MIP
% \end{itemize}
% \item Not included in bundled MiniZinc IDE
% \item Specialized visualization tools used here
  % \begin{itemize}
  % \item CP-Viz, Simonis et al. 2010
    % \end{itemize}
% \end{itemize}
% \end{frame}

\begin{frame}
\frametitle{Tutorial Based on ECLiPSe ELearning Course}
\begin{itemize}
\item Self-study course in constraint programming
\item Supported by Cisco Systems and Silicon Valley Community Foundation
\item Multi-media format, video lectures, slides, handout etc
\item {\small \url{https://eclipseclp.org/ELearning/index.html}}
\end{itemize}
\end{frame}

\begin{frame}
\frametitle{Also Part of CRT-AI Constraint Week}
\begin{itemize}
\item Annual one week course on CP and Optimization in Ireland
\item Part of national training program for PhD students in AI
\item \url{https://www.crt-ai.ie/}
\end{itemize}
\end{frame}



\begin{frame}
\frametitle{Constraint Programming - in a nutshell}
\begin{itemize}
\item Declarative description of problems with
\begin{itemize}
\item {\em Variables} which range over (finite) sets of values
\item {\em Constraints} over subsets of variables which restrict possible value combinations
\item A {\em solution} is a value assignment which satisfies all constraints
\end{itemize}

\item Constraint propagation/reasoning
\begin{itemize}
\item Removing inconsistent values for variables
\item Detect failure if constraint can not be satisfied
\item Interaction of constraints via shared variables
\item Incomplete
\end{itemize}

\item Search
\begin{itemize}
\item User controlled assignment of values to variables
\item Each step triggers constraint propagation 
\end{itemize}
\item Different domains require/allow different methods
\end{itemize}
\end{frame}

\begin{frame}
  \frametitle{Constraint Programming is Different}
  \begin{itemize}
  \item Declarative Programming
    \begin{itemize}
    \item Concentrate on what you want
      \item Not how to get there
      \item Program != Algorithm
      \item Program = Model
    \end{itemize}
    \item Applied to Combinatorial Problems
      \begin{itemize}
        \item No complete polynomial algorithms known (exist?)
        \item CP less ad-hoc than heuristics
        \item Models can evolve
  \end{itemize}
  \end{itemize}
  \end{frame}
    
\begin{frame}
  \frametitle{A Subtractive Process}
  \begin{textblock}{4}(8,-3)
    \includegraphics[width=4cm]{../introduction/images/stages}
  \end{textblock}
  \vfill
  \begin{quote}
    ``Oh, bosh, as Mr. Ruskin says. Sculpture, per se, is the simplest thing in the world. All you have to do is to take a big chunk of marble and a hammer and chisel, make up your mind what you are about to create and chip off all the marble you don't want.''-Paris Gaulois.
  \end{quote}
  
  {\tiny Source: \url{https://quoteinvestigator.com/2014/06/22/chip-away/}}
\end{frame}

\begin{frame}
\frametitle{Basic Process}
\begin{center}
\begin{tikzpicture}
\node[shape=rectangle,fill=pantone127-4] (problem) at (2,5) {Problem};
\node (human) at (2,4) {Human};
\node[shape=rectangle,fill=pantone127-4] (model) at (2,3) {Model};
\node (solver) at (2,2) {Constraint Solver/Search};
\node[shape=rectangle,fill=pantone157-8] (solution) at (2,1) {Solution};
\draw[-] (problem) -- (human);
\draw[->] (human) -- (model);
\draw[-] (model) -- (solver);
\draw[->] (solver) -- (solution);
\end{tikzpicture}
\end{center}
\end{frame}

\begin{frame}
\frametitle{More Realistic}
\begin{center}
\begin{tikzpicture}[xscale=4.0]
\node[shape=rectangle,fill=pantone127-4] (problem) at (2,5) {Problem};
\node (human) at (2,4) {Human};
\node[shape=rectangle,fill=pantone127-4] (model) at (2,3) {Model};
\node (solver) at (2,2) {Constraint Solver/Search};
\node[shape=rectangle,fill=pantone24-8] (hangs) at (1,1) {Hangs};
\node[shape=rectangle,fill=pantone157-8] (solution) at (2,1) {Solution};
\node[shape=rectangle,fill=pantone24-8] (wrong) at (3,1) {Wrong Solution};
\draw[-] (problem) -- (human);
\draw[->] (human) -- (model);
\draw[-] (model) -- (solver);
\draw[->] (solver) -- (hangs);
\draw[->] (solver) -- (solution);
\draw[->] (solver) -- (wrong);
\draw[->] (hangs) -- (1,4) -- (human);
\draw[->] (wrong) -- (3,4) -- (human);
\end{tikzpicture}
\end{center}
\end{frame}

\begin{frame}
\frametitle{Dual Role of Model}
\begin{itemize}
\item Allows Human to Express Problem
\begin{itemize}
\item Close to Problem Domain
\item Constraints as Abstractions
\end{itemize}
\item Allows Solver to Execute
\begin{itemize}
\item Variables as Communication Mechanism
\item Constraints as Algorithms
\end{itemize}
\end{itemize}
\end{frame}

\begin{frame}
\frametitle{Modelling Frameworks}
\begin{itemize}
\item \href{https://www.minizinc.org/}{MiniZinc} (NICTA, Monash University, Australia)
\item \href{https://github.com/eomahony/Numberjack}{NumberJack} (Insight, Ireland) 
\item \href{https://www-users.york.ac.uk/peter.nightingale/savilerow//}{EssencePrime/SavilleRow} (UK) 
\item \href{https://cpmpy.readthedocs.io/en/latest/}{CPMpy} (KU Leuven)
\item Allow use of multiple back-end solvers
\item Compile model into variants for each solver
\item A priori solver independent model(CP, MIP, SAT)
\end{itemize}
\end{frame} 

\begin{frame}
\frametitle{Framework Process}
\begin{center}
\begin{tikzpicture}[xscale=2]
\node[shape=rectangle,fill=pantone127-4] (problem) at (2.5,5) {Problem};
\node (human) at (2.5,4) {Human};
\node[shape=rectangle,fill=pantone127-4] (model) at (2.5,3) {Model};
\node (compile) at (2.5,2) {Compile/Reformulate};
\node (solver1) at (1,1) {CP};
\node (solver2) at (2,1) {MIP};
\node (solver3) at (3,1) {SAT};
\node (solver4) at (4,1) {Other};
\node[shape=rectangle,fill=pantone157-8] (solution1) at (1,0) {Solution};
\node[shape=rectangle,fill=pantone157-8] (solution2) at (2,0) {Solution};
\node[shape=rectangle,fill=pantone157-8] (solution3) at (3,0) {Solution};
\node[shape=rectangle,fill=pantone157-8] (solution4) at (4,0) {Solution};
\draw[-] (problem) -- (human);
\draw[->] (human) -- (model);
\draw[-] (model) -- (compile);
\draw[-] (compile) -- (solver1);
\draw[-] (compile) -- (solver2);
\draw[-] (compile) -- (solver3);
\draw[-] (compile) -- (solver4);
\draw[->] (solver1) -- (solution1);
\draw[->] (solver2) -- (solution2);
\draw[->] (solver3) -- (solution3);
\draw[->] (solver4) -- (solution4);
\end{tikzpicture}
\end{center}
\end{frame}

% \begin{frame}
  % \frametitle{Do It Now!}
  % \begin{itemize}
  % \item Download and install Minizinc
  % \item \url{https://www.minizinc.org/}
  % \end{itemize}
% \end{frame}


\begin{frame}
\frametitle{Why use Puzzles as Examples?}
\begin{itemize}
\item Easy to understand the problem
\item Solvable by hand without specialized knowledge
\item Possible to compare automated to manual solving process
\end{itemize}

\begin{quote}
The puzzle, though inanimate, is presented as a solvable problem without lasting negative consequences, a very low-risk low-reward situation. By being a puzzle, the object is attempting to convince the user that it must be completed.
\end{quote}
{\tiny Source: \href{https://www.everydayrhetoric.com/2020/03/the-rhetoric-of-puzzles/}{Every Day Rhetoric}}
\end{frame}


\mode<all>{
\part{Basic Constraint Propagation}
}
\begin{frame}
\partpage
\end{frame}
\mode<all>{
\input{../sendmore/body}
}

\mode<all>{
\part{Global Constraints}
}
\begin{frame}
\partpage
\end{frame}
\mode<all>{
\input{../sudoku/body}
}

\clearpage
\mode<all>{
\part{Customizing Search}
}
\begin{frame}
\partpage
\end{frame}
\mode<all>{
\input{../nqueen/body}
}


%% \clearpage
%% \mode<all>{
%%   \part{Problem Decomposition}
%% }
%% \begin{frame}
%%   \partpage
%% \end{frame}
%% \mode<all>{
%%   \input{../party/body}
%%   }

%% \clearpage
%% \mode<all>{
%%   \part{More Global Constraints}
%% }
%% \begin{frame}
%%   \partpage
%% \end{frame}
%% \mode<all>{
%%   \input{../car/body}
%%   }

%% \clearpage
%% \mode<all>{
%%   \part{Methodology}
%% }
%% \begin{frame}
%%   \partpage
%% \end{frame}
%% \mode<all>{
%%   \input{../methodology/body}
%%   }

\clearpage
\mode<all>{
\part{What is missing?}
}
\begin{frame}
\partpage
\end{frame}

\begin{frame}
\frametitle{Many Specialized Topics}
\begin{itemize}
\item How to design efficient core engine
\item Hybrids with LP/MIP tools
\item Hybrids with SAT
\item Symmetry breaking
\item Use of MDD/BDD to encode sets of solutions
\item High level modelling tools
\item Debugging/visualization
\end{itemize}
\end{frame}

\begin{frame}
\frametitle{Reformulation}
\begin{itemize}
\item Just because the user has modelled it this way, it doesn't mean we have to solve it that way
\begin{itemize}
\item Replace some constraint(s) by other, equivalent constraints
\item Because we don't have that constraint in our system
\item For performance
\end{itemize}
\end{itemize}
\end{frame}

\begin{frame}
\frametitle{Learning}
\begin{itemize}
\item While solving the problem we can learn how to strengthen the model/search 
\begin{itemize}
\item Understand which constraints/method contribute to propagation and change schedule
\item Learn no-good constraints by explaining failure
\item Adapt search strategy based on search experience
\end{itemize}
\end{itemize}
\end{frame}

%% \begin{frame}
%% \frametitle{Refined Process}
%% \begin{center}
%% \begin{tikzpicture}
%% \node[shape=rectangle,fill=pantone127-4] (problem) at (3,7) {Problem};
%% \node (human) at (3,6) {Human};
%% \node[shape=rectangle,fill=pantone127-4] (model) at (3,5) {User Model};
%% \node (reformulation) at (3,4) {Reformulation};
%% \node[shape=rectangle,fill=pantone127-4] (implementation) at (3,3) {Implementation Model};
%% \node (solver) at (3,2) {Constraint Solver/Search};
%% \node[shape=rectangle,fill=pantone157-8] (solution) at (3,1) {Solution};
%% \draw[-] (problem) -- (human);
%% \draw[->] (human) -- (model);
%% \draw[-] (model) -- (reformulation);
%% \draw[->] (reformulation) -- (implementation);
%% \draw[-] (implementation) -- (solver);
%% \draw[->] (solver) -- (0,2) -- (0,3) -- (implementation);
%% \draw[->] (solver) -- (solution);
%% \end{tikzpicture}
%% \end{center}
%% \end{frame}

%\begin{frame}
%\frametitle{What is CP actually used for?}
%\url{http://hsimonis.wordpress.com}
%\includegraphics[width=10cm]{../introduction/images/blog}
%\end{frame}

%% \begin{frame}
%% \frametitle{Exercises}
%% \begin{itemize}
%% \item Install Constraint Solver of your Choice
%% \item MiniZinc (MiningZinc) will be used in later sessions
%% \item Run Sendmoremoney, Sudoku, Queens examples
%% \item Can you choose which propagator to use for alldifferent?
%% \item Can you use different search strategies?
%% \end{itemize}
%% \end{frame}

\begin{frame}
\frametitle{More Learning Resources}
\begin{itemize}
\item Survey of Methods, Resources, and Formats for Teaching Constraint Programming
\begin{itemize}
\item by Tejas Santanam, Helmut Simonis
\item \url{https://doi.org/10.48550/arXiv.2403.12717}
\item Based on survey of community for WTCP 2023
\item \url{https://hsimonis.github.io/WTCP2023/}
\end{itemize}
\end{itemize}
\end{frame}


}

\clearpage
\mode<all>{
\part{Customizing Search}
}
\begin{frame}
\partpage
\end{frame}
\mode<all>{
\begin{frame}
\frametitle{Objectives}
\begin{itemize}
\item Overview of Core Constraint Programming
\item Three Main Concepts
\begin{itemize}
\item Constraint Propagation
\item Global Constraints
\item Customizing Search
\end{itemize}
%\item Get Some Experience with MiniZinc
\item Topics will be treated in more detail in later parts of the school
\item Based on Examples, not Formal Description
\end{itemize}
\end{frame}

\begin{frame}
\frametitle{Outline}
\begin{itemize}
\item Why Constraint Programming?
\item Constraint Propagation
\item Global Constraints
\item Customizing Search
%\item Choices in Modelling
%\item Decomposition
%\item More Global Constraints
%\item Methodology
%\item What is missing?
\end{itemize}
\end{frame}

% \begin{frame}
  % \frametitle{Using MiniZinc IDE}
  % \begin{itemize}
  % \item Developed in the Australian NICTA project
  % \item Maintained by Monash University
  % \item Modelling tool with multiple back-end solvers
  % \item Available from \url{https://www.minizinc.org/}
  % \end{itemize}
% \end{frame}


% \begin{frame}
% \frametitle{Examples in ECLiPSe}
% \begin{itemize}
% \item Open sourced constraint programming language
% \item Development goes back to 1985
% \item ECRC, ICL, IC-Parc, PTL, Cisco
% \item \url{https://eclipseclp.org/}
% \item Specialities
% \begin{itemize}
% \item Develop new solvers for specific domains
% \item Integration with MIP
% \end{itemize}
% \item Not included in bundled MiniZinc IDE
% \item Specialized visualization tools used here
  % \begin{itemize}
  % \item CP-Viz, Simonis et al. 2010
    % \end{itemize}
% \end{itemize}
% \end{frame}

\begin{frame}
\frametitle{Tutorial Based on ECLiPSe ELearning Course}
\begin{itemize}
\item Self-study course in constraint programming
\item Supported by Cisco Systems and Silicon Valley Community Foundation
\item Multi-media format, video lectures, slides, handout etc
\item {\small \url{https://eclipseclp.org/ELearning/index.html}}
\end{itemize}
\end{frame}

\begin{frame}
\frametitle{Also Part of CRT-AI Constraint Week}
\begin{itemize}
\item Annual one week course on CP and Optimization in Ireland
\item Part of national training program for PhD students in AI
\item \url{https://www.crt-ai.ie/}
\end{itemize}
\end{frame}



\begin{frame}
\frametitle{Constraint Programming - in a nutshell}
\begin{itemize}
\item Declarative description of problems with
\begin{itemize}
\item {\em Variables} which range over (finite) sets of values
\item {\em Constraints} over subsets of variables which restrict possible value combinations
\item A {\em solution} is a value assignment which satisfies all constraints
\end{itemize}

\item Constraint propagation/reasoning
\begin{itemize}
\item Removing inconsistent values for variables
\item Detect failure if constraint can not be satisfied
\item Interaction of constraints via shared variables
\item Incomplete
\end{itemize}

\item Search
\begin{itemize}
\item User controlled assignment of values to variables
\item Each step triggers constraint propagation 
\end{itemize}
\item Different domains require/allow different methods
\end{itemize}
\end{frame}

\begin{frame}
  \frametitle{Constraint Programming is Different}
  \begin{itemize}
  \item Declarative Programming
    \begin{itemize}
    \item Concentrate on what you want
      \item Not how to get there
      \item Program != Algorithm
      \item Program = Model
    \end{itemize}
    \item Applied to Combinatorial Problems
      \begin{itemize}
        \item No complete polynomial algorithms known (exist?)
        \item CP less ad-hoc than heuristics
        \item Models can evolve
  \end{itemize}
  \end{itemize}
  \end{frame}
    
\begin{frame}
  \frametitle{A Subtractive Process}
  \begin{textblock}{4}(8,-3)
    \includegraphics[width=4cm]{../introduction/images/stages}
  \end{textblock}
  \vfill
  \begin{quote}
    ``Oh, bosh, as Mr. Ruskin says. Sculpture, per se, is the simplest thing in the world. All you have to do is to take a big chunk of marble and a hammer and chisel, make up your mind what you are about to create and chip off all the marble you don't want.''-Paris Gaulois.
  \end{quote}
  
  {\tiny Source: \url{https://quoteinvestigator.com/2014/06/22/chip-away/}}
\end{frame}

\begin{frame}
\frametitle{Basic Process}
\begin{center}
\begin{tikzpicture}
\node[shape=rectangle,fill=pantone127-4] (problem) at (2,5) {Problem};
\node (human) at (2,4) {Human};
\node[shape=rectangle,fill=pantone127-4] (model) at (2,3) {Model};
\node (solver) at (2,2) {Constraint Solver/Search};
\node[shape=rectangle,fill=pantone157-8] (solution) at (2,1) {Solution};
\draw[-] (problem) -- (human);
\draw[->] (human) -- (model);
\draw[-] (model) -- (solver);
\draw[->] (solver) -- (solution);
\end{tikzpicture}
\end{center}
\end{frame}

\begin{frame}
\frametitle{More Realistic}
\begin{center}
\begin{tikzpicture}[xscale=4.0]
\node[shape=rectangle,fill=pantone127-4] (problem) at (2,5) {Problem};
\node (human) at (2,4) {Human};
\node[shape=rectangle,fill=pantone127-4] (model) at (2,3) {Model};
\node (solver) at (2,2) {Constraint Solver/Search};
\node[shape=rectangle,fill=pantone24-8] (hangs) at (1,1) {Hangs};
\node[shape=rectangle,fill=pantone157-8] (solution) at (2,1) {Solution};
\node[shape=rectangle,fill=pantone24-8] (wrong) at (3,1) {Wrong Solution};
\draw[-] (problem) -- (human);
\draw[->] (human) -- (model);
\draw[-] (model) -- (solver);
\draw[->] (solver) -- (hangs);
\draw[->] (solver) -- (solution);
\draw[->] (solver) -- (wrong);
\draw[->] (hangs) -- (1,4) -- (human);
\draw[->] (wrong) -- (3,4) -- (human);
\end{tikzpicture}
\end{center}
\end{frame}

\begin{frame}
\frametitle{Dual Role of Model}
\begin{itemize}
\item Allows Human to Express Problem
\begin{itemize}
\item Close to Problem Domain
\item Constraints as Abstractions
\end{itemize}
\item Allows Solver to Execute
\begin{itemize}
\item Variables as Communication Mechanism
\item Constraints as Algorithms
\end{itemize}
\end{itemize}
\end{frame}

\begin{frame}
\frametitle{Modelling Frameworks}
\begin{itemize}
\item \href{https://www.minizinc.org/}{MiniZinc} (NICTA, Monash University, Australia)
\item \href{https://github.com/eomahony/Numberjack}{NumberJack} (Insight, Ireland) 
\item \href{https://www-users.york.ac.uk/peter.nightingale/savilerow//}{EssencePrime/SavilleRow} (UK) 
\item \href{https://cpmpy.readthedocs.io/en/latest/}{CPMpy} (KU Leuven)
\item Allow use of multiple back-end solvers
\item Compile model into variants for each solver
\item A priori solver independent model(CP, MIP, SAT)
\end{itemize}
\end{frame} 

\begin{frame}
\frametitle{Framework Process}
\begin{center}
\begin{tikzpicture}[xscale=2]
\node[shape=rectangle,fill=pantone127-4] (problem) at (2.5,5) {Problem};
\node (human) at (2.5,4) {Human};
\node[shape=rectangle,fill=pantone127-4] (model) at (2.5,3) {Model};
\node (compile) at (2.5,2) {Compile/Reformulate};
\node (solver1) at (1,1) {CP};
\node (solver2) at (2,1) {MIP};
\node (solver3) at (3,1) {SAT};
\node (solver4) at (4,1) {Other};
\node[shape=rectangle,fill=pantone157-8] (solution1) at (1,0) {Solution};
\node[shape=rectangle,fill=pantone157-8] (solution2) at (2,0) {Solution};
\node[shape=rectangle,fill=pantone157-8] (solution3) at (3,0) {Solution};
\node[shape=rectangle,fill=pantone157-8] (solution4) at (4,0) {Solution};
\draw[-] (problem) -- (human);
\draw[->] (human) -- (model);
\draw[-] (model) -- (compile);
\draw[-] (compile) -- (solver1);
\draw[-] (compile) -- (solver2);
\draw[-] (compile) -- (solver3);
\draw[-] (compile) -- (solver4);
\draw[->] (solver1) -- (solution1);
\draw[->] (solver2) -- (solution2);
\draw[->] (solver3) -- (solution3);
\draw[->] (solver4) -- (solution4);
\end{tikzpicture}
\end{center}
\end{frame}

% \begin{frame}
  % \frametitle{Do It Now!}
  % \begin{itemize}
  % \item Download and install Minizinc
  % \item \url{https://www.minizinc.org/}
  % \end{itemize}
% \end{frame}


\begin{frame}
\frametitle{Why use Puzzles as Examples?}
\begin{itemize}
\item Easy to understand the problem
\item Solvable by hand without specialized knowledge
\item Possible to compare automated to manual solving process
\end{itemize}

\begin{quote}
The puzzle, though inanimate, is presented as a solvable problem without lasting negative consequences, a very low-risk low-reward situation. By being a puzzle, the object is attempting to convince the user that it must be completed.
\end{quote}
{\tiny Source: \href{https://www.everydayrhetoric.com/2020/03/the-rhetoric-of-puzzles/}{Every Day Rhetoric}}
\end{frame}


\mode<all>{
\part{Basic Constraint Propagation}
}
\begin{frame}
\partpage
\end{frame}
\mode<all>{
\input{../sendmore/body}
}

\mode<all>{
\part{Global Constraints}
}
\begin{frame}
\partpage
\end{frame}
\mode<all>{
\input{../sudoku/body}
}

\clearpage
\mode<all>{
\part{Customizing Search}
}
\begin{frame}
\partpage
\end{frame}
\mode<all>{
\input{../nqueen/body}
}


%% \clearpage
%% \mode<all>{
%%   \part{Problem Decomposition}
%% }
%% \begin{frame}
%%   \partpage
%% \end{frame}
%% \mode<all>{
%%   \input{../party/body}
%%   }

%% \clearpage
%% \mode<all>{
%%   \part{More Global Constraints}
%% }
%% \begin{frame}
%%   \partpage
%% \end{frame}
%% \mode<all>{
%%   \input{../car/body}
%%   }

%% \clearpage
%% \mode<all>{
%%   \part{Methodology}
%% }
%% \begin{frame}
%%   \partpage
%% \end{frame}
%% \mode<all>{
%%   \input{../methodology/body}
%%   }

\clearpage
\mode<all>{
\part{What is missing?}
}
\begin{frame}
\partpage
\end{frame}

\begin{frame}
\frametitle{Many Specialized Topics}
\begin{itemize}
\item How to design efficient core engine
\item Hybrids with LP/MIP tools
\item Hybrids with SAT
\item Symmetry breaking
\item Use of MDD/BDD to encode sets of solutions
\item High level modelling tools
\item Debugging/visualization
\end{itemize}
\end{frame}

\begin{frame}
\frametitle{Reformulation}
\begin{itemize}
\item Just because the user has modelled it this way, it doesn't mean we have to solve it that way
\begin{itemize}
\item Replace some constraint(s) by other, equivalent constraints
\item Because we don't have that constraint in our system
\item For performance
\end{itemize}
\end{itemize}
\end{frame}

\begin{frame}
\frametitle{Learning}
\begin{itemize}
\item While solving the problem we can learn how to strengthen the model/search 
\begin{itemize}
\item Understand which constraints/method contribute to propagation and change schedule
\item Learn no-good constraints by explaining failure
\item Adapt search strategy based on search experience
\end{itemize}
\end{itemize}
\end{frame}

%% \begin{frame}
%% \frametitle{Refined Process}
%% \begin{center}
%% \begin{tikzpicture}
%% \node[shape=rectangle,fill=pantone127-4] (problem) at (3,7) {Problem};
%% \node (human) at (3,6) {Human};
%% \node[shape=rectangle,fill=pantone127-4] (model) at (3,5) {User Model};
%% \node (reformulation) at (3,4) {Reformulation};
%% \node[shape=rectangle,fill=pantone127-4] (implementation) at (3,3) {Implementation Model};
%% \node (solver) at (3,2) {Constraint Solver/Search};
%% \node[shape=rectangle,fill=pantone157-8] (solution) at (3,1) {Solution};
%% \draw[-] (problem) -- (human);
%% \draw[->] (human) -- (model);
%% \draw[-] (model) -- (reformulation);
%% \draw[->] (reformulation) -- (implementation);
%% \draw[-] (implementation) -- (solver);
%% \draw[->] (solver) -- (0,2) -- (0,3) -- (implementation);
%% \draw[->] (solver) -- (solution);
%% \end{tikzpicture}
%% \end{center}
%% \end{frame}

%\begin{frame}
%\frametitle{What is CP actually used for?}
%\url{http://hsimonis.wordpress.com}
%\includegraphics[width=10cm]{../introduction/images/blog}
%\end{frame}

%% \begin{frame}
%% \frametitle{Exercises}
%% \begin{itemize}
%% \item Install Constraint Solver of your Choice
%% \item MiniZinc (MiningZinc) will be used in later sessions
%% \item Run Sendmoremoney, Sudoku, Queens examples
%% \item Can you choose which propagator to use for alldifferent?
%% \item Can you use different search strategies?
%% \end{itemize}
%% \end{frame}

\begin{frame}
\frametitle{More Learning Resources}
\begin{itemize}
\item Survey of Methods, Resources, and Formats for Teaching Constraint Programming
\begin{itemize}
\item by Tejas Santanam, Helmut Simonis
\item \url{https://doi.org/10.48550/arXiv.2403.12717}
\item Based on survey of community for WTCP 2023
\item \url{https://hsimonis.github.io/WTCP2023/}
\end{itemize}
\end{itemize}
\end{frame}


}


%% \clearpage
%% \mode<all>{
%%   \part{Problem Decomposition}
%% }
%% \begin{frame}
%%   \partpage
%% \end{frame}
%% \mode<all>{
%%   \begin{frame}
\frametitle{Objectives}
\begin{itemize}
\item Overview of Core Constraint Programming
\item Three Main Concepts
\begin{itemize}
\item Constraint Propagation
\item Global Constraints
\item Customizing Search
\end{itemize}
%\item Get Some Experience with MiniZinc
\item Topics will be treated in more detail in later parts of the school
\item Based on Examples, not Formal Description
\end{itemize}
\end{frame}

\begin{frame}
\frametitle{Outline}
\begin{itemize}
\item Why Constraint Programming?
\item Constraint Propagation
\item Global Constraints
\item Customizing Search
%\item Choices in Modelling
%\item Decomposition
%\item More Global Constraints
%\item Methodology
%\item What is missing?
\end{itemize}
\end{frame}

% \begin{frame}
  % \frametitle{Using MiniZinc IDE}
  % \begin{itemize}
  % \item Developed in the Australian NICTA project
  % \item Maintained by Monash University
  % \item Modelling tool with multiple back-end solvers
  % \item Available from \url{https://www.minizinc.org/}
  % \end{itemize}
% \end{frame}


% \begin{frame}
% \frametitle{Examples in ECLiPSe}
% \begin{itemize}
% \item Open sourced constraint programming language
% \item Development goes back to 1985
% \item ECRC, ICL, IC-Parc, PTL, Cisco
% \item \url{https://eclipseclp.org/}
% \item Specialities
% \begin{itemize}
% \item Develop new solvers for specific domains
% \item Integration with MIP
% \end{itemize}
% \item Not included in bundled MiniZinc IDE
% \item Specialized visualization tools used here
  % \begin{itemize}
  % \item CP-Viz, Simonis et al. 2010
    % \end{itemize}
% \end{itemize}
% \end{frame}

\begin{frame}
\frametitle{Tutorial Based on ECLiPSe ELearning Course}
\begin{itemize}
\item Self-study course in constraint programming
\item Supported by Cisco Systems and Silicon Valley Community Foundation
\item Multi-media format, video lectures, slides, handout etc
\item {\small \url{https://eclipseclp.org/ELearning/index.html}}
\end{itemize}
\end{frame}

\begin{frame}
\frametitle{Also Part of CRT-AI Constraint Week}
\begin{itemize}
\item Annual one week course on CP and Optimization in Ireland
\item Part of national training program for PhD students in AI
\item \url{https://www.crt-ai.ie/}
\end{itemize}
\end{frame}



\begin{frame}
\frametitle{Constraint Programming - in a nutshell}
\begin{itemize}
\item Declarative description of problems with
\begin{itemize}
\item {\em Variables} which range over (finite) sets of values
\item {\em Constraints} over subsets of variables which restrict possible value combinations
\item A {\em solution} is a value assignment which satisfies all constraints
\end{itemize}

\item Constraint propagation/reasoning
\begin{itemize}
\item Removing inconsistent values for variables
\item Detect failure if constraint can not be satisfied
\item Interaction of constraints via shared variables
\item Incomplete
\end{itemize}

\item Search
\begin{itemize}
\item User controlled assignment of values to variables
\item Each step triggers constraint propagation 
\end{itemize}
\item Different domains require/allow different methods
\end{itemize}
\end{frame}

\begin{frame}
  \frametitle{Constraint Programming is Different}
  \begin{itemize}
  \item Declarative Programming
    \begin{itemize}
    \item Concentrate on what you want
      \item Not how to get there
      \item Program != Algorithm
      \item Program = Model
    \end{itemize}
    \item Applied to Combinatorial Problems
      \begin{itemize}
        \item No complete polynomial algorithms known (exist?)
        \item CP less ad-hoc than heuristics
        \item Models can evolve
  \end{itemize}
  \end{itemize}
  \end{frame}
    
\begin{frame}
  \frametitle{A Subtractive Process}
  \begin{textblock}{4}(8,-3)
    \includegraphics[width=4cm]{../introduction/images/stages}
  \end{textblock}
  \vfill
  \begin{quote}
    ``Oh, bosh, as Mr. Ruskin says. Sculpture, per se, is the simplest thing in the world. All you have to do is to take a big chunk of marble and a hammer and chisel, make up your mind what you are about to create and chip off all the marble you don't want.''-Paris Gaulois.
  \end{quote}
  
  {\tiny Source: \url{https://quoteinvestigator.com/2014/06/22/chip-away/}}
\end{frame}

\begin{frame}
\frametitle{Basic Process}
\begin{center}
\begin{tikzpicture}
\node[shape=rectangle,fill=pantone127-4] (problem) at (2,5) {Problem};
\node (human) at (2,4) {Human};
\node[shape=rectangle,fill=pantone127-4] (model) at (2,3) {Model};
\node (solver) at (2,2) {Constraint Solver/Search};
\node[shape=rectangle,fill=pantone157-8] (solution) at (2,1) {Solution};
\draw[-] (problem) -- (human);
\draw[->] (human) -- (model);
\draw[-] (model) -- (solver);
\draw[->] (solver) -- (solution);
\end{tikzpicture}
\end{center}
\end{frame}

\begin{frame}
\frametitle{More Realistic}
\begin{center}
\begin{tikzpicture}[xscale=4.0]
\node[shape=rectangle,fill=pantone127-4] (problem) at (2,5) {Problem};
\node (human) at (2,4) {Human};
\node[shape=rectangle,fill=pantone127-4] (model) at (2,3) {Model};
\node (solver) at (2,2) {Constraint Solver/Search};
\node[shape=rectangle,fill=pantone24-8] (hangs) at (1,1) {Hangs};
\node[shape=rectangle,fill=pantone157-8] (solution) at (2,1) {Solution};
\node[shape=rectangle,fill=pantone24-8] (wrong) at (3,1) {Wrong Solution};
\draw[-] (problem) -- (human);
\draw[->] (human) -- (model);
\draw[-] (model) -- (solver);
\draw[->] (solver) -- (hangs);
\draw[->] (solver) -- (solution);
\draw[->] (solver) -- (wrong);
\draw[->] (hangs) -- (1,4) -- (human);
\draw[->] (wrong) -- (3,4) -- (human);
\end{tikzpicture}
\end{center}
\end{frame}

\begin{frame}
\frametitle{Dual Role of Model}
\begin{itemize}
\item Allows Human to Express Problem
\begin{itemize}
\item Close to Problem Domain
\item Constraints as Abstractions
\end{itemize}
\item Allows Solver to Execute
\begin{itemize}
\item Variables as Communication Mechanism
\item Constraints as Algorithms
\end{itemize}
\end{itemize}
\end{frame}

\begin{frame}
\frametitle{Modelling Frameworks}
\begin{itemize}
\item \href{https://www.minizinc.org/}{MiniZinc} (NICTA, Monash University, Australia)
\item \href{https://github.com/eomahony/Numberjack}{NumberJack} (Insight, Ireland) 
\item \href{https://www-users.york.ac.uk/peter.nightingale/savilerow//}{EssencePrime/SavilleRow} (UK) 
\item \href{https://cpmpy.readthedocs.io/en/latest/}{CPMpy} (KU Leuven)
\item Allow use of multiple back-end solvers
\item Compile model into variants for each solver
\item A priori solver independent model(CP, MIP, SAT)
\end{itemize}
\end{frame} 

\begin{frame}
\frametitle{Framework Process}
\begin{center}
\begin{tikzpicture}[xscale=2]
\node[shape=rectangle,fill=pantone127-4] (problem) at (2.5,5) {Problem};
\node (human) at (2.5,4) {Human};
\node[shape=rectangle,fill=pantone127-4] (model) at (2.5,3) {Model};
\node (compile) at (2.5,2) {Compile/Reformulate};
\node (solver1) at (1,1) {CP};
\node (solver2) at (2,1) {MIP};
\node (solver3) at (3,1) {SAT};
\node (solver4) at (4,1) {Other};
\node[shape=rectangle,fill=pantone157-8] (solution1) at (1,0) {Solution};
\node[shape=rectangle,fill=pantone157-8] (solution2) at (2,0) {Solution};
\node[shape=rectangle,fill=pantone157-8] (solution3) at (3,0) {Solution};
\node[shape=rectangle,fill=pantone157-8] (solution4) at (4,0) {Solution};
\draw[-] (problem) -- (human);
\draw[->] (human) -- (model);
\draw[-] (model) -- (compile);
\draw[-] (compile) -- (solver1);
\draw[-] (compile) -- (solver2);
\draw[-] (compile) -- (solver3);
\draw[-] (compile) -- (solver4);
\draw[->] (solver1) -- (solution1);
\draw[->] (solver2) -- (solution2);
\draw[->] (solver3) -- (solution3);
\draw[->] (solver4) -- (solution4);
\end{tikzpicture}
\end{center}
\end{frame}

% \begin{frame}
  % \frametitle{Do It Now!}
  % \begin{itemize}
  % \item Download and install Minizinc
  % \item \url{https://www.minizinc.org/}
  % \end{itemize}
% \end{frame}


\begin{frame}
\frametitle{Why use Puzzles as Examples?}
\begin{itemize}
\item Easy to understand the problem
\item Solvable by hand without specialized knowledge
\item Possible to compare automated to manual solving process
\end{itemize}

\begin{quote}
The puzzle, though inanimate, is presented as a solvable problem without lasting negative consequences, a very low-risk low-reward situation. By being a puzzle, the object is attempting to convince the user that it must be completed.
\end{quote}
{\tiny Source: \href{https://www.everydayrhetoric.com/2020/03/the-rhetoric-of-puzzles/}{Every Day Rhetoric}}
\end{frame}


\mode<all>{
\part{Basic Constraint Propagation}
}
\begin{frame}
\partpage
\end{frame}
\mode<all>{
\input{../sendmore/body}
}

\mode<all>{
\part{Global Constraints}
}
\begin{frame}
\partpage
\end{frame}
\mode<all>{
\input{../sudoku/body}
}

\clearpage
\mode<all>{
\part{Customizing Search}
}
\begin{frame}
\partpage
\end{frame}
\mode<all>{
\input{../nqueen/body}
}


%% \clearpage
%% \mode<all>{
%%   \part{Problem Decomposition}
%% }
%% \begin{frame}
%%   \partpage
%% \end{frame}
%% \mode<all>{
%%   \input{../party/body}
%%   }

%% \clearpage
%% \mode<all>{
%%   \part{More Global Constraints}
%% }
%% \begin{frame}
%%   \partpage
%% \end{frame}
%% \mode<all>{
%%   \input{../car/body}
%%   }

%% \clearpage
%% \mode<all>{
%%   \part{Methodology}
%% }
%% \begin{frame}
%%   \partpage
%% \end{frame}
%% \mode<all>{
%%   \input{../methodology/body}
%%   }

\clearpage
\mode<all>{
\part{What is missing?}
}
\begin{frame}
\partpage
\end{frame}

\begin{frame}
\frametitle{Many Specialized Topics}
\begin{itemize}
\item How to design efficient core engine
\item Hybrids with LP/MIP tools
\item Hybrids with SAT
\item Symmetry breaking
\item Use of MDD/BDD to encode sets of solutions
\item High level modelling tools
\item Debugging/visualization
\end{itemize}
\end{frame}

\begin{frame}
\frametitle{Reformulation}
\begin{itemize}
\item Just because the user has modelled it this way, it doesn't mean we have to solve it that way
\begin{itemize}
\item Replace some constraint(s) by other, equivalent constraints
\item Because we don't have that constraint in our system
\item For performance
\end{itemize}
\end{itemize}
\end{frame}

\begin{frame}
\frametitle{Learning}
\begin{itemize}
\item While solving the problem we can learn how to strengthen the model/search 
\begin{itemize}
\item Understand which constraints/method contribute to propagation and change schedule
\item Learn no-good constraints by explaining failure
\item Adapt search strategy based on search experience
\end{itemize}
\end{itemize}
\end{frame}

%% \begin{frame}
%% \frametitle{Refined Process}
%% \begin{center}
%% \begin{tikzpicture}
%% \node[shape=rectangle,fill=pantone127-4] (problem) at (3,7) {Problem};
%% \node (human) at (3,6) {Human};
%% \node[shape=rectangle,fill=pantone127-4] (model) at (3,5) {User Model};
%% \node (reformulation) at (3,4) {Reformulation};
%% \node[shape=rectangle,fill=pantone127-4] (implementation) at (3,3) {Implementation Model};
%% \node (solver) at (3,2) {Constraint Solver/Search};
%% \node[shape=rectangle,fill=pantone157-8] (solution) at (3,1) {Solution};
%% \draw[-] (problem) -- (human);
%% \draw[->] (human) -- (model);
%% \draw[-] (model) -- (reformulation);
%% \draw[->] (reformulation) -- (implementation);
%% \draw[-] (implementation) -- (solver);
%% \draw[->] (solver) -- (0,2) -- (0,3) -- (implementation);
%% \draw[->] (solver) -- (solution);
%% \end{tikzpicture}
%% \end{center}
%% \end{frame}

%\begin{frame}
%\frametitle{What is CP actually used for?}
%\url{http://hsimonis.wordpress.com}
%\includegraphics[width=10cm]{../introduction/images/blog}
%\end{frame}

%% \begin{frame}
%% \frametitle{Exercises}
%% \begin{itemize}
%% \item Install Constraint Solver of your Choice
%% \item MiniZinc (MiningZinc) will be used in later sessions
%% \item Run Sendmoremoney, Sudoku, Queens examples
%% \item Can you choose which propagator to use for alldifferent?
%% \item Can you use different search strategies?
%% \end{itemize}
%% \end{frame}

\begin{frame}
\frametitle{More Learning Resources}
\begin{itemize}
\item Survey of Methods, Resources, and Formats for Teaching Constraint Programming
\begin{itemize}
\item by Tejas Santanam, Helmut Simonis
\item \url{https://doi.org/10.48550/arXiv.2403.12717}
\item Based on survey of community for WTCP 2023
\item \url{https://hsimonis.github.io/WTCP2023/}
\end{itemize}
\end{itemize}
\end{frame}


%%   }

%% \clearpage
%% \mode<all>{
%%   \part{More Global Constraints}
%% }
%% \begin{frame}
%%   \partpage
%% \end{frame}
%% \mode<all>{
%%   \begin{frame}
\frametitle{Objectives}
\begin{itemize}
\item Overview of Core Constraint Programming
\item Three Main Concepts
\begin{itemize}
\item Constraint Propagation
\item Global Constraints
\item Customizing Search
\end{itemize}
%\item Get Some Experience with MiniZinc
\item Topics will be treated in more detail in later parts of the school
\item Based on Examples, not Formal Description
\end{itemize}
\end{frame}

\begin{frame}
\frametitle{Outline}
\begin{itemize}
\item Why Constraint Programming?
\item Constraint Propagation
\item Global Constraints
\item Customizing Search
%\item Choices in Modelling
%\item Decomposition
%\item More Global Constraints
%\item Methodology
%\item What is missing?
\end{itemize}
\end{frame}

% \begin{frame}
  % \frametitle{Using MiniZinc IDE}
  % \begin{itemize}
  % \item Developed in the Australian NICTA project
  % \item Maintained by Monash University
  % \item Modelling tool with multiple back-end solvers
  % \item Available from \url{https://www.minizinc.org/}
  % \end{itemize}
% \end{frame}


% \begin{frame}
% \frametitle{Examples in ECLiPSe}
% \begin{itemize}
% \item Open sourced constraint programming language
% \item Development goes back to 1985
% \item ECRC, ICL, IC-Parc, PTL, Cisco
% \item \url{https://eclipseclp.org/}
% \item Specialities
% \begin{itemize}
% \item Develop new solvers for specific domains
% \item Integration with MIP
% \end{itemize}
% \item Not included in bundled MiniZinc IDE
% \item Specialized visualization tools used here
  % \begin{itemize}
  % \item CP-Viz, Simonis et al. 2010
    % \end{itemize}
% \end{itemize}
% \end{frame}

\begin{frame}
\frametitle{Tutorial Based on ECLiPSe ELearning Course}
\begin{itemize}
\item Self-study course in constraint programming
\item Supported by Cisco Systems and Silicon Valley Community Foundation
\item Multi-media format, video lectures, slides, handout etc
\item {\small \url{https://eclipseclp.org/ELearning/index.html}}
\end{itemize}
\end{frame}

\begin{frame}
\frametitle{Also Part of CRT-AI Constraint Week}
\begin{itemize}
\item Annual one week course on CP and Optimization in Ireland
\item Part of national training program for PhD students in AI
\item \url{https://www.crt-ai.ie/}
\end{itemize}
\end{frame}



\begin{frame}
\frametitle{Constraint Programming - in a nutshell}
\begin{itemize}
\item Declarative description of problems with
\begin{itemize}
\item {\em Variables} which range over (finite) sets of values
\item {\em Constraints} over subsets of variables which restrict possible value combinations
\item A {\em solution} is a value assignment which satisfies all constraints
\end{itemize}

\item Constraint propagation/reasoning
\begin{itemize}
\item Removing inconsistent values for variables
\item Detect failure if constraint can not be satisfied
\item Interaction of constraints via shared variables
\item Incomplete
\end{itemize}

\item Search
\begin{itemize}
\item User controlled assignment of values to variables
\item Each step triggers constraint propagation 
\end{itemize}
\item Different domains require/allow different methods
\end{itemize}
\end{frame}

\begin{frame}
  \frametitle{Constraint Programming is Different}
  \begin{itemize}
  \item Declarative Programming
    \begin{itemize}
    \item Concentrate on what you want
      \item Not how to get there
      \item Program != Algorithm
      \item Program = Model
    \end{itemize}
    \item Applied to Combinatorial Problems
      \begin{itemize}
        \item No complete polynomial algorithms known (exist?)
        \item CP less ad-hoc than heuristics
        \item Models can evolve
  \end{itemize}
  \end{itemize}
  \end{frame}
    
\begin{frame}
  \frametitle{A Subtractive Process}
  \begin{textblock}{4}(8,-3)
    \includegraphics[width=4cm]{../introduction/images/stages}
  \end{textblock}
  \vfill
  \begin{quote}
    ``Oh, bosh, as Mr. Ruskin says. Sculpture, per se, is the simplest thing in the world. All you have to do is to take a big chunk of marble and a hammer and chisel, make up your mind what you are about to create and chip off all the marble you don't want.''-Paris Gaulois.
  \end{quote}
  
  {\tiny Source: \url{https://quoteinvestigator.com/2014/06/22/chip-away/}}
\end{frame}

\begin{frame}
\frametitle{Basic Process}
\begin{center}
\begin{tikzpicture}
\node[shape=rectangle,fill=pantone127-4] (problem) at (2,5) {Problem};
\node (human) at (2,4) {Human};
\node[shape=rectangle,fill=pantone127-4] (model) at (2,3) {Model};
\node (solver) at (2,2) {Constraint Solver/Search};
\node[shape=rectangle,fill=pantone157-8] (solution) at (2,1) {Solution};
\draw[-] (problem) -- (human);
\draw[->] (human) -- (model);
\draw[-] (model) -- (solver);
\draw[->] (solver) -- (solution);
\end{tikzpicture}
\end{center}
\end{frame}

\begin{frame}
\frametitle{More Realistic}
\begin{center}
\begin{tikzpicture}[xscale=4.0]
\node[shape=rectangle,fill=pantone127-4] (problem) at (2,5) {Problem};
\node (human) at (2,4) {Human};
\node[shape=rectangle,fill=pantone127-4] (model) at (2,3) {Model};
\node (solver) at (2,2) {Constraint Solver/Search};
\node[shape=rectangle,fill=pantone24-8] (hangs) at (1,1) {Hangs};
\node[shape=rectangle,fill=pantone157-8] (solution) at (2,1) {Solution};
\node[shape=rectangle,fill=pantone24-8] (wrong) at (3,1) {Wrong Solution};
\draw[-] (problem) -- (human);
\draw[->] (human) -- (model);
\draw[-] (model) -- (solver);
\draw[->] (solver) -- (hangs);
\draw[->] (solver) -- (solution);
\draw[->] (solver) -- (wrong);
\draw[->] (hangs) -- (1,4) -- (human);
\draw[->] (wrong) -- (3,4) -- (human);
\end{tikzpicture}
\end{center}
\end{frame}

\begin{frame}
\frametitle{Dual Role of Model}
\begin{itemize}
\item Allows Human to Express Problem
\begin{itemize}
\item Close to Problem Domain
\item Constraints as Abstractions
\end{itemize}
\item Allows Solver to Execute
\begin{itemize}
\item Variables as Communication Mechanism
\item Constraints as Algorithms
\end{itemize}
\end{itemize}
\end{frame}

\begin{frame}
\frametitle{Modelling Frameworks}
\begin{itemize}
\item \href{https://www.minizinc.org/}{MiniZinc} (NICTA, Monash University, Australia)
\item \href{https://github.com/eomahony/Numberjack}{NumberJack} (Insight, Ireland) 
\item \href{https://www-users.york.ac.uk/peter.nightingale/savilerow//}{EssencePrime/SavilleRow} (UK) 
\item \href{https://cpmpy.readthedocs.io/en/latest/}{CPMpy} (KU Leuven)
\item Allow use of multiple back-end solvers
\item Compile model into variants for each solver
\item A priori solver independent model(CP, MIP, SAT)
\end{itemize}
\end{frame} 

\begin{frame}
\frametitle{Framework Process}
\begin{center}
\begin{tikzpicture}[xscale=2]
\node[shape=rectangle,fill=pantone127-4] (problem) at (2.5,5) {Problem};
\node (human) at (2.5,4) {Human};
\node[shape=rectangle,fill=pantone127-4] (model) at (2.5,3) {Model};
\node (compile) at (2.5,2) {Compile/Reformulate};
\node (solver1) at (1,1) {CP};
\node (solver2) at (2,1) {MIP};
\node (solver3) at (3,1) {SAT};
\node (solver4) at (4,1) {Other};
\node[shape=rectangle,fill=pantone157-8] (solution1) at (1,0) {Solution};
\node[shape=rectangle,fill=pantone157-8] (solution2) at (2,0) {Solution};
\node[shape=rectangle,fill=pantone157-8] (solution3) at (3,0) {Solution};
\node[shape=rectangle,fill=pantone157-8] (solution4) at (4,0) {Solution};
\draw[-] (problem) -- (human);
\draw[->] (human) -- (model);
\draw[-] (model) -- (compile);
\draw[-] (compile) -- (solver1);
\draw[-] (compile) -- (solver2);
\draw[-] (compile) -- (solver3);
\draw[-] (compile) -- (solver4);
\draw[->] (solver1) -- (solution1);
\draw[->] (solver2) -- (solution2);
\draw[->] (solver3) -- (solution3);
\draw[->] (solver4) -- (solution4);
\end{tikzpicture}
\end{center}
\end{frame}

% \begin{frame}
  % \frametitle{Do It Now!}
  % \begin{itemize}
  % \item Download and install Minizinc
  % \item \url{https://www.minizinc.org/}
  % \end{itemize}
% \end{frame}


\begin{frame}
\frametitle{Why use Puzzles as Examples?}
\begin{itemize}
\item Easy to understand the problem
\item Solvable by hand without specialized knowledge
\item Possible to compare automated to manual solving process
\end{itemize}

\begin{quote}
The puzzle, though inanimate, is presented as a solvable problem without lasting negative consequences, a very low-risk low-reward situation. By being a puzzle, the object is attempting to convince the user that it must be completed.
\end{quote}
{\tiny Source: \href{https://www.everydayrhetoric.com/2020/03/the-rhetoric-of-puzzles/}{Every Day Rhetoric}}
\end{frame}


\mode<all>{
\part{Basic Constraint Propagation}
}
\begin{frame}
\partpage
\end{frame}
\mode<all>{
\input{../sendmore/body}
}

\mode<all>{
\part{Global Constraints}
}
\begin{frame}
\partpage
\end{frame}
\mode<all>{
\input{../sudoku/body}
}

\clearpage
\mode<all>{
\part{Customizing Search}
}
\begin{frame}
\partpage
\end{frame}
\mode<all>{
\input{../nqueen/body}
}


%% \clearpage
%% \mode<all>{
%%   \part{Problem Decomposition}
%% }
%% \begin{frame}
%%   \partpage
%% \end{frame}
%% \mode<all>{
%%   \input{../party/body}
%%   }

%% \clearpage
%% \mode<all>{
%%   \part{More Global Constraints}
%% }
%% \begin{frame}
%%   \partpage
%% \end{frame}
%% \mode<all>{
%%   \input{../car/body}
%%   }

%% \clearpage
%% \mode<all>{
%%   \part{Methodology}
%% }
%% \begin{frame}
%%   \partpage
%% \end{frame}
%% \mode<all>{
%%   \input{../methodology/body}
%%   }

\clearpage
\mode<all>{
\part{What is missing?}
}
\begin{frame}
\partpage
\end{frame}

\begin{frame}
\frametitle{Many Specialized Topics}
\begin{itemize}
\item How to design efficient core engine
\item Hybrids with LP/MIP tools
\item Hybrids with SAT
\item Symmetry breaking
\item Use of MDD/BDD to encode sets of solutions
\item High level modelling tools
\item Debugging/visualization
\end{itemize}
\end{frame}

\begin{frame}
\frametitle{Reformulation}
\begin{itemize}
\item Just because the user has modelled it this way, it doesn't mean we have to solve it that way
\begin{itemize}
\item Replace some constraint(s) by other, equivalent constraints
\item Because we don't have that constraint in our system
\item For performance
\end{itemize}
\end{itemize}
\end{frame}

\begin{frame}
\frametitle{Learning}
\begin{itemize}
\item While solving the problem we can learn how to strengthen the model/search 
\begin{itemize}
\item Understand which constraints/method contribute to propagation and change schedule
\item Learn no-good constraints by explaining failure
\item Adapt search strategy based on search experience
\end{itemize}
\end{itemize}
\end{frame}

%% \begin{frame}
%% \frametitle{Refined Process}
%% \begin{center}
%% \begin{tikzpicture}
%% \node[shape=rectangle,fill=pantone127-4] (problem) at (3,7) {Problem};
%% \node (human) at (3,6) {Human};
%% \node[shape=rectangle,fill=pantone127-4] (model) at (3,5) {User Model};
%% \node (reformulation) at (3,4) {Reformulation};
%% \node[shape=rectangle,fill=pantone127-4] (implementation) at (3,3) {Implementation Model};
%% \node (solver) at (3,2) {Constraint Solver/Search};
%% \node[shape=rectangle,fill=pantone157-8] (solution) at (3,1) {Solution};
%% \draw[-] (problem) -- (human);
%% \draw[->] (human) -- (model);
%% \draw[-] (model) -- (reformulation);
%% \draw[->] (reformulation) -- (implementation);
%% \draw[-] (implementation) -- (solver);
%% \draw[->] (solver) -- (0,2) -- (0,3) -- (implementation);
%% \draw[->] (solver) -- (solution);
%% \end{tikzpicture}
%% \end{center}
%% \end{frame}

%\begin{frame}
%\frametitle{What is CP actually used for?}
%\url{http://hsimonis.wordpress.com}
%\includegraphics[width=10cm]{../introduction/images/blog}
%\end{frame}

%% \begin{frame}
%% \frametitle{Exercises}
%% \begin{itemize}
%% \item Install Constraint Solver of your Choice
%% \item MiniZinc (MiningZinc) will be used in later sessions
%% \item Run Sendmoremoney, Sudoku, Queens examples
%% \item Can you choose which propagator to use for alldifferent?
%% \item Can you use different search strategies?
%% \end{itemize}
%% \end{frame}

\begin{frame}
\frametitle{More Learning Resources}
\begin{itemize}
\item Survey of Methods, Resources, and Formats for Teaching Constraint Programming
\begin{itemize}
\item by Tejas Santanam, Helmut Simonis
\item \url{https://doi.org/10.48550/arXiv.2403.12717}
\item Based on survey of community for WTCP 2023
\item \url{https://hsimonis.github.io/WTCP2023/}
\end{itemize}
\end{itemize}
\end{frame}


%%   }

%% \clearpage
%% \mode<all>{
%%   \part{Methodology}
%% }
%% \begin{frame}
%%   \partpage
%% \end{frame}
%% \mode<all>{
%%   \begin{frame}
\frametitle{Objectives}
\begin{itemize}
\item Overview of Core Constraint Programming
\item Three Main Concepts
\begin{itemize}
\item Constraint Propagation
\item Global Constraints
\item Customizing Search
\end{itemize}
%\item Get Some Experience with MiniZinc
\item Topics will be treated in more detail in later parts of the school
\item Based on Examples, not Formal Description
\end{itemize}
\end{frame}

\begin{frame}
\frametitle{Outline}
\begin{itemize}
\item Why Constraint Programming?
\item Constraint Propagation
\item Global Constraints
\item Customizing Search
%\item Choices in Modelling
%\item Decomposition
%\item More Global Constraints
%\item Methodology
%\item What is missing?
\end{itemize}
\end{frame}

% \begin{frame}
  % \frametitle{Using MiniZinc IDE}
  % \begin{itemize}
  % \item Developed in the Australian NICTA project
  % \item Maintained by Monash University
  % \item Modelling tool with multiple back-end solvers
  % \item Available from \url{https://www.minizinc.org/}
  % \end{itemize}
% \end{frame}


% \begin{frame}
% \frametitle{Examples in ECLiPSe}
% \begin{itemize}
% \item Open sourced constraint programming language
% \item Development goes back to 1985
% \item ECRC, ICL, IC-Parc, PTL, Cisco
% \item \url{https://eclipseclp.org/}
% \item Specialities
% \begin{itemize}
% \item Develop new solvers for specific domains
% \item Integration with MIP
% \end{itemize}
% \item Not included in bundled MiniZinc IDE
% \item Specialized visualization tools used here
  % \begin{itemize}
  % \item CP-Viz, Simonis et al. 2010
    % \end{itemize}
% \end{itemize}
% \end{frame}

\begin{frame}
\frametitle{Tutorial Based on ECLiPSe ELearning Course}
\begin{itemize}
\item Self-study course in constraint programming
\item Supported by Cisco Systems and Silicon Valley Community Foundation
\item Multi-media format, video lectures, slides, handout etc
\item {\small \url{https://eclipseclp.org/ELearning/index.html}}
\end{itemize}
\end{frame}

\begin{frame}
\frametitle{Also Part of CRT-AI Constraint Week}
\begin{itemize}
\item Annual one week course on CP and Optimization in Ireland
\item Part of national training program for PhD students in AI
\item \url{https://www.crt-ai.ie/}
\end{itemize}
\end{frame}



\begin{frame}
\frametitle{Constraint Programming - in a nutshell}
\begin{itemize}
\item Declarative description of problems with
\begin{itemize}
\item {\em Variables} which range over (finite) sets of values
\item {\em Constraints} over subsets of variables which restrict possible value combinations
\item A {\em solution} is a value assignment which satisfies all constraints
\end{itemize}

\item Constraint propagation/reasoning
\begin{itemize}
\item Removing inconsistent values for variables
\item Detect failure if constraint can not be satisfied
\item Interaction of constraints via shared variables
\item Incomplete
\end{itemize}

\item Search
\begin{itemize}
\item User controlled assignment of values to variables
\item Each step triggers constraint propagation 
\end{itemize}
\item Different domains require/allow different methods
\end{itemize}
\end{frame}

\begin{frame}
  \frametitle{Constraint Programming is Different}
  \begin{itemize}
  \item Declarative Programming
    \begin{itemize}
    \item Concentrate on what you want
      \item Not how to get there
      \item Program != Algorithm
      \item Program = Model
    \end{itemize}
    \item Applied to Combinatorial Problems
      \begin{itemize}
        \item No complete polynomial algorithms known (exist?)
        \item CP less ad-hoc than heuristics
        \item Models can evolve
  \end{itemize}
  \end{itemize}
  \end{frame}
    
\begin{frame}
  \frametitle{A Subtractive Process}
  \begin{textblock}{4}(8,-3)
    \includegraphics[width=4cm]{../introduction/images/stages}
  \end{textblock}
  \vfill
  \begin{quote}
    ``Oh, bosh, as Mr. Ruskin says. Sculpture, per se, is the simplest thing in the world. All you have to do is to take a big chunk of marble and a hammer and chisel, make up your mind what you are about to create and chip off all the marble you don't want.''-Paris Gaulois.
  \end{quote}
  
  {\tiny Source: \url{https://quoteinvestigator.com/2014/06/22/chip-away/}}
\end{frame}

\begin{frame}
\frametitle{Basic Process}
\begin{center}
\begin{tikzpicture}
\node[shape=rectangle,fill=pantone127-4] (problem) at (2,5) {Problem};
\node (human) at (2,4) {Human};
\node[shape=rectangle,fill=pantone127-4] (model) at (2,3) {Model};
\node (solver) at (2,2) {Constraint Solver/Search};
\node[shape=rectangle,fill=pantone157-8] (solution) at (2,1) {Solution};
\draw[-] (problem) -- (human);
\draw[->] (human) -- (model);
\draw[-] (model) -- (solver);
\draw[->] (solver) -- (solution);
\end{tikzpicture}
\end{center}
\end{frame}

\begin{frame}
\frametitle{More Realistic}
\begin{center}
\begin{tikzpicture}[xscale=4.0]
\node[shape=rectangle,fill=pantone127-4] (problem) at (2,5) {Problem};
\node (human) at (2,4) {Human};
\node[shape=rectangle,fill=pantone127-4] (model) at (2,3) {Model};
\node (solver) at (2,2) {Constraint Solver/Search};
\node[shape=rectangle,fill=pantone24-8] (hangs) at (1,1) {Hangs};
\node[shape=rectangle,fill=pantone157-8] (solution) at (2,1) {Solution};
\node[shape=rectangle,fill=pantone24-8] (wrong) at (3,1) {Wrong Solution};
\draw[-] (problem) -- (human);
\draw[->] (human) -- (model);
\draw[-] (model) -- (solver);
\draw[->] (solver) -- (hangs);
\draw[->] (solver) -- (solution);
\draw[->] (solver) -- (wrong);
\draw[->] (hangs) -- (1,4) -- (human);
\draw[->] (wrong) -- (3,4) -- (human);
\end{tikzpicture}
\end{center}
\end{frame}

\begin{frame}
\frametitle{Dual Role of Model}
\begin{itemize}
\item Allows Human to Express Problem
\begin{itemize}
\item Close to Problem Domain
\item Constraints as Abstractions
\end{itemize}
\item Allows Solver to Execute
\begin{itemize}
\item Variables as Communication Mechanism
\item Constraints as Algorithms
\end{itemize}
\end{itemize}
\end{frame}

\begin{frame}
\frametitle{Modelling Frameworks}
\begin{itemize}
\item \href{https://www.minizinc.org/}{MiniZinc} (NICTA, Monash University, Australia)
\item \href{https://github.com/eomahony/Numberjack}{NumberJack} (Insight, Ireland) 
\item \href{https://www-users.york.ac.uk/peter.nightingale/savilerow//}{EssencePrime/SavilleRow} (UK) 
\item \href{https://cpmpy.readthedocs.io/en/latest/}{CPMpy} (KU Leuven)
\item Allow use of multiple back-end solvers
\item Compile model into variants for each solver
\item A priori solver independent model(CP, MIP, SAT)
\end{itemize}
\end{frame} 

\begin{frame}
\frametitle{Framework Process}
\begin{center}
\begin{tikzpicture}[xscale=2]
\node[shape=rectangle,fill=pantone127-4] (problem) at (2.5,5) {Problem};
\node (human) at (2.5,4) {Human};
\node[shape=rectangle,fill=pantone127-4] (model) at (2.5,3) {Model};
\node (compile) at (2.5,2) {Compile/Reformulate};
\node (solver1) at (1,1) {CP};
\node (solver2) at (2,1) {MIP};
\node (solver3) at (3,1) {SAT};
\node (solver4) at (4,1) {Other};
\node[shape=rectangle,fill=pantone157-8] (solution1) at (1,0) {Solution};
\node[shape=rectangle,fill=pantone157-8] (solution2) at (2,0) {Solution};
\node[shape=rectangle,fill=pantone157-8] (solution3) at (3,0) {Solution};
\node[shape=rectangle,fill=pantone157-8] (solution4) at (4,0) {Solution};
\draw[-] (problem) -- (human);
\draw[->] (human) -- (model);
\draw[-] (model) -- (compile);
\draw[-] (compile) -- (solver1);
\draw[-] (compile) -- (solver2);
\draw[-] (compile) -- (solver3);
\draw[-] (compile) -- (solver4);
\draw[->] (solver1) -- (solution1);
\draw[->] (solver2) -- (solution2);
\draw[->] (solver3) -- (solution3);
\draw[->] (solver4) -- (solution4);
\end{tikzpicture}
\end{center}
\end{frame}

% \begin{frame}
  % \frametitle{Do It Now!}
  % \begin{itemize}
  % \item Download and install Minizinc
  % \item \url{https://www.minizinc.org/}
  % \end{itemize}
% \end{frame}


\begin{frame}
\frametitle{Why use Puzzles as Examples?}
\begin{itemize}
\item Easy to understand the problem
\item Solvable by hand without specialized knowledge
\item Possible to compare automated to manual solving process
\end{itemize}

\begin{quote}
The puzzle, though inanimate, is presented as a solvable problem without lasting negative consequences, a very low-risk low-reward situation. By being a puzzle, the object is attempting to convince the user that it must be completed.
\end{quote}
{\tiny Source: \href{https://www.everydayrhetoric.com/2020/03/the-rhetoric-of-puzzles/}{Every Day Rhetoric}}
\end{frame}


\mode<all>{
\part{Basic Constraint Propagation}
}
\begin{frame}
\partpage
\end{frame}
\mode<all>{
\input{../sendmore/body}
}

\mode<all>{
\part{Global Constraints}
}
\begin{frame}
\partpage
\end{frame}
\mode<all>{
\input{../sudoku/body}
}

\clearpage
\mode<all>{
\part{Customizing Search}
}
\begin{frame}
\partpage
\end{frame}
\mode<all>{
\input{../nqueen/body}
}


%% \clearpage
%% \mode<all>{
%%   \part{Problem Decomposition}
%% }
%% \begin{frame}
%%   \partpage
%% \end{frame}
%% \mode<all>{
%%   \input{../party/body}
%%   }

%% \clearpage
%% \mode<all>{
%%   \part{More Global Constraints}
%% }
%% \begin{frame}
%%   \partpage
%% \end{frame}
%% \mode<all>{
%%   \input{../car/body}
%%   }

%% \clearpage
%% \mode<all>{
%%   \part{Methodology}
%% }
%% \begin{frame}
%%   \partpage
%% \end{frame}
%% \mode<all>{
%%   \input{../methodology/body}
%%   }

\clearpage
\mode<all>{
\part{What is missing?}
}
\begin{frame}
\partpage
\end{frame}

\begin{frame}
\frametitle{Many Specialized Topics}
\begin{itemize}
\item How to design efficient core engine
\item Hybrids with LP/MIP tools
\item Hybrids with SAT
\item Symmetry breaking
\item Use of MDD/BDD to encode sets of solutions
\item High level modelling tools
\item Debugging/visualization
\end{itemize}
\end{frame}

\begin{frame}
\frametitle{Reformulation}
\begin{itemize}
\item Just because the user has modelled it this way, it doesn't mean we have to solve it that way
\begin{itemize}
\item Replace some constraint(s) by other, equivalent constraints
\item Because we don't have that constraint in our system
\item For performance
\end{itemize}
\end{itemize}
\end{frame}

\begin{frame}
\frametitle{Learning}
\begin{itemize}
\item While solving the problem we can learn how to strengthen the model/search 
\begin{itemize}
\item Understand which constraints/method contribute to propagation and change schedule
\item Learn no-good constraints by explaining failure
\item Adapt search strategy based on search experience
\end{itemize}
\end{itemize}
\end{frame}

%% \begin{frame}
%% \frametitle{Refined Process}
%% \begin{center}
%% \begin{tikzpicture}
%% \node[shape=rectangle,fill=pantone127-4] (problem) at (3,7) {Problem};
%% \node (human) at (3,6) {Human};
%% \node[shape=rectangle,fill=pantone127-4] (model) at (3,5) {User Model};
%% \node (reformulation) at (3,4) {Reformulation};
%% \node[shape=rectangle,fill=pantone127-4] (implementation) at (3,3) {Implementation Model};
%% \node (solver) at (3,2) {Constraint Solver/Search};
%% \node[shape=rectangle,fill=pantone157-8] (solution) at (3,1) {Solution};
%% \draw[-] (problem) -- (human);
%% \draw[->] (human) -- (model);
%% \draw[-] (model) -- (reformulation);
%% \draw[->] (reformulation) -- (implementation);
%% \draw[-] (implementation) -- (solver);
%% \draw[->] (solver) -- (0,2) -- (0,3) -- (implementation);
%% \draw[->] (solver) -- (solution);
%% \end{tikzpicture}
%% \end{center}
%% \end{frame}

%\begin{frame}
%\frametitle{What is CP actually used for?}
%\url{http://hsimonis.wordpress.com}
%\includegraphics[width=10cm]{../introduction/images/blog}
%\end{frame}

%% \begin{frame}
%% \frametitle{Exercises}
%% \begin{itemize}
%% \item Install Constraint Solver of your Choice
%% \item MiniZinc (MiningZinc) will be used in later sessions
%% \item Run Sendmoremoney, Sudoku, Queens examples
%% \item Can you choose which propagator to use for alldifferent?
%% \item Can you use different search strategies?
%% \end{itemize}
%% \end{frame}

\begin{frame}
\frametitle{More Learning Resources}
\begin{itemize}
\item Survey of Methods, Resources, and Formats for Teaching Constraint Programming
\begin{itemize}
\item by Tejas Santanam, Helmut Simonis
\item \url{https://doi.org/10.48550/arXiv.2403.12717}
\item Based on survey of community for WTCP 2023
\item \url{https://hsimonis.github.io/WTCP2023/}
\end{itemize}
\end{itemize}
\end{frame}


%%   }

\clearpage
\mode<all>{
\part{What is missing?}
}
\begin{frame}
\partpage
\end{frame}

\begin{frame}
\frametitle{Many Specialized Topics}
\begin{itemize}
\item How to design efficient core engine
\item Hybrids with LP/MIP tools
\item Hybrids with SAT
\item Symmetry breaking
\item Use of MDD/BDD to encode sets of solutions
\item High level modelling tools
\item Debugging/visualization
\end{itemize}
\end{frame}

\begin{frame}
\frametitle{Reformulation}
\begin{itemize}
\item Just because the user has modelled it this way, it doesn't mean we have to solve it that way
\begin{itemize}
\item Replace some constraint(s) by other, equivalent constraints
\item Because we don't have that constraint in our system
\item For performance
\end{itemize}
\end{itemize}
\end{frame}

\begin{frame}
\frametitle{Learning}
\begin{itemize}
\item While solving the problem we can learn how to strengthen the model/search 
\begin{itemize}
\item Understand which constraints/method contribute to propagation and change schedule
\item Learn no-good constraints by explaining failure
\item Adapt search strategy based on search experience
\end{itemize}
\end{itemize}
\end{frame}

%% \begin{frame}
%% \frametitle{Refined Process}
%% \begin{center}
%% \begin{tikzpicture}
%% \node[shape=rectangle,fill=pantone127-4] (problem) at (3,7) {Problem};
%% \node (human) at (3,6) {Human};
%% \node[shape=rectangle,fill=pantone127-4] (model) at (3,5) {User Model};
%% \node (reformulation) at (3,4) {Reformulation};
%% \node[shape=rectangle,fill=pantone127-4] (implementation) at (3,3) {Implementation Model};
%% \node (solver) at (3,2) {Constraint Solver/Search};
%% \node[shape=rectangle,fill=pantone157-8] (solution) at (3,1) {Solution};
%% \draw[-] (problem) -- (human);
%% \draw[->] (human) -- (model);
%% \draw[-] (model) -- (reformulation);
%% \draw[->] (reformulation) -- (implementation);
%% \draw[-] (implementation) -- (solver);
%% \draw[->] (solver) -- (0,2) -- (0,3) -- (implementation);
%% \draw[->] (solver) -- (solution);
%% \end{tikzpicture}
%% \end{center}
%% \end{frame}

%\begin{frame}
%\frametitle{What is CP actually used for?}
%\url{http://hsimonis.wordpress.com}
%\includegraphics[width=10cm]{../introduction/images/blog}
%\end{frame}

%% \begin{frame}
%% \frametitle{Exercises}
%% \begin{itemize}
%% \item Install Constraint Solver of your Choice
%% \item MiniZinc (MiningZinc) will be used in later sessions
%% \item Run Sendmoremoney, Sudoku, Queens examples
%% \item Can you choose which propagator to use for alldifferent?
%% \item Can you use different search strategies?
%% \end{itemize}
%% \end{frame}

\begin{frame}
\frametitle{More Learning Resources}
\begin{itemize}
\item Survey of Methods, Resources, and Formats for Teaching Constraint Programming
\begin{itemize}
\item by Tejas Santanam, Helmut Simonis
\item \url{https://doi.org/10.48550/arXiv.2403.12717}
\item Based on survey of community for WTCP 2023
\item \url{https://hsimonis.github.io/WTCP2023/}
\end{itemize}
\end{itemize}
\end{frame}


}

\mode<all>{
\part{Global Constraints}
}
\begin{frame}
\partpage
\end{frame}
\mode<all>{
\begin{frame}
\frametitle{Objectives}
\begin{itemize}
\item Overview of Core Constraint Programming
\item Three Main Concepts
\begin{itemize}
\item Constraint Propagation
\item Global Constraints
\item Customizing Search
\end{itemize}
%\item Get Some Experience with MiniZinc
\item Topics will be treated in more detail in later parts of the school
\item Based on Examples, not Formal Description
\end{itemize}
\end{frame}

\begin{frame}
\frametitle{Outline}
\begin{itemize}
\item Why Constraint Programming?
\item Constraint Propagation
\item Global Constraints
\item Customizing Search
%\item Choices in Modelling
%\item Decomposition
%\item More Global Constraints
%\item Methodology
%\item What is missing?
\end{itemize}
\end{frame}

% \begin{frame}
  % \frametitle{Using MiniZinc IDE}
  % \begin{itemize}
  % \item Developed in the Australian NICTA project
  % \item Maintained by Monash University
  % \item Modelling tool with multiple back-end solvers
  % \item Available from \url{https://www.minizinc.org/}
  % \end{itemize}
% \end{frame}


% \begin{frame}
% \frametitle{Examples in ECLiPSe}
% \begin{itemize}
% \item Open sourced constraint programming language
% \item Development goes back to 1985
% \item ECRC, ICL, IC-Parc, PTL, Cisco
% \item \url{https://eclipseclp.org/}
% \item Specialities
% \begin{itemize}
% \item Develop new solvers for specific domains
% \item Integration with MIP
% \end{itemize}
% \item Not included in bundled MiniZinc IDE
% \item Specialized visualization tools used here
  % \begin{itemize}
  % \item CP-Viz, Simonis et al. 2010
    % \end{itemize}
% \end{itemize}
% \end{frame}

\begin{frame}
\frametitle{Tutorial Based on ECLiPSe ELearning Course}
\begin{itemize}
\item Self-study course in constraint programming
\item Supported by Cisco Systems and Silicon Valley Community Foundation
\item Multi-media format, video lectures, slides, handout etc
\item {\small \url{https://eclipseclp.org/ELearning/index.html}}
\end{itemize}
\end{frame}

\begin{frame}
\frametitle{Also Part of CRT-AI Constraint Week}
\begin{itemize}
\item Annual one week course on CP and Optimization in Ireland
\item Part of national training program for PhD students in AI
\item \url{https://www.crt-ai.ie/}
\end{itemize}
\end{frame}



\begin{frame}
\frametitle{Constraint Programming - in a nutshell}
\begin{itemize}
\item Declarative description of problems with
\begin{itemize}
\item {\em Variables} which range over (finite) sets of values
\item {\em Constraints} over subsets of variables which restrict possible value combinations
\item A {\em solution} is a value assignment which satisfies all constraints
\end{itemize}

\item Constraint propagation/reasoning
\begin{itemize}
\item Removing inconsistent values for variables
\item Detect failure if constraint can not be satisfied
\item Interaction of constraints via shared variables
\item Incomplete
\end{itemize}

\item Search
\begin{itemize}
\item User controlled assignment of values to variables
\item Each step triggers constraint propagation 
\end{itemize}
\item Different domains require/allow different methods
\end{itemize}
\end{frame}

\begin{frame}
  \frametitle{Constraint Programming is Different}
  \begin{itemize}
  \item Declarative Programming
    \begin{itemize}
    \item Concentrate on what you want
      \item Not how to get there
      \item Program != Algorithm
      \item Program = Model
    \end{itemize}
    \item Applied to Combinatorial Problems
      \begin{itemize}
        \item No complete polynomial algorithms known (exist?)
        \item CP less ad-hoc than heuristics
        \item Models can evolve
  \end{itemize}
  \end{itemize}
  \end{frame}
    
\begin{frame}
  \frametitle{A Subtractive Process}
  \begin{textblock}{4}(8,-3)
    \includegraphics[width=4cm]{../introduction/images/stages}
  \end{textblock}
  \vfill
  \begin{quote}
    ``Oh, bosh, as Mr. Ruskin says. Sculpture, per se, is the simplest thing in the world. All you have to do is to take a big chunk of marble and a hammer and chisel, make up your mind what you are about to create and chip off all the marble you don't want.''-Paris Gaulois.
  \end{quote}
  
  {\tiny Source: \url{https://quoteinvestigator.com/2014/06/22/chip-away/}}
\end{frame}

\begin{frame}
\frametitle{Basic Process}
\begin{center}
\begin{tikzpicture}
\node[shape=rectangle,fill=pantone127-4] (problem) at (2,5) {Problem};
\node (human) at (2,4) {Human};
\node[shape=rectangle,fill=pantone127-4] (model) at (2,3) {Model};
\node (solver) at (2,2) {Constraint Solver/Search};
\node[shape=rectangle,fill=pantone157-8] (solution) at (2,1) {Solution};
\draw[-] (problem) -- (human);
\draw[->] (human) -- (model);
\draw[-] (model) -- (solver);
\draw[->] (solver) -- (solution);
\end{tikzpicture}
\end{center}
\end{frame}

\begin{frame}
\frametitle{More Realistic}
\begin{center}
\begin{tikzpicture}[xscale=4.0]
\node[shape=rectangle,fill=pantone127-4] (problem) at (2,5) {Problem};
\node (human) at (2,4) {Human};
\node[shape=rectangle,fill=pantone127-4] (model) at (2,3) {Model};
\node (solver) at (2,2) {Constraint Solver/Search};
\node[shape=rectangle,fill=pantone24-8] (hangs) at (1,1) {Hangs};
\node[shape=rectangle,fill=pantone157-8] (solution) at (2,1) {Solution};
\node[shape=rectangle,fill=pantone24-8] (wrong) at (3,1) {Wrong Solution};
\draw[-] (problem) -- (human);
\draw[->] (human) -- (model);
\draw[-] (model) -- (solver);
\draw[->] (solver) -- (hangs);
\draw[->] (solver) -- (solution);
\draw[->] (solver) -- (wrong);
\draw[->] (hangs) -- (1,4) -- (human);
\draw[->] (wrong) -- (3,4) -- (human);
\end{tikzpicture}
\end{center}
\end{frame}

\begin{frame}
\frametitle{Dual Role of Model}
\begin{itemize}
\item Allows Human to Express Problem
\begin{itemize}
\item Close to Problem Domain
\item Constraints as Abstractions
\end{itemize}
\item Allows Solver to Execute
\begin{itemize}
\item Variables as Communication Mechanism
\item Constraints as Algorithms
\end{itemize}
\end{itemize}
\end{frame}

\begin{frame}
\frametitle{Modelling Frameworks}
\begin{itemize}
\item \href{https://www.minizinc.org/}{MiniZinc} (NICTA, Monash University, Australia)
\item \href{https://github.com/eomahony/Numberjack}{NumberJack} (Insight, Ireland) 
\item \href{https://www-users.york.ac.uk/peter.nightingale/savilerow//}{EssencePrime/SavilleRow} (UK) 
\item \href{https://cpmpy.readthedocs.io/en/latest/}{CPMpy} (KU Leuven)
\item Allow use of multiple back-end solvers
\item Compile model into variants for each solver
\item A priori solver independent model(CP, MIP, SAT)
\end{itemize}
\end{frame} 

\begin{frame}
\frametitle{Framework Process}
\begin{center}
\begin{tikzpicture}[xscale=2]
\node[shape=rectangle,fill=pantone127-4] (problem) at (2.5,5) {Problem};
\node (human) at (2.5,4) {Human};
\node[shape=rectangle,fill=pantone127-4] (model) at (2.5,3) {Model};
\node (compile) at (2.5,2) {Compile/Reformulate};
\node (solver1) at (1,1) {CP};
\node (solver2) at (2,1) {MIP};
\node (solver3) at (3,1) {SAT};
\node (solver4) at (4,1) {Other};
\node[shape=rectangle,fill=pantone157-8] (solution1) at (1,0) {Solution};
\node[shape=rectangle,fill=pantone157-8] (solution2) at (2,0) {Solution};
\node[shape=rectangle,fill=pantone157-8] (solution3) at (3,0) {Solution};
\node[shape=rectangle,fill=pantone157-8] (solution4) at (4,0) {Solution};
\draw[-] (problem) -- (human);
\draw[->] (human) -- (model);
\draw[-] (model) -- (compile);
\draw[-] (compile) -- (solver1);
\draw[-] (compile) -- (solver2);
\draw[-] (compile) -- (solver3);
\draw[-] (compile) -- (solver4);
\draw[->] (solver1) -- (solution1);
\draw[->] (solver2) -- (solution2);
\draw[->] (solver3) -- (solution3);
\draw[->] (solver4) -- (solution4);
\end{tikzpicture}
\end{center}
\end{frame}

% \begin{frame}
  % \frametitle{Do It Now!}
  % \begin{itemize}
  % \item Download and install Minizinc
  % \item \url{https://www.minizinc.org/}
  % \end{itemize}
% \end{frame}


\begin{frame}
\frametitle{Why use Puzzles as Examples?}
\begin{itemize}
\item Easy to understand the problem
\item Solvable by hand without specialized knowledge
\item Possible to compare automated to manual solving process
\end{itemize}

\begin{quote}
The puzzle, though inanimate, is presented as a solvable problem without lasting negative consequences, a very low-risk low-reward situation. By being a puzzle, the object is attempting to convince the user that it must be completed.
\end{quote}
{\tiny Source: \href{https://www.everydayrhetoric.com/2020/03/the-rhetoric-of-puzzles/}{Every Day Rhetoric}}
\end{frame}


\mode<all>{
\part{Basic Constraint Propagation}
}
\begin{frame}
\partpage
\end{frame}
\mode<all>{
\begin{frame}
\frametitle{Objectives}
\begin{itemize}
\item Overview of Core Constraint Programming
\item Three Main Concepts
\begin{itemize}
\item Constraint Propagation
\item Global Constraints
\item Customizing Search
\end{itemize}
%\item Get Some Experience with MiniZinc
\item Topics will be treated in more detail in later parts of the school
\item Based on Examples, not Formal Description
\end{itemize}
\end{frame}

\begin{frame}
\frametitle{Outline}
\begin{itemize}
\item Why Constraint Programming?
\item Constraint Propagation
\item Global Constraints
\item Customizing Search
%\item Choices in Modelling
%\item Decomposition
%\item More Global Constraints
%\item Methodology
%\item What is missing?
\end{itemize}
\end{frame}

% \begin{frame}
  % \frametitle{Using MiniZinc IDE}
  % \begin{itemize}
  % \item Developed in the Australian NICTA project
  % \item Maintained by Monash University
  % \item Modelling tool with multiple back-end solvers
  % \item Available from \url{https://www.minizinc.org/}
  % \end{itemize}
% \end{frame}


% \begin{frame}
% \frametitle{Examples in ECLiPSe}
% \begin{itemize}
% \item Open sourced constraint programming language
% \item Development goes back to 1985
% \item ECRC, ICL, IC-Parc, PTL, Cisco
% \item \url{https://eclipseclp.org/}
% \item Specialities
% \begin{itemize}
% \item Develop new solvers for specific domains
% \item Integration with MIP
% \end{itemize}
% \item Not included in bundled MiniZinc IDE
% \item Specialized visualization tools used here
  % \begin{itemize}
  % \item CP-Viz, Simonis et al. 2010
    % \end{itemize}
% \end{itemize}
% \end{frame}

\begin{frame}
\frametitle{Tutorial Based on ECLiPSe ELearning Course}
\begin{itemize}
\item Self-study course in constraint programming
\item Supported by Cisco Systems and Silicon Valley Community Foundation
\item Multi-media format, video lectures, slides, handout etc
\item {\small \url{https://eclipseclp.org/ELearning/index.html}}
\end{itemize}
\end{frame}

\begin{frame}
\frametitle{Also Part of CRT-AI Constraint Week}
\begin{itemize}
\item Annual one week course on CP and Optimization in Ireland
\item Part of national training program for PhD students in AI
\item \url{https://www.crt-ai.ie/}
\end{itemize}
\end{frame}



\begin{frame}
\frametitle{Constraint Programming - in a nutshell}
\begin{itemize}
\item Declarative description of problems with
\begin{itemize}
\item {\em Variables} which range over (finite) sets of values
\item {\em Constraints} over subsets of variables which restrict possible value combinations
\item A {\em solution} is a value assignment which satisfies all constraints
\end{itemize}

\item Constraint propagation/reasoning
\begin{itemize}
\item Removing inconsistent values for variables
\item Detect failure if constraint can not be satisfied
\item Interaction of constraints via shared variables
\item Incomplete
\end{itemize}

\item Search
\begin{itemize}
\item User controlled assignment of values to variables
\item Each step triggers constraint propagation 
\end{itemize}
\item Different domains require/allow different methods
\end{itemize}
\end{frame}

\begin{frame}
  \frametitle{Constraint Programming is Different}
  \begin{itemize}
  \item Declarative Programming
    \begin{itemize}
    \item Concentrate on what you want
      \item Not how to get there
      \item Program != Algorithm
      \item Program = Model
    \end{itemize}
    \item Applied to Combinatorial Problems
      \begin{itemize}
        \item No complete polynomial algorithms known (exist?)
        \item CP less ad-hoc than heuristics
        \item Models can evolve
  \end{itemize}
  \end{itemize}
  \end{frame}
    
\begin{frame}
  \frametitle{A Subtractive Process}
  \begin{textblock}{4}(8,-3)
    \includegraphics[width=4cm]{../introduction/images/stages}
  \end{textblock}
  \vfill
  \begin{quote}
    ``Oh, bosh, as Mr. Ruskin says. Sculpture, per se, is the simplest thing in the world. All you have to do is to take a big chunk of marble and a hammer and chisel, make up your mind what you are about to create and chip off all the marble you don't want.''-Paris Gaulois.
  \end{quote}
  
  {\tiny Source: \url{https://quoteinvestigator.com/2014/06/22/chip-away/}}
\end{frame}

\begin{frame}
\frametitle{Basic Process}
\begin{center}
\begin{tikzpicture}
\node[shape=rectangle,fill=pantone127-4] (problem) at (2,5) {Problem};
\node (human) at (2,4) {Human};
\node[shape=rectangle,fill=pantone127-4] (model) at (2,3) {Model};
\node (solver) at (2,2) {Constraint Solver/Search};
\node[shape=rectangle,fill=pantone157-8] (solution) at (2,1) {Solution};
\draw[-] (problem) -- (human);
\draw[->] (human) -- (model);
\draw[-] (model) -- (solver);
\draw[->] (solver) -- (solution);
\end{tikzpicture}
\end{center}
\end{frame}

\begin{frame}
\frametitle{More Realistic}
\begin{center}
\begin{tikzpicture}[xscale=4.0]
\node[shape=rectangle,fill=pantone127-4] (problem) at (2,5) {Problem};
\node (human) at (2,4) {Human};
\node[shape=rectangle,fill=pantone127-4] (model) at (2,3) {Model};
\node (solver) at (2,2) {Constraint Solver/Search};
\node[shape=rectangle,fill=pantone24-8] (hangs) at (1,1) {Hangs};
\node[shape=rectangle,fill=pantone157-8] (solution) at (2,1) {Solution};
\node[shape=rectangle,fill=pantone24-8] (wrong) at (3,1) {Wrong Solution};
\draw[-] (problem) -- (human);
\draw[->] (human) -- (model);
\draw[-] (model) -- (solver);
\draw[->] (solver) -- (hangs);
\draw[->] (solver) -- (solution);
\draw[->] (solver) -- (wrong);
\draw[->] (hangs) -- (1,4) -- (human);
\draw[->] (wrong) -- (3,4) -- (human);
\end{tikzpicture}
\end{center}
\end{frame}

\begin{frame}
\frametitle{Dual Role of Model}
\begin{itemize}
\item Allows Human to Express Problem
\begin{itemize}
\item Close to Problem Domain
\item Constraints as Abstractions
\end{itemize}
\item Allows Solver to Execute
\begin{itemize}
\item Variables as Communication Mechanism
\item Constraints as Algorithms
\end{itemize}
\end{itemize}
\end{frame}

\begin{frame}
\frametitle{Modelling Frameworks}
\begin{itemize}
\item \href{https://www.minizinc.org/}{MiniZinc} (NICTA, Monash University, Australia)
\item \href{https://github.com/eomahony/Numberjack}{NumberJack} (Insight, Ireland) 
\item \href{https://www-users.york.ac.uk/peter.nightingale/savilerow//}{EssencePrime/SavilleRow} (UK) 
\item \href{https://cpmpy.readthedocs.io/en/latest/}{CPMpy} (KU Leuven)
\item Allow use of multiple back-end solvers
\item Compile model into variants for each solver
\item A priori solver independent model(CP, MIP, SAT)
\end{itemize}
\end{frame} 

\begin{frame}
\frametitle{Framework Process}
\begin{center}
\begin{tikzpicture}[xscale=2]
\node[shape=rectangle,fill=pantone127-4] (problem) at (2.5,5) {Problem};
\node (human) at (2.5,4) {Human};
\node[shape=rectangle,fill=pantone127-4] (model) at (2.5,3) {Model};
\node (compile) at (2.5,2) {Compile/Reformulate};
\node (solver1) at (1,1) {CP};
\node (solver2) at (2,1) {MIP};
\node (solver3) at (3,1) {SAT};
\node (solver4) at (4,1) {Other};
\node[shape=rectangle,fill=pantone157-8] (solution1) at (1,0) {Solution};
\node[shape=rectangle,fill=pantone157-8] (solution2) at (2,0) {Solution};
\node[shape=rectangle,fill=pantone157-8] (solution3) at (3,0) {Solution};
\node[shape=rectangle,fill=pantone157-8] (solution4) at (4,0) {Solution};
\draw[-] (problem) -- (human);
\draw[->] (human) -- (model);
\draw[-] (model) -- (compile);
\draw[-] (compile) -- (solver1);
\draw[-] (compile) -- (solver2);
\draw[-] (compile) -- (solver3);
\draw[-] (compile) -- (solver4);
\draw[->] (solver1) -- (solution1);
\draw[->] (solver2) -- (solution2);
\draw[->] (solver3) -- (solution3);
\draw[->] (solver4) -- (solution4);
\end{tikzpicture}
\end{center}
\end{frame}

% \begin{frame}
  % \frametitle{Do It Now!}
  % \begin{itemize}
  % \item Download and install Minizinc
  % \item \url{https://www.minizinc.org/}
  % \end{itemize}
% \end{frame}


\begin{frame}
\frametitle{Why use Puzzles as Examples?}
\begin{itemize}
\item Easy to understand the problem
\item Solvable by hand without specialized knowledge
\item Possible to compare automated to manual solving process
\end{itemize}

\begin{quote}
The puzzle, though inanimate, is presented as a solvable problem without lasting negative consequences, a very low-risk low-reward situation. By being a puzzle, the object is attempting to convince the user that it must be completed.
\end{quote}
{\tiny Source: \href{https://www.everydayrhetoric.com/2020/03/the-rhetoric-of-puzzles/}{Every Day Rhetoric}}
\end{frame}


\mode<all>{
\part{Basic Constraint Propagation}
}
\begin{frame}
\partpage
\end{frame}
\mode<all>{
\input{../sendmore/body}
}

\mode<all>{
\part{Global Constraints}
}
\begin{frame}
\partpage
\end{frame}
\mode<all>{
\input{../sudoku/body}
}

\clearpage
\mode<all>{
\part{Customizing Search}
}
\begin{frame}
\partpage
\end{frame}
\mode<all>{
\input{../nqueen/body}
}


%% \clearpage
%% \mode<all>{
%%   \part{Problem Decomposition}
%% }
%% \begin{frame}
%%   \partpage
%% \end{frame}
%% \mode<all>{
%%   \input{../party/body}
%%   }

%% \clearpage
%% \mode<all>{
%%   \part{More Global Constraints}
%% }
%% \begin{frame}
%%   \partpage
%% \end{frame}
%% \mode<all>{
%%   \input{../car/body}
%%   }

%% \clearpage
%% \mode<all>{
%%   \part{Methodology}
%% }
%% \begin{frame}
%%   \partpage
%% \end{frame}
%% \mode<all>{
%%   \input{../methodology/body}
%%   }

\clearpage
\mode<all>{
\part{What is missing?}
}
\begin{frame}
\partpage
\end{frame}

\begin{frame}
\frametitle{Many Specialized Topics}
\begin{itemize}
\item How to design efficient core engine
\item Hybrids with LP/MIP tools
\item Hybrids with SAT
\item Symmetry breaking
\item Use of MDD/BDD to encode sets of solutions
\item High level modelling tools
\item Debugging/visualization
\end{itemize}
\end{frame}

\begin{frame}
\frametitle{Reformulation}
\begin{itemize}
\item Just because the user has modelled it this way, it doesn't mean we have to solve it that way
\begin{itemize}
\item Replace some constraint(s) by other, equivalent constraints
\item Because we don't have that constraint in our system
\item For performance
\end{itemize}
\end{itemize}
\end{frame}

\begin{frame}
\frametitle{Learning}
\begin{itemize}
\item While solving the problem we can learn how to strengthen the model/search 
\begin{itemize}
\item Understand which constraints/method contribute to propagation and change schedule
\item Learn no-good constraints by explaining failure
\item Adapt search strategy based on search experience
\end{itemize}
\end{itemize}
\end{frame}

%% \begin{frame}
%% \frametitle{Refined Process}
%% \begin{center}
%% \begin{tikzpicture}
%% \node[shape=rectangle,fill=pantone127-4] (problem) at (3,7) {Problem};
%% \node (human) at (3,6) {Human};
%% \node[shape=rectangle,fill=pantone127-4] (model) at (3,5) {User Model};
%% \node (reformulation) at (3,4) {Reformulation};
%% \node[shape=rectangle,fill=pantone127-4] (implementation) at (3,3) {Implementation Model};
%% \node (solver) at (3,2) {Constraint Solver/Search};
%% \node[shape=rectangle,fill=pantone157-8] (solution) at (3,1) {Solution};
%% \draw[-] (problem) -- (human);
%% \draw[->] (human) -- (model);
%% \draw[-] (model) -- (reformulation);
%% \draw[->] (reformulation) -- (implementation);
%% \draw[-] (implementation) -- (solver);
%% \draw[->] (solver) -- (0,2) -- (0,3) -- (implementation);
%% \draw[->] (solver) -- (solution);
%% \end{tikzpicture}
%% \end{center}
%% \end{frame}

%\begin{frame}
%\frametitle{What is CP actually used for?}
%\url{http://hsimonis.wordpress.com}
%\includegraphics[width=10cm]{../introduction/images/blog}
%\end{frame}

%% \begin{frame}
%% \frametitle{Exercises}
%% \begin{itemize}
%% \item Install Constraint Solver of your Choice
%% \item MiniZinc (MiningZinc) will be used in later sessions
%% \item Run Sendmoremoney, Sudoku, Queens examples
%% \item Can you choose which propagator to use for alldifferent?
%% \item Can you use different search strategies?
%% \end{itemize}
%% \end{frame}

\begin{frame}
\frametitle{More Learning Resources}
\begin{itemize}
\item Survey of Methods, Resources, and Formats for Teaching Constraint Programming
\begin{itemize}
\item by Tejas Santanam, Helmut Simonis
\item \url{https://doi.org/10.48550/arXiv.2403.12717}
\item Based on survey of community for WTCP 2023
\item \url{https://hsimonis.github.io/WTCP2023/}
\end{itemize}
\end{itemize}
\end{frame}


}

\mode<all>{
\part{Global Constraints}
}
\begin{frame}
\partpage
\end{frame}
\mode<all>{
\begin{frame}
\frametitle{Objectives}
\begin{itemize}
\item Overview of Core Constraint Programming
\item Three Main Concepts
\begin{itemize}
\item Constraint Propagation
\item Global Constraints
\item Customizing Search
\end{itemize}
%\item Get Some Experience with MiniZinc
\item Topics will be treated in more detail in later parts of the school
\item Based on Examples, not Formal Description
\end{itemize}
\end{frame}

\begin{frame}
\frametitle{Outline}
\begin{itemize}
\item Why Constraint Programming?
\item Constraint Propagation
\item Global Constraints
\item Customizing Search
%\item Choices in Modelling
%\item Decomposition
%\item More Global Constraints
%\item Methodology
%\item What is missing?
\end{itemize}
\end{frame}

% \begin{frame}
  % \frametitle{Using MiniZinc IDE}
  % \begin{itemize}
  % \item Developed in the Australian NICTA project
  % \item Maintained by Monash University
  % \item Modelling tool with multiple back-end solvers
  % \item Available from \url{https://www.minizinc.org/}
  % \end{itemize}
% \end{frame}


% \begin{frame}
% \frametitle{Examples in ECLiPSe}
% \begin{itemize}
% \item Open sourced constraint programming language
% \item Development goes back to 1985
% \item ECRC, ICL, IC-Parc, PTL, Cisco
% \item \url{https://eclipseclp.org/}
% \item Specialities
% \begin{itemize}
% \item Develop new solvers for specific domains
% \item Integration with MIP
% \end{itemize}
% \item Not included in bundled MiniZinc IDE
% \item Specialized visualization tools used here
  % \begin{itemize}
  % \item CP-Viz, Simonis et al. 2010
    % \end{itemize}
% \end{itemize}
% \end{frame}

\begin{frame}
\frametitle{Tutorial Based on ECLiPSe ELearning Course}
\begin{itemize}
\item Self-study course in constraint programming
\item Supported by Cisco Systems and Silicon Valley Community Foundation
\item Multi-media format, video lectures, slides, handout etc
\item {\small \url{https://eclipseclp.org/ELearning/index.html}}
\end{itemize}
\end{frame}

\begin{frame}
\frametitle{Also Part of CRT-AI Constraint Week}
\begin{itemize}
\item Annual one week course on CP and Optimization in Ireland
\item Part of national training program for PhD students in AI
\item \url{https://www.crt-ai.ie/}
\end{itemize}
\end{frame}



\begin{frame}
\frametitle{Constraint Programming - in a nutshell}
\begin{itemize}
\item Declarative description of problems with
\begin{itemize}
\item {\em Variables} which range over (finite) sets of values
\item {\em Constraints} over subsets of variables which restrict possible value combinations
\item A {\em solution} is a value assignment which satisfies all constraints
\end{itemize}

\item Constraint propagation/reasoning
\begin{itemize}
\item Removing inconsistent values for variables
\item Detect failure if constraint can not be satisfied
\item Interaction of constraints via shared variables
\item Incomplete
\end{itemize}

\item Search
\begin{itemize}
\item User controlled assignment of values to variables
\item Each step triggers constraint propagation 
\end{itemize}
\item Different domains require/allow different methods
\end{itemize}
\end{frame}

\begin{frame}
  \frametitle{Constraint Programming is Different}
  \begin{itemize}
  \item Declarative Programming
    \begin{itemize}
    \item Concentrate on what you want
      \item Not how to get there
      \item Program != Algorithm
      \item Program = Model
    \end{itemize}
    \item Applied to Combinatorial Problems
      \begin{itemize}
        \item No complete polynomial algorithms known (exist?)
        \item CP less ad-hoc than heuristics
        \item Models can evolve
  \end{itemize}
  \end{itemize}
  \end{frame}
    
\begin{frame}
  \frametitle{A Subtractive Process}
  \begin{textblock}{4}(8,-3)
    \includegraphics[width=4cm]{../introduction/images/stages}
  \end{textblock}
  \vfill
  \begin{quote}
    ``Oh, bosh, as Mr. Ruskin says. Sculpture, per se, is the simplest thing in the world. All you have to do is to take a big chunk of marble and a hammer and chisel, make up your mind what you are about to create and chip off all the marble you don't want.''-Paris Gaulois.
  \end{quote}
  
  {\tiny Source: \url{https://quoteinvestigator.com/2014/06/22/chip-away/}}
\end{frame}

\begin{frame}
\frametitle{Basic Process}
\begin{center}
\begin{tikzpicture}
\node[shape=rectangle,fill=pantone127-4] (problem) at (2,5) {Problem};
\node (human) at (2,4) {Human};
\node[shape=rectangle,fill=pantone127-4] (model) at (2,3) {Model};
\node (solver) at (2,2) {Constraint Solver/Search};
\node[shape=rectangle,fill=pantone157-8] (solution) at (2,1) {Solution};
\draw[-] (problem) -- (human);
\draw[->] (human) -- (model);
\draw[-] (model) -- (solver);
\draw[->] (solver) -- (solution);
\end{tikzpicture}
\end{center}
\end{frame}

\begin{frame}
\frametitle{More Realistic}
\begin{center}
\begin{tikzpicture}[xscale=4.0]
\node[shape=rectangle,fill=pantone127-4] (problem) at (2,5) {Problem};
\node (human) at (2,4) {Human};
\node[shape=rectangle,fill=pantone127-4] (model) at (2,3) {Model};
\node (solver) at (2,2) {Constraint Solver/Search};
\node[shape=rectangle,fill=pantone24-8] (hangs) at (1,1) {Hangs};
\node[shape=rectangle,fill=pantone157-8] (solution) at (2,1) {Solution};
\node[shape=rectangle,fill=pantone24-8] (wrong) at (3,1) {Wrong Solution};
\draw[-] (problem) -- (human);
\draw[->] (human) -- (model);
\draw[-] (model) -- (solver);
\draw[->] (solver) -- (hangs);
\draw[->] (solver) -- (solution);
\draw[->] (solver) -- (wrong);
\draw[->] (hangs) -- (1,4) -- (human);
\draw[->] (wrong) -- (3,4) -- (human);
\end{tikzpicture}
\end{center}
\end{frame}

\begin{frame}
\frametitle{Dual Role of Model}
\begin{itemize}
\item Allows Human to Express Problem
\begin{itemize}
\item Close to Problem Domain
\item Constraints as Abstractions
\end{itemize}
\item Allows Solver to Execute
\begin{itemize}
\item Variables as Communication Mechanism
\item Constraints as Algorithms
\end{itemize}
\end{itemize}
\end{frame}

\begin{frame}
\frametitle{Modelling Frameworks}
\begin{itemize}
\item \href{https://www.minizinc.org/}{MiniZinc} (NICTA, Monash University, Australia)
\item \href{https://github.com/eomahony/Numberjack}{NumberJack} (Insight, Ireland) 
\item \href{https://www-users.york.ac.uk/peter.nightingale/savilerow//}{EssencePrime/SavilleRow} (UK) 
\item \href{https://cpmpy.readthedocs.io/en/latest/}{CPMpy} (KU Leuven)
\item Allow use of multiple back-end solvers
\item Compile model into variants for each solver
\item A priori solver independent model(CP, MIP, SAT)
\end{itemize}
\end{frame} 

\begin{frame}
\frametitle{Framework Process}
\begin{center}
\begin{tikzpicture}[xscale=2]
\node[shape=rectangle,fill=pantone127-4] (problem) at (2.5,5) {Problem};
\node (human) at (2.5,4) {Human};
\node[shape=rectangle,fill=pantone127-4] (model) at (2.5,3) {Model};
\node (compile) at (2.5,2) {Compile/Reformulate};
\node (solver1) at (1,1) {CP};
\node (solver2) at (2,1) {MIP};
\node (solver3) at (3,1) {SAT};
\node (solver4) at (4,1) {Other};
\node[shape=rectangle,fill=pantone157-8] (solution1) at (1,0) {Solution};
\node[shape=rectangle,fill=pantone157-8] (solution2) at (2,0) {Solution};
\node[shape=rectangle,fill=pantone157-8] (solution3) at (3,0) {Solution};
\node[shape=rectangle,fill=pantone157-8] (solution4) at (4,0) {Solution};
\draw[-] (problem) -- (human);
\draw[->] (human) -- (model);
\draw[-] (model) -- (compile);
\draw[-] (compile) -- (solver1);
\draw[-] (compile) -- (solver2);
\draw[-] (compile) -- (solver3);
\draw[-] (compile) -- (solver4);
\draw[->] (solver1) -- (solution1);
\draw[->] (solver2) -- (solution2);
\draw[->] (solver3) -- (solution3);
\draw[->] (solver4) -- (solution4);
\end{tikzpicture}
\end{center}
\end{frame}

% \begin{frame}
  % \frametitle{Do It Now!}
  % \begin{itemize}
  % \item Download and install Minizinc
  % \item \url{https://www.minizinc.org/}
  % \end{itemize}
% \end{frame}


\begin{frame}
\frametitle{Why use Puzzles as Examples?}
\begin{itemize}
\item Easy to understand the problem
\item Solvable by hand without specialized knowledge
\item Possible to compare automated to manual solving process
\end{itemize}

\begin{quote}
The puzzle, though inanimate, is presented as a solvable problem without lasting negative consequences, a very low-risk low-reward situation. By being a puzzle, the object is attempting to convince the user that it must be completed.
\end{quote}
{\tiny Source: \href{https://www.everydayrhetoric.com/2020/03/the-rhetoric-of-puzzles/}{Every Day Rhetoric}}
\end{frame}


\mode<all>{
\part{Basic Constraint Propagation}
}
\begin{frame}
\partpage
\end{frame}
\mode<all>{
\input{../sendmore/body}
}

\mode<all>{
\part{Global Constraints}
}
\begin{frame}
\partpage
\end{frame}
\mode<all>{
\input{../sudoku/body}
}

\clearpage
\mode<all>{
\part{Customizing Search}
}
\begin{frame}
\partpage
\end{frame}
\mode<all>{
\input{../nqueen/body}
}


%% \clearpage
%% \mode<all>{
%%   \part{Problem Decomposition}
%% }
%% \begin{frame}
%%   \partpage
%% \end{frame}
%% \mode<all>{
%%   \input{../party/body}
%%   }

%% \clearpage
%% \mode<all>{
%%   \part{More Global Constraints}
%% }
%% \begin{frame}
%%   \partpage
%% \end{frame}
%% \mode<all>{
%%   \input{../car/body}
%%   }

%% \clearpage
%% \mode<all>{
%%   \part{Methodology}
%% }
%% \begin{frame}
%%   \partpage
%% \end{frame}
%% \mode<all>{
%%   \input{../methodology/body}
%%   }

\clearpage
\mode<all>{
\part{What is missing?}
}
\begin{frame}
\partpage
\end{frame}

\begin{frame}
\frametitle{Many Specialized Topics}
\begin{itemize}
\item How to design efficient core engine
\item Hybrids with LP/MIP tools
\item Hybrids with SAT
\item Symmetry breaking
\item Use of MDD/BDD to encode sets of solutions
\item High level modelling tools
\item Debugging/visualization
\end{itemize}
\end{frame}

\begin{frame}
\frametitle{Reformulation}
\begin{itemize}
\item Just because the user has modelled it this way, it doesn't mean we have to solve it that way
\begin{itemize}
\item Replace some constraint(s) by other, equivalent constraints
\item Because we don't have that constraint in our system
\item For performance
\end{itemize}
\end{itemize}
\end{frame}

\begin{frame}
\frametitle{Learning}
\begin{itemize}
\item While solving the problem we can learn how to strengthen the model/search 
\begin{itemize}
\item Understand which constraints/method contribute to propagation and change schedule
\item Learn no-good constraints by explaining failure
\item Adapt search strategy based on search experience
\end{itemize}
\end{itemize}
\end{frame}

%% \begin{frame}
%% \frametitle{Refined Process}
%% \begin{center}
%% \begin{tikzpicture}
%% \node[shape=rectangle,fill=pantone127-4] (problem) at (3,7) {Problem};
%% \node (human) at (3,6) {Human};
%% \node[shape=rectangle,fill=pantone127-4] (model) at (3,5) {User Model};
%% \node (reformulation) at (3,4) {Reformulation};
%% \node[shape=rectangle,fill=pantone127-4] (implementation) at (3,3) {Implementation Model};
%% \node (solver) at (3,2) {Constraint Solver/Search};
%% \node[shape=rectangle,fill=pantone157-8] (solution) at (3,1) {Solution};
%% \draw[-] (problem) -- (human);
%% \draw[->] (human) -- (model);
%% \draw[-] (model) -- (reformulation);
%% \draw[->] (reformulation) -- (implementation);
%% \draw[-] (implementation) -- (solver);
%% \draw[->] (solver) -- (0,2) -- (0,3) -- (implementation);
%% \draw[->] (solver) -- (solution);
%% \end{tikzpicture}
%% \end{center}
%% \end{frame}

%\begin{frame}
%\frametitle{What is CP actually used for?}
%\url{http://hsimonis.wordpress.com}
%\includegraphics[width=10cm]{../introduction/images/blog}
%\end{frame}

%% \begin{frame}
%% \frametitle{Exercises}
%% \begin{itemize}
%% \item Install Constraint Solver of your Choice
%% \item MiniZinc (MiningZinc) will be used in later sessions
%% \item Run Sendmoremoney, Sudoku, Queens examples
%% \item Can you choose which propagator to use for alldifferent?
%% \item Can you use different search strategies?
%% \end{itemize}
%% \end{frame}

\begin{frame}
\frametitle{More Learning Resources}
\begin{itemize}
\item Survey of Methods, Resources, and Formats for Teaching Constraint Programming
\begin{itemize}
\item by Tejas Santanam, Helmut Simonis
\item \url{https://doi.org/10.48550/arXiv.2403.12717}
\item Based on survey of community for WTCP 2023
\item \url{https://hsimonis.github.io/WTCP2023/}
\end{itemize}
\end{itemize}
\end{frame}


}

\clearpage
\mode<all>{
\part{Customizing Search}
}
\begin{frame}
\partpage
\end{frame}
\mode<all>{
\begin{frame}
\frametitle{Objectives}
\begin{itemize}
\item Overview of Core Constraint Programming
\item Three Main Concepts
\begin{itemize}
\item Constraint Propagation
\item Global Constraints
\item Customizing Search
\end{itemize}
%\item Get Some Experience with MiniZinc
\item Topics will be treated in more detail in later parts of the school
\item Based on Examples, not Formal Description
\end{itemize}
\end{frame}

\begin{frame}
\frametitle{Outline}
\begin{itemize}
\item Why Constraint Programming?
\item Constraint Propagation
\item Global Constraints
\item Customizing Search
%\item Choices in Modelling
%\item Decomposition
%\item More Global Constraints
%\item Methodology
%\item What is missing?
\end{itemize}
\end{frame}

% \begin{frame}
  % \frametitle{Using MiniZinc IDE}
  % \begin{itemize}
  % \item Developed in the Australian NICTA project
  % \item Maintained by Monash University
  % \item Modelling tool with multiple back-end solvers
  % \item Available from \url{https://www.minizinc.org/}
  % \end{itemize}
% \end{frame}


% \begin{frame}
% \frametitle{Examples in ECLiPSe}
% \begin{itemize}
% \item Open sourced constraint programming language
% \item Development goes back to 1985
% \item ECRC, ICL, IC-Parc, PTL, Cisco
% \item \url{https://eclipseclp.org/}
% \item Specialities
% \begin{itemize}
% \item Develop new solvers for specific domains
% \item Integration with MIP
% \end{itemize}
% \item Not included in bundled MiniZinc IDE
% \item Specialized visualization tools used here
  % \begin{itemize}
  % \item CP-Viz, Simonis et al. 2010
    % \end{itemize}
% \end{itemize}
% \end{frame}

\begin{frame}
\frametitle{Tutorial Based on ECLiPSe ELearning Course}
\begin{itemize}
\item Self-study course in constraint programming
\item Supported by Cisco Systems and Silicon Valley Community Foundation
\item Multi-media format, video lectures, slides, handout etc
\item {\small \url{https://eclipseclp.org/ELearning/index.html}}
\end{itemize}
\end{frame}

\begin{frame}
\frametitle{Also Part of CRT-AI Constraint Week}
\begin{itemize}
\item Annual one week course on CP and Optimization in Ireland
\item Part of national training program for PhD students in AI
\item \url{https://www.crt-ai.ie/}
\end{itemize}
\end{frame}



\begin{frame}
\frametitle{Constraint Programming - in a nutshell}
\begin{itemize}
\item Declarative description of problems with
\begin{itemize}
\item {\em Variables} which range over (finite) sets of values
\item {\em Constraints} over subsets of variables which restrict possible value combinations
\item A {\em solution} is a value assignment which satisfies all constraints
\end{itemize}

\item Constraint propagation/reasoning
\begin{itemize}
\item Removing inconsistent values for variables
\item Detect failure if constraint can not be satisfied
\item Interaction of constraints via shared variables
\item Incomplete
\end{itemize}

\item Search
\begin{itemize}
\item User controlled assignment of values to variables
\item Each step triggers constraint propagation 
\end{itemize}
\item Different domains require/allow different methods
\end{itemize}
\end{frame}

\begin{frame}
  \frametitle{Constraint Programming is Different}
  \begin{itemize}
  \item Declarative Programming
    \begin{itemize}
    \item Concentrate on what you want
      \item Not how to get there
      \item Program != Algorithm
      \item Program = Model
    \end{itemize}
    \item Applied to Combinatorial Problems
      \begin{itemize}
        \item No complete polynomial algorithms known (exist?)
        \item CP less ad-hoc than heuristics
        \item Models can evolve
  \end{itemize}
  \end{itemize}
  \end{frame}
    
\begin{frame}
  \frametitle{A Subtractive Process}
  \begin{textblock}{4}(8,-3)
    \includegraphics[width=4cm]{../introduction/images/stages}
  \end{textblock}
  \vfill
  \begin{quote}
    ``Oh, bosh, as Mr. Ruskin says. Sculpture, per se, is the simplest thing in the world. All you have to do is to take a big chunk of marble and a hammer and chisel, make up your mind what you are about to create and chip off all the marble you don't want.''-Paris Gaulois.
  \end{quote}
  
  {\tiny Source: \url{https://quoteinvestigator.com/2014/06/22/chip-away/}}
\end{frame}

\begin{frame}
\frametitle{Basic Process}
\begin{center}
\begin{tikzpicture}
\node[shape=rectangle,fill=pantone127-4] (problem) at (2,5) {Problem};
\node (human) at (2,4) {Human};
\node[shape=rectangle,fill=pantone127-4] (model) at (2,3) {Model};
\node (solver) at (2,2) {Constraint Solver/Search};
\node[shape=rectangle,fill=pantone157-8] (solution) at (2,1) {Solution};
\draw[-] (problem) -- (human);
\draw[->] (human) -- (model);
\draw[-] (model) -- (solver);
\draw[->] (solver) -- (solution);
\end{tikzpicture}
\end{center}
\end{frame}

\begin{frame}
\frametitle{More Realistic}
\begin{center}
\begin{tikzpicture}[xscale=4.0]
\node[shape=rectangle,fill=pantone127-4] (problem) at (2,5) {Problem};
\node (human) at (2,4) {Human};
\node[shape=rectangle,fill=pantone127-4] (model) at (2,3) {Model};
\node (solver) at (2,2) {Constraint Solver/Search};
\node[shape=rectangle,fill=pantone24-8] (hangs) at (1,1) {Hangs};
\node[shape=rectangle,fill=pantone157-8] (solution) at (2,1) {Solution};
\node[shape=rectangle,fill=pantone24-8] (wrong) at (3,1) {Wrong Solution};
\draw[-] (problem) -- (human);
\draw[->] (human) -- (model);
\draw[-] (model) -- (solver);
\draw[->] (solver) -- (hangs);
\draw[->] (solver) -- (solution);
\draw[->] (solver) -- (wrong);
\draw[->] (hangs) -- (1,4) -- (human);
\draw[->] (wrong) -- (3,4) -- (human);
\end{tikzpicture}
\end{center}
\end{frame}

\begin{frame}
\frametitle{Dual Role of Model}
\begin{itemize}
\item Allows Human to Express Problem
\begin{itemize}
\item Close to Problem Domain
\item Constraints as Abstractions
\end{itemize}
\item Allows Solver to Execute
\begin{itemize}
\item Variables as Communication Mechanism
\item Constraints as Algorithms
\end{itemize}
\end{itemize}
\end{frame}

\begin{frame}
\frametitle{Modelling Frameworks}
\begin{itemize}
\item \href{https://www.minizinc.org/}{MiniZinc} (NICTA, Monash University, Australia)
\item \href{https://github.com/eomahony/Numberjack}{NumberJack} (Insight, Ireland) 
\item \href{https://www-users.york.ac.uk/peter.nightingale/savilerow//}{EssencePrime/SavilleRow} (UK) 
\item \href{https://cpmpy.readthedocs.io/en/latest/}{CPMpy} (KU Leuven)
\item Allow use of multiple back-end solvers
\item Compile model into variants for each solver
\item A priori solver independent model(CP, MIP, SAT)
\end{itemize}
\end{frame} 

\begin{frame}
\frametitle{Framework Process}
\begin{center}
\begin{tikzpicture}[xscale=2]
\node[shape=rectangle,fill=pantone127-4] (problem) at (2.5,5) {Problem};
\node (human) at (2.5,4) {Human};
\node[shape=rectangle,fill=pantone127-4] (model) at (2.5,3) {Model};
\node (compile) at (2.5,2) {Compile/Reformulate};
\node (solver1) at (1,1) {CP};
\node (solver2) at (2,1) {MIP};
\node (solver3) at (3,1) {SAT};
\node (solver4) at (4,1) {Other};
\node[shape=rectangle,fill=pantone157-8] (solution1) at (1,0) {Solution};
\node[shape=rectangle,fill=pantone157-8] (solution2) at (2,0) {Solution};
\node[shape=rectangle,fill=pantone157-8] (solution3) at (3,0) {Solution};
\node[shape=rectangle,fill=pantone157-8] (solution4) at (4,0) {Solution};
\draw[-] (problem) -- (human);
\draw[->] (human) -- (model);
\draw[-] (model) -- (compile);
\draw[-] (compile) -- (solver1);
\draw[-] (compile) -- (solver2);
\draw[-] (compile) -- (solver3);
\draw[-] (compile) -- (solver4);
\draw[->] (solver1) -- (solution1);
\draw[->] (solver2) -- (solution2);
\draw[->] (solver3) -- (solution3);
\draw[->] (solver4) -- (solution4);
\end{tikzpicture}
\end{center}
\end{frame}

% \begin{frame}
  % \frametitle{Do It Now!}
  % \begin{itemize}
  % \item Download and install Minizinc
  % \item \url{https://www.minizinc.org/}
  % \end{itemize}
% \end{frame}


\begin{frame}
\frametitle{Why use Puzzles as Examples?}
\begin{itemize}
\item Easy to understand the problem
\item Solvable by hand without specialized knowledge
\item Possible to compare automated to manual solving process
\end{itemize}

\begin{quote}
The puzzle, though inanimate, is presented as a solvable problem without lasting negative consequences, a very low-risk low-reward situation. By being a puzzle, the object is attempting to convince the user that it must be completed.
\end{quote}
{\tiny Source: \href{https://www.everydayrhetoric.com/2020/03/the-rhetoric-of-puzzles/}{Every Day Rhetoric}}
\end{frame}


\mode<all>{
\part{Basic Constraint Propagation}
}
\begin{frame}
\partpage
\end{frame}
\mode<all>{
\input{../sendmore/body}
}

\mode<all>{
\part{Global Constraints}
}
\begin{frame}
\partpage
\end{frame}
\mode<all>{
\input{../sudoku/body}
}

\clearpage
\mode<all>{
\part{Customizing Search}
}
\begin{frame}
\partpage
\end{frame}
\mode<all>{
\input{../nqueen/body}
}


%% \clearpage
%% \mode<all>{
%%   \part{Problem Decomposition}
%% }
%% \begin{frame}
%%   \partpage
%% \end{frame}
%% \mode<all>{
%%   \input{../party/body}
%%   }

%% \clearpage
%% \mode<all>{
%%   \part{More Global Constraints}
%% }
%% \begin{frame}
%%   \partpage
%% \end{frame}
%% \mode<all>{
%%   \input{../car/body}
%%   }

%% \clearpage
%% \mode<all>{
%%   \part{Methodology}
%% }
%% \begin{frame}
%%   \partpage
%% \end{frame}
%% \mode<all>{
%%   \input{../methodology/body}
%%   }

\clearpage
\mode<all>{
\part{What is missing?}
}
\begin{frame}
\partpage
\end{frame}

\begin{frame}
\frametitle{Many Specialized Topics}
\begin{itemize}
\item How to design efficient core engine
\item Hybrids with LP/MIP tools
\item Hybrids with SAT
\item Symmetry breaking
\item Use of MDD/BDD to encode sets of solutions
\item High level modelling tools
\item Debugging/visualization
\end{itemize}
\end{frame}

\begin{frame}
\frametitle{Reformulation}
\begin{itemize}
\item Just because the user has modelled it this way, it doesn't mean we have to solve it that way
\begin{itemize}
\item Replace some constraint(s) by other, equivalent constraints
\item Because we don't have that constraint in our system
\item For performance
\end{itemize}
\end{itemize}
\end{frame}

\begin{frame}
\frametitle{Learning}
\begin{itemize}
\item While solving the problem we can learn how to strengthen the model/search 
\begin{itemize}
\item Understand which constraints/method contribute to propagation and change schedule
\item Learn no-good constraints by explaining failure
\item Adapt search strategy based on search experience
\end{itemize}
\end{itemize}
\end{frame}

%% \begin{frame}
%% \frametitle{Refined Process}
%% \begin{center}
%% \begin{tikzpicture}
%% \node[shape=rectangle,fill=pantone127-4] (problem) at (3,7) {Problem};
%% \node (human) at (3,6) {Human};
%% \node[shape=rectangle,fill=pantone127-4] (model) at (3,5) {User Model};
%% \node (reformulation) at (3,4) {Reformulation};
%% \node[shape=rectangle,fill=pantone127-4] (implementation) at (3,3) {Implementation Model};
%% \node (solver) at (3,2) {Constraint Solver/Search};
%% \node[shape=rectangle,fill=pantone157-8] (solution) at (3,1) {Solution};
%% \draw[-] (problem) -- (human);
%% \draw[->] (human) -- (model);
%% \draw[-] (model) -- (reformulation);
%% \draw[->] (reformulation) -- (implementation);
%% \draw[-] (implementation) -- (solver);
%% \draw[->] (solver) -- (0,2) -- (0,3) -- (implementation);
%% \draw[->] (solver) -- (solution);
%% \end{tikzpicture}
%% \end{center}
%% \end{frame}

%\begin{frame}
%\frametitle{What is CP actually used for?}
%\url{http://hsimonis.wordpress.com}
%\includegraphics[width=10cm]{../introduction/images/blog}
%\end{frame}

%% \begin{frame}
%% \frametitle{Exercises}
%% \begin{itemize}
%% \item Install Constraint Solver of your Choice
%% \item MiniZinc (MiningZinc) will be used in later sessions
%% \item Run Sendmoremoney, Sudoku, Queens examples
%% \item Can you choose which propagator to use for alldifferent?
%% \item Can you use different search strategies?
%% \end{itemize}
%% \end{frame}

\begin{frame}
\frametitle{More Learning Resources}
\begin{itemize}
\item Survey of Methods, Resources, and Formats for Teaching Constraint Programming
\begin{itemize}
\item by Tejas Santanam, Helmut Simonis
\item \url{https://doi.org/10.48550/arXiv.2403.12717}
\item Based on survey of community for WTCP 2023
\item \url{https://hsimonis.github.io/WTCP2023/}
\end{itemize}
\end{itemize}
\end{frame}


}


%% \clearpage
%% \mode<all>{
%%   \part{Problem Decomposition}
%% }
%% \begin{frame}
%%   \partpage
%% \end{frame}
%% \mode<all>{
%%   \begin{frame}
\frametitle{Objectives}
\begin{itemize}
\item Overview of Core Constraint Programming
\item Three Main Concepts
\begin{itemize}
\item Constraint Propagation
\item Global Constraints
\item Customizing Search
\end{itemize}
%\item Get Some Experience with MiniZinc
\item Topics will be treated in more detail in later parts of the school
\item Based on Examples, not Formal Description
\end{itemize}
\end{frame}

\begin{frame}
\frametitle{Outline}
\begin{itemize}
\item Why Constraint Programming?
\item Constraint Propagation
\item Global Constraints
\item Customizing Search
%\item Choices in Modelling
%\item Decomposition
%\item More Global Constraints
%\item Methodology
%\item What is missing?
\end{itemize}
\end{frame}

% \begin{frame}
  % \frametitle{Using MiniZinc IDE}
  % \begin{itemize}
  % \item Developed in the Australian NICTA project
  % \item Maintained by Monash University
  % \item Modelling tool with multiple back-end solvers
  % \item Available from \url{https://www.minizinc.org/}
  % \end{itemize}
% \end{frame}


% \begin{frame}
% \frametitle{Examples in ECLiPSe}
% \begin{itemize}
% \item Open sourced constraint programming language
% \item Development goes back to 1985
% \item ECRC, ICL, IC-Parc, PTL, Cisco
% \item \url{https://eclipseclp.org/}
% \item Specialities
% \begin{itemize}
% \item Develop new solvers for specific domains
% \item Integration with MIP
% \end{itemize}
% \item Not included in bundled MiniZinc IDE
% \item Specialized visualization tools used here
  % \begin{itemize}
  % \item CP-Viz, Simonis et al. 2010
    % \end{itemize}
% \end{itemize}
% \end{frame}

\begin{frame}
\frametitle{Tutorial Based on ECLiPSe ELearning Course}
\begin{itemize}
\item Self-study course in constraint programming
\item Supported by Cisco Systems and Silicon Valley Community Foundation
\item Multi-media format, video lectures, slides, handout etc
\item {\small \url{https://eclipseclp.org/ELearning/index.html}}
\end{itemize}
\end{frame}

\begin{frame}
\frametitle{Also Part of CRT-AI Constraint Week}
\begin{itemize}
\item Annual one week course on CP and Optimization in Ireland
\item Part of national training program for PhD students in AI
\item \url{https://www.crt-ai.ie/}
\end{itemize}
\end{frame}



\begin{frame}
\frametitle{Constraint Programming - in a nutshell}
\begin{itemize}
\item Declarative description of problems with
\begin{itemize}
\item {\em Variables} which range over (finite) sets of values
\item {\em Constraints} over subsets of variables which restrict possible value combinations
\item A {\em solution} is a value assignment which satisfies all constraints
\end{itemize}

\item Constraint propagation/reasoning
\begin{itemize}
\item Removing inconsistent values for variables
\item Detect failure if constraint can not be satisfied
\item Interaction of constraints via shared variables
\item Incomplete
\end{itemize}

\item Search
\begin{itemize}
\item User controlled assignment of values to variables
\item Each step triggers constraint propagation 
\end{itemize}
\item Different domains require/allow different methods
\end{itemize}
\end{frame}

\begin{frame}
  \frametitle{Constraint Programming is Different}
  \begin{itemize}
  \item Declarative Programming
    \begin{itemize}
    \item Concentrate on what you want
      \item Not how to get there
      \item Program != Algorithm
      \item Program = Model
    \end{itemize}
    \item Applied to Combinatorial Problems
      \begin{itemize}
        \item No complete polynomial algorithms known (exist?)
        \item CP less ad-hoc than heuristics
        \item Models can evolve
  \end{itemize}
  \end{itemize}
  \end{frame}
    
\begin{frame}
  \frametitle{A Subtractive Process}
  \begin{textblock}{4}(8,-3)
    \includegraphics[width=4cm]{../introduction/images/stages}
  \end{textblock}
  \vfill
  \begin{quote}
    ``Oh, bosh, as Mr. Ruskin says. Sculpture, per se, is the simplest thing in the world. All you have to do is to take a big chunk of marble and a hammer and chisel, make up your mind what you are about to create and chip off all the marble you don't want.''-Paris Gaulois.
  \end{quote}
  
  {\tiny Source: \url{https://quoteinvestigator.com/2014/06/22/chip-away/}}
\end{frame}

\begin{frame}
\frametitle{Basic Process}
\begin{center}
\begin{tikzpicture}
\node[shape=rectangle,fill=pantone127-4] (problem) at (2,5) {Problem};
\node (human) at (2,4) {Human};
\node[shape=rectangle,fill=pantone127-4] (model) at (2,3) {Model};
\node (solver) at (2,2) {Constraint Solver/Search};
\node[shape=rectangle,fill=pantone157-8] (solution) at (2,1) {Solution};
\draw[-] (problem) -- (human);
\draw[->] (human) -- (model);
\draw[-] (model) -- (solver);
\draw[->] (solver) -- (solution);
\end{tikzpicture}
\end{center}
\end{frame}

\begin{frame}
\frametitle{More Realistic}
\begin{center}
\begin{tikzpicture}[xscale=4.0]
\node[shape=rectangle,fill=pantone127-4] (problem) at (2,5) {Problem};
\node (human) at (2,4) {Human};
\node[shape=rectangle,fill=pantone127-4] (model) at (2,3) {Model};
\node (solver) at (2,2) {Constraint Solver/Search};
\node[shape=rectangle,fill=pantone24-8] (hangs) at (1,1) {Hangs};
\node[shape=rectangle,fill=pantone157-8] (solution) at (2,1) {Solution};
\node[shape=rectangle,fill=pantone24-8] (wrong) at (3,1) {Wrong Solution};
\draw[-] (problem) -- (human);
\draw[->] (human) -- (model);
\draw[-] (model) -- (solver);
\draw[->] (solver) -- (hangs);
\draw[->] (solver) -- (solution);
\draw[->] (solver) -- (wrong);
\draw[->] (hangs) -- (1,4) -- (human);
\draw[->] (wrong) -- (3,4) -- (human);
\end{tikzpicture}
\end{center}
\end{frame}

\begin{frame}
\frametitle{Dual Role of Model}
\begin{itemize}
\item Allows Human to Express Problem
\begin{itemize}
\item Close to Problem Domain
\item Constraints as Abstractions
\end{itemize}
\item Allows Solver to Execute
\begin{itemize}
\item Variables as Communication Mechanism
\item Constraints as Algorithms
\end{itemize}
\end{itemize}
\end{frame}

\begin{frame}
\frametitle{Modelling Frameworks}
\begin{itemize}
\item \href{https://www.minizinc.org/}{MiniZinc} (NICTA, Monash University, Australia)
\item \href{https://github.com/eomahony/Numberjack}{NumberJack} (Insight, Ireland) 
\item \href{https://www-users.york.ac.uk/peter.nightingale/savilerow//}{EssencePrime/SavilleRow} (UK) 
\item \href{https://cpmpy.readthedocs.io/en/latest/}{CPMpy} (KU Leuven)
\item Allow use of multiple back-end solvers
\item Compile model into variants for each solver
\item A priori solver independent model(CP, MIP, SAT)
\end{itemize}
\end{frame} 

\begin{frame}
\frametitle{Framework Process}
\begin{center}
\begin{tikzpicture}[xscale=2]
\node[shape=rectangle,fill=pantone127-4] (problem) at (2.5,5) {Problem};
\node (human) at (2.5,4) {Human};
\node[shape=rectangle,fill=pantone127-4] (model) at (2.5,3) {Model};
\node (compile) at (2.5,2) {Compile/Reformulate};
\node (solver1) at (1,1) {CP};
\node (solver2) at (2,1) {MIP};
\node (solver3) at (3,1) {SAT};
\node (solver4) at (4,1) {Other};
\node[shape=rectangle,fill=pantone157-8] (solution1) at (1,0) {Solution};
\node[shape=rectangle,fill=pantone157-8] (solution2) at (2,0) {Solution};
\node[shape=rectangle,fill=pantone157-8] (solution3) at (3,0) {Solution};
\node[shape=rectangle,fill=pantone157-8] (solution4) at (4,0) {Solution};
\draw[-] (problem) -- (human);
\draw[->] (human) -- (model);
\draw[-] (model) -- (compile);
\draw[-] (compile) -- (solver1);
\draw[-] (compile) -- (solver2);
\draw[-] (compile) -- (solver3);
\draw[-] (compile) -- (solver4);
\draw[->] (solver1) -- (solution1);
\draw[->] (solver2) -- (solution2);
\draw[->] (solver3) -- (solution3);
\draw[->] (solver4) -- (solution4);
\end{tikzpicture}
\end{center}
\end{frame}

% \begin{frame}
  % \frametitle{Do It Now!}
  % \begin{itemize}
  % \item Download and install Minizinc
  % \item \url{https://www.minizinc.org/}
  % \end{itemize}
% \end{frame}


\begin{frame}
\frametitle{Why use Puzzles as Examples?}
\begin{itemize}
\item Easy to understand the problem
\item Solvable by hand without specialized knowledge
\item Possible to compare automated to manual solving process
\end{itemize}

\begin{quote}
The puzzle, though inanimate, is presented as a solvable problem without lasting negative consequences, a very low-risk low-reward situation. By being a puzzle, the object is attempting to convince the user that it must be completed.
\end{quote}
{\tiny Source: \href{https://www.everydayrhetoric.com/2020/03/the-rhetoric-of-puzzles/}{Every Day Rhetoric}}
\end{frame}


\mode<all>{
\part{Basic Constraint Propagation}
}
\begin{frame}
\partpage
\end{frame}
\mode<all>{
\input{../sendmore/body}
}

\mode<all>{
\part{Global Constraints}
}
\begin{frame}
\partpage
\end{frame}
\mode<all>{
\input{../sudoku/body}
}

\clearpage
\mode<all>{
\part{Customizing Search}
}
\begin{frame}
\partpage
\end{frame}
\mode<all>{
\input{../nqueen/body}
}


%% \clearpage
%% \mode<all>{
%%   \part{Problem Decomposition}
%% }
%% \begin{frame}
%%   \partpage
%% \end{frame}
%% \mode<all>{
%%   \input{../party/body}
%%   }

%% \clearpage
%% \mode<all>{
%%   \part{More Global Constraints}
%% }
%% \begin{frame}
%%   \partpage
%% \end{frame}
%% \mode<all>{
%%   \input{../car/body}
%%   }

%% \clearpage
%% \mode<all>{
%%   \part{Methodology}
%% }
%% \begin{frame}
%%   \partpage
%% \end{frame}
%% \mode<all>{
%%   \input{../methodology/body}
%%   }

\clearpage
\mode<all>{
\part{What is missing?}
}
\begin{frame}
\partpage
\end{frame}

\begin{frame}
\frametitle{Many Specialized Topics}
\begin{itemize}
\item How to design efficient core engine
\item Hybrids with LP/MIP tools
\item Hybrids with SAT
\item Symmetry breaking
\item Use of MDD/BDD to encode sets of solutions
\item High level modelling tools
\item Debugging/visualization
\end{itemize}
\end{frame}

\begin{frame}
\frametitle{Reformulation}
\begin{itemize}
\item Just because the user has modelled it this way, it doesn't mean we have to solve it that way
\begin{itemize}
\item Replace some constraint(s) by other, equivalent constraints
\item Because we don't have that constraint in our system
\item For performance
\end{itemize}
\end{itemize}
\end{frame}

\begin{frame}
\frametitle{Learning}
\begin{itemize}
\item While solving the problem we can learn how to strengthen the model/search 
\begin{itemize}
\item Understand which constraints/method contribute to propagation and change schedule
\item Learn no-good constraints by explaining failure
\item Adapt search strategy based on search experience
\end{itemize}
\end{itemize}
\end{frame}

%% \begin{frame}
%% \frametitle{Refined Process}
%% \begin{center}
%% \begin{tikzpicture}
%% \node[shape=rectangle,fill=pantone127-4] (problem) at (3,7) {Problem};
%% \node (human) at (3,6) {Human};
%% \node[shape=rectangle,fill=pantone127-4] (model) at (3,5) {User Model};
%% \node (reformulation) at (3,4) {Reformulation};
%% \node[shape=rectangle,fill=pantone127-4] (implementation) at (3,3) {Implementation Model};
%% \node (solver) at (3,2) {Constraint Solver/Search};
%% \node[shape=rectangle,fill=pantone157-8] (solution) at (3,1) {Solution};
%% \draw[-] (problem) -- (human);
%% \draw[->] (human) -- (model);
%% \draw[-] (model) -- (reformulation);
%% \draw[->] (reformulation) -- (implementation);
%% \draw[-] (implementation) -- (solver);
%% \draw[->] (solver) -- (0,2) -- (0,3) -- (implementation);
%% \draw[->] (solver) -- (solution);
%% \end{tikzpicture}
%% \end{center}
%% \end{frame}

%\begin{frame}
%\frametitle{What is CP actually used for?}
%\url{http://hsimonis.wordpress.com}
%\includegraphics[width=10cm]{../introduction/images/blog}
%\end{frame}

%% \begin{frame}
%% \frametitle{Exercises}
%% \begin{itemize}
%% \item Install Constraint Solver of your Choice
%% \item MiniZinc (MiningZinc) will be used in later sessions
%% \item Run Sendmoremoney, Sudoku, Queens examples
%% \item Can you choose which propagator to use for alldifferent?
%% \item Can you use different search strategies?
%% \end{itemize}
%% \end{frame}

\begin{frame}
\frametitle{More Learning Resources}
\begin{itemize}
\item Survey of Methods, Resources, and Formats for Teaching Constraint Programming
\begin{itemize}
\item by Tejas Santanam, Helmut Simonis
\item \url{https://doi.org/10.48550/arXiv.2403.12717}
\item Based on survey of community for WTCP 2023
\item \url{https://hsimonis.github.io/WTCP2023/}
\end{itemize}
\end{itemize}
\end{frame}


%%   }

%% \clearpage
%% \mode<all>{
%%   \part{More Global Constraints}
%% }
%% \begin{frame}
%%   \partpage
%% \end{frame}
%% \mode<all>{
%%   \begin{frame}
\frametitle{Objectives}
\begin{itemize}
\item Overview of Core Constraint Programming
\item Three Main Concepts
\begin{itemize}
\item Constraint Propagation
\item Global Constraints
\item Customizing Search
\end{itemize}
%\item Get Some Experience with MiniZinc
\item Topics will be treated in more detail in later parts of the school
\item Based on Examples, not Formal Description
\end{itemize}
\end{frame}

\begin{frame}
\frametitle{Outline}
\begin{itemize}
\item Why Constraint Programming?
\item Constraint Propagation
\item Global Constraints
\item Customizing Search
%\item Choices in Modelling
%\item Decomposition
%\item More Global Constraints
%\item Methodology
%\item What is missing?
\end{itemize}
\end{frame}

% \begin{frame}
  % \frametitle{Using MiniZinc IDE}
  % \begin{itemize}
  % \item Developed in the Australian NICTA project
  % \item Maintained by Monash University
  % \item Modelling tool with multiple back-end solvers
  % \item Available from \url{https://www.minizinc.org/}
  % \end{itemize}
% \end{frame}


% \begin{frame}
% \frametitle{Examples in ECLiPSe}
% \begin{itemize}
% \item Open sourced constraint programming language
% \item Development goes back to 1985
% \item ECRC, ICL, IC-Parc, PTL, Cisco
% \item \url{https://eclipseclp.org/}
% \item Specialities
% \begin{itemize}
% \item Develop new solvers for specific domains
% \item Integration with MIP
% \end{itemize}
% \item Not included in bundled MiniZinc IDE
% \item Specialized visualization tools used here
  % \begin{itemize}
  % \item CP-Viz, Simonis et al. 2010
    % \end{itemize}
% \end{itemize}
% \end{frame}

\begin{frame}
\frametitle{Tutorial Based on ECLiPSe ELearning Course}
\begin{itemize}
\item Self-study course in constraint programming
\item Supported by Cisco Systems and Silicon Valley Community Foundation
\item Multi-media format, video lectures, slides, handout etc
\item {\small \url{https://eclipseclp.org/ELearning/index.html}}
\end{itemize}
\end{frame}

\begin{frame}
\frametitle{Also Part of CRT-AI Constraint Week}
\begin{itemize}
\item Annual one week course on CP and Optimization in Ireland
\item Part of national training program for PhD students in AI
\item \url{https://www.crt-ai.ie/}
\end{itemize}
\end{frame}



\begin{frame}
\frametitle{Constraint Programming - in a nutshell}
\begin{itemize}
\item Declarative description of problems with
\begin{itemize}
\item {\em Variables} which range over (finite) sets of values
\item {\em Constraints} over subsets of variables which restrict possible value combinations
\item A {\em solution} is a value assignment which satisfies all constraints
\end{itemize}

\item Constraint propagation/reasoning
\begin{itemize}
\item Removing inconsistent values for variables
\item Detect failure if constraint can not be satisfied
\item Interaction of constraints via shared variables
\item Incomplete
\end{itemize}

\item Search
\begin{itemize}
\item User controlled assignment of values to variables
\item Each step triggers constraint propagation 
\end{itemize}
\item Different domains require/allow different methods
\end{itemize}
\end{frame}

\begin{frame}
  \frametitle{Constraint Programming is Different}
  \begin{itemize}
  \item Declarative Programming
    \begin{itemize}
    \item Concentrate on what you want
      \item Not how to get there
      \item Program != Algorithm
      \item Program = Model
    \end{itemize}
    \item Applied to Combinatorial Problems
      \begin{itemize}
        \item No complete polynomial algorithms known (exist?)
        \item CP less ad-hoc than heuristics
        \item Models can evolve
  \end{itemize}
  \end{itemize}
  \end{frame}
    
\begin{frame}
  \frametitle{A Subtractive Process}
  \begin{textblock}{4}(8,-3)
    \includegraphics[width=4cm]{../introduction/images/stages}
  \end{textblock}
  \vfill
  \begin{quote}
    ``Oh, bosh, as Mr. Ruskin says. Sculpture, per se, is the simplest thing in the world. All you have to do is to take a big chunk of marble and a hammer and chisel, make up your mind what you are about to create and chip off all the marble you don't want.''-Paris Gaulois.
  \end{quote}
  
  {\tiny Source: \url{https://quoteinvestigator.com/2014/06/22/chip-away/}}
\end{frame}

\begin{frame}
\frametitle{Basic Process}
\begin{center}
\begin{tikzpicture}
\node[shape=rectangle,fill=pantone127-4] (problem) at (2,5) {Problem};
\node (human) at (2,4) {Human};
\node[shape=rectangle,fill=pantone127-4] (model) at (2,3) {Model};
\node (solver) at (2,2) {Constraint Solver/Search};
\node[shape=rectangle,fill=pantone157-8] (solution) at (2,1) {Solution};
\draw[-] (problem) -- (human);
\draw[->] (human) -- (model);
\draw[-] (model) -- (solver);
\draw[->] (solver) -- (solution);
\end{tikzpicture}
\end{center}
\end{frame}

\begin{frame}
\frametitle{More Realistic}
\begin{center}
\begin{tikzpicture}[xscale=4.0]
\node[shape=rectangle,fill=pantone127-4] (problem) at (2,5) {Problem};
\node (human) at (2,4) {Human};
\node[shape=rectangle,fill=pantone127-4] (model) at (2,3) {Model};
\node (solver) at (2,2) {Constraint Solver/Search};
\node[shape=rectangle,fill=pantone24-8] (hangs) at (1,1) {Hangs};
\node[shape=rectangle,fill=pantone157-8] (solution) at (2,1) {Solution};
\node[shape=rectangle,fill=pantone24-8] (wrong) at (3,1) {Wrong Solution};
\draw[-] (problem) -- (human);
\draw[->] (human) -- (model);
\draw[-] (model) -- (solver);
\draw[->] (solver) -- (hangs);
\draw[->] (solver) -- (solution);
\draw[->] (solver) -- (wrong);
\draw[->] (hangs) -- (1,4) -- (human);
\draw[->] (wrong) -- (3,4) -- (human);
\end{tikzpicture}
\end{center}
\end{frame}

\begin{frame}
\frametitle{Dual Role of Model}
\begin{itemize}
\item Allows Human to Express Problem
\begin{itemize}
\item Close to Problem Domain
\item Constraints as Abstractions
\end{itemize}
\item Allows Solver to Execute
\begin{itemize}
\item Variables as Communication Mechanism
\item Constraints as Algorithms
\end{itemize}
\end{itemize}
\end{frame}

\begin{frame}
\frametitle{Modelling Frameworks}
\begin{itemize}
\item \href{https://www.minizinc.org/}{MiniZinc} (NICTA, Monash University, Australia)
\item \href{https://github.com/eomahony/Numberjack}{NumberJack} (Insight, Ireland) 
\item \href{https://www-users.york.ac.uk/peter.nightingale/savilerow//}{EssencePrime/SavilleRow} (UK) 
\item \href{https://cpmpy.readthedocs.io/en/latest/}{CPMpy} (KU Leuven)
\item Allow use of multiple back-end solvers
\item Compile model into variants for each solver
\item A priori solver independent model(CP, MIP, SAT)
\end{itemize}
\end{frame} 

\begin{frame}
\frametitle{Framework Process}
\begin{center}
\begin{tikzpicture}[xscale=2]
\node[shape=rectangle,fill=pantone127-4] (problem) at (2.5,5) {Problem};
\node (human) at (2.5,4) {Human};
\node[shape=rectangle,fill=pantone127-4] (model) at (2.5,3) {Model};
\node (compile) at (2.5,2) {Compile/Reformulate};
\node (solver1) at (1,1) {CP};
\node (solver2) at (2,1) {MIP};
\node (solver3) at (3,1) {SAT};
\node (solver4) at (4,1) {Other};
\node[shape=rectangle,fill=pantone157-8] (solution1) at (1,0) {Solution};
\node[shape=rectangle,fill=pantone157-8] (solution2) at (2,0) {Solution};
\node[shape=rectangle,fill=pantone157-8] (solution3) at (3,0) {Solution};
\node[shape=rectangle,fill=pantone157-8] (solution4) at (4,0) {Solution};
\draw[-] (problem) -- (human);
\draw[->] (human) -- (model);
\draw[-] (model) -- (compile);
\draw[-] (compile) -- (solver1);
\draw[-] (compile) -- (solver2);
\draw[-] (compile) -- (solver3);
\draw[-] (compile) -- (solver4);
\draw[->] (solver1) -- (solution1);
\draw[->] (solver2) -- (solution2);
\draw[->] (solver3) -- (solution3);
\draw[->] (solver4) -- (solution4);
\end{tikzpicture}
\end{center}
\end{frame}

% \begin{frame}
  % \frametitle{Do It Now!}
  % \begin{itemize}
  % \item Download and install Minizinc
  % \item \url{https://www.minizinc.org/}
  % \end{itemize}
% \end{frame}


\begin{frame}
\frametitle{Why use Puzzles as Examples?}
\begin{itemize}
\item Easy to understand the problem
\item Solvable by hand without specialized knowledge
\item Possible to compare automated to manual solving process
\end{itemize}

\begin{quote}
The puzzle, though inanimate, is presented as a solvable problem without lasting negative consequences, a very low-risk low-reward situation. By being a puzzle, the object is attempting to convince the user that it must be completed.
\end{quote}
{\tiny Source: \href{https://www.everydayrhetoric.com/2020/03/the-rhetoric-of-puzzles/}{Every Day Rhetoric}}
\end{frame}


\mode<all>{
\part{Basic Constraint Propagation}
}
\begin{frame}
\partpage
\end{frame}
\mode<all>{
\input{../sendmore/body}
}

\mode<all>{
\part{Global Constraints}
}
\begin{frame}
\partpage
\end{frame}
\mode<all>{
\input{../sudoku/body}
}

\clearpage
\mode<all>{
\part{Customizing Search}
}
\begin{frame}
\partpage
\end{frame}
\mode<all>{
\input{../nqueen/body}
}


%% \clearpage
%% \mode<all>{
%%   \part{Problem Decomposition}
%% }
%% \begin{frame}
%%   \partpage
%% \end{frame}
%% \mode<all>{
%%   \input{../party/body}
%%   }

%% \clearpage
%% \mode<all>{
%%   \part{More Global Constraints}
%% }
%% \begin{frame}
%%   \partpage
%% \end{frame}
%% \mode<all>{
%%   \input{../car/body}
%%   }

%% \clearpage
%% \mode<all>{
%%   \part{Methodology}
%% }
%% \begin{frame}
%%   \partpage
%% \end{frame}
%% \mode<all>{
%%   \input{../methodology/body}
%%   }

\clearpage
\mode<all>{
\part{What is missing?}
}
\begin{frame}
\partpage
\end{frame}

\begin{frame}
\frametitle{Many Specialized Topics}
\begin{itemize}
\item How to design efficient core engine
\item Hybrids with LP/MIP tools
\item Hybrids with SAT
\item Symmetry breaking
\item Use of MDD/BDD to encode sets of solutions
\item High level modelling tools
\item Debugging/visualization
\end{itemize}
\end{frame}

\begin{frame}
\frametitle{Reformulation}
\begin{itemize}
\item Just because the user has modelled it this way, it doesn't mean we have to solve it that way
\begin{itemize}
\item Replace some constraint(s) by other, equivalent constraints
\item Because we don't have that constraint in our system
\item For performance
\end{itemize}
\end{itemize}
\end{frame}

\begin{frame}
\frametitle{Learning}
\begin{itemize}
\item While solving the problem we can learn how to strengthen the model/search 
\begin{itemize}
\item Understand which constraints/method contribute to propagation and change schedule
\item Learn no-good constraints by explaining failure
\item Adapt search strategy based on search experience
\end{itemize}
\end{itemize}
\end{frame}

%% \begin{frame}
%% \frametitle{Refined Process}
%% \begin{center}
%% \begin{tikzpicture}
%% \node[shape=rectangle,fill=pantone127-4] (problem) at (3,7) {Problem};
%% \node (human) at (3,6) {Human};
%% \node[shape=rectangle,fill=pantone127-4] (model) at (3,5) {User Model};
%% \node (reformulation) at (3,4) {Reformulation};
%% \node[shape=rectangle,fill=pantone127-4] (implementation) at (3,3) {Implementation Model};
%% \node (solver) at (3,2) {Constraint Solver/Search};
%% \node[shape=rectangle,fill=pantone157-8] (solution) at (3,1) {Solution};
%% \draw[-] (problem) -- (human);
%% \draw[->] (human) -- (model);
%% \draw[-] (model) -- (reformulation);
%% \draw[->] (reformulation) -- (implementation);
%% \draw[-] (implementation) -- (solver);
%% \draw[->] (solver) -- (0,2) -- (0,3) -- (implementation);
%% \draw[->] (solver) -- (solution);
%% \end{tikzpicture}
%% \end{center}
%% \end{frame}

%\begin{frame}
%\frametitle{What is CP actually used for?}
%\url{http://hsimonis.wordpress.com}
%\includegraphics[width=10cm]{../introduction/images/blog}
%\end{frame}

%% \begin{frame}
%% \frametitle{Exercises}
%% \begin{itemize}
%% \item Install Constraint Solver of your Choice
%% \item MiniZinc (MiningZinc) will be used in later sessions
%% \item Run Sendmoremoney, Sudoku, Queens examples
%% \item Can you choose which propagator to use for alldifferent?
%% \item Can you use different search strategies?
%% \end{itemize}
%% \end{frame}

\begin{frame}
\frametitle{More Learning Resources}
\begin{itemize}
\item Survey of Methods, Resources, and Formats for Teaching Constraint Programming
\begin{itemize}
\item by Tejas Santanam, Helmut Simonis
\item \url{https://doi.org/10.48550/arXiv.2403.12717}
\item Based on survey of community for WTCP 2023
\item \url{https://hsimonis.github.io/WTCP2023/}
\end{itemize}
\end{itemize}
\end{frame}


%%   }

%% \clearpage
%% \mode<all>{
%%   \part{Methodology}
%% }
%% \begin{frame}
%%   \partpage
%% \end{frame}
%% \mode<all>{
%%   \begin{frame}
\frametitle{Objectives}
\begin{itemize}
\item Overview of Core Constraint Programming
\item Three Main Concepts
\begin{itemize}
\item Constraint Propagation
\item Global Constraints
\item Customizing Search
\end{itemize}
%\item Get Some Experience with MiniZinc
\item Topics will be treated in more detail in later parts of the school
\item Based on Examples, not Formal Description
\end{itemize}
\end{frame}

\begin{frame}
\frametitle{Outline}
\begin{itemize}
\item Why Constraint Programming?
\item Constraint Propagation
\item Global Constraints
\item Customizing Search
%\item Choices in Modelling
%\item Decomposition
%\item More Global Constraints
%\item Methodology
%\item What is missing?
\end{itemize}
\end{frame}

% \begin{frame}
  % \frametitle{Using MiniZinc IDE}
  % \begin{itemize}
  % \item Developed in the Australian NICTA project
  % \item Maintained by Monash University
  % \item Modelling tool with multiple back-end solvers
  % \item Available from \url{https://www.minizinc.org/}
  % \end{itemize}
% \end{frame}


% \begin{frame}
% \frametitle{Examples in ECLiPSe}
% \begin{itemize}
% \item Open sourced constraint programming language
% \item Development goes back to 1985
% \item ECRC, ICL, IC-Parc, PTL, Cisco
% \item \url{https://eclipseclp.org/}
% \item Specialities
% \begin{itemize}
% \item Develop new solvers for specific domains
% \item Integration with MIP
% \end{itemize}
% \item Not included in bundled MiniZinc IDE
% \item Specialized visualization tools used here
  % \begin{itemize}
  % \item CP-Viz, Simonis et al. 2010
    % \end{itemize}
% \end{itemize}
% \end{frame}

\begin{frame}
\frametitle{Tutorial Based on ECLiPSe ELearning Course}
\begin{itemize}
\item Self-study course in constraint programming
\item Supported by Cisco Systems and Silicon Valley Community Foundation
\item Multi-media format, video lectures, slides, handout etc
\item {\small \url{https://eclipseclp.org/ELearning/index.html}}
\end{itemize}
\end{frame}

\begin{frame}
\frametitle{Also Part of CRT-AI Constraint Week}
\begin{itemize}
\item Annual one week course on CP and Optimization in Ireland
\item Part of national training program for PhD students in AI
\item \url{https://www.crt-ai.ie/}
\end{itemize}
\end{frame}



\begin{frame}
\frametitle{Constraint Programming - in a nutshell}
\begin{itemize}
\item Declarative description of problems with
\begin{itemize}
\item {\em Variables} which range over (finite) sets of values
\item {\em Constraints} over subsets of variables which restrict possible value combinations
\item A {\em solution} is a value assignment which satisfies all constraints
\end{itemize}

\item Constraint propagation/reasoning
\begin{itemize}
\item Removing inconsistent values for variables
\item Detect failure if constraint can not be satisfied
\item Interaction of constraints via shared variables
\item Incomplete
\end{itemize}

\item Search
\begin{itemize}
\item User controlled assignment of values to variables
\item Each step triggers constraint propagation 
\end{itemize}
\item Different domains require/allow different methods
\end{itemize}
\end{frame}

\begin{frame}
  \frametitle{Constraint Programming is Different}
  \begin{itemize}
  \item Declarative Programming
    \begin{itemize}
    \item Concentrate on what you want
      \item Not how to get there
      \item Program != Algorithm
      \item Program = Model
    \end{itemize}
    \item Applied to Combinatorial Problems
      \begin{itemize}
        \item No complete polynomial algorithms known (exist?)
        \item CP less ad-hoc than heuristics
        \item Models can evolve
  \end{itemize}
  \end{itemize}
  \end{frame}
    
\begin{frame}
  \frametitle{A Subtractive Process}
  \begin{textblock}{4}(8,-3)
    \includegraphics[width=4cm]{../introduction/images/stages}
  \end{textblock}
  \vfill
  \begin{quote}
    ``Oh, bosh, as Mr. Ruskin says. Sculpture, per se, is the simplest thing in the world. All you have to do is to take a big chunk of marble and a hammer and chisel, make up your mind what you are about to create and chip off all the marble you don't want.''-Paris Gaulois.
  \end{quote}
  
  {\tiny Source: \url{https://quoteinvestigator.com/2014/06/22/chip-away/}}
\end{frame}

\begin{frame}
\frametitle{Basic Process}
\begin{center}
\begin{tikzpicture}
\node[shape=rectangle,fill=pantone127-4] (problem) at (2,5) {Problem};
\node (human) at (2,4) {Human};
\node[shape=rectangle,fill=pantone127-4] (model) at (2,3) {Model};
\node (solver) at (2,2) {Constraint Solver/Search};
\node[shape=rectangle,fill=pantone157-8] (solution) at (2,1) {Solution};
\draw[-] (problem) -- (human);
\draw[->] (human) -- (model);
\draw[-] (model) -- (solver);
\draw[->] (solver) -- (solution);
\end{tikzpicture}
\end{center}
\end{frame}

\begin{frame}
\frametitle{More Realistic}
\begin{center}
\begin{tikzpicture}[xscale=4.0]
\node[shape=rectangle,fill=pantone127-4] (problem) at (2,5) {Problem};
\node (human) at (2,4) {Human};
\node[shape=rectangle,fill=pantone127-4] (model) at (2,3) {Model};
\node (solver) at (2,2) {Constraint Solver/Search};
\node[shape=rectangle,fill=pantone24-8] (hangs) at (1,1) {Hangs};
\node[shape=rectangle,fill=pantone157-8] (solution) at (2,1) {Solution};
\node[shape=rectangle,fill=pantone24-8] (wrong) at (3,1) {Wrong Solution};
\draw[-] (problem) -- (human);
\draw[->] (human) -- (model);
\draw[-] (model) -- (solver);
\draw[->] (solver) -- (hangs);
\draw[->] (solver) -- (solution);
\draw[->] (solver) -- (wrong);
\draw[->] (hangs) -- (1,4) -- (human);
\draw[->] (wrong) -- (3,4) -- (human);
\end{tikzpicture}
\end{center}
\end{frame}

\begin{frame}
\frametitle{Dual Role of Model}
\begin{itemize}
\item Allows Human to Express Problem
\begin{itemize}
\item Close to Problem Domain
\item Constraints as Abstractions
\end{itemize}
\item Allows Solver to Execute
\begin{itemize}
\item Variables as Communication Mechanism
\item Constraints as Algorithms
\end{itemize}
\end{itemize}
\end{frame}

\begin{frame}
\frametitle{Modelling Frameworks}
\begin{itemize}
\item \href{https://www.minizinc.org/}{MiniZinc} (NICTA, Monash University, Australia)
\item \href{https://github.com/eomahony/Numberjack}{NumberJack} (Insight, Ireland) 
\item \href{https://www-users.york.ac.uk/peter.nightingale/savilerow//}{EssencePrime/SavilleRow} (UK) 
\item \href{https://cpmpy.readthedocs.io/en/latest/}{CPMpy} (KU Leuven)
\item Allow use of multiple back-end solvers
\item Compile model into variants for each solver
\item A priori solver independent model(CP, MIP, SAT)
\end{itemize}
\end{frame} 

\begin{frame}
\frametitle{Framework Process}
\begin{center}
\begin{tikzpicture}[xscale=2]
\node[shape=rectangle,fill=pantone127-4] (problem) at (2.5,5) {Problem};
\node (human) at (2.5,4) {Human};
\node[shape=rectangle,fill=pantone127-4] (model) at (2.5,3) {Model};
\node (compile) at (2.5,2) {Compile/Reformulate};
\node (solver1) at (1,1) {CP};
\node (solver2) at (2,1) {MIP};
\node (solver3) at (3,1) {SAT};
\node (solver4) at (4,1) {Other};
\node[shape=rectangle,fill=pantone157-8] (solution1) at (1,0) {Solution};
\node[shape=rectangle,fill=pantone157-8] (solution2) at (2,0) {Solution};
\node[shape=rectangle,fill=pantone157-8] (solution3) at (3,0) {Solution};
\node[shape=rectangle,fill=pantone157-8] (solution4) at (4,0) {Solution};
\draw[-] (problem) -- (human);
\draw[->] (human) -- (model);
\draw[-] (model) -- (compile);
\draw[-] (compile) -- (solver1);
\draw[-] (compile) -- (solver2);
\draw[-] (compile) -- (solver3);
\draw[-] (compile) -- (solver4);
\draw[->] (solver1) -- (solution1);
\draw[->] (solver2) -- (solution2);
\draw[->] (solver3) -- (solution3);
\draw[->] (solver4) -- (solution4);
\end{tikzpicture}
\end{center}
\end{frame}

% \begin{frame}
  % \frametitle{Do It Now!}
  % \begin{itemize}
  % \item Download and install Minizinc
  % \item \url{https://www.minizinc.org/}
  % \end{itemize}
% \end{frame}


\begin{frame}
\frametitle{Why use Puzzles as Examples?}
\begin{itemize}
\item Easy to understand the problem
\item Solvable by hand without specialized knowledge
\item Possible to compare automated to manual solving process
\end{itemize}

\begin{quote}
The puzzle, though inanimate, is presented as a solvable problem without lasting negative consequences, a very low-risk low-reward situation. By being a puzzle, the object is attempting to convince the user that it must be completed.
\end{quote}
{\tiny Source: \href{https://www.everydayrhetoric.com/2020/03/the-rhetoric-of-puzzles/}{Every Day Rhetoric}}
\end{frame}


\mode<all>{
\part{Basic Constraint Propagation}
}
\begin{frame}
\partpage
\end{frame}
\mode<all>{
\input{../sendmore/body}
}

\mode<all>{
\part{Global Constraints}
}
\begin{frame}
\partpage
\end{frame}
\mode<all>{
\input{../sudoku/body}
}

\clearpage
\mode<all>{
\part{Customizing Search}
}
\begin{frame}
\partpage
\end{frame}
\mode<all>{
\input{../nqueen/body}
}


%% \clearpage
%% \mode<all>{
%%   \part{Problem Decomposition}
%% }
%% \begin{frame}
%%   \partpage
%% \end{frame}
%% \mode<all>{
%%   \input{../party/body}
%%   }

%% \clearpage
%% \mode<all>{
%%   \part{More Global Constraints}
%% }
%% \begin{frame}
%%   \partpage
%% \end{frame}
%% \mode<all>{
%%   \input{../car/body}
%%   }

%% \clearpage
%% \mode<all>{
%%   \part{Methodology}
%% }
%% \begin{frame}
%%   \partpage
%% \end{frame}
%% \mode<all>{
%%   \input{../methodology/body}
%%   }

\clearpage
\mode<all>{
\part{What is missing?}
}
\begin{frame}
\partpage
\end{frame}

\begin{frame}
\frametitle{Many Specialized Topics}
\begin{itemize}
\item How to design efficient core engine
\item Hybrids with LP/MIP tools
\item Hybrids with SAT
\item Symmetry breaking
\item Use of MDD/BDD to encode sets of solutions
\item High level modelling tools
\item Debugging/visualization
\end{itemize}
\end{frame}

\begin{frame}
\frametitle{Reformulation}
\begin{itemize}
\item Just because the user has modelled it this way, it doesn't mean we have to solve it that way
\begin{itemize}
\item Replace some constraint(s) by other, equivalent constraints
\item Because we don't have that constraint in our system
\item For performance
\end{itemize}
\end{itemize}
\end{frame}

\begin{frame}
\frametitle{Learning}
\begin{itemize}
\item While solving the problem we can learn how to strengthen the model/search 
\begin{itemize}
\item Understand which constraints/method contribute to propagation and change schedule
\item Learn no-good constraints by explaining failure
\item Adapt search strategy based on search experience
\end{itemize}
\end{itemize}
\end{frame}

%% \begin{frame}
%% \frametitle{Refined Process}
%% \begin{center}
%% \begin{tikzpicture}
%% \node[shape=rectangle,fill=pantone127-4] (problem) at (3,7) {Problem};
%% \node (human) at (3,6) {Human};
%% \node[shape=rectangle,fill=pantone127-4] (model) at (3,5) {User Model};
%% \node (reformulation) at (3,4) {Reformulation};
%% \node[shape=rectangle,fill=pantone127-4] (implementation) at (3,3) {Implementation Model};
%% \node (solver) at (3,2) {Constraint Solver/Search};
%% \node[shape=rectangle,fill=pantone157-8] (solution) at (3,1) {Solution};
%% \draw[-] (problem) -- (human);
%% \draw[->] (human) -- (model);
%% \draw[-] (model) -- (reformulation);
%% \draw[->] (reformulation) -- (implementation);
%% \draw[-] (implementation) -- (solver);
%% \draw[->] (solver) -- (0,2) -- (0,3) -- (implementation);
%% \draw[->] (solver) -- (solution);
%% \end{tikzpicture}
%% \end{center}
%% \end{frame}

%\begin{frame}
%\frametitle{What is CP actually used for?}
%\url{http://hsimonis.wordpress.com}
%\includegraphics[width=10cm]{../introduction/images/blog}
%\end{frame}

%% \begin{frame}
%% \frametitle{Exercises}
%% \begin{itemize}
%% \item Install Constraint Solver of your Choice
%% \item MiniZinc (MiningZinc) will be used in later sessions
%% \item Run Sendmoremoney, Sudoku, Queens examples
%% \item Can you choose which propagator to use for alldifferent?
%% \item Can you use different search strategies?
%% \end{itemize}
%% \end{frame}

\begin{frame}
\frametitle{More Learning Resources}
\begin{itemize}
\item Survey of Methods, Resources, and Formats for Teaching Constraint Programming
\begin{itemize}
\item by Tejas Santanam, Helmut Simonis
\item \url{https://doi.org/10.48550/arXiv.2403.12717}
\item Based on survey of community for WTCP 2023
\item \url{https://hsimonis.github.io/WTCP2023/}
\end{itemize}
\end{itemize}
\end{frame}


%%   }

\clearpage
\mode<all>{
\part{What is missing?}
}
\begin{frame}
\partpage
\end{frame}

\begin{frame}
\frametitle{Many Specialized Topics}
\begin{itemize}
\item How to design efficient core engine
\item Hybrids with LP/MIP tools
\item Hybrids with SAT
\item Symmetry breaking
\item Use of MDD/BDD to encode sets of solutions
\item High level modelling tools
\item Debugging/visualization
\end{itemize}
\end{frame}

\begin{frame}
\frametitle{Reformulation}
\begin{itemize}
\item Just because the user has modelled it this way, it doesn't mean we have to solve it that way
\begin{itemize}
\item Replace some constraint(s) by other, equivalent constraints
\item Because we don't have that constraint in our system
\item For performance
\end{itemize}
\end{itemize}
\end{frame}

\begin{frame}
\frametitle{Learning}
\begin{itemize}
\item While solving the problem we can learn how to strengthen the model/search 
\begin{itemize}
\item Understand which constraints/method contribute to propagation and change schedule
\item Learn no-good constraints by explaining failure
\item Adapt search strategy based on search experience
\end{itemize}
\end{itemize}
\end{frame}

%% \begin{frame}
%% \frametitle{Refined Process}
%% \begin{center}
%% \begin{tikzpicture}
%% \node[shape=rectangle,fill=pantone127-4] (problem) at (3,7) {Problem};
%% \node (human) at (3,6) {Human};
%% \node[shape=rectangle,fill=pantone127-4] (model) at (3,5) {User Model};
%% \node (reformulation) at (3,4) {Reformulation};
%% \node[shape=rectangle,fill=pantone127-4] (implementation) at (3,3) {Implementation Model};
%% \node (solver) at (3,2) {Constraint Solver/Search};
%% \node[shape=rectangle,fill=pantone157-8] (solution) at (3,1) {Solution};
%% \draw[-] (problem) -- (human);
%% \draw[->] (human) -- (model);
%% \draw[-] (model) -- (reformulation);
%% \draw[->] (reformulation) -- (implementation);
%% \draw[-] (implementation) -- (solver);
%% \draw[->] (solver) -- (0,2) -- (0,3) -- (implementation);
%% \draw[->] (solver) -- (solution);
%% \end{tikzpicture}
%% \end{center}
%% \end{frame}

%\begin{frame}
%\frametitle{What is CP actually used for?}
%\url{http://hsimonis.wordpress.com}
%\includegraphics[width=10cm]{../introduction/images/blog}
%\end{frame}

%% \begin{frame}
%% \frametitle{Exercises}
%% \begin{itemize}
%% \item Install Constraint Solver of your Choice
%% \item MiniZinc (MiningZinc) will be used in later sessions
%% \item Run Sendmoremoney, Sudoku, Queens examples
%% \item Can you choose which propagator to use for alldifferent?
%% \item Can you use different search strategies?
%% \end{itemize}
%% \end{frame}

\begin{frame}
\frametitle{More Learning Resources}
\begin{itemize}
\item Survey of Methods, Resources, and Formats for Teaching Constraint Programming
\begin{itemize}
\item by Tejas Santanam, Helmut Simonis
\item \url{https://doi.org/10.48550/arXiv.2403.12717}
\item Based on survey of community for WTCP 2023
\item \url{https://hsimonis.github.io/WTCP2023/}
\end{itemize}
\end{itemize}
\end{frame}


}

\clearpage
\mode<all>{
\part{Customizing Search}
}
\begin{frame}
\partpage
\end{frame}
\mode<all>{
\begin{frame}
\frametitle{Objectives}
\begin{itemize}
\item Overview of Core Constraint Programming
\item Three Main Concepts
\begin{itemize}
\item Constraint Propagation
\item Global Constraints
\item Customizing Search
\end{itemize}
%\item Get Some Experience with MiniZinc
\item Topics will be treated in more detail in later parts of the school
\item Based on Examples, not Formal Description
\end{itemize}
\end{frame}

\begin{frame}
\frametitle{Outline}
\begin{itemize}
\item Why Constraint Programming?
\item Constraint Propagation
\item Global Constraints
\item Customizing Search
%\item Choices in Modelling
%\item Decomposition
%\item More Global Constraints
%\item Methodology
%\item What is missing?
\end{itemize}
\end{frame}

% \begin{frame}
  % \frametitle{Using MiniZinc IDE}
  % \begin{itemize}
  % \item Developed in the Australian NICTA project
  % \item Maintained by Monash University
  % \item Modelling tool with multiple back-end solvers
  % \item Available from \url{https://www.minizinc.org/}
  % \end{itemize}
% \end{frame}


% \begin{frame}
% \frametitle{Examples in ECLiPSe}
% \begin{itemize}
% \item Open sourced constraint programming language
% \item Development goes back to 1985
% \item ECRC, ICL, IC-Parc, PTL, Cisco
% \item \url{https://eclipseclp.org/}
% \item Specialities
% \begin{itemize}
% \item Develop new solvers for specific domains
% \item Integration with MIP
% \end{itemize}
% \item Not included in bundled MiniZinc IDE
% \item Specialized visualization tools used here
  % \begin{itemize}
  % \item CP-Viz, Simonis et al. 2010
    % \end{itemize}
% \end{itemize}
% \end{frame}

\begin{frame}
\frametitle{Tutorial Based on ECLiPSe ELearning Course}
\begin{itemize}
\item Self-study course in constraint programming
\item Supported by Cisco Systems and Silicon Valley Community Foundation
\item Multi-media format, video lectures, slides, handout etc
\item {\small \url{https://eclipseclp.org/ELearning/index.html}}
\end{itemize}
\end{frame}

\begin{frame}
\frametitle{Also Part of CRT-AI Constraint Week}
\begin{itemize}
\item Annual one week course on CP and Optimization in Ireland
\item Part of national training program for PhD students in AI
\item \url{https://www.crt-ai.ie/}
\end{itemize}
\end{frame}



\begin{frame}
\frametitle{Constraint Programming - in a nutshell}
\begin{itemize}
\item Declarative description of problems with
\begin{itemize}
\item {\em Variables} which range over (finite) sets of values
\item {\em Constraints} over subsets of variables which restrict possible value combinations
\item A {\em solution} is a value assignment which satisfies all constraints
\end{itemize}

\item Constraint propagation/reasoning
\begin{itemize}
\item Removing inconsistent values for variables
\item Detect failure if constraint can not be satisfied
\item Interaction of constraints via shared variables
\item Incomplete
\end{itemize}

\item Search
\begin{itemize}
\item User controlled assignment of values to variables
\item Each step triggers constraint propagation 
\end{itemize}
\item Different domains require/allow different methods
\end{itemize}
\end{frame}

\begin{frame}
  \frametitle{Constraint Programming is Different}
  \begin{itemize}
  \item Declarative Programming
    \begin{itemize}
    \item Concentrate on what you want
      \item Not how to get there
      \item Program != Algorithm
      \item Program = Model
    \end{itemize}
    \item Applied to Combinatorial Problems
      \begin{itemize}
        \item No complete polynomial algorithms known (exist?)
        \item CP less ad-hoc than heuristics
        \item Models can evolve
  \end{itemize}
  \end{itemize}
  \end{frame}
    
\begin{frame}
  \frametitle{A Subtractive Process}
  \begin{textblock}{4}(8,-3)
    \includegraphics[width=4cm]{../introduction/images/stages}
  \end{textblock}
  \vfill
  \begin{quote}
    ``Oh, bosh, as Mr. Ruskin says. Sculpture, per se, is the simplest thing in the world. All you have to do is to take a big chunk of marble and a hammer and chisel, make up your mind what you are about to create and chip off all the marble you don't want.''-Paris Gaulois.
  \end{quote}
  
  {\tiny Source: \url{https://quoteinvestigator.com/2014/06/22/chip-away/}}
\end{frame}

\begin{frame}
\frametitle{Basic Process}
\begin{center}
\begin{tikzpicture}
\node[shape=rectangle,fill=pantone127-4] (problem) at (2,5) {Problem};
\node (human) at (2,4) {Human};
\node[shape=rectangle,fill=pantone127-4] (model) at (2,3) {Model};
\node (solver) at (2,2) {Constraint Solver/Search};
\node[shape=rectangle,fill=pantone157-8] (solution) at (2,1) {Solution};
\draw[-] (problem) -- (human);
\draw[->] (human) -- (model);
\draw[-] (model) -- (solver);
\draw[->] (solver) -- (solution);
\end{tikzpicture}
\end{center}
\end{frame}

\begin{frame}
\frametitle{More Realistic}
\begin{center}
\begin{tikzpicture}[xscale=4.0]
\node[shape=rectangle,fill=pantone127-4] (problem) at (2,5) {Problem};
\node (human) at (2,4) {Human};
\node[shape=rectangle,fill=pantone127-4] (model) at (2,3) {Model};
\node (solver) at (2,2) {Constraint Solver/Search};
\node[shape=rectangle,fill=pantone24-8] (hangs) at (1,1) {Hangs};
\node[shape=rectangle,fill=pantone157-8] (solution) at (2,1) {Solution};
\node[shape=rectangle,fill=pantone24-8] (wrong) at (3,1) {Wrong Solution};
\draw[-] (problem) -- (human);
\draw[->] (human) -- (model);
\draw[-] (model) -- (solver);
\draw[->] (solver) -- (hangs);
\draw[->] (solver) -- (solution);
\draw[->] (solver) -- (wrong);
\draw[->] (hangs) -- (1,4) -- (human);
\draw[->] (wrong) -- (3,4) -- (human);
\end{tikzpicture}
\end{center}
\end{frame}

\begin{frame}
\frametitle{Dual Role of Model}
\begin{itemize}
\item Allows Human to Express Problem
\begin{itemize}
\item Close to Problem Domain
\item Constraints as Abstractions
\end{itemize}
\item Allows Solver to Execute
\begin{itemize}
\item Variables as Communication Mechanism
\item Constraints as Algorithms
\end{itemize}
\end{itemize}
\end{frame}

\begin{frame}
\frametitle{Modelling Frameworks}
\begin{itemize}
\item \href{https://www.minizinc.org/}{MiniZinc} (NICTA, Monash University, Australia)
\item \href{https://github.com/eomahony/Numberjack}{NumberJack} (Insight, Ireland) 
\item \href{https://www-users.york.ac.uk/peter.nightingale/savilerow//}{EssencePrime/SavilleRow} (UK) 
\item \href{https://cpmpy.readthedocs.io/en/latest/}{CPMpy} (KU Leuven)
\item Allow use of multiple back-end solvers
\item Compile model into variants for each solver
\item A priori solver independent model(CP, MIP, SAT)
\end{itemize}
\end{frame} 

\begin{frame}
\frametitle{Framework Process}
\begin{center}
\begin{tikzpicture}[xscale=2]
\node[shape=rectangle,fill=pantone127-4] (problem) at (2.5,5) {Problem};
\node (human) at (2.5,4) {Human};
\node[shape=rectangle,fill=pantone127-4] (model) at (2.5,3) {Model};
\node (compile) at (2.5,2) {Compile/Reformulate};
\node (solver1) at (1,1) {CP};
\node (solver2) at (2,1) {MIP};
\node (solver3) at (3,1) {SAT};
\node (solver4) at (4,1) {Other};
\node[shape=rectangle,fill=pantone157-8] (solution1) at (1,0) {Solution};
\node[shape=rectangle,fill=pantone157-8] (solution2) at (2,0) {Solution};
\node[shape=rectangle,fill=pantone157-8] (solution3) at (3,0) {Solution};
\node[shape=rectangle,fill=pantone157-8] (solution4) at (4,0) {Solution};
\draw[-] (problem) -- (human);
\draw[->] (human) -- (model);
\draw[-] (model) -- (compile);
\draw[-] (compile) -- (solver1);
\draw[-] (compile) -- (solver2);
\draw[-] (compile) -- (solver3);
\draw[-] (compile) -- (solver4);
\draw[->] (solver1) -- (solution1);
\draw[->] (solver2) -- (solution2);
\draw[->] (solver3) -- (solution3);
\draw[->] (solver4) -- (solution4);
\end{tikzpicture}
\end{center}
\end{frame}

% \begin{frame}
  % \frametitle{Do It Now!}
  % \begin{itemize}
  % \item Download and install Minizinc
  % \item \url{https://www.minizinc.org/}
  % \end{itemize}
% \end{frame}


\begin{frame}
\frametitle{Why use Puzzles as Examples?}
\begin{itemize}
\item Easy to understand the problem
\item Solvable by hand without specialized knowledge
\item Possible to compare automated to manual solving process
\end{itemize}

\begin{quote}
The puzzle, though inanimate, is presented as a solvable problem without lasting negative consequences, a very low-risk low-reward situation. By being a puzzle, the object is attempting to convince the user that it must be completed.
\end{quote}
{\tiny Source: \href{https://www.everydayrhetoric.com/2020/03/the-rhetoric-of-puzzles/}{Every Day Rhetoric}}
\end{frame}


\mode<all>{
\part{Basic Constraint Propagation}
}
\begin{frame}
\partpage
\end{frame}
\mode<all>{
\begin{frame}
\frametitle{Objectives}
\begin{itemize}
\item Overview of Core Constraint Programming
\item Three Main Concepts
\begin{itemize}
\item Constraint Propagation
\item Global Constraints
\item Customizing Search
\end{itemize}
%\item Get Some Experience with MiniZinc
\item Topics will be treated in more detail in later parts of the school
\item Based on Examples, not Formal Description
\end{itemize}
\end{frame}

\begin{frame}
\frametitle{Outline}
\begin{itemize}
\item Why Constraint Programming?
\item Constraint Propagation
\item Global Constraints
\item Customizing Search
%\item Choices in Modelling
%\item Decomposition
%\item More Global Constraints
%\item Methodology
%\item What is missing?
\end{itemize}
\end{frame}

% \begin{frame}
  % \frametitle{Using MiniZinc IDE}
  % \begin{itemize}
  % \item Developed in the Australian NICTA project
  % \item Maintained by Monash University
  % \item Modelling tool with multiple back-end solvers
  % \item Available from \url{https://www.minizinc.org/}
  % \end{itemize}
% \end{frame}


% \begin{frame}
% \frametitle{Examples in ECLiPSe}
% \begin{itemize}
% \item Open sourced constraint programming language
% \item Development goes back to 1985
% \item ECRC, ICL, IC-Parc, PTL, Cisco
% \item \url{https://eclipseclp.org/}
% \item Specialities
% \begin{itemize}
% \item Develop new solvers for specific domains
% \item Integration with MIP
% \end{itemize}
% \item Not included in bundled MiniZinc IDE
% \item Specialized visualization tools used here
  % \begin{itemize}
  % \item CP-Viz, Simonis et al. 2010
    % \end{itemize}
% \end{itemize}
% \end{frame}

\begin{frame}
\frametitle{Tutorial Based on ECLiPSe ELearning Course}
\begin{itemize}
\item Self-study course in constraint programming
\item Supported by Cisco Systems and Silicon Valley Community Foundation
\item Multi-media format, video lectures, slides, handout etc
\item {\small \url{https://eclipseclp.org/ELearning/index.html}}
\end{itemize}
\end{frame}

\begin{frame}
\frametitle{Also Part of CRT-AI Constraint Week}
\begin{itemize}
\item Annual one week course on CP and Optimization in Ireland
\item Part of national training program for PhD students in AI
\item \url{https://www.crt-ai.ie/}
\end{itemize}
\end{frame}



\begin{frame}
\frametitle{Constraint Programming - in a nutshell}
\begin{itemize}
\item Declarative description of problems with
\begin{itemize}
\item {\em Variables} which range over (finite) sets of values
\item {\em Constraints} over subsets of variables which restrict possible value combinations
\item A {\em solution} is a value assignment which satisfies all constraints
\end{itemize}

\item Constraint propagation/reasoning
\begin{itemize}
\item Removing inconsistent values for variables
\item Detect failure if constraint can not be satisfied
\item Interaction of constraints via shared variables
\item Incomplete
\end{itemize}

\item Search
\begin{itemize}
\item User controlled assignment of values to variables
\item Each step triggers constraint propagation 
\end{itemize}
\item Different domains require/allow different methods
\end{itemize}
\end{frame}

\begin{frame}
  \frametitle{Constraint Programming is Different}
  \begin{itemize}
  \item Declarative Programming
    \begin{itemize}
    \item Concentrate on what you want
      \item Not how to get there
      \item Program != Algorithm
      \item Program = Model
    \end{itemize}
    \item Applied to Combinatorial Problems
      \begin{itemize}
        \item No complete polynomial algorithms known (exist?)
        \item CP less ad-hoc than heuristics
        \item Models can evolve
  \end{itemize}
  \end{itemize}
  \end{frame}
    
\begin{frame}
  \frametitle{A Subtractive Process}
  \begin{textblock}{4}(8,-3)
    \includegraphics[width=4cm]{../introduction/images/stages}
  \end{textblock}
  \vfill
  \begin{quote}
    ``Oh, bosh, as Mr. Ruskin says. Sculpture, per se, is the simplest thing in the world. All you have to do is to take a big chunk of marble and a hammer and chisel, make up your mind what you are about to create and chip off all the marble you don't want.''-Paris Gaulois.
  \end{quote}
  
  {\tiny Source: \url{https://quoteinvestigator.com/2014/06/22/chip-away/}}
\end{frame}

\begin{frame}
\frametitle{Basic Process}
\begin{center}
\begin{tikzpicture}
\node[shape=rectangle,fill=pantone127-4] (problem) at (2,5) {Problem};
\node (human) at (2,4) {Human};
\node[shape=rectangle,fill=pantone127-4] (model) at (2,3) {Model};
\node (solver) at (2,2) {Constraint Solver/Search};
\node[shape=rectangle,fill=pantone157-8] (solution) at (2,1) {Solution};
\draw[-] (problem) -- (human);
\draw[->] (human) -- (model);
\draw[-] (model) -- (solver);
\draw[->] (solver) -- (solution);
\end{tikzpicture}
\end{center}
\end{frame}

\begin{frame}
\frametitle{More Realistic}
\begin{center}
\begin{tikzpicture}[xscale=4.0]
\node[shape=rectangle,fill=pantone127-4] (problem) at (2,5) {Problem};
\node (human) at (2,4) {Human};
\node[shape=rectangle,fill=pantone127-4] (model) at (2,3) {Model};
\node (solver) at (2,2) {Constraint Solver/Search};
\node[shape=rectangle,fill=pantone24-8] (hangs) at (1,1) {Hangs};
\node[shape=rectangle,fill=pantone157-8] (solution) at (2,1) {Solution};
\node[shape=rectangle,fill=pantone24-8] (wrong) at (3,1) {Wrong Solution};
\draw[-] (problem) -- (human);
\draw[->] (human) -- (model);
\draw[-] (model) -- (solver);
\draw[->] (solver) -- (hangs);
\draw[->] (solver) -- (solution);
\draw[->] (solver) -- (wrong);
\draw[->] (hangs) -- (1,4) -- (human);
\draw[->] (wrong) -- (3,4) -- (human);
\end{tikzpicture}
\end{center}
\end{frame}

\begin{frame}
\frametitle{Dual Role of Model}
\begin{itemize}
\item Allows Human to Express Problem
\begin{itemize}
\item Close to Problem Domain
\item Constraints as Abstractions
\end{itemize}
\item Allows Solver to Execute
\begin{itemize}
\item Variables as Communication Mechanism
\item Constraints as Algorithms
\end{itemize}
\end{itemize}
\end{frame}

\begin{frame}
\frametitle{Modelling Frameworks}
\begin{itemize}
\item \href{https://www.minizinc.org/}{MiniZinc} (NICTA, Monash University, Australia)
\item \href{https://github.com/eomahony/Numberjack}{NumberJack} (Insight, Ireland) 
\item \href{https://www-users.york.ac.uk/peter.nightingale/savilerow//}{EssencePrime/SavilleRow} (UK) 
\item \href{https://cpmpy.readthedocs.io/en/latest/}{CPMpy} (KU Leuven)
\item Allow use of multiple back-end solvers
\item Compile model into variants for each solver
\item A priori solver independent model(CP, MIP, SAT)
\end{itemize}
\end{frame} 

\begin{frame}
\frametitle{Framework Process}
\begin{center}
\begin{tikzpicture}[xscale=2]
\node[shape=rectangle,fill=pantone127-4] (problem) at (2.5,5) {Problem};
\node (human) at (2.5,4) {Human};
\node[shape=rectangle,fill=pantone127-4] (model) at (2.5,3) {Model};
\node (compile) at (2.5,2) {Compile/Reformulate};
\node (solver1) at (1,1) {CP};
\node (solver2) at (2,1) {MIP};
\node (solver3) at (3,1) {SAT};
\node (solver4) at (4,1) {Other};
\node[shape=rectangle,fill=pantone157-8] (solution1) at (1,0) {Solution};
\node[shape=rectangle,fill=pantone157-8] (solution2) at (2,0) {Solution};
\node[shape=rectangle,fill=pantone157-8] (solution3) at (3,0) {Solution};
\node[shape=rectangle,fill=pantone157-8] (solution4) at (4,0) {Solution};
\draw[-] (problem) -- (human);
\draw[->] (human) -- (model);
\draw[-] (model) -- (compile);
\draw[-] (compile) -- (solver1);
\draw[-] (compile) -- (solver2);
\draw[-] (compile) -- (solver3);
\draw[-] (compile) -- (solver4);
\draw[->] (solver1) -- (solution1);
\draw[->] (solver2) -- (solution2);
\draw[->] (solver3) -- (solution3);
\draw[->] (solver4) -- (solution4);
\end{tikzpicture}
\end{center}
\end{frame}

% \begin{frame}
  % \frametitle{Do It Now!}
  % \begin{itemize}
  % \item Download and install Minizinc
  % \item \url{https://www.minizinc.org/}
  % \end{itemize}
% \end{frame}


\begin{frame}
\frametitle{Why use Puzzles as Examples?}
\begin{itemize}
\item Easy to understand the problem
\item Solvable by hand without specialized knowledge
\item Possible to compare automated to manual solving process
\end{itemize}

\begin{quote}
The puzzle, though inanimate, is presented as a solvable problem without lasting negative consequences, a very low-risk low-reward situation. By being a puzzle, the object is attempting to convince the user that it must be completed.
\end{quote}
{\tiny Source: \href{https://www.everydayrhetoric.com/2020/03/the-rhetoric-of-puzzles/}{Every Day Rhetoric}}
\end{frame}


\mode<all>{
\part{Basic Constraint Propagation}
}
\begin{frame}
\partpage
\end{frame}
\mode<all>{
\input{../sendmore/body}
}

\mode<all>{
\part{Global Constraints}
}
\begin{frame}
\partpage
\end{frame}
\mode<all>{
\input{../sudoku/body}
}

\clearpage
\mode<all>{
\part{Customizing Search}
}
\begin{frame}
\partpage
\end{frame}
\mode<all>{
\input{../nqueen/body}
}


%% \clearpage
%% \mode<all>{
%%   \part{Problem Decomposition}
%% }
%% \begin{frame}
%%   \partpage
%% \end{frame}
%% \mode<all>{
%%   \input{../party/body}
%%   }

%% \clearpage
%% \mode<all>{
%%   \part{More Global Constraints}
%% }
%% \begin{frame}
%%   \partpage
%% \end{frame}
%% \mode<all>{
%%   \input{../car/body}
%%   }

%% \clearpage
%% \mode<all>{
%%   \part{Methodology}
%% }
%% \begin{frame}
%%   \partpage
%% \end{frame}
%% \mode<all>{
%%   \input{../methodology/body}
%%   }

\clearpage
\mode<all>{
\part{What is missing?}
}
\begin{frame}
\partpage
\end{frame}

\begin{frame}
\frametitle{Many Specialized Topics}
\begin{itemize}
\item How to design efficient core engine
\item Hybrids with LP/MIP tools
\item Hybrids with SAT
\item Symmetry breaking
\item Use of MDD/BDD to encode sets of solutions
\item High level modelling tools
\item Debugging/visualization
\end{itemize}
\end{frame}

\begin{frame}
\frametitle{Reformulation}
\begin{itemize}
\item Just because the user has modelled it this way, it doesn't mean we have to solve it that way
\begin{itemize}
\item Replace some constraint(s) by other, equivalent constraints
\item Because we don't have that constraint in our system
\item For performance
\end{itemize}
\end{itemize}
\end{frame}

\begin{frame}
\frametitle{Learning}
\begin{itemize}
\item While solving the problem we can learn how to strengthen the model/search 
\begin{itemize}
\item Understand which constraints/method contribute to propagation and change schedule
\item Learn no-good constraints by explaining failure
\item Adapt search strategy based on search experience
\end{itemize}
\end{itemize}
\end{frame}

%% \begin{frame}
%% \frametitle{Refined Process}
%% \begin{center}
%% \begin{tikzpicture}
%% \node[shape=rectangle,fill=pantone127-4] (problem) at (3,7) {Problem};
%% \node (human) at (3,6) {Human};
%% \node[shape=rectangle,fill=pantone127-4] (model) at (3,5) {User Model};
%% \node (reformulation) at (3,4) {Reformulation};
%% \node[shape=rectangle,fill=pantone127-4] (implementation) at (3,3) {Implementation Model};
%% \node (solver) at (3,2) {Constraint Solver/Search};
%% \node[shape=rectangle,fill=pantone157-8] (solution) at (3,1) {Solution};
%% \draw[-] (problem) -- (human);
%% \draw[->] (human) -- (model);
%% \draw[-] (model) -- (reformulation);
%% \draw[->] (reformulation) -- (implementation);
%% \draw[-] (implementation) -- (solver);
%% \draw[->] (solver) -- (0,2) -- (0,3) -- (implementation);
%% \draw[->] (solver) -- (solution);
%% \end{tikzpicture}
%% \end{center}
%% \end{frame}

%\begin{frame}
%\frametitle{What is CP actually used for?}
%\url{http://hsimonis.wordpress.com}
%\includegraphics[width=10cm]{../introduction/images/blog}
%\end{frame}

%% \begin{frame}
%% \frametitle{Exercises}
%% \begin{itemize}
%% \item Install Constraint Solver of your Choice
%% \item MiniZinc (MiningZinc) will be used in later sessions
%% \item Run Sendmoremoney, Sudoku, Queens examples
%% \item Can you choose which propagator to use for alldifferent?
%% \item Can you use different search strategies?
%% \end{itemize}
%% \end{frame}

\begin{frame}
\frametitle{More Learning Resources}
\begin{itemize}
\item Survey of Methods, Resources, and Formats for Teaching Constraint Programming
\begin{itemize}
\item by Tejas Santanam, Helmut Simonis
\item \url{https://doi.org/10.48550/arXiv.2403.12717}
\item Based on survey of community for WTCP 2023
\item \url{https://hsimonis.github.io/WTCP2023/}
\end{itemize}
\end{itemize}
\end{frame}


}

\mode<all>{
\part{Global Constraints}
}
\begin{frame}
\partpage
\end{frame}
\mode<all>{
\begin{frame}
\frametitle{Objectives}
\begin{itemize}
\item Overview of Core Constraint Programming
\item Three Main Concepts
\begin{itemize}
\item Constraint Propagation
\item Global Constraints
\item Customizing Search
\end{itemize}
%\item Get Some Experience with MiniZinc
\item Topics will be treated in more detail in later parts of the school
\item Based on Examples, not Formal Description
\end{itemize}
\end{frame}

\begin{frame}
\frametitle{Outline}
\begin{itemize}
\item Why Constraint Programming?
\item Constraint Propagation
\item Global Constraints
\item Customizing Search
%\item Choices in Modelling
%\item Decomposition
%\item More Global Constraints
%\item Methodology
%\item What is missing?
\end{itemize}
\end{frame}

% \begin{frame}
  % \frametitle{Using MiniZinc IDE}
  % \begin{itemize}
  % \item Developed in the Australian NICTA project
  % \item Maintained by Monash University
  % \item Modelling tool with multiple back-end solvers
  % \item Available from \url{https://www.minizinc.org/}
  % \end{itemize}
% \end{frame}


% \begin{frame}
% \frametitle{Examples in ECLiPSe}
% \begin{itemize}
% \item Open sourced constraint programming language
% \item Development goes back to 1985
% \item ECRC, ICL, IC-Parc, PTL, Cisco
% \item \url{https://eclipseclp.org/}
% \item Specialities
% \begin{itemize}
% \item Develop new solvers for specific domains
% \item Integration with MIP
% \end{itemize}
% \item Not included in bundled MiniZinc IDE
% \item Specialized visualization tools used here
  % \begin{itemize}
  % \item CP-Viz, Simonis et al. 2010
    % \end{itemize}
% \end{itemize}
% \end{frame}

\begin{frame}
\frametitle{Tutorial Based on ECLiPSe ELearning Course}
\begin{itemize}
\item Self-study course in constraint programming
\item Supported by Cisco Systems and Silicon Valley Community Foundation
\item Multi-media format, video lectures, slides, handout etc
\item {\small \url{https://eclipseclp.org/ELearning/index.html}}
\end{itemize}
\end{frame}

\begin{frame}
\frametitle{Also Part of CRT-AI Constraint Week}
\begin{itemize}
\item Annual one week course on CP and Optimization in Ireland
\item Part of national training program for PhD students in AI
\item \url{https://www.crt-ai.ie/}
\end{itemize}
\end{frame}



\begin{frame}
\frametitle{Constraint Programming - in a nutshell}
\begin{itemize}
\item Declarative description of problems with
\begin{itemize}
\item {\em Variables} which range over (finite) sets of values
\item {\em Constraints} over subsets of variables which restrict possible value combinations
\item A {\em solution} is a value assignment which satisfies all constraints
\end{itemize}

\item Constraint propagation/reasoning
\begin{itemize}
\item Removing inconsistent values for variables
\item Detect failure if constraint can not be satisfied
\item Interaction of constraints via shared variables
\item Incomplete
\end{itemize}

\item Search
\begin{itemize}
\item User controlled assignment of values to variables
\item Each step triggers constraint propagation 
\end{itemize}
\item Different domains require/allow different methods
\end{itemize}
\end{frame}

\begin{frame}
  \frametitle{Constraint Programming is Different}
  \begin{itemize}
  \item Declarative Programming
    \begin{itemize}
    \item Concentrate on what you want
      \item Not how to get there
      \item Program != Algorithm
      \item Program = Model
    \end{itemize}
    \item Applied to Combinatorial Problems
      \begin{itemize}
        \item No complete polynomial algorithms known (exist?)
        \item CP less ad-hoc than heuristics
        \item Models can evolve
  \end{itemize}
  \end{itemize}
  \end{frame}
    
\begin{frame}
  \frametitle{A Subtractive Process}
  \begin{textblock}{4}(8,-3)
    \includegraphics[width=4cm]{../introduction/images/stages}
  \end{textblock}
  \vfill
  \begin{quote}
    ``Oh, bosh, as Mr. Ruskin says. Sculpture, per se, is the simplest thing in the world. All you have to do is to take a big chunk of marble and a hammer and chisel, make up your mind what you are about to create and chip off all the marble you don't want.''-Paris Gaulois.
  \end{quote}
  
  {\tiny Source: \url{https://quoteinvestigator.com/2014/06/22/chip-away/}}
\end{frame}

\begin{frame}
\frametitle{Basic Process}
\begin{center}
\begin{tikzpicture}
\node[shape=rectangle,fill=pantone127-4] (problem) at (2,5) {Problem};
\node (human) at (2,4) {Human};
\node[shape=rectangle,fill=pantone127-4] (model) at (2,3) {Model};
\node (solver) at (2,2) {Constraint Solver/Search};
\node[shape=rectangle,fill=pantone157-8] (solution) at (2,1) {Solution};
\draw[-] (problem) -- (human);
\draw[->] (human) -- (model);
\draw[-] (model) -- (solver);
\draw[->] (solver) -- (solution);
\end{tikzpicture}
\end{center}
\end{frame}

\begin{frame}
\frametitle{More Realistic}
\begin{center}
\begin{tikzpicture}[xscale=4.0]
\node[shape=rectangle,fill=pantone127-4] (problem) at (2,5) {Problem};
\node (human) at (2,4) {Human};
\node[shape=rectangle,fill=pantone127-4] (model) at (2,3) {Model};
\node (solver) at (2,2) {Constraint Solver/Search};
\node[shape=rectangle,fill=pantone24-8] (hangs) at (1,1) {Hangs};
\node[shape=rectangle,fill=pantone157-8] (solution) at (2,1) {Solution};
\node[shape=rectangle,fill=pantone24-8] (wrong) at (3,1) {Wrong Solution};
\draw[-] (problem) -- (human);
\draw[->] (human) -- (model);
\draw[-] (model) -- (solver);
\draw[->] (solver) -- (hangs);
\draw[->] (solver) -- (solution);
\draw[->] (solver) -- (wrong);
\draw[->] (hangs) -- (1,4) -- (human);
\draw[->] (wrong) -- (3,4) -- (human);
\end{tikzpicture}
\end{center}
\end{frame}

\begin{frame}
\frametitle{Dual Role of Model}
\begin{itemize}
\item Allows Human to Express Problem
\begin{itemize}
\item Close to Problem Domain
\item Constraints as Abstractions
\end{itemize}
\item Allows Solver to Execute
\begin{itemize}
\item Variables as Communication Mechanism
\item Constraints as Algorithms
\end{itemize}
\end{itemize}
\end{frame}

\begin{frame}
\frametitle{Modelling Frameworks}
\begin{itemize}
\item \href{https://www.minizinc.org/}{MiniZinc} (NICTA, Monash University, Australia)
\item \href{https://github.com/eomahony/Numberjack}{NumberJack} (Insight, Ireland) 
\item \href{https://www-users.york.ac.uk/peter.nightingale/savilerow//}{EssencePrime/SavilleRow} (UK) 
\item \href{https://cpmpy.readthedocs.io/en/latest/}{CPMpy} (KU Leuven)
\item Allow use of multiple back-end solvers
\item Compile model into variants for each solver
\item A priori solver independent model(CP, MIP, SAT)
\end{itemize}
\end{frame} 

\begin{frame}
\frametitle{Framework Process}
\begin{center}
\begin{tikzpicture}[xscale=2]
\node[shape=rectangle,fill=pantone127-4] (problem) at (2.5,5) {Problem};
\node (human) at (2.5,4) {Human};
\node[shape=rectangle,fill=pantone127-4] (model) at (2.5,3) {Model};
\node (compile) at (2.5,2) {Compile/Reformulate};
\node (solver1) at (1,1) {CP};
\node (solver2) at (2,1) {MIP};
\node (solver3) at (3,1) {SAT};
\node (solver4) at (4,1) {Other};
\node[shape=rectangle,fill=pantone157-8] (solution1) at (1,0) {Solution};
\node[shape=rectangle,fill=pantone157-8] (solution2) at (2,0) {Solution};
\node[shape=rectangle,fill=pantone157-8] (solution3) at (3,0) {Solution};
\node[shape=rectangle,fill=pantone157-8] (solution4) at (4,0) {Solution};
\draw[-] (problem) -- (human);
\draw[->] (human) -- (model);
\draw[-] (model) -- (compile);
\draw[-] (compile) -- (solver1);
\draw[-] (compile) -- (solver2);
\draw[-] (compile) -- (solver3);
\draw[-] (compile) -- (solver4);
\draw[->] (solver1) -- (solution1);
\draw[->] (solver2) -- (solution2);
\draw[->] (solver3) -- (solution3);
\draw[->] (solver4) -- (solution4);
\end{tikzpicture}
\end{center}
\end{frame}

% \begin{frame}
  % \frametitle{Do It Now!}
  % \begin{itemize}
  % \item Download and install Minizinc
  % \item \url{https://www.minizinc.org/}
  % \end{itemize}
% \end{frame}


\begin{frame}
\frametitle{Why use Puzzles as Examples?}
\begin{itemize}
\item Easy to understand the problem
\item Solvable by hand without specialized knowledge
\item Possible to compare automated to manual solving process
\end{itemize}

\begin{quote}
The puzzle, though inanimate, is presented as a solvable problem without lasting negative consequences, a very low-risk low-reward situation. By being a puzzle, the object is attempting to convince the user that it must be completed.
\end{quote}
{\tiny Source: \href{https://www.everydayrhetoric.com/2020/03/the-rhetoric-of-puzzles/}{Every Day Rhetoric}}
\end{frame}


\mode<all>{
\part{Basic Constraint Propagation}
}
\begin{frame}
\partpage
\end{frame}
\mode<all>{
\input{../sendmore/body}
}

\mode<all>{
\part{Global Constraints}
}
\begin{frame}
\partpage
\end{frame}
\mode<all>{
\input{../sudoku/body}
}

\clearpage
\mode<all>{
\part{Customizing Search}
}
\begin{frame}
\partpage
\end{frame}
\mode<all>{
\input{../nqueen/body}
}


%% \clearpage
%% \mode<all>{
%%   \part{Problem Decomposition}
%% }
%% \begin{frame}
%%   \partpage
%% \end{frame}
%% \mode<all>{
%%   \input{../party/body}
%%   }

%% \clearpage
%% \mode<all>{
%%   \part{More Global Constraints}
%% }
%% \begin{frame}
%%   \partpage
%% \end{frame}
%% \mode<all>{
%%   \input{../car/body}
%%   }

%% \clearpage
%% \mode<all>{
%%   \part{Methodology}
%% }
%% \begin{frame}
%%   \partpage
%% \end{frame}
%% \mode<all>{
%%   \input{../methodology/body}
%%   }

\clearpage
\mode<all>{
\part{What is missing?}
}
\begin{frame}
\partpage
\end{frame}

\begin{frame}
\frametitle{Many Specialized Topics}
\begin{itemize}
\item How to design efficient core engine
\item Hybrids with LP/MIP tools
\item Hybrids with SAT
\item Symmetry breaking
\item Use of MDD/BDD to encode sets of solutions
\item High level modelling tools
\item Debugging/visualization
\end{itemize}
\end{frame}

\begin{frame}
\frametitle{Reformulation}
\begin{itemize}
\item Just because the user has modelled it this way, it doesn't mean we have to solve it that way
\begin{itemize}
\item Replace some constraint(s) by other, equivalent constraints
\item Because we don't have that constraint in our system
\item For performance
\end{itemize}
\end{itemize}
\end{frame}

\begin{frame}
\frametitle{Learning}
\begin{itemize}
\item While solving the problem we can learn how to strengthen the model/search 
\begin{itemize}
\item Understand which constraints/method contribute to propagation and change schedule
\item Learn no-good constraints by explaining failure
\item Adapt search strategy based on search experience
\end{itemize}
\end{itemize}
\end{frame}

%% \begin{frame}
%% \frametitle{Refined Process}
%% \begin{center}
%% \begin{tikzpicture}
%% \node[shape=rectangle,fill=pantone127-4] (problem) at (3,7) {Problem};
%% \node (human) at (3,6) {Human};
%% \node[shape=rectangle,fill=pantone127-4] (model) at (3,5) {User Model};
%% \node (reformulation) at (3,4) {Reformulation};
%% \node[shape=rectangle,fill=pantone127-4] (implementation) at (3,3) {Implementation Model};
%% \node (solver) at (3,2) {Constraint Solver/Search};
%% \node[shape=rectangle,fill=pantone157-8] (solution) at (3,1) {Solution};
%% \draw[-] (problem) -- (human);
%% \draw[->] (human) -- (model);
%% \draw[-] (model) -- (reformulation);
%% \draw[->] (reformulation) -- (implementation);
%% \draw[-] (implementation) -- (solver);
%% \draw[->] (solver) -- (0,2) -- (0,3) -- (implementation);
%% \draw[->] (solver) -- (solution);
%% \end{tikzpicture}
%% \end{center}
%% \end{frame}

%\begin{frame}
%\frametitle{What is CP actually used for?}
%\url{http://hsimonis.wordpress.com}
%\includegraphics[width=10cm]{../introduction/images/blog}
%\end{frame}

%% \begin{frame}
%% \frametitle{Exercises}
%% \begin{itemize}
%% \item Install Constraint Solver of your Choice
%% \item MiniZinc (MiningZinc) will be used in later sessions
%% \item Run Sendmoremoney, Sudoku, Queens examples
%% \item Can you choose which propagator to use for alldifferent?
%% \item Can you use different search strategies?
%% \end{itemize}
%% \end{frame}

\begin{frame}
\frametitle{More Learning Resources}
\begin{itemize}
\item Survey of Methods, Resources, and Formats for Teaching Constraint Programming
\begin{itemize}
\item by Tejas Santanam, Helmut Simonis
\item \url{https://doi.org/10.48550/arXiv.2403.12717}
\item Based on survey of community for WTCP 2023
\item \url{https://hsimonis.github.io/WTCP2023/}
\end{itemize}
\end{itemize}
\end{frame}


}

\clearpage
\mode<all>{
\part{Customizing Search}
}
\begin{frame}
\partpage
\end{frame}
\mode<all>{
\begin{frame}
\frametitle{Objectives}
\begin{itemize}
\item Overview of Core Constraint Programming
\item Three Main Concepts
\begin{itemize}
\item Constraint Propagation
\item Global Constraints
\item Customizing Search
\end{itemize}
%\item Get Some Experience with MiniZinc
\item Topics will be treated in more detail in later parts of the school
\item Based on Examples, not Formal Description
\end{itemize}
\end{frame}

\begin{frame}
\frametitle{Outline}
\begin{itemize}
\item Why Constraint Programming?
\item Constraint Propagation
\item Global Constraints
\item Customizing Search
%\item Choices in Modelling
%\item Decomposition
%\item More Global Constraints
%\item Methodology
%\item What is missing?
\end{itemize}
\end{frame}

% \begin{frame}
  % \frametitle{Using MiniZinc IDE}
  % \begin{itemize}
  % \item Developed in the Australian NICTA project
  % \item Maintained by Monash University
  % \item Modelling tool with multiple back-end solvers
  % \item Available from \url{https://www.minizinc.org/}
  % \end{itemize}
% \end{frame}


% \begin{frame}
% \frametitle{Examples in ECLiPSe}
% \begin{itemize}
% \item Open sourced constraint programming language
% \item Development goes back to 1985
% \item ECRC, ICL, IC-Parc, PTL, Cisco
% \item \url{https://eclipseclp.org/}
% \item Specialities
% \begin{itemize}
% \item Develop new solvers for specific domains
% \item Integration with MIP
% \end{itemize}
% \item Not included in bundled MiniZinc IDE
% \item Specialized visualization tools used here
  % \begin{itemize}
  % \item CP-Viz, Simonis et al. 2010
    % \end{itemize}
% \end{itemize}
% \end{frame}

\begin{frame}
\frametitle{Tutorial Based on ECLiPSe ELearning Course}
\begin{itemize}
\item Self-study course in constraint programming
\item Supported by Cisco Systems and Silicon Valley Community Foundation
\item Multi-media format, video lectures, slides, handout etc
\item {\small \url{https://eclipseclp.org/ELearning/index.html}}
\end{itemize}
\end{frame}

\begin{frame}
\frametitle{Also Part of CRT-AI Constraint Week}
\begin{itemize}
\item Annual one week course on CP and Optimization in Ireland
\item Part of national training program for PhD students in AI
\item \url{https://www.crt-ai.ie/}
\end{itemize}
\end{frame}



\begin{frame}
\frametitle{Constraint Programming - in a nutshell}
\begin{itemize}
\item Declarative description of problems with
\begin{itemize}
\item {\em Variables} which range over (finite) sets of values
\item {\em Constraints} over subsets of variables which restrict possible value combinations
\item A {\em solution} is a value assignment which satisfies all constraints
\end{itemize}

\item Constraint propagation/reasoning
\begin{itemize}
\item Removing inconsistent values for variables
\item Detect failure if constraint can not be satisfied
\item Interaction of constraints via shared variables
\item Incomplete
\end{itemize}

\item Search
\begin{itemize}
\item User controlled assignment of values to variables
\item Each step triggers constraint propagation 
\end{itemize}
\item Different domains require/allow different methods
\end{itemize}
\end{frame}

\begin{frame}
  \frametitle{Constraint Programming is Different}
  \begin{itemize}
  \item Declarative Programming
    \begin{itemize}
    \item Concentrate on what you want
      \item Not how to get there
      \item Program != Algorithm
      \item Program = Model
    \end{itemize}
    \item Applied to Combinatorial Problems
      \begin{itemize}
        \item No complete polynomial algorithms known (exist?)
        \item CP less ad-hoc than heuristics
        \item Models can evolve
  \end{itemize}
  \end{itemize}
  \end{frame}
    
\begin{frame}
  \frametitle{A Subtractive Process}
  \begin{textblock}{4}(8,-3)
    \includegraphics[width=4cm]{../introduction/images/stages}
  \end{textblock}
  \vfill
  \begin{quote}
    ``Oh, bosh, as Mr. Ruskin says. Sculpture, per se, is the simplest thing in the world. All you have to do is to take a big chunk of marble and a hammer and chisel, make up your mind what you are about to create and chip off all the marble you don't want.''-Paris Gaulois.
  \end{quote}
  
  {\tiny Source: \url{https://quoteinvestigator.com/2014/06/22/chip-away/}}
\end{frame}

\begin{frame}
\frametitle{Basic Process}
\begin{center}
\begin{tikzpicture}
\node[shape=rectangle,fill=pantone127-4] (problem) at (2,5) {Problem};
\node (human) at (2,4) {Human};
\node[shape=rectangle,fill=pantone127-4] (model) at (2,3) {Model};
\node (solver) at (2,2) {Constraint Solver/Search};
\node[shape=rectangle,fill=pantone157-8] (solution) at (2,1) {Solution};
\draw[-] (problem) -- (human);
\draw[->] (human) -- (model);
\draw[-] (model) -- (solver);
\draw[->] (solver) -- (solution);
\end{tikzpicture}
\end{center}
\end{frame}

\begin{frame}
\frametitle{More Realistic}
\begin{center}
\begin{tikzpicture}[xscale=4.0]
\node[shape=rectangle,fill=pantone127-4] (problem) at (2,5) {Problem};
\node (human) at (2,4) {Human};
\node[shape=rectangle,fill=pantone127-4] (model) at (2,3) {Model};
\node (solver) at (2,2) {Constraint Solver/Search};
\node[shape=rectangle,fill=pantone24-8] (hangs) at (1,1) {Hangs};
\node[shape=rectangle,fill=pantone157-8] (solution) at (2,1) {Solution};
\node[shape=rectangle,fill=pantone24-8] (wrong) at (3,1) {Wrong Solution};
\draw[-] (problem) -- (human);
\draw[->] (human) -- (model);
\draw[-] (model) -- (solver);
\draw[->] (solver) -- (hangs);
\draw[->] (solver) -- (solution);
\draw[->] (solver) -- (wrong);
\draw[->] (hangs) -- (1,4) -- (human);
\draw[->] (wrong) -- (3,4) -- (human);
\end{tikzpicture}
\end{center}
\end{frame}

\begin{frame}
\frametitle{Dual Role of Model}
\begin{itemize}
\item Allows Human to Express Problem
\begin{itemize}
\item Close to Problem Domain
\item Constraints as Abstractions
\end{itemize}
\item Allows Solver to Execute
\begin{itemize}
\item Variables as Communication Mechanism
\item Constraints as Algorithms
\end{itemize}
\end{itemize}
\end{frame}

\begin{frame}
\frametitle{Modelling Frameworks}
\begin{itemize}
\item \href{https://www.minizinc.org/}{MiniZinc} (NICTA, Monash University, Australia)
\item \href{https://github.com/eomahony/Numberjack}{NumberJack} (Insight, Ireland) 
\item \href{https://www-users.york.ac.uk/peter.nightingale/savilerow//}{EssencePrime/SavilleRow} (UK) 
\item \href{https://cpmpy.readthedocs.io/en/latest/}{CPMpy} (KU Leuven)
\item Allow use of multiple back-end solvers
\item Compile model into variants for each solver
\item A priori solver independent model(CP, MIP, SAT)
\end{itemize}
\end{frame} 

\begin{frame}
\frametitle{Framework Process}
\begin{center}
\begin{tikzpicture}[xscale=2]
\node[shape=rectangle,fill=pantone127-4] (problem) at (2.5,5) {Problem};
\node (human) at (2.5,4) {Human};
\node[shape=rectangle,fill=pantone127-4] (model) at (2.5,3) {Model};
\node (compile) at (2.5,2) {Compile/Reformulate};
\node (solver1) at (1,1) {CP};
\node (solver2) at (2,1) {MIP};
\node (solver3) at (3,1) {SAT};
\node (solver4) at (4,1) {Other};
\node[shape=rectangle,fill=pantone157-8] (solution1) at (1,0) {Solution};
\node[shape=rectangle,fill=pantone157-8] (solution2) at (2,0) {Solution};
\node[shape=rectangle,fill=pantone157-8] (solution3) at (3,0) {Solution};
\node[shape=rectangle,fill=pantone157-8] (solution4) at (4,0) {Solution};
\draw[-] (problem) -- (human);
\draw[->] (human) -- (model);
\draw[-] (model) -- (compile);
\draw[-] (compile) -- (solver1);
\draw[-] (compile) -- (solver2);
\draw[-] (compile) -- (solver3);
\draw[-] (compile) -- (solver4);
\draw[->] (solver1) -- (solution1);
\draw[->] (solver2) -- (solution2);
\draw[->] (solver3) -- (solution3);
\draw[->] (solver4) -- (solution4);
\end{tikzpicture}
\end{center}
\end{frame}

% \begin{frame}
  % \frametitle{Do It Now!}
  % \begin{itemize}
  % \item Download and install Minizinc
  % \item \url{https://www.minizinc.org/}
  % \end{itemize}
% \end{frame}


\begin{frame}
\frametitle{Why use Puzzles as Examples?}
\begin{itemize}
\item Easy to understand the problem
\item Solvable by hand without specialized knowledge
\item Possible to compare automated to manual solving process
\end{itemize}

\begin{quote}
The puzzle, though inanimate, is presented as a solvable problem without lasting negative consequences, a very low-risk low-reward situation. By being a puzzle, the object is attempting to convince the user that it must be completed.
\end{quote}
{\tiny Source: \href{https://www.everydayrhetoric.com/2020/03/the-rhetoric-of-puzzles/}{Every Day Rhetoric}}
\end{frame}


\mode<all>{
\part{Basic Constraint Propagation}
}
\begin{frame}
\partpage
\end{frame}
\mode<all>{
\input{../sendmore/body}
}

\mode<all>{
\part{Global Constraints}
}
\begin{frame}
\partpage
\end{frame}
\mode<all>{
\input{../sudoku/body}
}

\clearpage
\mode<all>{
\part{Customizing Search}
}
\begin{frame}
\partpage
\end{frame}
\mode<all>{
\input{../nqueen/body}
}


%% \clearpage
%% \mode<all>{
%%   \part{Problem Decomposition}
%% }
%% \begin{frame}
%%   \partpage
%% \end{frame}
%% \mode<all>{
%%   \input{../party/body}
%%   }

%% \clearpage
%% \mode<all>{
%%   \part{More Global Constraints}
%% }
%% \begin{frame}
%%   \partpage
%% \end{frame}
%% \mode<all>{
%%   \input{../car/body}
%%   }

%% \clearpage
%% \mode<all>{
%%   \part{Methodology}
%% }
%% \begin{frame}
%%   \partpage
%% \end{frame}
%% \mode<all>{
%%   \input{../methodology/body}
%%   }

\clearpage
\mode<all>{
\part{What is missing?}
}
\begin{frame}
\partpage
\end{frame}

\begin{frame}
\frametitle{Many Specialized Topics}
\begin{itemize}
\item How to design efficient core engine
\item Hybrids with LP/MIP tools
\item Hybrids with SAT
\item Symmetry breaking
\item Use of MDD/BDD to encode sets of solutions
\item High level modelling tools
\item Debugging/visualization
\end{itemize}
\end{frame}

\begin{frame}
\frametitle{Reformulation}
\begin{itemize}
\item Just because the user has modelled it this way, it doesn't mean we have to solve it that way
\begin{itemize}
\item Replace some constraint(s) by other, equivalent constraints
\item Because we don't have that constraint in our system
\item For performance
\end{itemize}
\end{itemize}
\end{frame}

\begin{frame}
\frametitle{Learning}
\begin{itemize}
\item While solving the problem we can learn how to strengthen the model/search 
\begin{itemize}
\item Understand which constraints/method contribute to propagation and change schedule
\item Learn no-good constraints by explaining failure
\item Adapt search strategy based on search experience
\end{itemize}
\end{itemize}
\end{frame}

%% \begin{frame}
%% \frametitle{Refined Process}
%% \begin{center}
%% \begin{tikzpicture}
%% \node[shape=rectangle,fill=pantone127-4] (problem) at (3,7) {Problem};
%% \node (human) at (3,6) {Human};
%% \node[shape=rectangle,fill=pantone127-4] (model) at (3,5) {User Model};
%% \node (reformulation) at (3,4) {Reformulation};
%% \node[shape=rectangle,fill=pantone127-4] (implementation) at (3,3) {Implementation Model};
%% \node (solver) at (3,2) {Constraint Solver/Search};
%% \node[shape=rectangle,fill=pantone157-8] (solution) at (3,1) {Solution};
%% \draw[-] (problem) -- (human);
%% \draw[->] (human) -- (model);
%% \draw[-] (model) -- (reformulation);
%% \draw[->] (reformulation) -- (implementation);
%% \draw[-] (implementation) -- (solver);
%% \draw[->] (solver) -- (0,2) -- (0,3) -- (implementation);
%% \draw[->] (solver) -- (solution);
%% \end{tikzpicture}
%% \end{center}
%% \end{frame}

%\begin{frame}
%\frametitle{What is CP actually used for?}
%\url{http://hsimonis.wordpress.com}
%\includegraphics[width=10cm]{../introduction/images/blog}
%\end{frame}

%% \begin{frame}
%% \frametitle{Exercises}
%% \begin{itemize}
%% \item Install Constraint Solver of your Choice
%% \item MiniZinc (MiningZinc) will be used in later sessions
%% \item Run Sendmoremoney, Sudoku, Queens examples
%% \item Can you choose which propagator to use for alldifferent?
%% \item Can you use different search strategies?
%% \end{itemize}
%% \end{frame}

\begin{frame}
\frametitle{More Learning Resources}
\begin{itemize}
\item Survey of Methods, Resources, and Formats for Teaching Constraint Programming
\begin{itemize}
\item by Tejas Santanam, Helmut Simonis
\item \url{https://doi.org/10.48550/arXiv.2403.12717}
\item Based on survey of community for WTCP 2023
\item \url{https://hsimonis.github.io/WTCP2023/}
\end{itemize}
\end{itemize}
\end{frame}


}


%% \clearpage
%% \mode<all>{
%%   \part{Problem Decomposition}
%% }
%% \begin{frame}
%%   \partpage
%% \end{frame}
%% \mode<all>{
%%   \begin{frame}
\frametitle{Objectives}
\begin{itemize}
\item Overview of Core Constraint Programming
\item Three Main Concepts
\begin{itemize}
\item Constraint Propagation
\item Global Constraints
\item Customizing Search
\end{itemize}
%\item Get Some Experience with MiniZinc
\item Topics will be treated in more detail in later parts of the school
\item Based on Examples, not Formal Description
\end{itemize}
\end{frame}

\begin{frame}
\frametitle{Outline}
\begin{itemize}
\item Why Constraint Programming?
\item Constraint Propagation
\item Global Constraints
\item Customizing Search
%\item Choices in Modelling
%\item Decomposition
%\item More Global Constraints
%\item Methodology
%\item What is missing?
\end{itemize}
\end{frame}

% \begin{frame}
  % \frametitle{Using MiniZinc IDE}
  % \begin{itemize}
  % \item Developed in the Australian NICTA project
  % \item Maintained by Monash University
  % \item Modelling tool with multiple back-end solvers
  % \item Available from \url{https://www.minizinc.org/}
  % \end{itemize}
% \end{frame}


% \begin{frame}
% \frametitle{Examples in ECLiPSe}
% \begin{itemize}
% \item Open sourced constraint programming language
% \item Development goes back to 1985
% \item ECRC, ICL, IC-Parc, PTL, Cisco
% \item \url{https://eclipseclp.org/}
% \item Specialities
% \begin{itemize}
% \item Develop new solvers for specific domains
% \item Integration with MIP
% \end{itemize}
% \item Not included in bundled MiniZinc IDE
% \item Specialized visualization tools used here
  % \begin{itemize}
  % \item CP-Viz, Simonis et al. 2010
    % \end{itemize}
% \end{itemize}
% \end{frame}

\begin{frame}
\frametitle{Tutorial Based on ECLiPSe ELearning Course}
\begin{itemize}
\item Self-study course in constraint programming
\item Supported by Cisco Systems and Silicon Valley Community Foundation
\item Multi-media format, video lectures, slides, handout etc
\item {\small \url{https://eclipseclp.org/ELearning/index.html}}
\end{itemize}
\end{frame}

\begin{frame}
\frametitle{Also Part of CRT-AI Constraint Week}
\begin{itemize}
\item Annual one week course on CP and Optimization in Ireland
\item Part of national training program for PhD students in AI
\item \url{https://www.crt-ai.ie/}
\end{itemize}
\end{frame}



\begin{frame}
\frametitle{Constraint Programming - in a nutshell}
\begin{itemize}
\item Declarative description of problems with
\begin{itemize}
\item {\em Variables} which range over (finite) sets of values
\item {\em Constraints} over subsets of variables which restrict possible value combinations
\item A {\em solution} is a value assignment which satisfies all constraints
\end{itemize}

\item Constraint propagation/reasoning
\begin{itemize}
\item Removing inconsistent values for variables
\item Detect failure if constraint can not be satisfied
\item Interaction of constraints via shared variables
\item Incomplete
\end{itemize}

\item Search
\begin{itemize}
\item User controlled assignment of values to variables
\item Each step triggers constraint propagation 
\end{itemize}
\item Different domains require/allow different methods
\end{itemize}
\end{frame}

\begin{frame}
  \frametitle{Constraint Programming is Different}
  \begin{itemize}
  \item Declarative Programming
    \begin{itemize}
    \item Concentrate on what you want
      \item Not how to get there
      \item Program != Algorithm
      \item Program = Model
    \end{itemize}
    \item Applied to Combinatorial Problems
      \begin{itemize}
        \item No complete polynomial algorithms known (exist?)
        \item CP less ad-hoc than heuristics
        \item Models can evolve
  \end{itemize}
  \end{itemize}
  \end{frame}
    
\begin{frame}
  \frametitle{A Subtractive Process}
  \begin{textblock}{4}(8,-3)
    \includegraphics[width=4cm]{../introduction/images/stages}
  \end{textblock}
  \vfill
  \begin{quote}
    ``Oh, bosh, as Mr. Ruskin says. Sculpture, per se, is the simplest thing in the world. All you have to do is to take a big chunk of marble and a hammer and chisel, make up your mind what you are about to create and chip off all the marble you don't want.''-Paris Gaulois.
  \end{quote}
  
  {\tiny Source: \url{https://quoteinvestigator.com/2014/06/22/chip-away/}}
\end{frame}

\begin{frame}
\frametitle{Basic Process}
\begin{center}
\begin{tikzpicture}
\node[shape=rectangle,fill=pantone127-4] (problem) at (2,5) {Problem};
\node (human) at (2,4) {Human};
\node[shape=rectangle,fill=pantone127-4] (model) at (2,3) {Model};
\node (solver) at (2,2) {Constraint Solver/Search};
\node[shape=rectangle,fill=pantone157-8] (solution) at (2,1) {Solution};
\draw[-] (problem) -- (human);
\draw[->] (human) -- (model);
\draw[-] (model) -- (solver);
\draw[->] (solver) -- (solution);
\end{tikzpicture}
\end{center}
\end{frame}

\begin{frame}
\frametitle{More Realistic}
\begin{center}
\begin{tikzpicture}[xscale=4.0]
\node[shape=rectangle,fill=pantone127-4] (problem) at (2,5) {Problem};
\node (human) at (2,4) {Human};
\node[shape=rectangle,fill=pantone127-4] (model) at (2,3) {Model};
\node (solver) at (2,2) {Constraint Solver/Search};
\node[shape=rectangle,fill=pantone24-8] (hangs) at (1,1) {Hangs};
\node[shape=rectangle,fill=pantone157-8] (solution) at (2,1) {Solution};
\node[shape=rectangle,fill=pantone24-8] (wrong) at (3,1) {Wrong Solution};
\draw[-] (problem) -- (human);
\draw[->] (human) -- (model);
\draw[-] (model) -- (solver);
\draw[->] (solver) -- (hangs);
\draw[->] (solver) -- (solution);
\draw[->] (solver) -- (wrong);
\draw[->] (hangs) -- (1,4) -- (human);
\draw[->] (wrong) -- (3,4) -- (human);
\end{tikzpicture}
\end{center}
\end{frame}

\begin{frame}
\frametitle{Dual Role of Model}
\begin{itemize}
\item Allows Human to Express Problem
\begin{itemize}
\item Close to Problem Domain
\item Constraints as Abstractions
\end{itemize}
\item Allows Solver to Execute
\begin{itemize}
\item Variables as Communication Mechanism
\item Constraints as Algorithms
\end{itemize}
\end{itemize}
\end{frame}

\begin{frame}
\frametitle{Modelling Frameworks}
\begin{itemize}
\item \href{https://www.minizinc.org/}{MiniZinc} (NICTA, Monash University, Australia)
\item \href{https://github.com/eomahony/Numberjack}{NumberJack} (Insight, Ireland) 
\item \href{https://www-users.york.ac.uk/peter.nightingale/savilerow//}{EssencePrime/SavilleRow} (UK) 
\item \href{https://cpmpy.readthedocs.io/en/latest/}{CPMpy} (KU Leuven)
\item Allow use of multiple back-end solvers
\item Compile model into variants for each solver
\item A priori solver independent model(CP, MIP, SAT)
\end{itemize}
\end{frame} 

\begin{frame}
\frametitle{Framework Process}
\begin{center}
\begin{tikzpicture}[xscale=2]
\node[shape=rectangle,fill=pantone127-4] (problem) at (2.5,5) {Problem};
\node (human) at (2.5,4) {Human};
\node[shape=rectangle,fill=pantone127-4] (model) at (2.5,3) {Model};
\node (compile) at (2.5,2) {Compile/Reformulate};
\node (solver1) at (1,1) {CP};
\node (solver2) at (2,1) {MIP};
\node (solver3) at (3,1) {SAT};
\node (solver4) at (4,1) {Other};
\node[shape=rectangle,fill=pantone157-8] (solution1) at (1,0) {Solution};
\node[shape=rectangle,fill=pantone157-8] (solution2) at (2,0) {Solution};
\node[shape=rectangle,fill=pantone157-8] (solution3) at (3,0) {Solution};
\node[shape=rectangle,fill=pantone157-8] (solution4) at (4,0) {Solution};
\draw[-] (problem) -- (human);
\draw[->] (human) -- (model);
\draw[-] (model) -- (compile);
\draw[-] (compile) -- (solver1);
\draw[-] (compile) -- (solver2);
\draw[-] (compile) -- (solver3);
\draw[-] (compile) -- (solver4);
\draw[->] (solver1) -- (solution1);
\draw[->] (solver2) -- (solution2);
\draw[->] (solver3) -- (solution3);
\draw[->] (solver4) -- (solution4);
\end{tikzpicture}
\end{center}
\end{frame}

% \begin{frame}
  % \frametitle{Do It Now!}
  % \begin{itemize}
  % \item Download and install Minizinc
  % \item \url{https://www.minizinc.org/}
  % \end{itemize}
% \end{frame}


\begin{frame}
\frametitle{Why use Puzzles as Examples?}
\begin{itemize}
\item Easy to understand the problem
\item Solvable by hand without specialized knowledge
\item Possible to compare automated to manual solving process
\end{itemize}

\begin{quote}
The puzzle, though inanimate, is presented as a solvable problem without lasting negative consequences, a very low-risk low-reward situation. By being a puzzle, the object is attempting to convince the user that it must be completed.
\end{quote}
{\tiny Source: \href{https://www.everydayrhetoric.com/2020/03/the-rhetoric-of-puzzles/}{Every Day Rhetoric}}
\end{frame}


\mode<all>{
\part{Basic Constraint Propagation}
}
\begin{frame}
\partpage
\end{frame}
\mode<all>{
\input{../sendmore/body}
}

\mode<all>{
\part{Global Constraints}
}
\begin{frame}
\partpage
\end{frame}
\mode<all>{
\input{../sudoku/body}
}

\clearpage
\mode<all>{
\part{Customizing Search}
}
\begin{frame}
\partpage
\end{frame}
\mode<all>{
\input{../nqueen/body}
}


%% \clearpage
%% \mode<all>{
%%   \part{Problem Decomposition}
%% }
%% \begin{frame}
%%   \partpage
%% \end{frame}
%% \mode<all>{
%%   \input{../party/body}
%%   }

%% \clearpage
%% \mode<all>{
%%   \part{More Global Constraints}
%% }
%% \begin{frame}
%%   \partpage
%% \end{frame}
%% \mode<all>{
%%   \input{../car/body}
%%   }

%% \clearpage
%% \mode<all>{
%%   \part{Methodology}
%% }
%% \begin{frame}
%%   \partpage
%% \end{frame}
%% \mode<all>{
%%   \input{../methodology/body}
%%   }

\clearpage
\mode<all>{
\part{What is missing?}
}
\begin{frame}
\partpage
\end{frame}

\begin{frame}
\frametitle{Many Specialized Topics}
\begin{itemize}
\item How to design efficient core engine
\item Hybrids with LP/MIP tools
\item Hybrids with SAT
\item Symmetry breaking
\item Use of MDD/BDD to encode sets of solutions
\item High level modelling tools
\item Debugging/visualization
\end{itemize}
\end{frame}

\begin{frame}
\frametitle{Reformulation}
\begin{itemize}
\item Just because the user has modelled it this way, it doesn't mean we have to solve it that way
\begin{itemize}
\item Replace some constraint(s) by other, equivalent constraints
\item Because we don't have that constraint in our system
\item For performance
\end{itemize}
\end{itemize}
\end{frame}

\begin{frame}
\frametitle{Learning}
\begin{itemize}
\item While solving the problem we can learn how to strengthen the model/search 
\begin{itemize}
\item Understand which constraints/method contribute to propagation and change schedule
\item Learn no-good constraints by explaining failure
\item Adapt search strategy based on search experience
\end{itemize}
\end{itemize}
\end{frame}

%% \begin{frame}
%% \frametitle{Refined Process}
%% \begin{center}
%% \begin{tikzpicture}
%% \node[shape=rectangle,fill=pantone127-4] (problem) at (3,7) {Problem};
%% \node (human) at (3,6) {Human};
%% \node[shape=rectangle,fill=pantone127-4] (model) at (3,5) {User Model};
%% \node (reformulation) at (3,4) {Reformulation};
%% \node[shape=rectangle,fill=pantone127-4] (implementation) at (3,3) {Implementation Model};
%% \node (solver) at (3,2) {Constraint Solver/Search};
%% \node[shape=rectangle,fill=pantone157-8] (solution) at (3,1) {Solution};
%% \draw[-] (problem) -- (human);
%% \draw[->] (human) -- (model);
%% \draw[-] (model) -- (reformulation);
%% \draw[->] (reformulation) -- (implementation);
%% \draw[-] (implementation) -- (solver);
%% \draw[->] (solver) -- (0,2) -- (0,3) -- (implementation);
%% \draw[->] (solver) -- (solution);
%% \end{tikzpicture}
%% \end{center}
%% \end{frame}

%\begin{frame}
%\frametitle{What is CP actually used for?}
%\url{http://hsimonis.wordpress.com}
%\includegraphics[width=10cm]{../introduction/images/blog}
%\end{frame}

%% \begin{frame}
%% \frametitle{Exercises}
%% \begin{itemize}
%% \item Install Constraint Solver of your Choice
%% \item MiniZinc (MiningZinc) will be used in later sessions
%% \item Run Sendmoremoney, Sudoku, Queens examples
%% \item Can you choose which propagator to use for alldifferent?
%% \item Can you use different search strategies?
%% \end{itemize}
%% \end{frame}

\begin{frame}
\frametitle{More Learning Resources}
\begin{itemize}
\item Survey of Methods, Resources, and Formats for Teaching Constraint Programming
\begin{itemize}
\item by Tejas Santanam, Helmut Simonis
\item \url{https://doi.org/10.48550/arXiv.2403.12717}
\item Based on survey of community for WTCP 2023
\item \url{https://hsimonis.github.io/WTCP2023/}
\end{itemize}
\end{itemize}
\end{frame}


%%   }

%% \clearpage
%% \mode<all>{
%%   \part{More Global Constraints}
%% }
%% \begin{frame}
%%   \partpage
%% \end{frame}
%% \mode<all>{
%%   \begin{frame}
\frametitle{Objectives}
\begin{itemize}
\item Overview of Core Constraint Programming
\item Three Main Concepts
\begin{itemize}
\item Constraint Propagation
\item Global Constraints
\item Customizing Search
\end{itemize}
%\item Get Some Experience with MiniZinc
\item Topics will be treated in more detail in later parts of the school
\item Based on Examples, not Formal Description
\end{itemize}
\end{frame}

\begin{frame}
\frametitle{Outline}
\begin{itemize}
\item Why Constraint Programming?
\item Constraint Propagation
\item Global Constraints
\item Customizing Search
%\item Choices in Modelling
%\item Decomposition
%\item More Global Constraints
%\item Methodology
%\item What is missing?
\end{itemize}
\end{frame}

% \begin{frame}
  % \frametitle{Using MiniZinc IDE}
  % \begin{itemize}
  % \item Developed in the Australian NICTA project
  % \item Maintained by Monash University
  % \item Modelling tool with multiple back-end solvers
  % \item Available from \url{https://www.minizinc.org/}
  % \end{itemize}
% \end{frame}


% \begin{frame}
% \frametitle{Examples in ECLiPSe}
% \begin{itemize}
% \item Open sourced constraint programming language
% \item Development goes back to 1985
% \item ECRC, ICL, IC-Parc, PTL, Cisco
% \item \url{https://eclipseclp.org/}
% \item Specialities
% \begin{itemize}
% \item Develop new solvers for specific domains
% \item Integration with MIP
% \end{itemize}
% \item Not included in bundled MiniZinc IDE
% \item Specialized visualization tools used here
  % \begin{itemize}
  % \item CP-Viz, Simonis et al. 2010
    % \end{itemize}
% \end{itemize}
% \end{frame}

\begin{frame}
\frametitle{Tutorial Based on ECLiPSe ELearning Course}
\begin{itemize}
\item Self-study course in constraint programming
\item Supported by Cisco Systems and Silicon Valley Community Foundation
\item Multi-media format, video lectures, slides, handout etc
\item {\small \url{https://eclipseclp.org/ELearning/index.html}}
\end{itemize}
\end{frame}

\begin{frame}
\frametitle{Also Part of CRT-AI Constraint Week}
\begin{itemize}
\item Annual one week course on CP and Optimization in Ireland
\item Part of national training program for PhD students in AI
\item \url{https://www.crt-ai.ie/}
\end{itemize}
\end{frame}



\begin{frame}
\frametitle{Constraint Programming - in a nutshell}
\begin{itemize}
\item Declarative description of problems with
\begin{itemize}
\item {\em Variables} which range over (finite) sets of values
\item {\em Constraints} over subsets of variables which restrict possible value combinations
\item A {\em solution} is a value assignment which satisfies all constraints
\end{itemize}

\item Constraint propagation/reasoning
\begin{itemize}
\item Removing inconsistent values for variables
\item Detect failure if constraint can not be satisfied
\item Interaction of constraints via shared variables
\item Incomplete
\end{itemize}

\item Search
\begin{itemize}
\item User controlled assignment of values to variables
\item Each step triggers constraint propagation 
\end{itemize}
\item Different domains require/allow different methods
\end{itemize}
\end{frame}

\begin{frame}
  \frametitle{Constraint Programming is Different}
  \begin{itemize}
  \item Declarative Programming
    \begin{itemize}
    \item Concentrate on what you want
      \item Not how to get there
      \item Program != Algorithm
      \item Program = Model
    \end{itemize}
    \item Applied to Combinatorial Problems
      \begin{itemize}
        \item No complete polynomial algorithms known (exist?)
        \item CP less ad-hoc than heuristics
        \item Models can evolve
  \end{itemize}
  \end{itemize}
  \end{frame}
    
\begin{frame}
  \frametitle{A Subtractive Process}
  \begin{textblock}{4}(8,-3)
    \includegraphics[width=4cm]{../introduction/images/stages}
  \end{textblock}
  \vfill
  \begin{quote}
    ``Oh, bosh, as Mr. Ruskin says. Sculpture, per se, is the simplest thing in the world. All you have to do is to take a big chunk of marble and a hammer and chisel, make up your mind what you are about to create and chip off all the marble you don't want.''-Paris Gaulois.
  \end{quote}
  
  {\tiny Source: \url{https://quoteinvestigator.com/2014/06/22/chip-away/}}
\end{frame}

\begin{frame}
\frametitle{Basic Process}
\begin{center}
\begin{tikzpicture}
\node[shape=rectangle,fill=pantone127-4] (problem) at (2,5) {Problem};
\node (human) at (2,4) {Human};
\node[shape=rectangle,fill=pantone127-4] (model) at (2,3) {Model};
\node (solver) at (2,2) {Constraint Solver/Search};
\node[shape=rectangle,fill=pantone157-8] (solution) at (2,1) {Solution};
\draw[-] (problem) -- (human);
\draw[->] (human) -- (model);
\draw[-] (model) -- (solver);
\draw[->] (solver) -- (solution);
\end{tikzpicture}
\end{center}
\end{frame}

\begin{frame}
\frametitle{More Realistic}
\begin{center}
\begin{tikzpicture}[xscale=4.0]
\node[shape=rectangle,fill=pantone127-4] (problem) at (2,5) {Problem};
\node (human) at (2,4) {Human};
\node[shape=rectangle,fill=pantone127-4] (model) at (2,3) {Model};
\node (solver) at (2,2) {Constraint Solver/Search};
\node[shape=rectangle,fill=pantone24-8] (hangs) at (1,1) {Hangs};
\node[shape=rectangle,fill=pantone157-8] (solution) at (2,1) {Solution};
\node[shape=rectangle,fill=pantone24-8] (wrong) at (3,1) {Wrong Solution};
\draw[-] (problem) -- (human);
\draw[->] (human) -- (model);
\draw[-] (model) -- (solver);
\draw[->] (solver) -- (hangs);
\draw[->] (solver) -- (solution);
\draw[->] (solver) -- (wrong);
\draw[->] (hangs) -- (1,4) -- (human);
\draw[->] (wrong) -- (3,4) -- (human);
\end{tikzpicture}
\end{center}
\end{frame}

\begin{frame}
\frametitle{Dual Role of Model}
\begin{itemize}
\item Allows Human to Express Problem
\begin{itemize}
\item Close to Problem Domain
\item Constraints as Abstractions
\end{itemize}
\item Allows Solver to Execute
\begin{itemize}
\item Variables as Communication Mechanism
\item Constraints as Algorithms
\end{itemize}
\end{itemize}
\end{frame}

\begin{frame}
\frametitle{Modelling Frameworks}
\begin{itemize}
\item \href{https://www.minizinc.org/}{MiniZinc} (NICTA, Monash University, Australia)
\item \href{https://github.com/eomahony/Numberjack}{NumberJack} (Insight, Ireland) 
\item \href{https://www-users.york.ac.uk/peter.nightingale/savilerow//}{EssencePrime/SavilleRow} (UK) 
\item \href{https://cpmpy.readthedocs.io/en/latest/}{CPMpy} (KU Leuven)
\item Allow use of multiple back-end solvers
\item Compile model into variants for each solver
\item A priori solver independent model(CP, MIP, SAT)
\end{itemize}
\end{frame} 

\begin{frame}
\frametitle{Framework Process}
\begin{center}
\begin{tikzpicture}[xscale=2]
\node[shape=rectangle,fill=pantone127-4] (problem) at (2.5,5) {Problem};
\node (human) at (2.5,4) {Human};
\node[shape=rectangle,fill=pantone127-4] (model) at (2.5,3) {Model};
\node (compile) at (2.5,2) {Compile/Reformulate};
\node (solver1) at (1,1) {CP};
\node (solver2) at (2,1) {MIP};
\node (solver3) at (3,1) {SAT};
\node (solver4) at (4,1) {Other};
\node[shape=rectangle,fill=pantone157-8] (solution1) at (1,0) {Solution};
\node[shape=rectangle,fill=pantone157-8] (solution2) at (2,0) {Solution};
\node[shape=rectangle,fill=pantone157-8] (solution3) at (3,0) {Solution};
\node[shape=rectangle,fill=pantone157-8] (solution4) at (4,0) {Solution};
\draw[-] (problem) -- (human);
\draw[->] (human) -- (model);
\draw[-] (model) -- (compile);
\draw[-] (compile) -- (solver1);
\draw[-] (compile) -- (solver2);
\draw[-] (compile) -- (solver3);
\draw[-] (compile) -- (solver4);
\draw[->] (solver1) -- (solution1);
\draw[->] (solver2) -- (solution2);
\draw[->] (solver3) -- (solution3);
\draw[->] (solver4) -- (solution4);
\end{tikzpicture}
\end{center}
\end{frame}

% \begin{frame}
  % \frametitle{Do It Now!}
  % \begin{itemize}
  % \item Download and install Minizinc
  % \item \url{https://www.minizinc.org/}
  % \end{itemize}
% \end{frame}


\begin{frame}
\frametitle{Why use Puzzles as Examples?}
\begin{itemize}
\item Easy to understand the problem
\item Solvable by hand without specialized knowledge
\item Possible to compare automated to manual solving process
\end{itemize}

\begin{quote}
The puzzle, though inanimate, is presented as a solvable problem without lasting negative consequences, a very low-risk low-reward situation. By being a puzzle, the object is attempting to convince the user that it must be completed.
\end{quote}
{\tiny Source: \href{https://www.everydayrhetoric.com/2020/03/the-rhetoric-of-puzzles/}{Every Day Rhetoric}}
\end{frame}


\mode<all>{
\part{Basic Constraint Propagation}
}
\begin{frame}
\partpage
\end{frame}
\mode<all>{
\input{../sendmore/body}
}

\mode<all>{
\part{Global Constraints}
}
\begin{frame}
\partpage
\end{frame}
\mode<all>{
\input{../sudoku/body}
}

\clearpage
\mode<all>{
\part{Customizing Search}
}
\begin{frame}
\partpage
\end{frame}
\mode<all>{
\input{../nqueen/body}
}


%% \clearpage
%% \mode<all>{
%%   \part{Problem Decomposition}
%% }
%% \begin{frame}
%%   \partpage
%% \end{frame}
%% \mode<all>{
%%   \input{../party/body}
%%   }

%% \clearpage
%% \mode<all>{
%%   \part{More Global Constraints}
%% }
%% \begin{frame}
%%   \partpage
%% \end{frame}
%% \mode<all>{
%%   \input{../car/body}
%%   }

%% \clearpage
%% \mode<all>{
%%   \part{Methodology}
%% }
%% \begin{frame}
%%   \partpage
%% \end{frame}
%% \mode<all>{
%%   \input{../methodology/body}
%%   }

\clearpage
\mode<all>{
\part{What is missing?}
}
\begin{frame}
\partpage
\end{frame}

\begin{frame}
\frametitle{Many Specialized Topics}
\begin{itemize}
\item How to design efficient core engine
\item Hybrids with LP/MIP tools
\item Hybrids with SAT
\item Symmetry breaking
\item Use of MDD/BDD to encode sets of solutions
\item High level modelling tools
\item Debugging/visualization
\end{itemize}
\end{frame}

\begin{frame}
\frametitle{Reformulation}
\begin{itemize}
\item Just because the user has modelled it this way, it doesn't mean we have to solve it that way
\begin{itemize}
\item Replace some constraint(s) by other, equivalent constraints
\item Because we don't have that constraint in our system
\item For performance
\end{itemize}
\end{itemize}
\end{frame}

\begin{frame}
\frametitle{Learning}
\begin{itemize}
\item While solving the problem we can learn how to strengthen the model/search 
\begin{itemize}
\item Understand which constraints/method contribute to propagation and change schedule
\item Learn no-good constraints by explaining failure
\item Adapt search strategy based on search experience
\end{itemize}
\end{itemize}
\end{frame}

%% \begin{frame}
%% \frametitle{Refined Process}
%% \begin{center}
%% \begin{tikzpicture}
%% \node[shape=rectangle,fill=pantone127-4] (problem) at (3,7) {Problem};
%% \node (human) at (3,6) {Human};
%% \node[shape=rectangle,fill=pantone127-4] (model) at (3,5) {User Model};
%% \node (reformulation) at (3,4) {Reformulation};
%% \node[shape=rectangle,fill=pantone127-4] (implementation) at (3,3) {Implementation Model};
%% \node (solver) at (3,2) {Constraint Solver/Search};
%% \node[shape=rectangle,fill=pantone157-8] (solution) at (3,1) {Solution};
%% \draw[-] (problem) -- (human);
%% \draw[->] (human) -- (model);
%% \draw[-] (model) -- (reformulation);
%% \draw[->] (reformulation) -- (implementation);
%% \draw[-] (implementation) -- (solver);
%% \draw[->] (solver) -- (0,2) -- (0,3) -- (implementation);
%% \draw[->] (solver) -- (solution);
%% \end{tikzpicture}
%% \end{center}
%% \end{frame}

%\begin{frame}
%\frametitle{What is CP actually used for?}
%\url{http://hsimonis.wordpress.com}
%\includegraphics[width=10cm]{../introduction/images/blog}
%\end{frame}

%% \begin{frame}
%% \frametitle{Exercises}
%% \begin{itemize}
%% \item Install Constraint Solver of your Choice
%% \item MiniZinc (MiningZinc) will be used in later sessions
%% \item Run Sendmoremoney, Sudoku, Queens examples
%% \item Can you choose which propagator to use for alldifferent?
%% \item Can you use different search strategies?
%% \end{itemize}
%% \end{frame}

\begin{frame}
\frametitle{More Learning Resources}
\begin{itemize}
\item Survey of Methods, Resources, and Formats for Teaching Constraint Programming
\begin{itemize}
\item by Tejas Santanam, Helmut Simonis
\item \url{https://doi.org/10.48550/arXiv.2403.12717}
\item Based on survey of community for WTCP 2023
\item \url{https://hsimonis.github.io/WTCP2023/}
\end{itemize}
\end{itemize}
\end{frame}


%%   }

%% \clearpage
%% \mode<all>{
%%   \part{Methodology}
%% }
%% \begin{frame}
%%   \partpage
%% \end{frame}
%% \mode<all>{
%%   \begin{frame}
\frametitle{Objectives}
\begin{itemize}
\item Overview of Core Constraint Programming
\item Three Main Concepts
\begin{itemize}
\item Constraint Propagation
\item Global Constraints
\item Customizing Search
\end{itemize}
%\item Get Some Experience with MiniZinc
\item Topics will be treated in more detail in later parts of the school
\item Based on Examples, not Formal Description
\end{itemize}
\end{frame}

\begin{frame}
\frametitle{Outline}
\begin{itemize}
\item Why Constraint Programming?
\item Constraint Propagation
\item Global Constraints
\item Customizing Search
%\item Choices in Modelling
%\item Decomposition
%\item More Global Constraints
%\item Methodology
%\item What is missing?
\end{itemize}
\end{frame}

% \begin{frame}
  % \frametitle{Using MiniZinc IDE}
  % \begin{itemize}
  % \item Developed in the Australian NICTA project
  % \item Maintained by Monash University
  % \item Modelling tool with multiple back-end solvers
  % \item Available from \url{https://www.minizinc.org/}
  % \end{itemize}
% \end{frame}


% \begin{frame}
% \frametitle{Examples in ECLiPSe}
% \begin{itemize}
% \item Open sourced constraint programming language
% \item Development goes back to 1985
% \item ECRC, ICL, IC-Parc, PTL, Cisco
% \item \url{https://eclipseclp.org/}
% \item Specialities
% \begin{itemize}
% \item Develop new solvers for specific domains
% \item Integration with MIP
% \end{itemize}
% \item Not included in bundled MiniZinc IDE
% \item Specialized visualization tools used here
  % \begin{itemize}
  % \item CP-Viz, Simonis et al. 2010
    % \end{itemize}
% \end{itemize}
% \end{frame}

\begin{frame}
\frametitle{Tutorial Based on ECLiPSe ELearning Course}
\begin{itemize}
\item Self-study course in constraint programming
\item Supported by Cisco Systems and Silicon Valley Community Foundation
\item Multi-media format, video lectures, slides, handout etc
\item {\small \url{https://eclipseclp.org/ELearning/index.html}}
\end{itemize}
\end{frame}

\begin{frame}
\frametitle{Also Part of CRT-AI Constraint Week}
\begin{itemize}
\item Annual one week course on CP and Optimization in Ireland
\item Part of national training program for PhD students in AI
\item \url{https://www.crt-ai.ie/}
\end{itemize}
\end{frame}



\begin{frame}
\frametitle{Constraint Programming - in a nutshell}
\begin{itemize}
\item Declarative description of problems with
\begin{itemize}
\item {\em Variables} which range over (finite) sets of values
\item {\em Constraints} over subsets of variables which restrict possible value combinations
\item A {\em solution} is a value assignment which satisfies all constraints
\end{itemize}

\item Constraint propagation/reasoning
\begin{itemize}
\item Removing inconsistent values for variables
\item Detect failure if constraint can not be satisfied
\item Interaction of constraints via shared variables
\item Incomplete
\end{itemize}

\item Search
\begin{itemize}
\item User controlled assignment of values to variables
\item Each step triggers constraint propagation 
\end{itemize}
\item Different domains require/allow different methods
\end{itemize}
\end{frame}

\begin{frame}
  \frametitle{Constraint Programming is Different}
  \begin{itemize}
  \item Declarative Programming
    \begin{itemize}
    \item Concentrate on what you want
      \item Not how to get there
      \item Program != Algorithm
      \item Program = Model
    \end{itemize}
    \item Applied to Combinatorial Problems
      \begin{itemize}
        \item No complete polynomial algorithms known (exist?)
        \item CP less ad-hoc than heuristics
        \item Models can evolve
  \end{itemize}
  \end{itemize}
  \end{frame}
    
\begin{frame}
  \frametitle{A Subtractive Process}
  \begin{textblock}{4}(8,-3)
    \includegraphics[width=4cm]{../introduction/images/stages}
  \end{textblock}
  \vfill
  \begin{quote}
    ``Oh, bosh, as Mr. Ruskin says. Sculpture, per se, is the simplest thing in the world. All you have to do is to take a big chunk of marble and a hammer and chisel, make up your mind what you are about to create and chip off all the marble you don't want.''-Paris Gaulois.
  \end{quote}
  
  {\tiny Source: \url{https://quoteinvestigator.com/2014/06/22/chip-away/}}
\end{frame}

\begin{frame}
\frametitle{Basic Process}
\begin{center}
\begin{tikzpicture}
\node[shape=rectangle,fill=pantone127-4] (problem) at (2,5) {Problem};
\node (human) at (2,4) {Human};
\node[shape=rectangle,fill=pantone127-4] (model) at (2,3) {Model};
\node (solver) at (2,2) {Constraint Solver/Search};
\node[shape=rectangle,fill=pantone157-8] (solution) at (2,1) {Solution};
\draw[-] (problem) -- (human);
\draw[->] (human) -- (model);
\draw[-] (model) -- (solver);
\draw[->] (solver) -- (solution);
\end{tikzpicture}
\end{center}
\end{frame}

\begin{frame}
\frametitle{More Realistic}
\begin{center}
\begin{tikzpicture}[xscale=4.0]
\node[shape=rectangle,fill=pantone127-4] (problem) at (2,5) {Problem};
\node (human) at (2,4) {Human};
\node[shape=rectangle,fill=pantone127-4] (model) at (2,3) {Model};
\node (solver) at (2,2) {Constraint Solver/Search};
\node[shape=rectangle,fill=pantone24-8] (hangs) at (1,1) {Hangs};
\node[shape=rectangle,fill=pantone157-8] (solution) at (2,1) {Solution};
\node[shape=rectangle,fill=pantone24-8] (wrong) at (3,1) {Wrong Solution};
\draw[-] (problem) -- (human);
\draw[->] (human) -- (model);
\draw[-] (model) -- (solver);
\draw[->] (solver) -- (hangs);
\draw[->] (solver) -- (solution);
\draw[->] (solver) -- (wrong);
\draw[->] (hangs) -- (1,4) -- (human);
\draw[->] (wrong) -- (3,4) -- (human);
\end{tikzpicture}
\end{center}
\end{frame}

\begin{frame}
\frametitle{Dual Role of Model}
\begin{itemize}
\item Allows Human to Express Problem
\begin{itemize}
\item Close to Problem Domain
\item Constraints as Abstractions
\end{itemize}
\item Allows Solver to Execute
\begin{itemize}
\item Variables as Communication Mechanism
\item Constraints as Algorithms
\end{itemize}
\end{itemize}
\end{frame}

\begin{frame}
\frametitle{Modelling Frameworks}
\begin{itemize}
\item \href{https://www.minizinc.org/}{MiniZinc} (NICTA, Monash University, Australia)
\item \href{https://github.com/eomahony/Numberjack}{NumberJack} (Insight, Ireland) 
\item \href{https://www-users.york.ac.uk/peter.nightingale/savilerow//}{EssencePrime/SavilleRow} (UK) 
\item \href{https://cpmpy.readthedocs.io/en/latest/}{CPMpy} (KU Leuven)
\item Allow use of multiple back-end solvers
\item Compile model into variants for each solver
\item A priori solver independent model(CP, MIP, SAT)
\end{itemize}
\end{frame} 

\begin{frame}
\frametitle{Framework Process}
\begin{center}
\begin{tikzpicture}[xscale=2]
\node[shape=rectangle,fill=pantone127-4] (problem) at (2.5,5) {Problem};
\node (human) at (2.5,4) {Human};
\node[shape=rectangle,fill=pantone127-4] (model) at (2.5,3) {Model};
\node (compile) at (2.5,2) {Compile/Reformulate};
\node (solver1) at (1,1) {CP};
\node (solver2) at (2,1) {MIP};
\node (solver3) at (3,1) {SAT};
\node (solver4) at (4,1) {Other};
\node[shape=rectangle,fill=pantone157-8] (solution1) at (1,0) {Solution};
\node[shape=rectangle,fill=pantone157-8] (solution2) at (2,0) {Solution};
\node[shape=rectangle,fill=pantone157-8] (solution3) at (3,0) {Solution};
\node[shape=rectangle,fill=pantone157-8] (solution4) at (4,0) {Solution};
\draw[-] (problem) -- (human);
\draw[->] (human) -- (model);
\draw[-] (model) -- (compile);
\draw[-] (compile) -- (solver1);
\draw[-] (compile) -- (solver2);
\draw[-] (compile) -- (solver3);
\draw[-] (compile) -- (solver4);
\draw[->] (solver1) -- (solution1);
\draw[->] (solver2) -- (solution2);
\draw[->] (solver3) -- (solution3);
\draw[->] (solver4) -- (solution4);
\end{tikzpicture}
\end{center}
\end{frame}

% \begin{frame}
  % \frametitle{Do It Now!}
  % \begin{itemize}
  % \item Download and install Minizinc
  % \item \url{https://www.minizinc.org/}
  % \end{itemize}
% \end{frame}


\begin{frame}
\frametitle{Why use Puzzles as Examples?}
\begin{itemize}
\item Easy to understand the problem
\item Solvable by hand without specialized knowledge
\item Possible to compare automated to manual solving process
\end{itemize}

\begin{quote}
The puzzle, though inanimate, is presented as a solvable problem without lasting negative consequences, a very low-risk low-reward situation. By being a puzzle, the object is attempting to convince the user that it must be completed.
\end{quote}
{\tiny Source: \href{https://www.everydayrhetoric.com/2020/03/the-rhetoric-of-puzzles/}{Every Day Rhetoric}}
\end{frame}


\mode<all>{
\part{Basic Constraint Propagation}
}
\begin{frame}
\partpage
\end{frame}
\mode<all>{
\input{../sendmore/body}
}

\mode<all>{
\part{Global Constraints}
}
\begin{frame}
\partpage
\end{frame}
\mode<all>{
\input{../sudoku/body}
}

\clearpage
\mode<all>{
\part{Customizing Search}
}
\begin{frame}
\partpage
\end{frame}
\mode<all>{
\input{../nqueen/body}
}


%% \clearpage
%% \mode<all>{
%%   \part{Problem Decomposition}
%% }
%% \begin{frame}
%%   \partpage
%% \end{frame}
%% \mode<all>{
%%   \input{../party/body}
%%   }

%% \clearpage
%% \mode<all>{
%%   \part{More Global Constraints}
%% }
%% \begin{frame}
%%   \partpage
%% \end{frame}
%% \mode<all>{
%%   \input{../car/body}
%%   }

%% \clearpage
%% \mode<all>{
%%   \part{Methodology}
%% }
%% \begin{frame}
%%   \partpage
%% \end{frame}
%% \mode<all>{
%%   \input{../methodology/body}
%%   }

\clearpage
\mode<all>{
\part{What is missing?}
}
\begin{frame}
\partpage
\end{frame}

\begin{frame}
\frametitle{Many Specialized Topics}
\begin{itemize}
\item How to design efficient core engine
\item Hybrids with LP/MIP tools
\item Hybrids with SAT
\item Symmetry breaking
\item Use of MDD/BDD to encode sets of solutions
\item High level modelling tools
\item Debugging/visualization
\end{itemize}
\end{frame}

\begin{frame}
\frametitle{Reformulation}
\begin{itemize}
\item Just because the user has modelled it this way, it doesn't mean we have to solve it that way
\begin{itemize}
\item Replace some constraint(s) by other, equivalent constraints
\item Because we don't have that constraint in our system
\item For performance
\end{itemize}
\end{itemize}
\end{frame}

\begin{frame}
\frametitle{Learning}
\begin{itemize}
\item While solving the problem we can learn how to strengthen the model/search 
\begin{itemize}
\item Understand which constraints/method contribute to propagation and change schedule
\item Learn no-good constraints by explaining failure
\item Adapt search strategy based on search experience
\end{itemize}
\end{itemize}
\end{frame}

%% \begin{frame}
%% \frametitle{Refined Process}
%% \begin{center}
%% \begin{tikzpicture}
%% \node[shape=rectangle,fill=pantone127-4] (problem) at (3,7) {Problem};
%% \node (human) at (3,6) {Human};
%% \node[shape=rectangle,fill=pantone127-4] (model) at (3,5) {User Model};
%% \node (reformulation) at (3,4) {Reformulation};
%% \node[shape=rectangle,fill=pantone127-4] (implementation) at (3,3) {Implementation Model};
%% \node (solver) at (3,2) {Constraint Solver/Search};
%% \node[shape=rectangle,fill=pantone157-8] (solution) at (3,1) {Solution};
%% \draw[-] (problem) -- (human);
%% \draw[->] (human) -- (model);
%% \draw[-] (model) -- (reformulation);
%% \draw[->] (reformulation) -- (implementation);
%% \draw[-] (implementation) -- (solver);
%% \draw[->] (solver) -- (0,2) -- (0,3) -- (implementation);
%% \draw[->] (solver) -- (solution);
%% \end{tikzpicture}
%% \end{center}
%% \end{frame}

%\begin{frame}
%\frametitle{What is CP actually used for?}
%\url{http://hsimonis.wordpress.com}
%\includegraphics[width=10cm]{../introduction/images/blog}
%\end{frame}

%% \begin{frame}
%% \frametitle{Exercises}
%% \begin{itemize}
%% \item Install Constraint Solver of your Choice
%% \item MiniZinc (MiningZinc) will be used in later sessions
%% \item Run Sendmoremoney, Sudoku, Queens examples
%% \item Can you choose which propagator to use for alldifferent?
%% \item Can you use different search strategies?
%% \end{itemize}
%% \end{frame}

\begin{frame}
\frametitle{More Learning Resources}
\begin{itemize}
\item Survey of Methods, Resources, and Formats for Teaching Constraint Programming
\begin{itemize}
\item by Tejas Santanam, Helmut Simonis
\item \url{https://doi.org/10.48550/arXiv.2403.12717}
\item Based on survey of community for WTCP 2023
\item \url{https://hsimonis.github.io/WTCP2023/}
\end{itemize}
\end{itemize}
\end{frame}


%%   }

\clearpage
\mode<all>{
\part{What is missing?}
}
\begin{frame}
\partpage
\end{frame}

\begin{frame}
\frametitle{Many Specialized Topics}
\begin{itemize}
\item How to design efficient core engine
\item Hybrids with LP/MIP tools
\item Hybrids with SAT
\item Symmetry breaking
\item Use of MDD/BDD to encode sets of solutions
\item High level modelling tools
\item Debugging/visualization
\end{itemize}
\end{frame}

\begin{frame}
\frametitle{Reformulation}
\begin{itemize}
\item Just because the user has modelled it this way, it doesn't mean we have to solve it that way
\begin{itemize}
\item Replace some constraint(s) by other, equivalent constraints
\item Because we don't have that constraint in our system
\item For performance
\end{itemize}
\end{itemize}
\end{frame}

\begin{frame}
\frametitle{Learning}
\begin{itemize}
\item While solving the problem we can learn how to strengthen the model/search 
\begin{itemize}
\item Understand which constraints/method contribute to propagation and change schedule
\item Learn no-good constraints by explaining failure
\item Adapt search strategy based on search experience
\end{itemize}
\end{itemize}
\end{frame}

%% \begin{frame}
%% \frametitle{Refined Process}
%% \begin{center}
%% \begin{tikzpicture}
%% \node[shape=rectangle,fill=pantone127-4] (problem) at (3,7) {Problem};
%% \node (human) at (3,6) {Human};
%% \node[shape=rectangle,fill=pantone127-4] (model) at (3,5) {User Model};
%% \node (reformulation) at (3,4) {Reformulation};
%% \node[shape=rectangle,fill=pantone127-4] (implementation) at (3,3) {Implementation Model};
%% \node (solver) at (3,2) {Constraint Solver/Search};
%% \node[shape=rectangle,fill=pantone157-8] (solution) at (3,1) {Solution};
%% \draw[-] (problem) -- (human);
%% \draw[->] (human) -- (model);
%% \draw[-] (model) -- (reformulation);
%% \draw[->] (reformulation) -- (implementation);
%% \draw[-] (implementation) -- (solver);
%% \draw[->] (solver) -- (0,2) -- (0,3) -- (implementation);
%% \draw[->] (solver) -- (solution);
%% \end{tikzpicture}
%% \end{center}
%% \end{frame}

%\begin{frame}
%\frametitle{What is CP actually used for?}
%\url{http://hsimonis.wordpress.com}
%\includegraphics[width=10cm]{../introduction/images/blog}
%\end{frame}

%% \begin{frame}
%% \frametitle{Exercises}
%% \begin{itemize}
%% \item Install Constraint Solver of your Choice
%% \item MiniZinc (MiningZinc) will be used in later sessions
%% \item Run Sendmoremoney, Sudoku, Queens examples
%% \item Can you choose which propagator to use for alldifferent?
%% \item Can you use different search strategies?
%% \end{itemize}
%% \end{frame}

\begin{frame}
\frametitle{More Learning Resources}
\begin{itemize}
\item Survey of Methods, Resources, and Formats for Teaching Constraint Programming
\begin{itemize}
\item by Tejas Santanam, Helmut Simonis
\item \url{https://doi.org/10.48550/arXiv.2403.12717}
\item Based on survey of community for WTCP 2023
\item \url{https://hsimonis.github.io/WTCP2023/}
\end{itemize}
\end{itemize}
\end{frame}


}


%% \clearpage
%% \mode<all>{
%%   \part{Problem Decomposition}
%% }
%% \begin{frame}
%%   \partpage
%% \end{frame}
%% \mode<all>{
%%   \begin{frame}
\frametitle{Objectives}
\begin{itemize}
\item Overview of Core Constraint Programming
\item Three Main Concepts
\begin{itemize}
\item Constraint Propagation
\item Global Constraints
\item Customizing Search
\end{itemize}
%\item Get Some Experience with MiniZinc
\item Topics will be treated in more detail in later parts of the school
\item Based on Examples, not Formal Description
\end{itemize}
\end{frame}

\begin{frame}
\frametitle{Outline}
\begin{itemize}
\item Why Constraint Programming?
\item Constraint Propagation
\item Global Constraints
\item Customizing Search
%\item Choices in Modelling
%\item Decomposition
%\item More Global Constraints
%\item Methodology
%\item What is missing?
\end{itemize}
\end{frame}

% \begin{frame}
  % \frametitle{Using MiniZinc IDE}
  % \begin{itemize}
  % \item Developed in the Australian NICTA project
  % \item Maintained by Monash University
  % \item Modelling tool with multiple back-end solvers
  % \item Available from \url{https://www.minizinc.org/}
  % \end{itemize}
% \end{frame}


% \begin{frame}
% \frametitle{Examples in ECLiPSe}
% \begin{itemize}
% \item Open sourced constraint programming language
% \item Development goes back to 1985
% \item ECRC, ICL, IC-Parc, PTL, Cisco
% \item \url{https://eclipseclp.org/}
% \item Specialities
% \begin{itemize}
% \item Develop new solvers for specific domains
% \item Integration with MIP
% \end{itemize}
% \item Not included in bundled MiniZinc IDE
% \item Specialized visualization tools used here
  % \begin{itemize}
  % \item CP-Viz, Simonis et al. 2010
    % \end{itemize}
% \end{itemize}
% \end{frame}

\begin{frame}
\frametitle{Tutorial Based on ECLiPSe ELearning Course}
\begin{itemize}
\item Self-study course in constraint programming
\item Supported by Cisco Systems and Silicon Valley Community Foundation
\item Multi-media format, video lectures, slides, handout etc
\item {\small \url{https://eclipseclp.org/ELearning/index.html}}
\end{itemize}
\end{frame}

\begin{frame}
\frametitle{Also Part of CRT-AI Constraint Week}
\begin{itemize}
\item Annual one week course on CP and Optimization in Ireland
\item Part of national training program for PhD students in AI
\item \url{https://www.crt-ai.ie/}
\end{itemize}
\end{frame}



\begin{frame}
\frametitle{Constraint Programming - in a nutshell}
\begin{itemize}
\item Declarative description of problems with
\begin{itemize}
\item {\em Variables} which range over (finite) sets of values
\item {\em Constraints} over subsets of variables which restrict possible value combinations
\item A {\em solution} is a value assignment which satisfies all constraints
\end{itemize}

\item Constraint propagation/reasoning
\begin{itemize}
\item Removing inconsistent values for variables
\item Detect failure if constraint can not be satisfied
\item Interaction of constraints via shared variables
\item Incomplete
\end{itemize}

\item Search
\begin{itemize}
\item User controlled assignment of values to variables
\item Each step triggers constraint propagation 
\end{itemize}
\item Different domains require/allow different methods
\end{itemize}
\end{frame}

\begin{frame}
  \frametitle{Constraint Programming is Different}
  \begin{itemize}
  \item Declarative Programming
    \begin{itemize}
    \item Concentrate on what you want
      \item Not how to get there
      \item Program != Algorithm
      \item Program = Model
    \end{itemize}
    \item Applied to Combinatorial Problems
      \begin{itemize}
        \item No complete polynomial algorithms known (exist?)
        \item CP less ad-hoc than heuristics
        \item Models can evolve
  \end{itemize}
  \end{itemize}
  \end{frame}
    
\begin{frame}
  \frametitle{A Subtractive Process}
  \begin{textblock}{4}(8,-3)
    \includegraphics[width=4cm]{../introduction/images/stages}
  \end{textblock}
  \vfill
  \begin{quote}
    ``Oh, bosh, as Mr. Ruskin says. Sculpture, per se, is the simplest thing in the world. All you have to do is to take a big chunk of marble and a hammer and chisel, make up your mind what you are about to create and chip off all the marble you don't want.''-Paris Gaulois.
  \end{quote}
  
  {\tiny Source: \url{https://quoteinvestigator.com/2014/06/22/chip-away/}}
\end{frame}

\begin{frame}
\frametitle{Basic Process}
\begin{center}
\begin{tikzpicture}
\node[shape=rectangle,fill=pantone127-4] (problem) at (2,5) {Problem};
\node (human) at (2,4) {Human};
\node[shape=rectangle,fill=pantone127-4] (model) at (2,3) {Model};
\node (solver) at (2,2) {Constraint Solver/Search};
\node[shape=rectangle,fill=pantone157-8] (solution) at (2,1) {Solution};
\draw[-] (problem) -- (human);
\draw[->] (human) -- (model);
\draw[-] (model) -- (solver);
\draw[->] (solver) -- (solution);
\end{tikzpicture}
\end{center}
\end{frame}

\begin{frame}
\frametitle{More Realistic}
\begin{center}
\begin{tikzpicture}[xscale=4.0]
\node[shape=rectangle,fill=pantone127-4] (problem) at (2,5) {Problem};
\node (human) at (2,4) {Human};
\node[shape=rectangle,fill=pantone127-4] (model) at (2,3) {Model};
\node (solver) at (2,2) {Constraint Solver/Search};
\node[shape=rectangle,fill=pantone24-8] (hangs) at (1,1) {Hangs};
\node[shape=rectangle,fill=pantone157-8] (solution) at (2,1) {Solution};
\node[shape=rectangle,fill=pantone24-8] (wrong) at (3,1) {Wrong Solution};
\draw[-] (problem) -- (human);
\draw[->] (human) -- (model);
\draw[-] (model) -- (solver);
\draw[->] (solver) -- (hangs);
\draw[->] (solver) -- (solution);
\draw[->] (solver) -- (wrong);
\draw[->] (hangs) -- (1,4) -- (human);
\draw[->] (wrong) -- (3,4) -- (human);
\end{tikzpicture}
\end{center}
\end{frame}

\begin{frame}
\frametitle{Dual Role of Model}
\begin{itemize}
\item Allows Human to Express Problem
\begin{itemize}
\item Close to Problem Domain
\item Constraints as Abstractions
\end{itemize}
\item Allows Solver to Execute
\begin{itemize}
\item Variables as Communication Mechanism
\item Constraints as Algorithms
\end{itemize}
\end{itemize}
\end{frame}

\begin{frame}
\frametitle{Modelling Frameworks}
\begin{itemize}
\item \href{https://www.minizinc.org/}{MiniZinc} (NICTA, Monash University, Australia)
\item \href{https://github.com/eomahony/Numberjack}{NumberJack} (Insight, Ireland) 
\item \href{https://www-users.york.ac.uk/peter.nightingale/savilerow//}{EssencePrime/SavilleRow} (UK) 
\item \href{https://cpmpy.readthedocs.io/en/latest/}{CPMpy} (KU Leuven)
\item Allow use of multiple back-end solvers
\item Compile model into variants for each solver
\item A priori solver independent model(CP, MIP, SAT)
\end{itemize}
\end{frame} 

\begin{frame}
\frametitle{Framework Process}
\begin{center}
\begin{tikzpicture}[xscale=2]
\node[shape=rectangle,fill=pantone127-4] (problem) at (2.5,5) {Problem};
\node (human) at (2.5,4) {Human};
\node[shape=rectangle,fill=pantone127-4] (model) at (2.5,3) {Model};
\node (compile) at (2.5,2) {Compile/Reformulate};
\node (solver1) at (1,1) {CP};
\node (solver2) at (2,1) {MIP};
\node (solver3) at (3,1) {SAT};
\node (solver4) at (4,1) {Other};
\node[shape=rectangle,fill=pantone157-8] (solution1) at (1,0) {Solution};
\node[shape=rectangle,fill=pantone157-8] (solution2) at (2,0) {Solution};
\node[shape=rectangle,fill=pantone157-8] (solution3) at (3,0) {Solution};
\node[shape=rectangle,fill=pantone157-8] (solution4) at (4,0) {Solution};
\draw[-] (problem) -- (human);
\draw[->] (human) -- (model);
\draw[-] (model) -- (compile);
\draw[-] (compile) -- (solver1);
\draw[-] (compile) -- (solver2);
\draw[-] (compile) -- (solver3);
\draw[-] (compile) -- (solver4);
\draw[->] (solver1) -- (solution1);
\draw[->] (solver2) -- (solution2);
\draw[->] (solver3) -- (solution3);
\draw[->] (solver4) -- (solution4);
\end{tikzpicture}
\end{center}
\end{frame}

% \begin{frame}
  % \frametitle{Do It Now!}
  % \begin{itemize}
  % \item Download and install Minizinc
  % \item \url{https://www.minizinc.org/}
  % \end{itemize}
% \end{frame}


\begin{frame}
\frametitle{Why use Puzzles as Examples?}
\begin{itemize}
\item Easy to understand the problem
\item Solvable by hand without specialized knowledge
\item Possible to compare automated to manual solving process
\end{itemize}

\begin{quote}
The puzzle, though inanimate, is presented as a solvable problem without lasting negative consequences, a very low-risk low-reward situation. By being a puzzle, the object is attempting to convince the user that it must be completed.
\end{quote}
{\tiny Source: \href{https://www.everydayrhetoric.com/2020/03/the-rhetoric-of-puzzles/}{Every Day Rhetoric}}
\end{frame}


\mode<all>{
\part{Basic Constraint Propagation}
}
\begin{frame}
\partpage
\end{frame}
\mode<all>{
\begin{frame}
\frametitle{Objectives}
\begin{itemize}
\item Overview of Core Constraint Programming
\item Three Main Concepts
\begin{itemize}
\item Constraint Propagation
\item Global Constraints
\item Customizing Search
\end{itemize}
%\item Get Some Experience with MiniZinc
\item Topics will be treated in more detail in later parts of the school
\item Based on Examples, not Formal Description
\end{itemize}
\end{frame}

\begin{frame}
\frametitle{Outline}
\begin{itemize}
\item Why Constraint Programming?
\item Constraint Propagation
\item Global Constraints
\item Customizing Search
%\item Choices in Modelling
%\item Decomposition
%\item More Global Constraints
%\item Methodology
%\item What is missing?
\end{itemize}
\end{frame}

% \begin{frame}
  % \frametitle{Using MiniZinc IDE}
  % \begin{itemize}
  % \item Developed in the Australian NICTA project
  % \item Maintained by Monash University
  % \item Modelling tool with multiple back-end solvers
  % \item Available from \url{https://www.minizinc.org/}
  % \end{itemize}
% \end{frame}


% \begin{frame}
% \frametitle{Examples in ECLiPSe}
% \begin{itemize}
% \item Open sourced constraint programming language
% \item Development goes back to 1985
% \item ECRC, ICL, IC-Parc, PTL, Cisco
% \item \url{https://eclipseclp.org/}
% \item Specialities
% \begin{itemize}
% \item Develop new solvers for specific domains
% \item Integration with MIP
% \end{itemize}
% \item Not included in bundled MiniZinc IDE
% \item Specialized visualization tools used here
  % \begin{itemize}
  % \item CP-Viz, Simonis et al. 2010
    % \end{itemize}
% \end{itemize}
% \end{frame}

\begin{frame}
\frametitle{Tutorial Based on ECLiPSe ELearning Course}
\begin{itemize}
\item Self-study course in constraint programming
\item Supported by Cisco Systems and Silicon Valley Community Foundation
\item Multi-media format, video lectures, slides, handout etc
\item {\small \url{https://eclipseclp.org/ELearning/index.html}}
\end{itemize}
\end{frame}

\begin{frame}
\frametitle{Also Part of CRT-AI Constraint Week}
\begin{itemize}
\item Annual one week course on CP and Optimization in Ireland
\item Part of national training program for PhD students in AI
\item \url{https://www.crt-ai.ie/}
\end{itemize}
\end{frame}



\begin{frame}
\frametitle{Constraint Programming - in a nutshell}
\begin{itemize}
\item Declarative description of problems with
\begin{itemize}
\item {\em Variables} which range over (finite) sets of values
\item {\em Constraints} over subsets of variables which restrict possible value combinations
\item A {\em solution} is a value assignment which satisfies all constraints
\end{itemize}

\item Constraint propagation/reasoning
\begin{itemize}
\item Removing inconsistent values for variables
\item Detect failure if constraint can not be satisfied
\item Interaction of constraints via shared variables
\item Incomplete
\end{itemize}

\item Search
\begin{itemize}
\item User controlled assignment of values to variables
\item Each step triggers constraint propagation 
\end{itemize}
\item Different domains require/allow different methods
\end{itemize}
\end{frame}

\begin{frame}
  \frametitle{Constraint Programming is Different}
  \begin{itemize}
  \item Declarative Programming
    \begin{itemize}
    \item Concentrate on what you want
      \item Not how to get there
      \item Program != Algorithm
      \item Program = Model
    \end{itemize}
    \item Applied to Combinatorial Problems
      \begin{itemize}
        \item No complete polynomial algorithms known (exist?)
        \item CP less ad-hoc than heuristics
        \item Models can evolve
  \end{itemize}
  \end{itemize}
  \end{frame}
    
\begin{frame}
  \frametitle{A Subtractive Process}
  \begin{textblock}{4}(8,-3)
    \includegraphics[width=4cm]{../introduction/images/stages}
  \end{textblock}
  \vfill
  \begin{quote}
    ``Oh, bosh, as Mr. Ruskin says. Sculpture, per se, is the simplest thing in the world. All you have to do is to take a big chunk of marble and a hammer and chisel, make up your mind what you are about to create and chip off all the marble you don't want.''-Paris Gaulois.
  \end{quote}
  
  {\tiny Source: \url{https://quoteinvestigator.com/2014/06/22/chip-away/}}
\end{frame}

\begin{frame}
\frametitle{Basic Process}
\begin{center}
\begin{tikzpicture}
\node[shape=rectangle,fill=pantone127-4] (problem) at (2,5) {Problem};
\node (human) at (2,4) {Human};
\node[shape=rectangle,fill=pantone127-4] (model) at (2,3) {Model};
\node (solver) at (2,2) {Constraint Solver/Search};
\node[shape=rectangle,fill=pantone157-8] (solution) at (2,1) {Solution};
\draw[-] (problem) -- (human);
\draw[->] (human) -- (model);
\draw[-] (model) -- (solver);
\draw[->] (solver) -- (solution);
\end{tikzpicture}
\end{center}
\end{frame}

\begin{frame}
\frametitle{More Realistic}
\begin{center}
\begin{tikzpicture}[xscale=4.0]
\node[shape=rectangle,fill=pantone127-4] (problem) at (2,5) {Problem};
\node (human) at (2,4) {Human};
\node[shape=rectangle,fill=pantone127-4] (model) at (2,3) {Model};
\node (solver) at (2,2) {Constraint Solver/Search};
\node[shape=rectangle,fill=pantone24-8] (hangs) at (1,1) {Hangs};
\node[shape=rectangle,fill=pantone157-8] (solution) at (2,1) {Solution};
\node[shape=rectangle,fill=pantone24-8] (wrong) at (3,1) {Wrong Solution};
\draw[-] (problem) -- (human);
\draw[->] (human) -- (model);
\draw[-] (model) -- (solver);
\draw[->] (solver) -- (hangs);
\draw[->] (solver) -- (solution);
\draw[->] (solver) -- (wrong);
\draw[->] (hangs) -- (1,4) -- (human);
\draw[->] (wrong) -- (3,4) -- (human);
\end{tikzpicture}
\end{center}
\end{frame}

\begin{frame}
\frametitle{Dual Role of Model}
\begin{itemize}
\item Allows Human to Express Problem
\begin{itemize}
\item Close to Problem Domain
\item Constraints as Abstractions
\end{itemize}
\item Allows Solver to Execute
\begin{itemize}
\item Variables as Communication Mechanism
\item Constraints as Algorithms
\end{itemize}
\end{itemize}
\end{frame}

\begin{frame}
\frametitle{Modelling Frameworks}
\begin{itemize}
\item \href{https://www.minizinc.org/}{MiniZinc} (NICTA, Monash University, Australia)
\item \href{https://github.com/eomahony/Numberjack}{NumberJack} (Insight, Ireland) 
\item \href{https://www-users.york.ac.uk/peter.nightingale/savilerow//}{EssencePrime/SavilleRow} (UK) 
\item \href{https://cpmpy.readthedocs.io/en/latest/}{CPMpy} (KU Leuven)
\item Allow use of multiple back-end solvers
\item Compile model into variants for each solver
\item A priori solver independent model(CP, MIP, SAT)
\end{itemize}
\end{frame} 

\begin{frame}
\frametitle{Framework Process}
\begin{center}
\begin{tikzpicture}[xscale=2]
\node[shape=rectangle,fill=pantone127-4] (problem) at (2.5,5) {Problem};
\node (human) at (2.5,4) {Human};
\node[shape=rectangle,fill=pantone127-4] (model) at (2.5,3) {Model};
\node (compile) at (2.5,2) {Compile/Reformulate};
\node (solver1) at (1,1) {CP};
\node (solver2) at (2,1) {MIP};
\node (solver3) at (3,1) {SAT};
\node (solver4) at (4,1) {Other};
\node[shape=rectangle,fill=pantone157-8] (solution1) at (1,0) {Solution};
\node[shape=rectangle,fill=pantone157-8] (solution2) at (2,0) {Solution};
\node[shape=rectangle,fill=pantone157-8] (solution3) at (3,0) {Solution};
\node[shape=rectangle,fill=pantone157-8] (solution4) at (4,0) {Solution};
\draw[-] (problem) -- (human);
\draw[->] (human) -- (model);
\draw[-] (model) -- (compile);
\draw[-] (compile) -- (solver1);
\draw[-] (compile) -- (solver2);
\draw[-] (compile) -- (solver3);
\draw[-] (compile) -- (solver4);
\draw[->] (solver1) -- (solution1);
\draw[->] (solver2) -- (solution2);
\draw[->] (solver3) -- (solution3);
\draw[->] (solver4) -- (solution4);
\end{tikzpicture}
\end{center}
\end{frame}

% \begin{frame}
  % \frametitle{Do It Now!}
  % \begin{itemize}
  % \item Download and install Minizinc
  % \item \url{https://www.minizinc.org/}
  % \end{itemize}
% \end{frame}


\begin{frame}
\frametitle{Why use Puzzles as Examples?}
\begin{itemize}
\item Easy to understand the problem
\item Solvable by hand without specialized knowledge
\item Possible to compare automated to manual solving process
\end{itemize}

\begin{quote}
The puzzle, though inanimate, is presented as a solvable problem without lasting negative consequences, a very low-risk low-reward situation. By being a puzzle, the object is attempting to convince the user that it must be completed.
\end{quote}
{\tiny Source: \href{https://www.everydayrhetoric.com/2020/03/the-rhetoric-of-puzzles/}{Every Day Rhetoric}}
\end{frame}


\mode<all>{
\part{Basic Constraint Propagation}
}
\begin{frame}
\partpage
\end{frame}
\mode<all>{
\input{../sendmore/body}
}

\mode<all>{
\part{Global Constraints}
}
\begin{frame}
\partpage
\end{frame}
\mode<all>{
\input{../sudoku/body}
}

\clearpage
\mode<all>{
\part{Customizing Search}
}
\begin{frame}
\partpage
\end{frame}
\mode<all>{
\input{../nqueen/body}
}


%% \clearpage
%% \mode<all>{
%%   \part{Problem Decomposition}
%% }
%% \begin{frame}
%%   \partpage
%% \end{frame}
%% \mode<all>{
%%   \input{../party/body}
%%   }

%% \clearpage
%% \mode<all>{
%%   \part{More Global Constraints}
%% }
%% \begin{frame}
%%   \partpage
%% \end{frame}
%% \mode<all>{
%%   \input{../car/body}
%%   }

%% \clearpage
%% \mode<all>{
%%   \part{Methodology}
%% }
%% \begin{frame}
%%   \partpage
%% \end{frame}
%% \mode<all>{
%%   \input{../methodology/body}
%%   }

\clearpage
\mode<all>{
\part{What is missing?}
}
\begin{frame}
\partpage
\end{frame}

\begin{frame}
\frametitle{Many Specialized Topics}
\begin{itemize}
\item How to design efficient core engine
\item Hybrids with LP/MIP tools
\item Hybrids with SAT
\item Symmetry breaking
\item Use of MDD/BDD to encode sets of solutions
\item High level modelling tools
\item Debugging/visualization
\end{itemize}
\end{frame}

\begin{frame}
\frametitle{Reformulation}
\begin{itemize}
\item Just because the user has modelled it this way, it doesn't mean we have to solve it that way
\begin{itemize}
\item Replace some constraint(s) by other, equivalent constraints
\item Because we don't have that constraint in our system
\item For performance
\end{itemize}
\end{itemize}
\end{frame}

\begin{frame}
\frametitle{Learning}
\begin{itemize}
\item While solving the problem we can learn how to strengthen the model/search 
\begin{itemize}
\item Understand which constraints/method contribute to propagation and change schedule
\item Learn no-good constraints by explaining failure
\item Adapt search strategy based on search experience
\end{itemize}
\end{itemize}
\end{frame}

%% \begin{frame}
%% \frametitle{Refined Process}
%% \begin{center}
%% \begin{tikzpicture}
%% \node[shape=rectangle,fill=pantone127-4] (problem) at (3,7) {Problem};
%% \node (human) at (3,6) {Human};
%% \node[shape=rectangle,fill=pantone127-4] (model) at (3,5) {User Model};
%% \node (reformulation) at (3,4) {Reformulation};
%% \node[shape=rectangle,fill=pantone127-4] (implementation) at (3,3) {Implementation Model};
%% \node (solver) at (3,2) {Constraint Solver/Search};
%% \node[shape=rectangle,fill=pantone157-8] (solution) at (3,1) {Solution};
%% \draw[-] (problem) -- (human);
%% \draw[->] (human) -- (model);
%% \draw[-] (model) -- (reformulation);
%% \draw[->] (reformulation) -- (implementation);
%% \draw[-] (implementation) -- (solver);
%% \draw[->] (solver) -- (0,2) -- (0,3) -- (implementation);
%% \draw[->] (solver) -- (solution);
%% \end{tikzpicture}
%% \end{center}
%% \end{frame}

%\begin{frame}
%\frametitle{What is CP actually used for?}
%\url{http://hsimonis.wordpress.com}
%\includegraphics[width=10cm]{../introduction/images/blog}
%\end{frame}

%% \begin{frame}
%% \frametitle{Exercises}
%% \begin{itemize}
%% \item Install Constraint Solver of your Choice
%% \item MiniZinc (MiningZinc) will be used in later sessions
%% \item Run Sendmoremoney, Sudoku, Queens examples
%% \item Can you choose which propagator to use for alldifferent?
%% \item Can you use different search strategies?
%% \end{itemize}
%% \end{frame}

\begin{frame}
\frametitle{More Learning Resources}
\begin{itemize}
\item Survey of Methods, Resources, and Formats for Teaching Constraint Programming
\begin{itemize}
\item by Tejas Santanam, Helmut Simonis
\item \url{https://doi.org/10.48550/arXiv.2403.12717}
\item Based on survey of community for WTCP 2023
\item \url{https://hsimonis.github.io/WTCP2023/}
\end{itemize}
\end{itemize}
\end{frame}


}

\mode<all>{
\part{Global Constraints}
}
\begin{frame}
\partpage
\end{frame}
\mode<all>{
\begin{frame}
\frametitle{Objectives}
\begin{itemize}
\item Overview of Core Constraint Programming
\item Three Main Concepts
\begin{itemize}
\item Constraint Propagation
\item Global Constraints
\item Customizing Search
\end{itemize}
%\item Get Some Experience with MiniZinc
\item Topics will be treated in more detail in later parts of the school
\item Based on Examples, not Formal Description
\end{itemize}
\end{frame}

\begin{frame}
\frametitle{Outline}
\begin{itemize}
\item Why Constraint Programming?
\item Constraint Propagation
\item Global Constraints
\item Customizing Search
%\item Choices in Modelling
%\item Decomposition
%\item More Global Constraints
%\item Methodology
%\item What is missing?
\end{itemize}
\end{frame}

% \begin{frame}
  % \frametitle{Using MiniZinc IDE}
  % \begin{itemize}
  % \item Developed in the Australian NICTA project
  % \item Maintained by Monash University
  % \item Modelling tool with multiple back-end solvers
  % \item Available from \url{https://www.minizinc.org/}
  % \end{itemize}
% \end{frame}


% \begin{frame}
% \frametitle{Examples in ECLiPSe}
% \begin{itemize}
% \item Open sourced constraint programming language
% \item Development goes back to 1985
% \item ECRC, ICL, IC-Parc, PTL, Cisco
% \item \url{https://eclipseclp.org/}
% \item Specialities
% \begin{itemize}
% \item Develop new solvers for specific domains
% \item Integration with MIP
% \end{itemize}
% \item Not included in bundled MiniZinc IDE
% \item Specialized visualization tools used here
  % \begin{itemize}
  % \item CP-Viz, Simonis et al. 2010
    % \end{itemize}
% \end{itemize}
% \end{frame}

\begin{frame}
\frametitle{Tutorial Based on ECLiPSe ELearning Course}
\begin{itemize}
\item Self-study course in constraint programming
\item Supported by Cisco Systems and Silicon Valley Community Foundation
\item Multi-media format, video lectures, slides, handout etc
\item {\small \url{https://eclipseclp.org/ELearning/index.html}}
\end{itemize}
\end{frame}

\begin{frame}
\frametitle{Also Part of CRT-AI Constraint Week}
\begin{itemize}
\item Annual one week course on CP and Optimization in Ireland
\item Part of national training program for PhD students in AI
\item \url{https://www.crt-ai.ie/}
\end{itemize}
\end{frame}



\begin{frame}
\frametitle{Constraint Programming - in a nutshell}
\begin{itemize}
\item Declarative description of problems with
\begin{itemize}
\item {\em Variables} which range over (finite) sets of values
\item {\em Constraints} over subsets of variables which restrict possible value combinations
\item A {\em solution} is a value assignment which satisfies all constraints
\end{itemize}

\item Constraint propagation/reasoning
\begin{itemize}
\item Removing inconsistent values for variables
\item Detect failure if constraint can not be satisfied
\item Interaction of constraints via shared variables
\item Incomplete
\end{itemize}

\item Search
\begin{itemize}
\item User controlled assignment of values to variables
\item Each step triggers constraint propagation 
\end{itemize}
\item Different domains require/allow different methods
\end{itemize}
\end{frame}

\begin{frame}
  \frametitle{Constraint Programming is Different}
  \begin{itemize}
  \item Declarative Programming
    \begin{itemize}
    \item Concentrate on what you want
      \item Not how to get there
      \item Program != Algorithm
      \item Program = Model
    \end{itemize}
    \item Applied to Combinatorial Problems
      \begin{itemize}
        \item No complete polynomial algorithms known (exist?)
        \item CP less ad-hoc than heuristics
        \item Models can evolve
  \end{itemize}
  \end{itemize}
  \end{frame}
    
\begin{frame}
  \frametitle{A Subtractive Process}
  \begin{textblock}{4}(8,-3)
    \includegraphics[width=4cm]{../introduction/images/stages}
  \end{textblock}
  \vfill
  \begin{quote}
    ``Oh, bosh, as Mr. Ruskin says. Sculpture, per se, is the simplest thing in the world. All you have to do is to take a big chunk of marble and a hammer and chisel, make up your mind what you are about to create and chip off all the marble you don't want.''-Paris Gaulois.
  \end{quote}
  
  {\tiny Source: \url{https://quoteinvestigator.com/2014/06/22/chip-away/}}
\end{frame}

\begin{frame}
\frametitle{Basic Process}
\begin{center}
\begin{tikzpicture}
\node[shape=rectangle,fill=pantone127-4] (problem) at (2,5) {Problem};
\node (human) at (2,4) {Human};
\node[shape=rectangle,fill=pantone127-4] (model) at (2,3) {Model};
\node (solver) at (2,2) {Constraint Solver/Search};
\node[shape=rectangle,fill=pantone157-8] (solution) at (2,1) {Solution};
\draw[-] (problem) -- (human);
\draw[->] (human) -- (model);
\draw[-] (model) -- (solver);
\draw[->] (solver) -- (solution);
\end{tikzpicture}
\end{center}
\end{frame}

\begin{frame}
\frametitle{More Realistic}
\begin{center}
\begin{tikzpicture}[xscale=4.0]
\node[shape=rectangle,fill=pantone127-4] (problem) at (2,5) {Problem};
\node (human) at (2,4) {Human};
\node[shape=rectangle,fill=pantone127-4] (model) at (2,3) {Model};
\node (solver) at (2,2) {Constraint Solver/Search};
\node[shape=rectangle,fill=pantone24-8] (hangs) at (1,1) {Hangs};
\node[shape=rectangle,fill=pantone157-8] (solution) at (2,1) {Solution};
\node[shape=rectangle,fill=pantone24-8] (wrong) at (3,1) {Wrong Solution};
\draw[-] (problem) -- (human);
\draw[->] (human) -- (model);
\draw[-] (model) -- (solver);
\draw[->] (solver) -- (hangs);
\draw[->] (solver) -- (solution);
\draw[->] (solver) -- (wrong);
\draw[->] (hangs) -- (1,4) -- (human);
\draw[->] (wrong) -- (3,4) -- (human);
\end{tikzpicture}
\end{center}
\end{frame}

\begin{frame}
\frametitle{Dual Role of Model}
\begin{itemize}
\item Allows Human to Express Problem
\begin{itemize}
\item Close to Problem Domain
\item Constraints as Abstractions
\end{itemize}
\item Allows Solver to Execute
\begin{itemize}
\item Variables as Communication Mechanism
\item Constraints as Algorithms
\end{itemize}
\end{itemize}
\end{frame}

\begin{frame}
\frametitle{Modelling Frameworks}
\begin{itemize}
\item \href{https://www.minizinc.org/}{MiniZinc} (NICTA, Monash University, Australia)
\item \href{https://github.com/eomahony/Numberjack}{NumberJack} (Insight, Ireland) 
\item \href{https://www-users.york.ac.uk/peter.nightingale/savilerow//}{EssencePrime/SavilleRow} (UK) 
\item \href{https://cpmpy.readthedocs.io/en/latest/}{CPMpy} (KU Leuven)
\item Allow use of multiple back-end solvers
\item Compile model into variants for each solver
\item A priori solver independent model(CP, MIP, SAT)
\end{itemize}
\end{frame} 

\begin{frame}
\frametitle{Framework Process}
\begin{center}
\begin{tikzpicture}[xscale=2]
\node[shape=rectangle,fill=pantone127-4] (problem) at (2.5,5) {Problem};
\node (human) at (2.5,4) {Human};
\node[shape=rectangle,fill=pantone127-4] (model) at (2.5,3) {Model};
\node (compile) at (2.5,2) {Compile/Reformulate};
\node (solver1) at (1,1) {CP};
\node (solver2) at (2,1) {MIP};
\node (solver3) at (3,1) {SAT};
\node (solver4) at (4,1) {Other};
\node[shape=rectangle,fill=pantone157-8] (solution1) at (1,0) {Solution};
\node[shape=rectangle,fill=pantone157-8] (solution2) at (2,0) {Solution};
\node[shape=rectangle,fill=pantone157-8] (solution3) at (3,0) {Solution};
\node[shape=rectangle,fill=pantone157-8] (solution4) at (4,0) {Solution};
\draw[-] (problem) -- (human);
\draw[->] (human) -- (model);
\draw[-] (model) -- (compile);
\draw[-] (compile) -- (solver1);
\draw[-] (compile) -- (solver2);
\draw[-] (compile) -- (solver3);
\draw[-] (compile) -- (solver4);
\draw[->] (solver1) -- (solution1);
\draw[->] (solver2) -- (solution2);
\draw[->] (solver3) -- (solution3);
\draw[->] (solver4) -- (solution4);
\end{tikzpicture}
\end{center}
\end{frame}

% \begin{frame}
  % \frametitle{Do It Now!}
  % \begin{itemize}
  % \item Download and install Minizinc
  % \item \url{https://www.minizinc.org/}
  % \end{itemize}
% \end{frame}


\begin{frame}
\frametitle{Why use Puzzles as Examples?}
\begin{itemize}
\item Easy to understand the problem
\item Solvable by hand without specialized knowledge
\item Possible to compare automated to manual solving process
\end{itemize}

\begin{quote}
The puzzle, though inanimate, is presented as a solvable problem without lasting negative consequences, a very low-risk low-reward situation. By being a puzzle, the object is attempting to convince the user that it must be completed.
\end{quote}
{\tiny Source: \href{https://www.everydayrhetoric.com/2020/03/the-rhetoric-of-puzzles/}{Every Day Rhetoric}}
\end{frame}


\mode<all>{
\part{Basic Constraint Propagation}
}
\begin{frame}
\partpage
\end{frame}
\mode<all>{
\input{../sendmore/body}
}

\mode<all>{
\part{Global Constraints}
}
\begin{frame}
\partpage
\end{frame}
\mode<all>{
\input{../sudoku/body}
}

\clearpage
\mode<all>{
\part{Customizing Search}
}
\begin{frame}
\partpage
\end{frame}
\mode<all>{
\input{../nqueen/body}
}


%% \clearpage
%% \mode<all>{
%%   \part{Problem Decomposition}
%% }
%% \begin{frame}
%%   \partpage
%% \end{frame}
%% \mode<all>{
%%   \input{../party/body}
%%   }

%% \clearpage
%% \mode<all>{
%%   \part{More Global Constraints}
%% }
%% \begin{frame}
%%   \partpage
%% \end{frame}
%% \mode<all>{
%%   \input{../car/body}
%%   }

%% \clearpage
%% \mode<all>{
%%   \part{Methodology}
%% }
%% \begin{frame}
%%   \partpage
%% \end{frame}
%% \mode<all>{
%%   \input{../methodology/body}
%%   }

\clearpage
\mode<all>{
\part{What is missing?}
}
\begin{frame}
\partpage
\end{frame}

\begin{frame}
\frametitle{Many Specialized Topics}
\begin{itemize}
\item How to design efficient core engine
\item Hybrids with LP/MIP tools
\item Hybrids with SAT
\item Symmetry breaking
\item Use of MDD/BDD to encode sets of solutions
\item High level modelling tools
\item Debugging/visualization
\end{itemize}
\end{frame}

\begin{frame}
\frametitle{Reformulation}
\begin{itemize}
\item Just because the user has modelled it this way, it doesn't mean we have to solve it that way
\begin{itemize}
\item Replace some constraint(s) by other, equivalent constraints
\item Because we don't have that constraint in our system
\item For performance
\end{itemize}
\end{itemize}
\end{frame}

\begin{frame}
\frametitle{Learning}
\begin{itemize}
\item While solving the problem we can learn how to strengthen the model/search 
\begin{itemize}
\item Understand which constraints/method contribute to propagation and change schedule
\item Learn no-good constraints by explaining failure
\item Adapt search strategy based on search experience
\end{itemize}
\end{itemize}
\end{frame}

%% \begin{frame}
%% \frametitle{Refined Process}
%% \begin{center}
%% \begin{tikzpicture}
%% \node[shape=rectangle,fill=pantone127-4] (problem) at (3,7) {Problem};
%% \node (human) at (3,6) {Human};
%% \node[shape=rectangle,fill=pantone127-4] (model) at (3,5) {User Model};
%% \node (reformulation) at (3,4) {Reformulation};
%% \node[shape=rectangle,fill=pantone127-4] (implementation) at (3,3) {Implementation Model};
%% \node (solver) at (3,2) {Constraint Solver/Search};
%% \node[shape=rectangle,fill=pantone157-8] (solution) at (3,1) {Solution};
%% \draw[-] (problem) -- (human);
%% \draw[->] (human) -- (model);
%% \draw[-] (model) -- (reformulation);
%% \draw[->] (reformulation) -- (implementation);
%% \draw[-] (implementation) -- (solver);
%% \draw[->] (solver) -- (0,2) -- (0,3) -- (implementation);
%% \draw[->] (solver) -- (solution);
%% \end{tikzpicture}
%% \end{center}
%% \end{frame}

%\begin{frame}
%\frametitle{What is CP actually used for?}
%\url{http://hsimonis.wordpress.com}
%\includegraphics[width=10cm]{../introduction/images/blog}
%\end{frame}

%% \begin{frame}
%% \frametitle{Exercises}
%% \begin{itemize}
%% \item Install Constraint Solver of your Choice
%% \item MiniZinc (MiningZinc) will be used in later sessions
%% \item Run Sendmoremoney, Sudoku, Queens examples
%% \item Can you choose which propagator to use for alldifferent?
%% \item Can you use different search strategies?
%% \end{itemize}
%% \end{frame}

\begin{frame}
\frametitle{More Learning Resources}
\begin{itemize}
\item Survey of Methods, Resources, and Formats for Teaching Constraint Programming
\begin{itemize}
\item by Tejas Santanam, Helmut Simonis
\item \url{https://doi.org/10.48550/arXiv.2403.12717}
\item Based on survey of community for WTCP 2023
\item \url{https://hsimonis.github.io/WTCP2023/}
\end{itemize}
\end{itemize}
\end{frame}


}

\clearpage
\mode<all>{
\part{Customizing Search}
}
\begin{frame}
\partpage
\end{frame}
\mode<all>{
\begin{frame}
\frametitle{Objectives}
\begin{itemize}
\item Overview of Core Constraint Programming
\item Three Main Concepts
\begin{itemize}
\item Constraint Propagation
\item Global Constraints
\item Customizing Search
\end{itemize}
%\item Get Some Experience with MiniZinc
\item Topics will be treated in more detail in later parts of the school
\item Based on Examples, not Formal Description
\end{itemize}
\end{frame}

\begin{frame}
\frametitle{Outline}
\begin{itemize}
\item Why Constraint Programming?
\item Constraint Propagation
\item Global Constraints
\item Customizing Search
%\item Choices in Modelling
%\item Decomposition
%\item More Global Constraints
%\item Methodology
%\item What is missing?
\end{itemize}
\end{frame}

% \begin{frame}
  % \frametitle{Using MiniZinc IDE}
  % \begin{itemize}
  % \item Developed in the Australian NICTA project
  % \item Maintained by Monash University
  % \item Modelling tool with multiple back-end solvers
  % \item Available from \url{https://www.minizinc.org/}
  % \end{itemize}
% \end{frame}


% \begin{frame}
% \frametitle{Examples in ECLiPSe}
% \begin{itemize}
% \item Open sourced constraint programming language
% \item Development goes back to 1985
% \item ECRC, ICL, IC-Parc, PTL, Cisco
% \item \url{https://eclipseclp.org/}
% \item Specialities
% \begin{itemize}
% \item Develop new solvers for specific domains
% \item Integration with MIP
% \end{itemize}
% \item Not included in bundled MiniZinc IDE
% \item Specialized visualization tools used here
  % \begin{itemize}
  % \item CP-Viz, Simonis et al. 2010
    % \end{itemize}
% \end{itemize}
% \end{frame}

\begin{frame}
\frametitle{Tutorial Based on ECLiPSe ELearning Course}
\begin{itemize}
\item Self-study course in constraint programming
\item Supported by Cisco Systems and Silicon Valley Community Foundation
\item Multi-media format, video lectures, slides, handout etc
\item {\small \url{https://eclipseclp.org/ELearning/index.html}}
\end{itemize}
\end{frame}

\begin{frame}
\frametitle{Also Part of CRT-AI Constraint Week}
\begin{itemize}
\item Annual one week course on CP and Optimization in Ireland
\item Part of national training program for PhD students in AI
\item \url{https://www.crt-ai.ie/}
\end{itemize}
\end{frame}



\begin{frame}
\frametitle{Constraint Programming - in a nutshell}
\begin{itemize}
\item Declarative description of problems with
\begin{itemize}
\item {\em Variables} which range over (finite) sets of values
\item {\em Constraints} over subsets of variables which restrict possible value combinations
\item A {\em solution} is a value assignment which satisfies all constraints
\end{itemize}

\item Constraint propagation/reasoning
\begin{itemize}
\item Removing inconsistent values for variables
\item Detect failure if constraint can not be satisfied
\item Interaction of constraints via shared variables
\item Incomplete
\end{itemize}

\item Search
\begin{itemize}
\item User controlled assignment of values to variables
\item Each step triggers constraint propagation 
\end{itemize}
\item Different domains require/allow different methods
\end{itemize}
\end{frame}

\begin{frame}
  \frametitle{Constraint Programming is Different}
  \begin{itemize}
  \item Declarative Programming
    \begin{itemize}
    \item Concentrate on what you want
      \item Not how to get there
      \item Program != Algorithm
      \item Program = Model
    \end{itemize}
    \item Applied to Combinatorial Problems
      \begin{itemize}
        \item No complete polynomial algorithms known (exist?)
        \item CP less ad-hoc than heuristics
        \item Models can evolve
  \end{itemize}
  \end{itemize}
  \end{frame}
    
\begin{frame}
  \frametitle{A Subtractive Process}
  \begin{textblock}{4}(8,-3)
    \includegraphics[width=4cm]{../introduction/images/stages}
  \end{textblock}
  \vfill
  \begin{quote}
    ``Oh, bosh, as Mr. Ruskin says. Sculpture, per se, is the simplest thing in the world. All you have to do is to take a big chunk of marble and a hammer and chisel, make up your mind what you are about to create and chip off all the marble you don't want.''-Paris Gaulois.
  \end{quote}
  
  {\tiny Source: \url{https://quoteinvestigator.com/2014/06/22/chip-away/}}
\end{frame}

\begin{frame}
\frametitle{Basic Process}
\begin{center}
\begin{tikzpicture}
\node[shape=rectangle,fill=pantone127-4] (problem) at (2,5) {Problem};
\node (human) at (2,4) {Human};
\node[shape=rectangle,fill=pantone127-4] (model) at (2,3) {Model};
\node (solver) at (2,2) {Constraint Solver/Search};
\node[shape=rectangle,fill=pantone157-8] (solution) at (2,1) {Solution};
\draw[-] (problem) -- (human);
\draw[->] (human) -- (model);
\draw[-] (model) -- (solver);
\draw[->] (solver) -- (solution);
\end{tikzpicture}
\end{center}
\end{frame}

\begin{frame}
\frametitle{More Realistic}
\begin{center}
\begin{tikzpicture}[xscale=4.0]
\node[shape=rectangle,fill=pantone127-4] (problem) at (2,5) {Problem};
\node (human) at (2,4) {Human};
\node[shape=rectangle,fill=pantone127-4] (model) at (2,3) {Model};
\node (solver) at (2,2) {Constraint Solver/Search};
\node[shape=rectangle,fill=pantone24-8] (hangs) at (1,1) {Hangs};
\node[shape=rectangle,fill=pantone157-8] (solution) at (2,1) {Solution};
\node[shape=rectangle,fill=pantone24-8] (wrong) at (3,1) {Wrong Solution};
\draw[-] (problem) -- (human);
\draw[->] (human) -- (model);
\draw[-] (model) -- (solver);
\draw[->] (solver) -- (hangs);
\draw[->] (solver) -- (solution);
\draw[->] (solver) -- (wrong);
\draw[->] (hangs) -- (1,4) -- (human);
\draw[->] (wrong) -- (3,4) -- (human);
\end{tikzpicture}
\end{center}
\end{frame}

\begin{frame}
\frametitle{Dual Role of Model}
\begin{itemize}
\item Allows Human to Express Problem
\begin{itemize}
\item Close to Problem Domain
\item Constraints as Abstractions
\end{itemize}
\item Allows Solver to Execute
\begin{itemize}
\item Variables as Communication Mechanism
\item Constraints as Algorithms
\end{itemize}
\end{itemize}
\end{frame}

\begin{frame}
\frametitle{Modelling Frameworks}
\begin{itemize}
\item \href{https://www.minizinc.org/}{MiniZinc} (NICTA, Monash University, Australia)
\item \href{https://github.com/eomahony/Numberjack}{NumberJack} (Insight, Ireland) 
\item \href{https://www-users.york.ac.uk/peter.nightingale/savilerow//}{EssencePrime/SavilleRow} (UK) 
\item \href{https://cpmpy.readthedocs.io/en/latest/}{CPMpy} (KU Leuven)
\item Allow use of multiple back-end solvers
\item Compile model into variants for each solver
\item A priori solver independent model(CP, MIP, SAT)
\end{itemize}
\end{frame} 

\begin{frame}
\frametitle{Framework Process}
\begin{center}
\begin{tikzpicture}[xscale=2]
\node[shape=rectangle,fill=pantone127-4] (problem) at (2.5,5) {Problem};
\node (human) at (2.5,4) {Human};
\node[shape=rectangle,fill=pantone127-4] (model) at (2.5,3) {Model};
\node (compile) at (2.5,2) {Compile/Reformulate};
\node (solver1) at (1,1) {CP};
\node (solver2) at (2,1) {MIP};
\node (solver3) at (3,1) {SAT};
\node (solver4) at (4,1) {Other};
\node[shape=rectangle,fill=pantone157-8] (solution1) at (1,0) {Solution};
\node[shape=rectangle,fill=pantone157-8] (solution2) at (2,0) {Solution};
\node[shape=rectangle,fill=pantone157-8] (solution3) at (3,0) {Solution};
\node[shape=rectangle,fill=pantone157-8] (solution4) at (4,0) {Solution};
\draw[-] (problem) -- (human);
\draw[->] (human) -- (model);
\draw[-] (model) -- (compile);
\draw[-] (compile) -- (solver1);
\draw[-] (compile) -- (solver2);
\draw[-] (compile) -- (solver3);
\draw[-] (compile) -- (solver4);
\draw[->] (solver1) -- (solution1);
\draw[->] (solver2) -- (solution2);
\draw[->] (solver3) -- (solution3);
\draw[->] (solver4) -- (solution4);
\end{tikzpicture}
\end{center}
\end{frame}

% \begin{frame}
  % \frametitle{Do It Now!}
  % \begin{itemize}
  % \item Download and install Minizinc
  % \item \url{https://www.minizinc.org/}
  % \end{itemize}
% \end{frame}


\begin{frame}
\frametitle{Why use Puzzles as Examples?}
\begin{itemize}
\item Easy to understand the problem
\item Solvable by hand without specialized knowledge
\item Possible to compare automated to manual solving process
\end{itemize}

\begin{quote}
The puzzle, though inanimate, is presented as a solvable problem without lasting negative consequences, a very low-risk low-reward situation. By being a puzzle, the object is attempting to convince the user that it must be completed.
\end{quote}
{\tiny Source: \href{https://www.everydayrhetoric.com/2020/03/the-rhetoric-of-puzzles/}{Every Day Rhetoric}}
\end{frame}


\mode<all>{
\part{Basic Constraint Propagation}
}
\begin{frame}
\partpage
\end{frame}
\mode<all>{
\input{../sendmore/body}
}

\mode<all>{
\part{Global Constraints}
}
\begin{frame}
\partpage
\end{frame}
\mode<all>{
\input{../sudoku/body}
}

\clearpage
\mode<all>{
\part{Customizing Search}
}
\begin{frame}
\partpage
\end{frame}
\mode<all>{
\input{../nqueen/body}
}


%% \clearpage
%% \mode<all>{
%%   \part{Problem Decomposition}
%% }
%% \begin{frame}
%%   \partpage
%% \end{frame}
%% \mode<all>{
%%   \input{../party/body}
%%   }

%% \clearpage
%% \mode<all>{
%%   \part{More Global Constraints}
%% }
%% \begin{frame}
%%   \partpage
%% \end{frame}
%% \mode<all>{
%%   \input{../car/body}
%%   }

%% \clearpage
%% \mode<all>{
%%   \part{Methodology}
%% }
%% \begin{frame}
%%   \partpage
%% \end{frame}
%% \mode<all>{
%%   \input{../methodology/body}
%%   }

\clearpage
\mode<all>{
\part{What is missing?}
}
\begin{frame}
\partpage
\end{frame}

\begin{frame}
\frametitle{Many Specialized Topics}
\begin{itemize}
\item How to design efficient core engine
\item Hybrids with LP/MIP tools
\item Hybrids with SAT
\item Symmetry breaking
\item Use of MDD/BDD to encode sets of solutions
\item High level modelling tools
\item Debugging/visualization
\end{itemize}
\end{frame}

\begin{frame}
\frametitle{Reformulation}
\begin{itemize}
\item Just because the user has modelled it this way, it doesn't mean we have to solve it that way
\begin{itemize}
\item Replace some constraint(s) by other, equivalent constraints
\item Because we don't have that constraint in our system
\item For performance
\end{itemize}
\end{itemize}
\end{frame}

\begin{frame}
\frametitle{Learning}
\begin{itemize}
\item While solving the problem we can learn how to strengthen the model/search 
\begin{itemize}
\item Understand which constraints/method contribute to propagation and change schedule
\item Learn no-good constraints by explaining failure
\item Adapt search strategy based on search experience
\end{itemize}
\end{itemize}
\end{frame}

%% \begin{frame}
%% \frametitle{Refined Process}
%% \begin{center}
%% \begin{tikzpicture}
%% \node[shape=rectangle,fill=pantone127-4] (problem) at (3,7) {Problem};
%% \node (human) at (3,6) {Human};
%% \node[shape=rectangle,fill=pantone127-4] (model) at (3,5) {User Model};
%% \node (reformulation) at (3,4) {Reformulation};
%% \node[shape=rectangle,fill=pantone127-4] (implementation) at (3,3) {Implementation Model};
%% \node (solver) at (3,2) {Constraint Solver/Search};
%% \node[shape=rectangle,fill=pantone157-8] (solution) at (3,1) {Solution};
%% \draw[-] (problem) -- (human);
%% \draw[->] (human) -- (model);
%% \draw[-] (model) -- (reformulation);
%% \draw[->] (reformulation) -- (implementation);
%% \draw[-] (implementation) -- (solver);
%% \draw[->] (solver) -- (0,2) -- (0,3) -- (implementation);
%% \draw[->] (solver) -- (solution);
%% \end{tikzpicture}
%% \end{center}
%% \end{frame}

%\begin{frame}
%\frametitle{What is CP actually used for?}
%\url{http://hsimonis.wordpress.com}
%\includegraphics[width=10cm]{../introduction/images/blog}
%\end{frame}

%% \begin{frame}
%% \frametitle{Exercises}
%% \begin{itemize}
%% \item Install Constraint Solver of your Choice
%% \item MiniZinc (MiningZinc) will be used in later sessions
%% \item Run Sendmoremoney, Sudoku, Queens examples
%% \item Can you choose which propagator to use for alldifferent?
%% \item Can you use different search strategies?
%% \end{itemize}
%% \end{frame}

\begin{frame}
\frametitle{More Learning Resources}
\begin{itemize}
\item Survey of Methods, Resources, and Formats for Teaching Constraint Programming
\begin{itemize}
\item by Tejas Santanam, Helmut Simonis
\item \url{https://doi.org/10.48550/arXiv.2403.12717}
\item Based on survey of community for WTCP 2023
\item \url{https://hsimonis.github.io/WTCP2023/}
\end{itemize}
\end{itemize}
\end{frame}


}


%% \clearpage
%% \mode<all>{
%%   \part{Problem Decomposition}
%% }
%% \begin{frame}
%%   \partpage
%% \end{frame}
%% \mode<all>{
%%   \begin{frame}
\frametitle{Objectives}
\begin{itemize}
\item Overview of Core Constraint Programming
\item Three Main Concepts
\begin{itemize}
\item Constraint Propagation
\item Global Constraints
\item Customizing Search
\end{itemize}
%\item Get Some Experience with MiniZinc
\item Topics will be treated in more detail in later parts of the school
\item Based on Examples, not Formal Description
\end{itemize}
\end{frame}

\begin{frame}
\frametitle{Outline}
\begin{itemize}
\item Why Constraint Programming?
\item Constraint Propagation
\item Global Constraints
\item Customizing Search
%\item Choices in Modelling
%\item Decomposition
%\item More Global Constraints
%\item Methodology
%\item What is missing?
\end{itemize}
\end{frame}

% \begin{frame}
  % \frametitle{Using MiniZinc IDE}
  % \begin{itemize}
  % \item Developed in the Australian NICTA project
  % \item Maintained by Monash University
  % \item Modelling tool with multiple back-end solvers
  % \item Available from \url{https://www.minizinc.org/}
  % \end{itemize}
% \end{frame}


% \begin{frame}
% \frametitle{Examples in ECLiPSe}
% \begin{itemize}
% \item Open sourced constraint programming language
% \item Development goes back to 1985
% \item ECRC, ICL, IC-Parc, PTL, Cisco
% \item \url{https://eclipseclp.org/}
% \item Specialities
% \begin{itemize}
% \item Develop new solvers for specific domains
% \item Integration with MIP
% \end{itemize}
% \item Not included in bundled MiniZinc IDE
% \item Specialized visualization tools used here
  % \begin{itemize}
  % \item CP-Viz, Simonis et al. 2010
    % \end{itemize}
% \end{itemize}
% \end{frame}

\begin{frame}
\frametitle{Tutorial Based on ECLiPSe ELearning Course}
\begin{itemize}
\item Self-study course in constraint programming
\item Supported by Cisco Systems and Silicon Valley Community Foundation
\item Multi-media format, video lectures, slides, handout etc
\item {\small \url{https://eclipseclp.org/ELearning/index.html}}
\end{itemize}
\end{frame}

\begin{frame}
\frametitle{Also Part of CRT-AI Constraint Week}
\begin{itemize}
\item Annual one week course on CP and Optimization in Ireland
\item Part of national training program for PhD students in AI
\item \url{https://www.crt-ai.ie/}
\end{itemize}
\end{frame}



\begin{frame}
\frametitle{Constraint Programming - in a nutshell}
\begin{itemize}
\item Declarative description of problems with
\begin{itemize}
\item {\em Variables} which range over (finite) sets of values
\item {\em Constraints} over subsets of variables which restrict possible value combinations
\item A {\em solution} is a value assignment which satisfies all constraints
\end{itemize}

\item Constraint propagation/reasoning
\begin{itemize}
\item Removing inconsistent values for variables
\item Detect failure if constraint can not be satisfied
\item Interaction of constraints via shared variables
\item Incomplete
\end{itemize}

\item Search
\begin{itemize}
\item User controlled assignment of values to variables
\item Each step triggers constraint propagation 
\end{itemize}
\item Different domains require/allow different methods
\end{itemize}
\end{frame}

\begin{frame}
  \frametitle{Constraint Programming is Different}
  \begin{itemize}
  \item Declarative Programming
    \begin{itemize}
    \item Concentrate on what you want
      \item Not how to get there
      \item Program != Algorithm
      \item Program = Model
    \end{itemize}
    \item Applied to Combinatorial Problems
      \begin{itemize}
        \item No complete polynomial algorithms known (exist?)
        \item CP less ad-hoc than heuristics
        \item Models can evolve
  \end{itemize}
  \end{itemize}
  \end{frame}
    
\begin{frame}
  \frametitle{A Subtractive Process}
  \begin{textblock}{4}(8,-3)
    \includegraphics[width=4cm]{../introduction/images/stages}
  \end{textblock}
  \vfill
  \begin{quote}
    ``Oh, bosh, as Mr. Ruskin says. Sculpture, per se, is the simplest thing in the world. All you have to do is to take a big chunk of marble and a hammer and chisel, make up your mind what you are about to create and chip off all the marble you don't want.''-Paris Gaulois.
  \end{quote}
  
  {\tiny Source: \url{https://quoteinvestigator.com/2014/06/22/chip-away/}}
\end{frame}

\begin{frame}
\frametitle{Basic Process}
\begin{center}
\begin{tikzpicture}
\node[shape=rectangle,fill=pantone127-4] (problem) at (2,5) {Problem};
\node (human) at (2,4) {Human};
\node[shape=rectangle,fill=pantone127-4] (model) at (2,3) {Model};
\node (solver) at (2,2) {Constraint Solver/Search};
\node[shape=rectangle,fill=pantone157-8] (solution) at (2,1) {Solution};
\draw[-] (problem) -- (human);
\draw[->] (human) -- (model);
\draw[-] (model) -- (solver);
\draw[->] (solver) -- (solution);
\end{tikzpicture}
\end{center}
\end{frame}

\begin{frame}
\frametitle{More Realistic}
\begin{center}
\begin{tikzpicture}[xscale=4.0]
\node[shape=rectangle,fill=pantone127-4] (problem) at (2,5) {Problem};
\node (human) at (2,4) {Human};
\node[shape=rectangle,fill=pantone127-4] (model) at (2,3) {Model};
\node (solver) at (2,2) {Constraint Solver/Search};
\node[shape=rectangle,fill=pantone24-8] (hangs) at (1,1) {Hangs};
\node[shape=rectangle,fill=pantone157-8] (solution) at (2,1) {Solution};
\node[shape=rectangle,fill=pantone24-8] (wrong) at (3,1) {Wrong Solution};
\draw[-] (problem) -- (human);
\draw[->] (human) -- (model);
\draw[-] (model) -- (solver);
\draw[->] (solver) -- (hangs);
\draw[->] (solver) -- (solution);
\draw[->] (solver) -- (wrong);
\draw[->] (hangs) -- (1,4) -- (human);
\draw[->] (wrong) -- (3,4) -- (human);
\end{tikzpicture}
\end{center}
\end{frame}

\begin{frame}
\frametitle{Dual Role of Model}
\begin{itemize}
\item Allows Human to Express Problem
\begin{itemize}
\item Close to Problem Domain
\item Constraints as Abstractions
\end{itemize}
\item Allows Solver to Execute
\begin{itemize}
\item Variables as Communication Mechanism
\item Constraints as Algorithms
\end{itemize}
\end{itemize}
\end{frame}

\begin{frame}
\frametitle{Modelling Frameworks}
\begin{itemize}
\item \href{https://www.minizinc.org/}{MiniZinc} (NICTA, Monash University, Australia)
\item \href{https://github.com/eomahony/Numberjack}{NumberJack} (Insight, Ireland) 
\item \href{https://www-users.york.ac.uk/peter.nightingale/savilerow//}{EssencePrime/SavilleRow} (UK) 
\item \href{https://cpmpy.readthedocs.io/en/latest/}{CPMpy} (KU Leuven)
\item Allow use of multiple back-end solvers
\item Compile model into variants for each solver
\item A priori solver independent model(CP, MIP, SAT)
\end{itemize}
\end{frame} 

\begin{frame}
\frametitle{Framework Process}
\begin{center}
\begin{tikzpicture}[xscale=2]
\node[shape=rectangle,fill=pantone127-4] (problem) at (2.5,5) {Problem};
\node (human) at (2.5,4) {Human};
\node[shape=rectangle,fill=pantone127-4] (model) at (2.5,3) {Model};
\node (compile) at (2.5,2) {Compile/Reformulate};
\node (solver1) at (1,1) {CP};
\node (solver2) at (2,1) {MIP};
\node (solver3) at (3,1) {SAT};
\node (solver4) at (4,1) {Other};
\node[shape=rectangle,fill=pantone157-8] (solution1) at (1,0) {Solution};
\node[shape=rectangle,fill=pantone157-8] (solution2) at (2,0) {Solution};
\node[shape=rectangle,fill=pantone157-8] (solution3) at (3,0) {Solution};
\node[shape=rectangle,fill=pantone157-8] (solution4) at (4,0) {Solution};
\draw[-] (problem) -- (human);
\draw[->] (human) -- (model);
\draw[-] (model) -- (compile);
\draw[-] (compile) -- (solver1);
\draw[-] (compile) -- (solver2);
\draw[-] (compile) -- (solver3);
\draw[-] (compile) -- (solver4);
\draw[->] (solver1) -- (solution1);
\draw[->] (solver2) -- (solution2);
\draw[->] (solver3) -- (solution3);
\draw[->] (solver4) -- (solution4);
\end{tikzpicture}
\end{center}
\end{frame}

% \begin{frame}
  % \frametitle{Do It Now!}
  % \begin{itemize}
  % \item Download and install Minizinc
  % \item \url{https://www.minizinc.org/}
  % \end{itemize}
% \end{frame}


\begin{frame}
\frametitle{Why use Puzzles as Examples?}
\begin{itemize}
\item Easy to understand the problem
\item Solvable by hand without specialized knowledge
\item Possible to compare automated to manual solving process
\end{itemize}

\begin{quote}
The puzzle, though inanimate, is presented as a solvable problem without lasting negative consequences, a very low-risk low-reward situation. By being a puzzle, the object is attempting to convince the user that it must be completed.
\end{quote}
{\tiny Source: \href{https://www.everydayrhetoric.com/2020/03/the-rhetoric-of-puzzles/}{Every Day Rhetoric}}
\end{frame}


\mode<all>{
\part{Basic Constraint Propagation}
}
\begin{frame}
\partpage
\end{frame}
\mode<all>{
\input{../sendmore/body}
}

\mode<all>{
\part{Global Constraints}
}
\begin{frame}
\partpage
\end{frame}
\mode<all>{
\input{../sudoku/body}
}

\clearpage
\mode<all>{
\part{Customizing Search}
}
\begin{frame}
\partpage
\end{frame}
\mode<all>{
\input{../nqueen/body}
}


%% \clearpage
%% \mode<all>{
%%   \part{Problem Decomposition}
%% }
%% \begin{frame}
%%   \partpage
%% \end{frame}
%% \mode<all>{
%%   \input{../party/body}
%%   }

%% \clearpage
%% \mode<all>{
%%   \part{More Global Constraints}
%% }
%% \begin{frame}
%%   \partpage
%% \end{frame}
%% \mode<all>{
%%   \input{../car/body}
%%   }

%% \clearpage
%% \mode<all>{
%%   \part{Methodology}
%% }
%% \begin{frame}
%%   \partpage
%% \end{frame}
%% \mode<all>{
%%   \input{../methodology/body}
%%   }

\clearpage
\mode<all>{
\part{What is missing?}
}
\begin{frame}
\partpage
\end{frame}

\begin{frame}
\frametitle{Many Specialized Topics}
\begin{itemize}
\item How to design efficient core engine
\item Hybrids with LP/MIP tools
\item Hybrids with SAT
\item Symmetry breaking
\item Use of MDD/BDD to encode sets of solutions
\item High level modelling tools
\item Debugging/visualization
\end{itemize}
\end{frame}

\begin{frame}
\frametitle{Reformulation}
\begin{itemize}
\item Just because the user has modelled it this way, it doesn't mean we have to solve it that way
\begin{itemize}
\item Replace some constraint(s) by other, equivalent constraints
\item Because we don't have that constraint in our system
\item For performance
\end{itemize}
\end{itemize}
\end{frame}

\begin{frame}
\frametitle{Learning}
\begin{itemize}
\item While solving the problem we can learn how to strengthen the model/search 
\begin{itemize}
\item Understand which constraints/method contribute to propagation and change schedule
\item Learn no-good constraints by explaining failure
\item Adapt search strategy based on search experience
\end{itemize}
\end{itemize}
\end{frame}

%% \begin{frame}
%% \frametitle{Refined Process}
%% \begin{center}
%% \begin{tikzpicture}
%% \node[shape=rectangle,fill=pantone127-4] (problem) at (3,7) {Problem};
%% \node (human) at (3,6) {Human};
%% \node[shape=rectangle,fill=pantone127-4] (model) at (3,5) {User Model};
%% \node (reformulation) at (3,4) {Reformulation};
%% \node[shape=rectangle,fill=pantone127-4] (implementation) at (3,3) {Implementation Model};
%% \node (solver) at (3,2) {Constraint Solver/Search};
%% \node[shape=rectangle,fill=pantone157-8] (solution) at (3,1) {Solution};
%% \draw[-] (problem) -- (human);
%% \draw[->] (human) -- (model);
%% \draw[-] (model) -- (reformulation);
%% \draw[->] (reformulation) -- (implementation);
%% \draw[-] (implementation) -- (solver);
%% \draw[->] (solver) -- (0,2) -- (0,3) -- (implementation);
%% \draw[->] (solver) -- (solution);
%% \end{tikzpicture}
%% \end{center}
%% \end{frame}

%\begin{frame}
%\frametitle{What is CP actually used for?}
%\url{http://hsimonis.wordpress.com}
%\includegraphics[width=10cm]{../introduction/images/blog}
%\end{frame}

%% \begin{frame}
%% \frametitle{Exercises}
%% \begin{itemize}
%% \item Install Constraint Solver of your Choice
%% \item MiniZinc (MiningZinc) will be used in later sessions
%% \item Run Sendmoremoney, Sudoku, Queens examples
%% \item Can you choose which propagator to use for alldifferent?
%% \item Can you use different search strategies?
%% \end{itemize}
%% \end{frame}

\begin{frame}
\frametitle{More Learning Resources}
\begin{itemize}
\item Survey of Methods, Resources, and Formats for Teaching Constraint Programming
\begin{itemize}
\item by Tejas Santanam, Helmut Simonis
\item \url{https://doi.org/10.48550/arXiv.2403.12717}
\item Based on survey of community for WTCP 2023
\item \url{https://hsimonis.github.io/WTCP2023/}
\end{itemize}
\end{itemize}
\end{frame}


%%   }

%% \clearpage
%% \mode<all>{
%%   \part{More Global Constraints}
%% }
%% \begin{frame}
%%   \partpage
%% \end{frame}
%% \mode<all>{
%%   \begin{frame}
\frametitle{Objectives}
\begin{itemize}
\item Overview of Core Constraint Programming
\item Three Main Concepts
\begin{itemize}
\item Constraint Propagation
\item Global Constraints
\item Customizing Search
\end{itemize}
%\item Get Some Experience with MiniZinc
\item Topics will be treated in more detail in later parts of the school
\item Based on Examples, not Formal Description
\end{itemize}
\end{frame}

\begin{frame}
\frametitle{Outline}
\begin{itemize}
\item Why Constraint Programming?
\item Constraint Propagation
\item Global Constraints
\item Customizing Search
%\item Choices in Modelling
%\item Decomposition
%\item More Global Constraints
%\item Methodology
%\item What is missing?
\end{itemize}
\end{frame}

% \begin{frame}
  % \frametitle{Using MiniZinc IDE}
  % \begin{itemize}
  % \item Developed in the Australian NICTA project
  % \item Maintained by Monash University
  % \item Modelling tool with multiple back-end solvers
  % \item Available from \url{https://www.minizinc.org/}
  % \end{itemize}
% \end{frame}


% \begin{frame}
% \frametitle{Examples in ECLiPSe}
% \begin{itemize}
% \item Open sourced constraint programming language
% \item Development goes back to 1985
% \item ECRC, ICL, IC-Parc, PTL, Cisco
% \item \url{https://eclipseclp.org/}
% \item Specialities
% \begin{itemize}
% \item Develop new solvers for specific domains
% \item Integration with MIP
% \end{itemize}
% \item Not included in bundled MiniZinc IDE
% \item Specialized visualization tools used here
  % \begin{itemize}
  % \item CP-Viz, Simonis et al. 2010
    % \end{itemize}
% \end{itemize}
% \end{frame}

\begin{frame}
\frametitle{Tutorial Based on ECLiPSe ELearning Course}
\begin{itemize}
\item Self-study course in constraint programming
\item Supported by Cisco Systems and Silicon Valley Community Foundation
\item Multi-media format, video lectures, slides, handout etc
\item {\small \url{https://eclipseclp.org/ELearning/index.html}}
\end{itemize}
\end{frame}

\begin{frame}
\frametitle{Also Part of CRT-AI Constraint Week}
\begin{itemize}
\item Annual one week course on CP and Optimization in Ireland
\item Part of national training program for PhD students in AI
\item \url{https://www.crt-ai.ie/}
\end{itemize}
\end{frame}



\begin{frame}
\frametitle{Constraint Programming - in a nutshell}
\begin{itemize}
\item Declarative description of problems with
\begin{itemize}
\item {\em Variables} which range over (finite) sets of values
\item {\em Constraints} over subsets of variables which restrict possible value combinations
\item A {\em solution} is a value assignment which satisfies all constraints
\end{itemize}

\item Constraint propagation/reasoning
\begin{itemize}
\item Removing inconsistent values for variables
\item Detect failure if constraint can not be satisfied
\item Interaction of constraints via shared variables
\item Incomplete
\end{itemize}

\item Search
\begin{itemize}
\item User controlled assignment of values to variables
\item Each step triggers constraint propagation 
\end{itemize}
\item Different domains require/allow different methods
\end{itemize}
\end{frame}

\begin{frame}
  \frametitle{Constraint Programming is Different}
  \begin{itemize}
  \item Declarative Programming
    \begin{itemize}
    \item Concentrate on what you want
      \item Not how to get there
      \item Program != Algorithm
      \item Program = Model
    \end{itemize}
    \item Applied to Combinatorial Problems
      \begin{itemize}
        \item No complete polynomial algorithms known (exist?)
        \item CP less ad-hoc than heuristics
        \item Models can evolve
  \end{itemize}
  \end{itemize}
  \end{frame}
    
\begin{frame}
  \frametitle{A Subtractive Process}
  \begin{textblock}{4}(8,-3)
    \includegraphics[width=4cm]{../introduction/images/stages}
  \end{textblock}
  \vfill
  \begin{quote}
    ``Oh, bosh, as Mr. Ruskin says. Sculpture, per se, is the simplest thing in the world. All you have to do is to take a big chunk of marble and a hammer and chisel, make up your mind what you are about to create and chip off all the marble you don't want.''-Paris Gaulois.
  \end{quote}
  
  {\tiny Source: \url{https://quoteinvestigator.com/2014/06/22/chip-away/}}
\end{frame}

\begin{frame}
\frametitle{Basic Process}
\begin{center}
\begin{tikzpicture}
\node[shape=rectangle,fill=pantone127-4] (problem) at (2,5) {Problem};
\node (human) at (2,4) {Human};
\node[shape=rectangle,fill=pantone127-4] (model) at (2,3) {Model};
\node (solver) at (2,2) {Constraint Solver/Search};
\node[shape=rectangle,fill=pantone157-8] (solution) at (2,1) {Solution};
\draw[-] (problem) -- (human);
\draw[->] (human) -- (model);
\draw[-] (model) -- (solver);
\draw[->] (solver) -- (solution);
\end{tikzpicture}
\end{center}
\end{frame}

\begin{frame}
\frametitle{More Realistic}
\begin{center}
\begin{tikzpicture}[xscale=4.0]
\node[shape=rectangle,fill=pantone127-4] (problem) at (2,5) {Problem};
\node (human) at (2,4) {Human};
\node[shape=rectangle,fill=pantone127-4] (model) at (2,3) {Model};
\node (solver) at (2,2) {Constraint Solver/Search};
\node[shape=rectangle,fill=pantone24-8] (hangs) at (1,1) {Hangs};
\node[shape=rectangle,fill=pantone157-8] (solution) at (2,1) {Solution};
\node[shape=rectangle,fill=pantone24-8] (wrong) at (3,1) {Wrong Solution};
\draw[-] (problem) -- (human);
\draw[->] (human) -- (model);
\draw[-] (model) -- (solver);
\draw[->] (solver) -- (hangs);
\draw[->] (solver) -- (solution);
\draw[->] (solver) -- (wrong);
\draw[->] (hangs) -- (1,4) -- (human);
\draw[->] (wrong) -- (3,4) -- (human);
\end{tikzpicture}
\end{center}
\end{frame}

\begin{frame}
\frametitle{Dual Role of Model}
\begin{itemize}
\item Allows Human to Express Problem
\begin{itemize}
\item Close to Problem Domain
\item Constraints as Abstractions
\end{itemize}
\item Allows Solver to Execute
\begin{itemize}
\item Variables as Communication Mechanism
\item Constraints as Algorithms
\end{itemize}
\end{itemize}
\end{frame}

\begin{frame}
\frametitle{Modelling Frameworks}
\begin{itemize}
\item \href{https://www.minizinc.org/}{MiniZinc} (NICTA, Monash University, Australia)
\item \href{https://github.com/eomahony/Numberjack}{NumberJack} (Insight, Ireland) 
\item \href{https://www-users.york.ac.uk/peter.nightingale/savilerow//}{EssencePrime/SavilleRow} (UK) 
\item \href{https://cpmpy.readthedocs.io/en/latest/}{CPMpy} (KU Leuven)
\item Allow use of multiple back-end solvers
\item Compile model into variants for each solver
\item A priori solver independent model(CP, MIP, SAT)
\end{itemize}
\end{frame} 

\begin{frame}
\frametitle{Framework Process}
\begin{center}
\begin{tikzpicture}[xscale=2]
\node[shape=rectangle,fill=pantone127-4] (problem) at (2.5,5) {Problem};
\node (human) at (2.5,4) {Human};
\node[shape=rectangle,fill=pantone127-4] (model) at (2.5,3) {Model};
\node (compile) at (2.5,2) {Compile/Reformulate};
\node (solver1) at (1,1) {CP};
\node (solver2) at (2,1) {MIP};
\node (solver3) at (3,1) {SAT};
\node (solver4) at (4,1) {Other};
\node[shape=rectangle,fill=pantone157-8] (solution1) at (1,0) {Solution};
\node[shape=rectangle,fill=pantone157-8] (solution2) at (2,0) {Solution};
\node[shape=rectangle,fill=pantone157-8] (solution3) at (3,0) {Solution};
\node[shape=rectangle,fill=pantone157-8] (solution4) at (4,0) {Solution};
\draw[-] (problem) -- (human);
\draw[->] (human) -- (model);
\draw[-] (model) -- (compile);
\draw[-] (compile) -- (solver1);
\draw[-] (compile) -- (solver2);
\draw[-] (compile) -- (solver3);
\draw[-] (compile) -- (solver4);
\draw[->] (solver1) -- (solution1);
\draw[->] (solver2) -- (solution2);
\draw[->] (solver3) -- (solution3);
\draw[->] (solver4) -- (solution4);
\end{tikzpicture}
\end{center}
\end{frame}

% \begin{frame}
  % \frametitle{Do It Now!}
  % \begin{itemize}
  % \item Download and install Minizinc
  % \item \url{https://www.minizinc.org/}
  % \end{itemize}
% \end{frame}


\begin{frame}
\frametitle{Why use Puzzles as Examples?}
\begin{itemize}
\item Easy to understand the problem
\item Solvable by hand without specialized knowledge
\item Possible to compare automated to manual solving process
\end{itemize}

\begin{quote}
The puzzle, though inanimate, is presented as a solvable problem without lasting negative consequences, a very low-risk low-reward situation. By being a puzzle, the object is attempting to convince the user that it must be completed.
\end{quote}
{\tiny Source: \href{https://www.everydayrhetoric.com/2020/03/the-rhetoric-of-puzzles/}{Every Day Rhetoric}}
\end{frame}


\mode<all>{
\part{Basic Constraint Propagation}
}
\begin{frame}
\partpage
\end{frame}
\mode<all>{
\input{../sendmore/body}
}

\mode<all>{
\part{Global Constraints}
}
\begin{frame}
\partpage
\end{frame}
\mode<all>{
\input{../sudoku/body}
}

\clearpage
\mode<all>{
\part{Customizing Search}
}
\begin{frame}
\partpage
\end{frame}
\mode<all>{
\input{../nqueen/body}
}


%% \clearpage
%% \mode<all>{
%%   \part{Problem Decomposition}
%% }
%% \begin{frame}
%%   \partpage
%% \end{frame}
%% \mode<all>{
%%   \input{../party/body}
%%   }

%% \clearpage
%% \mode<all>{
%%   \part{More Global Constraints}
%% }
%% \begin{frame}
%%   \partpage
%% \end{frame}
%% \mode<all>{
%%   \input{../car/body}
%%   }

%% \clearpage
%% \mode<all>{
%%   \part{Methodology}
%% }
%% \begin{frame}
%%   \partpage
%% \end{frame}
%% \mode<all>{
%%   \input{../methodology/body}
%%   }

\clearpage
\mode<all>{
\part{What is missing?}
}
\begin{frame}
\partpage
\end{frame}

\begin{frame}
\frametitle{Many Specialized Topics}
\begin{itemize}
\item How to design efficient core engine
\item Hybrids with LP/MIP tools
\item Hybrids with SAT
\item Symmetry breaking
\item Use of MDD/BDD to encode sets of solutions
\item High level modelling tools
\item Debugging/visualization
\end{itemize}
\end{frame}

\begin{frame}
\frametitle{Reformulation}
\begin{itemize}
\item Just because the user has modelled it this way, it doesn't mean we have to solve it that way
\begin{itemize}
\item Replace some constraint(s) by other, equivalent constraints
\item Because we don't have that constraint in our system
\item For performance
\end{itemize}
\end{itemize}
\end{frame}

\begin{frame}
\frametitle{Learning}
\begin{itemize}
\item While solving the problem we can learn how to strengthen the model/search 
\begin{itemize}
\item Understand which constraints/method contribute to propagation and change schedule
\item Learn no-good constraints by explaining failure
\item Adapt search strategy based on search experience
\end{itemize}
\end{itemize}
\end{frame}

%% \begin{frame}
%% \frametitle{Refined Process}
%% \begin{center}
%% \begin{tikzpicture}
%% \node[shape=rectangle,fill=pantone127-4] (problem) at (3,7) {Problem};
%% \node (human) at (3,6) {Human};
%% \node[shape=rectangle,fill=pantone127-4] (model) at (3,5) {User Model};
%% \node (reformulation) at (3,4) {Reformulation};
%% \node[shape=rectangle,fill=pantone127-4] (implementation) at (3,3) {Implementation Model};
%% \node (solver) at (3,2) {Constraint Solver/Search};
%% \node[shape=rectangle,fill=pantone157-8] (solution) at (3,1) {Solution};
%% \draw[-] (problem) -- (human);
%% \draw[->] (human) -- (model);
%% \draw[-] (model) -- (reformulation);
%% \draw[->] (reformulation) -- (implementation);
%% \draw[-] (implementation) -- (solver);
%% \draw[->] (solver) -- (0,2) -- (0,3) -- (implementation);
%% \draw[->] (solver) -- (solution);
%% \end{tikzpicture}
%% \end{center}
%% \end{frame}

%\begin{frame}
%\frametitle{What is CP actually used for?}
%\url{http://hsimonis.wordpress.com}
%\includegraphics[width=10cm]{../introduction/images/blog}
%\end{frame}

%% \begin{frame}
%% \frametitle{Exercises}
%% \begin{itemize}
%% \item Install Constraint Solver of your Choice
%% \item MiniZinc (MiningZinc) will be used in later sessions
%% \item Run Sendmoremoney, Sudoku, Queens examples
%% \item Can you choose which propagator to use for alldifferent?
%% \item Can you use different search strategies?
%% \end{itemize}
%% \end{frame}

\begin{frame}
\frametitle{More Learning Resources}
\begin{itemize}
\item Survey of Methods, Resources, and Formats for Teaching Constraint Programming
\begin{itemize}
\item by Tejas Santanam, Helmut Simonis
\item \url{https://doi.org/10.48550/arXiv.2403.12717}
\item Based on survey of community for WTCP 2023
\item \url{https://hsimonis.github.io/WTCP2023/}
\end{itemize}
\end{itemize}
\end{frame}


%%   }

%% \clearpage
%% \mode<all>{
%%   \part{Methodology}
%% }
%% \begin{frame}
%%   \partpage
%% \end{frame}
%% \mode<all>{
%%   \begin{frame}
\frametitle{Objectives}
\begin{itemize}
\item Overview of Core Constraint Programming
\item Three Main Concepts
\begin{itemize}
\item Constraint Propagation
\item Global Constraints
\item Customizing Search
\end{itemize}
%\item Get Some Experience with MiniZinc
\item Topics will be treated in more detail in later parts of the school
\item Based on Examples, not Formal Description
\end{itemize}
\end{frame}

\begin{frame}
\frametitle{Outline}
\begin{itemize}
\item Why Constraint Programming?
\item Constraint Propagation
\item Global Constraints
\item Customizing Search
%\item Choices in Modelling
%\item Decomposition
%\item More Global Constraints
%\item Methodology
%\item What is missing?
\end{itemize}
\end{frame}

% \begin{frame}
  % \frametitle{Using MiniZinc IDE}
  % \begin{itemize}
  % \item Developed in the Australian NICTA project
  % \item Maintained by Monash University
  % \item Modelling tool with multiple back-end solvers
  % \item Available from \url{https://www.minizinc.org/}
  % \end{itemize}
% \end{frame}


% \begin{frame}
% \frametitle{Examples in ECLiPSe}
% \begin{itemize}
% \item Open sourced constraint programming language
% \item Development goes back to 1985
% \item ECRC, ICL, IC-Parc, PTL, Cisco
% \item \url{https://eclipseclp.org/}
% \item Specialities
% \begin{itemize}
% \item Develop new solvers for specific domains
% \item Integration with MIP
% \end{itemize}
% \item Not included in bundled MiniZinc IDE
% \item Specialized visualization tools used here
  % \begin{itemize}
  % \item CP-Viz, Simonis et al. 2010
    % \end{itemize}
% \end{itemize}
% \end{frame}

\begin{frame}
\frametitle{Tutorial Based on ECLiPSe ELearning Course}
\begin{itemize}
\item Self-study course in constraint programming
\item Supported by Cisco Systems and Silicon Valley Community Foundation
\item Multi-media format, video lectures, slides, handout etc
\item {\small \url{https://eclipseclp.org/ELearning/index.html}}
\end{itemize}
\end{frame}

\begin{frame}
\frametitle{Also Part of CRT-AI Constraint Week}
\begin{itemize}
\item Annual one week course on CP and Optimization in Ireland
\item Part of national training program for PhD students in AI
\item \url{https://www.crt-ai.ie/}
\end{itemize}
\end{frame}



\begin{frame}
\frametitle{Constraint Programming - in a nutshell}
\begin{itemize}
\item Declarative description of problems with
\begin{itemize}
\item {\em Variables} which range over (finite) sets of values
\item {\em Constraints} over subsets of variables which restrict possible value combinations
\item A {\em solution} is a value assignment which satisfies all constraints
\end{itemize}

\item Constraint propagation/reasoning
\begin{itemize}
\item Removing inconsistent values for variables
\item Detect failure if constraint can not be satisfied
\item Interaction of constraints via shared variables
\item Incomplete
\end{itemize}

\item Search
\begin{itemize}
\item User controlled assignment of values to variables
\item Each step triggers constraint propagation 
\end{itemize}
\item Different domains require/allow different methods
\end{itemize}
\end{frame}

\begin{frame}
  \frametitle{Constraint Programming is Different}
  \begin{itemize}
  \item Declarative Programming
    \begin{itemize}
    \item Concentrate on what you want
      \item Not how to get there
      \item Program != Algorithm
      \item Program = Model
    \end{itemize}
    \item Applied to Combinatorial Problems
      \begin{itemize}
        \item No complete polynomial algorithms known (exist?)
        \item CP less ad-hoc than heuristics
        \item Models can evolve
  \end{itemize}
  \end{itemize}
  \end{frame}
    
\begin{frame}
  \frametitle{A Subtractive Process}
  \begin{textblock}{4}(8,-3)
    \includegraphics[width=4cm]{../introduction/images/stages}
  \end{textblock}
  \vfill
  \begin{quote}
    ``Oh, bosh, as Mr. Ruskin says. Sculpture, per se, is the simplest thing in the world. All you have to do is to take a big chunk of marble and a hammer and chisel, make up your mind what you are about to create and chip off all the marble you don't want.''-Paris Gaulois.
  \end{quote}
  
  {\tiny Source: \url{https://quoteinvestigator.com/2014/06/22/chip-away/}}
\end{frame}

\begin{frame}
\frametitle{Basic Process}
\begin{center}
\begin{tikzpicture}
\node[shape=rectangle,fill=pantone127-4] (problem) at (2,5) {Problem};
\node (human) at (2,4) {Human};
\node[shape=rectangle,fill=pantone127-4] (model) at (2,3) {Model};
\node (solver) at (2,2) {Constraint Solver/Search};
\node[shape=rectangle,fill=pantone157-8] (solution) at (2,1) {Solution};
\draw[-] (problem) -- (human);
\draw[->] (human) -- (model);
\draw[-] (model) -- (solver);
\draw[->] (solver) -- (solution);
\end{tikzpicture}
\end{center}
\end{frame}

\begin{frame}
\frametitle{More Realistic}
\begin{center}
\begin{tikzpicture}[xscale=4.0]
\node[shape=rectangle,fill=pantone127-4] (problem) at (2,5) {Problem};
\node (human) at (2,4) {Human};
\node[shape=rectangle,fill=pantone127-4] (model) at (2,3) {Model};
\node (solver) at (2,2) {Constraint Solver/Search};
\node[shape=rectangle,fill=pantone24-8] (hangs) at (1,1) {Hangs};
\node[shape=rectangle,fill=pantone157-8] (solution) at (2,1) {Solution};
\node[shape=rectangle,fill=pantone24-8] (wrong) at (3,1) {Wrong Solution};
\draw[-] (problem) -- (human);
\draw[->] (human) -- (model);
\draw[-] (model) -- (solver);
\draw[->] (solver) -- (hangs);
\draw[->] (solver) -- (solution);
\draw[->] (solver) -- (wrong);
\draw[->] (hangs) -- (1,4) -- (human);
\draw[->] (wrong) -- (3,4) -- (human);
\end{tikzpicture}
\end{center}
\end{frame}

\begin{frame}
\frametitle{Dual Role of Model}
\begin{itemize}
\item Allows Human to Express Problem
\begin{itemize}
\item Close to Problem Domain
\item Constraints as Abstractions
\end{itemize}
\item Allows Solver to Execute
\begin{itemize}
\item Variables as Communication Mechanism
\item Constraints as Algorithms
\end{itemize}
\end{itemize}
\end{frame}

\begin{frame}
\frametitle{Modelling Frameworks}
\begin{itemize}
\item \href{https://www.minizinc.org/}{MiniZinc} (NICTA, Monash University, Australia)
\item \href{https://github.com/eomahony/Numberjack}{NumberJack} (Insight, Ireland) 
\item \href{https://www-users.york.ac.uk/peter.nightingale/savilerow//}{EssencePrime/SavilleRow} (UK) 
\item \href{https://cpmpy.readthedocs.io/en/latest/}{CPMpy} (KU Leuven)
\item Allow use of multiple back-end solvers
\item Compile model into variants for each solver
\item A priori solver independent model(CP, MIP, SAT)
\end{itemize}
\end{frame} 

\begin{frame}
\frametitle{Framework Process}
\begin{center}
\begin{tikzpicture}[xscale=2]
\node[shape=rectangle,fill=pantone127-4] (problem) at (2.5,5) {Problem};
\node (human) at (2.5,4) {Human};
\node[shape=rectangle,fill=pantone127-4] (model) at (2.5,3) {Model};
\node (compile) at (2.5,2) {Compile/Reformulate};
\node (solver1) at (1,1) {CP};
\node (solver2) at (2,1) {MIP};
\node (solver3) at (3,1) {SAT};
\node (solver4) at (4,1) {Other};
\node[shape=rectangle,fill=pantone157-8] (solution1) at (1,0) {Solution};
\node[shape=rectangle,fill=pantone157-8] (solution2) at (2,0) {Solution};
\node[shape=rectangle,fill=pantone157-8] (solution3) at (3,0) {Solution};
\node[shape=rectangle,fill=pantone157-8] (solution4) at (4,0) {Solution};
\draw[-] (problem) -- (human);
\draw[->] (human) -- (model);
\draw[-] (model) -- (compile);
\draw[-] (compile) -- (solver1);
\draw[-] (compile) -- (solver2);
\draw[-] (compile) -- (solver3);
\draw[-] (compile) -- (solver4);
\draw[->] (solver1) -- (solution1);
\draw[->] (solver2) -- (solution2);
\draw[->] (solver3) -- (solution3);
\draw[->] (solver4) -- (solution4);
\end{tikzpicture}
\end{center}
\end{frame}

% \begin{frame}
  % \frametitle{Do It Now!}
  % \begin{itemize}
  % \item Download and install Minizinc
  % \item \url{https://www.minizinc.org/}
  % \end{itemize}
% \end{frame}


\begin{frame}
\frametitle{Why use Puzzles as Examples?}
\begin{itemize}
\item Easy to understand the problem
\item Solvable by hand without specialized knowledge
\item Possible to compare automated to manual solving process
\end{itemize}

\begin{quote}
The puzzle, though inanimate, is presented as a solvable problem without lasting negative consequences, a very low-risk low-reward situation. By being a puzzle, the object is attempting to convince the user that it must be completed.
\end{quote}
{\tiny Source: \href{https://www.everydayrhetoric.com/2020/03/the-rhetoric-of-puzzles/}{Every Day Rhetoric}}
\end{frame}


\mode<all>{
\part{Basic Constraint Propagation}
}
\begin{frame}
\partpage
\end{frame}
\mode<all>{
\input{../sendmore/body}
}

\mode<all>{
\part{Global Constraints}
}
\begin{frame}
\partpage
\end{frame}
\mode<all>{
\input{../sudoku/body}
}

\clearpage
\mode<all>{
\part{Customizing Search}
}
\begin{frame}
\partpage
\end{frame}
\mode<all>{
\input{../nqueen/body}
}


%% \clearpage
%% \mode<all>{
%%   \part{Problem Decomposition}
%% }
%% \begin{frame}
%%   \partpage
%% \end{frame}
%% \mode<all>{
%%   \input{../party/body}
%%   }

%% \clearpage
%% \mode<all>{
%%   \part{More Global Constraints}
%% }
%% \begin{frame}
%%   \partpage
%% \end{frame}
%% \mode<all>{
%%   \input{../car/body}
%%   }

%% \clearpage
%% \mode<all>{
%%   \part{Methodology}
%% }
%% \begin{frame}
%%   \partpage
%% \end{frame}
%% \mode<all>{
%%   \input{../methodology/body}
%%   }

\clearpage
\mode<all>{
\part{What is missing?}
}
\begin{frame}
\partpage
\end{frame}

\begin{frame}
\frametitle{Many Specialized Topics}
\begin{itemize}
\item How to design efficient core engine
\item Hybrids with LP/MIP tools
\item Hybrids with SAT
\item Symmetry breaking
\item Use of MDD/BDD to encode sets of solutions
\item High level modelling tools
\item Debugging/visualization
\end{itemize}
\end{frame}

\begin{frame}
\frametitle{Reformulation}
\begin{itemize}
\item Just because the user has modelled it this way, it doesn't mean we have to solve it that way
\begin{itemize}
\item Replace some constraint(s) by other, equivalent constraints
\item Because we don't have that constraint in our system
\item For performance
\end{itemize}
\end{itemize}
\end{frame}

\begin{frame}
\frametitle{Learning}
\begin{itemize}
\item While solving the problem we can learn how to strengthen the model/search 
\begin{itemize}
\item Understand which constraints/method contribute to propagation and change schedule
\item Learn no-good constraints by explaining failure
\item Adapt search strategy based on search experience
\end{itemize}
\end{itemize}
\end{frame}

%% \begin{frame}
%% \frametitle{Refined Process}
%% \begin{center}
%% \begin{tikzpicture}
%% \node[shape=rectangle,fill=pantone127-4] (problem) at (3,7) {Problem};
%% \node (human) at (3,6) {Human};
%% \node[shape=rectangle,fill=pantone127-4] (model) at (3,5) {User Model};
%% \node (reformulation) at (3,4) {Reformulation};
%% \node[shape=rectangle,fill=pantone127-4] (implementation) at (3,3) {Implementation Model};
%% \node (solver) at (3,2) {Constraint Solver/Search};
%% \node[shape=rectangle,fill=pantone157-8] (solution) at (3,1) {Solution};
%% \draw[-] (problem) -- (human);
%% \draw[->] (human) -- (model);
%% \draw[-] (model) -- (reformulation);
%% \draw[->] (reformulation) -- (implementation);
%% \draw[-] (implementation) -- (solver);
%% \draw[->] (solver) -- (0,2) -- (0,3) -- (implementation);
%% \draw[->] (solver) -- (solution);
%% \end{tikzpicture}
%% \end{center}
%% \end{frame}

%\begin{frame}
%\frametitle{What is CP actually used for?}
%\url{http://hsimonis.wordpress.com}
%\includegraphics[width=10cm]{../introduction/images/blog}
%\end{frame}

%% \begin{frame}
%% \frametitle{Exercises}
%% \begin{itemize}
%% \item Install Constraint Solver of your Choice
%% \item MiniZinc (MiningZinc) will be used in later sessions
%% \item Run Sendmoremoney, Sudoku, Queens examples
%% \item Can you choose which propagator to use for alldifferent?
%% \item Can you use different search strategies?
%% \end{itemize}
%% \end{frame}

\begin{frame}
\frametitle{More Learning Resources}
\begin{itemize}
\item Survey of Methods, Resources, and Formats for Teaching Constraint Programming
\begin{itemize}
\item by Tejas Santanam, Helmut Simonis
\item \url{https://doi.org/10.48550/arXiv.2403.12717}
\item Based on survey of community for WTCP 2023
\item \url{https://hsimonis.github.io/WTCP2023/}
\end{itemize}
\end{itemize}
\end{frame}


%%   }

\clearpage
\mode<all>{
\part{What is missing?}
}
\begin{frame}
\partpage
\end{frame}

\begin{frame}
\frametitle{Many Specialized Topics}
\begin{itemize}
\item How to design efficient core engine
\item Hybrids with LP/MIP tools
\item Hybrids with SAT
\item Symmetry breaking
\item Use of MDD/BDD to encode sets of solutions
\item High level modelling tools
\item Debugging/visualization
\end{itemize}
\end{frame}

\begin{frame}
\frametitle{Reformulation}
\begin{itemize}
\item Just because the user has modelled it this way, it doesn't mean we have to solve it that way
\begin{itemize}
\item Replace some constraint(s) by other, equivalent constraints
\item Because we don't have that constraint in our system
\item For performance
\end{itemize}
\end{itemize}
\end{frame}

\begin{frame}
\frametitle{Learning}
\begin{itemize}
\item While solving the problem we can learn how to strengthen the model/search 
\begin{itemize}
\item Understand which constraints/method contribute to propagation and change schedule
\item Learn no-good constraints by explaining failure
\item Adapt search strategy based on search experience
\end{itemize}
\end{itemize}
\end{frame}

%% \begin{frame}
%% \frametitle{Refined Process}
%% \begin{center}
%% \begin{tikzpicture}
%% \node[shape=rectangle,fill=pantone127-4] (problem) at (3,7) {Problem};
%% \node (human) at (3,6) {Human};
%% \node[shape=rectangle,fill=pantone127-4] (model) at (3,5) {User Model};
%% \node (reformulation) at (3,4) {Reformulation};
%% \node[shape=rectangle,fill=pantone127-4] (implementation) at (3,3) {Implementation Model};
%% \node (solver) at (3,2) {Constraint Solver/Search};
%% \node[shape=rectangle,fill=pantone157-8] (solution) at (3,1) {Solution};
%% \draw[-] (problem) -- (human);
%% \draw[->] (human) -- (model);
%% \draw[-] (model) -- (reformulation);
%% \draw[->] (reformulation) -- (implementation);
%% \draw[-] (implementation) -- (solver);
%% \draw[->] (solver) -- (0,2) -- (0,3) -- (implementation);
%% \draw[->] (solver) -- (solution);
%% \end{tikzpicture}
%% \end{center}
%% \end{frame}

%\begin{frame}
%\frametitle{What is CP actually used for?}
%\url{http://hsimonis.wordpress.com}
%\includegraphics[width=10cm]{../introduction/images/blog}
%\end{frame}

%% \begin{frame}
%% \frametitle{Exercises}
%% \begin{itemize}
%% \item Install Constraint Solver of your Choice
%% \item MiniZinc (MiningZinc) will be used in later sessions
%% \item Run Sendmoremoney, Sudoku, Queens examples
%% \item Can you choose which propagator to use for alldifferent?
%% \item Can you use different search strategies?
%% \end{itemize}
%% \end{frame}

\begin{frame}
\frametitle{More Learning Resources}
\begin{itemize}
\item Survey of Methods, Resources, and Formats for Teaching Constraint Programming
\begin{itemize}
\item by Tejas Santanam, Helmut Simonis
\item \url{https://doi.org/10.48550/arXiv.2403.12717}
\item Based on survey of community for WTCP 2023
\item \url{https://hsimonis.github.io/WTCP2023/}
\end{itemize}
\end{itemize}
\end{frame}


%%   }

%% \clearpage
%% \mode<all>{
%%   \part{More Global Constraints}
%% }
%% \begin{frame}
%%   \partpage
%% \end{frame}
%% \mode<all>{
%%   \begin{frame}
\frametitle{Objectives}
\begin{itemize}
\item Overview of Core Constraint Programming
\item Three Main Concepts
\begin{itemize}
\item Constraint Propagation
\item Global Constraints
\item Customizing Search
\end{itemize}
%\item Get Some Experience with MiniZinc
\item Topics will be treated in more detail in later parts of the school
\item Based on Examples, not Formal Description
\end{itemize}
\end{frame}

\begin{frame}
\frametitle{Outline}
\begin{itemize}
\item Why Constraint Programming?
\item Constraint Propagation
\item Global Constraints
\item Customizing Search
%\item Choices in Modelling
%\item Decomposition
%\item More Global Constraints
%\item Methodology
%\item What is missing?
\end{itemize}
\end{frame}

% \begin{frame}
  % \frametitle{Using MiniZinc IDE}
  % \begin{itemize}
  % \item Developed in the Australian NICTA project
  % \item Maintained by Monash University
  % \item Modelling tool with multiple back-end solvers
  % \item Available from \url{https://www.minizinc.org/}
  % \end{itemize}
% \end{frame}


% \begin{frame}
% \frametitle{Examples in ECLiPSe}
% \begin{itemize}
% \item Open sourced constraint programming language
% \item Development goes back to 1985
% \item ECRC, ICL, IC-Parc, PTL, Cisco
% \item \url{https://eclipseclp.org/}
% \item Specialities
% \begin{itemize}
% \item Develop new solvers for specific domains
% \item Integration with MIP
% \end{itemize}
% \item Not included in bundled MiniZinc IDE
% \item Specialized visualization tools used here
  % \begin{itemize}
  % \item CP-Viz, Simonis et al. 2010
    % \end{itemize}
% \end{itemize}
% \end{frame}

\begin{frame}
\frametitle{Tutorial Based on ECLiPSe ELearning Course}
\begin{itemize}
\item Self-study course in constraint programming
\item Supported by Cisco Systems and Silicon Valley Community Foundation
\item Multi-media format, video lectures, slides, handout etc
\item {\small \url{https://eclipseclp.org/ELearning/index.html}}
\end{itemize}
\end{frame}

\begin{frame}
\frametitle{Also Part of CRT-AI Constraint Week}
\begin{itemize}
\item Annual one week course on CP and Optimization in Ireland
\item Part of national training program for PhD students in AI
\item \url{https://www.crt-ai.ie/}
\end{itemize}
\end{frame}



\begin{frame}
\frametitle{Constraint Programming - in a nutshell}
\begin{itemize}
\item Declarative description of problems with
\begin{itemize}
\item {\em Variables} which range over (finite) sets of values
\item {\em Constraints} over subsets of variables which restrict possible value combinations
\item A {\em solution} is a value assignment which satisfies all constraints
\end{itemize}

\item Constraint propagation/reasoning
\begin{itemize}
\item Removing inconsistent values for variables
\item Detect failure if constraint can not be satisfied
\item Interaction of constraints via shared variables
\item Incomplete
\end{itemize}

\item Search
\begin{itemize}
\item User controlled assignment of values to variables
\item Each step triggers constraint propagation 
\end{itemize}
\item Different domains require/allow different methods
\end{itemize}
\end{frame}

\begin{frame}
  \frametitle{Constraint Programming is Different}
  \begin{itemize}
  \item Declarative Programming
    \begin{itemize}
    \item Concentrate on what you want
      \item Not how to get there
      \item Program != Algorithm
      \item Program = Model
    \end{itemize}
    \item Applied to Combinatorial Problems
      \begin{itemize}
        \item No complete polynomial algorithms known (exist?)
        \item CP less ad-hoc than heuristics
        \item Models can evolve
  \end{itemize}
  \end{itemize}
  \end{frame}
    
\begin{frame}
  \frametitle{A Subtractive Process}
  \begin{textblock}{4}(8,-3)
    \includegraphics[width=4cm]{../introduction/images/stages}
  \end{textblock}
  \vfill
  \begin{quote}
    ``Oh, bosh, as Mr. Ruskin says. Sculpture, per se, is the simplest thing in the world. All you have to do is to take a big chunk of marble and a hammer and chisel, make up your mind what you are about to create and chip off all the marble you don't want.''-Paris Gaulois.
  \end{quote}
  
  {\tiny Source: \url{https://quoteinvestigator.com/2014/06/22/chip-away/}}
\end{frame}

\begin{frame}
\frametitle{Basic Process}
\begin{center}
\begin{tikzpicture}
\node[shape=rectangle,fill=pantone127-4] (problem) at (2,5) {Problem};
\node (human) at (2,4) {Human};
\node[shape=rectangle,fill=pantone127-4] (model) at (2,3) {Model};
\node (solver) at (2,2) {Constraint Solver/Search};
\node[shape=rectangle,fill=pantone157-8] (solution) at (2,1) {Solution};
\draw[-] (problem) -- (human);
\draw[->] (human) -- (model);
\draw[-] (model) -- (solver);
\draw[->] (solver) -- (solution);
\end{tikzpicture}
\end{center}
\end{frame}

\begin{frame}
\frametitle{More Realistic}
\begin{center}
\begin{tikzpicture}[xscale=4.0]
\node[shape=rectangle,fill=pantone127-4] (problem) at (2,5) {Problem};
\node (human) at (2,4) {Human};
\node[shape=rectangle,fill=pantone127-4] (model) at (2,3) {Model};
\node (solver) at (2,2) {Constraint Solver/Search};
\node[shape=rectangle,fill=pantone24-8] (hangs) at (1,1) {Hangs};
\node[shape=rectangle,fill=pantone157-8] (solution) at (2,1) {Solution};
\node[shape=rectangle,fill=pantone24-8] (wrong) at (3,1) {Wrong Solution};
\draw[-] (problem) -- (human);
\draw[->] (human) -- (model);
\draw[-] (model) -- (solver);
\draw[->] (solver) -- (hangs);
\draw[->] (solver) -- (solution);
\draw[->] (solver) -- (wrong);
\draw[->] (hangs) -- (1,4) -- (human);
\draw[->] (wrong) -- (3,4) -- (human);
\end{tikzpicture}
\end{center}
\end{frame}

\begin{frame}
\frametitle{Dual Role of Model}
\begin{itemize}
\item Allows Human to Express Problem
\begin{itemize}
\item Close to Problem Domain
\item Constraints as Abstractions
\end{itemize}
\item Allows Solver to Execute
\begin{itemize}
\item Variables as Communication Mechanism
\item Constraints as Algorithms
\end{itemize}
\end{itemize}
\end{frame}

\begin{frame}
\frametitle{Modelling Frameworks}
\begin{itemize}
\item \href{https://www.minizinc.org/}{MiniZinc} (NICTA, Monash University, Australia)
\item \href{https://github.com/eomahony/Numberjack}{NumberJack} (Insight, Ireland) 
\item \href{https://www-users.york.ac.uk/peter.nightingale/savilerow//}{EssencePrime/SavilleRow} (UK) 
\item \href{https://cpmpy.readthedocs.io/en/latest/}{CPMpy} (KU Leuven)
\item Allow use of multiple back-end solvers
\item Compile model into variants for each solver
\item A priori solver independent model(CP, MIP, SAT)
\end{itemize}
\end{frame} 

\begin{frame}
\frametitle{Framework Process}
\begin{center}
\begin{tikzpicture}[xscale=2]
\node[shape=rectangle,fill=pantone127-4] (problem) at (2.5,5) {Problem};
\node (human) at (2.5,4) {Human};
\node[shape=rectangle,fill=pantone127-4] (model) at (2.5,3) {Model};
\node (compile) at (2.5,2) {Compile/Reformulate};
\node (solver1) at (1,1) {CP};
\node (solver2) at (2,1) {MIP};
\node (solver3) at (3,1) {SAT};
\node (solver4) at (4,1) {Other};
\node[shape=rectangle,fill=pantone157-8] (solution1) at (1,0) {Solution};
\node[shape=rectangle,fill=pantone157-8] (solution2) at (2,0) {Solution};
\node[shape=rectangle,fill=pantone157-8] (solution3) at (3,0) {Solution};
\node[shape=rectangle,fill=pantone157-8] (solution4) at (4,0) {Solution};
\draw[-] (problem) -- (human);
\draw[->] (human) -- (model);
\draw[-] (model) -- (compile);
\draw[-] (compile) -- (solver1);
\draw[-] (compile) -- (solver2);
\draw[-] (compile) -- (solver3);
\draw[-] (compile) -- (solver4);
\draw[->] (solver1) -- (solution1);
\draw[->] (solver2) -- (solution2);
\draw[->] (solver3) -- (solution3);
\draw[->] (solver4) -- (solution4);
\end{tikzpicture}
\end{center}
\end{frame}

% \begin{frame}
  % \frametitle{Do It Now!}
  % \begin{itemize}
  % \item Download and install Minizinc
  % \item \url{https://www.minizinc.org/}
  % \end{itemize}
% \end{frame}


\begin{frame}
\frametitle{Why use Puzzles as Examples?}
\begin{itemize}
\item Easy to understand the problem
\item Solvable by hand without specialized knowledge
\item Possible to compare automated to manual solving process
\end{itemize}

\begin{quote}
The puzzle, though inanimate, is presented as a solvable problem without lasting negative consequences, a very low-risk low-reward situation. By being a puzzle, the object is attempting to convince the user that it must be completed.
\end{quote}
{\tiny Source: \href{https://www.everydayrhetoric.com/2020/03/the-rhetoric-of-puzzles/}{Every Day Rhetoric}}
\end{frame}


\mode<all>{
\part{Basic Constraint Propagation}
}
\begin{frame}
\partpage
\end{frame}
\mode<all>{
\begin{frame}
\frametitle{Objectives}
\begin{itemize}
\item Overview of Core Constraint Programming
\item Three Main Concepts
\begin{itemize}
\item Constraint Propagation
\item Global Constraints
\item Customizing Search
\end{itemize}
%\item Get Some Experience with MiniZinc
\item Topics will be treated in more detail in later parts of the school
\item Based on Examples, not Formal Description
\end{itemize}
\end{frame}

\begin{frame}
\frametitle{Outline}
\begin{itemize}
\item Why Constraint Programming?
\item Constraint Propagation
\item Global Constraints
\item Customizing Search
%\item Choices in Modelling
%\item Decomposition
%\item More Global Constraints
%\item Methodology
%\item What is missing?
\end{itemize}
\end{frame}

% \begin{frame}
  % \frametitle{Using MiniZinc IDE}
  % \begin{itemize}
  % \item Developed in the Australian NICTA project
  % \item Maintained by Monash University
  % \item Modelling tool with multiple back-end solvers
  % \item Available from \url{https://www.minizinc.org/}
  % \end{itemize}
% \end{frame}


% \begin{frame}
% \frametitle{Examples in ECLiPSe}
% \begin{itemize}
% \item Open sourced constraint programming language
% \item Development goes back to 1985
% \item ECRC, ICL, IC-Parc, PTL, Cisco
% \item \url{https://eclipseclp.org/}
% \item Specialities
% \begin{itemize}
% \item Develop new solvers for specific domains
% \item Integration with MIP
% \end{itemize}
% \item Not included in bundled MiniZinc IDE
% \item Specialized visualization tools used here
  % \begin{itemize}
  % \item CP-Viz, Simonis et al. 2010
    % \end{itemize}
% \end{itemize}
% \end{frame}

\begin{frame}
\frametitle{Tutorial Based on ECLiPSe ELearning Course}
\begin{itemize}
\item Self-study course in constraint programming
\item Supported by Cisco Systems and Silicon Valley Community Foundation
\item Multi-media format, video lectures, slides, handout etc
\item {\small \url{https://eclipseclp.org/ELearning/index.html}}
\end{itemize}
\end{frame}

\begin{frame}
\frametitle{Also Part of CRT-AI Constraint Week}
\begin{itemize}
\item Annual one week course on CP and Optimization in Ireland
\item Part of national training program for PhD students in AI
\item \url{https://www.crt-ai.ie/}
\end{itemize}
\end{frame}



\begin{frame}
\frametitle{Constraint Programming - in a nutshell}
\begin{itemize}
\item Declarative description of problems with
\begin{itemize}
\item {\em Variables} which range over (finite) sets of values
\item {\em Constraints} over subsets of variables which restrict possible value combinations
\item A {\em solution} is a value assignment which satisfies all constraints
\end{itemize}

\item Constraint propagation/reasoning
\begin{itemize}
\item Removing inconsistent values for variables
\item Detect failure if constraint can not be satisfied
\item Interaction of constraints via shared variables
\item Incomplete
\end{itemize}

\item Search
\begin{itemize}
\item User controlled assignment of values to variables
\item Each step triggers constraint propagation 
\end{itemize}
\item Different domains require/allow different methods
\end{itemize}
\end{frame}

\begin{frame}
  \frametitle{Constraint Programming is Different}
  \begin{itemize}
  \item Declarative Programming
    \begin{itemize}
    \item Concentrate on what you want
      \item Not how to get there
      \item Program != Algorithm
      \item Program = Model
    \end{itemize}
    \item Applied to Combinatorial Problems
      \begin{itemize}
        \item No complete polynomial algorithms known (exist?)
        \item CP less ad-hoc than heuristics
        \item Models can evolve
  \end{itemize}
  \end{itemize}
  \end{frame}
    
\begin{frame}
  \frametitle{A Subtractive Process}
  \begin{textblock}{4}(8,-3)
    \includegraphics[width=4cm]{../introduction/images/stages}
  \end{textblock}
  \vfill
  \begin{quote}
    ``Oh, bosh, as Mr. Ruskin says. Sculpture, per se, is the simplest thing in the world. All you have to do is to take a big chunk of marble and a hammer and chisel, make up your mind what you are about to create and chip off all the marble you don't want.''-Paris Gaulois.
  \end{quote}
  
  {\tiny Source: \url{https://quoteinvestigator.com/2014/06/22/chip-away/}}
\end{frame}

\begin{frame}
\frametitle{Basic Process}
\begin{center}
\begin{tikzpicture}
\node[shape=rectangle,fill=pantone127-4] (problem) at (2,5) {Problem};
\node (human) at (2,4) {Human};
\node[shape=rectangle,fill=pantone127-4] (model) at (2,3) {Model};
\node (solver) at (2,2) {Constraint Solver/Search};
\node[shape=rectangle,fill=pantone157-8] (solution) at (2,1) {Solution};
\draw[-] (problem) -- (human);
\draw[->] (human) -- (model);
\draw[-] (model) -- (solver);
\draw[->] (solver) -- (solution);
\end{tikzpicture}
\end{center}
\end{frame}

\begin{frame}
\frametitle{More Realistic}
\begin{center}
\begin{tikzpicture}[xscale=4.0]
\node[shape=rectangle,fill=pantone127-4] (problem) at (2,5) {Problem};
\node (human) at (2,4) {Human};
\node[shape=rectangle,fill=pantone127-4] (model) at (2,3) {Model};
\node (solver) at (2,2) {Constraint Solver/Search};
\node[shape=rectangle,fill=pantone24-8] (hangs) at (1,1) {Hangs};
\node[shape=rectangle,fill=pantone157-8] (solution) at (2,1) {Solution};
\node[shape=rectangle,fill=pantone24-8] (wrong) at (3,1) {Wrong Solution};
\draw[-] (problem) -- (human);
\draw[->] (human) -- (model);
\draw[-] (model) -- (solver);
\draw[->] (solver) -- (hangs);
\draw[->] (solver) -- (solution);
\draw[->] (solver) -- (wrong);
\draw[->] (hangs) -- (1,4) -- (human);
\draw[->] (wrong) -- (3,4) -- (human);
\end{tikzpicture}
\end{center}
\end{frame}

\begin{frame}
\frametitle{Dual Role of Model}
\begin{itemize}
\item Allows Human to Express Problem
\begin{itemize}
\item Close to Problem Domain
\item Constraints as Abstractions
\end{itemize}
\item Allows Solver to Execute
\begin{itemize}
\item Variables as Communication Mechanism
\item Constraints as Algorithms
\end{itemize}
\end{itemize}
\end{frame}

\begin{frame}
\frametitle{Modelling Frameworks}
\begin{itemize}
\item \href{https://www.minizinc.org/}{MiniZinc} (NICTA, Monash University, Australia)
\item \href{https://github.com/eomahony/Numberjack}{NumberJack} (Insight, Ireland) 
\item \href{https://www-users.york.ac.uk/peter.nightingale/savilerow//}{EssencePrime/SavilleRow} (UK) 
\item \href{https://cpmpy.readthedocs.io/en/latest/}{CPMpy} (KU Leuven)
\item Allow use of multiple back-end solvers
\item Compile model into variants for each solver
\item A priori solver independent model(CP, MIP, SAT)
\end{itemize}
\end{frame} 

\begin{frame}
\frametitle{Framework Process}
\begin{center}
\begin{tikzpicture}[xscale=2]
\node[shape=rectangle,fill=pantone127-4] (problem) at (2.5,5) {Problem};
\node (human) at (2.5,4) {Human};
\node[shape=rectangle,fill=pantone127-4] (model) at (2.5,3) {Model};
\node (compile) at (2.5,2) {Compile/Reformulate};
\node (solver1) at (1,1) {CP};
\node (solver2) at (2,1) {MIP};
\node (solver3) at (3,1) {SAT};
\node (solver4) at (4,1) {Other};
\node[shape=rectangle,fill=pantone157-8] (solution1) at (1,0) {Solution};
\node[shape=rectangle,fill=pantone157-8] (solution2) at (2,0) {Solution};
\node[shape=rectangle,fill=pantone157-8] (solution3) at (3,0) {Solution};
\node[shape=rectangle,fill=pantone157-8] (solution4) at (4,0) {Solution};
\draw[-] (problem) -- (human);
\draw[->] (human) -- (model);
\draw[-] (model) -- (compile);
\draw[-] (compile) -- (solver1);
\draw[-] (compile) -- (solver2);
\draw[-] (compile) -- (solver3);
\draw[-] (compile) -- (solver4);
\draw[->] (solver1) -- (solution1);
\draw[->] (solver2) -- (solution2);
\draw[->] (solver3) -- (solution3);
\draw[->] (solver4) -- (solution4);
\end{tikzpicture}
\end{center}
\end{frame}

% \begin{frame}
  % \frametitle{Do It Now!}
  % \begin{itemize}
  % \item Download and install Minizinc
  % \item \url{https://www.minizinc.org/}
  % \end{itemize}
% \end{frame}


\begin{frame}
\frametitle{Why use Puzzles as Examples?}
\begin{itemize}
\item Easy to understand the problem
\item Solvable by hand without specialized knowledge
\item Possible to compare automated to manual solving process
\end{itemize}

\begin{quote}
The puzzle, though inanimate, is presented as a solvable problem without lasting negative consequences, a very low-risk low-reward situation. By being a puzzle, the object is attempting to convince the user that it must be completed.
\end{quote}
{\tiny Source: \href{https://www.everydayrhetoric.com/2020/03/the-rhetoric-of-puzzles/}{Every Day Rhetoric}}
\end{frame}


\mode<all>{
\part{Basic Constraint Propagation}
}
\begin{frame}
\partpage
\end{frame}
\mode<all>{
\input{../sendmore/body}
}

\mode<all>{
\part{Global Constraints}
}
\begin{frame}
\partpage
\end{frame}
\mode<all>{
\input{../sudoku/body}
}

\clearpage
\mode<all>{
\part{Customizing Search}
}
\begin{frame}
\partpage
\end{frame}
\mode<all>{
\input{../nqueen/body}
}


%% \clearpage
%% \mode<all>{
%%   \part{Problem Decomposition}
%% }
%% \begin{frame}
%%   \partpage
%% \end{frame}
%% \mode<all>{
%%   \input{../party/body}
%%   }

%% \clearpage
%% \mode<all>{
%%   \part{More Global Constraints}
%% }
%% \begin{frame}
%%   \partpage
%% \end{frame}
%% \mode<all>{
%%   \input{../car/body}
%%   }

%% \clearpage
%% \mode<all>{
%%   \part{Methodology}
%% }
%% \begin{frame}
%%   \partpage
%% \end{frame}
%% \mode<all>{
%%   \input{../methodology/body}
%%   }

\clearpage
\mode<all>{
\part{What is missing?}
}
\begin{frame}
\partpage
\end{frame}

\begin{frame}
\frametitle{Many Specialized Topics}
\begin{itemize}
\item How to design efficient core engine
\item Hybrids with LP/MIP tools
\item Hybrids with SAT
\item Symmetry breaking
\item Use of MDD/BDD to encode sets of solutions
\item High level modelling tools
\item Debugging/visualization
\end{itemize}
\end{frame}

\begin{frame}
\frametitle{Reformulation}
\begin{itemize}
\item Just because the user has modelled it this way, it doesn't mean we have to solve it that way
\begin{itemize}
\item Replace some constraint(s) by other, equivalent constraints
\item Because we don't have that constraint in our system
\item For performance
\end{itemize}
\end{itemize}
\end{frame}

\begin{frame}
\frametitle{Learning}
\begin{itemize}
\item While solving the problem we can learn how to strengthen the model/search 
\begin{itemize}
\item Understand which constraints/method contribute to propagation and change schedule
\item Learn no-good constraints by explaining failure
\item Adapt search strategy based on search experience
\end{itemize}
\end{itemize}
\end{frame}

%% \begin{frame}
%% \frametitle{Refined Process}
%% \begin{center}
%% \begin{tikzpicture}
%% \node[shape=rectangle,fill=pantone127-4] (problem) at (3,7) {Problem};
%% \node (human) at (3,6) {Human};
%% \node[shape=rectangle,fill=pantone127-4] (model) at (3,5) {User Model};
%% \node (reformulation) at (3,4) {Reformulation};
%% \node[shape=rectangle,fill=pantone127-4] (implementation) at (3,3) {Implementation Model};
%% \node (solver) at (3,2) {Constraint Solver/Search};
%% \node[shape=rectangle,fill=pantone157-8] (solution) at (3,1) {Solution};
%% \draw[-] (problem) -- (human);
%% \draw[->] (human) -- (model);
%% \draw[-] (model) -- (reformulation);
%% \draw[->] (reformulation) -- (implementation);
%% \draw[-] (implementation) -- (solver);
%% \draw[->] (solver) -- (0,2) -- (0,3) -- (implementation);
%% \draw[->] (solver) -- (solution);
%% \end{tikzpicture}
%% \end{center}
%% \end{frame}

%\begin{frame}
%\frametitle{What is CP actually used for?}
%\url{http://hsimonis.wordpress.com}
%\includegraphics[width=10cm]{../introduction/images/blog}
%\end{frame}

%% \begin{frame}
%% \frametitle{Exercises}
%% \begin{itemize}
%% \item Install Constraint Solver of your Choice
%% \item MiniZinc (MiningZinc) will be used in later sessions
%% \item Run Sendmoremoney, Sudoku, Queens examples
%% \item Can you choose which propagator to use for alldifferent?
%% \item Can you use different search strategies?
%% \end{itemize}
%% \end{frame}

\begin{frame}
\frametitle{More Learning Resources}
\begin{itemize}
\item Survey of Methods, Resources, and Formats for Teaching Constraint Programming
\begin{itemize}
\item by Tejas Santanam, Helmut Simonis
\item \url{https://doi.org/10.48550/arXiv.2403.12717}
\item Based on survey of community for WTCP 2023
\item \url{https://hsimonis.github.io/WTCP2023/}
\end{itemize}
\end{itemize}
\end{frame}


}

\mode<all>{
\part{Global Constraints}
}
\begin{frame}
\partpage
\end{frame}
\mode<all>{
\begin{frame}
\frametitle{Objectives}
\begin{itemize}
\item Overview of Core Constraint Programming
\item Three Main Concepts
\begin{itemize}
\item Constraint Propagation
\item Global Constraints
\item Customizing Search
\end{itemize}
%\item Get Some Experience with MiniZinc
\item Topics will be treated in more detail in later parts of the school
\item Based on Examples, not Formal Description
\end{itemize}
\end{frame}

\begin{frame}
\frametitle{Outline}
\begin{itemize}
\item Why Constraint Programming?
\item Constraint Propagation
\item Global Constraints
\item Customizing Search
%\item Choices in Modelling
%\item Decomposition
%\item More Global Constraints
%\item Methodology
%\item What is missing?
\end{itemize}
\end{frame}

% \begin{frame}
  % \frametitle{Using MiniZinc IDE}
  % \begin{itemize}
  % \item Developed in the Australian NICTA project
  % \item Maintained by Monash University
  % \item Modelling tool with multiple back-end solvers
  % \item Available from \url{https://www.minizinc.org/}
  % \end{itemize}
% \end{frame}


% \begin{frame}
% \frametitle{Examples in ECLiPSe}
% \begin{itemize}
% \item Open sourced constraint programming language
% \item Development goes back to 1985
% \item ECRC, ICL, IC-Parc, PTL, Cisco
% \item \url{https://eclipseclp.org/}
% \item Specialities
% \begin{itemize}
% \item Develop new solvers for specific domains
% \item Integration with MIP
% \end{itemize}
% \item Not included in bundled MiniZinc IDE
% \item Specialized visualization tools used here
  % \begin{itemize}
  % \item CP-Viz, Simonis et al. 2010
    % \end{itemize}
% \end{itemize}
% \end{frame}

\begin{frame}
\frametitle{Tutorial Based on ECLiPSe ELearning Course}
\begin{itemize}
\item Self-study course in constraint programming
\item Supported by Cisco Systems and Silicon Valley Community Foundation
\item Multi-media format, video lectures, slides, handout etc
\item {\small \url{https://eclipseclp.org/ELearning/index.html}}
\end{itemize}
\end{frame}

\begin{frame}
\frametitle{Also Part of CRT-AI Constraint Week}
\begin{itemize}
\item Annual one week course on CP and Optimization in Ireland
\item Part of national training program for PhD students in AI
\item \url{https://www.crt-ai.ie/}
\end{itemize}
\end{frame}



\begin{frame}
\frametitle{Constraint Programming - in a nutshell}
\begin{itemize}
\item Declarative description of problems with
\begin{itemize}
\item {\em Variables} which range over (finite) sets of values
\item {\em Constraints} over subsets of variables which restrict possible value combinations
\item A {\em solution} is a value assignment which satisfies all constraints
\end{itemize}

\item Constraint propagation/reasoning
\begin{itemize}
\item Removing inconsistent values for variables
\item Detect failure if constraint can not be satisfied
\item Interaction of constraints via shared variables
\item Incomplete
\end{itemize}

\item Search
\begin{itemize}
\item User controlled assignment of values to variables
\item Each step triggers constraint propagation 
\end{itemize}
\item Different domains require/allow different methods
\end{itemize}
\end{frame}

\begin{frame}
  \frametitle{Constraint Programming is Different}
  \begin{itemize}
  \item Declarative Programming
    \begin{itemize}
    \item Concentrate on what you want
      \item Not how to get there
      \item Program != Algorithm
      \item Program = Model
    \end{itemize}
    \item Applied to Combinatorial Problems
      \begin{itemize}
        \item No complete polynomial algorithms known (exist?)
        \item CP less ad-hoc than heuristics
        \item Models can evolve
  \end{itemize}
  \end{itemize}
  \end{frame}
    
\begin{frame}
  \frametitle{A Subtractive Process}
  \begin{textblock}{4}(8,-3)
    \includegraphics[width=4cm]{../introduction/images/stages}
  \end{textblock}
  \vfill
  \begin{quote}
    ``Oh, bosh, as Mr. Ruskin says. Sculpture, per se, is the simplest thing in the world. All you have to do is to take a big chunk of marble and a hammer and chisel, make up your mind what you are about to create and chip off all the marble you don't want.''-Paris Gaulois.
  \end{quote}
  
  {\tiny Source: \url{https://quoteinvestigator.com/2014/06/22/chip-away/}}
\end{frame}

\begin{frame}
\frametitle{Basic Process}
\begin{center}
\begin{tikzpicture}
\node[shape=rectangle,fill=pantone127-4] (problem) at (2,5) {Problem};
\node (human) at (2,4) {Human};
\node[shape=rectangle,fill=pantone127-4] (model) at (2,3) {Model};
\node (solver) at (2,2) {Constraint Solver/Search};
\node[shape=rectangle,fill=pantone157-8] (solution) at (2,1) {Solution};
\draw[-] (problem) -- (human);
\draw[->] (human) -- (model);
\draw[-] (model) -- (solver);
\draw[->] (solver) -- (solution);
\end{tikzpicture}
\end{center}
\end{frame}

\begin{frame}
\frametitle{More Realistic}
\begin{center}
\begin{tikzpicture}[xscale=4.0]
\node[shape=rectangle,fill=pantone127-4] (problem) at (2,5) {Problem};
\node (human) at (2,4) {Human};
\node[shape=rectangle,fill=pantone127-4] (model) at (2,3) {Model};
\node (solver) at (2,2) {Constraint Solver/Search};
\node[shape=rectangle,fill=pantone24-8] (hangs) at (1,1) {Hangs};
\node[shape=rectangle,fill=pantone157-8] (solution) at (2,1) {Solution};
\node[shape=rectangle,fill=pantone24-8] (wrong) at (3,1) {Wrong Solution};
\draw[-] (problem) -- (human);
\draw[->] (human) -- (model);
\draw[-] (model) -- (solver);
\draw[->] (solver) -- (hangs);
\draw[->] (solver) -- (solution);
\draw[->] (solver) -- (wrong);
\draw[->] (hangs) -- (1,4) -- (human);
\draw[->] (wrong) -- (3,4) -- (human);
\end{tikzpicture}
\end{center}
\end{frame}

\begin{frame}
\frametitle{Dual Role of Model}
\begin{itemize}
\item Allows Human to Express Problem
\begin{itemize}
\item Close to Problem Domain
\item Constraints as Abstractions
\end{itemize}
\item Allows Solver to Execute
\begin{itemize}
\item Variables as Communication Mechanism
\item Constraints as Algorithms
\end{itemize}
\end{itemize}
\end{frame}

\begin{frame}
\frametitle{Modelling Frameworks}
\begin{itemize}
\item \href{https://www.minizinc.org/}{MiniZinc} (NICTA, Monash University, Australia)
\item \href{https://github.com/eomahony/Numberjack}{NumberJack} (Insight, Ireland) 
\item \href{https://www-users.york.ac.uk/peter.nightingale/savilerow//}{EssencePrime/SavilleRow} (UK) 
\item \href{https://cpmpy.readthedocs.io/en/latest/}{CPMpy} (KU Leuven)
\item Allow use of multiple back-end solvers
\item Compile model into variants for each solver
\item A priori solver independent model(CP, MIP, SAT)
\end{itemize}
\end{frame} 

\begin{frame}
\frametitle{Framework Process}
\begin{center}
\begin{tikzpicture}[xscale=2]
\node[shape=rectangle,fill=pantone127-4] (problem) at (2.5,5) {Problem};
\node (human) at (2.5,4) {Human};
\node[shape=rectangle,fill=pantone127-4] (model) at (2.5,3) {Model};
\node (compile) at (2.5,2) {Compile/Reformulate};
\node (solver1) at (1,1) {CP};
\node (solver2) at (2,1) {MIP};
\node (solver3) at (3,1) {SAT};
\node (solver4) at (4,1) {Other};
\node[shape=rectangle,fill=pantone157-8] (solution1) at (1,0) {Solution};
\node[shape=rectangle,fill=pantone157-8] (solution2) at (2,0) {Solution};
\node[shape=rectangle,fill=pantone157-8] (solution3) at (3,0) {Solution};
\node[shape=rectangle,fill=pantone157-8] (solution4) at (4,0) {Solution};
\draw[-] (problem) -- (human);
\draw[->] (human) -- (model);
\draw[-] (model) -- (compile);
\draw[-] (compile) -- (solver1);
\draw[-] (compile) -- (solver2);
\draw[-] (compile) -- (solver3);
\draw[-] (compile) -- (solver4);
\draw[->] (solver1) -- (solution1);
\draw[->] (solver2) -- (solution2);
\draw[->] (solver3) -- (solution3);
\draw[->] (solver4) -- (solution4);
\end{tikzpicture}
\end{center}
\end{frame}

% \begin{frame}
  % \frametitle{Do It Now!}
  % \begin{itemize}
  % \item Download and install Minizinc
  % \item \url{https://www.minizinc.org/}
  % \end{itemize}
% \end{frame}


\begin{frame}
\frametitle{Why use Puzzles as Examples?}
\begin{itemize}
\item Easy to understand the problem
\item Solvable by hand without specialized knowledge
\item Possible to compare automated to manual solving process
\end{itemize}

\begin{quote}
The puzzle, though inanimate, is presented as a solvable problem without lasting negative consequences, a very low-risk low-reward situation. By being a puzzle, the object is attempting to convince the user that it must be completed.
\end{quote}
{\tiny Source: \href{https://www.everydayrhetoric.com/2020/03/the-rhetoric-of-puzzles/}{Every Day Rhetoric}}
\end{frame}


\mode<all>{
\part{Basic Constraint Propagation}
}
\begin{frame}
\partpage
\end{frame}
\mode<all>{
\input{../sendmore/body}
}

\mode<all>{
\part{Global Constraints}
}
\begin{frame}
\partpage
\end{frame}
\mode<all>{
\input{../sudoku/body}
}

\clearpage
\mode<all>{
\part{Customizing Search}
}
\begin{frame}
\partpage
\end{frame}
\mode<all>{
\input{../nqueen/body}
}


%% \clearpage
%% \mode<all>{
%%   \part{Problem Decomposition}
%% }
%% \begin{frame}
%%   \partpage
%% \end{frame}
%% \mode<all>{
%%   \input{../party/body}
%%   }

%% \clearpage
%% \mode<all>{
%%   \part{More Global Constraints}
%% }
%% \begin{frame}
%%   \partpage
%% \end{frame}
%% \mode<all>{
%%   \input{../car/body}
%%   }

%% \clearpage
%% \mode<all>{
%%   \part{Methodology}
%% }
%% \begin{frame}
%%   \partpage
%% \end{frame}
%% \mode<all>{
%%   \input{../methodology/body}
%%   }

\clearpage
\mode<all>{
\part{What is missing?}
}
\begin{frame}
\partpage
\end{frame}

\begin{frame}
\frametitle{Many Specialized Topics}
\begin{itemize}
\item How to design efficient core engine
\item Hybrids with LP/MIP tools
\item Hybrids with SAT
\item Symmetry breaking
\item Use of MDD/BDD to encode sets of solutions
\item High level modelling tools
\item Debugging/visualization
\end{itemize}
\end{frame}

\begin{frame}
\frametitle{Reformulation}
\begin{itemize}
\item Just because the user has modelled it this way, it doesn't mean we have to solve it that way
\begin{itemize}
\item Replace some constraint(s) by other, equivalent constraints
\item Because we don't have that constraint in our system
\item For performance
\end{itemize}
\end{itemize}
\end{frame}

\begin{frame}
\frametitle{Learning}
\begin{itemize}
\item While solving the problem we can learn how to strengthen the model/search 
\begin{itemize}
\item Understand which constraints/method contribute to propagation and change schedule
\item Learn no-good constraints by explaining failure
\item Adapt search strategy based on search experience
\end{itemize}
\end{itemize}
\end{frame}

%% \begin{frame}
%% \frametitle{Refined Process}
%% \begin{center}
%% \begin{tikzpicture}
%% \node[shape=rectangle,fill=pantone127-4] (problem) at (3,7) {Problem};
%% \node (human) at (3,6) {Human};
%% \node[shape=rectangle,fill=pantone127-4] (model) at (3,5) {User Model};
%% \node (reformulation) at (3,4) {Reformulation};
%% \node[shape=rectangle,fill=pantone127-4] (implementation) at (3,3) {Implementation Model};
%% \node (solver) at (3,2) {Constraint Solver/Search};
%% \node[shape=rectangle,fill=pantone157-8] (solution) at (3,1) {Solution};
%% \draw[-] (problem) -- (human);
%% \draw[->] (human) -- (model);
%% \draw[-] (model) -- (reformulation);
%% \draw[->] (reformulation) -- (implementation);
%% \draw[-] (implementation) -- (solver);
%% \draw[->] (solver) -- (0,2) -- (0,3) -- (implementation);
%% \draw[->] (solver) -- (solution);
%% \end{tikzpicture}
%% \end{center}
%% \end{frame}

%\begin{frame}
%\frametitle{What is CP actually used for?}
%\url{http://hsimonis.wordpress.com}
%\includegraphics[width=10cm]{../introduction/images/blog}
%\end{frame}

%% \begin{frame}
%% \frametitle{Exercises}
%% \begin{itemize}
%% \item Install Constraint Solver of your Choice
%% \item MiniZinc (MiningZinc) will be used in later sessions
%% \item Run Sendmoremoney, Sudoku, Queens examples
%% \item Can you choose which propagator to use for alldifferent?
%% \item Can you use different search strategies?
%% \end{itemize}
%% \end{frame}

\begin{frame}
\frametitle{More Learning Resources}
\begin{itemize}
\item Survey of Methods, Resources, and Formats for Teaching Constraint Programming
\begin{itemize}
\item by Tejas Santanam, Helmut Simonis
\item \url{https://doi.org/10.48550/arXiv.2403.12717}
\item Based on survey of community for WTCP 2023
\item \url{https://hsimonis.github.io/WTCP2023/}
\end{itemize}
\end{itemize}
\end{frame}


}

\clearpage
\mode<all>{
\part{Customizing Search}
}
\begin{frame}
\partpage
\end{frame}
\mode<all>{
\begin{frame}
\frametitle{Objectives}
\begin{itemize}
\item Overview of Core Constraint Programming
\item Three Main Concepts
\begin{itemize}
\item Constraint Propagation
\item Global Constraints
\item Customizing Search
\end{itemize}
%\item Get Some Experience with MiniZinc
\item Topics will be treated in more detail in later parts of the school
\item Based on Examples, not Formal Description
\end{itemize}
\end{frame}

\begin{frame}
\frametitle{Outline}
\begin{itemize}
\item Why Constraint Programming?
\item Constraint Propagation
\item Global Constraints
\item Customizing Search
%\item Choices in Modelling
%\item Decomposition
%\item More Global Constraints
%\item Methodology
%\item What is missing?
\end{itemize}
\end{frame}

% \begin{frame}
  % \frametitle{Using MiniZinc IDE}
  % \begin{itemize}
  % \item Developed in the Australian NICTA project
  % \item Maintained by Monash University
  % \item Modelling tool with multiple back-end solvers
  % \item Available from \url{https://www.minizinc.org/}
  % \end{itemize}
% \end{frame}


% \begin{frame}
% \frametitle{Examples in ECLiPSe}
% \begin{itemize}
% \item Open sourced constraint programming language
% \item Development goes back to 1985
% \item ECRC, ICL, IC-Parc, PTL, Cisco
% \item \url{https://eclipseclp.org/}
% \item Specialities
% \begin{itemize}
% \item Develop new solvers for specific domains
% \item Integration with MIP
% \end{itemize}
% \item Not included in bundled MiniZinc IDE
% \item Specialized visualization tools used here
  % \begin{itemize}
  % \item CP-Viz, Simonis et al. 2010
    % \end{itemize}
% \end{itemize}
% \end{frame}

\begin{frame}
\frametitle{Tutorial Based on ECLiPSe ELearning Course}
\begin{itemize}
\item Self-study course in constraint programming
\item Supported by Cisco Systems and Silicon Valley Community Foundation
\item Multi-media format, video lectures, slides, handout etc
\item {\small \url{https://eclipseclp.org/ELearning/index.html}}
\end{itemize}
\end{frame}

\begin{frame}
\frametitle{Also Part of CRT-AI Constraint Week}
\begin{itemize}
\item Annual one week course on CP and Optimization in Ireland
\item Part of national training program for PhD students in AI
\item \url{https://www.crt-ai.ie/}
\end{itemize}
\end{frame}



\begin{frame}
\frametitle{Constraint Programming - in a nutshell}
\begin{itemize}
\item Declarative description of problems with
\begin{itemize}
\item {\em Variables} which range over (finite) sets of values
\item {\em Constraints} over subsets of variables which restrict possible value combinations
\item A {\em solution} is a value assignment which satisfies all constraints
\end{itemize}

\item Constraint propagation/reasoning
\begin{itemize}
\item Removing inconsistent values for variables
\item Detect failure if constraint can not be satisfied
\item Interaction of constraints via shared variables
\item Incomplete
\end{itemize}

\item Search
\begin{itemize}
\item User controlled assignment of values to variables
\item Each step triggers constraint propagation 
\end{itemize}
\item Different domains require/allow different methods
\end{itemize}
\end{frame}

\begin{frame}
  \frametitle{Constraint Programming is Different}
  \begin{itemize}
  \item Declarative Programming
    \begin{itemize}
    \item Concentrate on what you want
      \item Not how to get there
      \item Program != Algorithm
      \item Program = Model
    \end{itemize}
    \item Applied to Combinatorial Problems
      \begin{itemize}
        \item No complete polynomial algorithms known (exist?)
        \item CP less ad-hoc than heuristics
        \item Models can evolve
  \end{itemize}
  \end{itemize}
  \end{frame}
    
\begin{frame}
  \frametitle{A Subtractive Process}
  \begin{textblock}{4}(8,-3)
    \includegraphics[width=4cm]{../introduction/images/stages}
  \end{textblock}
  \vfill
  \begin{quote}
    ``Oh, bosh, as Mr. Ruskin says. Sculpture, per se, is the simplest thing in the world. All you have to do is to take a big chunk of marble and a hammer and chisel, make up your mind what you are about to create and chip off all the marble you don't want.''-Paris Gaulois.
  \end{quote}
  
  {\tiny Source: \url{https://quoteinvestigator.com/2014/06/22/chip-away/}}
\end{frame}

\begin{frame}
\frametitle{Basic Process}
\begin{center}
\begin{tikzpicture}
\node[shape=rectangle,fill=pantone127-4] (problem) at (2,5) {Problem};
\node (human) at (2,4) {Human};
\node[shape=rectangle,fill=pantone127-4] (model) at (2,3) {Model};
\node (solver) at (2,2) {Constraint Solver/Search};
\node[shape=rectangle,fill=pantone157-8] (solution) at (2,1) {Solution};
\draw[-] (problem) -- (human);
\draw[->] (human) -- (model);
\draw[-] (model) -- (solver);
\draw[->] (solver) -- (solution);
\end{tikzpicture}
\end{center}
\end{frame}

\begin{frame}
\frametitle{More Realistic}
\begin{center}
\begin{tikzpicture}[xscale=4.0]
\node[shape=rectangle,fill=pantone127-4] (problem) at (2,5) {Problem};
\node (human) at (2,4) {Human};
\node[shape=rectangle,fill=pantone127-4] (model) at (2,3) {Model};
\node (solver) at (2,2) {Constraint Solver/Search};
\node[shape=rectangle,fill=pantone24-8] (hangs) at (1,1) {Hangs};
\node[shape=rectangle,fill=pantone157-8] (solution) at (2,1) {Solution};
\node[shape=rectangle,fill=pantone24-8] (wrong) at (3,1) {Wrong Solution};
\draw[-] (problem) -- (human);
\draw[->] (human) -- (model);
\draw[-] (model) -- (solver);
\draw[->] (solver) -- (hangs);
\draw[->] (solver) -- (solution);
\draw[->] (solver) -- (wrong);
\draw[->] (hangs) -- (1,4) -- (human);
\draw[->] (wrong) -- (3,4) -- (human);
\end{tikzpicture}
\end{center}
\end{frame}

\begin{frame}
\frametitle{Dual Role of Model}
\begin{itemize}
\item Allows Human to Express Problem
\begin{itemize}
\item Close to Problem Domain
\item Constraints as Abstractions
\end{itemize}
\item Allows Solver to Execute
\begin{itemize}
\item Variables as Communication Mechanism
\item Constraints as Algorithms
\end{itemize}
\end{itemize}
\end{frame}

\begin{frame}
\frametitle{Modelling Frameworks}
\begin{itemize}
\item \href{https://www.minizinc.org/}{MiniZinc} (NICTA, Monash University, Australia)
\item \href{https://github.com/eomahony/Numberjack}{NumberJack} (Insight, Ireland) 
\item \href{https://www-users.york.ac.uk/peter.nightingale/savilerow//}{EssencePrime/SavilleRow} (UK) 
\item \href{https://cpmpy.readthedocs.io/en/latest/}{CPMpy} (KU Leuven)
\item Allow use of multiple back-end solvers
\item Compile model into variants for each solver
\item A priori solver independent model(CP, MIP, SAT)
\end{itemize}
\end{frame} 

\begin{frame}
\frametitle{Framework Process}
\begin{center}
\begin{tikzpicture}[xscale=2]
\node[shape=rectangle,fill=pantone127-4] (problem) at (2.5,5) {Problem};
\node (human) at (2.5,4) {Human};
\node[shape=rectangle,fill=pantone127-4] (model) at (2.5,3) {Model};
\node (compile) at (2.5,2) {Compile/Reformulate};
\node (solver1) at (1,1) {CP};
\node (solver2) at (2,1) {MIP};
\node (solver3) at (3,1) {SAT};
\node (solver4) at (4,1) {Other};
\node[shape=rectangle,fill=pantone157-8] (solution1) at (1,0) {Solution};
\node[shape=rectangle,fill=pantone157-8] (solution2) at (2,0) {Solution};
\node[shape=rectangle,fill=pantone157-8] (solution3) at (3,0) {Solution};
\node[shape=rectangle,fill=pantone157-8] (solution4) at (4,0) {Solution};
\draw[-] (problem) -- (human);
\draw[->] (human) -- (model);
\draw[-] (model) -- (compile);
\draw[-] (compile) -- (solver1);
\draw[-] (compile) -- (solver2);
\draw[-] (compile) -- (solver3);
\draw[-] (compile) -- (solver4);
\draw[->] (solver1) -- (solution1);
\draw[->] (solver2) -- (solution2);
\draw[->] (solver3) -- (solution3);
\draw[->] (solver4) -- (solution4);
\end{tikzpicture}
\end{center}
\end{frame}

% \begin{frame}
  % \frametitle{Do It Now!}
  % \begin{itemize}
  % \item Download and install Minizinc
  % \item \url{https://www.minizinc.org/}
  % \end{itemize}
% \end{frame}


\begin{frame}
\frametitle{Why use Puzzles as Examples?}
\begin{itemize}
\item Easy to understand the problem
\item Solvable by hand without specialized knowledge
\item Possible to compare automated to manual solving process
\end{itemize}

\begin{quote}
The puzzle, though inanimate, is presented as a solvable problem without lasting negative consequences, a very low-risk low-reward situation. By being a puzzle, the object is attempting to convince the user that it must be completed.
\end{quote}
{\tiny Source: \href{https://www.everydayrhetoric.com/2020/03/the-rhetoric-of-puzzles/}{Every Day Rhetoric}}
\end{frame}


\mode<all>{
\part{Basic Constraint Propagation}
}
\begin{frame}
\partpage
\end{frame}
\mode<all>{
\input{../sendmore/body}
}

\mode<all>{
\part{Global Constraints}
}
\begin{frame}
\partpage
\end{frame}
\mode<all>{
\input{../sudoku/body}
}

\clearpage
\mode<all>{
\part{Customizing Search}
}
\begin{frame}
\partpage
\end{frame}
\mode<all>{
\input{../nqueen/body}
}


%% \clearpage
%% \mode<all>{
%%   \part{Problem Decomposition}
%% }
%% \begin{frame}
%%   \partpage
%% \end{frame}
%% \mode<all>{
%%   \input{../party/body}
%%   }

%% \clearpage
%% \mode<all>{
%%   \part{More Global Constraints}
%% }
%% \begin{frame}
%%   \partpage
%% \end{frame}
%% \mode<all>{
%%   \input{../car/body}
%%   }

%% \clearpage
%% \mode<all>{
%%   \part{Methodology}
%% }
%% \begin{frame}
%%   \partpage
%% \end{frame}
%% \mode<all>{
%%   \input{../methodology/body}
%%   }

\clearpage
\mode<all>{
\part{What is missing?}
}
\begin{frame}
\partpage
\end{frame}

\begin{frame}
\frametitle{Many Specialized Topics}
\begin{itemize}
\item How to design efficient core engine
\item Hybrids with LP/MIP tools
\item Hybrids with SAT
\item Symmetry breaking
\item Use of MDD/BDD to encode sets of solutions
\item High level modelling tools
\item Debugging/visualization
\end{itemize}
\end{frame}

\begin{frame}
\frametitle{Reformulation}
\begin{itemize}
\item Just because the user has modelled it this way, it doesn't mean we have to solve it that way
\begin{itemize}
\item Replace some constraint(s) by other, equivalent constraints
\item Because we don't have that constraint in our system
\item For performance
\end{itemize}
\end{itemize}
\end{frame}

\begin{frame}
\frametitle{Learning}
\begin{itemize}
\item While solving the problem we can learn how to strengthen the model/search 
\begin{itemize}
\item Understand which constraints/method contribute to propagation and change schedule
\item Learn no-good constraints by explaining failure
\item Adapt search strategy based on search experience
\end{itemize}
\end{itemize}
\end{frame}

%% \begin{frame}
%% \frametitle{Refined Process}
%% \begin{center}
%% \begin{tikzpicture}
%% \node[shape=rectangle,fill=pantone127-4] (problem) at (3,7) {Problem};
%% \node (human) at (3,6) {Human};
%% \node[shape=rectangle,fill=pantone127-4] (model) at (3,5) {User Model};
%% \node (reformulation) at (3,4) {Reformulation};
%% \node[shape=rectangle,fill=pantone127-4] (implementation) at (3,3) {Implementation Model};
%% \node (solver) at (3,2) {Constraint Solver/Search};
%% \node[shape=rectangle,fill=pantone157-8] (solution) at (3,1) {Solution};
%% \draw[-] (problem) -- (human);
%% \draw[->] (human) -- (model);
%% \draw[-] (model) -- (reformulation);
%% \draw[->] (reformulation) -- (implementation);
%% \draw[-] (implementation) -- (solver);
%% \draw[->] (solver) -- (0,2) -- (0,3) -- (implementation);
%% \draw[->] (solver) -- (solution);
%% \end{tikzpicture}
%% \end{center}
%% \end{frame}

%\begin{frame}
%\frametitle{What is CP actually used for?}
%\url{http://hsimonis.wordpress.com}
%\includegraphics[width=10cm]{../introduction/images/blog}
%\end{frame}

%% \begin{frame}
%% \frametitle{Exercises}
%% \begin{itemize}
%% \item Install Constraint Solver of your Choice
%% \item MiniZinc (MiningZinc) will be used in later sessions
%% \item Run Sendmoremoney, Sudoku, Queens examples
%% \item Can you choose which propagator to use for alldifferent?
%% \item Can you use different search strategies?
%% \end{itemize}
%% \end{frame}

\begin{frame}
\frametitle{More Learning Resources}
\begin{itemize}
\item Survey of Methods, Resources, and Formats for Teaching Constraint Programming
\begin{itemize}
\item by Tejas Santanam, Helmut Simonis
\item \url{https://doi.org/10.48550/arXiv.2403.12717}
\item Based on survey of community for WTCP 2023
\item \url{https://hsimonis.github.io/WTCP2023/}
\end{itemize}
\end{itemize}
\end{frame}


}


%% \clearpage
%% \mode<all>{
%%   \part{Problem Decomposition}
%% }
%% \begin{frame}
%%   \partpage
%% \end{frame}
%% \mode<all>{
%%   \begin{frame}
\frametitle{Objectives}
\begin{itemize}
\item Overview of Core Constraint Programming
\item Three Main Concepts
\begin{itemize}
\item Constraint Propagation
\item Global Constraints
\item Customizing Search
\end{itemize}
%\item Get Some Experience with MiniZinc
\item Topics will be treated in more detail in later parts of the school
\item Based on Examples, not Formal Description
\end{itemize}
\end{frame}

\begin{frame}
\frametitle{Outline}
\begin{itemize}
\item Why Constraint Programming?
\item Constraint Propagation
\item Global Constraints
\item Customizing Search
%\item Choices in Modelling
%\item Decomposition
%\item More Global Constraints
%\item Methodology
%\item What is missing?
\end{itemize}
\end{frame}

% \begin{frame}
  % \frametitle{Using MiniZinc IDE}
  % \begin{itemize}
  % \item Developed in the Australian NICTA project
  % \item Maintained by Monash University
  % \item Modelling tool with multiple back-end solvers
  % \item Available from \url{https://www.minizinc.org/}
  % \end{itemize}
% \end{frame}


% \begin{frame}
% \frametitle{Examples in ECLiPSe}
% \begin{itemize}
% \item Open sourced constraint programming language
% \item Development goes back to 1985
% \item ECRC, ICL, IC-Parc, PTL, Cisco
% \item \url{https://eclipseclp.org/}
% \item Specialities
% \begin{itemize}
% \item Develop new solvers for specific domains
% \item Integration with MIP
% \end{itemize}
% \item Not included in bundled MiniZinc IDE
% \item Specialized visualization tools used here
  % \begin{itemize}
  % \item CP-Viz, Simonis et al. 2010
    % \end{itemize}
% \end{itemize}
% \end{frame}

\begin{frame}
\frametitle{Tutorial Based on ECLiPSe ELearning Course}
\begin{itemize}
\item Self-study course in constraint programming
\item Supported by Cisco Systems and Silicon Valley Community Foundation
\item Multi-media format, video lectures, slides, handout etc
\item {\small \url{https://eclipseclp.org/ELearning/index.html}}
\end{itemize}
\end{frame}

\begin{frame}
\frametitle{Also Part of CRT-AI Constraint Week}
\begin{itemize}
\item Annual one week course on CP and Optimization in Ireland
\item Part of national training program for PhD students in AI
\item \url{https://www.crt-ai.ie/}
\end{itemize}
\end{frame}



\begin{frame}
\frametitle{Constraint Programming - in a nutshell}
\begin{itemize}
\item Declarative description of problems with
\begin{itemize}
\item {\em Variables} which range over (finite) sets of values
\item {\em Constraints} over subsets of variables which restrict possible value combinations
\item A {\em solution} is a value assignment which satisfies all constraints
\end{itemize}

\item Constraint propagation/reasoning
\begin{itemize}
\item Removing inconsistent values for variables
\item Detect failure if constraint can not be satisfied
\item Interaction of constraints via shared variables
\item Incomplete
\end{itemize}

\item Search
\begin{itemize}
\item User controlled assignment of values to variables
\item Each step triggers constraint propagation 
\end{itemize}
\item Different domains require/allow different methods
\end{itemize}
\end{frame}

\begin{frame}
  \frametitle{Constraint Programming is Different}
  \begin{itemize}
  \item Declarative Programming
    \begin{itemize}
    \item Concentrate on what you want
      \item Not how to get there
      \item Program != Algorithm
      \item Program = Model
    \end{itemize}
    \item Applied to Combinatorial Problems
      \begin{itemize}
        \item No complete polynomial algorithms known (exist?)
        \item CP less ad-hoc than heuristics
        \item Models can evolve
  \end{itemize}
  \end{itemize}
  \end{frame}
    
\begin{frame}
  \frametitle{A Subtractive Process}
  \begin{textblock}{4}(8,-3)
    \includegraphics[width=4cm]{../introduction/images/stages}
  \end{textblock}
  \vfill
  \begin{quote}
    ``Oh, bosh, as Mr. Ruskin says. Sculpture, per se, is the simplest thing in the world. All you have to do is to take a big chunk of marble and a hammer and chisel, make up your mind what you are about to create and chip off all the marble you don't want.''-Paris Gaulois.
  \end{quote}
  
  {\tiny Source: \url{https://quoteinvestigator.com/2014/06/22/chip-away/}}
\end{frame}

\begin{frame}
\frametitle{Basic Process}
\begin{center}
\begin{tikzpicture}
\node[shape=rectangle,fill=pantone127-4] (problem) at (2,5) {Problem};
\node (human) at (2,4) {Human};
\node[shape=rectangle,fill=pantone127-4] (model) at (2,3) {Model};
\node (solver) at (2,2) {Constraint Solver/Search};
\node[shape=rectangle,fill=pantone157-8] (solution) at (2,1) {Solution};
\draw[-] (problem) -- (human);
\draw[->] (human) -- (model);
\draw[-] (model) -- (solver);
\draw[->] (solver) -- (solution);
\end{tikzpicture}
\end{center}
\end{frame}

\begin{frame}
\frametitle{More Realistic}
\begin{center}
\begin{tikzpicture}[xscale=4.0]
\node[shape=rectangle,fill=pantone127-4] (problem) at (2,5) {Problem};
\node (human) at (2,4) {Human};
\node[shape=rectangle,fill=pantone127-4] (model) at (2,3) {Model};
\node (solver) at (2,2) {Constraint Solver/Search};
\node[shape=rectangle,fill=pantone24-8] (hangs) at (1,1) {Hangs};
\node[shape=rectangle,fill=pantone157-8] (solution) at (2,1) {Solution};
\node[shape=rectangle,fill=pantone24-8] (wrong) at (3,1) {Wrong Solution};
\draw[-] (problem) -- (human);
\draw[->] (human) -- (model);
\draw[-] (model) -- (solver);
\draw[->] (solver) -- (hangs);
\draw[->] (solver) -- (solution);
\draw[->] (solver) -- (wrong);
\draw[->] (hangs) -- (1,4) -- (human);
\draw[->] (wrong) -- (3,4) -- (human);
\end{tikzpicture}
\end{center}
\end{frame}

\begin{frame}
\frametitle{Dual Role of Model}
\begin{itemize}
\item Allows Human to Express Problem
\begin{itemize}
\item Close to Problem Domain
\item Constraints as Abstractions
\end{itemize}
\item Allows Solver to Execute
\begin{itemize}
\item Variables as Communication Mechanism
\item Constraints as Algorithms
\end{itemize}
\end{itemize}
\end{frame}

\begin{frame}
\frametitle{Modelling Frameworks}
\begin{itemize}
\item \href{https://www.minizinc.org/}{MiniZinc} (NICTA, Monash University, Australia)
\item \href{https://github.com/eomahony/Numberjack}{NumberJack} (Insight, Ireland) 
\item \href{https://www-users.york.ac.uk/peter.nightingale/savilerow//}{EssencePrime/SavilleRow} (UK) 
\item \href{https://cpmpy.readthedocs.io/en/latest/}{CPMpy} (KU Leuven)
\item Allow use of multiple back-end solvers
\item Compile model into variants for each solver
\item A priori solver independent model(CP, MIP, SAT)
\end{itemize}
\end{frame} 

\begin{frame}
\frametitle{Framework Process}
\begin{center}
\begin{tikzpicture}[xscale=2]
\node[shape=rectangle,fill=pantone127-4] (problem) at (2.5,5) {Problem};
\node (human) at (2.5,4) {Human};
\node[shape=rectangle,fill=pantone127-4] (model) at (2.5,3) {Model};
\node (compile) at (2.5,2) {Compile/Reformulate};
\node (solver1) at (1,1) {CP};
\node (solver2) at (2,1) {MIP};
\node (solver3) at (3,1) {SAT};
\node (solver4) at (4,1) {Other};
\node[shape=rectangle,fill=pantone157-8] (solution1) at (1,0) {Solution};
\node[shape=rectangle,fill=pantone157-8] (solution2) at (2,0) {Solution};
\node[shape=rectangle,fill=pantone157-8] (solution3) at (3,0) {Solution};
\node[shape=rectangle,fill=pantone157-8] (solution4) at (4,0) {Solution};
\draw[-] (problem) -- (human);
\draw[->] (human) -- (model);
\draw[-] (model) -- (compile);
\draw[-] (compile) -- (solver1);
\draw[-] (compile) -- (solver2);
\draw[-] (compile) -- (solver3);
\draw[-] (compile) -- (solver4);
\draw[->] (solver1) -- (solution1);
\draw[->] (solver2) -- (solution2);
\draw[->] (solver3) -- (solution3);
\draw[->] (solver4) -- (solution4);
\end{tikzpicture}
\end{center}
\end{frame}

% \begin{frame}
  % \frametitle{Do It Now!}
  % \begin{itemize}
  % \item Download and install Minizinc
  % \item \url{https://www.minizinc.org/}
  % \end{itemize}
% \end{frame}


\begin{frame}
\frametitle{Why use Puzzles as Examples?}
\begin{itemize}
\item Easy to understand the problem
\item Solvable by hand without specialized knowledge
\item Possible to compare automated to manual solving process
\end{itemize}

\begin{quote}
The puzzle, though inanimate, is presented as a solvable problem without lasting negative consequences, a very low-risk low-reward situation. By being a puzzle, the object is attempting to convince the user that it must be completed.
\end{quote}
{\tiny Source: \href{https://www.everydayrhetoric.com/2020/03/the-rhetoric-of-puzzles/}{Every Day Rhetoric}}
\end{frame}


\mode<all>{
\part{Basic Constraint Propagation}
}
\begin{frame}
\partpage
\end{frame}
\mode<all>{
\input{../sendmore/body}
}

\mode<all>{
\part{Global Constraints}
}
\begin{frame}
\partpage
\end{frame}
\mode<all>{
\input{../sudoku/body}
}

\clearpage
\mode<all>{
\part{Customizing Search}
}
\begin{frame}
\partpage
\end{frame}
\mode<all>{
\input{../nqueen/body}
}


%% \clearpage
%% \mode<all>{
%%   \part{Problem Decomposition}
%% }
%% \begin{frame}
%%   \partpage
%% \end{frame}
%% \mode<all>{
%%   \input{../party/body}
%%   }

%% \clearpage
%% \mode<all>{
%%   \part{More Global Constraints}
%% }
%% \begin{frame}
%%   \partpage
%% \end{frame}
%% \mode<all>{
%%   \input{../car/body}
%%   }

%% \clearpage
%% \mode<all>{
%%   \part{Methodology}
%% }
%% \begin{frame}
%%   \partpage
%% \end{frame}
%% \mode<all>{
%%   \input{../methodology/body}
%%   }

\clearpage
\mode<all>{
\part{What is missing?}
}
\begin{frame}
\partpage
\end{frame}

\begin{frame}
\frametitle{Many Specialized Topics}
\begin{itemize}
\item How to design efficient core engine
\item Hybrids with LP/MIP tools
\item Hybrids with SAT
\item Symmetry breaking
\item Use of MDD/BDD to encode sets of solutions
\item High level modelling tools
\item Debugging/visualization
\end{itemize}
\end{frame}

\begin{frame}
\frametitle{Reformulation}
\begin{itemize}
\item Just because the user has modelled it this way, it doesn't mean we have to solve it that way
\begin{itemize}
\item Replace some constraint(s) by other, equivalent constraints
\item Because we don't have that constraint in our system
\item For performance
\end{itemize}
\end{itemize}
\end{frame}

\begin{frame}
\frametitle{Learning}
\begin{itemize}
\item While solving the problem we can learn how to strengthen the model/search 
\begin{itemize}
\item Understand which constraints/method contribute to propagation and change schedule
\item Learn no-good constraints by explaining failure
\item Adapt search strategy based on search experience
\end{itemize}
\end{itemize}
\end{frame}

%% \begin{frame}
%% \frametitle{Refined Process}
%% \begin{center}
%% \begin{tikzpicture}
%% \node[shape=rectangle,fill=pantone127-4] (problem) at (3,7) {Problem};
%% \node (human) at (3,6) {Human};
%% \node[shape=rectangle,fill=pantone127-4] (model) at (3,5) {User Model};
%% \node (reformulation) at (3,4) {Reformulation};
%% \node[shape=rectangle,fill=pantone127-4] (implementation) at (3,3) {Implementation Model};
%% \node (solver) at (3,2) {Constraint Solver/Search};
%% \node[shape=rectangle,fill=pantone157-8] (solution) at (3,1) {Solution};
%% \draw[-] (problem) -- (human);
%% \draw[->] (human) -- (model);
%% \draw[-] (model) -- (reformulation);
%% \draw[->] (reformulation) -- (implementation);
%% \draw[-] (implementation) -- (solver);
%% \draw[->] (solver) -- (0,2) -- (0,3) -- (implementation);
%% \draw[->] (solver) -- (solution);
%% \end{tikzpicture}
%% \end{center}
%% \end{frame}

%\begin{frame}
%\frametitle{What is CP actually used for?}
%\url{http://hsimonis.wordpress.com}
%\includegraphics[width=10cm]{../introduction/images/blog}
%\end{frame}

%% \begin{frame}
%% \frametitle{Exercises}
%% \begin{itemize}
%% \item Install Constraint Solver of your Choice
%% \item MiniZinc (MiningZinc) will be used in later sessions
%% \item Run Sendmoremoney, Sudoku, Queens examples
%% \item Can you choose which propagator to use for alldifferent?
%% \item Can you use different search strategies?
%% \end{itemize}
%% \end{frame}

\begin{frame}
\frametitle{More Learning Resources}
\begin{itemize}
\item Survey of Methods, Resources, and Formats for Teaching Constraint Programming
\begin{itemize}
\item by Tejas Santanam, Helmut Simonis
\item \url{https://doi.org/10.48550/arXiv.2403.12717}
\item Based on survey of community for WTCP 2023
\item \url{https://hsimonis.github.io/WTCP2023/}
\end{itemize}
\end{itemize}
\end{frame}


%%   }

%% \clearpage
%% \mode<all>{
%%   \part{More Global Constraints}
%% }
%% \begin{frame}
%%   \partpage
%% \end{frame}
%% \mode<all>{
%%   \begin{frame}
\frametitle{Objectives}
\begin{itemize}
\item Overview of Core Constraint Programming
\item Three Main Concepts
\begin{itemize}
\item Constraint Propagation
\item Global Constraints
\item Customizing Search
\end{itemize}
%\item Get Some Experience with MiniZinc
\item Topics will be treated in more detail in later parts of the school
\item Based on Examples, not Formal Description
\end{itemize}
\end{frame}

\begin{frame}
\frametitle{Outline}
\begin{itemize}
\item Why Constraint Programming?
\item Constraint Propagation
\item Global Constraints
\item Customizing Search
%\item Choices in Modelling
%\item Decomposition
%\item More Global Constraints
%\item Methodology
%\item What is missing?
\end{itemize}
\end{frame}

% \begin{frame}
  % \frametitle{Using MiniZinc IDE}
  % \begin{itemize}
  % \item Developed in the Australian NICTA project
  % \item Maintained by Monash University
  % \item Modelling tool with multiple back-end solvers
  % \item Available from \url{https://www.minizinc.org/}
  % \end{itemize}
% \end{frame}


% \begin{frame}
% \frametitle{Examples in ECLiPSe}
% \begin{itemize}
% \item Open sourced constraint programming language
% \item Development goes back to 1985
% \item ECRC, ICL, IC-Parc, PTL, Cisco
% \item \url{https://eclipseclp.org/}
% \item Specialities
% \begin{itemize}
% \item Develop new solvers for specific domains
% \item Integration with MIP
% \end{itemize}
% \item Not included in bundled MiniZinc IDE
% \item Specialized visualization tools used here
  % \begin{itemize}
  % \item CP-Viz, Simonis et al. 2010
    % \end{itemize}
% \end{itemize}
% \end{frame}

\begin{frame}
\frametitle{Tutorial Based on ECLiPSe ELearning Course}
\begin{itemize}
\item Self-study course in constraint programming
\item Supported by Cisco Systems and Silicon Valley Community Foundation
\item Multi-media format, video lectures, slides, handout etc
\item {\small \url{https://eclipseclp.org/ELearning/index.html}}
\end{itemize}
\end{frame}

\begin{frame}
\frametitle{Also Part of CRT-AI Constraint Week}
\begin{itemize}
\item Annual one week course on CP and Optimization in Ireland
\item Part of national training program for PhD students in AI
\item \url{https://www.crt-ai.ie/}
\end{itemize}
\end{frame}



\begin{frame}
\frametitle{Constraint Programming - in a nutshell}
\begin{itemize}
\item Declarative description of problems with
\begin{itemize}
\item {\em Variables} which range over (finite) sets of values
\item {\em Constraints} over subsets of variables which restrict possible value combinations
\item A {\em solution} is a value assignment which satisfies all constraints
\end{itemize}

\item Constraint propagation/reasoning
\begin{itemize}
\item Removing inconsistent values for variables
\item Detect failure if constraint can not be satisfied
\item Interaction of constraints via shared variables
\item Incomplete
\end{itemize}

\item Search
\begin{itemize}
\item User controlled assignment of values to variables
\item Each step triggers constraint propagation 
\end{itemize}
\item Different domains require/allow different methods
\end{itemize}
\end{frame}

\begin{frame}
  \frametitle{Constraint Programming is Different}
  \begin{itemize}
  \item Declarative Programming
    \begin{itemize}
    \item Concentrate on what you want
      \item Not how to get there
      \item Program != Algorithm
      \item Program = Model
    \end{itemize}
    \item Applied to Combinatorial Problems
      \begin{itemize}
        \item No complete polynomial algorithms known (exist?)
        \item CP less ad-hoc than heuristics
        \item Models can evolve
  \end{itemize}
  \end{itemize}
  \end{frame}
    
\begin{frame}
  \frametitle{A Subtractive Process}
  \begin{textblock}{4}(8,-3)
    \includegraphics[width=4cm]{../introduction/images/stages}
  \end{textblock}
  \vfill
  \begin{quote}
    ``Oh, bosh, as Mr. Ruskin says. Sculpture, per se, is the simplest thing in the world. All you have to do is to take a big chunk of marble and a hammer and chisel, make up your mind what you are about to create and chip off all the marble you don't want.''-Paris Gaulois.
  \end{quote}
  
  {\tiny Source: \url{https://quoteinvestigator.com/2014/06/22/chip-away/}}
\end{frame}

\begin{frame}
\frametitle{Basic Process}
\begin{center}
\begin{tikzpicture}
\node[shape=rectangle,fill=pantone127-4] (problem) at (2,5) {Problem};
\node (human) at (2,4) {Human};
\node[shape=rectangle,fill=pantone127-4] (model) at (2,3) {Model};
\node (solver) at (2,2) {Constraint Solver/Search};
\node[shape=rectangle,fill=pantone157-8] (solution) at (2,1) {Solution};
\draw[-] (problem) -- (human);
\draw[->] (human) -- (model);
\draw[-] (model) -- (solver);
\draw[->] (solver) -- (solution);
\end{tikzpicture}
\end{center}
\end{frame}

\begin{frame}
\frametitle{More Realistic}
\begin{center}
\begin{tikzpicture}[xscale=4.0]
\node[shape=rectangle,fill=pantone127-4] (problem) at (2,5) {Problem};
\node (human) at (2,4) {Human};
\node[shape=rectangle,fill=pantone127-4] (model) at (2,3) {Model};
\node (solver) at (2,2) {Constraint Solver/Search};
\node[shape=rectangle,fill=pantone24-8] (hangs) at (1,1) {Hangs};
\node[shape=rectangle,fill=pantone157-8] (solution) at (2,1) {Solution};
\node[shape=rectangle,fill=pantone24-8] (wrong) at (3,1) {Wrong Solution};
\draw[-] (problem) -- (human);
\draw[->] (human) -- (model);
\draw[-] (model) -- (solver);
\draw[->] (solver) -- (hangs);
\draw[->] (solver) -- (solution);
\draw[->] (solver) -- (wrong);
\draw[->] (hangs) -- (1,4) -- (human);
\draw[->] (wrong) -- (3,4) -- (human);
\end{tikzpicture}
\end{center}
\end{frame}

\begin{frame}
\frametitle{Dual Role of Model}
\begin{itemize}
\item Allows Human to Express Problem
\begin{itemize}
\item Close to Problem Domain
\item Constraints as Abstractions
\end{itemize}
\item Allows Solver to Execute
\begin{itemize}
\item Variables as Communication Mechanism
\item Constraints as Algorithms
\end{itemize}
\end{itemize}
\end{frame}

\begin{frame}
\frametitle{Modelling Frameworks}
\begin{itemize}
\item \href{https://www.minizinc.org/}{MiniZinc} (NICTA, Monash University, Australia)
\item \href{https://github.com/eomahony/Numberjack}{NumberJack} (Insight, Ireland) 
\item \href{https://www-users.york.ac.uk/peter.nightingale/savilerow//}{EssencePrime/SavilleRow} (UK) 
\item \href{https://cpmpy.readthedocs.io/en/latest/}{CPMpy} (KU Leuven)
\item Allow use of multiple back-end solvers
\item Compile model into variants for each solver
\item A priori solver independent model(CP, MIP, SAT)
\end{itemize}
\end{frame} 

\begin{frame}
\frametitle{Framework Process}
\begin{center}
\begin{tikzpicture}[xscale=2]
\node[shape=rectangle,fill=pantone127-4] (problem) at (2.5,5) {Problem};
\node (human) at (2.5,4) {Human};
\node[shape=rectangle,fill=pantone127-4] (model) at (2.5,3) {Model};
\node (compile) at (2.5,2) {Compile/Reformulate};
\node (solver1) at (1,1) {CP};
\node (solver2) at (2,1) {MIP};
\node (solver3) at (3,1) {SAT};
\node (solver4) at (4,1) {Other};
\node[shape=rectangle,fill=pantone157-8] (solution1) at (1,0) {Solution};
\node[shape=rectangle,fill=pantone157-8] (solution2) at (2,0) {Solution};
\node[shape=rectangle,fill=pantone157-8] (solution3) at (3,0) {Solution};
\node[shape=rectangle,fill=pantone157-8] (solution4) at (4,0) {Solution};
\draw[-] (problem) -- (human);
\draw[->] (human) -- (model);
\draw[-] (model) -- (compile);
\draw[-] (compile) -- (solver1);
\draw[-] (compile) -- (solver2);
\draw[-] (compile) -- (solver3);
\draw[-] (compile) -- (solver4);
\draw[->] (solver1) -- (solution1);
\draw[->] (solver2) -- (solution2);
\draw[->] (solver3) -- (solution3);
\draw[->] (solver4) -- (solution4);
\end{tikzpicture}
\end{center}
\end{frame}

% \begin{frame}
  % \frametitle{Do It Now!}
  % \begin{itemize}
  % \item Download and install Minizinc
  % \item \url{https://www.minizinc.org/}
  % \end{itemize}
% \end{frame}


\begin{frame}
\frametitle{Why use Puzzles as Examples?}
\begin{itemize}
\item Easy to understand the problem
\item Solvable by hand without specialized knowledge
\item Possible to compare automated to manual solving process
\end{itemize}

\begin{quote}
The puzzle, though inanimate, is presented as a solvable problem without lasting negative consequences, a very low-risk low-reward situation. By being a puzzle, the object is attempting to convince the user that it must be completed.
\end{quote}
{\tiny Source: \href{https://www.everydayrhetoric.com/2020/03/the-rhetoric-of-puzzles/}{Every Day Rhetoric}}
\end{frame}


\mode<all>{
\part{Basic Constraint Propagation}
}
\begin{frame}
\partpage
\end{frame}
\mode<all>{
\input{../sendmore/body}
}

\mode<all>{
\part{Global Constraints}
}
\begin{frame}
\partpage
\end{frame}
\mode<all>{
\input{../sudoku/body}
}

\clearpage
\mode<all>{
\part{Customizing Search}
}
\begin{frame}
\partpage
\end{frame}
\mode<all>{
\input{../nqueen/body}
}


%% \clearpage
%% \mode<all>{
%%   \part{Problem Decomposition}
%% }
%% \begin{frame}
%%   \partpage
%% \end{frame}
%% \mode<all>{
%%   \input{../party/body}
%%   }

%% \clearpage
%% \mode<all>{
%%   \part{More Global Constraints}
%% }
%% \begin{frame}
%%   \partpage
%% \end{frame}
%% \mode<all>{
%%   \input{../car/body}
%%   }

%% \clearpage
%% \mode<all>{
%%   \part{Methodology}
%% }
%% \begin{frame}
%%   \partpage
%% \end{frame}
%% \mode<all>{
%%   \input{../methodology/body}
%%   }

\clearpage
\mode<all>{
\part{What is missing?}
}
\begin{frame}
\partpage
\end{frame}

\begin{frame}
\frametitle{Many Specialized Topics}
\begin{itemize}
\item How to design efficient core engine
\item Hybrids with LP/MIP tools
\item Hybrids with SAT
\item Symmetry breaking
\item Use of MDD/BDD to encode sets of solutions
\item High level modelling tools
\item Debugging/visualization
\end{itemize}
\end{frame}

\begin{frame}
\frametitle{Reformulation}
\begin{itemize}
\item Just because the user has modelled it this way, it doesn't mean we have to solve it that way
\begin{itemize}
\item Replace some constraint(s) by other, equivalent constraints
\item Because we don't have that constraint in our system
\item For performance
\end{itemize}
\end{itemize}
\end{frame}

\begin{frame}
\frametitle{Learning}
\begin{itemize}
\item While solving the problem we can learn how to strengthen the model/search 
\begin{itemize}
\item Understand which constraints/method contribute to propagation and change schedule
\item Learn no-good constraints by explaining failure
\item Adapt search strategy based on search experience
\end{itemize}
\end{itemize}
\end{frame}

%% \begin{frame}
%% \frametitle{Refined Process}
%% \begin{center}
%% \begin{tikzpicture}
%% \node[shape=rectangle,fill=pantone127-4] (problem) at (3,7) {Problem};
%% \node (human) at (3,6) {Human};
%% \node[shape=rectangle,fill=pantone127-4] (model) at (3,5) {User Model};
%% \node (reformulation) at (3,4) {Reformulation};
%% \node[shape=rectangle,fill=pantone127-4] (implementation) at (3,3) {Implementation Model};
%% \node (solver) at (3,2) {Constraint Solver/Search};
%% \node[shape=rectangle,fill=pantone157-8] (solution) at (3,1) {Solution};
%% \draw[-] (problem) -- (human);
%% \draw[->] (human) -- (model);
%% \draw[-] (model) -- (reformulation);
%% \draw[->] (reformulation) -- (implementation);
%% \draw[-] (implementation) -- (solver);
%% \draw[->] (solver) -- (0,2) -- (0,3) -- (implementation);
%% \draw[->] (solver) -- (solution);
%% \end{tikzpicture}
%% \end{center}
%% \end{frame}

%\begin{frame}
%\frametitle{What is CP actually used for?}
%\url{http://hsimonis.wordpress.com}
%\includegraphics[width=10cm]{../introduction/images/blog}
%\end{frame}

%% \begin{frame}
%% \frametitle{Exercises}
%% \begin{itemize}
%% \item Install Constraint Solver of your Choice
%% \item MiniZinc (MiningZinc) will be used in later sessions
%% \item Run Sendmoremoney, Sudoku, Queens examples
%% \item Can you choose which propagator to use for alldifferent?
%% \item Can you use different search strategies?
%% \end{itemize}
%% \end{frame}

\begin{frame}
\frametitle{More Learning Resources}
\begin{itemize}
\item Survey of Methods, Resources, and Formats for Teaching Constraint Programming
\begin{itemize}
\item by Tejas Santanam, Helmut Simonis
\item \url{https://doi.org/10.48550/arXiv.2403.12717}
\item Based on survey of community for WTCP 2023
\item \url{https://hsimonis.github.io/WTCP2023/}
\end{itemize}
\end{itemize}
\end{frame}


%%   }

%% \clearpage
%% \mode<all>{
%%   \part{Methodology}
%% }
%% \begin{frame}
%%   \partpage
%% \end{frame}
%% \mode<all>{
%%   \begin{frame}
\frametitle{Objectives}
\begin{itemize}
\item Overview of Core Constraint Programming
\item Three Main Concepts
\begin{itemize}
\item Constraint Propagation
\item Global Constraints
\item Customizing Search
\end{itemize}
%\item Get Some Experience with MiniZinc
\item Topics will be treated in more detail in later parts of the school
\item Based on Examples, not Formal Description
\end{itemize}
\end{frame}

\begin{frame}
\frametitle{Outline}
\begin{itemize}
\item Why Constraint Programming?
\item Constraint Propagation
\item Global Constraints
\item Customizing Search
%\item Choices in Modelling
%\item Decomposition
%\item More Global Constraints
%\item Methodology
%\item What is missing?
\end{itemize}
\end{frame}

% \begin{frame}
  % \frametitle{Using MiniZinc IDE}
  % \begin{itemize}
  % \item Developed in the Australian NICTA project
  % \item Maintained by Monash University
  % \item Modelling tool with multiple back-end solvers
  % \item Available from \url{https://www.minizinc.org/}
  % \end{itemize}
% \end{frame}


% \begin{frame}
% \frametitle{Examples in ECLiPSe}
% \begin{itemize}
% \item Open sourced constraint programming language
% \item Development goes back to 1985
% \item ECRC, ICL, IC-Parc, PTL, Cisco
% \item \url{https://eclipseclp.org/}
% \item Specialities
% \begin{itemize}
% \item Develop new solvers for specific domains
% \item Integration with MIP
% \end{itemize}
% \item Not included in bundled MiniZinc IDE
% \item Specialized visualization tools used here
  % \begin{itemize}
  % \item CP-Viz, Simonis et al. 2010
    % \end{itemize}
% \end{itemize}
% \end{frame}

\begin{frame}
\frametitle{Tutorial Based on ECLiPSe ELearning Course}
\begin{itemize}
\item Self-study course in constraint programming
\item Supported by Cisco Systems and Silicon Valley Community Foundation
\item Multi-media format, video lectures, slides, handout etc
\item {\small \url{https://eclipseclp.org/ELearning/index.html}}
\end{itemize}
\end{frame}

\begin{frame}
\frametitle{Also Part of CRT-AI Constraint Week}
\begin{itemize}
\item Annual one week course on CP and Optimization in Ireland
\item Part of national training program for PhD students in AI
\item \url{https://www.crt-ai.ie/}
\end{itemize}
\end{frame}



\begin{frame}
\frametitle{Constraint Programming - in a nutshell}
\begin{itemize}
\item Declarative description of problems with
\begin{itemize}
\item {\em Variables} which range over (finite) sets of values
\item {\em Constraints} over subsets of variables which restrict possible value combinations
\item A {\em solution} is a value assignment which satisfies all constraints
\end{itemize}

\item Constraint propagation/reasoning
\begin{itemize}
\item Removing inconsistent values for variables
\item Detect failure if constraint can not be satisfied
\item Interaction of constraints via shared variables
\item Incomplete
\end{itemize}

\item Search
\begin{itemize}
\item User controlled assignment of values to variables
\item Each step triggers constraint propagation 
\end{itemize}
\item Different domains require/allow different methods
\end{itemize}
\end{frame}

\begin{frame}
  \frametitle{Constraint Programming is Different}
  \begin{itemize}
  \item Declarative Programming
    \begin{itemize}
    \item Concentrate on what you want
      \item Not how to get there
      \item Program != Algorithm
      \item Program = Model
    \end{itemize}
    \item Applied to Combinatorial Problems
      \begin{itemize}
        \item No complete polynomial algorithms known (exist?)
        \item CP less ad-hoc than heuristics
        \item Models can evolve
  \end{itemize}
  \end{itemize}
  \end{frame}
    
\begin{frame}
  \frametitle{A Subtractive Process}
  \begin{textblock}{4}(8,-3)
    \includegraphics[width=4cm]{../introduction/images/stages}
  \end{textblock}
  \vfill
  \begin{quote}
    ``Oh, bosh, as Mr. Ruskin says. Sculpture, per se, is the simplest thing in the world. All you have to do is to take a big chunk of marble and a hammer and chisel, make up your mind what you are about to create and chip off all the marble you don't want.''-Paris Gaulois.
  \end{quote}
  
  {\tiny Source: \url{https://quoteinvestigator.com/2014/06/22/chip-away/}}
\end{frame}

\begin{frame}
\frametitle{Basic Process}
\begin{center}
\begin{tikzpicture}
\node[shape=rectangle,fill=pantone127-4] (problem) at (2,5) {Problem};
\node (human) at (2,4) {Human};
\node[shape=rectangle,fill=pantone127-4] (model) at (2,3) {Model};
\node (solver) at (2,2) {Constraint Solver/Search};
\node[shape=rectangle,fill=pantone157-8] (solution) at (2,1) {Solution};
\draw[-] (problem) -- (human);
\draw[->] (human) -- (model);
\draw[-] (model) -- (solver);
\draw[->] (solver) -- (solution);
\end{tikzpicture}
\end{center}
\end{frame}

\begin{frame}
\frametitle{More Realistic}
\begin{center}
\begin{tikzpicture}[xscale=4.0]
\node[shape=rectangle,fill=pantone127-4] (problem) at (2,5) {Problem};
\node (human) at (2,4) {Human};
\node[shape=rectangle,fill=pantone127-4] (model) at (2,3) {Model};
\node (solver) at (2,2) {Constraint Solver/Search};
\node[shape=rectangle,fill=pantone24-8] (hangs) at (1,1) {Hangs};
\node[shape=rectangle,fill=pantone157-8] (solution) at (2,1) {Solution};
\node[shape=rectangle,fill=pantone24-8] (wrong) at (3,1) {Wrong Solution};
\draw[-] (problem) -- (human);
\draw[->] (human) -- (model);
\draw[-] (model) -- (solver);
\draw[->] (solver) -- (hangs);
\draw[->] (solver) -- (solution);
\draw[->] (solver) -- (wrong);
\draw[->] (hangs) -- (1,4) -- (human);
\draw[->] (wrong) -- (3,4) -- (human);
\end{tikzpicture}
\end{center}
\end{frame}

\begin{frame}
\frametitle{Dual Role of Model}
\begin{itemize}
\item Allows Human to Express Problem
\begin{itemize}
\item Close to Problem Domain
\item Constraints as Abstractions
\end{itemize}
\item Allows Solver to Execute
\begin{itemize}
\item Variables as Communication Mechanism
\item Constraints as Algorithms
\end{itemize}
\end{itemize}
\end{frame}

\begin{frame}
\frametitle{Modelling Frameworks}
\begin{itemize}
\item \href{https://www.minizinc.org/}{MiniZinc} (NICTA, Monash University, Australia)
\item \href{https://github.com/eomahony/Numberjack}{NumberJack} (Insight, Ireland) 
\item \href{https://www-users.york.ac.uk/peter.nightingale/savilerow//}{EssencePrime/SavilleRow} (UK) 
\item \href{https://cpmpy.readthedocs.io/en/latest/}{CPMpy} (KU Leuven)
\item Allow use of multiple back-end solvers
\item Compile model into variants for each solver
\item A priori solver independent model(CP, MIP, SAT)
\end{itemize}
\end{frame} 

\begin{frame}
\frametitle{Framework Process}
\begin{center}
\begin{tikzpicture}[xscale=2]
\node[shape=rectangle,fill=pantone127-4] (problem) at (2.5,5) {Problem};
\node (human) at (2.5,4) {Human};
\node[shape=rectangle,fill=pantone127-4] (model) at (2.5,3) {Model};
\node (compile) at (2.5,2) {Compile/Reformulate};
\node (solver1) at (1,1) {CP};
\node (solver2) at (2,1) {MIP};
\node (solver3) at (3,1) {SAT};
\node (solver4) at (4,1) {Other};
\node[shape=rectangle,fill=pantone157-8] (solution1) at (1,0) {Solution};
\node[shape=rectangle,fill=pantone157-8] (solution2) at (2,0) {Solution};
\node[shape=rectangle,fill=pantone157-8] (solution3) at (3,0) {Solution};
\node[shape=rectangle,fill=pantone157-8] (solution4) at (4,0) {Solution};
\draw[-] (problem) -- (human);
\draw[->] (human) -- (model);
\draw[-] (model) -- (compile);
\draw[-] (compile) -- (solver1);
\draw[-] (compile) -- (solver2);
\draw[-] (compile) -- (solver3);
\draw[-] (compile) -- (solver4);
\draw[->] (solver1) -- (solution1);
\draw[->] (solver2) -- (solution2);
\draw[->] (solver3) -- (solution3);
\draw[->] (solver4) -- (solution4);
\end{tikzpicture}
\end{center}
\end{frame}

% \begin{frame}
  % \frametitle{Do It Now!}
  % \begin{itemize}
  % \item Download and install Minizinc
  % \item \url{https://www.minizinc.org/}
  % \end{itemize}
% \end{frame}


\begin{frame}
\frametitle{Why use Puzzles as Examples?}
\begin{itemize}
\item Easy to understand the problem
\item Solvable by hand without specialized knowledge
\item Possible to compare automated to manual solving process
\end{itemize}

\begin{quote}
The puzzle, though inanimate, is presented as a solvable problem without lasting negative consequences, a very low-risk low-reward situation. By being a puzzle, the object is attempting to convince the user that it must be completed.
\end{quote}
{\tiny Source: \href{https://www.everydayrhetoric.com/2020/03/the-rhetoric-of-puzzles/}{Every Day Rhetoric}}
\end{frame}


\mode<all>{
\part{Basic Constraint Propagation}
}
\begin{frame}
\partpage
\end{frame}
\mode<all>{
\input{../sendmore/body}
}

\mode<all>{
\part{Global Constraints}
}
\begin{frame}
\partpage
\end{frame}
\mode<all>{
\input{../sudoku/body}
}

\clearpage
\mode<all>{
\part{Customizing Search}
}
\begin{frame}
\partpage
\end{frame}
\mode<all>{
\input{../nqueen/body}
}


%% \clearpage
%% \mode<all>{
%%   \part{Problem Decomposition}
%% }
%% \begin{frame}
%%   \partpage
%% \end{frame}
%% \mode<all>{
%%   \input{../party/body}
%%   }

%% \clearpage
%% \mode<all>{
%%   \part{More Global Constraints}
%% }
%% \begin{frame}
%%   \partpage
%% \end{frame}
%% \mode<all>{
%%   \input{../car/body}
%%   }

%% \clearpage
%% \mode<all>{
%%   \part{Methodology}
%% }
%% \begin{frame}
%%   \partpage
%% \end{frame}
%% \mode<all>{
%%   \input{../methodology/body}
%%   }

\clearpage
\mode<all>{
\part{What is missing?}
}
\begin{frame}
\partpage
\end{frame}

\begin{frame}
\frametitle{Many Specialized Topics}
\begin{itemize}
\item How to design efficient core engine
\item Hybrids with LP/MIP tools
\item Hybrids with SAT
\item Symmetry breaking
\item Use of MDD/BDD to encode sets of solutions
\item High level modelling tools
\item Debugging/visualization
\end{itemize}
\end{frame}

\begin{frame}
\frametitle{Reformulation}
\begin{itemize}
\item Just because the user has modelled it this way, it doesn't mean we have to solve it that way
\begin{itemize}
\item Replace some constraint(s) by other, equivalent constraints
\item Because we don't have that constraint in our system
\item For performance
\end{itemize}
\end{itemize}
\end{frame}

\begin{frame}
\frametitle{Learning}
\begin{itemize}
\item While solving the problem we can learn how to strengthen the model/search 
\begin{itemize}
\item Understand which constraints/method contribute to propagation and change schedule
\item Learn no-good constraints by explaining failure
\item Adapt search strategy based on search experience
\end{itemize}
\end{itemize}
\end{frame}

%% \begin{frame}
%% \frametitle{Refined Process}
%% \begin{center}
%% \begin{tikzpicture}
%% \node[shape=rectangle,fill=pantone127-4] (problem) at (3,7) {Problem};
%% \node (human) at (3,6) {Human};
%% \node[shape=rectangle,fill=pantone127-4] (model) at (3,5) {User Model};
%% \node (reformulation) at (3,4) {Reformulation};
%% \node[shape=rectangle,fill=pantone127-4] (implementation) at (3,3) {Implementation Model};
%% \node (solver) at (3,2) {Constraint Solver/Search};
%% \node[shape=rectangle,fill=pantone157-8] (solution) at (3,1) {Solution};
%% \draw[-] (problem) -- (human);
%% \draw[->] (human) -- (model);
%% \draw[-] (model) -- (reformulation);
%% \draw[->] (reformulation) -- (implementation);
%% \draw[-] (implementation) -- (solver);
%% \draw[->] (solver) -- (0,2) -- (0,3) -- (implementation);
%% \draw[->] (solver) -- (solution);
%% \end{tikzpicture}
%% \end{center}
%% \end{frame}

%\begin{frame}
%\frametitle{What is CP actually used for?}
%\url{http://hsimonis.wordpress.com}
%\includegraphics[width=10cm]{../introduction/images/blog}
%\end{frame}

%% \begin{frame}
%% \frametitle{Exercises}
%% \begin{itemize}
%% \item Install Constraint Solver of your Choice
%% \item MiniZinc (MiningZinc) will be used in later sessions
%% \item Run Sendmoremoney, Sudoku, Queens examples
%% \item Can you choose which propagator to use for alldifferent?
%% \item Can you use different search strategies?
%% \end{itemize}
%% \end{frame}

\begin{frame}
\frametitle{More Learning Resources}
\begin{itemize}
\item Survey of Methods, Resources, and Formats for Teaching Constraint Programming
\begin{itemize}
\item by Tejas Santanam, Helmut Simonis
\item \url{https://doi.org/10.48550/arXiv.2403.12717}
\item Based on survey of community for WTCP 2023
\item \url{https://hsimonis.github.io/WTCP2023/}
\end{itemize}
\end{itemize}
\end{frame}


%%   }

\clearpage
\mode<all>{
\part{What is missing?}
}
\begin{frame}
\partpage
\end{frame}

\begin{frame}
\frametitle{Many Specialized Topics}
\begin{itemize}
\item How to design efficient core engine
\item Hybrids with LP/MIP tools
\item Hybrids with SAT
\item Symmetry breaking
\item Use of MDD/BDD to encode sets of solutions
\item High level modelling tools
\item Debugging/visualization
\end{itemize}
\end{frame}

\begin{frame}
\frametitle{Reformulation}
\begin{itemize}
\item Just because the user has modelled it this way, it doesn't mean we have to solve it that way
\begin{itemize}
\item Replace some constraint(s) by other, equivalent constraints
\item Because we don't have that constraint in our system
\item For performance
\end{itemize}
\end{itemize}
\end{frame}

\begin{frame}
\frametitle{Learning}
\begin{itemize}
\item While solving the problem we can learn how to strengthen the model/search 
\begin{itemize}
\item Understand which constraints/method contribute to propagation and change schedule
\item Learn no-good constraints by explaining failure
\item Adapt search strategy based on search experience
\end{itemize}
\end{itemize}
\end{frame}

%% \begin{frame}
%% \frametitle{Refined Process}
%% \begin{center}
%% \begin{tikzpicture}
%% \node[shape=rectangle,fill=pantone127-4] (problem) at (3,7) {Problem};
%% \node (human) at (3,6) {Human};
%% \node[shape=rectangle,fill=pantone127-4] (model) at (3,5) {User Model};
%% \node (reformulation) at (3,4) {Reformulation};
%% \node[shape=rectangle,fill=pantone127-4] (implementation) at (3,3) {Implementation Model};
%% \node (solver) at (3,2) {Constraint Solver/Search};
%% \node[shape=rectangle,fill=pantone157-8] (solution) at (3,1) {Solution};
%% \draw[-] (problem) -- (human);
%% \draw[->] (human) -- (model);
%% \draw[-] (model) -- (reformulation);
%% \draw[->] (reformulation) -- (implementation);
%% \draw[-] (implementation) -- (solver);
%% \draw[->] (solver) -- (0,2) -- (0,3) -- (implementation);
%% \draw[->] (solver) -- (solution);
%% \end{tikzpicture}
%% \end{center}
%% \end{frame}

%\begin{frame}
%\frametitle{What is CP actually used for?}
%\url{http://hsimonis.wordpress.com}
%\includegraphics[width=10cm]{../introduction/images/blog}
%\end{frame}

%% \begin{frame}
%% \frametitle{Exercises}
%% \begin{itemize}
%% \item Install Constraint Solver of your Choice
%% \item MiniZinc (MiningZinc) will be used in later sessions
%% \item Run Sendmoremoney, Sudoku, Queens examples
%% \item Can you choose which propagator to use for alldifferent?
%% \item Can you use different search strategies?
%% \end{itemize}
%% \end{frame}

\begin{frame}
\frametitle{More Learning Resources}
\begin{itemize}
\item Survey of Methods, Resources, and Formats for Teaching Constraint Programming
\begin{itemize}
\item by Tejas Santanam, Helmut Simonis
\item \url{https://doi.org/10.48550/arXiv.2403.12717}
\item Based on survey of community for WTCP 2023
\item \url{https://hsimonis.github.io/WTCP2023/}
\end{itemize}
\end{itemize}
\end{frame}


%%   }

%% \clearpage
%% \mode<all>{
%%   \part{Methodology}
%% }
%% \begin{frame}
%%   \partpage
%% \end{frame}
%% \mode<all>{
%%   \begin{frame}
\frametitle{Objectives}
\begin{itemize}
\item Overview of Core Constraint Programming
\item Three Main Concepts
\begin{itemize}
\item Constraint Propagation
\item Global Constraints
\item Customizing Search
\end{itemize}
%\item Get Some Experience with MiniZinc
\item Topics will be treated in more detail in later parts of the school
\item Based on Examples, not Formal Description
\end{itemize}
\end{frame}

\begin{frame}
\frametitle{Outline}
\begin{itemize}
\item Why Constraint Programming?
\item Constraint Propagation
\item Global Constraints
\item Customizing Search
%\item Choices in Modelling
%\item Decomposition
%\item More Global Constraints
%\item Methodology
%\item What is missing?
\end{itemize}
\end{frame}

% \begin{frame}
  % \frametitle{Using MiniZinc IDE}
  % \begin{itemize}
  % \item Developed in the Australian NICTA project
  % \item Maintained by Monash University
  % \item Modelling tool with multiple back-end solvers
  % \item Available from \url{https://www.minizinc.org/}
  % \end{itemize}
% \end{frame}


% \begin{frame}
% \frametitle{Examples in ECLiPSe}
% \begin{itemize}
% \item Open sourced constraint programming language
% \item Development goes back to 1985
% \item ECRC, ICL, IC-Parc, PTL, Cisco
% \item \url{https://eclipseclp.org/}
% \item Specialities
% \begin{itemize}
% \item Develop new solvers for specific domains
% \item Integration with MIP
% \end{itemize}
% \item Not included in bundled MiniZinc IDE
% \item Specialized visualization tools used here
  % \begin{itemize}
  % \item CP-Viz, Simonis et al. 2010
    % \end{itemize}
% \end{itemize}
% \end{frame}

\begin{frame}
\frametitle{Tutorial Based on ECLiPSe ELearning Course}
\begin{itemize}
\item Self-study course in constraint programming
\item Supported by Cisco Systems and Silicon Valley Community Foundation
\item Multi-media format, video lectures, slides, handout etc
\item {\small \url{https://eclipseclp.org/ELearning/index.html}}
\end{itemize}
\end{frame}

\begin{frame}
\frametitle{Also Part of CRT-AI Constraint Week}
\begin{itemize}
\item Annual one week course on CP and Optimization in Ireland
\item Part of national training program for PhD students in AI
\item \url{https://www.crt-ai.ie/}
\end{itemize}
\end{frame}



\begin{frame}
\frametitle{Constraint Programming - in a nutshell}
\begin{itemize}
\item Declarative description of problems with
\begin{itemize}
\item {\em Variables} which range over (finite) sets of values
\item {\em Constraints} over subsets of variables which restrict possible value combinations
\item A {\em solution} is a value assignment which satisfies all constraints
\end{itemize}

\item Constraint propagation/reasoning
\begin{itemize}
\item Removing inconsistent values for variables
\item Detect failure if constraint can not be satisfied
\item Interaction of constraints via shared variables
\item Incomplete
\end{itemize}

\item Search
\begin{itemize}
\item User controlled assignment of values to variables
\item Each step triggers constraint propagation 
\end{itemize}
\item Different domains require/allow different methods
\end{itemize}
\end{frame}

\begin{frame}
  \frametitle{Constraint Programming is Different}
  \begin{itemize}
  \item Declarative Programming
    \begin{itemize}
    \item Concentrate on what you want
      \item Not how to get there
      \item Program != Algorithm
      \item Program = Model
    \end{itemize}
    \item Applied to Combinatorial Problems
      \begin{itemize}
        \item No complete polynomial algorithms known (exist?)
        \item CP less ad-hoc than heuristics
        \item Models can evolve
  \end{itemize}
  \end{itemize}
  \end{frame}
    
\begin{frame}
  \frametitle{A Subtractive Process}
  \begin{textblock}{4}(8,-3)
    \includegraphics[width=4cm]{../introduction/images/stages}
  \end{textblock}
  \vfill
  \begin{quote}
    ``Oh, bosh, as Mr. Ruskin says. Sculpture, per se, is the simplest thing in the world. All you have to do is to take a big chunk of marble and a hammer and chisel, make up your mind what you are about to create and chip off all the marble you don't want.''-Paris Gaulois.
  \end{quote}
  
  {\tiny Source: \url{https://quoteinvestigator.com/2014/06/22/chip-away/}}
\end{frame}

\begin{frame}
\frametitle{Basic Process}
\begin{center}
\begin{tikzpicture}
\node[shape=rectangle,fill=pantone127-4] (problem) at (2,5) {Problem};
\node (human) at (2,4) {Human};
\node[shape=rectangle,fill=pantone127-4] (model) at (2,3) {Model};
\node (solver) at (2,2) {Constraint Solver/Search};
\node[shape=rectangle,fill=pantone157-8] (solution) at (2,1) {Solution};
\draw[-] (problem) -- (human);
\draw[->] (human) -- (model);
\draw[-] (model) -- (solver);
\draw[->] (solver) -- (solution);
\end{tikzpicture}
\end{center}
\end{frame}

\begin{frame}
\frametitle{More Realistic}
\begin{center}
\begin{tikzpicture}[xscale=4.0]
\node[shape=rectangle,fill=pantone127-4] (problem) at (2,5) {Problem};
\node (human) at (2,4) {Human};
\node[shape=rectangle,fill=pantone127-4] (model) at (2,3) {Model};
\node (solver) at (2,2) {Constraint Solver/Search};
\node[shape=rectangle,fill=pantone24-8] (hangs) at (1,1) {Hangs};
\node[shape=rectangle,fill=pantone157-8] (solution) at (2,1) {Solution};
\node[shape=rectangle,fill=pantone24-8] (wrong) at (3,1) {Wrong Solution};
\draw[-] (problem) -- (human);
\draw[->] (human) -- (model);
\draw[-] (model) -- (solver);
\draw[->] (solver) -- (hangs);
\draw[->] (solver) -- (solution);
\draw[->] (solver) -- (wrong);
\draw[->] (hangs) -- (1,4) -- (human);
\draw[->] (wrong) -- (3,4) -- (human);
\end{tikzpicture}
\end{center}
\end{frame}

\begin{frame}
\frametitle{Dual Role of Model}
\begin{itemize}
\item Allows Human to Express Problem
\begin{itemize}
\item Close to Problem Domain
\item Constraints as Abstractions
\end{itemize}
\item Allows Solver to Execute
\begin{itemize}
\item Variables as Communication Mechanism
\item Constraints as Algorithms
\end{itemize}
\end{itemize}
\end{frame}

\begin{frame}
\frametitle{Modelling Frameworks}
\begin{itemize}
\item \href{https://www.minizinc.org/}{MiniZinc} (NICTA, Monash University, Australia)
\item \href{https://github.com/eomahony/Numberjack}{NumberJack} (Insight, Ireland) 
\item \href{https://www-users.york.ac.uk/peter.nightingale/savilerow//}{EssencePrime/SavilleRow} (UK) 
\item \href{https://cpmpy.readthedocs.io/en/latest/}{CPMpy} (KU Leuven)
\item Allow use of multiple back-end solvers
\item Compile model into variants for each solver
\item A priori solver independent model(CP, MIP, SAT)
\end{itemize}
\end{frame} 

\begin{frame}
\frametitle{Framework Process}
\begin{center}
\begin{tikzpicture}[xscale=2]
\node[shape=rectangle,fill=pantone127-4] (problem) at (2.5,5) {Problem};
\node (human) at (2.5,4) {Human};
\node[shape=rectangle,fill=pantone127-4] (model) at (2.5,3) {Model};
\node (compile) at (2.5,2) {Compile/Reformulate};
\node (solver1) at (1,1) {CP};
\node (solver2) at (2,1) {MIP};
\node (solver3) at (3,1) {SAT};
\node (solver4) at (4,1) {Other};
\node[shape=rectangle,fill=pantone157-8] (solution1) at (1,0) {Solution};
\node[shape=rectangle,fill=pantone157-8] (solution2) at (2,0) {Solution};
\node[shape=rectangle,fill=pantone157-8] (solution3) at (3,0) {Solution};
\node[shape=rectangle,fill=pantone157-8] (solution4) at (4,0) {Solution};
\draw[-] (problem) -- (human);
\draw[->] (human) -- (model);
\draw[-] (model) -- (compile);
\draw[-] (compile) -- (solver1);
\draw[-] (compile) -- (solver2);
\draw[-] (compile) -- (solver3);
\draw[-] (compile) -- (solver4);
\draw[->] (solver1) -- (solution1);
\draw[->] (solver2) -- (solution2);
\draw[->] (solver3) -- (solution3);
\draw[->] (solver4) -- (solution4);
\end{tikzpicture}
\end{center}
\end{frame}

% \begin{frame}
  % \frametitle{Do It Now!}
  % \begin{itemize}
  % \item Download and install Minizinc
  % \item \url{https://www.minizinc.org/}
  % \end{itemize}
% \end{frame}


\begin{frame}
\frametitle{Why use Puzzles as Examples?}
\begin{itemize}
\item Easy to understand the problem
\item Solvable by hand without specialized knowledge
\item Possible to compare automated to manual solving process
\end{itemize}

\begin{quote}
The puzzle, though inanimate, is presented as a solvable problem without lasting negative consequences, a very low-risk low-reward situation. By being a puzzle, the object is attempting to convince the user that it must be completed.
\end{quote}
{\tiny Source: \href{https://www.everydayrhetoric.com/2020/03/the-rhetoric-of-puzzles/}{Every Day Rhetoric}}
\end{frame}


\mode<all>{
\part{Basic Constraint Propagation}
}
\begin{frame}
\partpage
\end{frame}
\mode<all>{
\begin{frame}
\frametitle{Objectives}
\begin{itemize}
\item Overview of Core Constraint Programming
\item Three Main Concepts
\begin{itemize}
\item Constraint Propagation
\item Global Constraints
\item Customizing Search
\end{itemize}
%\item Get Some Experience with MiniZinc
\item Topics will be treated in more detail in later parts of the school
\item Based on Examples, not Formal Description
\end{itemize}
\end{frame}

\begin{frame}
\frametitle{Outline}
\begin{itemize}
\item Why Constraint Programming?
\item Constraint Propagation
\item Global Constraints
\item Customizing Search
%\item Choices in Modelling
%\item Decomposition
%\item More Global Constraints
%\item Methodology
%\item What is missing?
\end{itemize}
\end{frame}

% \begin{frame}
  % \frametitle{Using MiniZinc IDE}
  % \begin{itemize}
  % \item Developed in the Australian NICTA project
  % \item Maintained by Monash University
  % \item Modelling tool with multiple back-end solvers
  % \item Available from \url{https://www.minizinc.org/}
  % \end{itemize}
% \end{frame}


% \begin{frame}
% \frametitle{Examples in ECLiPSe}
% \begin{itemize}
% \item Open sourced constraint programming language
% \item Development goes back to 1985
% \item ECRC, ICL, IC-Parc, PTL, Cisco
% \item \url{https://eclipseclp.org/}
% \item Specialities
% \begin{itemize}
% \item Develop new solvers for specific domains
% \item Integration with MIP
% \end{itemize}
% \item Not included in bundled MiniZinc IDE
% \item Specialized visualization tools used here
  % \begin{itemize}
  % \item CP-Viz, Simonis et al. 2010
    % \end{itemize}
% \end{itemize}
% \end{frame}

\begin{frame}
\frametitle{Tutorial Based on ECLiPSe ELearning Course}
\begin{itemize}
\item Self-study course in constraint programming
\item Supported by Cisco Systems and Silicon Valley Community Foundation
\item Multi-media format, video lectures, slides, handout etc
\item {\small \url{https://eclipseclp.org/ELearning/index.html}}
\end{itemize}
\end{frame}

\begin{frame}
\frametitle{Also Part of CRT-AI Constraint Week}
\begin{itemize}
\item Annual one week course on CP and Optimization in Ireland
\item Part of national training program for PhD students in AI
\item \url{https://www.crt-ai.ie/}
\end{itemize}
\end{frame}



\begin{frame}
\frametitle{Constraint Programming - in a nutshell}
\begin{itemize}
\item Declarative description of problems with
\begin{itemize}
\item {\em Variables} which range over (finite) sets of values
\item {\em Constraints} over subsets of variables which restrict possible value combinations
\item A {\em solution} is a value assignment which satisfies all constraints
\end{itemize}

\item Constraint propagation/reasoning
\begin{itemize}
\item Removing inconsistent values for variables
\item Detect failure if constraint can not be satisfied
\item Interaction of constraints via shared variables
\item Incomplete
\end{itemize}

\item Search
\begin{itemize}
\item User controlled assignment of values to variables
\item Each step triggers constraint propagation 
\end{itemize}
\item Different domains require/allow different methods
\end{itemize}
\end{frame}

\begin{frame}
  \frametitle{Constraint Programming is Different}
  \begin{itemize}
  \item Declarative Programming
    \begin{itemize}
    \item Concentrate on what you want
      \item Not how to get there
      \item Program != Algorithm
      \item Program = Model
    \end{itemize}
    \item Applied to Combinatorial Problems
      \begin{itemize}
        \item No complete polynomial algorithms known (exist?)
        \item CP less ad-hoc than heuristics
        \item Models can evolve
  \end{itemize}
  \end{itemize}
  \end{frame}
    
\begin{frame}
  \frametitle{A Subtractive Process}
  \begin{textblock}{4}(8,-3)
    \includegraphics[width=4cm]{../introduction/images/stages}
  \end{textblock}
  \vfill
  \begin{quote}
    ``Oh, bosh, as Mr. Ruskin says. Sculpture, per se, is the simplest thing in the world. All you have to do is to take a big chunk of marble and a hammer and chisel, make up your mind what you are about to create and chip off all the marble you don't want.''-Paris Gaulois.
  \end{quote}
  
  {\tiny Source: \url{https://quoteinvestigator.com/2014/06/22/chip-away/}}
\end{frame}

\begin{frame}
\frametitle{Basic Process}
\begin{center}
\begin{tikzpicture}
\node[shape=rectangle,fill=pantone127-4] (problem) at (2,5) {Problem};
\node (human) at (2,4) {Human};
\node[shape=rectangle,fill=pantone127-4] (model) at (2,3) {Model};
\node (solver) at (2,2) {Constraint Solver/Search};
\node[shape=rectangle,fill=pantone157-8] (solution) at (2,1) {Solution};
\draw[-] (problem) -- (human);
\draw[->] (human) -- (model);
\draw[-] (model) -- (solver);
\draw[->] (solver) -- (solution);
\end{tikzpicture}
\end{center}
\end{frame}

\begin{frame}
\frametitle{More Realistic}
\begin{center}
\begin{tikzpicture}[xscale=4.0]
\node[shape=rectangle,fill=pantone127-4] (problem) at (2,5) {Problem};
\node (human) at (2,4) {Human};
\node[shape=rectangle,fill=pantone127-4] (model) at (2,3) {Model};
\node (solver) at (2,2) {Constraint Solver/Search};
\node[shape=rectangle,fill=pantone24-8] (hangs) at (1,1) {Hangs};
\node[shape=rectangle,fill=pantone157-8] (solution) at (2,1) {Solution};
\node[shape=rectangle,fill=pantone24-8] (wrong) at (3,1) {Wrong Solution};
\draw[-] (problem) -- (human);
\draw[->] (human) -- (model);
\draw[-] (model) -- (solver);
\draw[->] (solver) -- (hangs);
\draw[->] (solver) -- (solution);
\draw[->] (solver) -- (wrong);
\draw[->] (hangs) -- (1,4) -- (human);
\draw[->] (wrong) -- (3,4) -- (human);
\end{tikzpicture}
\end{center}
\end{frame}

\begin{frame}
\frametitle{Dual Role of Model}
\begin{itemize}
\item Allows Human to Express Problem
\begin{itemize}
\item Close to Problem Domain
\item Constraints as Abstractions
\end{itemize}
\item Allows Solver to Execute
\begin{itemize}
\item Variables as Communication Mechanism
\item Constraints as Algorithms
\end{itemize}
\end{itemize}
\end{frame}

\begin{frame}
\frametitle{Modelling Frameworks}
\begin{itemize}
\item \href{https://www.minizinc.org/}{MiniZinc} (NICTA, Monash University, Australia)
\item \href{https://github.com/eomahony/Numberjack}{NumberJack} (Insight, Ireland) 
\item \href{https://www-users.york.ac.uk/peter.nightingale/savilerow//}{EssencePrime/SavilleRow} (UK) 
\item \href{https://cpmpy.readthedocs.io/en/latest/}{CPMpy} (KU Leuven)
\item Allow use of multiple back-end solvers
\item Compile model into variants for each solver
\item A priori solver independent model(CP, MIP, SAT)
\end{itemize}
\end{frame} 

\begin{frame}
\frametitle{Framework Process}
\begin{center}
\begin{tikzpicture}[xscale=2]
\node[shape=rectangle,fill=pantone127-4] (problem) at (2.5,5) {Problem};
\node (human) at (2.5,4) {Human};
\node[shape=rectangle,fill=pantone127-4] (model) at (2.5,3) {Model};
\node (compile) at (2.5,2) {Compile/Reformulate};
\node (solver1) at (1,1) {CP};
\node (solver2) at (2,1) {MIP};
\node (solver3) at (3,1) {SAT};
\node (solver4) at (4,1) {Other};
\node[shape=rectangle,fill=pantone157-8] (solution1) at (1,0) {Solution};
\node[shape=rectangle,fill=pantone157-8] (solution2) at (2,0) {Solution};
\node[shape=rectangle,fill=pantone157-8] (solution3) at (3,0) {Solution};
\node[shape=rectangle,fill=pantone157-8] (solution4) at (4,0) {Solution};
\draw[-] (problem) -- (human);
\draw[->] (human) -- (model);
\draw[-] (model) -- (compile);
\draw[-] (compile) -- (solver1);
\draw[-] (compile) -- (solver2);
\draw[-] (compile) -- (solver3);
\draw[-] (compile) -- (solver4);
\draw[->] (solver1) -- (solution1);
\draw[->] (solver2) -- (solution2);
\draw[->] (solver3) -- (solution3);
\draw[->] (solver4) -- (solution4);
\end{tikzpicture}
\end{center}
\end{frame}

% \begin{frame}
  % \frametitle{Do It Now!}
  % \begin{itemize}
  % \item Download and install Minizinc
  % \item \url{https://www.minizinc.org/}
  % \end{itemize}
% \end{frame}


\begin{frame}
\frametitle{Why use Puzzles as Examples?}
\begin{itemize}
\item Easy to understand the problem
\item Solvable by hand without specialized knowledge
\item Possible to compare automated to manual solving process
\end{itemize}

\begin{quote}
The puzzle, though inanimate, is presented as a solvable problem without lasting negative consequences, a very low-risk low-reward situation. By being a puzzle, the object is attempting to convince the user that it must be completed.
\end{quote}
{\tiny Source: \href{https://www.everydayrhetoric.com/2020/03/the-rhetoric-of-puzzles/}{Every Day Rhetoric}}
\end{frame}


\mode<all>{
\part{Basic Constraint Propagation}
}
\begin{frame}
\partpage
\end{frame}
\mode<all>{
\input{../sendmore/body}
}

\mode<all>{
\part{Global Constraints}
}
\begin{frame}
\partpage
\end{frame}
\mode<all>{
\input{../sudoku/body}
}

\clearpage
\mode<all>{
\part{Customizing Search}
}
\begin{frame}
\partpage
\end{frame}
\mode<all>{
\input{../nqueen/body}
}


%% \clearpage
%% \mode<all>{
%%   \part{Problem Decomposition}
%% }
%% \begin{frame}
%%   \partpage
%% \end{frame}
%% \mode<all>{
%%   \input{../party/body}
%%   }

%% \clearpage
%% \mode<all>{
%%   \part{More Global Constraints}
%% }
%% \begin{frame}
%%   \partpage
%% \end{frame}
%% \mode<all>{
%%   \input{../car/body}
%%   }

%% \clearpage
%% \mode<all>{
%%   \part{Methodology}
%% }
%% \begin{frame}
%%   \partpage
%% \end{frame}
%% \mode<all>{
%%   \input{../methodology/body}
%%   }

\clearpage
\mode<all>{
\part{What is missing?}
}
\begin{frame}
\partpage
\end{frame}

\begin{frame}
\frametitle{Many Specialized Topics}
\begin{itemize}
\item How to design efficient core engine
\item Hybrids with LP/MIP tools
\item Hybrids with SAT
\item Symmetry breaking
\item Use of MDD/BDD to encode sets of solutions
\item High level modelling tools
\item Debugging/visualization
\end{itemize}
\end{frame}

\begin{frame}
\frametitle{Reformulation}
\begin{itemize}
\item Just because the user has modelled it this way, it doesn't mean we have to solve it that way
\begin{itemize}
\item Replace some constraint(s) by other, equivalent constraints
\item Because we don't have that constraint in our system
\item For performance
\end{itemize}
\end{itemize}
\end{frame}

\begin{frame}
\frametitle{Learning}
\begin{itemize}
\item While solving the problem we can learn how to strengthen the model/search 
\begin{itemize}
\item Understand which constraints/method contribute to propagation and change schedule
\item Learn no-good constraints by explaining failure
\item Adapt search strategy based on search experience
\end{itemize}
\end{itemize}
\end{frame}

%% \begin{frame}
%% \frametitle{Refined Process}
%% \begin{center}
%% \begin{tikzpicture}
%% \node[shape=rectangle,fill=pantone127-4] (problem) at (3,7) {Problem};
%% \node (human) at (3,6) {Human};
%% \node[shape=rectangle,fill=pantone127-4] (model) at (3,5) {User Model};
%% \node (reformulation) at (3,4) {Reformulation};
%% \node[shape=rectangle,fill=pantone127-4] (implementation) at (3,3) {Implementation Model};
%% \node (solver) at (3,2) {Constraint Solver/Search};
%% \node[shape=rectangle,fill=pantone157-8] (solution) at (3,1) {Solution};
%% \draw[-] (problem) -- (human);
%% \draw[->] (human) -- (model);
%% \draw[-] (model) -- (reformulation);
%% \draw[->] (reformulation) -- (implementation);
%% \draw[-] (implementation) -- (solver);
%% \draw[->] (solver) -- (0,2) -- (0,3) -- (implementation);
%% \draw[->] (solver) -- (solution);
%% \end{tikzpicture}
%% \end{center}
%% \end{frame}

%\begin{frame}
%\frametitle{What is CP actually used for?}
%\url{http://hsimonis.wordpress.com}
%\includegraphics[width=10cm]{../introduction/images/blog}
%\end{frame}

%% \begin{frame}
%% \frametitle{Exercises}
%% \begin{itemize}
%% \item Install Constraint Solver of your Choice
%% \item MiniZinc (MiningZinc) will be used in later sessions
%% \item Run Sendmoremoney, Sudoku, Queens examples
%% \item Can you choose which propagator to use for alldifferent?
%% \item Can you use different search strategies?
%% \end{itemize}
%% \end{frame}

\begin{frame}
\frametitle{More Learning Resources}
\begin{itemize}
\item Survey of Methods, Resources, and Formats for Teaching Constraint Programming
\begin{itemize}
\item by Tejas Santanam, Helmut Simonis
\item \url{https://doi.org/10.48550/arXiv.2403.12717}
\item Based on survey of community for WTCP 2023
\item \url{https://hsimonis.github.io/WTCP2023/}
\end{itemize}
\end{itemize}
\end{frame}


}

\mode<all>{
\part{Global Constraints}
}
\begin{frame}
\partpage
\end{frame}
\mode<all>{
\begin{frame}
\frametitle{Objectives}
\begin{itemize}
\item Overview of Core Constraint Programming
\item Three Main Concepts
\begin{itemize}
\item Constraint Propagation
\item Global Constraints
\item Customizing Search
\end{itemize}
%\item Get Some Experience with MiniZinc
\item Topics will be treated in more detail in later parts of the school
\item Based on Examples, not Formal Description
\end{itemize}
\end{frame}

\begin{frame}
\frametitle{Outline}
\begin{itemize}
\item Why Constraint Programming?
\item Constraint Propagation
\item Global Constraints
\item Customizing Search
%\item Choices in Modelling
%\item Decomposition
%\item More Global Constraints
%\item Methodology
%\item What is missing?
\end{itemize}
\end{frame}

% \begin{frame}
  % \frametitle{Using MiniZinc IDE}
  % \begin{itemize}
  % \item Developed in the Australian NICTA project
  % \item Maintained by Monash University
  % \item Modelling tool with multiple back-end solvers
  % \item Available from \url{https://www.minizinc.org/}
  % \end{itemize}
% \end{frame}


% \begin{frame}
% \frametitle{Examples in ECLiPSe}
% \begin{itemize}
% \item Open sourced constraint programming language
% \item Development goes back to 1985
% \item ECRC, ICL, IC-Parc, PTL, Cisco
% \item \url{https://eclipseclp.org/}
% \item Specialities
% \begin{itemize}
% \item Develop new solvers for specific domains
% \item Integration with MIP
% \end{itemize}
% \item Not included in bundled MiniZinc IDE
% \item Specialized visualization tools used here
  % \begin{itemize}
  % \item CP-Viz, Simonis et al. 2010
    % \end{itemize}
% \end{itemize}
% \end{frame}

\begin{frame}
\frametitle{Tutorial Based on ECLiPSe ELearning Course}
\begin{itemize}
\item Self-study course in constraint programming
\item Supported by Cisco Systems and Silicon Valley Community Foundation
\item Multi-media format, video lectures, slides, handout etc
\item {\small \url{https://eclipseclp.org/ELearning/index.html}}
\end{itemize}
\end{frame}

\begin{frame}
\frametitle{Also Part of CRT-AI Constraint Week}
\begin{itemize}
\item Annual one week course on CP and Optimization in Ireland
\item Part of national training program for PhD students in AI
\item \url{https://www.crt-ai.ie/}
\end{itemize}
\end{frame}



\begin{frame}
\frametitle{Constraint Programming - in a nutshell}
\begin{itemize}
\item Declarative description of problems with
\begin{itemize}
\item {\em Variables} which range over (finite) sets of values
\item {\em Constraints} over subsets of variables which restrict possible value combinations
\item A {\em solution} is a value assignment which satisfies all constraints
\end{itemize}

\item Constraint propagation/reasoning
\begin{itemize}
\item Removing inconsistent values for variables
\item Detect failure if constraint can not be satisfied
\item Interaction of constraints via shared variables
\item Incomplete
\end{itemize}

\item Search
\begin{itemize}
\item User controlled assignment of values to variables
\item Each step triggers constraint propagation 
\end{itemize}
\item Different domains require/allow different methods
\end{itemize}
\end{frame}

\begin{frame}
  \frametitle{Constraint Programming is Different}
  \begin{itemize}
  \item Declarative Programming
    \begin{itemize}
    \item Concentrate on what you want
      \item Not how to get there
      \item Program != Algorithm
      \item Program = Model
    \end{itemize}
    \item Applied to Combinatorial Problems
      \begin{itemize}
        \item No complete polynomial algorithms known (exist?)
        \item CP less ad-hoc than heuristics
        \item Models can evolve
  \end{itemize}
  \end{itemize}
  \end{frame}
    
\begin{frame}
  \frametitle{A Subtractive Process}
  \begin{textblock}{4}(8,-3)
    \includegraphics[width=4cm]{../introduction/images/stages}
  \end{textblock}
  \vfill
  \begin{quote}
    ``Oh, bosh, as Mr. Ruskin says. Sculpture, per se, is the simplest thing in the world. All you have to do is to take a big chunk of marble and a hammer and chisel, make up your mind what you are about to create and chip off all the marble you don't want.''-Paris Gaulois.
  \end{quote}
  
  {\tiny Source: \url{https://quoteinvestigator.com/2014/06/22/chip-away/}}
\end{frame}

\begin{frame}
\frametitle{Basic Process}
\begin{center}
\begin{tikzpicture}
\node[shape=rectangle,fill=pantone127-4] (problem) at (2,5) {Problem};
\node (human) at (2,4) {Human};
\node[shape=rectangle,fill=pantone127-4] (model) at (2,3) {Model};
\node (solver) at (2,2) {Constraint Solver/Search};
\node[shape=rectangle,fill=pantone157-8] (solution) at (2,1) {Solution};
\draw[-] (problem) -- (human);
\draw[->] (human) -- (model);
\draw[-] (model) -- (solver);
\draw[->] (solver) -- (solution);
\end{tikzpicture}
\end{center}
\end{frame}

\begin{frame}
\frametitle{More Realistic}
\begin{center}
\begin{tikzpicture}[xscale=4.0]
\node[shape=rectangle,fill=pantone127-4] (problem) at (2,5) {Problem};
\node (human) at (2,4) {Human};
\node[shape=rectangle,fill=pantone127-4] (model) at (2,3) {Model};
\node (solver) at (2,2) {Constraint Solver/Search};
\node[shape=rectangle,fill=pantone24-8] (hangs) at (1,1) {Hangs};
\node[shape=rectangle,fill=pantone157-8] (solution) at (2,1) {Solution};
\node[shape=rectangle,fill=pantone24-8] (wrong) at (3,1) {Wrong Solution};
\draw[-] (problem) -- (human);
\draw[->] (human) -- (model);
\draw[-] (model) -- (solver);
\draw[->] (solver) -- (hangs);
\draw[->] (solver) -- (solution);
\draw[->] (solver) -- (wrong);
\draw[->] (hangs) -- (1,4) -- (human);
\draw[->] (wrong) -- (3,4) -- (human);
\end{tikzpicture}
\end{center}
\end{frame}

\begin{frame}
\frametitle{Dual Role of Model}
\begin{itemize}
\item Allows Human to Express Problem
\begin{itemize}
\item Close to Problem Domain
\item Constraints as Abstractions
\end{itemize}
\item Allows Solver to Execute
\begin{itemize}
\item Variables as Communication Mechanism
\item Constraints as Algorithms
\end{itemize}
\end{itemize}
\end{frame}

\begin{frame}
\frametitle{Modelling Frameworks}
\begin{itemize}
\item \href{https://www.minizinc.org/}{MiniZinc} (NICTA, Monash University, Australia)
\item \href{https://github.com/eomahony/Numberjack}{NumberJack} (Insight, Ireland) 
\item \href{https://www-users.york.ac.uk/peter.nightingale/savilerow//}{EssencePrime/SavilleRow} (UK) 
\item \href{https://cpmpy.readthedocs.io/en/latest/}{CPMpy} (KU Leuven)
\item Allow use of multiple back-end solvers
\item Compile model into variants for each solver
\item A priori solver independent model(CP, MIP, SAT)
\end{itemize}
\end{frame} 

\begin{frame}
\frametitle{Framework Process}
\begin{center}
\begin{tikzpicture}[xscale=2]
\node[shape=rectangle,fill=pantone127-4] (problem) at (2.5,5) {Problem};
\node (human) at (2.5,4) {Human};
\node[shape=rectangle,fill=pantone127-4] (model) at (2.5,3) {Model};
\node (compile) at (2.5,2) {Compile/Reformulate};
\node (solver1) at (1,1) {CP};
\node (solver2) at (2,1) {MIP};
\node (solver3) at (3,1) {SAT};
\node (solver4) at (4,1) {Other};
\node[shape=rectangle,fill=pantone157-8] (solution1) at (1,0) {Solution};
\node[shape=rectangle,fill=pantone157-8] (solution2) at (2,0) {Solution};
\node[shape=rectangle,fill=pantone157-8] (solution3) at (3,0) {Solution};
\node[shape=rectangle,fill=pantone157-8] (solution4) at (4,0) {Solution};
\draw[-] (problem) -- (human);
\draw[->] (human) -- (model);
\draw[-] (model) -- (compile);
\draw[-] (compile) -- (solver1);
\draw[-] (compile) -- (solver2);
\draw[-] (compile) -- (solver3);
\draw[-] (compile) -- (solver4);
\draw[->] (solver1) -- (solution1);
\draw[->] (solver2) -- (solution2);
\draw[->] (solver3) -- (solution3);
\draw[->] (solver4) -- (solution4);
\end{tikzpicture}
\end{center}
\end{frame}

% \begin{frame}
  % \frametitle{Do It Now!}
  % \begin{itemize}
  % \item Download and install Minizinc
  % \item \url{https://www.minizinc.org/}
  % \end{itemize}
% \end{frame}


\begin{frame}
\frametitle{Why use Puzzles as Examples?}
\begin{itemize}
\item Easy to understand the problem
\item Solvable by hand without specialized knowledge
\item Possible to compare automated to manual solving process
\end{itemize}

\begin{quote}
The puzzle, though inanimate, is presented as a solvable problem without lasting negative consequences, a very low-risk low-reward situation. By being a puzzle, the object is attempting to convince the user that it must be completed.
\end{quote}
{\tiny Source: \href{https://www.everydayrhetoric.com/2020/03/the-rhetoric-of-puzzles/}{Every Day Rhetoric}}
\end{frame}


\mode<all>{
\part{Basic Constraint Propagation}
}
\begin{frame}
\partpage
\end{frame}
\mode<all>{
\input{../sendmore/body}
}

\mode<all>{
\part{Global Constraints}
}
\begin{frame}
\partpage
\end{frame}
\mode<all>{
\input{../sudoku/body}
}

\clearpage
\mode<all>{
\part{Customizing Search}
}
\begin{frame}
\partpage
\end{frame}
\mode<all>{
\input{../nqueen/body}
}


%% \clearpage
%% \mode<all>{
%%   \part{Problem Decomposition}
%% }
%% \begin{frame}
%%   \partpage
%% \end{frame}
%% \mode<all>{
%%   \input{../party/body}
%%   }

%% \clearpage
%% \mode<all>{
%%   \part{More Global Constraints}
%% }
%% \begin{frame}
%%   \partpage
%% \end{frame}
%% \mode<all>{
%%   \input{../car/body}
%%   }

%% \clearpage
%% \mode<all>{
%%   \part{Methodology}
%% }
%% \begin{frame}
%%   \partpage
%% \end{frame}
%% \mode<all>{
%%   \input{../methodology/body}
%%   }

\clearpage
\mode<all>{
\part{What is missing?}
}
\begin{frame}
\partpage
\end{frame}

\begin{frame}
\frametitle{Many Specialized Topics}
\begin{itemize}
\item How to design efficient core engine
\item Hybrids with LP/MIP tools
\item Hybrids with SAT
\item Symmetry breaking
\item Use of MDD/BDD to encode sets of solutions
\item High level modelling tools
\item Debugging/visualization
\end{itemize}
\end{frame}

\begin{frame}
\frametitle{Reformulation}
\begin{itemize}
\item Just because the user has modelled it this way, it doesn't mean we have to solve it that way
\begin{itemize}
\item Replace some constraint(s) by other, equivalent constraints
\item Because we don't have that constraint in our system
\item For performance
\end{itemize}
\end{itemize}
\end{frame}

\begin{frame}
\frametitle{Learning}
\begin{itemize}
\item While solving the problem we can learn how to strengthen the model/search 
\begin{itemize}
\item Understand which constraints/method contribute to propagation and change schedule
\item Learn no-good constraints by explaining failure
\item Adapt search strategy based on search experience
\end{itemize}
\end{itemize}
\end{frame}

%% \begin{frame}
%% \frametitle{Refined Process}
%% \begin{center}
%% \begin{tikzpicture}
%% \node[shape=rectangle,fill=pantone127-4] (problem) at (3,7) {Problem};
%% \node (human) at (3,6) {Human};
%% \node[shape=rectangle,fill=pantone127-4] (model) at (3,5) {User Model};
%% \node (reformulation) at (3,4) {Reformulation};
%% \node[shape=rectangle,fill=pantone127-4] (implementation) at (3,3) {Implementation Model};
%% \node (solver) at (3,2) {Constraint Solver/Search};
%% \node[shape=rectangle,fill=pantone157-8] (solution) at (3,1) {Solution};
%% \draw[-] (problem) -- (human);
%% \draw[->] (human) -- (model);
%% \draw[-] (model) -- (reformulation);
%% \draw[->] (reformulation) -- (implementation);
%% \draw[-] (implementation) -- (solver);
%% \draw[->] (solver) -- (0,2) -- (0,3) -- (implementation);
%% \draw[->] (solver) -- (solution);
%% \end{tikzpicture}
%% \end{center}
%% \end{frame}

%\begin{frame}
%\frametitle{What is CP actually used for?}
%\url{http://hsimonis.wordpress.com}
%\includegraphics[width=10cm]{../introduction/images/blog}
%\end{frame}

%% \begin{frame}
%% \frametitle{Exercises}
%% \begin{itemize}
%% \item Install Constraint Solver of your Choice
%% \item MiniZinc (MiningZinc) will be used in later sessions
%% \item Run Sendmoremoney, Sudoku, Queens examples
%% \item Can you choose which propagator to use for alldifferent?
%% \item Can you use different search strategies?
%% \end{itemize}
%% \end{frame}

\begin{frame}
\frametitle{More Learning Resources}
\begin{itemize}
\item Survey of Methods, Resources, and Formats for Teaching Constraint Programming
\begin{itemize}
\item by Tejas Santanam, Helmut Simonis
\item \url{https://doi.org/10.48550/arXiv.2403.12717}
\item Based on survey of community for WTCP 2023
\item \url{https://hsimonis.github.io/WTCP2023/}
\end{itemize}
\end{itemize}
\end{frame}


}

\clearpage
\mode<all>{
\part{Customizing Search}
}
\begin{frame}
\partpage
\end{frame}
\mode<all>{
\begin{frame}
\frametitle{Objectives}
\begin{itemize}
\item Overview of Core Constraint Programming
\item Three Main Concepts
\begin{itemize}
\item Constraint Propagation
\item Global Constraints
\item Customizing Search
\end{itemize}
%\item Get Some Experience with MiniZinc
\item Topics will be treated in more detail in later parts of the school
\item Based on Examples, not Formal Description
\end{itemize}
\end{frame}

\begin{frame}
\frametitle{Outline}
\begin{itemize}
\item Why Constraint Programming?
\item Constraint Propagation
\item Global Constraints
\item Customizing Search
%\item Choices in Modelling
%\item Decomposition
%\item More Global Constraints
%\item Methodology
%\item What is missing?
\end{itemize}
\end{frame}

% \begin{frame}
  % \frametitle{Using MiniZinc IDE}
  % \begin{itemize}
  % \item Developed in the Australian NICTA project
  % \item Maintained by Monash University
  % \item Modelling tool with multiple back-end solvers
  % \item Available from \url{https://www.minizinc.org/}
  % \end{itemize}
% \end{frame}


% \begin{frame}
% \frametitle{Examples in ECLiPSe}
% \begin{itemize}
% \item Open sourced constraint programming language
% \item Development goes back to 1985
% \item ECRC, ICL, IC-Parc, PTL, Cisco
% \item \url{https://eclipseclp.org/}
% \item Specialities
% \begin{itemize}
% \item Develop new solvers for specific domains
% \item Integration with MIP
% \end{itemize}
% \item Not included in bundled MiniZinc IDE
% \item Specialized visualization tools used here
  % \begin{itemize}
  % \item CP-Viz, Simonis et al. 2010
    % \end{itemize}
% \end{itemize}
% \end{frame}

\begin{frame}
\frametitle{Tutorial Based on ECLiPSe ELearning Course}
\begin{itemize}
\item Self-study course in constraint programming
\item Supported by Cisco Systems and Silicon Valley Community Foundation
\item Multi-media format, video lectures, slides, handout etc
\item {\small \url{https://eclipseclp.org/ELearning/index.html}}
\end{itemize}
\end{frame}

\begin{frame}
\frametitle{Also Part of CRT-AI Constraint Week}
\begin{itemize}
\item Annual one week course on CP and Optimization in Ireland
\item Part of national training program for PhD students in AI
\item \url{https://www.crt-ai.ie/}
\end{itemize}
\end{frame}



\begin{frame}
\frametitle{Constraint Programming - in a nutshell}
\begin{itemize}
\item Declarative description of problems with
\begin{itemize}
\item {\em Variables} which range over (finite) sets of values
\item {\em Constraints} over subsets of variables which restrict possible value combinations
\item A {\em solution} is a value assignment which satisfies all constraints
\end{itemize}

\item Constraint propagation/reasoning
\begin{itemize}
\item Removing inconsistent values for variables
\item Detect failure if constraint can not be satisfied
\item Interaction of constraints via shared variables
\item Incomplete
\end{itemize}

\item Search
\begin{itemize}
\item User controlled assignment of values to variables
\item Each step triggers constraint propagation 
\end{itemize}
\item Different domains require/allow different methods
\end{itemize}
\end{frame}

\begin{frame}
  \frametitle{Constraint Programming is Different}
  \begin{itemize}
  \item Declarative Programming
    \begin{itemize}
    \item Concentrate on what you want
      \item Not how to get there
      \item Program != Algorithm
      \item Program = Model
    \end{itemize}
    \item Applied to Combinatorial Problems
      \begin{itemize}
        \item No complete polynomial algorithms known (exist?)
        \item CP less ad-hoc than heuristics
        \item Models can evolve
  \end{itemize}
  \end{itemize}
  \end{frame}
    
\begin{frame}
  \frametitle{A Subtractive Process}
  \begin{textblock}{4}(8,-3)
    \includegraphics[width=4cm]{../introduction/images/stages}
  \end{textblock}
  \vfill
  \begin{quote}
    ``Oh, bosh, as Mr. Ruskin says. Sculpture, per se, is the simplest thing in the world. All you have to do is to take a big chunk of marble and a hammer and chisel, make up your mind what you are about to create and chip off all the marble you don't want.''-Paris Gaulois.
  \end{quote}
  
  {\tiny Source: \url{https://quoteinvestigator.com/2014/06/22/chip-away/}}
\end{frame}

\begin{frame}
\frametitle{Basic Process}
\begin{center}
\begin{tikzpicture}
\node[shape=rectangle,fill=pantone127-4] (problem) at (2,5) {Problem};
\node (human) at (2,4) {Human};
\node[shape=rectangle,fill=pantone127-4] (model) at (2,3) {Model};
\node (solver) at (2,2) {Constraint Solver/Search};
\node[shape=rectangle,fill=pantone157-8] (solution) at (2,1) {Solution};
\draw[-] (problem) -- (human);
\draw[->] (human) -- (model);
\draw[-] (model) -- (solver);
\draw[->] (solver) -- (solution);
\end{tikzpicture}
\end{center}
\end{frame}

\begin{frame}
\frametitle{More Realistic}
\begin{center}
\begin{tikzpicture}[xscale=4.0]
\node[shape=rectangle,fill=pantone127-4] (problem) at (2,5) {Problem};
\node (human) at (2,4) {Human};
\node[shape=rectangle,fill=pantone127-4] (model) at (2,3) {Model};
\node (solver) at (2,2) {Constraint Solver/Search};
\node[shape=rectangle,fill=pantone24-8] (hangs) at (1,1) {Hangs};
\node[shape=rectangle,fill=pantone157-8] (solution) at (2,1) {Solution};
\node[shape=rectangle,fill=pantone24-8] (wrong) at (3,1) {Wrong Solution};
\draw[-] (problem) -- (human);
\draw[->] (human) -- (model);
\draw[-] (model) -- (solver);
\draw[->] (solver) -- (hangs);
\draw[->] (solver) -- (solution);
\draw[->] (solver) -- (wrong);
\draw[->] (hangs) -- (1,4) -- (human);
\draw[->] (wrong) -- (3,4) -- (human);
\end{tikzpicture}
\end{center}
\end{frame}

\begin{frame}
\frametitle{Dual Role of Model}
\begin{itemize}
\item Allows Human to Express Problem
\begin{itemize}
\item Close to Problem Domain
\item Constraints as Abstractions
\end{itemize}
\item Allows Solver to Execute
\begin{itemize}
\item Variables as Communication Mechanism
\item Constraints as Algorithms
\end{itemize}
\end{itemize}
\end{frame}

\begin{frame}
\frametitle{Modelling Frameworks}
\begin{itemize}
\item \href{https://www.minizinc.org/}{MiniZinc} (NICTA, Monash University, Australia)
\item \href{https://github.com/eomahony/Numberjack}{NumberJack} (Insight, Ireland) 
\item \href{https://www-users.york.ac.uk/peter.nightingale/savilerow//}{EssencePrime/SavilleRow} (UK) 
\item \href{https://cpmpy.readthedocs.io/en/latest/}{CPMpy} (KU Leuven)
\item Allow use of multiple back-end solvers
\item Compile model into variants for each solver
\item A priori solver independent model(CP, MIP, SAT)
\end{itemize}
\end{frame} 

\begin{frame}
\frametitle{Framework Process}
\begin{center}
\begin{tikzpicture}[xscale=2]
\node[shape=rectangle,fill=pantone127-4] (problem) at (2.5,5) {Problem};
\node (human) at (2.5,4) {Human};
\node[shape=rectangle,fill=pantone127-4] (model) at (2.5,3) {Model};
\node (compile) at (2.5,2) {Compile/Reformulate};
\node (solver1) at (1,1) {CP};
\node (solver2) at (2,1) {MIP};
\node (solver3) at (3,1) {SAT};
\node (solver4) at (4,1) {Other};
\node[shape=rectangle,fill=pantone157-8] (solution1) at (1,0) {Solution};
\node[shape=rectangle,fill=pantone157-8] (solution2) at (2,0) {Solution};
\node[shape=rectangle,fill=pantone157-8] (solution3) at (3,0) {Solution};
\node[shape=rectangle,fill=pantone157-8] (solution4) at (4,0) {Solution};
\draw[-] (problem) -- (human);
\draw[->] (human) -- (model);
\draw[-] (model) -- (compile);
\draw[-] (compile) -- (solver1);
\draw[-] (compile) -- (solver2);
\draw[-] (compile) -- (solver3);
\draw[-] (compile) -- (solver4);
\draw[->] (solver1) -- (solution1);
\draw[->] (solver2) -- (solution2);
\draw[->] (solver3) -- (solution3);
\draw[->] (solver4) -- (solution4);
\end{tikzpicture}
\end{center}
\end{frame}

% \begin{frame}
  % \frametitle{Do It Now!}
  % \begin{itemize}
  % \item Download and install Minizinc
  % \item \url{https://www.minizinc.org/}
  % \end{itemize}
% \end{frame}


\begin{frame}
\frametitle{Why use Puzzles as Examples?}
\begin{itemize}
\item Easy to understand the problem
\item Solvable by hand without specialized knowledge
\item Possible to compare automated to manual solving process
\end{itemize}

\begin{quote}
The puzzle, though inanimate, is presented as a solvable problem without lasting negative consequences, a very low-risk low-reward situation. By being a puzzle, the object is attempting to convince the user that it must be completed.
\end{quote}
{\tiny Source: \href{https://www.everydayrhetoric.com/2020/03/the-rhetoric-of-puzzles/}{Every Day Rhetoric}}
\end{frame}


\mode<all>{
\part{Basic Constraint Propagation}
}
\begin{frame}
\partpage
\end{frame}
\mode<all>{
\input{../sendmore/body}
}

\mode<all>{
\part{Global Constraints}
}
\begin{frame}
\partpage
\end{frame}
\mode<all>{
\input{../sudoku/body}
}

\clearpage
\mode<all>{
\part{Customizing Search}
}
\begin{frame}
\partpage
\end{frame}
\mode<all>{
\input{../nqueen/body}
}


%% \clearpage
%% \mode<all>{
%%   \part{Problem Decomposition}
%% }
%% \begin{frame}
%%   \partpage
%% \end{frame}
%% \mode<all>{
%%   \input{../party/body}
%%   }

%% \clearpage
%% \mode<all>{
%%   \part{More Global Constraints}
%% }
%% \begin{frame}
%%   \partpage
%% \end{frame}
%% \mode<all>{
%%   \input{../car/body}
%%   }

%% \clearpage
%% \mode<all>{
%%   \part{Methodology}
%% }
%% \begin{frame}
%%   \partpage
%% \end{frame}
%% \mode<all>{
%%   \input{../methodology/body}
%%   }

\clearpage
\mode<all>{
\part{What is missing?}
}
\begin{frame}
\partpage
\end{frame}

\begin{frame}
\frametitle{Many Specialized Topics}
\begin{itemize}
\item How to design efficient core engine
\item Hybrids with LP/MIP tools
\item Hybrids with SAT
\item Symmetry breaking
\item Use of MDD/BDD to encode sets of solutions
\item High level modelling tools
\item Debugging/visualization
\end{itemize}
\end{frame}

\begin{frame}
\frametitle{Reformulation}
\begin{itemize}
\item Just because the user has modelled it this way, it doesn't mean we have to solve it that way
\begin{itemize}
\item Replace some constraint(s) by other, equivalent constraints
\item Because we don't have that constraint in our system
\item For performance
\end{itemize}
\end{itemize}
\end{frame}

\begin{frame}
\frametitle{Learning}
\begin{itemize}
\item While solving the problem we can learn how to strengthen the model/search 
\begin{itemize}
\item Understand which constraints/method contribute to propagation and change schedule
\item Learn no-good constraints by explaining failure
\item Adapt search strategy based on search experience
\end{itemize}
\end{itemize}
\end{frame}

%% \begin{frame}
%% \frametitle{Refined Process}
%% \begin{center}
%% \begin{tikzpicture}
%% \node[shape=rectangle,fill=pantone127-4] (problem) at (3,7) {Problem};
%% \node (human) at (3,6) {Human};
%% \node[shape=rectangle,fill=pantone127-4] (model) at (3,5) {User Model};
%% \node (reformulation) at (3,4) {Reformulation};
%% \node[shape=rectangle,fill=pantone127-4] (implementation) at (3,3) {Implementation Model};
%% \node (solver) at (3,2) {Constraint Solver/Search};
%% \node[shape=rectangle,fill=pantone157-8] (solution) at (3,1) {Solution};
%% \draw[-] (problem) -- (human);
%% \draw[->] (human) -- (model);
%% \draw[-] (model) -- (reformulation);
%% \draw[->] (reformulation) -- (implementation);
%% \draw[-] (implementation) -- (solver);
%% \draw[->] (solver) -- (0,2) -- (0,3) -- (implementation);
%% \draw[->] (solver) -- (solution);
%% \end{tikzpicture}
%% \end{center}
%% \end{frame}

%\begin{frame}
%\frametitle{What is CP actually used for?}
%\url{http://hsimonis.wordpress.com}
%\includegraphics[width=10cm]{../introduction/images/blog}
%\end{frame}

%% \begin{frame}
%% \frametitle{Exercises}
%% \begin{itemize}
%% \item Install Constraint Solver of your Choice
%% \item MiniZinc (MiningZinc) will be used in later sessions
%% \item Run Sendmoremoney, Sudoku, Queens examples
%% \item Can you choose which propagator to use for alldifferent?
%% \item Can you use different search strategies?
%% \end{itemize}
%% \end{frame}

\begin{frame}
\frametitle{More Learning Resources}
\begin{itemize}
\item Survey of Methods, Resources, and Formats for Teaching Constraint Programming
\begin{itemize}
\item by Tejas Santanam, Helmut Simonis
\item \url{https://doi.org/10.48550/arXiv.2403.12717}
\item Based on survey of community for WTCP 2023
\item \url{https://hsimonis.github.io/WTCP2023/}
\end{itemize}
\end{itemize}
\end{frame}


}


%% \clearpage
%% \mode<all>{
%%   \part{Problem Decomposition}
%% }
%% \begin{frame}
%%   \partpage
%% \end{frame}
%% \mode<all>{
%%   \begin{frame}
\frametitle{Objectives}
\begin{itemize}
\item Overview of Core Constraint Programming
\item Three Main Concepts
\begin{itemize}
\item Constraint Propagation
\item Global Constraints
\item Customizing Search
\end{itemize}
%\item Get Some Experience with MiniZinc
\item Topics will be treated in more detail in later parts of the school
\item Based on Examples, not Formal Description
\end{itemize}
\end{frame}

\begin{frame}
\frametitle{Outline}
\begin{itemize}
\item Why Constraint Programming?
\item Constraint Propagation
\item Global Constraints
\item Customizing Search
%\item Choices in Modelling
%\item Decomposition
%\item More Global Constraints
%\item Methodology
%\item What is missing?
\end{itemize}
\end{frame}

% \begin{frame}
  % \frametitle{Using MiniZinc IDE}
  % \begin{itemize}
  % \item Developed in the Australian NICTA project
  % \item Maintained by Monash University
  % \item Modelling tool with multiple back-end solvers
  % \item Available from \url{https://www.minizinc.org/}
  % \end{itemize}
% \end{frame}


% \begin{frame}
% \frametitle{Examples in ECLiPSe}
% \begin{itemize}
% \item Open sourced constraint programming language
% \item Development goes back to 1985
% \item ECRC, ICL, IC-Parc, PTL, Cisco
% \item \url{https://eclipseclp.org/}
% \item Specialities
% \begin{itemize}
% \item Develop new solvers for specific domains
% \item Integration with MIP
% \end{itemize}
% \item Not included in bundled MiniZinc IDE
% \item Specialized visualization tools used here
  % \begin{itemize}
  % \item CP-Viz, Simonis et al. 2010
    % \end{itemize}
% \end{itemize}
% \end{frame}

\begin{frame}
\frametitle{Tutorial Based on ECLiPSe ELearning Course}
\begin{itemize}
\item Self-study course in constraint programming
\item Supported by Cisco Systems and Silicon Valley Community Foundation
\item Multi-media format, video lectures, slides, handout etc
\item {\small \url{https://eclipseclp.org/ELearning/index.html}}
\end{itemize}
\end{frame}

\begin{frame}
\frametitle{Also Part of CRT-AI Constraint Week}
\begin{itemize}
\item Annual one week course on CP and Optimization in Ireland
\item Part of national training program for PhD students in AI
\item \url{https://www.crt-ai.ie/}
\end{itemize}
\end{frame}



\begin{frame}
\frametitle{Constraint Programming - in a nutshell}
\begin{itemize}
\item Declarative description of problems with
\begin{itemize}
\item {\em Variables} which range over (finite) sets of values
\item {\em Constraints} over subsets of variables which restrict possible value combinations
\item A {\em solution} is a value assignment which satisfies all constraints
\end{itemize}

\item Constraint propagation/reasoning
\begin{itemize}
\item Removing inconsistent values for variables
\item Detect failure if constraint can not be satisfied
\item Interaction of constraints via shared variables
\item Incomplete
\end{itemize}

\item Search
\begin{itemize}
\item User controlled assignment of values to variables
\item Each step triggers constraint propagation 
\end{itemize}
\item Different domains require/allow different methods
\end{itemize}
\end{frame}

\begin{frame}
  \frametitle{Constraint Programming is Different}
  \begin{itemize}
  \item Declarative Programming
    \begin{itemize}
    \item Concentrate on what you want
      \item Not how to get there
      \item Program != Algorithm
      \item Program = Model
    \end{itemize}
    \item Applied to Combinatorial Problems
      \begin{itemize}
        \item No complete polynomial algorithms known (exist?)
        \item CP less ad-hoc than heuristics
        \item Models can evolve
  \end{itemize}
  \end{itemize}
  \end{frame}
    
\begin{frame}
  \frametitle{A Subtractive Process}
  \begin{textblock}{4}(8,-3)
    \includegraphics[width=4cm]{../introduction/images/stages}
  \end{textblock}
  \vfill
  \begin{quote}
    ``Oh, bosh, as Mr. Ruskin says. Sculpture, per se, is the simplest thing in the world. All you have to do is to take a big chunk of marble and a hammer and chisel, make up your mind what you are about to create and chip off all the marble you don't want.''-Paris Gaulois.
  \end{quote}
  
  {\tiny Source: \url{https://quoteinvestigator.com/2014/06/22/chip-away/}}
\end{frame}

\begin{frame}
\frametitle{Basic Process}
\begin{center}
\begin{tikzpicture}
\node[shape=rectangle,fill=pantone127-4] (problem) at (2,5) {Problem};
\node (human) at (2,4) {Human};
\node[shape=rectangle,fill=pantone127-4] (model) at (2,3) {Model};
\node (solver) at (2,2) {Constraint Solver/Search};
\node[shape=rectangle,fill=pantone157-8] (solution) at (2,1) {Solution};
\draw[-] (problem) -- (human);
\draw[->] (human) -- (model);
\draw[-] (model) -- (solver);
\draw[->] (solver) -- (solution);
\end{tikzpicture}
\end{center}
\end{frame}

\begin{frame}
\frametitle{More Realistic}
\begin{center}
\begin{tikzpicture}[xscale=4.0]
\node[shape=rectangle,fill=pantone127-4] (problem) at (2,5) {Problem};
\node (human) at (2,4) {Human};
\node[shape=rectangle,fill=pantone127-4] (model) at (2,3) {Model};
\node (solver) at (2,2) {Constraint Solver/Search};
\node[shape=rectangle,fill=pantone24-8] (hangs) at (1,1) {Hangs};
\node[shape=rectangle,fill=pantone157-8] (solution) at (2,1) {Solution};
\node[shape=rectangle,fill=pantone24-8] (wrong) at (3,1) {Wrong Solution};
\draw[-] (problem) -- (human);
\draw[->] (human) -- (model);
\draw[-] (model) -- (solver);
\draw[->] (solver) -- (hangs);
\draw[->] (solver) -- (solution);
\draw[->] (solver) -- (wrong);
\draw[->] (hangs) -- (1,4) -- (human);
\draw[->] (wrong) -- (3,4) -- (human);
\end{tikzpicture}
\end{center}
\end{frame}

\begin{frame}
\frametitle{Dual Role of Model}
\begin{itemize}
\item Allows Human to Express Problem
\begin{itemize}
\item Close to Problem Domain
\item Constraints as Abstractions
\end{itemize}
\item Allows Solver to Execute
\begin{itemize}
\item Variables as Communication Mechanism
\item Constraints as Algorithms
\end{itemize}
\end{itemize}
\end{frame}

\begin{frame}
\frametitle{Modelling Frameworks}
\begin{itemize}
\item \href{https://www.minizinc.org/}{MiniZinc} (NICTA, Monash University, Australia)
\item \href{https://github.com/eomahony/Numberjack}{NumberJack} (Insight, Ireland) 
\item \href{https://www-users.york.ac.uk/peter.nightingale/savilerow//}{EssencePrime/SavilleRow} (UK) 
\item \href{https://cpmpy.readthedocs.io/en/latest/}{CPMpy} (KU Leuven)
\item Allow use of multiple back-end solvers
\item Compile model into variants for each solver
\item A priori solver independent model(CP, MIP, SAT)
\end{itemize}
\end{frame} 

\begin{frame}
\frametitle{Framework Process}
\begin{center}
\begin{tikzpicture}[xscale=2]
\node[shape=rectangle,fill=pantone127-4] (problem) at (2.5,5) {Problem};
\node (human) at (2.5,4) {Human};
\node[shape=rectangle,fill=pantone127-4] (model) at (2.5,3) {Model};
\node (compile) at (2.5,2) {Compile/Reformulate};
\node (solver1) at (1,1) {CP};
\node (solver2) at (2,1) {MIP};
\node (solver3) at (3,1) {SAT};
\node (solver4) at (4,1) {Other};
\node[shape=rectangle,fill=pantone157-8] (solution1) at (1,0) {Solution};
\node[shape=rectangle,fill=pantone157-8] (solution2) at (2,0) {Solution};
\node[shape=rectangle,fill=pantone157-8] (solution3) at (3,0) {Solution};
\node[shape=rectangle,fill=pantone157-8] (solution4) at (4,0) {Solution};
\draw[-] (problem) -- (human);
\draw[->] (human) -- (model);
\draw[-] (model) -- (compile);
\draw[-] (compile) -- (solver1);
\draw[-] (compile) -- (solver2);
\draw[-] (compile) -- (solver3);
\draw[-] (compile) -- (solver4);
\draw[->] (solver1) -- (solution1);
\draw[->] (solver2) -- (solution2);
\draw[->] (solver3) -- (solution3);
\draw[->] (solver4) -- (solution4);
\end{tikzpicture}
\end{center}
\end{frame}

% \begin{frame}
  % \frametitle{Do It Now!}
  % \begin{itemize}
  % \item Download and install Minizinc
  % \item \url{https://www.minizinc.org/}
  % \end{itemize}
% \end{frame}


\begin{frame}
\frametitle{Why use Puzzles as Examples?}
\begin{itemize}
\item Easy to understand the problem
\item Solvable by hand without specialized knowledge
\item Possible to compare automated to manual solving process
\end{itemize}

\begin{quote}
The puzzle, though inanimate, is presented as a solvable problem without lasting negative consequences, a very low-risk low-reward situation. By being a puzzle, the object is attempting to convince the user that it must be completed.
\end{quote}
{\tiny Source: \href{https://www.everydayrhetoric.com/2020/03/the-rhetoric-of-puzzles/}{Every Day Rhetoric}}
\end{frame}


\mode<all>{
\part{Basic Constraint Propagation}
}
\begin{frame}
\partpage
\end{frame}
\mode<all>{
\input{../sendmore/body}
}

\mode<all>{
\part{Global Constraints}
}
\begin{frame}
\partpage
\end{frame}
\mode<all>{
\input{../sudoku/body}
}

\clearpage
\mode<all>{
\part{Customizing Search}
}
\begin{frame}
\partpage
\end{frame}
\mode<all>{
\input{../nqueen/body}
}


%% \clearpage
%% \mode<all>{
%%   \part{Problem Decomposition}
%% }
%% \begin{frame}
%%   \partpage
%% \end{frame}
%% \mode<all>{
%%   \input{../party/body}
%%   }

%% \clearpage
%% \mode<all>{
%%   \part{More Global Constraints}
%% }
%% \begin{frame}
%%   \partpage
%% \end{frame}
%% \mode<all>{
%%   \input{../car/body}
%%   }

%% \clearpage
%% \mode<all>{
%%   \part{Methodology}
%% }
%% \begin{frame}
%%   \partpage
%% \end{frame}
%% \mode<all>{
%%   \input{../methodology/body}
%%   }

\clearpage
\mode<all>{
\part{What is missing?}
}
\begin{frame}
\partpage
\end{frame}

\begin{frame}
\frametitle{Many Specialized Topics}
\begin{itemize}
\item How to design efficient core engine
\item Hybrids with LP/MIP tools
\item Hybrids with SAT
\item Symmetry breaking
\item Use of MDD/BDD to encode sets of solutions
\item High level modelling tools
\item Debugging/visualization
\end{itemize}
\end{frame}

\begin{frame}
\frametitle{Reformulation}
\begin{itemize}
\item Just because the user has modelled it this way, it doesn't mean we have to solve it that way
\begin{itemize}
\item Replace some constraint(s) by other, equivalent constraints
\item Because we don't have that constraint in our system
\item For performance
\end{itemize}
\end{itemize}
\end{frame}

\begin{frame}
\frametitle{Learning}
\begin{itemize}
\item While solving the problem we can learn how to strengthen the model/search 
\begin{itemize}
\item Understand which constraints/method contribute to propagation and change schedule
\item Learn no-good constraints by explaining failure
\item Adapt search strategy based on search experience
\end{itemize}
\end{itemize}
\end{frame}

%% \begin{frame}
%% \frametitle{Refined Process}
%% \begin{center}
%% \begin{tikzpicture}
%% \node[shape=rectangle,fill=pantone127-4] (problem) at (3,7) {Problem};
%% \node (human) at (3,6) {Human};
%% \node[shape=rectangle,fill=pantone127-4] (model) at (3,5) {User Model};
%% \node (reformulation) at (3,4) {Reformulation};
%% \node[shape=rectangle,fill=pantone127-4] (implementation) at (3,3) {Implementation Model};
%% \node (solver) at (3,2) {Constraint Solver/Search};
%% \node[shape=rectangle,fill=pantone157-8] (solution) at (3,1) {Solution};
%% \draw[-] (problem) -- (human);
%% \draw[->] (human) -- (model);
%% \draw[-] (model) -- (reformulation);
%% \draw[->] (reformulation) -- (implementation);
%% \draw[-] (implementation) -- (solver);
%% \draw[->] (solver) -- (0,2) -- (0,3) -- (implementation);
%% \draw[->] (solver) -- (solution);
%% \end{tikzpicture}
%% \end{center}
%% \end{frame}

%\begin{frame}
%\frametitle{What is CP actually used for?}
%\url{http://hsimonis.wordpress.com}
%\includegraphics[width=10cm]{../introduction/images/blog}
%\end{frame}

%% \begin{frame}
%% \frametitle{Exercises}
%% \begin{itemize}
%% \item Install Constraint Solver of your Choice
%% \item MiniZinc (MiningZinc) will be used in later sessions
%% \item Run Sendmoremoney, Sudoku, Queens examples
%% \item Can you choose which propagator to use for alldifferent?
%% \item Can you use different search strategies?
%% \end{itemize}
%% \end{frame}

\begin{frame}
\frametitle{More Learning Resources}
\begin{itemize}
\item Survey of Methods, Resources, and Formats for Teaching Constraint Programming
\begin{itemize}
\item by Tejas Santanam, Helmut Simonis
\item \url{https://doi.org/10.48550/arXiv.2403.12717}
\item Based on survey of community for WTCP 2023
\item \url{https://hsimonis.github.io/WTCP2023/}
\end{itemize}
\end{itemize}
\end{frame}


%%   }

%% \clearpage
%% \mode<all>{
%%   \part{More Global Constraints}
%% }
%% \begin{frame}
%%   \partpage
%% \end{frame}
%% \mode<all>{
%%   \begin{frame}
\frametitle{Objectives}
\begin{itemize}
\item Overview of Core Constraint Programming
\item Three Main Concepts
\begin{itemize}
\item Constraint Propagation
\item Global Constraints
\item Customizing Search
\end{itemize}
%\item Get Some Experience with MiniZinc
\item Topics will be treated in more detail in later parts of the school
\item Based on Examples, not Formal Description
\end{itemize}
\end{frame}

\begin{frame}
\frametitle{Outline}
\begin{itemize}
\item Why Constraint Programming?
\item Constraint Propagation
\item Global Constraints
\item Customizing Search
%\item Choices in Modelling
%\item Decomposition
%\item More Global Constraints
%\item Methodology
%\item What is missing?
\end{itemize}
\end{frame}

% \begin{frame}
  % \frametitle{Using MiniZinc IDE}
  % \begin{itemize}
  % \item Developed in the Australian NICTA project
  % \item Maintained by Monash University
  % \item Modelling tool with multiple back-end solvers
  % \item Available from \url{https://www.minizinc.org/}
  % \end{itemize}
% \end{frame}


% \begin{frame}
% \frametitle{Examples in ECLiPSe}
% \begin{itemize}
% \item Open sourced constraint programming language
% \item Development goes back to 1985
% \item ECRC, ICL, IC-Parc, PTL, Cisco
% \item \url{https://eclipseclp.org/}
% \item Specialities
% \begin{itemize}
% \item Develop new solvers for specific domains
% \item Integration with MIP
% \end{itemize}
% \item Not included in bundled MiniZinc IDE
% \item Specialized visualization tools used here
  % \begin{itemize}
  % \item CP-Viz, Simonis et al. 2010
    % \end{itemize}
% \end{itemize}
% \end{frame}

\begin{frame}
\frametitle{Tutorial Based on ECLiPSe ELearning Course}
\begin{itemize}
\item Self-study course in constraint programming
\item Supported by Cisco Systems and Silicon Valley Community Foundation
\item Multi-media format, video lectures, slides, handout etc
\item {\small \url{https://eclipseclp.org/ELearning/index.html}}
\end{itemize}
\end{frame}

\begin{frame}
\frametitle{Also Part of CRT-AI Constraint Week}
\begin{itemize}
\item Annual one week course on CP and Optimization in Ireland
\item Part of national training program for PhD students in AI
\item \url{https://www.crt-ai.ie/}
\end{itemize}
\end{frame}



\begin{frame}
\frametitle{Constraint Programming - in a nutshell}
\begin{itemize}
\item Declarative description of problems with
\begin{itemize}
\item {\em Variables} which range over (finite) sets of values
\item {\em Constraints} over subsets of variables which restrict possible value combinations
\item A {\em solution} is a value assignment which satisfies all constraints
\end{itemize}

\item Constraint propagation/reasoning
\begin{itemize}
\item Removing inconsistent values for variables
\item Detect failure if constraint can not be satisfied
\item Interaction of constraints via shared variables
\item Incomplete
\end{itemize}

\item Search
\begin{itemize}
\item User controlled assignment of values to variables
\item Each step triggers constraint propagation 
\end{itemize}
\item Different domains require/allow different methods
\end{itemize}
\end{frame}

\begin{frame}
  \frametitle{Constraint Programming is Different}
  \begin{itemize}
  \item Declarative Programming
    \begin{itemize}
    \item Concentrate on what you want
      \item Not how to get there
      \item Program != Algorithm
      \item Program = Model
    \end{itemize}
    \item Applied to Combinatorial Problems
      \begin{itemize}
        \item No complete polynomial algorithms known (exist?)
        \item CP less ad-hoc than heuristics
        \item Models can evolve
  \end{itemize}
  \end{itemize}
  \end{frame}
    
\begin{frame}
  \frametitle{A Subtractive Process}
  \begin{textblock}{4}(8,-3)
    \includegraphics[width=4cm]{../introduction/images/stages}
  \end{textblock}
  \vfill
  \begin{quote}
    ``Oh, bosh, as Mr. Ruskin says. Sculpture, per se, is the simplest thing in the world. All you have to do is to take a big chunk of marble and a hammer and chisel, make up your mind what you are about to create and chip off all the marble you don't want.''-Paris Gaulois.
  \end{quote}
  
  {\tiny Source: \url{https://quoteinvestigator.com/2014/06/22/chip-away/}}
\end{frame}

\begin{frame}
\frametitle{Basic Process}
\begin{center}
\begin{tikzpicture}
\node[shape=rectangle,fill=pantone127-4] (problem) at (2,5) {Problem};
\node (human) at (2,4) {Human};
\node[shape=rectangle,fill=pantone127-4] (model) at (2,3) {Model};
\node (solver) at (2,2) {Constraint Solver/Search};
\node[shape=rectangle,fill=pantone157-8] (solution) at (2,1) {Solution};
\draw[-] (problem) -- (human);
\draw[->] (human) -- (model);
\draw[-] (model) -- (solver);
\draw[->] (solver) -- (solution);
\end{tikzpicture}
\end{center}
\end{frame}

\begin{frame}
\frametitle{More Realistic}
\begin{center}
\begin{tikzpicture}[xscale=4.0]
\node[shape=rectangle,fill=pantone127-4] (problem) at (2,5) {Problem};
\node (human) at (2,4) {Human};
\node[shape=rectangle,fill=pantone127-4] (model) at (2,3) {Model};
\node (solver) at (2,2) {Constraint Solver/Search};
\node[shape=rectangle,fill=pantone24-8] (hangs) at (1,1) {Hangs};
\node[shape=rectangle,fill=pantone157-8] (solution) at (2,1) {Solution};
\node[shape=rectangle,fill=pantone24-8] (wrong) at (3,1) {Wrong Solution};
\draw[-] (problem) -- (human);
\draw[->] (human) -- (model);
\draw[-] (model) -- (solver);
\draw[->] (solver) -- (hangs);
\draw[->] (solver) -- (solution);
\draw[->] (solver) -- (wrong);
\draw[->] (hangs) -- (1,4) -- (human);
\draw[->] (wrong) -- (3,4) -- (human);
\end{tikzpicture}
\end{center}
\end{frame}

\begin{frame}
\frametitle{Dual Role of Model}
\begin{itemize}
\item Allows Human to Express Problem
\begin{itemize}
\item Close to Problem Domain
\item Constraints as Abstractions
\end{itemize}
\item Allows Solver to Execute
\begin{itemize}
\item Variables as Communication Mechanism
\item Constraints as Algorithms
\end{itemize}
\end{itemize}
\end{frame}

\begin{frame}
\frametitle{Modelling Frameworks}
\begin{itemize}
\item \href{https://www.minizinc.org/}{MiniZinc} (NICTA, Monash University, Australia)
\item \href{https://github.com/eomahony/Numberjack}{NumberJack} (Insight, Ireland) 
\item \href{https://www-users.york.ac.uk/peter.nightingale/savilerow//}{EssencePrime/SavilleRow} (UK) 
\item \href{https://cpmpy.readthedocs.io/en/latest/}{CPMpy} (KU Leuven)
\item Allow use of multiple back-end solvers
\item Compile model into variants for each solver
\item A priori solver independent model(CP, MIP, SAT)
\end{itemize}
\end{frame} 

\begin{frame}
\frametitle{Framework Process}
\begin{center}
\begin{tikzpicture}[xscale=2]
\node[shape=rectangle,fill=pantone127-4] (problem) at (2.5,5) {Problem};
\node (human) at (2.5,4) {Human};
\node[shape=rectangle,fill=pantone127-4] (model) at (2.5,3) {Model};
\node (compile) at (2.5,2) {Compile/Reformulate};
\node (solver1) at (1,1) {CP};
\node (solver2) at (2,1) {MIP};
\node (solver3) at (3,1) {SAT};
\node (solver4) at (4,1) {Other};
\node[shape=rectangle,fill=pantone157-8] (solution1) at (1,0) {Solution};
\node[shape=rectangle,fill=pantone157-8] (solution2) at (2,0) {Solution};
\node[shape=rectangle,fill=pantone157-8] (solution3) at (3,0) {Solution};
\node[shape=rectangle,fill=pantone157-8] (solution4) at (4,0) {Solution};
\draw[-] (problem) -- (human);
\draw[->] (human) -- (model);
\draw[-] (model) -- (compile);
\draw[-] (compile) -- (solver1);
\draw[-] (compile) -- (solver2);
\draw[-] (compile) -- (solver3);
\draw[-] (compile) -- (solver4);
\draw[->] (solver1) -- (solution1);
\draw[->] (solver2) -- (solution2);
\draw[->] (solver3) -- (solution3);
\draw[->] (solver4) -- (solution4);
\end{tikzpicture}
\end{center}
\end{frame}

% \begin{frame}
  % \frametitle{Do It Now!}
  % \begin{itemize}
  % \item Download and install Minizinc
  % \item \url{https://www.minizinc.org/}
  % \end{itemize}
% \end{frame}


\begin{frame}
\frametitle{Why use Puzzles as Examples?}
\begin{itemize}
\item Easy to understand the problem
\item Solvable by hand without specialized knowledge
\item Possible to compare automated to manual solving process
\end{itemize}

\begin{quote}
The puzzle, though inanimate, is presented as a solvable problem without lasting negative consequences, a very low-risk low-reward situation. By being a puzzle, the object is attempting to convince the user that it must be completed.
\end{quote}
{\tiny Source: \href{https://www.everydayrhetoric.com/2020/03/the-rhetoric-of-puzzles/}{Every Day Rhetoric}}
\end{frame}


\mode<all>{
\part{Basic Constraint Propagation}
}
\begin{frame}
\partpage
\end{frame}
\mode<all>{
\input{../sendmore/body}
}

\mode<all>{
\part{Global Constraints}
}
\begin{frame}
\partpage
\end{frame}
\mode<all>{
\input{../sudoku/body}
}

\clearpage
\mode<all>{
\part{Customizing Search}
}
\begin{frame}
\partpage
\end{frame}
\mode<all>{
\input{../nqueen/body}
}


%% \clearpage
%% \mode<all>{
%%   \part{Problem Decomposition}
%% }
%% \begin{frame}
%%   \partpage
%% \end{frame}
%% \mode<all>{
%%   \input{../party/body}
%%   }

%% \clearpage
%% \mode<all>{
%%   \part{More Global Constraints}
%% }
%% \begin{frame}
%%   \partpage
%% \end{frame}
%% \mode<all>{
%%   \input{../car/body}
%%   }

%% \clearpage
%% \mode<all>{
%%   \part{Methodology}
%% }
%% \begin{frame}
%%   \partpage
%% \end{frame}
%% \mode<all>{
%%   \input{../methodology/body}
%%   }

\clearpage
\mode<all>{
\part{What is missing?}
}
\begin{frame}
\partpage
\end{frame}

\begin{frame}
\frametitle{Many Specialized Topics}
\begin{itemize}
\item How to design efficient core engine
\item Hybrids with LP/MIP tools
\item Hybrids with SAT
\item Symmetry breaking
\item Use of MDD/BDD to encode sets of solutions
\item High level modelling tools
\item Debugging/visualization
\end{itemize}
\end{frame}

\begin{frame}
\frametitle{Reformulation}
\begin{itemize}
\item Just because the user has modelled it this way, it doesn't mean we have to solve it that way
\begin{itemize}
\item Replace some constraint(s) by other, equivalent constraints
\item Because we don't have that constraint in our system
\item For performance
\end{itemize}
\end{itemize}
\end{frame}

\begin{frame}
\frametitle{Learning}
\begin{itemize}
\item While solving the problem we can learn how to strengthen the model/search 
\begin{itemize}
\item Understand which constraints/method contribute to propagation and change schedule
\item Learn no-good constraints by explaining failure
\item Adapt search strategy based on search experience
\end{itemize}
\end{itemize}
\end{frame}

%% \begin{frame}
%% \frametitle{Refined Process}
%% \begin{center}
%% \begin{tikzpicture}
%% \node[shape=rectangle,fill=pantone127-4] (problem) at (3,7) {Problem};
%% \node (human) at (3,6) {Human};
%% \node[shape=rectangle,fill=pantone127-4] (model) at (3,5) {User Model};
%% \node (reformulation) at (3,4) {Reformulation};
%% \node[shape=rectangle,fill=pantone127-4] (implementation) at (3,3) {Implementation Model};
%% \node (solver) at (3,2) {Constraint Solver/Search};
%% \node[shape=rectangle,fill=pantone157-8] (solution) at (3,1) {Solution};
%% \draw[-] (problem) -- (human);
%% \draw[->] (human) -- (model);
%% \draw[-] (model) -- (reformulation);
%% \draw[->] (reformulation) -- (implementation);
%% \draw[-] (implementation) -- (solver);
%% \draw[->] (solver) -- (0,2) -- (0,3) -- (implementation);
%% \draw[->] (solver) -- (solution);
%% \end{tikzpicture}
%% \end{center}
%% \end{frame}

%\begin{frame}
%\frametitle{What is CP actually used for?}
%\url{http://hsimonis.wordpress.com}
%\includegraphics[width=10cm]{../introduction/images/blog}
%\end{frame}

%% \begin{frame}
%% \frametitle{Exercises}
%% \begin{itemize}
%% \item Install Constraint Solver of your Choice
%% \item MiniZinc (MiningZinc) will be used in later sessions
%% \item Run Sendmoremoney, Sudoku, Queens examples
%% \item Can you choose which propagator to use for alldifferent?
%% \item Can you use different search strategies?
%% \end{itemize}
%% \end{frame}

\begin{frame}
\frametitle{More Learning Resources}
\begin{itemize}
\item Survey of Methods, Resources, and Formats for Teaching Constraint Programming
\begin{itemize}
\item by Tejas Santanam, Helmut Simonis
\item \url{https://doi.org/10.48550/arXiv.2403.12717}
\item Based on survey of community for WTCP 2023
\item \url{https://hsimonis.github.io/WTCP2023/}
\end{itemize}
\end{itemize}
\end{frame}


%%   }

%% \clearpage
%% \mode<all>{
%%   \part{Methodology}
%% }
%% \begin{frame}
%%   \partpage
%% \end{frame}
%% \mode<all>{
%%   \begin{frame}
\frametitle{Objectives}
\begin{itemize}
\item Overview of Core Constraint Programming
\item Three Main Concepts
\begin{itemize}
\item Constraint Propagation
\item Global Constraints
\item Customizing Search
\end{itemize}
%\item Get Some Experience with MiniZinc
\item Topics will be treated in more detail in later parts of the school
\item Based on Examples, not Formal Description
\end{itemize}
\end{frame}

\begin{frame}
\frametitle{Outline}
\begin{itemize}
\item Why Constraint Programming?
\item Constraint Propagation
\item Global Constraints
\item Customizing Search
%\item Choices in Modelling
%\item Decomposition
%\item More Global Constraints
%\item Methodology
%\item What is missing?
\end{itemize}
\end{frame}

% \begin{frame}
  % \frametitle{Using MiniZinc IDE}
  % \begin{itemize}
  % \item Developed in the Australian NICTA project
  % \item Maintained by Monash University
  % \item Modelling tool with multiple back-end solvers
  % \item Available from \url{https://www.minizinc.org/}
  % \end{itemize}
% \end{frame}


% \begin{frame}
% \frametitle{Examples in ECLiPSe}
% \begin{itemize}
% \item Open sourced constraint programming language
% \item Development goes back to 1985
% \item ECRC, ICL, IC-Parc, PTL, Cisco
% \item \url{https://eclipseclp.org/}
% \item Specialities
% \begin{itemize}
% \item Develop new solvers for specific domains
% \item Integration with MIP
% \end{itemize}
% \item Not included in bundled MiniZinc IDE
% \item Specialized visualization tools used here
  % \begin{itemize}
  % \item CP-Viz, Simonis et al. 2010
    % \end{itemize}
% \end{itemize}
% \end{frame}

\begin{frame}
\frametitle{Tutorial Based on ECLiPSe ELearning Course}
\begin{itemize}
\item Self-study course in constraint programming
\item Supported by Cisco Systems and Silicon Valley Community Foundation
\item Multi-media format, video lectures, slides, handout etc
\item {\small \url{https://eclipseclp.org/ELearning/index.html}}
\end{itemize}
\end{frame}

\begin{frame}
\frametitle{Also Part of CRT-AI Constraint Week}
\begin{itemize}
\item Annual one week course on CP and Optimization in Ireland
\item Part of national training program for PhD students in AI
\item \url{https://www.crt-ai.ie/}
\end{itemize}
\end{frame}



\begin{frame}
\frametitle{Constraint Programming - in a nutshell}
\begin{itemize}
\item Declarative description of problems with
\begin{itemize}
\item {\em Variables} which range over (finite) sets of values
\item {\em Constraints} over subsets of variables which restrict possible value combinations
\item A {\em solution} is a value assignment which satisfies all constraints
\end{itemize}

\item Constraint propagation/reasoning
\begin{itemize}
\item Removing inconsistent values for variables
\item Detect failure if constraint can not be satisfied
\item Interaction of constraints via shared variables
\item Incomplete
\end{itemize}

\item Search
\begin{itemize}
\item User controlled assignment of values to variables
\item Each step triggers constraint propagation 
\end{itemize}
\item Different domains require/allow different methods
\end{itemize}
\end{frame}

\begin{frame}
  \frametitle{Constraint Programming is Different}
  \begin{itemize}
  \item Declarative Programming
    \begin{itemize}
    \item Concentrate on what you want
      \item Not how to get there
      \item Program != Algorithm
      \item Program = Model
    \end{itemize}
    \item Applied to Combinatorial Problems
      \begin{itemize}
        \item No complete polynomial algorithms known (exist?)
        \item CP less ad-hoc than heuristics
        \item Models can evolve
  \end{itemize}
  \end{itemize}
  \end{frame}
    
\begin{frame}
  \frametitle{A Subtractive Process}
  \begin{textblock}{4}(8,-3)
    \includegraphics[width=4cm]{../introduction/images/stages}
  \end{textblock}
  \vfill
  \begin{quote}
    ``Oh, bosh, as Mr. Ruskin says. Sculpture, per se, is the simplest thing in the world. All you have to do is to take a big chunk of marble and a hammer and chisel, make up your mind what you are about to create and chip off all the marble you don't want.''-Paris Gaulois.
  \end{quote}
  
  {\tiny Source: \url{https://quoteinvestigator.com/2014/06/22/chip-away/}}
\end{frame}

\begin{frame}
\frametitle{Basic Process}
\begin{center}
\begin{tikzpicture}
\node[shape=rectangle,fill=pantone127-4] (problem) at (2,5) {Problem};
\node (human) at (2,4) {Human};
\node[shape=rectangle,fill=pantone127-4] (model) at (2,3) {Model};
\node (solver) at (2,2) {Constraint Solver/Search};
\node[shape=rectangle,fill=pantone157-8] (solution) at (2,1) {Solution};
\draw[-] (problem) -- (human);
\draw[->] (human) -- (model);
\draw[-] (model) -- (solver);
\draw[->] (solver) -- (solution);
\end{tikzpicture}
\end{center}
\end{frame}

\begin{frame}
\frametitle{More Realistic}
\begin{center}
\begin{tikzpicture}[xscale=4.0]
\node[shape=rectangle,fill=pantone127-4] (problem) at (2,5) {Problem};
\node (human) at (2,4) {Human};
\node[shape=rectangle,fill=pantone127-4] (model) at (2,3) {Model};
\node (solver) at (2,2) {Constraint Solver/Search};
\node[shape=rectangle,fill=pantone24-8] (hangs) at (1,1) {Hangs};
\node[shape=rectangle,fill=pantone157-8] (solution) at (2,1) {Solution};
\node[shape=rectangle,fill=pantone24-8] (wrong) at (3,1) {Wrong Solution};
\draw[-] (problem) -- (human);
\draw[->] (human) -- (model);
\draw[-] (model) -- (solver);
\draw[->] (solver) -- (hangs);
\draw[->] (solver) -- (solution);
\draw[->] (solver) -- (wrong);
\draw[->] (hangs) -- (1,4) -- (human);
\draw[->] (wrong) -- (3,4) -- (human);
\end{tikzpicture}
\end{center}
\end{frame}

\begin{frame}
\frametitle{Dual Role of Model}
\begin{itemize}
\item Allows Human to Express Problem
\begin{itemize}
\item Close to Problem Domain
\item Constraints as Abstractions
\end{itemize}
\item Allows Solver to Execute
\begin{itemize}
\item Variables as Communication Mechanism
\item Constraints as Algorithms
\end{itemize}
\end{itemize}
\end{frame}

\begin{frame}
\frametitle{Modelling Frameworks}
\begin{itemize}
\item \href{https://www.minizinc.org/}{MiniZinc} (NICTA, Monash University, Australia)
\item \href{https://github.com/eomahony/Numberjack}{NumberJack} (Insight, Ireland) 
\item \href{https://www-users.york.ac.uk/peter.nightingale/savilerow//}{EssencePrime/SavilleRow} (UK) 
\item \href{https://cpmpy.readthedocs.io/en/latest/}{CPMpy} (KU Leuven)
\item Allow use of multiple back-end solvers
\item Compile model into variants for each solver
\item A priori solver independent model(CP, MIP, SAT)
\end{itemize}
\end{frame} 

\begin{frame}
\frametitle{Framework Process}
\begin{center}
\begin{tikzpicture}[xscale=2]
\node[shape=rectangle,fill=pantone127-4] (problem) at (2.5,5) {Problem};
\node (human) at (2.5,4) {Human};
\node[shape=rectangle,fill=pantone127-4] (model) at (2.5,3) {Model};
\node (compile) at (2.5,2) {Compile/Reformulate};
\node (solver1) at (1,1) {CP};
\node (solver2) at (2,1) {MIP};
\node (solver3) at (3,1) {SAT};
\node (solver4) at (4,1) {Other};
\node[shape=rectangle,fill=pantone157-8] (solution1) at (1,0) {Solution};
\node[shape=rectangle,fill=pantone157-8] (solution2) at (2,0) {Solution};
\node[shape=rectangle,fill=pantone157-8] (solution3) at (3,0) {Solution};
\node[shape=rectangle,fill=pantone157-8] (solution4) at (4,0) {Solution};
\draw[-] (problem) -- (human);
\draw[->] (human) -- (model);
\draw[-] (model) -- (compile);
\draw[-] (compile) -- (solver1);
\draw[-] (compile) -- (solver2);
\draw[-] (compile) -- (solver3);
\draw[-] (compile) -- (solver4);
\draw[->] (solver1) -- (solution1);
\draw[->] (solver2) -- (solution2);
\draw[->] (solver3) -- (solution3);
\draw[->] (solver4) -- (solution4);
\end{tikzpicture}
\end{center}
\end{frame}

% \begin{frame}
  % \frametitle{Do It Now!}
  % \begin{itemize}
  % \item Download and install Minizinc
  % \item \url{https://www.minizinc.org/}
  % \end{itemize}
% \end{frame}


\begin{frame}
\frametitle{Why use Puzzles as Examples?}
\begin{itemize}
\item Easy to understand the problem
\item Solvable by hand without specialized knowledge
\item Possible to compare automated to manual solving process
\end{itemize}

\begin{quote}
The puzzle, though inanimate, is presented as a solvable problem without lasting negative consequences, a very low-risk low-reward situation. By being a puzzle, the object is attempting to convince the user that it must be completed.
\end{quote}
{\tiny Source: \href{https://www.everydayrhetoric.com/2020/03/the-rhetoric-of-puzzles/}{Every Day Rhetoric}}
\end{frame}


\mode<all>{
\part{Basic Constraint Propagation}
}
\begin{frame}
\partpage
\end{frame}
\mode<all>{
\input{../sendmore/body}
}

\mode<all>{
\part{Global Constraints}
}
\begin{frame}
\partpage
\end{frame}
\mode<all>{
\input{../sudoku/body}
}

\clearpage
\mode<all>{
\part{Customizing Search}
}
\begin{frame}
\partpage
\end{frame}
\mode<all>{
\input{../nqueen/body}
}


%% \clearpage
%% \mode<all>{
%%   \part{Problem Decomposition}
%% }
%% \begin{frame}
%%   \partpage
%% \end{frame}
%% \mode<all>{
%%   \input{../party/body}
%%   }

%% \clearpage
%% \mode<all>{
%%   \part{More Global Constraints}
%% }
%% \begin{frame}
%%   \partpage
%% \end{frame}
%% \mode<all>{
%%   \input{../car/body}
%%   }

%% \clearpage
%% \mode<all>{
%%   \part{Methodology}
%% }
%% \begin{frame}
%%   \partpage
%% \end{frame}
%% \mode<all>{
%%   \input{../methodology/body}
%%   }

\clearpage
\mode<all>{
\part{What is missing?}
}
\begin{frame}
\partpage
\end{frame}

\begin{frame}
\frametitle{Many Specialized Topics}
\begin{itemize}
\item How to design efficient core engine
\item Hybrids with LP/MIP tools
\item Hybrids with SAT
\item Symmetry breaking
\item Use of MDD/BDD to encode sets of solutions
\item High level modelling tools
\item Debugging/visualization
\end{itemize}
\end{frame}

\begin{frame}
\frametitle{Reformulation}
\begin{itemize}
\item Just because the user has modelled it this way, it doesn't mean we have to solve it that way
\begin{itemize}
\item Replace some constraint(s) by other, equivalent constraints
\item Because we don't have that constraint in our system
\item For performance
\end{itemize}
\end{itemize}
\end{frame}

\begin{frame}
\frametitle{Learning}
\begin{itemize}
\item While solving the problem we can learn how to strengthen the model/search 
\begin{itemize}
\item Understand which constraints/method contribute to propagation and change schedule
\item Learn no-good constraints by explaining failure
\item Adapt search strategy based on search experience
\end{itemize}
\end{itemize}
\end{frame}

%% \begin{frame}
%% \frametitle{Refined Process}
%% \begin{center}
%% \begin{tikzpicture}
%% \node[shape=rectangle,fill=pantone127-4] (problem) at (3,7) {Problem};
%% \node (human) at (3,6) {Human};
%% \node[shape=rectangle,fill=pantone127-4] (model) at (3,5) {User Model};
%% \node (reformulation) at (3,4) {Reformulation};
%% \node[shape=rectangle,fill=pantone127-4] (implementation) at (3,3) {Implementation Model};
%% \node (solver) at (3,2) {Constraint Solver/Search};
%% \node[shape=rectangle,fill=pantone157-8] (solution) at (3,1) {Solution};
%% \draw[-] (problem) -- (human);
%% \draw[->] (human) -- (model);
%% \draw[-] (model) -- (reformulation);
%% \draw[->] (reformulation) -- (implementation);
%% \draw[-] (implementation) -- (solver);
%% \draw[->] (solver) -- (0,2) -- (0,3) -- (implementation);
%% \draw[->] (solver) -- (solution);
%% \end{tikzpicture}
%% \end{center}
%% \end{frame}

%\begin{frame}
%\frametitle{What is CP actually used for?}
%\url{http://hsimonis.wordpress.com}
%\includegraphics[width=10cm]{../introduction/images/blog}
%\end{frame}

%% \begin{frame}
%% \frametitle{Exercises}
%% \begin{itemize}
%% \item Install Constraint Solver of your Choice
%% \item MiniZinc (MiningZinc) will be used in later sessions
%% \item Run Sendmoremoney, Sudoku, Queens examples
%% \item Can you choose which propagator to use for alldifferent?
%% \item Can you use different search strategies?
%% \end{itemize}
%% \end{frame}

\begin{frame}
\frametitle{More Learning Resources}
\begin{itemize}
\item Survey of Methods, Resources, and Formats for Teaching Constraint Programming
\begin{itemize}
\item by Tejas Santanam, Helmut Simonis
\item \url{https://doi.org/10.48550/arXiv.2403.12717}
\item Based on survey of community for WTCP 2023
\item \url{https://hsimonis.github.io/WTCP2023/}
\end{itemize}
\end{itemize}
\end{frame}


%%   }

\clearpage
\mode<all>{
\part{What is missing?}
}
\begin{frame}
\partpage
\end{frame}

\begin{frame}
\frametitle{Many Specialized Topics}
\begin{itemize}
\item How to design efficient core engine
\item Hybrids with LP/MIP tools
\item Hybrids with SAT
\item Symmetry breaking
\item Use of MDD/BDD to encode sets of solutions
\item High level modelling tools
\item Debugging/visualization
\end{itemize}
\end{frame}

\begin{frame}
\frametitle{Reformulation}
\begin{itemize}
\item Just because the user has modelled it this way, it doesn't mean we have to solve it that way
\begin{itemize}
\item Replace some constraint(s) by other, equivalent constraints
\item Because we don't have that constraint in our system
\item For performance
\end{itemize}
\end{itemize}
\end{frame}

\begin{frame}
\frametitle{Learning}
\begin{itemize}
\item While solving the problem we can learn how to strengthen the model/search 
\begin{itemize}
\item Understand which constraints/method contribute to propagation and change schedule
\item Learn no-good constraints by explaining failure
\item Adapt search strategy based on search experience
\end{itemize}
\end{itemize}
\end{frame}

%% \begin{frame}
%% \frametitle{Refined Process}
%% \begin{center}
%% \begin{tikzpicture}
%% \node[shape=rectangle,fill=pantone127-4] (problem) at (3,7) {Problem};
%% \node (human) at (3,6) {Human};
%% \node[shape=rectangle,fill=pantone127-4] (model) at (3,5) {User Model};
%% \node (reformulation) at (3,4) {Reformulation};
%% \node[shape=rectangle,fill=pantone127-4] (implementation) at (3,3) {Implementation Model};
%% \node (solver) at (3,2) {Constraint Solver/Search};
%% \node[shape=rectangle,fill=pantone157-8] (solution) at (3,1) {Solution};
%% \draw[-] (problem) -- (human);
%% \draw[->] (human) -- (model);
%% \draw[-] (model) -- (reformulation);
%% \draw[->] (reformulation) -- (implementation);
%% \draw[-] (implementation) -- (solver);
%% \draw[->] (solver) -- (0,2) -- (0,3) -- (implementation);
%% \draw[->] (solver) -- (solution);
%% \end{tikzpicture}
%% \end{center}
%% \end{frame}

%\begin{frame}
%\frametitle{What is CP actually used for?}
%\url{http://hsimonis.wordpress.com}
%\includegraphics[width=10cm]{../introduction/images/blog}
%\end{frame}

%% \begin{frame}
%% \frametitle{Exercises}
%% \begin{itemize}
%% \item Install Constraint Solver of your Choice
%% \item MiniZinc (MiningZinc) will be used in later sessions
%% \item Run Sendmoremoney, Sudoku, Queens examples
%% \item Can you choose which propagator to use for alldifferent?
%% \item Can you use different search strategies?
%% \end{itemize}
%% \end{frame}

\begin{frame}
\frametitle{More Learning Resources}
\begin{itemize}
\item Survey of Methods, Resources, and Formats for Teaching Constraint Programming
\begin{itemize}
\item by Tejas Santanam, Helmut Simonis
\item \url{https://doi.org/10.48550/arXiv.2403.12717}
\item Based on survey of community for WTCP 2023
\item \url{https://hsimonis.github.io/WTCP2023/}
\end{itemize}
\end{itemize}
\end{frame}


%%   }

\clearpage
\mode<all>{
\part{What is missing?}
}
\begin{frame}
\partpage
\end{frame}

\begin{frame}
\frametitle{Many Specialized Topics}
\begin{itemize}
\item How to design efficient core engine
\item Hybrids with LP/MIP tools
\item Hybrids with SAT
\item Symmetry breaking
\item Use of MDD/BDD to encode sets of solutions
\item High level modelling tools
\item Debugging/visualization
\end{itemize}
\end{frame}

\begin{frame}
\frametitle{Reformulation}
\begin{itemize}
\item Just because the user has modelled it this way, it doesn't mean we have to solve it that way
\begin{itemize}
\item Replace some constraint(s) by other, equivalent constraints
\item Because we don't have that constraint in our system
\item For performance
\end{itemize}
\end{itemize}
\end{frame}

\begin{frame}
\frametitle{Learning}
\begin{itemize}
\item While solving the problem we can learn how to strengthen the model/search 
\begin{itemize}
\item Understand which constraints/method contribute to propagation and change schedule
\item Learn no-good constraints by explaining failure
\item Adapt search strategy based on search experience
\end{itemize}
\end{itemize}
\end{frame}

%% \begin{frame}
%% \frametitle{Refined Process}
%% \begin{center}
%% \begin{tikzpicture}
%% \node[shape=rectangle,fill=pantone127-4] (problem) at (3,7) {Problem};
%% \node (human) at (3,6) {Human};
%% \node[shape=rectangle,fill=pantone127-4] (model) at (3,5) {User Model};
%% \node (reformulation) at (3,4) {Reformulation};
%% \node[shape=rectangle,fill=pantone127-4] (implementation) at (3,3) {Implementation Model};
%% \node (solver) at (3,2) {Constraint Solver/Search};
%% \node[shape=rectangle,fill=pantone157-8] (solution) at (3,1) {Solution};
%% \draw[-] (problem) -- (human);
%% \draw[->] (human) -- (model);
%% \draw[-] (model) -- (reformulation);
%% \draw[->] (reformulation) -- (implementation);
%% \draw[-] (implementation) -- (solver);
%% \draw[->] (solver) -- (0,2) -- (0,3) -- (implementation);
%% \draw[->] (solver) -- (solution);
%% \end{tikzpicture}
%% \end{center}
%% \end{frame}

%\begin{frame}
%\frametitle{What is CP actually used for?}
%\url{http://hsimonis.wordpress.com}
%\includegraphics[width=10cm]{../introduction/images/blog}
%\end{frame}

%% \begin{frame}
%% \frametitle{Exercises}
%% \begin{itemize}
%% \item Install Constraint Solver of your Choice
%% \item MiniZinc (MiningZinc) will be used in later sessions
%% \item Run Sendmoremoney, Sudoku, Queens examples
%% \item Can you choose which propagator to use for alldifferent?
%% \item Can you use different search strategies?
%% \end{itemize}
%% \end{frame}

\begin{frame}
\frametitle{More Learning Resources}
\begin{itemize}
\item Survey of Methods, Resources, and Formats for Teaching Constraint Programming
\begin{itemize}
\item by Tejas Santanam, Helmut Simonis
\item \url{https://doi.org/10.48550/arXiv.2403.12717}
\item Based on survey of community for WTCP 2023
\item \url{https://hsimonis.github.io/WTCP2023/}
\end{itemize}
\end{itemize}
\end{frame}


%%   }

\clearpage
\mode<all>{
\part{What is missing?}
}
\begin{frame}
\partpage
\end{frame}

\begin{frame}
\frametitle{Many Specialized Topics}
\begin{itemize}
\item How to design efficient core engine
\item Hybrids with LP/MIP tools
\item Hybrids with SAT
\item Symmetry breaking
\item Use of MDD/BDD to encode sets of solutions
\item High level modelling tools
\item Debugging/visualization
\end{itemize}
\end{frame}

\begin{frame}
\frametitle{Reformulation}
\begin{itemize}
\item Just because the user has modelled it this way, it doesn't mean we have to solve it that way
\begin{itemize}
\item Replace some constraint(s) by other, equivalent constraints
\item Because we don't have that constraint in our system
\item For performance
\end{itemize}
\end{itemize}
\end{frame}

\begin{frame}
\frametitle{Learning}
\begin{itemize}
\item While solving the problem we can learn how to strengthen the model/search 
\begin{itemize}
\item Understand which constraints/method contribute to propagation and change schedule
\item Learn no-good constraints by explaining failure
\item Adapt search strategy based on search experience
\end{itemize}
\end{itemize}
\end{frame}

%% \begin{frame}
%% \frametitle{Refined Process}
%% \begin{center}
%% \begin{tikzpicture}
%% \node[shape=rectangle,fill=pantone127-4] (problem) at (3,7) {Problem};
%% \node (human) at (3,6) {Human};
%% \node[shape=rectangle,fill=pantone127-4] (model) at (3,5) {User Model};
%% \node (reformulation) at (3,4) {Reformulation};
%% \node[shape=rectangle,fill=pantone127-4] (implementation) at (3,3) {Implementation Model};
%% \node (solver) at (3,2) {Constraint Solver/Search};
%% \node[shape=rectangle,fill=pantone157-8] (solution) at (3,1) {Solution};
%% \draw[-] (problem) -- (human);
%% \draw[->] (human) -- (model);
%% \draw[-] (model) -- (reformulation);
%% \draw[->] (reformulation) -- (implementation);
%% \draw[-] (implementation) -- (solver);
%% \draw[->] (solver) -- (0,2) -- (0,3) -- (implementation);
%% \draw[->] (solver) -- (solution);
%% \end{tikzpicture}
%% \end{center}
%% \end{frame}

%\begin{frame}
%\frametitle{What is CP actually used for?}
%\url{http://hsimonis.wordpress.com}
%\includegraphics[width=10cm]{../introduction/images/blog}
%\end{frame}

%% \begin{frame}
%% \frametitle{Exercises}
%% \begin{itemize}
%% \item Install Constraint Solver of your Choice
%% \item MiniZinc (MiningZinc) will be used in later sessions
%% \item Run Sendmoremoney, Sudoku, Queens examples
%% \item Can you choose which propagator to use for alldifferent?
%% \item Can you use different search strategies?
%% \end{itemize}
%% \end{frame}

\begin{frame}
\frametitle{More Learning Resources}
\begin{itemize}
\item Survey of Methods, Resources, and Formats for Teaching Constraint Programming
\begin{itemize}
\item by Tejas Santanam, Helmut Simonis
\item \url{https://doi.org/10.48550/arXiv.2403.12717}
\item Based on survey of community for WTCP 2023
\item \url{https://hsimonis.github.io/WTCP2023/}
\end{itemize}
\end{itemize}
\end{frame}

